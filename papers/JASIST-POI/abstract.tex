\abstract{
\noindent
The category of a point-of-interest (POI) is 
important for location based services (LBS). Popular map services try to
include POI type information or allow map users to tag POIs from a 
predefined set of categories, but these tags can be inaccurate and far from complete.
The classification of POIs have been studied in the context of location-based
social network and models are largely based on user visiting behaviors.
%Thus the concept of Point Of Interest (POI), representing any point on the map users may be interested in, emerges, and becomes the basic component of Location-based Social Network (LBSN). In order to fully utilize the POIs on the map, one essential information about a POI is its category. A lot of classification approaches based on analyzing user behavior exist. 
This paper explores another aspect of a POI: the geographical 
location of the POI, and studies how spatial features influence the results of
POI classification. 
%During our exploration of different kinds of spatial features, 
We find that different spatial features work well for different categories, 
%Features influence each other in combination, thus 
and a best feature combination exists for each category. 
We show that while each feature individually benefits the classification, 
the best combination provides significant improvements over user behavior 
features alone. What's more, spatial features are robust to noise and sparse
data.
}

\keywords{LBS, LBSN, POI, classification, feature, geographical coordinates}
