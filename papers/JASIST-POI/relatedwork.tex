\section{Related Work}
\label{RelatedWork}
In this section, we review related works mainly from two aspects: 
POI categorization, and other tasks utilizing spatial features.

\subsection{POI Categorization}
Ye et al.'s work is most related to our work in terms of purpose and approach. 
The first part of their features comes from explicit patterns, 
including total number of check-ins, total number of unique visitors, 
maximum number of check-ins by a single visitor, distribution of check-in 
time in a week and distribution of check-in time in 24-hour scale. 
We found these features are very useful in our data, too, 
and the reasons were clearly described in \cite{yemao}. 
However, the relatedness score, which is the second part of their features, 
is ineffective for our data set from Foursquare. The relatedness score is 
conducted from two undirected bipartite graph applying Random Walk: 
the User-Place graph representing user's check-in record, 
and the Time-Place graph indicating the time of check-ins 
at POIs. We believe the main reason comes from data set 
difference. Ye et al.'s data set comes from Whrrl. They 
filtered out users who have less than 40 check-ins, and users whose 
check-in places' entropy larger than 0.5. However, our data set 
from Foursquare don't have so many individual check-ins, and thus
doing such filtering will make the data sparser. Ye et al.'s work 
does not work in sparse data because: (1) there's no clear relatedness between 
POIs visited by the same user; and (2) the number of check-ins for one 
POI may not sufficient to indicate the places' check-in distribution over time, 
thus the relatedness indicated by time is also not convincing. 
According to our experiment on our data set, adding the relatedness 
score does not gain better result, on the contrary the result gets 
worse. Therefore, we only compare with the BASE features from 
Ye. et al.'s work. We add spatial features 
to the BASE features and show significant improvement.

Along with the popularity of smart phone with GPS sensors, a large amount of work \cite{Liao:2007:EPA:1229555.1229562,placer11,MDCconditional,phoneImageAudio,topicmodelMDC1,MDCdescriptive} 
focusing on utilizing smart phone trace data emerge. 
One of the most early work is \cite{Liao:2007:EPA:1229555.1229562}, 
in which Liao et al. combines features from both time interval 
of POI visiting and the presence of some special types of POI, e.g., bus stops. 
They also applied a hierarchical conditional random field (CRF) in their approach 
to emphasize the visiting sequence. Along with this, Chen et al.\cite{placer11} 
used a hidden Markov model instead of a CRF. Semantic annotation of place, 
dedicated Task 1 in Nokia Mobile Data Challenge (MDC1)\cite{eberle2012mobile}, 
attracts many attentions\cite{MDCconditional,topicmodelMDC1,MDCdescriptive}. 
In that task, the data provides both the phone's sensors' data 
(e.g., GPS sensor, Wifi sensor, bluetooth sensor) and the specific 
phone status data (e.g., charging status, calling or messaging, 
phone-silent setting). Topic model has also been used 
in classification where the POI's category is treated as topic, 
POI is treated as documents, and the features are the words that 
compose a document \cite{topicmodelMDC1}. 
The winner of the task \cite{MDCconditional}, used several state-of-the-art 
classifiers. They concluded that manipulating the features by heuristics 
and emphasizing on time interval is essential in winning the task. 
Montoliu et al.\cite{MDCdescriptive} combined several binary classifiers 
to solve the multi-class problem for MDC1.
Chon\cite{phoneImageAudio} explores another interesting aspect from smart phone 
images and audio clips.
Again these data needs fully monitoring, thus is hard to obtain and 
cause privacy problems. Besides, we notice that the category set in these problems 
are limited to approximately 10, while in our paper, we explore 278 fine-grained 
categories.

Other than smart phone data source, Hegde et al.\cite{interestprofile} utilized words that 
users posted on online social networks, for example micro-blogging systems, 
to build user preferences and interests, then generate descriptive tags for POIs 
according to groups of users' check-ins and interests. 
These tags are mainly for semantic understanding rather than classification. 
Pianese\cite{UserRoutine} aims to get user traces by processing users' 
social network trails, for example from Twitter and Foursquare.

Some works\cite{Kim:2011:EUF:2030112.2030142,placer1} do not induce 
labels from the features extracted from POI themselves. 
Instead, they get information from users' choice on the labels. 
Lian\cite{Lian:2011:LLN:2093973.2093990} solve the problem of 
selecting appropriate name for the check-in from 
different POIs on the same spot, and they do this based on users' previous behaviors. 
In our data, user check-in on a specific POI is rather than just a coordinates. 
In fact there are many pieces of work\cite{placer7,placer9,DBLP:journals/pvldb/CaoCJ10a,DBLP:conf/mdm/LiuWY06}
done on characterizing places given GPS trails. However they do not 
provide explicit semantic labels for places. 
Moreover, there are more proposals for the prediction and recommendation 
of POI for users regarding spatial aspects than for classification. 
Ye et al.\cite{yournextmove} used a hidden Markov model to predict  
next probable category that a user would visit. Cheng et al.\cite{cheng2012fused}
introduced matrix factorization considering geographical influence into personalized POI recommendation.

\subsection{Spatial features in other tasks}
Some work use spatial features to improve performance 
in recommendation task.
Zheng et al.\cite{LocationZheng} introduce categories distribution 
in a fixed size rectangle to measure the functionality of certain area. 
To emphasize the importance of uncommon kinds of POIs, for example schools, 
among ubiquitous kinds of POIs, such as restaurants, they apply TFIDF score to 
recalculate POIs' category distribution. In their recommendation task, 
they showed a smaller region size would result in better performance. 
However, in our work, we did a step further and found out that we should set 
different size for different categories. 
Krumm et al.\cite{LocationPlacer} aim at solving a similar task with us, 
labeling the POI with a category. They introduced an annotator called ``Placer'' to solve 
the classification problem. However the diary survey data from government studies they used
is rather rare data source. The data records people's daily trajectory, thus exposes large
amount of privacy with long time monitoring. In their work, Krumm et al. proposed to 
use category distribution within different diameters and category distance as features 
to predict the category for a POI. However, according to our experiments, 
using all the features with different diameters would introduce redundancy 
and choosing one appropriate distance would be a better choice. We also tried several 
data processing methods on the category distance feature, and found out that a log 
operation or twice log operation further benefit the accuracy of prediction. 
Overall, both of the aforementioned work utilize the same spatial features for all categories, 
which may not be effective in the real world data, because different categories have different 
characteristics on different spatial features which we have shown in our experiments. 
In our work, we focus on analyzing the influence of each spatial feature 
to each category, which enables much more delicate use on spatial features.

%
%
%There are also interesting works focusing directly on user behaviour itself rather than utilizing it on tagging or classification. \cite{userbehavior} fully analyzed users' check-in behavior based on frequent users on Foursquare, and getting interesting conclusion from the perspective of user as well as from POIs. For example they studied the time interval of people staying in one place, the return frequency of people coming back to certain places. However, POIs covered by frequent users is only a small part comparing to all, thus delicate features based on these conclusions lack discrimination for most of the POIs, not to mention the POIs which lost even general statistic characteristics because of small quantity of check-ins.


