\section{Other Features}
\label{chp:others}
\subsection{NAME Features}
The most direct way to discover the category of a POI is looking at its name.
%Similar with human guessing the category of a POI, the first intuition we get from a POI is its name.
For example, when we see a POI named ``Mouth Restaurant'', we would easily figure out that it belongs to ``Food''.
%Given the fact that words appear in POIs' names are far more less than in vocabulary,
We build the NAME feature in a straightforward way. We break POIs' names into words,
then we collect all the words and filter out those showed up only once in POIs.
Then the remaining words forms a dictionary, and NAME feature for a POI is the
binary representation of the POI's name in the dictionary.

Simple as it is, it is obvious that NAME feature is a powerful feature.
However, there's still cases that name features cannot help but other features work.
For example, when we meet the name ``Charlie's Corner'', it is not so obvious what category
it should belong to by looking at the name. However, it shows good visit time distribution,
as well as good spatial environment character with other ``Food'' near it,
thus we still can classify it to ``Food'', which is the correct label, over other categories.

\subsection{BASE Features}
\label{sec:base}
As discussed above, the statistics of user behavior
features in Ye et al.'s work \cite{yemao} are also effective features for identifying
a POI's category. We also employ these feature in our experiments.
These features include the total number of check-ins, unique visitors,
maximum number of check-ins by a single visitor, distribution of check-in
time in a week and in 24-hour scale. %Such features represent the user behavior very well.


