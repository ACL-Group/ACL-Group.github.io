%# -*- coding: utf-8-unix -*-
% !TEX program = xelatex
% !TEX root = ../thesis.tex
% !TEX encoding = UTF-8 Unicode
\begin{thanks}

%  感谢所有测试和使用交大学位论文 \LaTeX 模板的同学!
%
%  感谢那位最先制作出博士学位论文 \LaTeX 模板的交大物理系同学!
%
%  感谢William Wang同学对模板移植做出的巨大贡献!
%  
%  感谢 \href{https://github.com/weijianwen}{@weijianwen} 学长一直以来的开发和维护工作!
%  
%  感谢 \href{https://github.com/sjtug}{@sjtug} 以及 \href{https://github.com/dyweb}{@dyweb} 对 0.9.5 之后版本的开发和维护工作!
%  
%  感谢所有为模板贡献过代码的同学们, 以及所有测试和使用模板的各位同学!
%

时光荏苒,在交大已经度过了六年的时光。
博士生的求学之路,应该是人生第一段称得上艰苦的旅行,
一路走来经历了很多,要感谢很多人,让我能一步步走到今天。

衷心感谢我的导师朱其立教授。
您对科研的不懈追求以及对学术的严谨态度,一点点感染着我,
使我不满足于追求表面的结果,而是去多问自己为什么,多去探寻问题的本质。
%您总是拿出大量的时间和每一位学生进行讨论,分析问题的敏锐嗅觉和 表率让我深深敬佩
在和您的无数次讨论交流中,
经常有学术上的争论,但我总能无压力地讨论自己的想法。
您对问题的敏锐嗅觉和对每一位学生的充分尊重,让我深深敬佩。
您经常工作到午夜,耐心地为我们的学术论文进行修改, 
还有一次次的Noodle time,大家一起愉快而又充满干劲地向论文提交发起冲刺。
六年来自己的科研进步离不开您的悉心指导,尤其在基础薄弱的前几年始终对我充满信心。
在科研之余,您经常和我们一起运动,组织一年一度甚至两度的实验室集体出游,
邀请无法回家的同学来您家中,共度中秋、元旦和除夕。
为师者常有,亦师亦友却可遇不可求,衷心感谢您对我在科研上的指导和生活中的关心,
这是我此生莫大的荣幸。

特别感谢同一个课题小组的三位学弟,骆徐圣、陈显扬和林封利。
很庆幸几年来能和你们一起努力,每一次通宵赶论文有你们的陪伴,
每一篇发表成果都离不开你们的倾力付出。
骆徐圣和我合作三年,不仅科研一起奋战,生活中也是非常好的朋友,
在各方面给了我非常多的鼓励,也与我分享对人生、情感的感悟,
相信这些都会对我一生受用。

%特别感谢三位co-author  xusheng,xianyang,fengli
%1. 通宵赶ddl的友谊
%2. xusheng对我影响最大,除了科研上的思想交流,生活中给了我很多帮助和启发,
%对人生、对情感方面,都会一生受用。
感谢实验室的三位师兄,赵凯祺、蔡智源、王拯,不仅带领我走进科研的殿堂,
而且让我更快融入实验室大家庭,至今依旧怀念当年ABG的美好时光。
感谢其他几位博士生,罗志一、刘乙竹、黄姗姗,
在实验室朝夕相处,怀揣共同的理想,面临相似的困难,平日互相支持与鼓励,和彼此都有着深厚的友谊。
感谢姜凯、孙伟、赵天宛、方文静、龚禹、徐栋、沙雨辰、梁玉鼎、唐洋洋、许方正、林禹臣、章梦雪、黄圣蕾、张海军、贾琪
以及更多的ADAPTers,
和大家相处非常愉快,从大家身上学到了许多。%你们也始终给予我肯定与包容,

感谢我的室友李冉,不仅有着共同的爱好与话题,
平日也经常畅聊对未来生活的规划与思考,让我对科研之外的生活有了更多的理解。
感谢王珏、孔奎权、王小乐等光彪楼演唱厅的小伙伴们,
很欣慰能被你们亲切称呼一声``{大师兄}'' ,
你们的存在让我的博士生活不再单调。

最后,特别感谢我的父母和所有家人,
在博士的求学之路中,是你们给予我无条件的包容和支持,
让我无需担心生活上的种种压力。
在我经历科研挫折、一度失去自信的时候,
总能在家中给我鼓励,让我能放下包袱,继续奋力前行。

%感谢自己
%"struggling for PhD degree" 大家都笑了 挣扎
%见识到了自己的能力,挫折、战胜挫折

\end{thanks}
