%# -*- coding: utf-8-unix -*-
% !TEX program = xelatex
% !TEX root = ../thesis.tex
% !TEX encoding = UTF-8 Unicode

\section{实验}%eval

本节主要介绍我们所使用的自动问答数据集,以及用于比较的已有问答模型。
具体实验包括在多个数据集上的端到端测试,以及一系列切除测试,%(ablation test)
用来分析方法中不同模块的重要性。

%In this section, we introduce the QA datasets and state-of-the-art systems
%that we compare.
%We show the end-to-end results of the KBQA task,
%and perform detail analysis to investigate the importance of different modules
%used in our approach.



\subsection{实验设置}%experimental setup



%TODO: QALD: too small, dismiss
\textbf{自动问答数据集:}
我们在实验中使用了三个开放领域的数据集,分别为ComplexQuestions\cite{bao2016constraint},
WebQuestions\cite{berant2013semantic}以及SimpleQuestions\cite{bordes2015large},
对应缩写为CompQ,WebQ和SimpQ。
CompQ数据集来源于Bing搜索引擎日志,一共包含2,100个具有复杂语义的问题,
以及人工标注的答案,
%大致的类型比例
前1,300个问句为训练集,后800为测试集。
WebQ数据集收集了5,810个通过Google Suggest API抓取的问题,以及对应的人工标注答案,
约有15\%的问句为复杂语义,同样数据集被分为3,778句训练集,以及2,032句测试集。
SimpQ一共包含108,442个具有简单语义的问句以及标注的答案,
答案形式为\textless 相关实体,谓词\textgreater 对,
%哪来的
我们主要利用该数据集进行补充实验,验证回答复杂问题的模型在简单语义场景中的性能。
对于其它自动问答的数据集,例如QALD,由于测试集数量过小,我们没有在这之上进行实验。
%WebQ/CompQ/SimpQ的下载地址呢

%\noindent
\textbf{知识库:}
对于在CompQ和WebQ上进行的实验,
我们跟随文献\parencite{berant2013semantic,xu2016question}的实验设置,
使用完整版本的Freebase\footnote{
该版本具体数据可从\url{https://github.com/syxu828/QuestionAnsweringOverFB}下载。}
作为知识库,共包含约46,000,000个不同实体,以及5,323种不同谓词。
同时通过开源图数据库Virtuoso\footnote{\url{http://virtuoso.openlinksw.com}.}
对Freebase进行访问与查询。
%\footnote{
%We remove predicates in \textit{user}, \textit{base} and \textit{freebase} domain,
%as well as predicates whose objects are neither entities or literals, 
%like \textit{common.topic.article} and \textit{common.topic.image}.
%}
对于SimpQ上进行的实验,我们使用数据集中提供的FB2M知识库,
它是Freebase的一个子集,包含大约2,000,000个实体和10,000,000个事实三元组。


\textbf{模型实现及调参细节:}
对本节中的所有实验,我们使用基于GloVe\cite{pennington2014glove}
预训练的词向量作为模型词向量矩阵的初始化。
词向量维度$d$,以及双向GRU层的隐藏状态维度均设为300。
损失函数中的$\lambda$的调参范围为\{0.1, 0.2, 0.5\},
实体链接优化的集成阈值$K$范围为\{1, 2, 3, 5, 10, +INF\},
训练批量大小$B$范围为\{16, 32, 64\}.



\subsection{端对端实验比较}%eval. result

我们首先对WebQ和CompQ数据集进行端到端测试。
实验所使用的评价指标为所有测试问题的平均$F_1$分数。
%具体答案怎么生成,最好的schema
Berant等人\cite{berant2013semantic}提供的官方评测代码\footnote{
\url{http://www-nlp.stanford.edu/software/sempre}.
}通过预测答案和标准答案的完全字面匹配计算每个问题的$F_1$分数,
对于CompQ数据集,其中标注的实体名称和Freebase内实体名称存在大小写不一致的情况,
因此我们参照Bao等人\cite{bao2016constraint}的做法,
计算$F_1$分数时忽略大小写。
% talk shorter??
通过对验证集进行调参,WebQ数据集的实验参数为$\lambda=0.5$,$B=32$,$K=3$,
CompQ数据集的参数为$\lambda=0.5$,$B=32$,$K=5$。


\tabref{tab:compqa-e2e}列出了在两个数据集上的具体实验结果。
%比较的都是些什么东西 不只是SP,但基本不是KGE(可以聊一下)
%是不是可以考虑把SP+NN的方法标记出来
%we mainly compare with SP or SP + NN methods, as close to our method,
%and competitive among various approaches.
Yih等人\cite{yih2015semantic}在CompQ上的实验结果基于Bao等人\cite{bao2016constraint}
对其模型的实现。
%无法复现,没有代码
在CompQ数据集上,我们提出的神经网络模型超过了其它已有方法,
将平均$F_1$分数提升了1.9,
而在WebQ数据集上,与大量已有工作进行对比,我们的模型排在第二位,
文献\parencite{jain2016question}基于记忆网络模型,成为分数最高的系统,
其方法并不基于语义解析,
无法直观解释一个答案是基于怎样的语义而生成,
并且问答过程涉及的隐含语义与单一谓词路径相似,难以应对类型、时间、顺序等语义限制。
%less interpretable
需要指出的是,Xu等人\cite{xu2016question}利用维基百科的非结构化文本进行
候选答案的验证,过滤掉满足主路径语义,但不匹配剩余语义的答案。
由于此方法引入了大量由人工社区提供的额外知识,它达到了一个略高于我们方法的分数(53.3),
但将此步骤去掉之后,模型分数跌落至47.0。
此外,文献\parencite{yih2015semantic,bao2016constraint}
额外使用了ClueWeb数据集\cite{gabrilovich2013facc1}
学习谓词与自然语言词组之间的语义匹配关系。
根据Yih等人公布的比较结果,
把这一部分信息移除之后,WebQ数据集上的$F_1$分数将下降了约0.9。
此外,结果显示,扩充实体链接可以进一步提升问答系统的整体性能,
在两个数据集上都获得了大约0.8的提升,是对语义匹配模型的一个良好补充。
我们认为,和其它使用了S-MART链接工具的问答系统相比,%此处可以cite
我们的结果可以与之直接比较,
这是因为S-MART的算法同样基于维基百科的半结构化信息进行学习,
例如重定向链接、消歧义页面、锚文本%anchor text
等信息,实体链接扩充的步骤没有并没有引入额外的知识,
因此可以直接比较。
%还有好几个可以说明的点

\begin{table}[ht]
    \centering
    \bicaption{CompQ和WebQ数据集上的实验结果,评价指标为平均$F_1$分数}
              {Average $F_1$ scores on CompQ and WebQ datasets.}
    \begin{tabular} {l|c|c}
        \hline
        Method  &   CompQ  & WebQ \\
        \hline
        Dong et al. (2015)    \parencite{dong2015question}            &   -   & 40.8  \\
        Yao et al. (2015)     \parencite{yao2015lean}                 &   -   & 44.3  \\
        Bast et al. (2015)    \parencite{bast2015more}                &   -   & 49.4  \\
        Berant et al. (2015)  \parencite{berant2015imitation}         &   -   & 49.7  \\
        Yih et al. (2015)     \parencite{yih2015semantic}             & 36.9  & 52.5  \\
        Reddy et al. (2016)   \parencite{reddy2016transforming}       &   -   & 50.3  \\
        Xu et al. (2016)      \parencite{xu2016question} (w/o text)   &   -   & 47.0  \\
        Bao et al. (2016)     \parencite{bao2016constraint}           & 40.9  & 52.4  \\
        Jain (2017)           \parencite{jain2016question}            &   -   & \textbf{55.6}  \\
        Abujabal et al. (2017)\parencite{abujabal2017automated}       &   -   & 51.0  \\
        Cui et al. (2017)     \parencite{cui2017kbqa}                 &   -   & 34.0  \\     
        Hu et al. (2018)      \parencite{hu2018answering}             &   -   & 49.6  \\
        Talmor et al. (2018)  \parencite{talmor2018web}               & 39.7  &   -   \\
        \hline
        Ours (w/o linking enrich)       & 42.0  & 52.0  \\
        Ours (w/ linking enrich)        & \textbf{42.8}  & 52.7  \\
        \hline
    \end{tabular}
    \label{tab:compqa-e2e}
\end{table}
%TODO: if time allows, stat. the F1 of simple / complex questions in WQ.
%TODO: also try to talk about enrichment analysis.


针对语义匹配本身,我们在SimpQ数据集上进行了测试。
由于SimpQ提供了标注的相关实体,我们可以消除实体链接步骤带来的差错,
单独衡量语义匹配的性能。
我们根据相关实体的名字,倒推出它在问句中对应的短语,
将其替换为\textless E\textgreater 之后,预测问句所表达的知识库谓词,
使用准确率作为评价指标。
\tabref{tab:compqa-simpq}列出了具体的实验结果。
相关文献主要针对简单问题,尝试了许多模型变种,
例如文献\parencite{qu2018question}的准确率最高,
该模型利用循环神经网络对问句语义进行建模,
同时利用卷积神经网络,从问句和谓词名称的词级别二维相似度矩阵中学习隐藏匹配样式。
文献\parencite{yu2017improved}使用了双层双向LSTM网络对问句进行编码,
并在两层中使用残差连接方式捕捉不同粒度的语义。
我们的语义匹配准确率略低一些,
考虑到重点在于多个语义成分的组合,而不是回答简单问题,
我们的模型更加轻量,同时93.1\%的准确率也确保了模型的有效性。

\begin{table}[ht]
    \centering
    \bicaption{SimpQ数据集上的语义匹配测试结果}{Accuracy on the SimpQ dataset.}
    \begin{tabular} {l|c|c}
        \hline
        Method  &   Relation Inputs     & Accuracy   \\
        \hline
        BiLSTM w/ words             & words         & 91.2 \\
        BiLSTM w/ rel\_name         & rel\_name     & 88.9 \\
        Yih et al. (2015) \parencite{yih2015semantic}     & char-3-gram   & 90.0 \\
        Yin et al. (2016) \parencite{yin2016simple}       & words         & 91.3 \\
        Yu et al.  (2017) \parencite{yu2017improved}      & words+rel\_name    & 93.3 \\
        Qu et al.  (2018) \parencite{qu2018question}      & words+rel\_separated    & \textbf{93.7} \\
        \hline
        Ours                        & words+path    & 93.1 \\
        \hline
    \end{tabular}
    \label{tab:compqa-simpq}
\end{table}

\subsection{模型分析}%ablation

本节主要对模型的各个主要进行分析测试,并讨论模型回答错误的一些例子。


\subsubsection{谓词路径表示}
%word和id repr,换个名字可好
我们改变模型对谓词路径的编码方式,并在CompQ和WebQ上进行分析测试。
首先对于谓词名字序列,我们尝试使用双向GRU层
(和问句编码部分结构一致,但不共享参数)拼接隐藏状态的方式
替代词向量平均。
对于谓词编号序列,我们将对路径整体编码方式改为谓词向量的平均。
%谓词向量好像之前没提过

实验结果如\tabref{tab:compqa-abl-pw}所示。
观察发现,前三行的基线方法移除了名字序列或编号序列,
在两个数据集上的$F_1$分数明显低于后三行的方法。
这说明了谓词的名字序列和编号序列所提供的语义可以互相补充。
另一方面,对比最后两行实验,
在CompQ数据集上,对名字序列使用词向量平均要优于使用双向GRU,
而在WebQ上,这个差距变得更小,
我们认为原因主要来自于训练数据量的区别,
WebQ的训练集大小约为CompQ的三倍,
因此可以支持更复杂的模型。
%考虑到avg还是比RNN要好,这里确定不狠踩一脚吗?
%Since the number of distinct predicate sequences are limited,
%leveraging both w and id outperforms other approaches.
%Since the usage of words in FB could be slightly different from NL scenario,
%as path embedding is fitting the residues between q and relation names,
%and the repr of word and id are more likely to be complementary to each other.
%Meanwhile,

\begin{table}[ht]
    \centering
    \bicaption{对谓词表示的分析结果。}{Ablation results on path representation.}
    \begin{tabular} {c|c|c|c}
        \hline
        Word repr.  &  Id repr.  &   CompQ $F_1$  & WebQ $F_1$ \\
        \hline
        None        &  PathEmb  &   41.11   & 51.86 \\      %%  XH
        Average     &  None     &   42.18   & 51.74 \\      %   BX
        BiGRU       &  None     &   41.80   & 51.87 \\      %   RX
        Average     &  Average  &   42.16   & 52.00 \\      %   BB
        BiGRU       &  PathEmb  &   41.52   & 52.33 \\      %   RH
        Average     &  PathEmb  &   \textbf{42.84}   & \textbf{52.66} \\      %   BH
        \hline
    \end{tabular}
    \label{tab:compqa-abl-pw}
\end{table}


%%Ablation 1: Bao / Sep / Comp. (20:00)
%%Kernel: what's compact / separate / bao's difference
%%Kernel: 
%%Ablation 2: Q- encoding, compare with SimpQ, if possible (21:40)
%%What dependency can do and what they can't.
%%can: syntactic information (functional), as compression; can't: lose information
%%Example: "end up marrying" "gain independence from" ...
%\textbf{Question representation:}
%\tabref{tab:abl-qw} shows the ablation result on all the datasets.
%when dependency path information is augmented with sentential information,
%the performance boosts by relatively xx.x on average.
%introducing strong syntactic and functional features,
%also local features (like attention)
%however, performances drops by xx.x if only use dependency,
%crucial words not in the path: such as 
%"gain independence from"
%"end up marrying"
%%What dependency can do and what they can't.
%%can: syntactic information (functional), as compression; can't: lose information
%%Example: ``end up marrying`` ``gain independence from`` ...
%\textbf{Question representation:}
%\tabref{tab:abl-qw} shows the ablation result on all the datasets.
%when dependency path information is augmented with sentential information,
%the performance boosts by relatively xx.x on average.
%introducing strong syntactic and functional features,
%also local features (like attention)
%however, performances drops by xx.x if only use dependency,
%crucial words not in the path: such as 
%``gain independence from``
%``end up marrying``


\subsubsection{问句表示及语义组合}
为了说明语义组合的有效性,我们建立一个基线模型:
不使用\eqnref{eqn:maxpool}对应的最大池化操作,
替代方式是分别计算每个问句表示和每个语义成分之间的相似度,
并将各部分相似度分值相加,作为查询图与问句的整体相似度:
$S_{rm}(q, G) = \sum_{i}{cos(\bi{p}^{(i)}, \bi{q}_p^{(i)})}$。
对于问句的编码方式,我们进行一系列比对实验,
观察不使用字面序列或依存语法路径对整体性能带来的影响。


\tabref{tab:compqa-abl-qw}显示了在CompQ和WebQ上的具体比较结果。
相比仅使用问句字面信息的模型,当依存语法分析提供的路径信息被使用后,
问答系统整体性能平均提升了0.42。
在隐藏语义的角度,答案和相关实体之间的依存语法路径主要包含了
词之间的语法依赖,以及每个词的功能化特征,
是对整个问句序列信息的良好补充。
然而,如果对问句编码只使用依存语法信息,$F_1$分数会大幅度下降约2.17。
对于具有特殊语法结构的问题,如果仅关注疑问词和实体短语间的路径,
会使得模型丢失句中表达语义的关键词,
例如以下两例:
``who did \textit{draco malloy} end up \textbf{marrying}'' 以及
``who did the \textit{philippines} gain \textbf{independence} from'' ,
其中相关实体用斜体标出,代表语义的关键词为粗体。
经过观察发现,WebQ中大约有5\%的问句具有类似的结构,
在丢失关键语义信息后很难预测出正确的查询图。


语义组合的比较结果显示,模型中使用的最大池化操作要一致优于对应的基线方法。
在WebQ上的提升要低于CompQ,
主要原因是WebQ中约85\%的问句依然是简单语义形式,
无法体现语义组合的区别。
移除依存语法信息和池化操作的模型可以视为一个基础的
利用深度学习改善语义解析的问答模型。
在复杂语义场景中,局部信息和语义组合的引入,
两者结合使得CompQ数据集上效果提升1.28。


我们通过以下例子,进一步阐述模型中语义组合带来的优势。
给定问句``who is gimli's father in the hobbit'' ,
由于``gimli'' 的实体链接结果中既存在自然人,也存在名字一样的虚拟角色,
我们主要关注下面两个可能代表真实语义的查询图:
%此处可以画图填充
\begin{enumerate}
    \item ($?$, $children$, $gimli\_person$);
    \item ($?$, $fictional\_children$, $gimli\_character$) $\wedge$ ($?$, $appear\_in$, $hobbit$)。
\end{enumerate}
两个查询图涉及到三个不同的语义成分,
如果独立观察其中每一个语义成分,谓词$children$与问句整体的匹配程度最高,
因为``father'' 一词包含了很强的语义信息,训练数据中也包含较多``'s father'' 和$children$的关联,
因此它们的关联特征容易被学习。
相比之下,$fictional\_children$过于生僻,而$appear\_in$与``father'' 无关联,
这两个语义成分的相似度远不如$children$,因此基线模型认为第一个查询图更加正确。
%有没有办法show出两条边各自的分数?
而我们的模型中,不同语义成分的隐藏特征通过池化方式汇集起来,
分别将各自突出的隐藏语义传递出去,构成查询图整体的语义向量。
与单独的$children$语义向量相比,查询图整体语义能兼顾
与``'s father'' 以及``in the hobbit'' 匹配,
因此模型能正确预测第二个查询图为答案。


%To demonstrate the effective of this part of our model,
%we construct two alternative baselines.
%For the first baseline, we remove the max pooling operation (\eqnref{xx}) and
%calculate the cosine similarity of each individual component,
%then sum them together as the relation matching score:
%$s(q, p) = sum blabla$.
%The second baseline is inspired by \citet{bao2016constraint}:
%the output of relation matching module is a 5-dim vector
%serving as rich relation matching features in the final layer.
%Each value in this vector indicates the sum of similarity scores
%between the question and the semantic component in 5 different categories:
%main, entity, type, time, ordinal, respectively.
%As the results shown in \tabref{tab:abl-sem},
%we observe that
%there's a stable gap between our approach and the first baseline,
%%TODO: t-test if possible
%showing that our model is able to capture the semantic interaction between components,
%rather than treating the query structure as a set of isolated components.
%We also point out that for both baselines,
%shared paths between positive and negative query structures are canceled out,
%as a result, the learning step cannot make full use of training pairs.
%% Bao lower than sep? not sure.
%
%%talk about "gimli's father", using figures if possible
%
%
%%Ablation 2: Q- encoding, compare with SimpQ, if possible (21:40)
%%What dependency can do and what they can't.
%%can: syntactic information (functional), as compression; can't: lose information
%%Example: "end up marrying" "gain independence from" ...
%\textbf{Question representation:}
%\tabref{tab:abl-qw} shows the ablation result on all the datasets.
%when dependency path information is augmented with sentential information,
%the performance boosts by relatively xx.x on average.
%introducing strong syntactic and functional features,
%also local features (like attention)
%
%however, performances drops by xx.x if only use dependency,
%crucial words not in the path: such as 
%"gain independence from"
%"end up marrying"
%
%webq: more simple questions (80\%) not big difference between sep and comp
%compq: significant gap

\begin{table}[ht]
    \centering
    \bicaption{问句表示和语义组合的分析测试。}{Ablation results on question representation and compositional strategy.}
    \begin{tabular} {c|c|c|c}
        \hline
        Composition     & Q\_repr   &   CompQ $F_1$   & WebQ $F_1$ \\
        \hline
        Baseline        &   sentential    &   41.56   & 52.14 \\
        Baseline        &   both          &   42.35   & 52.39 \\
        \hline
        Ours            &   dependency    &   41.48   & 49.69 \\
        Ours            &   sentential    &   42.59   & 52.28 \\
        Ours            &   both          &   \textbf{42.84}   & \textbf{52.66} \\
        \hline
    \end{tabular}
    \label{tab:compqa-abl-qw}
\end{table}




\subsubsection{错误分析}
%Analyze the cases where not the highest result is returned.
%take compQ as example.
%
%1. entity linking error
%several small categories
%
%2. relation matching error
%heat map
%
%3. not perfect (missing edges)
%
%
%percentage
%example:
%what's wrong
%what's right
%reason

我们从CompQ数据集中完全回答错误的问题中随机挑选100个例子进行分析,
并归纳出下列几类错误原因。

\emph{主路径错误} (10\%):
模型完全没有理解问句语义,哪怕最主要的语义也没有预测出来。
这类错误对应的问题通常较难回答,例如
``What native american sports heroes earning two gold medals in the 1912 Olympics'' 。%10/100 e.g. 1417

\emph{语义限制错误} (42\%):
模型预测的查询图中包含正确的主路径,但其余语义限制存在偏差。
比较典型的一类限制是隐含时间限制,例如问句
``Who was US president when Traicho Kostov was teenager'' 无法准确回答,
因为``when Traicho Kostov was teenager'' 暗示了时间限制,
受限于候选生成方法,这类限制无法被识别。
%这个例子有问题啊,讲道理主路径对的话,F1不会为0的
%你确定比例这么高??这可是完全错误的例子诶
%35/100, e.g. 1930 1843

\emph{实体链接错误} (16\%):这类错误的主要原因是问句中的一些实体词组具有高度歧义。
例如问句``What character did Robert Pattinson play in Harry Potter'' ,
而``Harry Potter'' 可以对应7部不同的电影,因此很难猜测问句中指的是哪一部。
%15/100  e.g. 2064 1664

\emph{杂项} (32\%): 包含了一些较明显的答案标注错误,以及问题本身语义不明确或不合逻辑。
例如问句``Where is Byron Nelson 2012'' ,
根据标注答案可以帮助确定问句中``Byron Nelson'' 的具体所指,
然而此人已于2006年去世,因此该问题的真实意图难以捉摸,
或许提问者想问的是他的逝世地点,或葬于何处。%25/100 e.g. 1301 1947


