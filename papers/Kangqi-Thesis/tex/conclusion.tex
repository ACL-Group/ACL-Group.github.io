%# -*- coding: utf-8-unix -*-
% !TEX program = xelatex
% !TEX root = ../thesis.tex
% !TEX encoding = UTF-8 Unicode

\chapter{总结与展望}
\label{chap:conclusion}

\section{论文工作总结与主要贡献}

自然语言理解是人工智能的重要分支。
如何让机器理解人类语言的含义,是一系列任务的研究重点,
尤其是对于问答系统、阅读理解、多轮对话等下游任务,
它们都依赖于机器对语义的充分认知。
伴随着互联网中海量结构化信息积累,
知识库的诞生和相关技术的发展给自然语言理解
提供了一种有效的解决方案,
即以知识库中的实体、类型和谓词为载体,
描述自然语言中的实体、实体间的关系,甚至蕴含多个关系的复杂句子。
在此背景下,本文对基于知识库的自然语言理解分为三个递进的层面,
即实体理解、关系理解和问句理解。
针对这三个层面理解问题,本文展开了一系列研究,
并提出了具有针对性的语义匹配模型。


实体理解的目标,是将自然语言文本中表示实体的短语映射至知识库的对应实体,
是一种直接匹配的过程。
本文进行了中文到英文的跨语言场景中,对表格文本进行链接的研究。
%和传统的实体链接任务相比较,
表格链接过程中,同行列的实体具有明显的相关性,
这是传统实体链接任务所不具备的特性,
也是链接模型的关注重点。
而知识库和链接文本不在同一个语言中,使得模型无法利用任何字面上的相似信息,
这给链接任务带来了更多挑战。
本文是学术界首次研究跨语言的表格链接任务,
本文提出了基于跨语言词向量和深度神经网络的链接模型,
目标在于克服翻译步骤带来的错误传播,以及自动学习不同粒度的语义匹配特征。
具体而言,本文提出的方法贡献如下:
\begin{enumerate}
\item{候选实体生成中,利用多种翻译工具进行过渡,并保留足量候选,将黑盒翻译工具出错的影响尽可能降低;}
\item{训练跨语言词向量,使得中英文单词、实体的特征表示在连续语义空间中互通,
保证在不依赖字面相似特征和共现统计特征的情况下,实现高质量的链接;}
\item{定义了三种语义匹配特征,即单个单元格到实体的指示特征,
单元格行列信息到实体的上下文特征,及同列实体之间的一致性特征,
通过神经网络对三类特征进行表示学习,并提出了逐位方差进行一致性特征计算的方式;}
\item{模型遵循联合训练框架,以整张表格级别的匹配程度作为目标函数,
并利用基于成对排序损失的RankNet进行训练,充分利用负样本表格生成产生的偏序关系;}
\item{实验表明,本文提出的模型在跨语言表格链接任务中明显优于其它基线模型,
同时模型对一致性特征的建模以及联合训练框架均带来实质性的帮助。}
\end{enumerate}


关系理解的目标,是将自然语言中的二元关系通过知识库中的谓词进行表示。
相对于实体理解的直接匹配过程,关系理解较难做到二元关系和谓词的一一对应,
一方面在于关系的多义性,更重要原因在于知识库和自然语言之间存在语义间隔,
使得一些语法简单的关系,在知识库中却对应复杂的语义。
基于这两个不同的挑战,本文对二元关系进行了两种不同粒度的研究。

粗粒度的关系语义研究中,本文旨在分析关系在大跨度上的多义性,
挖掘关系的主语和宾语所具有的不同类型搭配。
本文提出了挖掘关系具有代表性类型搭配的方法,其思路在于尽可能使用
具体的类型匹配更多的已知关系三元组,主要贡献列举如下:
\begin{enumerate}
\item{提出了一种主宾语联合进行实体链接的方式,利用关系名称和主宾语间谓词路径存在的关联特征,
提升整体链接准确率;}
\item{去除关系名称中不影响类型搭配的成分,并利用语法变换将相似语义关系归为一组,
使长尾关系能够被有效利用;}
\item{利用松弛类型包含构建更丰富的知识库类型层次关系,并可用于其它任务中;}
\item{人工测评实验表明,本文提出的方法可以改善互信息模型对热门类型搭配的惩罚情况,
同时推理出的代表性的类型搭配也具有不错的质量。}
\end{enumerate}

细粒度的关系语义研究中,本文旨在深入挖掘关系语义的精确表达,
定义了具有树形结构的模式图,它是知识库中满足特定语义的子图的抽象表达,
同时具有良好的可解释性。
本文提出了基于复杂模式图的规则推导模型,
由已知关系三元组出发,挖掘语义相近的候选模式图,
并学习它们的概率分布,从而以结构匹配的形式描述关系语义,并运用于知识库补全任务中。
本文提出的方法贡献如下:
\begin{enumerate}
\item{定义了具有 ``{路径+分支}'' 结构的模式图,它是对传统规则推导模型中,
基于谓词路径形式的规则扩展,对复杂语义关系具有更强的表示能力;}
\item{利用深度优先搜索采集不同的模式图,并通过优先队列实现搜索过程的高效剪枝,
在获取和关系语义较为接近的模式图同时,维持不同模式图间的多样性;}
\item{将二元关系语义表示为候选模式图上的概率分布,可以更好地应对关系的多义性,
同时任何一个查询图自身都具有独立的描述能力,使人类易于理解;}
\item{模式图概率通过生成模型学习,实现了宽泛和具体模式图之间的平衡;}
\item{多个自然语言关系的模式图实例表明,基于模式图的结构有能力准确描述复杂关系
语义,并且质量显著好于其它基于路径的规则推导模型;}
\item{本文提出的模型能有效运用于知识库补全任务中,在主宾语预测和三元组分类两个子任务上,
效果优于其它规则推导模型,以及新兴的知识库向量模型。}
\end{enumerate}


问句理解的目标,是学习问句和答案之间的推理匹配。
本文关注于通过知识库回答客观事实类问题,
由于单个问句可能包含未知答案和其它实体的多个关系,
和语义仅对应单个谓词的问句相比,复杂问句的回答更具有挑战性,
体现在如何对复杂问句进行语义描述,以及如何度量和问句的语义匹配程度。
针对以上挑战,本文提出了面向复杂语义问句的问答模型。
对于问句的语义表示,本文沿用关系理解中的模式图思路,
由问句出发生成可解释性高的查询图,
以表示答案实体与问句中多个相关实体、类型、时间等信息的关联。
同时,模型通过神经网络训练问句与查询图的匹配程度,
为复杂查询图整体学习连续空间中的特征表示,捕捉不同成分间的语义交互。
具体贡献如下:
\begin{enumerate}
\item{沿用模式图思路,利用多阶段生成方式构建问句的候选查询图,并在前人基础上
对类型语义限制和时间语义限制进行改进;}
\item{提出了一个轻量级的神经网络模型,以计算问句和查询图的语义匹配程度,
据我们所知,这是知识库问答研究中首次尝试学习复杂查询图整体的连续语义表示;}
\item{对问句的表示学习引入依存语法路径,作为问句字面序列信息的补充,
以体现问句与特定语义成分的关联;}
\item{通过集成方法,对已有实体链接工具的结果进行扩充,在链接准确率不受较大影响的前提下,
提升候选查询图的召回;}
\item{本文提出的模型在复杂问题数据集上取得了最优的效果,在简单问题数据集上依然保持竞争力,
更多对比实验显示,学习查询图整体的连续特征表示有助于提升问答系统的效果。}
\end{enumerate}


\section{未来工作展望}

由于时间关系,本文的工作中还存在一些没有得到解决的问题,列举如下:

\begin{enumerate}
\item{表格链接,以及关系三元组的实体链接中,都存在着无法链接到具体实体的短语。
除了较容易识别的数字、时间以外,考虑到知识库并不完整,
部分实体(尤其是人名)不存在于知识库中,此时模型需要识别出这样的短语,而不是强行链接。
我们对表格链接的任务定义绕开了此问题,而对三元组的实体链接则忽略了这种情况,
这是一个需要改进的方向。}
\item{关系三元组的链接方式较为粗糙,采用了主谓宾各自匹配度连乘的方式,
并没有使用模型训练各部分权重。
\secref{sec:tinf-approach}提到的集成链接方案并不是最优的解决办法,
未来将利用神经网络表示三元组各自成分的链接特征,从而提升这一步骤的准确率。}
\item{知识库问答研究中,我们尝试使用注意力层\cite{bahdanau2014neural}取代依存语法序列,
让语义匹配模型自动学习和特定谓词最相关的问句短语,但实验显示注意力层对问答指标几乎没有改进。
一个可能的解释是,输入的问句长度大多在10左右,而不是类似一段话的形式,
因此注意力模型效果不明显。在今后的研究中,会在这个问题上继续调研。}
\end{enumerate}
%尚未解决的问题 
%1. TabEL 和tinf  的unlinkable   两者都会碰到这个问题
%2. tinf  方法简单  没有学习重要性
%3. OpenIE的优化 patty  筛东西
%4. attention

此外,在未来的研究工作中,我们以问句理解为核心,关注以下两个主要研究问题。

关系理解和问句理解具有很高的相关性。
给定问句中的二元关系,若已知其主宾语类型搭配,
那么对于候选查询图而言,答案类型与类型搭配的查询图更有可能表示了正确的语义。
类似地,二元关系所对应的模式图也可指引问句查询图的排序,提供额外的匹配特征。
我们在过去的工作中,对主宾语类型搭配与自动问答的结合进行了一定的尝试,
但效果提升有限,除了类型搭配本身出现偏差,
将问句与特定二元关系的对应是另一个瓶颈。
基于语法转换的方式进行映射过于确定,由于用户提问可能不具有严谨的语法,
可能需要使用更加灵活的方式实现这一对应。
在未来的研究中,我们将尝试由陈述句出发生成疑问句,并引入一定的非严谨语法形式,
以此构建训练数据,学习更加准确的问句到二元关系的映射。
真实问答系统的问答对数据(例如Yahoo Answers,以及大量FAQ资源)可以帮助对不规范的语法进行
理解和近似,使模型学习启发性的知识。

在现有的问答模型中,候选结构的生成过程是一次性的,
对于测试问句,必须先生成所有查询图,再从中挑选最匹配的结构。
为了保证候选生成速度,搜索规模需要受限,例如主路径长度限制为2,
对于某些特殊问句,则无法生成出正确的查询图。
因此,一种可能的改进方式,是将查询结构的生成看做序列,
通过使用序列到序列模型,以问句为输入,输出查询图的生成序列。
Golub等人\cite{golub2016character}使用了这样的模型用于回答简单问句,
而Jain\cite{jain2016question}使用记忆网络模型在WebQuestions上取得了最佳的效果,
其模型的多层设计暗含了谓词的多步跳转。
对于复杂问句,虽结构复杂,但多阶段生成过程很容易转换成序列形式,
如何将复杂语义结构与序列到序列模型结合,是未来的一个研究方向。



%1. qa和tinf的结合
%2. 融合  kbe和qa
%3. candgen beam search风格




