%# -*- coding: utf-8-unix -*-
% !TEX program = xelatex
% !TEX root = ../thesis.tex
% !TEX encoding = UTF-8 Unicode

\subsection{候选实体生成}
\label{sec:tabel-candgen}

%To generate the candidate entity table,

我们对中文表格$X$的每一个单元格内容生成一系列英文知识库中的候选实体。
在本章的研究中,我们使用英文维基百科作为知识库。
由于提出的方法不使用任何中文知识库进行过渡,
为了实现语言转换,我们首先利用已有的翻译工具生成中文词组对应的多种翻译结果。
接下来,对于每一个翻译结果,我们都使用预先定义的启发式规则,将英文词组转换为候选实体。
这些实体的来源主要包括:
1) 名称与翻译完全匹配的实体;
2) 维基百科中,完全匹配的锚文本所指向的实体;
3) 通过计算编辑距离(Edit Distance)进行模糊匹配,并且相似度足够高的实体。
%可以具体讲一下细节
以中文词组 ``{疑犯追踪}'' 举例,不同的翻译工具生成的结果不同,
例如 ``person of interest'' 或者 ``suspect tracking'' 。
整体候选实体来自于每一个翻译结果的映射,例如维基百科中的实体
``person of interest'' , ``person of interest (tv series)'' 以及 ``suspect (1987 film)'' 。

