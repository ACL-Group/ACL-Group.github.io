%# -*- coding: utf-8-unix -*-
% !TEX program = xelatex
% !TEX root = ../thesis.tex
% !TEX encoding = UTF-8 Unicode

\section{任务规范定义}
\label{sec:tabel-problem}

%The task of cross-lingual table linking: table info & entity info

%1. table info
%The task of cross-lingual table linking involves two kinds of information:
%mention tables and entities in the knowledge base.

输入的互联网表格$X$是一个具有$R$行和$C$列的矩阵,
每一个单元格$x_{ij}$的内容是由语言$L_1$(例如中文)描述的词语序列。
给定由另一种语言$L_2$(例如英文)编写,并包含大量实体$e$的知识库$K$,
跨语言表格链接的任务是寻找$X$对应的目标链接表格$E$,
使得链接表格中的每一个实体$e_{ij} \in K$对应单元格$x_{ij}$内容的消歧义表示。

在具体场景中,输入的表格包括一些无法被链接的单元格,
例如数字、日期、时间以及一些知识库中尚不存在的新兴实体。
一些已有工作\cite{ibrahim2016making}主要负责在互联网表格中识别这些数字或时间实体,
因此在本章中,我们不关注一个单元格是否能被链接的判断方式。
具体到任务定义中,
$P$为输入表格中所有可以被链接的单元格坐标$(i, j)$所构成的集合,
并且我们假设在训练集和测试集中,
每个输入表格$X$对应的可链接位置集合$P$都是已知的。

传统的实体链接方法通常在模型中定义一个评分函数$S(x, e)$,
用于衡量文本$x$与目标实体$e$之间的相关程度。
在表格链接任务中,这样的做法等同于将不同的单元格分割开,单独计算相似度。
然而缺陷在于,相邻或是同行列的目标实体之间的交互完全无法体现在链接模型中。
为了将目标链接表格中不同实体间的耦合关系融入任务中,
我们定义了在表格层面的评分函数,并以此预测最佳的链接表格$\hat{E}$,
如下所示:%TODO: argmax 显示错误
\begin{equation}
  \label{eqn:joint-score}
  \hat{E} = \argmax_{E \in GEN(X)} S(X,E),
\end{equation}
\noindent
其中$GEN(X)$表示由$X$生成的所有候选链接表格。
该函数描述了输入表格与候选实体表格之间的整体相关性分数。
%可以在这里扯joint和non-joint的区别。
