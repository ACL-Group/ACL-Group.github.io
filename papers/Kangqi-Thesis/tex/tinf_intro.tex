%# -*- coding: utf-8-unix -*-
% !TEX program = xelatex
% !TEX root = ../thesis.tex
% !TEX encoding = UTF-8 Unicode

\subsection{引言}
\label{sec:tinf-intro}
% 1. give a example of ambiguous relation
% 2. talk about selectional preference
% 3. talk about open ie
% Goal: Why we do this work? What's the help of pairwise selectional preference?
% Reverb has high quality

% open ie part
% 1. recent years, open ie system played an important rule

开放式信息抽取(Open Information Extraction)任务的目标是从
从开放领域的文本语料库中挖掘命名实体或概念,并抽取出连接这些实体的
各种不同的自然语言关系。
之所以称为开放式抽取,是因为要挖掘的关系不局限于特定领域
也不基于固定的匹配规则。
学术界中,较为先进的开放式信息抽取系统
\cite{carlson2010toward,fader2011identifying,schmitz2012open,nakashole2012patty}
可以从海量互联网语料库中,以很高的准确率提取百万甚至更高级别数量的关系实例,
($arg_1$, $rel$, $arg_2$)三元组形式,我们将其称为关系三元组。
其中,$rel$为二元关系,
一般表示为短语(词级别描述)或依存语法路径(语法级别描述)。
$arg_1$和$arg_2$是关系的两个参数,即主语和宾语,同样表现为短语形式。


% 4. it's interesting to see the semantic information hidden in these tuples.
开放式信息抽取提供给我们海量关系实例的同时,
我们有兴趣将这些实例进行归纳,寻找更加抽象的语义表示。
我们关注的重点就是这些关系所具有的不同含义。
以关系 ``play in'' 为例,开放式信息抽取系统可以提供
一系列具有($X$, play in, $Y$)形式的三元组。
例如ReVerb系统\cite{fader2011identifying}可抽取出三元组
(Goel Grey, played in, Cabaret)以及(Tom Brady, play in, National Football League)。
%\begin{center}
%$\langle \text{Goel Grey}, played\ in, \text{Cabaret} \rangle$ \\
%$\langle \text{Tom Brady}, play\ in, \text{National Football League} \rangle$
%\end{center}
%本章的目标是通过已有的关系实例,自动推理出
给定某关系已有的三元组实例,我们可以推理出一系列
由类型三元组描述的关系模式,即主宾语类型搭配($t_1$, play in, $t_2$)。
其中$t_1$以及$t_2$为标准化的实体类型,其来源为含有类型定义的知识库,
例如WordNet\cite{miller1995wordnet},%如果还有别的地方提到了WN,那么这里就不要提了
Yago\cite{suchanek2007WWW},Freebase\cite{bollacker2008freebase}
以及Probase\cite{wu2012probase}。
每一个关系模式都可以用来表示一组特定的 ``play in'' 关系实例,
其中主宾语分别属于对应的类型。
对于上例 ``play in'' ,我们可以给出两个可能的模式:
($film\_actor$, play in, $film$),以及($pro\_athlete$, play in, $sports\_league$)。
%\begin{center}
%$\langle \textbf{film\ actor},\ play\ in,\ \textbf{film} \rangle$ \\
%$\langle \textbf{athlete},\ play\ in,\ \textbf{sports\ league} \rangle$
%\end{center}
%根据已有的关系模式,我们可以得知,
由此可见,二元关系 ``play in'' 具有明显歧义,
不仅可以描述 ``{运动员—体育联盟}'' 联系,还可以描述 ``{演员—电影}'' 之间的联系。
对于歧义较少的关系,我们依然可以推理出不同的
主宾语类型搭配,例如关系 ``is the mayor of'' 可以推理出
($person$, is the mayor of, $location$),以及($politician$, is the mayor of, $city$)
等不同模式,在类型上具有不同的粒度,后者显然更加具体。

对于自然语言理解任务,例如上下文相关的实体消歧,还有开放领域自动问答,
关系模式是一个有用的信息。
假设我们要对句子 ``\textit{Granger} played in \textit{the NBA}'' 进行实体识别。
``\textit{Granger}'' 对应一个人名,但由于只提供了姓氏,因此具有较高歧义。
而 ``\textit{the NBA}'' 几乎可以确定是人们熟知的体育联盟。
再结合上面列举的 ``play in'' 所具有的关系模式,
实体识别模型便可以获得额外特征,即 ``\textit{Granger}'' 更有可能代表运动员,
也就使得篮球运动员``Danny Granger'' 更容易被正确识别。
考虑到这个实体并不非常著名,与之相关的关系实例数量可能较少,
但类型特征依然可以提供很大的帮助。
%Suppose we've known the argument at one side of a relation, type pairs will help us inferring what kind of entities are more likely
%to occur at the other side.

%Structured knowledge base (KB) is a taxonomy containing real world entities, types, relations between entities
%and ``IsA'' relation between entities and types. Structured KBs such as WordNet \cite{miller1995wordnet},
%Yago \cite{suchanek2007WWW} and Freebase \cite{bollacker2008freebase} are widely used in information extraction
%and semantic learning tasks. In order to make relation schemas understood by human, we leverage types
%in the KB as the output of relation schemas.

%For the purpose of relation schema inference, any ontology or taxonomy of
%entities and concepts (or types) connected by isA relation can be
%used. Examples include WordNet~\cite{miller1995wordnet},
%Yago~\cite{suchanek2007WWW}, Probase~\cite{WuLWZ12}
%and Freebase~\cite{bollacker2008freebase}.
%In this paper, we choose to use Freebase as our target ontology
%Because Freebase~\cite{bollacker2008freebase} is a widely used
%community supported ontology for entity linking and question answering,
%this paper seeks to infer schemas using Freebase types.
%
%Freebase contains more than 40 million entities, and has
%a type hierarchical structure with more than 1,700 real types
%\footnote{Freebase types are identified by type id, for example, $sports.pro\_athlete$ stands for ``professional athlete''.}.
%Each entity belongs to at least one type.
%When compared with other knowledge bases, Freebase has a much greater focus on named entities than {\tt WordNet}.
%Besides, the type hierarchy of {\tt Yago} is too fine-grained, which is not suitable for schema inferring.
%Considering aspects mentioned above, we adapt Freebase as our knowledge base in our work.
%

%The most relevant technique to achieve our goal is
%\textit{selectional preference} (SP)~\cite{}, which computes the most
%appropriate type for a particular argument (e.g., subject or object) of a
%predicate. There are different approaches in computing SP. Class-based
%approach~\cite{resnik1996selectional} seeks to map each argument of a relation
%to entities in a taxonomy such as WordNet, and abstract the arguments into
%human readable types from the taxonomy. Other non-class based approaches
%\cite{erk2007simple,ritter2010latent} cannot produce human readable type names
%because they either rely on distributional properties or produce types as
%latent variables. As a result, non-class based methods are not suitable for
%inferring schemas which must be readable by humans. The approach proposed
%in this paper is a variant of class-based SP, whose primary difference is
%that the most preferred types for both arguments of a binary relation
%are computed simultaneously, rather than on each individual arguments.


%TODO:凑篇幅而论,这个地方肯定要讲具体的SP了。

为了生成关系模式,一种已有的方案是基于选择偏好(Selectional Preference)技术
\cite{resnik1996selectional,erk2007simple,ritter2010latent},
它可以对关系中的主宾语实体计算各自具代表性的类型。
选择偏好技术主要思路来自关系与类型之间的互信息计算\cite{erk2007simple},
这种方式倾向于选择当前关系所独有的类型,
换句话说,如果一个类型普遍适用于不同关系中的实体描述,
那么它便不容易被选为代表类型。
% add disadvantage of SP
然而在开放式信息抽取中,很多关系实际上是相关的,甚至非常相近,
例如 ``play in'', ``take part in'' 以及 ``is involved in'' 。
这些关系实际上具有相同的语义,因此主宾语的类型搭配也应该相似,
而选择偏好技术会因为关系的不同而对这些类型都进行弱化。
%TODO:好像还要加不少东西,SP也不可能放在外面讲

因此本章中,给定一个关系和一系列具体的三元组,
我们的任务是寻找那些最具体的类型搭配,而同时包含尽可能多的关系实例。
%Intuitively, when we human infer the schema for a relation,
%we prefer to choose those schemas which are suitable under most circumstances.
%Moreover, it would be better if a schema is more specific.
%Based on this idea, we propose an approach to infer best type
%schemas for binary relation.
我们的方法首先将关系实例中的主宾语映射为知识库中的实体,
即为每个三元组生成($e_1$, $e_2$)实体对。
接着根据不同实体所属的类型,寻找可以覆盖尽可能多实体对的类型搭配
($t_1$, $t_2$)。
最后,当不同的类型搭配覆盖的实体对较为接近或一致时,
我们利用知识库中已有的$IsA$关系,扩充知识库中类型之间的层次结构,
以此寻找更加具体的类型搭配。


%
%Learned by previous examples, the key challenge for relation type inferring
%is that, type distributions of both arguments
%should be modeled simultaneously, and the preference of different schemas should be comparable with each other.
%\KQ{If type distributions for $arg1$ and $arg2$ are modeled separately, we lose the information in combining.}
%As the previous example ``play in'' shows, if we only know preferred types for X and Y independently,
%it's hard to tell what kind of Y is likely to be when X is an actor.
%
%% sp part (intuition: get the human readable types)
%% 1. what is sp ?
%Selectional Preference (SP) is the technique to get certain types that is more likely than other
%types to be the argument of one relation.
%% 2. with such constraint, we can let the computer know whether a relation argument is suitable or not.
%%%%With this kind of type constraint, we can compare the possibility of types
%% 3. basic sp: resnik 1996 on wordnet
%One branch of SP is knowledge based, types are mapped to structured knowledge bases, such as WordNet \cite{miller1995wordnet}, Yago \cite{suchanek2007WWW} and Freebase \cite{bollacker2008freebase}. The earliest work in SP is proposed by Resnik \shortcite{resnik1996selectional}, which is based on WordNet.
%% 4. recently, sp with topic models
%The other branch is statistical based, argument types are generated by topic models like Latent Dirichlet
%Allocation \cite{blei2003latent}.
%% 5. leverage external taxonomy, for example, Freebase provide its type taxonomy, which is well-defined by ..., containing 1000+ useful types. (Comparing to Yago and DBPedia), and human readable.
%One main advantage of knowledge based SP is that, given a well-defined type taxonomy, the argument types of can be easily understood by human.
%Thus, our work is built on knowledge based SP.



%% 6. we can use fb ty%pes to represent result. (list the previous examples)
%Using Freebase type taxonomy as the external knowledge, for the previous two relations,
%the preferred relation schemas can be represented as:
%
%\begin{center}
%$\langle politician,\ is\ the\ mayor\ of,\ citytown \rangle$
%$\langle athlete,\ play\ in,\ sports\_team \rangle$
%$\langle film\_actor,\ play\ in,\ film \rangle$
%\end{center}
%
%\noindent
%Where argument types are corresponding to Freebase types.
%showing the different interpretations of relations.


% our contribution
% 1. our goal is to find different selectional preference for one relation. (located to human readable types)
%In this paper, our goal is to generate argument types for binary relations, generating all possible
%relation schemas, which are human readable.
% 2. build rvsp, a xxxx based on xxxx.

本章的贡献可以总结为以下三个部分:
\begin{enumerate}
\item{我们具体定义了基于开放式信息抽取的二元关系模式推理问题;}
\item{我们设计了基于Freebase和实体链接任务的方法,
对一类关系的主宾语所具有的类型分布进行联合建模;}
%\cite{lin2012entity,ratinov2011local,hoffart2011robust,rao2013entity,cai2013wikification},
\item{我们在ReVerb数据集上进行实验,根据人工标注的类型搭配结果,
对不同二元关系生成的最佳模式进行测评。与传统选择偏好方法比较,
我们的模型在MRR指标上得到了10\%的相对提升。}
\end{enumerate}

%This paper makes the following contributions: i) we defined the schema
%inference problem for binary relations from Open IE;
%ii) we developed a prototype system based on Freebase and
%entity linking~\cite{lin2012entity,ratinov2011local,hoffart2011robust,rao2013entity,CaiZZW13}, which simultaneously models the type distributions
%of two arguments for each binary relation;
%iii) our experiment on ReVerb triples showed that the top inferred schemas
%receive decent mean reciprocal rank (MRR) of 0.337,
%with respect to the human labeled ground truth.
