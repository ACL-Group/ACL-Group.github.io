%%%%%%%%%%%%%%%%%%%%%%%%%%%%%%%%%%%%%%%%%
% Plain Cover Letter
% LaTeX Template
% Version 1.0 (28/5/13)
%
% This template has been downloaded from:
% http://www.LaTeXTemplates.com
%
% Original author:
% Rensselaer Polytechnic Institute 
% http://www.rpi.edu/dept/arc/training/latex/resumes/
%
% License:
% CC BY-NC-SA 3.0 (http://creativecommons.org/licenses/by-nc-sa/3.0/)
%
%%%%%%%%%%%%%%%%%%%%%%%%%%%%%%%%%%%%%%%%%

%----------------------------------------------------------------------------------------
%	PACKAGES AND OTHER DOCUMENT CONFIGURATIONS
%----------------------------------------------------------------------------------------

\documentclass[11pt]{letter} % Default font size of the document, change to 10pt to fit more text

\usepackage{newcent} % Default font is the New Century Schoolbook PostScript font 
%\usepackage{helvet} % Uncomment this (while commenting the above line) to use the Helvetica font

% Margins
\topmargin=-1in % Moves the top of the document 1 inch above the default
\textheight=8.5in % Total height of the text on the page before text goes on to the next page, this can be increased in a longer letter
\oddsidemargin=-10pt % Position of the left margin, can be negative or positive if you want more or less room
\textwidth=6.5in % Total width of the text, increase this if the left margin was decreased and vice-versa

\let\raggedleft\raggedright % Pushes the date (at the top) to the left, comment this line to have the date on the right

\usepackage{color}

\begin{document}

%----------------------------------------------------------------------------------------
%	ADDRESSEE SECTION
%----------------------------------------------------------------------------------------

\begin{letter}{Dr. Ruslan Mitkov \\
Editor-in-Chief  \\
Natural Language Engineering} 

%----------------------------------------------------------------------------------------
%	YOUR NAME & ADDRESS SECTION
%----------------------------------------------------------------------------------------

\begin{center}
\large\bf Yizhu Liu, Xinyue Chen, Xusheng Luo and Kenny Q. Zhu \\ % Your name
%\vspace{20pt} \hrule height 1pt % If you would like a horizontal line separating the name from the address, uncomment the line to the left of this text
Department of Computer Science and Engineering \\ Shanghai Jiao Tong University \\ 800 Dongchuan Road, Shanghai, China 200240 \\
liuyizhu@sjtu.edu.cn
 % Your address and phone number
\end{center} 
\vfill

\signature{Yizhu Liu} % Your name for the signature at the bottom

%----------------------------------------------------------------------------------------
%	LETTER CONTENT SECTION
%----------------------------------------------------------------------------------------

\opening{Dear Dr. Ruslan Mitkov,} 

I am pleased to resubmit an original research article entitled 
"Reducing Repetition in Convolutional Abstractive Summarization" by Yizhu Liu, Xinyue Chen, Xusheng Luo and Kenny Q. Zhu 
for consideration for publication in \textit{Natural Language Engineering}. 
Abstractive Summarization is one of the most challenging and interesting problems 
in the field of Natural Language Processing (NLP). 
It is a process of generating a concise and meaningful summary of text from multiple text resources 
such as news articals.
The neural-based abstractive summarization always produces repetitive sentences.
This manuscript finds two possible reasons behind the repetition problem in neural-based abstractive summarization: 
\begin{itemize}
\item The models always attend to the same location in source.
\item The models may attend to similar but different sentences in source. 
\end{itemize}
In response, we propose a section-aware attention mechanism (ATTF) and a sentence-level backtracking decoder (SBD).

In this manuscript, our proposed model improves the quality of generated abstractive summaries.
ATTF and SBD can generate more readable summaries, which are free of grammatical and factual errors
as well as logically consistent with source document.
Extensive experiments show that our proposed model outperforms multiple baselines.
 
We believe that this manuscript is appropriate for publication by
\textit{Natural Language Engineering}
because it belongs to the topic of text information extraction and text generation. 
In addition, our manuscript finds the reason for repetition in neural-based abstractive summarization.
This may help to extract important information of texts and generate new texts.
We also present two new evaluation metrics for future work in this research area.

What's more, we adopted all the suggestions raised by the Editorial Office. Details are as follows:
\begin{enumerate}
   \item LANGUAGE AND GRAMMAR: render your text free of grammatical errors and stylistic inconsistencies.
   \begin{itemize}
				\item[] \textbf{AUTHORS' RESPONSE}: 
				We are very sorry for our incorrect writing.
				We have proofreading the paper thoroughly and fixed those errors with an English native speaker.

                \hspace*{0.6cm}
				For example,
				the statements of 
				``Repeatedness Correlation measures test how well ...'' 
				are changed as
				``Repeatedness Correlation measures how well ...''
				(Section 3.2, page 10)
   \end{itemize}
   \item LENGTH:expand on what you have already written under Related Work and Conclusion.
   \begin{itemize}
				\item[] \textbf{AUTHORS' RESPONSE}: 
				According to your suggestions, we have enriched the content of Related Work and Conclusion. We also revised some parts of our manuscript. The length of our paper increased from 18 pp to 20 pp (excluding references).

                \hspace*{0.6cm}
				For example, we discuss some typical methods used in neural-based abstractive summarization and explain why we choose convolutional neural network as our basic summarization model. 
				(Section 4, page 17-19)
   \end{itemize}
\end{enumerate}

This manuscript has not been published and is not under
consideration for publication elsewhere. 
If you feel that the manuscript is appropriate for your journal, 
we suggest the following reviewers: 
\begin{enumerate}
	\item[-] Seung-won Hwang (seungwonh@yonsei.ac.kr)
	\item[-] Haixun Wang (haixun@gmail.com)
	\item[-] Wentao Wu (Wentao.Wu@microsoft.com)
	\item[-] Eric Lo (ericlo@cse.cuhk.edu.hk)
\end{enumerate}

Thank you for your consideration!

\closing{Sincerely yours,}


%\encl{Curriculum vitae, employment form} % List your enclosed documents here, comment this out to get rid of the "encl:"

%----------------------------------------------------------------------------------------

\end{letter}

\end{document}
