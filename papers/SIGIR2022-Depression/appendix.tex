\section{Depression Templates}

\begin{table}[htbp]
  \small
  \centering
  \begin{tabular}{l|l}
  \hline
  Dimension & Template \\
  \hline
  Feeling Depressed  &  I feel depressed. \\
  Diagnosis &  I am diagnosed with depression. \\
  Treatment &  I am treating my depression. \\
  \hline
  Sadness & I feel sad.  \\
  Pessimism & I am discouraged about my future.  \\
  Past Failure & I always fail. \\
  Loss of Pleasure & I don't get pleasure from things. \\
  Guilty Feelings & I feel quite guilty. \\
  Punishment Feelings & I expected to be punished. \\
  Self-Dislike & I am disappointed in myself. \\
  Self-Criticalness & I always criticize myself for my faults. \\
  Suicidal Thoughts or Wishes & I have thoughts of killing myself. \\
  Crying & I always cry. \\
  Agitation & I am hard to stay still. \\
  Loss of Interest & It's hard to get interested in things. \\
  Indecisiveness & I have trouble making decisions. \\
  Worthlessness & I feel worthless. \\
  Loss of Energy & I don't have energy to do things. \\
  Changes in Sleeping Pattern & I have changes in my sleeping pattern. \\
  Irritability & I am always irritable. \\
  Changes in Appetite & I have changes in my appetite. \\
  Concentration Difficulty & I feel hard to concentrate on things. \\
  Tiredness  & I am too tired to do things. \\
  Loss of Interest in Sex & I have lost my interest in sex. \\
  \hline
  \end{tabular}
  \caption{The main templates and their corresponding dimensions we used in our experiments, including 3 direct depression descriptions and 21 indirect symptoms derived from BDI-II \citep{beck1996beck}. }
  \label{table:bdi}
\end{table}


We provide the complete templates in Table \ref{table:bdi}. We mainly use a combination of 3 direct depression descriptions and the 21 indirect symptoms derived from BDI-II. We also experimented other well-known depression scales like HDRS \citep{hamilton1986hamilton}, CES-D \citep{Lenore1977CES-D} and PHQ-9 \citep{kroenke2001phq} (We will release our revised templates for theses scales along with our code). The original scales usually contain different descriptions under the same dimension to distinguish different level of intensity or frequency. However, we find that current sentence representations have difficulty in capturing such nuanced differences. We thus condense the descriptions of each dimension into one general template (A few may have more, if there are significant intra-dimension difference).
