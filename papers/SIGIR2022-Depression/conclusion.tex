\section{Conclusions}

In this work, we tackle the problem of online depression detection with a novel, psychiatry-guided method of risky post screening and hierarchical attentional network. The accurate selection of risky posts out of the long user history constitutes a solid foundation for prediction as well as enables the usage of large pretrained language model, resulting in SOTA on 3 datasets and good generalizability. Furthermore, our framework can work on online early detection scenarios with high efficiency, supported by the proposed evolving queue algorithm, which can greatly reduce the required number of model inferences. The use of attention mechanism and depression scales provides our method with strong interpretability in the form of attention weights and diagnostic basis, which we hope can facilitate its further application in clinical practice as a reliable assistant. Future works may include exploring better ways to combine direct depression expressions and indirect symptoms in a unified model.
