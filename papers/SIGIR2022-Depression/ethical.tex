\section{Ethical and Broader Impact Statement}

This work aims to help people suffering from depression, but have not yet been diagnosed due to the difficulty in receiving clinical help or the stigmatization of the disease. It can be a sensitive topic so it is important to discuss the potential risks and limitations of our work. The proposed method can conduct early depression detection on social media. However, the performance is far from prefect, so the models' early alerts still require careful examinations from professional practitioners. The proposed method can provide diagnostic bases as explanations to the patients or clinical practitioners, but the diagnostic basis may not precisely matched the actual symptom implied in the post. Therefore, the diagnostic basis should be checked before adoption. Moreover, the datasets are annotated with proxy signals of depression, which may not be representative of the true population of depression patients. In practice, the model should be trained on a more carefully-curated dataset for reliable predictions.

The datasets used in this work are either publicly available or used under their corresponding data usage agreement. All posts in examples were de-identified and paraphrased for anonymity.