\section{Related Work}
Classifying relations between entities in a certain sentence plays a key role in NLP applications and thus has been a hot research topic recently.
Feature-based methods~\cite{sem} and neural network techniques are most common. 
\citeauthor{socher2011semi}(2011) design a RNN method using the constituency parse tree. \citeauthor{ebrahimi2015chain} (2015) narrows down the attention to shortest dependency path(SDP) of given sentences. 
\citeauthor{xu2015classifying}(2015) introduce multi-channel  SDP-based LSTM model to classify relations incooperating several different kinds of information of a sentence improved by \citeauthor{xu2016improved}(2016), which performs the best on the SemEval-2010 Task 8 and is one of our baseline methods.
%\footnote{\url{}}

The most closely related work to ours is the extraction of
visual commonsense knowledge by \citeauthor{yatskar2016stating}. 
This work learns the textual representation of seven types of fine-grained 
visual relations such as ``touches'', ``above'' and ``disconnected from'' 
by jointly modeling the relative position of the 80 kinds of objects in 300,000 images
and the textual caption for the image in MS-COCO dataset\cite{lin2014microsoft}. 
The authors also generalized their extracted knowledge using WordNet. 
Since the commonsense knowledge in image captions is limited by its selected objects and not scalable for its expensive human labor, we propose a framework to use large text which is scalable and involves more real world description. 
Another important related work is from \citeauthor{li2016commonsense}, which enriches
several popular relation types in ConceptNet by deep neural networks.
\lnear~relation was not studied in this work. Besides, this work studies
a different problem known as knowledge base completion which seeks to
add more edges into a knowledge graph, with little information from
text corpora. 
