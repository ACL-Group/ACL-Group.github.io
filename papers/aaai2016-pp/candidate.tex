\section{Candidate Schema Generation}
In this section, we propose a searching algorithm to generation
a bunch of candidate schemas (CS) used in the paraphrasing task.
Besides, we also explore the size of hit pairs and all hit pairs in $EP$.
%which satisfies what?

%basic framework 5 sents
%1 Intro
Basically, the searching process is made up of three parts.
%2 BFS
In the first part, we generate a list of most simple schema candidates,
which are descriptive over a part of $EP$.
%3 DFS
With simple schema candidates as the starting point, in the second part, we expand schemas by 
recursively adding new schema predicates to previous candidates, making schemas more and more 
specific.
%4 HP maintain
During the searching process, we maintain ground graphs of each candidate, and
calculate number of hit pairs that a candidate schema has.
%5 Selection
Finally, we select a finite number of well descriptive schemas from all candidates,
measured by a cost function which is similar with \eqnref{eqn:f}

\subsection{Simple Schema Retrieval}
%BFS part (with hit pair gen.)
%1. what is simple path
At first, we state the ``most simple'' schema as a \textbf{solid} path, that is,
all vertices in the schema are variables, connecting $x_{subj}$ and $x_{obj}$.
%2. intuition
Intuitively, this path represents a join operation over necessary predicates in $KB$, 
which is the skeleton of more complex schemas.

%3-6. where the path comes

In order to find this kind of 

In order to search descriptive paths for $EP$, we use breadth-first search algorithm 
to find paths for each entity pair $\langle e_1, e_2 \rangle$.
The basic idea is to use a breadth-first search algorithm to 
Due to the existence of various predicates and popular entity in $KB$, a exhausted search
could result in many long paths, but meaningless in natural language.
Therefore, we use a parameter $k$ to limit the maximum length of a path, which
controls the searching scope.
In practice, we use a bidirectional strategy to save time:
We start from both $e_1$ and $e_2$, then search $k/2$ layers at each side simultaneously,
recording different intermediate entity $e'$ with all paths between $\langle e_1, e' \rangle$
and $\langle e', e_2 \rangle$.
By merging paths on $e'$, we retrieve all paths from $e_1$ to $e_2$.
\KQ{Add figures showing example paths in the running ex.}

%7-9. how to bring subgraphs. (including sampling)
In the process of breadth-first search, we actually recorded all entities in each individual schema $S$ (a path),
so all ground graphs within $EP$ are stored. \KQ{Add a figure to show how an example
ground graph is generated in BFS.}
Therefore, we keep these ground graphs and count the number of hit pairs of $S$ in $EP$.
Besides, we aim to calculate the size of $|HP(S)|$ within the scope of whole $KB$, 
which is processed by querying relations in $KB$.

\KQ{Suppose we have an example schema $x_{subj}--rel_1--x_1--rel_2--x_2--rel_3--x_{obj}$, adding into a fig.}
By querying $rel_1$ in $KB$, we retrieve all possible $\langle e_{subj}, e_1 \rangle$ pairs.
Then we take all possible $e_1$ and further query $rel_2$, resulting in possible $\langle e_{subj}, e_1, e_2 \rangle$ triples.
Finally, by querying $rel_3$, all $\langle e_{subj}, e_1, e_2, e_{obj} \rangle$ tuples could be retrieved,
representing different ground graphs of the schema.
\KQ{Replace all $rel_x$ with specific predicates.}
However, due to the difference of intermediate entities, the size of ground graphs could be too large to store and process.
Thus, we sample subgraphs in each query step. 
For example, suppose we get too many $\langle e_{subj}, e_1, e_2 \rangle$ triples,
we first group triple by $\langle e_{subj}, e_2 \rangle$, and then equally select some group of triple,
which reduces the size of ground graphs to a satisfiable limit.

\subsection{Schema Expansion}
%DFS part
%6 sents
%1. input?
In the second part of searching process, we take the simple schemas (paths) and 
corresponding ground graphs as input, and explore more specific schemas.
%2. how to expand
Based on the definition of schema graph, there are two kinds of adding strategies listed below:
\begin{itemize}
    \item[-] Add a new variable, and use a \textit{solid} relation to connect the variable with an existing variable, 
    but \textbf{except} $x_{subj}$ and $x_{obj}$;
    \item[-] Add a specific entity (or type), and use a \textit{dashed} (or \textit{isa}) relation to
    connect the entity (or type) with an existing variable.
\end{itemize}
%3. dfs
While there are many specific possible adding edges for one schema, we adopt
a depth-first search to recursively add edges one by one.
%4. formula
Therefore, we need to determine which edge is to added for the current state of schema in the DFS process.
Among several potential schemas (after adding an edge), we use a cost function to measure the quality
of each schema:
\begin{equation}
\label{eqn:dfs}
\begin{aligned}
g(EP,\, S) = & Cost(S) + Cost(EP(S)\, |\, S) \\
             & + 2 \log |E| \cdot (|EP| - |EP(S)|).
\end{aligned}
\end{equation}

\noindent
Note that the difference of $g$ and $f$ (\eqnref{eqn:f}) is to increase
$2 \log |E|$ bits for each entity pair not in $EP(S)$.
Since a schema cannot cover all pair in $EP$, the intuition of $g$ is the cost of 
transmitting $EP$ by using $S$ only: if $S$ cannot describe some entity pair in $EP$,
we have to directly transmit these two entities.
Therefore, we can compare the quality between general and specific schemas.

%5. limit
Also, we need to limit the scope schemas in DFS process. We follow the idea of $k$
in the step of simple schema retrieval, and limits the variables in the expanded
schema.

