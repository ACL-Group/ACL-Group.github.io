\section{Introduction}
\label{sec:intro}

%\begin{itemize}
%\item Binary predicates in standard knowledge base are identified by certain 
%canonical forms (e.g., parent\_of), sometimes even crytic (e.g., ...)
%To support natural language queries (in QA e.g.), we need to map natural
%language predicates to predicates (or a set of connected predicates, which
%we call schema) in the knowledge base. 
%(e.g., the mother of $\rightarrow$ parent\_of and 
%gender = female. Maybe a figure here). We need to argue why this problem
%is critical and a must in QA (and other important applications).
%\item State of the arts in solving this problem (give some cites and brief
%descriptions) and their limitations. Say informally what are the challenges 
%in this problem.
%\item Our approach is to define graph schema inferred
%from the knowledge graph. We formalize the problem as given a set of
%entity pairs extracted by a natural language predicate, return a set of graph
%schemas that cover all the entities pairs and simultaneously optimizes 
%a the cost of transmitting the pairs and the schemas, according to the
%minimal description length (MDL) principle. The optimization problem is shown
%to be NP-hard and we thus propose an local search based 
%approximation algorithm to solve it.
%\item Our main contributions are:
%end{itemize}

% what to do and why to do
%1. what's binary relation
Open Information Extraction (Open IE) is a recent popular techhnique
to mine large amount of relations  as well
as their instances from open-domain natural language
data sources such as the world wide web pages. 
State-of-the-art Open IE systems, such as ReVerb \cite{fader2011identifying},
NELL \cite{} and PATTY \cite{nakashole2012patty} extract binary 
relation (e.g., grand-father-of, was-born-in) between named entity pairs, 
which can then be transformed into or used to enrich existing structured
knowledge bases such as DBpedia~\cite{}, Freebase~\cite{} and YAGO~\cite{}.
These knowledge bases are typically represented in the form of a graph 
connecting named entities, concepts and their types using standard, predefined 
predicates as edges. \figref{fig:freebase} shows a fragment of
Freebase. \KZ{Show a ground graph of grand-father-of using real entities.} 
As a result, there is a gap between the relations
extracted using natural language patterns and the predicates available in
the knowledge base. For example, Freebase doesn't have the grand-father-of
predicate, or even father-of predicate, but instead has the parent-of 
and gender-of predicates. To enable open IE data integration and to
support advanced applications such as question answering, it is necessary
to translate a natural language relation into one or more standard, 
machine-readable predicates in the knowledge base, joined together into a
representation known as the {\em relation schema}. For example,
the grand-father-of relation can be represented by the Freebase schema shown 
in \figref{fig:grandpa}.
 
\begin{figure}[th]
\caption{A snippet of Freebase.}
\label{fig:freebase}
\end{figure}


\begin{figure}[th]
\caption{A relation schema for grand-father-of in Freebase.}
\label{fig:grandpa}
\end{figure}

%2. what's knowledge base try to do
%Structured knowledge base (KB) is a graph based taxonomy containing real world 
%entities,  types,  binary predicates between entities and ``IsA'' relations 
%between entities and types.
%(Machine readable, containing millions / billions of facts)

%Structured KBs such as WordNet \cite{miller1995wordnet},
%Yago \cite{suchanek2007WWW} and Freebase \cite{bollacker2008freebase} are widely used in information extraction
%and semantic learning tasks. In order to make relation schemas understood by human, we leverage types
%in the KB as the output of relation schemas.

%3. what to do is to extract schema
%The paraphrasing task is to map a natural language relation into structured
%canonical forms in KB, which is understood by both machine and human.
%
%We call the representation as \textit{relation schema} throughout this paper.
%
%\KQ{how to give a clear impression on schema}
%%\cite{bollacker2008freebase}
%%%Figure show schema on FB as the first impression (mention FB's size here)
%informal
One can see that a relational schema is a template of many subgraphs 
with the concrete structure in the knowledge base.
The goal of this paper is to enable effective and efficient translation 
process, which we call ``paraphrasing.'' 


%add. why we need to schema
%Paraphrasing is a open task, since the structured schema is an important knowledge
%used in many down-stream applications, such as question answering, text entailment
%and short text similarity querying.
%(Arguing)

%4. claim the gap between kb and nl on description
Besides, the task is not trivial, due to the semantic gaps between natural language
and knowledge base. Yet some knowledge base is lack of predicates, however, the gap
couldn't be removed, even for Freebase containing thousands of binary predicates.

%5. simple exmple & chain example (mediator)
%6. branching example
%a) place_of_birth v.s. <people, was born in, place>
%b) mediator: film.actor.film --> film.performance.film   v.s.   "starring in"
%c) branching: female spouse v.s. "wife of"
Recap the example schemas shown before. 
\figref{fig:fb-schema} (a) shows the difference between relation words ``was born in''
and predicate names ``place of birth''; \figref{fig:fb-schema} brings the gap to 
structure level, where the schema includes a intermediate node (used to maintain the
ternary relation ``actor plays a character in a film''), making it more complex;
Meanwhile, the gap in \figref{fig:fb-schema} is larger even the relation ``wife of''
is so common in the real world. \KQ{mention predicate path here?}

%traditional method & limits
%0. informally, composite relation
Informally speaking, the input of paraphrasing is a natural language relation $R$ with a
list of entity pairs $R(e_1, e_2)$, and the output is a list of schema graphs, where each
schema graphs is a composite of several predicates connected by joins and constraints.
We define the notion \textbf{simple schema}, if the schema graph is a path of predicates
in KB connecting entities from $e_1$ to $e_2$. 
To this end, state-of-the-art systems have been proposed for solving this task.

%(Lei Zou) (EMNLP 2011) (AAAI 2012) (Tran 2009)
For the part of supervised systems, the first branch is graph-walk based 
\cite{lao2010relational,lao2011random}. A candidate schema is drawn from the predicate path 
between some entity pairs, and the probabilistic distribution of random walking from $e_1$ 
to $e_2$ in KB on the schema is used as a feature to train the importance of each path.
The second branch is logic based \cite{zhang2012ontological}, where the system uses hand 
crafted soft rules to mine various features that leads to good relation schemas. 
Since soft rules are independent of different relations, it brings the limitation that
weights won't be changed across different inputs.

Besides, unsupervised models are also used. For example, Zou et al. \shortcite{zou2014natural}
followed the idea of TF-IDF score \cite{} to calculate the best schema with respect to a
speicifc relation. While different input relations could have overlap meaning, this situation 
causes a lower score of schemas representing the overlapping part.

In addition, among all different paraphrasing solvers, the process of candidate schema 
searching could always be a big challenge. All systems discussed above only generate
simple schemas (predicate paths), which is also a limitation for searching more
specific and meaningful schemas.

% our approach
%1. IMPORTANT data-driven
In this paper, we present an unsupervised data-driven approach to solve paraphrasing problem.
%2. MDL principle, give a formal def. on the ability to desribe data
We formalize the problem as given a list of entity pairs extracted from a natural language relation,
return a ranked list of schemas that cover all the entity pairs and simultaneously optimizes
the cost of transmitting both schemas and entity pairs, according to the \textit{Minimum
Description Length} (MDL) principle. 
%3. optimization problem, and be able to find diff. schemas for an ambiguous relation.
The optimization problem is shown to be NP-hard and we thus propose an local search
based approximate algorithm to solve it.
Furthermore, we \textbf{do not limit} schemas to be simple schemas only, which potentially help
us find better schemas. To the best of our knowledge, our work is the first trial of
mapping natural language predicates to a broader scope of structured representations.


%Contributions
This paper makes the following contributions:
\begin{itemize}
	\item we propose a brand new framework for the paraphrasing task on the basis of MDL principle;
	\item we search schemas beyond predicate paths, increasing the chance of finding better schemas;
	\item our experiments on NELL and Freebase show that our system improves the paraphrasing
		  result by about 10\%, compared with other state-of-the-art systems.
\end{itemize}



