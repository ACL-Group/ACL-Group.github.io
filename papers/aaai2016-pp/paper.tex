\def\year{2015}
%File: formatting-instruction.tex
\documentclass[letterpaper]{article}

\usepackage{aaai}
\usepackage{times}
\usepackage{helvet}
\usepackage{courier}
\usepackage{amsmath,amsfonts,amssymb,amsthm,amsopn}
\usepackage{times}
\usepackage{url}
\usepackage{color}
\usepackage{latexsym}
\usepackage{epsfig}
\usepackage{graphicx}
\usepackage{booktabs}
\usepackage{diagbox}
\usepackage{array}
\usepackage{multicol}
\usepackage{threeparttable}
\usepackage{epstopdf}

\newcommand{\figref}[1]{Figure \ref{#1}}
\newcommand{\tabref}[1]{Table \ref{#1}}
\newcommand{\eqnref}[1]{Eq. (\ref{#1})}
\newcommand{\defref}[1]{Def. \ref{#1}}

\newcolumntype{I}{!{\vrule width 1pt}}
\newlength\savedwidth
\newcommand\whline{\noalign{\global\savedwidth\arrayrulewidth
                            \global\arrayrulewidth 1pt}%
                   \hline
                   \noalign{\global\arrayrulewidth\savedwidth}}
\newlength{\Oldarrayrulewidth}
\newcommand{\Cline}[2]{%
  \noalign{\global\setlength{\Oldarrayrulewidth}{\arrayrulewidth}}%
  \noalign{\global\setlength{\arrayrulewidth}{#1}}\cline{#2}%
  \noalign{\global\setlength{\arrayrulewidth}{\Oldarrayrulewidth}}}


\theoremstyle{plain}
\newtheorem{thm}{Theorem}
\newtheorem{lem}[thm]{Lemma}
\newtheorem{prop}[thm]{Proposition}
\newtheorem*{cor}{Corollary}

\theoremstyle{definition}
\newtheorem{defn}{Definition}
%\newtheorem{defn}{Definition}[section]
\newtheorem{conj}{Conjecture}[section]
\newtheorem{exmp}{Example}[section]

\theoremstyle{remark}
\newtheorem*{rem}{Remark}
\newtheorem*{note}{Note}

\newcommand{\KZ}[1]{\textcolor{blue}{Kenny: #1}}
\newcommand{\KQ}[1]{\textcolor{red}{Kangqi: #1}}
\newcommand{\XS}[1]{\textcolor{blue}{Xusheng: #1}}


\frenchspacing
\setlength{\pdfpagewidth}{8.5in}
\setlength{\pdfpageheight}{11in}
\pdfinfo{
/Title (Insert Your Title Here)
/Author (Put All Your Authors Here, Separated by Commas)}
\setcounter{secnumdepth}{0}


\begin{document}
\title{Paraphrasing Binary Relations from Natural Language to Knowledge Graphs}
%\author{Kangqi Luo, Xusheng Luo \and Kenny Q. Zhu\\
%Shanghai Jiao Tong University\\
%}
\maketitle


\begin{abstract}
% 5 sentences
%1. gap between NL and KB
The task of paraphrasing to translate a natural language relation into structured knowledge.
Due to the incompleteness of a knowledge base (KB), paraphrasing is important for
bridging semantic gaps between NL representations and canonical forms in KB.
%2. presents
We presents a framework to model the semantic representation of a binary relations based on
a small set of relation instances, and to infer a set of relation schemas.
%3. what's the schema look like
Each relation schema is a graph which combines different predicates in KB.
%4. MDL principle
Our framework follows the \textit{minimum description length} principle, which optimizes the
cost of transmitting relation instances and the schemas.
%5. apply to  &  competitive
Comparing with state-of-the-art paraphrasing systems, the experiments show that our system can improve
both schema quality and the F1 score of binary relation instances extracted by schemas over different knowledge bases.

\end{abstract}
\section{Introduction}

Protein$-$protein interactions (PPIs) are of central importance for the majority of biological functions, such as signal transduction, metabolic pathways, molecular dynamics, and protein networks\cite{Hoffmann.Krallinger.ea:2005}, for they serve as the most fundamental building blocks of the entire interacademic systems of any organisms. Collecting data on pairwise interaction relationships is essential for multiple purpose, including identification of modules with certain functionality\cite{Spirin.Mirny.03}, mapping diseases to dominated genes\cite{Ideker.Sharan.08}, and after all, understanding wholistic metabolic/genetic networks from a system biology perspective.

A lot of databases have been built to store protein and genetic interactions from major model organism species and are available in various standardized formats, such as MINT\cite{Zanzoni.Montecchi-Palazzi.ea:2002}, BIND\cite{Bader.ea:2003}, BIOGRID\cite{DBLP:journals/nar/StarkBRBBT06}, etc. Among those mainstream databases, the data largely rely on voluntary reports by scientists or researchers, besides, comprehensive curation efforts become indispensable for the sake of accuracy. However, the amount of biology-related literatures with respect to protein interactions grows explosively and thus make it either impossible or impractical to manually detect PPI information anymore.

Considering huge amount of PPI information with great wealth hidden in published papers, in recent years, numerous mining techniques have been proposed that aim to extract PPI information automatically from free text, especially machine learning, information retrieval, and natural language processing\cite{DBLP:journals/bib/WinnenburgWPDS08}.These approaches can be roughly categorized into three classes: co$-$occurrence, rule$-$based, and machine learning. 

Co$-$occurrence is the approach with most simplicity and naivete. Just as its name implies, this method intends to find out pairs of proteins that co-occur in the same context. The scope of "same context" ranges from phrase, sentence, paragraph to whole abstract, even document. The underlying assumption is that whenever two proteins are mentioned together by authors, chances are high that there is some kind of relationship between them. However, however, in-context closeness even semantic relation does not necessarily represent actual biological interaction. As a consequence, a large fraction of candidate pairs are mismatched inevitably, causing a high recall but low precision.

The second approach is rule-based extraction, in other words, pattern matching. There are many types of rules, most of them concern natural language processing (NLP). One way is to specify hand-crafted regular expressions before hand, which mostly lean on language usage preference. Besides, by using full or partial (shallow) parsing strategies, more information would be acquired, such as part-of-speech taggers, local dependencies between syntactic components, context-free grammar\cite{DBLP:journals/bioinformatics/TemkinG03}, and full sentence structure. Compared to co$-$occurrence, rule-based approach enjoy better precision but much lower recall. In addition, since the rules are usually derived from training data, that is to say, the improper choice of training data would be significantly lethal, therefore quality of extraction is invariably instable and may not applicable to other data.

The third and most commonly used approach use machine learning techniques, in this case, the task to extract protein$-$protein interactions turns out to be a binary classification problem. Each protein pairs are represented along with a set of features, which is associated with their context, then a well$-$defined classifier gives the answer whether the candidate protein pairs is classified to be qualified PPI. (TO BE FURTHER FILLED!!!)

In this paper, we introduce a general bootstrapping framework for Protein$-$protein interaction extraction from natural text.Our method differs from most of the previous works in three aspects:

(1)The extraction process is driven by only tiny fraction of training data, which are regarded as seed data. In each round, it would derive reliable patterns automatically from seed data, then extract more positive PPI pairs consequently, what's more, the seed data would be augmented by the newly extracted results with high confidence.

(2)multiple graph kernel. 

(3)various evaluation.





\section{Problem Definition}
\label{sec:problem}

In this section we formally define the problem of short title extraction.
A char is a single Chinese or English character.
A segmented word (or term) $x$ is a sequence of several chars such as 
``Nike'' or ``牛仔裤''(jean).
A product title, denoted as $X$, is a sequence of words $\{x_1, x_2, ..., x_n\}$.
Let $Y$ be a sequence of labels $\{y_1, y_2, ..., y_n\}$ over $X$, where $y_i \in \{0, 1\}$.
The corresponding short title is a subsequence of $X$, denoted as $S = \{x_i\}$, 
where $y_i = 1$ and $|S| \le n$.

%we are interesting in obtaining a short title which can represent the most important information about the product.

We regard short title extraction task as a sequence classification problem.
Each word is sequentially visited in the original product title order
and a binary decision is made.
We do this by scoring each word $x_i$ within $X$ and predicting a label $y_i \in \{0, 1\}$, 
indicating whether the word should or should not be included in the short title $S$.
As we apply supervised training, the objective is to maximize the likelihood of all word labels
$Y=\{y_1,y_2,...,y_n\}$, given the input product title $X$ and model parameters $\theta$:
\begin{equation}
\label{eqn:problem}
\log{p(Y|X,\theta)}=\sum_{i=1}^{n}{\log{p(y_i|X,\theta)}}.
\end{equation}

%Our problem is different from Sequece Labelling problem, as ...

%In a more restrictive scenario, the number of words $m$ in the short title is strictly limited, where $m$ is some fixed number and $m \le \sum_{i=1}^{n} len(x_i)$. $len(x_i)$ is the number of words (chars) in term $x_i$.



\section{Information-Theoretic Scoring}
\label{sec:scoring}

%>>>>0. we follow information theory to define the cost function
In this section, we follow information theory to define the function
$f(EP, S)$, which is used to measure the cost of a schema describing
a set of entity pairs.

%>>>>1. what's MDL? bytes transmitting data & desc
    % with the form Cost(EP, OS) = Cost(OS) + Cost(EP | OS), which is most suitable in our work
% some similar words in popl2008 can be put here, for example, what is compact and precise
The \textit{Minimum Description Length} (MDL) is a common principle from information theory,
which states that a good description is one that minimizes the cost (in bits) of transmitting
the data.
%cite Grunwald 2007 here
Formally, the transmitting cost consists of two parts in our task:
\begin{equation}
    \label{eqn:mdl}
    Cost(EP,\, OS) = Cost(OS) + Cost(EP\, |\, OS),
\end{equation}

\noindent
where $Cost(OS)$ is the number of bits to transmit the description (output schemas) itself, and
$Cost(EP\, |\, OS)$ is the number of bits to transmit the data (entity pairs) given schema graphs.

%>>>>2. Cost(S)
%Obviously, a good  is a connected graph. If $S$ contains some vertices which are not connected
%to $x_{subj}$ and $x_{obj}$ through predicates, those vertices are nothing with the relation that
%$S$ wants to describe. Removing those vertices can decrease the cost of transmitting $S$ itself without
%harming the effectiveness for transmitting entity pairs.
%Due to multiple schemas available, the cost of transmitting $\vec{S}$ equals to the sum of transmitting
%each schema.

Intuitively, the cost in bits of transmitting output schemas is the summation cost
of transmitting each individual schema: $Cost(OS) = \sum\nolimits_{S \in OS} Cost(S)$.
Considering a single schema $S$, each predicate in the schema is transmitted, including
two nodes (entity, type or variable), a predicate and its category.

We give the formal definition of $Cost(S)$ as follows:
\begin{equation}
\label{eqn:costs}
\begin{aligned}
    & Cost(S)      = \sum\nolimits_{p_s \in P_S} Cost(p_s), \\
    & Cost(p_s)    = Cost(v_1) + Cost(pred) + Cost(v_2) + \log{|C|}, \\
    & Cost(v)      = \left\{
        \begin{aligned}
        \log{|X|} & ~     & v \in X  \\
        \log{|E|} & ~     & v \in E' \\
        \log{|T|} & ~     & v \in T'
        \end{aligned}
    \right., \\
    & Cost(pred)   = \left\{
        \begin{aligned}
        & 0         & ~     & pred = \text{isa}  \\
        & \log{|L|} & ~     & pred \in L \\
        \end{aligned}
    \right.,
\end{aligned}
\end{equation}

\noindent
note that the predicate name can only be ``isa'' if the category is $isa$,
therefore we don't need to transmit this duplicated information.

%>>>>3. Cost'(EP | S) (EP \in Hit(S))
Next, we focus on the cost of transmitting entity pairs given output schemas.
due to the fact that a single schema is not able to cover all pairs in $EP$,
each schema is used to transmit those entity pairs which belong to its hit pairs.

We define $EP(S)$ as the entity pairs transmitted by $S$, that is, $EP(S) = EP \cap HP(S)$.
Then the cost of transmitting all pairs equals to the summation cost of each schema
transmitting its own pairs:
\begin{equation}
    Cost(EP\, |\, OS) = \sum\nolimits_{S \in OS} Cost(EP(S)\, |\, S)
\end{equation}

Note that $HP(S)$ can be easily retrieved by querying the schema in the $KB$.
The retrieval step resembles querying a view in relational database,
where the resulting $HP(S)$ is a table with 2 columns, representing $e_{subj}$ and $e_{obj}$.
Note that $S$ can only hit at most $|E|^2$ different entity pairs in $KB$, the more specific
a schema is, the fewer pairs it hits.

Now recap $Cost(EP(S)\, |\, S)$.
Since $EP(S) \subseteq HP(S)$, when transmitting one entity pair,
we can either send two entities directly, or send the row number of the pair in $HP(S)$.
The former one costs $2\log|E|$ bits, while the latter costs $\log{|HP(S)|}$ bits,
which is always no larger than $2\log|E|$.
We call this method as \textbf{inclusive} strategy.

There is an alternative \textbf{exclusive} strategy: suppose $EP(S)$ covers more than
half of pairs in $HP(S)$, instead of transmitting all pairs belonging to $EP(S)$,
we can only transmit pairs excluded from the set. Under this circumstance, the exclusive
strategy goes more efficient.

Combining both inclusive and exclusive strategies, the cost of transmitting entity pairs
given a schema is defined as \eqnref{eqn:costd}:
\begin{equation}
\label{eqn:costd}
\begin{aligned}
Cost&(EP(S)\, |\, S) = 1 + \log|HP(S)| \cdot   \\
    &\min\{|EP(S)|,\, |HP(S)| - |EP(S)|\},
\end{aligned}
\end{equation}
\noindent
where it cost extra 1 bit to describe whether to use inclusive or exclusive strategy.

%>>>>4. Removing assignment and define f.
Finally, we define the MDL-based cost function $f$
over a schema and entity pairs as follows:
\begin{equation}
\label{eqn:f}
f(EP,\, S) = Cost(S) + Cost(EP(S)\, |\, S).
\end{equation}
As we can see, minimizing the whole cost $Z$ in our paraphrasing problem (see \defref{def:pp})
is equivalent to minimizing the transmitting cost in \eqnref{eqn:mdl}.
In the next section, we will discuss the method to generate candidate schemas
and necessary information ($|HP(S)|$ and $|EP(S)|$) used in the cost function.

%
%%>>>>4. state "assignment matrix" A, because one schema is not enough, we need to assign tasks.
%    % 1-to-many, we will explain why later
%    % use a diagram to show "hit" v.s. "assign"
%    % formal def. of assign
%    % Then we have f(EP, S) = min_A \in A_space {Cost(S) + Cost(Grp_A(S) | S)}
%In order to properly define the cost function $f$ based on MDL principle, we are going to
%explore what does a single schema contributes to the transmitting cost in \eqnref{eqn:mdl}.
%Since each schema in $OS$ is independent of each other, therefore
%$Cost(OS) = \sum\nolimits_{S \in OS} Cost(S)$, which is the summation cost of transmitting each schema.
%
%Now we focus on $Cost(EP\, |\, OS)$. Due to the fact that a single schema is not able to cover
%all pairs in $EP$, we assign each schema with a small part of entity pairs,
%such that each pair will be described by some schemas.
%
%\begin{defn}
%\textit{Assignment Matrix}
%
%Let $EP$ be entity pairs $\{ep_1, ..., ep_n\}$, $OS$ be output schemas $\{S_1, ..., S_m\}$.
%An \textit{assignment matrix} $A_{n \times m}$ is a 0/1 matrix satisfying:
%\begin{itemize}
%    \item[-] $A_{ij} = 1 \Rightarrow ep_i \in HP(S_j)$,
%    \item[-] $\forall ep_i, \exists S_j$, such that $A_{ij} = 1$.
%\end{itemize}
%Besides, we define $EP_A(S)$ as all entity pairs assigned to $S$.
%\end{defn}
%
%Given an assignment matrix $A$, the cost of transmitting entity pairs is
%made up of each schema transmitting its own pairs:
%$Cost(EP\, |\, OS) = \sum\nolimits_{S \in OS} Cost(EP_A(S)\, |\, S)$.
%
%\KQ{figure again, different assignments.}
%Figure b shows two example assignments on ``parentOf'' entity pairs.
%\KQ{Note that one entity pair can be assigned to more than one schema.
%We will explain why later.}
%As we can see in the example, more than one assignment is available.
%Let $\mathbb{A}$ be the space of all assignments, we define $\hat{A}$ as
%the assignment at minimum cost:
%\begin{equation}
%\hat{A} = \arg\underset{A \in \mathbb{A}}{\min} \sum\limits_{S \in OS} Cost(S) + Cost(EP_A(S)\, |\, S)
%\end{equation}
%Therefore, we define MDL-based cost function $f$ based on the assignment bringing
%the minimum cost, as follows:
%\begin{equation}
%f(EP,\, S) = Cost(S) + Cost(EP_{\hat{A}}(S)\, |\, S)
%\end{equation}
%
%
%
%
%%>>>>5. summary
%In summary, we follow the MDL principle and define the cost function of a schema
%describing entity pairs. In the next section, we model our paraphrasing problem
%as a task of integer linear programming.

%
%The cost of transmitting entity pairs given schemas is defined as Equation 3.
%The whole cost is the sum of transmitting each individual entity pair.
%For each $\langle e_1, e_2 \rangle,$ if it's hit by at least one schema in $\vec{S}$,
%the schema with smallest hits is chosen to transmit the pair;
%otherwise, we transmit $e_1$ and $e_2$ directly.
%Besides, we need extra $\log{(|\vec{S}|+1)}$ bits to indicate which schema it chooses or not.
%
%\begin{equation}
%\begin{aligned}
%    & Cost(EP|\vec{S})= \sum\limits_{ep \in EP} Cost(ep | \vec{S}), \\
%    & Cost(ep | \vec{S}) = \log{(|\vec{S}|+1)} + \\
%    & \left\{
%        \begin{aligned}
%        & \min_{S : ep \in HP(S)} & \log{|HP(S)|} &    & \exists S \in \vec{S}, ep \in HP(S)\\
%        &                         & 2 \log{|E|}   &    & \text{otherwise}  \\
%        \end{aligned}
%    \right. \\
%\end{aligned}
%\end{equation}

%
%The intuition that a good $S$ can reduce the cost of transmitting $EP$ is,
%the entity pairs are highly related with the hitting pairs of $S$, such that we
%do not need to transmit every entity in $EP$, instead, we just transmit constraints
%to control which pairs are transmitted from its hitting set $HP(S)$. \KQ{I'm not sure the word
%``constraint'' is suitable or not}
%
%For example, suppose user has already get $S$ \KQ{we use a true graph $S$ representing ``mother of'' relation},
%and the user want to transmit $\langle Bill Gates, Mary Gates\rangle$ in $EP$. Since $HP(S)$ has the pair,
%specifying ``subject is $Bill Gates$'' is informative enough to return the pair in $HP(S)$.
%
%%Given an entity pair $\langle e_1, e_2 \rangle$, the the intuition of reducing transmission
%%cost is that, with the knowledge of $S$ and $KB$,
%
%The intuition that a good $S$ can reduce the cost of transmitting $EP$ is,
%given an entity $e$, if a ground graph $G$ generated from $S$ has $e_{subj} = e$,
%then the object argument $e_{obj}$ is likely to form a pair with $e$ in $EP$.
%Similar intuition holds when $e$ occurs in the object side.
%%Alright, that sentence is so long......
%
%For example, suppose a group $\{ \langle e', e_1 \rangle, ..., \langle e', e_{20} \rangle \}$ with
%the same $e'$ at the subject side.
%If $S$ generates 20 ground graphs with $e_{subj} = e'$, hitting every pair in the group, then
%we can transmit entity $e'$ only, instead of 21 different entities.
%Even if $S$ generates 25 related ground graphs, where 5 hits are not in $EP$, the entities
%for transmitting is 6, which is still much fewer than original 21 entities.
%
%We define the cost of transmitting a pair group as below:
%\begin{itemize}     % we can add a running example figure illustrating these notions
%  \item[*] A pair group $grp$ is a set of pairs $\{\langle e', e_1 \rangle, ..., \langle e', e_n \rangle\}$
%  sharing subjects, or $\{\langle e_1, e' \rangle, ..., \langle e_n, e' \rangle\}$ sharing objects.
%  The side of sharing entity $e'$ is called \textbf{sharing side}, the other side called \textbf{different side}.
%  \item[*] A hit group $hgrp(S, grp)$ is a set of entity pairs $\langle e_{subj}, e_{obj} \rangle$
%  hit by $S$, where all $e_{subj} = e'$ if $grp$ shares subject, or vice versa.
%  \item[*] $ED(grp)$ is the set of \textbf{E}ntities at the \textbf{D}ifferent side of $grp$.
%  \item[*] $EH(S, grp)$ is the set of \textbf{E}ntities at the different side
%  of \textbf{H}it group $hgrp(S, grp)$.
%  \item[*] The cost of transmitting the pair group given $S$ is:
%\end{itemize}
%
%\begin{equation}
%\begin{aligned}
%Cost & (grp|S) = \{\min \{|grp|, |ED(grp) \cup EH(S, grp)| \\
%     & - |ED(grp) \cap EH(S, grp)|\} + 1\} * |E|
%\end{aligned}
%\end{equation}
%
%
%Learnt from Equation 3, for one $grp$, the sharing entity $e'$ must be transmitted.
%For the remaining part, we have two strategies:
%transmitting entities at different side one by one,
%or only transmitting part of entities not in ED or EH.
%
%$Cost(EP | S)$ is defined as Equation 4.
%We assign each pair in $EP$ into groups (could be more than one), which satisfies that
%each relation group shares one subject or object. For one assignment, the bits of transmitting
%$EP$ equals to the summation cost of transmitting each group.
%Therefore, the cost of transmitting $EP$ is the minimum cost among all possible assignments.
%
%\begin{equation}
%Cost(EP | S) = \min_{ass \in assSpace} \sum\limits_{i=1}^{|ass|} Cost(grp_i | S)
%\end{equation}
%
%% Add formula here.
%% Therefore, this is really a searching problem.
%



\section{Candidate Schema Generation}
In this section, we propose a searching algorithm to generation
a bunch of candidate schemas (CS) used in the paraphrasing task.
Besides, we also explore the size of hit pairs and all hit pairs in $EP$.
%which satisfies what?

%basic framework 5 sents
%1 Intro
Basically, the searching process is made up of three parts.
%2 BFS
In the first part, we generate a list of most simple schema candidates,
which are descriptive over a part of $EP$.
%3 DFS
With simple schema candidates as the starting point, in the second part, we expand schemas by 
recursively adding new schema predicates to previous candidates, making schemas more and more 
specific.
%4 HP maintain
During the searching process, we maintain ground graphs of each candidate, and
calculate number of hit pairs that a candidate schema has.
%5 Selection
Finally, we select a finite number of well descriptive schemas from all candidates,
measured by a cost function which is similar with \eqnref{eqn:f}

\subsection{Simple Schema Retrieval}
%BFS part (with hit pair gen.)
%1. what is simple path
At first, we state the ``most simple'' schema as a \textbf{solid} path, that is,
all vertices in the schema are variables, connecting $x_{subj}$ and $x_{obj}$.
%2. intuition
Intuitively, this path represents a join operation over necessary predicates in $KB$, 
which is the skeleton of more complex schemas.

%3-6. where the path comes

In order to find this kind of 

In order to search descriptive paths for $EP$, we use breadth-first search algorithm 
to find paths for each entity pair $\langle e_1, e_2 \rangle$.
The basic idea is to use a breadth-first search algorithm to 
Due to the existence of various predicates and popular entity in $KB$, a exhausted search
could result in many long paths, but meaningless in natural language.
Therefore, we use a parameter $k$ to limit the maximum length of a path, which
controls the searching scope.
In practice, we use a bidirectional strategy to save time:
We start from both $e_1$ and $e_2$, then search $k/2$ layers at each side simultaneously,
recording different intermediate entity $e'$ with all paths between $\langle e_1, e' \rangle$
and $\langle e', e_2 \rangle$.
By merging paths on $e'$, we retrieve all paths from $e_1$ to $e_2$.
\KQ{Add figures showing example paths in the running ex.}

%7-9. how to bring subgraphs. (including sampling)
In the process of breadth-first search, we actually recorded all entities in each individual schema $S$ (a path),
so all ground graphs within $EP$ are stored. \KQ{Add a figure to show how an example
ground graph is generated in BFS.}
Therefore, we keep these ground graphs and count the number of hit pairs of $S$ in $EP$.
Besides, we aim to calculate the size of $|HP(S)|$ within the scope of whole $KB$, 
which is processed by querying relations in $KB$.

\KQ{Suppose we have an example schema $x_{subj}--rel_1--x_1--rel_2--x_2--rel_3--x_{obj}$, adding into a fig.}
By querying $rel_1$ in $KB$, we retrieve all possible $\langle e_{subj}, e_1 \rangle$ pairs.
Then we take all possible $e_1$ and further query $rel_2$, resulting in possible $\langle e_{subj}, e_1, e_2 \rangle$ triples.
Finally, by querying $rel_3$, all $\langle e_{subj}, e_1, e_2, e_{obj} \rangle$ tuples could be retrieved,
representing different ground graphs of the schema.
\KQ{Replace all $rel_x$ with specific predicates.}
However, due to the difference of intermediate entities, the size of ground graphs could be too large to store and process.
Thus, we sample subgraphs in each query step. 
For example, suppose we get too many $\langle e_{subj}, e_1, e_2 \rangle$ triples,
we first group triple by $\langle e_{subj}, e_2 \rangle$, and then equally select some group of triple,
which reduces the size of ground graphs to a satisfiable limit.

\subsection{Schema Expansion}
%DFS part
%6 sents
%1. input?
In the second part of searching process, we take the simple schemas (paths) and 
corresponding ground graphs as input, and explore more specific schemas.
%2. how to expand
Based on the definition of schema graph, there are two kinds of adding strategies listed below:
\begin{itemize}
    \item[-] Add a new variable, and use a \textit{solid} relation to connect the variable with an existing variable, 
    but \textbf{except} $x_{subj}$ and $x_{obj}$;
    \item[-] Add a specific entity (or type), and use a \textit{dashed} (or \textit{isa}) relation to
    connect the entity (or type) with an existing variable.
\end{itemize}
%3. dfs
While there are many specific possible adding edges for one schema, we adopt
a depth-first search to recursively add edges one by one.
%4. formula
Therefore, we need to determine which edge is to added for the current state of schema in the DFS process.
Among several potential schemas (after adding an edge), we use a cost function to measure the quality
of each schema:
\begin{equation}
\label{eqn:dfs}
\begin{aligned}
g(EP,\, S) = & Cost(S) + Cost(EP(S)\, |\, S) \\
             & + 2 \log |E| \cdot (|EP| - |EP(S)|).
\end{aligned}
\end{equation}

\noindent
Note that the difference of $g$ and $f$ (\eqnref{eqn:f}) is to increase
$2 \log |E|$ bits for each entity pair not in $EP(S)$.
Since a schema cannot cover all pair in $EP$, the intuition of $g$ is the cost of 
transmitting $EP$ by using $S$ only: if $S$ cannot describe some entity pair in $EP$,
we have to directly transmit these two entities.
Therefore, we can compare the quality between general and specific schemas.

%5. limit
Also, we need to limit the scope schemas in DFS process. We follow the idea of $k$
in the step of simple schema retrieval, and limits the variables in the expanded
schema.



\section{Search Best Schemas from Candidates}

%Goal: Given candidate and a list of info, find the best schemas sat. minimum cost based on MDL
In the final step of schema inferring, given all candidate schemas along with its costs and hit numbers
as input, the output is to pick a set a schemas $\vec{S}$ at the smallest cost of transmitting both
schemas and entity pairs.



%Problem: Nonlinear Programming
We model this subtask as a problem of integer linear programming (ILP).
Before describing the objective function and constraints, 
we state some necessary symbols:

\begin{itemize}
  \item[*] Entity Pairs: $\{ep_1, ..., ep_n\}$,
  \item[*] Candidate Schemas: $\{S_1, ..., S_m\}$,
  \item[*] $l_{ij} \in \{0, 1\}$ indicates if $ep_i$ is hit by $S_j$,
  \item[*] $cost_j$ indicates the cost of $j$-th schema: $Cost(S_j)$,
  \item[*] $h_j$ indicates the total hit number of $S_j$,
  \item[*] $x_{ij} \in \{0, 1\}$: the \textbf{variable} indicating whether
  $ep_i$ is assigned to $S_j$.
  \item[*] $y_j \in \{0, 1\}$: the \textbf{variable} indicating whether
  at least one $ep$ is assigned to $S_j$.
\end{itemize}
Firstly, we follow the definition of $Cost(EP, \vec{S})$ and describe the
objective function:
\begin{equation}
\begin{aligned}
  \min ~ Z = & \sum\nolimits_j \{ y_j * (cost_j + 1 + \\
             & \log h_j * \min \{\sum\nolimits_i x_{ij}, h_j - \sum\nolimits_i x_{ij} \}~ )~ \},
\end{aligned}
\end{equation}
where the part except $cost_j$ is the cost of transmitting an entity group
by $j$-th schema, that is $Cost(G_{EP}(S_j)~ |~ S_j)$.

Since the definition of $Z$ contains a minimum operation, 
we need to perform some transformation, producing a linear objective.
Therefore, we split each schema into two dummy nodes. Each node indicates
a \textit{included} or \textit{excluded} way to transmit entity pairs.
The extended symbols are shown below:

\begin{itemize}
  \item[*] $k \in \{0, 1\}$ indicates the way to describe entity pairs
  given a schema: 0 for \textit{included}, 1 for \textit{excluded},
  \item[*] Dummy node $S_{jk}$ indicates a schema $S_j$ and the $k$-th
  way to describe entity pairs given $S_j$,
  \item[*] $c_{jk}$ indicates the \textit{virtual description cost} of
  $S_{jk}$, where $c_{j0} = cost_j + 1$, $c_{j1} = cost_j + 1 + h_j \log h_j$,
  \item[*] $d_{jk}$ indicates the \textit{virtual additional cost} of transmitting
  a row included in (or excluded from) hit set of $S_j$, where $d_{j0} = 
  \log h_j$, $d_{j1} = - \log h_j$,
  \item[*] $x_{ijk} \in \{0, 1\}$: the \textbf{variable} indicating whether
  $ep_i$ is assigned to $S_{jk}$,
  \item[*] $y_{jk} \in \{0, 1\}$: the \textbf{variable} indicating whether
  at least one $ep$ is assigned to $S_{jk}$.
\end{itemize}

The objective function is: 
\begin{equation}
\begin{aligned}
  \min ~ Z & = \sum\nolimits_j \{ y_{j0} * (cost_j + 1 + \log h_j * \sum\nolimits_i x_{ij0} ) \\
           & + y_{j1} * (cost_j + 1 + \log h_j * (h_j - \sum\nolimits_i x_{ij1}) ) \}  \\
           & = \sum\nolimits_k \sum\nolimits_j \{ y_{jk} c_{jk} + y_{jk} d_{jk} \sum\nolimits_i x_{ijk} \}
\end{aligned}
\end{equation}

Based on the definition of $x_{ijk}$ and $y_{jk}$, 
we restrict that $\sum\nolimits_i x_{ijk} = 0$ \textbf{iff} $y_{jk} = 0$ (constraint 2 listed below).
Therefore, $y_{jk} d_{jk} \sum\nolimits_i x_{ijk}$ should always equal to $d_{jk} \sum\nolimits_i x_{ijk}$.
Finally, the function in Eq. 7 is transformed to a linear objective: 

\begin{equation}
  \min ~ Z = \sum\nolimits_k \sum\nolimits_j 
        \{ c_{jk} y_{jk} + d_{jk} \sum\nolimits_i x_{ijk} \},
\end{equation}

where the following constraints apply:

1. If a schema $S_j$ is chosen, then only one way (\textit{included} or \textit{excluded}) 
of describing entity pairs should be used:
\begin{center}
  $\sum\nolimits_k y_{jk} \leq 1, \forall j$
\end{center}

2. $S_{jk}$ should be chosen if at least one pair is assigned to it,
and shouldn't be chosen if none of pairs are assigned:
\begin{center}
  $\frac 1n \sum\nolimits_i x_{ijk} \leq y_{jk} \leq \sum\nolimits_i x_{ijk}, \forall j,k$
\end{center}

3. Each pair can only be assigned to schemas hitting it:
\begin{center}
  $x_{ijk} \leq l_{ij}, \forall i,j,k$
\end{center}

4. Each pair should be assigned to \textbf{at least} one schema:
\begin{center}
  $\sum\nolimits_k \sum\nolimits_j x_{ijk} \geq 1, \forall i$
\end{center}


%\bibliographystyle{acl}
%\bibliography{qa}

\section{Experiments}
\label{sec:experiments}
In this section, we conduct
extensive experiments on slogan generation 
to evaluate the performance
of the proposed model SALE.
We introduce the dataset, 
the competing models and parameter settings,
as well as the evaluation metrics.
We also demonstrate the experimental results in a series of evaluations
and perform further analyses on the effectiveness of our approach
in generating accurate, fluent, informative and attractive slogans.

\subsection{Dataset}
\label{sec:dataset}
We first introduce the text corpora we create
for slogan generation task in e-commerce.
Then we describe the evaluation dataset we used in
our following experiments.
The datasets are released at \url{https://202.120.38.146/slogan/}.

\subsubsection{Dataset for Slogan Generation}
\label{sec:corpora}
%In this section, we describe the experimental setup,
%especially the hyper-parameter configurations of 
%the Seq2Seq framework we used in following experiments. 
%We also detail dataset used in our experiments.
Slogan generation in E-commerce is a relative new problem.
Thus, there is a lack of dataset for this task.
We created a new dataset, containing 
the basic information of the topics attending to potential focuses or selling-points,
including the topic and its item preference, as well as the slogan.
The data are collected from Taobao, a large-scale website for e-commerce in China.

We use the pattern of ``\emph{PV} + \emph{CG}" 
to construct topics from frequent phrases mining from largely amount
of query logs and product titles.
The product titles are composed by the sellers and content producers on the
website.
We construct multiple item preferences for each topic by sampling items from 
secondary categories as well as human intervention to 
make the items with an item preference concentrate more on a specific focus 
or selling-point.
Thus, in each instance, a topic is annotated with an item preference semi-automatically
by leveraging the category ontology introduced in \secref{sec:introduction}.
Then, we recruit experts to write a slogan for each data instance.
Overall, the dataset contains 857 topics and 
in total 3,555 $(x, p, y)$ instances after preprocessing.

We use four splits named (train/dev/LMdev/test) in our experiments.
Note that, the LMdev split is for 
hyper-parameter $\beta$ tuning (see \secref{sec:shallow_fusion} in details).
The splits are randomly divided based on topics 
proportionally by 90\%, 5\%, 1.5\% and 3.5\%.
Thus each split of (train/dev/LMdev/test) includes 771, 43, 13, 30 topics separately,
and correspond to 3132 training instances, 231 development instances, 
50 LM development instances, as well as 142 test instances.

%For the evaluation dataset, 
\subsubsection{Evaluation Dataset}
\label{sec:eval_dataset}
We perform algorithm evaluation and human evaluation
in our experiments (see \secref{sec:metrics} in details).
Thus we provide two evaluation datasets separately for each.
We directly use all the 142 instances of test split in \secref{sec:corpora},
referred as FULLtest,
for the algorithm evaluation which are based on automatic scoring systems,
such as BLEU.
Besides, we randomly sample 50 instances from the test split
to form a small evaluation dataset for human evaluation,
referred as HUMtest.


\subsection{Compared Methods}
\label{sec:compared}
In this section, we introduce the baseline and choices for 
our model components, as well as the parameter settings
used in those models.

\subsubsection{Baselines and SALE}
\label{sec:baselines}
According to the problem statement (in \secref{sec:problem})
and the proposed item preference fusion methods (in \secref{sec:preference}),
the models for comparison backed by Seq2Seq framework are mainly one-way input models and two-way input models.

One-way input models (prefixed by \emph{One}) takes in one-way input as the source sequence,
and the slogan as its target sequence, without considering 
semantics enhancement or incorporating pretrained language model.
There are three one-way input baselines with different inputs.
\textbf{One-T} (\textbf{t}opic) model
takes the topic itself as its source sequence,
while \textbf{One-P} (item \textbf{p}reference) model takes
the titles of items as its source sequence.
Then, 
%while 
\textbf{One-CAT} (con\textbf{cat}nating) model 
concatenates the topic and its item preference with special token \emph{SEP}
as a separator, and takes the sequence of concatenation as its source sequence.

The two-way input models (prefixed by \emph{Two})
are designed to treat topics and item preferences heterogeneously.
We propose two kinds of two-way input models
based on different heterogeneous inputs fusion methods 
(see details in~\secref{sec:preference}).
%we propose two fusion methods in \secref{sec:preference}
%to combine the heterogeneous inputs.
\textbf{Two-BiAttn} (\textbf{bi}directional \textbf{att}ending) model use the two-way bidirectional attending
to combine the representations of topics and that of item preferences.
\textbf{Two-CAT} (con\textbf{cat}nating) model use two-way concatenating strategy 
to fuse the heterogeneous outputs of encoders.

For \textbf{SALE}, 
we incorporate the semantics enhancement module
(in \secref{sec:semantics}) to enrich
the deep contextualized representations
backed by Two-CAT baseline.
\textbf{SALE+PLM} integrates
%On the basis of SALE, 
%we integrate
pre-trained language model (PLM) 
into SALE at inference time in order to improve
the generalization and robustness of the model.
%Specially, SALE identifies the \emph{is-a} relations
%among heterogeneous inputs and
%increase the semantic capacity of the model for better contextualized representations
%knowledge-aware module

\subsubsection{Parameter Settings}
We use an architecture of 8 stacked convolutional layers 
for both the topic encoder and the item preference encoder
as well as the decoder parts with kernel width as 3.
To enable deep convolutional networks, 
we add residual connections~\cite{he2016deep} from the input of each convolution
to the output of the layer as well.
For each convolutional layer, we set the hidden vector size as 512
and the embedding size as 256.
To alleviate the overfitting problem, we add the dropout ($p=0.2$)
layer~\cite{srivastava2014dropout} for all convolutional layers and fully connected layers.

To optimize the proposed models,
we use Nesterov's accelerated gradient method
~\cite{sutskever2013importance} with gradient clipping 0.1
~\cite{pascanu2013difficulty},
momentum 0.99, and 
learning rate 0.2.
We terminate the training process when the learning rate drops 
below 10e-5.
We set beam size as 5 for the beam search algorithm
in the testing step.
The hyper-parameter $\beta$ of SALE-PLM (in ~\eqnref{eq:shallow_fusion})
was selected to maximize the generation performance
on the LMdev split by grid search, from the range 1e-4 and 0.1.


\subsection{Evaluation Metrics}
\label{sec:metrics}
We perform both algorithm evaluation and human evaluation
in our experiments.
Specially, we evaluate our model on generation quality which includes
the automatic scoring metrics such as
BLEU and lexical diversity,
as well as a number of human-evaluation metrics.

\paragraph{BLEU}
The BLEU algorithm~\cite{papineni2002bleu} compares consecutive phrases of the 
generated slogan with the consecutive phrases it finds
in the reference slogan, and counts the number of matches, in a weighted fashion.
A higher BLEU score indicates a higher degree of similarity with the reference
slogan.
We compare all competing models on test split in terms of the BLEU score as a sanity check.
We also use BLEU score as the standard metric to finetune
hyper-parameter $\beta$ in SALE+PLM model.

\paragraph{Lexical Diversity}
A common problem in automatic text generation is that the system tends to generate safe
answers with enough diversity~\cite{li2016deep}.
A low diversity score often means generated contents are general and vague, 
while higher diversity means the generated contents are more informative and 
interesting.
Following~\cite{ChenLZYZ019}, we calculate the number of distinct n-grams produced on the test split
as the measurement of the diversity of generated descriptions.

\paragraph{Human-evaluation Metrics}
Automatic scoring metrics including BLEU score and lexical diversity are competitive and inexpensive to operate.
However, they do not consider
other important aspects such as intelligibility and grammatical correctness (or fluency) of slogan.
We use several human-evaluation metrics
to evaluate competing models on various perspectives.
\begin{itemize}
	\item \textbf{Overall quality} is designed to measure the
	overall generation quality of model.
	\item \textbf{Relevancy} is used to measure the content relevancy of generated slogan to the given topic and items.
	\item \textbf{Fluency} focus on the intelligibility and grammatical correctness of generated slogan.
	\item \textbf{Interestingness} takes personification and attractiveness into account.
\end{itemize}


\subsection{Performance Comparisons and Analysis}
\label{sec:results}


In this section, we conduct an analysis of our proposed model
to evaluate the contribution of item preference fusion module and
semantics enhancement module as well as the integration of 
pre-trained language model.

We evaluate competing models on FULLtest and HUMtest 
as we described in \secref{sec:eval_dataset}.
The comparison results of slogan generation are shown in 
\tabref{tab:auto_eval} and \tabref{tab:human_eval}.
For human evaluation, we recruit three experts as annotators 
and ask them to give scores on each aspect of generated slogan, 
range from 1 to 5,
then average the scores of each aspect on HUMtest as 
human evaluation results.


\begin{table*}[th]
	%	\small
	\centering
	\caption{Slogan generation results comparison with baseline methods using FULLtest.}
	\label{tab:auto_eval}
	\begin{tabular}{lcccc}
		\hline
		Model %& Overall quality 
		& BLEU &  Diversity (n=2) ($\times 10^2$ )& Diversity (n=3) ($\times 10^2$ ) & Diversity (n=4) ($\times 10^2$ ) \\
		\hline
		One-T %&  3.30  
		&  28.34 &  2.25   &  2.45  &  2.37 \\
		One-P %&  3.86  
		&  41.11 &   4.74 &    5.99 & 6.33 \\
		One-CAT  % & 3.82  
		& 38.86  &  3.86 &  4.77  & 4.94 \\
		Two-BiAttn  % & 3.86  
		& 36.99  &  4.82 &  5.87  &  6.03   \\
		Two-CAT % & 3.95
		& 40.59  &  4.89 &  5.99  &  6.22 \\
		\hline\hline
		SALE % & \textbf{4.16}  
		& 42.31  & 4.87  &  6.20 &  6.55  \\
		SALE+PLM % & -  
		& \textbf{42.36}   &  \textbf{4.89} & \textbf{6.23}  &  \textbf{6.57}  \\
		%		SingleSG$_{\mathrm{concept}}$ & 28.34 &  3.30 & 3.32 & 4.31 & 4.38 \\
		%		SingleSG$_{\mathrm{items}}$& 41.11 & 3.84 & 4.0 & 4.30 & 4.22  \\
		%		MultiSG-{biattn} & 36.99 & 3.86 & 4.05 & 4.17 & 4.11 \\
		%		MultiSG-{cat} & 40.59 & 3.95 & 4.13 & 4.34 & 4.23  \\
		\hline 
	\end{tabular}
\end{table*}



\begin{table}[th]
	\small
	\centering
	\caption{Human evaluation for slogan generation task using HUMtest.}
	\label{tab:human_eval}
	\begin{tabular}{lcccc}
		\hline
		Model & Overall quality & Relevancy &  Fluency & Interestingness \\
		\hline
		One-T &  3.30  &  3.32 &  4.31   &  4.38 \\
		One-P &  3.84 &  4.0 &   4.30 &    4.22  \\
		One-CAT  &  3.62  & 3.94  & 4.24  & 4.18  \\
		Two-BiAttn  & 3.86  & 4.05  &  4.16  &  4.11     \\
		Two-CAT & 3.95  & 4.13  &  4.34 &  4.23   \\
		\hline\hline
		SALE & \textbf{4.16}  & \textbf{4.32}  & \textbf{4.53}  &  \textbf{4.43}  \\
		SALE+PLM & -  & -   &  - &   -  \\
		%		SingleSG$_{\mathrm{concept}}$ & 28.34 &  3.30 & 3.32 & 4.31 & 4.38 \\
		%		SingleSG$_{\mathrm{items}}$& 41.11 & 3.84 & 4.0 & 4.30 & 4.22  \\
		%		MultiSG-{biattn} & 36.99 & 3.86 & 4.05 & 4.17 & 4.11 \\
		%		MultiSG-{cat} & 40.59 & 3.95 & 4.13 & 4.34 & 4.23  \\
		\hline 
	\end{tabular}
\end{table}



Firstly, we show the importance of item preference for  slogan generation.
We introduce item preference features for specific topic
using category ontology as discussed in \secref{sec:preference}.
Topic and its item reference are simply concatenate into one input sequence
in One-CAT model. 
As we can see that One-CAT substantially outperforms One-T which only use topic as input
with an advantage of +0.32 overall quality (relatively 9.7\%), +10.5 BLEU, +110.7\% diversity ($n=2$), +144\% diversity ($n=3$) and +81\% diversity ($n=4$),
Thus, item preference plays an important role in slogan generation task.

However, the results show that
One-P model which only takes in item preference as input outperforms
One-CAT on various metrics.
This imposes that topic and item preference are two heterogeneous inputs,
thus we should treat them differently in the model using item preference fusion method.
Next, we analyze the contribution of item preference fusion methods proposed
in \secref{sec:preference}
by comparing One-CAT, Two-BiAttn and Two-CAT.
We can see that show that two-way concatenating method for Two-CAT
substantially outperforms two-way directional attending for Two-BiAttn.
Though Two-CAT model slightly decreases on BLEU compared to One-P,
Two-CAT outperforms One-P according to human evaluation shown in \tabref{tab:human_eval}.
This suggests that two-way bidirectional attending fusion 
makes the semantics corruption between two heterogeneously deep contextualized representations.
Therefore, two-way concatenating fusion method is more effective for 
heterogeneous inputs combination.

Our proposed \emph{is-a} knowledge-aware model SALE 
is backed by Two-CAT, equipping with the semantics enhancement module.
Results show the effectiveness of semantics enhancement module
proposed in \secref{sec:semantics}.
As shown in \tabref{tab:human_eval} and \tabref{tab:auto_eval}, 
SALE outperforms Two-CAT by a substantial margin.
Specially, semantics enhancement improves the
diversity scores ($n=3, 4$) 3.5\%, 5.3\% separately .
SALE also achieves an improvement of 1.72 (relatively 4.24\%) in terms of BLEU, 
as well as an improvement of 0.57 in terms of overall quality.
We can see that SALE outperforms all previous baselines on every aspect.
Thus SALE is able to generate accurate, fluency, informative and attractive slogans.
We further illustrate this in \secref{sec:cases}. 



Lastly, we analyze the contribution of pre-trained language model integration
comparing results of SALE and SALE+PLM.
As shown in \tabref{tab:auto_eval}, 
incorporating PLM at inference stably improves the diversity 
that performs best at every n-gram diversity scores ($n=2,3,4$).
Note that, we finetuned hyper-parameter $\beta$ for SALE+PLM 
in terms of BLEU score on LMdev split,
and SALE+PLM with $\beta = 2e\mathrm{-}4$ achieves best BLEU score as 42.94.
Thus, we use $\beta=2e\mathrm{-}4$ for SALE+PLM model in test.
As shown in \tabref{tab:auto_eval}, SALE+PLM 
outperforms all competing models in terms of BLEU score as 42.36 
on FULLtest dataset.
Since the results generated by SALE and SALE+PLM are nearly the same on HUMtest,
their human evaluation results are same, 
we do not show result of SALE+PLM in \tabref{tab:human_eval}.
We can see that in this case the improvement of PLM integration is minor but stable, 
on both BLEU score and diversity scores.
We argue that such PLM integration makes our model more robust.

% the contribution of item preferences in one-way input models: OneT, OneP, OneCAT
% item preference fusion methods for two-way input models: OneCAT, Two-BiAttn, Two-CAT
% is-a knowledge-aware model SALE: Two-CAT, SALE, SALE+PLM




\subsection{Case Studies}
\label{sec:cases}
In this section, we perform case studies to observe 
how our propose methods influence the generation so that
the model can generate different slogans for a specific topic
according to different item preferences.
Besides, our proposed \emph{is-a} knowledge-aware model SALE generate higher quality slogans
benefiting from semantics enhancement.

%The running example in \tabref{sec:introduction} illustrates 

In \tabref{tab:vary_preference}, % \tabref{tab:vary_preference}
two item preferences are provided for topic ``早教玩具" (early education toys).
The first preference consists of musical toys such as ``音乐拍拍鼓" (musical patting drum),
which focuses on music education for children,
while the second preference is mainly about ``手摇铃" (rattle) which focuses on improving concentration ability 
as well as soothe emotions for babies.
The second example of 
\tabref{tab:vary_preference} is the running example we discussed in \secref{sec:introduction}.
The proposed model SALE successfully captures those focuses 
and generate attractive slogans accordingly.
For example, SALE generates \emph{music enlightenment}
for the focus of musical patting drum
and generates \emph{soothe baby's emotion } for the focus of 
rattle.
As shown in \tabref{tab:vary_preference}, 
we use red color to mark the preferences and its effects for slogan generation.

We also demonstrate the effectiveness of semantics enhancement
by comparing slogans generated by SALE and Two-CAT in \tabref{tab:semantics}.
%as shown in \tabref{tab:semantics}.
\tabref{tab:semantics_a} shows two example topics associated with an item preference each.
The entities involved in \emph{is-a} relations
have been marked as blue. 
The third column of \tabref{tab:semantics_a} demonstrates the 
identified relations.
\tabref{tab:semantics_b} compares slogans generated by SALE and Two-CAT
for the topics in \tabref{tab:semantics_a}.
Results show that SALE enhanced by \emph{is-a} knowledge 
tends to integrate the inferred user needs into slogan,
for example \emph{the first choice when preparing a gift for mom} for ``large size mother-dress"
and \emph{always protect you} for ``outdoor sports protective gear",
which further promotes user interests.


%\KZ{You need to translate these into English.}
% Please add the following required packages to your document preamble:
% \usepackage{multirow}
\begin{table*}[th!]
\begin{center}
\caption{Two examples of generated slogans by the proposed model SALE, varying
	the item preference while fixing the topic as input.}
\label{tab:vary_preference}
\small
%\subfloat[Example of slogans generated by SALE.]{
%	\label{tab:vary_preference_a}
		\begin{tabular}{c|c|c}
		\hline
%		\multicolumn{1}{c}{topic}  
		topic                                                                    
		& item preference                   
%		& semantic relations                                                                                                                    
		& slogan                                                                         
		\\ \hline
		\multirow{2}{*}{\begin{tabular}[l]{@{}l@{}} \\ 早 教 玩 具 \\ early education toys\end{tabular}} 
		& \begin{tabular}[l]{p{65mm}l@{}}
%			澳 贝 青 蛙 小 鼓 音 乐 手 拍 鼓\\ (ao bei frog)\\ 
			儿 童 益 智 早 教 玩 具 宝 宝 \color{red}{音 乐 拍 拍 鼓} \\ 
			intelligence early childhood education toys \quad 
			\color{red}{musical patting drum for baby}
%			\\ 宝 宝 音 乐 拍 拍 鼓 儿 童 益 智 电 动 玩 具 \\ (译文)
		\end{tabular} 
		& \begin{tabular}[l]{p{65mm}l@{}}
		    \textcolor{red}{音 乐} \color{black}{早 教} \color{red}{启 蒙} , 
			\color{black}{宝 宝 智 能 } \color{red}{手 拍 鼓} \\ 
			early childhood education \quad \color{red}{music} \quad \color{red}{enlightenment}
			\textcolor{black}{, intelligent} \quad \color{red}{patting drum} \quad
			\color{black}{for baby} 
		\end{tabular} \\ \cline{2-3} 
		& \begin{tabular}[l]{p{65mm}l@{}} 宝 宝 益 智 早 教 婴 幼 儿 \color{red}{手 摇 铃} \\
			\textcolor{red}{rattle} \color{black}{for baby intelligence early education}
%			\\ 澳 贝 新 生 婴 儿 牙 胶 手 摇 铃\\ (译文) 
		\end{tabular}            
%		& \begin{tabular}[c]{@{}l@{}}
%			手 拍 鼓, \emph{hypo}, 玩 具
%		\end{tabular}                                                    
		& \begin{tabular}[l]{p{65mm}l@{}}婴 儿 益 智 \color{red}{摇 铃}, \color{red}{安 抚} \color{black}{宝 宝} \color{red}{情 绪} 
			\color{black}{神 器}\\ intelligence development \color{red}{rattle} 
			\color{black}{for baby, the best tool to} \color{red}{soothe} \color{black}{the baby}
		\end{tabular}    \\ \hline
%	\end{tabular}
%}
%\cut{%%%%%%%%%%%%
%\qquad
%\subfloat[]{
%	\label{tab:vary_preference_b}
%	\begin{tabular}{c|c|c}
%		\hline
		%		\multicolumn{1}{c}{topic}  
%		topic                                                                    
%		& item preference                   
%		%		& semantic relations                                                                                                                    
%		& slogan                                                                         
%		\\ \hline
		\multirow{2}{*}{\begin{tabular}[l]{@{}l@{}} \\ 玻 璃 灯 具\\ glass light fixture \end{tabular}} 
		& \begin{tabular}[l]{p{65mm}l@{}}客 厅 \color{red}{ 现 代 简 约 吸 顶 灯} \color{black}{两 室 一 厅 套 装 灯} \\ 
			\textcolor{red}{morden style living room ceiling light} \quad light set for two-bedroom apartment
%			\\ 创 意 led 客 厅 吸 顶 灯 水 晶 灯 \\ (译文)
		\end{tabular} 
		& \begin{tabular}[l]{p{65mm}l@{}}\textcolor{red}{现 代} \color{black}{元 素} \color{red}{吸 顶 灯} , 彰 显 \color{red}{极 简} 魅 力 \\ 
		\textcolor{red}{ceiling lights} \color{black}{in} \textcolor{red}{modern} \color{black}{style, shining} \textcolor{red}{minimalist} \color{black}{charm} 
		\end{tabular} \\ \cline{2-3}
		& \begin{tabular}[l]{p{65mm}l@{}}
%			台 灯 卧 室 床 头 灯 温 馨 浪 漫 ins 少 女 个 性 创 意 \\(译文) \\ 
			钟 爱 一 生 台 灯 卧 室 \color{red}{暖 光} \color{black}{床 头 灯} \color{red}{温 馨} \color{black}{布 艺} \\ 
			the favorite table lamp \quad table lamp with \color{red}{warm ligth} \color{black}{for living room} \quad \color{red}{warm} \color{black}{cloth art}
		\end{tabular}            
		& \begin{tabular}[l]{p{65mm}l@{}} 一 灯 一 世 界, 一 亮 一 \color{red}{温 馨} \\ 
		lights in your world, bright and \textcolor{red}{warm}  
		\end{tabular} \\ \hline
	\end{tabular}
%    }%%%%%%%%%%
%}
\end{center}
\end{table*}


\begin{table*}[th!]
	\begin{center}
		\caption{The influence of semantics enhancement for slogan generation.}
		\label{tab:semantics}
		\small
		\subfloat[Examples of relation identification in SALE.]{
			\label{tab:semantics_a}
			\begin{tabular}{c|c|c}
				\hline
				%		\multicolumn{1}{c}{topic}  
				topic                                                                    
				& item preference       
				%		& semantic relations                                                                                                                    
				& semantic relations                                                                         
				\\ \hline
				\begin{tabular}{p{10em}}
				大 码 \color{blue}{妈 妈 装 }\\ large size \color{blue}{mother-dress}
				\end{tabular}
				& \begin{tabular}{p{20em}}
				中 老 年 \color{blue}{女 装} \color{black}{秋 装 长 袖 连 衣 裙 夏 中 年 妈 妈 装 打 底 衫 秋 春 季 大 码 连 衣 裙 子} \\ middle-aged and old \color{blue}{women's clothing} \quad \color{black}{autumn long sleeves \quad summer dress  \quad blouses for middled-aged women  \quad large size dresses for spring and autumn
				}   \end{tabular} 
				& \begin{tabular}{p{12em}<{\centering}} (女 装, \emph{hyper}, 妈 妈 装)  \\ (women's clothing, \emph{hyper}, mother-dress ) \end{tabular} \\ 
				\hline
				
				\begin{tabular}{p{10em}}
					户 外 运 动 \color{blue}{ 护 具 }\\ outdoor sports \color{blue}{protective gear}
				\end{tabular}
				& \begin{tabular}{p{20em}}
					裤 袜 加 长 \color{blue}{护 小 腿 } \color{black}{超 薄 跑 步 健 身 } \color{blue}{护 膝} \color{black}{护 具 男 女 运 动 装 备}   \\
					lengthen legging pantyhose \quad \color{blue}{leg protector} \quad \color{black}{ultra thin sports} \color{blue}{knee pads} \quad \color{black}{sports protective gear for men and women} 
				\end{tabular} 
				& \begin{tabular}{p{12em}<{\centering}} (护 小 腿, \emph{hypo}, 护 具)  \\ (leg protector, \emph{hypo}, protective gear) \\
				 (护 膝, \emph{hypo}, 护 具) \\ (knee pad, \emph{hypo} protective gear) \end{tabular} \\ 
			    \hline
				
			\end{tabular}
		}
	\qquad
		\subfloat[Comparision of capturing potential user needs.]{
		\label{tab:semantics_b}
		\begin{tabular}{c|c}
			\hline  
			model                                     & slogans                           \\ \hline
			\begin{tabular}{p{7em}<{\centering}}Two-CAT\end{tabular}
			& \begin{tabular}{p{32em}} 中 老 年 连 衣 裙 , 时 尚 \\ 
			middle-aged and old women's dress, fashion\end{tabular} \\ 
			\begin{tabular}{p{12em}<{\centering}}SALE\end{tabular}	
			& \begin{tabular}{p{32em}} 中 老 年 连 衣 裙 , \color{blue}{送 妈 妈}  \color{black}{的 首 选}\\
			middle-aged and old women's dress, the best choice \color{blue}{for mommy}
		 \end{tabular} \\ 
			\hline
			\begin{tabular}{p{7em}<{\centering}}Two-CAT\end{tabular}
			& \begin{tabular}{p{32em}} 运 动 套 装 , 穿 出 潮 流 感 \\ 
			sports sweatsuit, fashion \end{tabular} \\ 
			\begin{tabular}{p{12em}<{\centering}}SALE\end{tabular}	
			& \begin{tabular}{p{32em}} 运 动 不 能 少 , 时 刻 \color{blue}{保 护} 你 \\ 
			exercise is indispensable, \color{blue}{protecting} \color{black}{you at all time}
			\end{tabular} \\ 
			\hline
			
		\end{tabular}
	}

	\end{center}
\end{table*}
%大 码 妈 妈 装
%中 老 年 女 装 秋 装 长 袖 连 衣 裙 夏 中 年 妈 妈 装 打 底 衫 秋 春 季 大 码 连 衣 裙 子
%中 老 年 连 衣 裙 , 送 妈 妈 的 首 选
%中 老 年 连 衣 裙 , 时 尚 时 尚
%
%户 外 运 动 护 具
%篮 球 骑 行 登 山 健 身 护 腿
%裤 袜 加 长 护 小 腿 超 薄 跑 步 健 身 护 膝 护 具 男 女 运 动 装 备
%运 动 套 装 , 穿 出 潮 流 感
%运 动 不 能 少 , 时 刻 保 护 你

%玻 璃 灯 具
%
%(glass light fixture)
%
%欧 式 吸 顶 灯 圆 形 LED 吸 顶 灯 具
%创 意 led 客 厅 吸 顶 灯 水 晶 灯


%音 乐 早 教 启 蒙 , 宝 宝 智 能 手 拍 鼓
%
%(译文)
%
%儿 童 安 抚 摇 铃 , 哄 娃 益 智 两 手 齐 抓
%
%(译文)


%儿 童 早 教
%
%(early childhood education)
%
%
%澳 贝 青 蛙 小 鼓 音 乐 手 拍 鼓,
%(译文)
%儿 童 益 智 早 教 玩 具 澳 贝 宝 宝 音 乐 拍 拍 鼓
%(译文)
%宝 宝 音 乐 拍 拍 鼓 儿 童 益 智 电 动 玩 具 
%(译文)
%
%
%音 乐 早 教 启 蒙 , 宝 宝 智 能 手 拍 鼓 
%(译文)



% Please add the following required packages to your document preamble:
% \usepackage{multirow}

%\begin{table*}[th!]
%\begin{center}
%\caption{Study cases for generated slogans.}
%\label{tab:vary_preference}
%\subfloat[Each pair of slogans is generated by varying the item preference while fixing the topic as input. ]{
%        \label{tab:case_a}
%\begin{tabular}{p{1.5em}<{\centering}|p{27em}|p{5em}}
%	\hline
%	\multicolumn{1}{c}{\multirow{2}{*}{topic: 儿 童 早 教 \\ (early childrenhood education)} }
%
%	\hline
%	\end{tabular}
%}
%\end{center}
%\end{table*}

%\caption{Each pair of slogans is generated by varying the item preference while fixing the topic as input. }

%长 袖 大 码 妈 妈 装
%
%儿 童 玩 具
%宝 宝 巴 士 正 品 奇 奇 妙 妙 形 象 熊 猫 公 仔 宝 宝 的 好 伙 伴 礼 物 娃 娃 毛 绒 玩 具
%
%儿 童 玩 具 , 玩 出 百 变 造 型
%创 意 玩 具 , 捏 出 百 变 造 型
%
% 1 2 个 月 儿 童 早 教
% 宝 宝 手 拍 鼓 , 早 教 益 智 好 伙 伴
% 音 乐 早 教 启 蒙 , 宝 宝 智 能 手 拍 鼓
% 儿 童 早 教 益 智 玩 具 清 单
% 益 智 音 乐 玩 具 , 开 发 宝 宝 无 限 智 力



\section{Related Work}
This section surveys previous works on question generation and tree encoding
respectively.

Text question generation has attracted the attention 
after the work of ~\citeauthor{du2017learning}~\shortcite{du2017learning}, who uses deep seq2seq model 
to generate questions from a raw text paragraph. 
Before that, text question generation relied heavily on hand-craft 
question patterns~\cite{HeilmanS10,LabutovBV15,MostowC09} which is time and 
labor consuming. 

However, this pure seq2seq model is not focused and 
has no control over part in the paragraph to generate question. 
~\citeauthor{zhou2017neural}~\shortcite{zhou2017neural} proposed to encode 
key phrase information using binary indicators to generate 
key-aware questions and they assumes the answer to be key phrase. 
Considering key phrase (answer) is unavailable in reality, 
~\citeauthor{SubramanianWYT17}~\shortcite{SubramanianWYT17} applied 
a two-stage approach. First, key phrases are extracted by 
pointer network~\cite{ptrnet}. Second, 
key phrases are encoded in the same way as 
Zhou et al. With the intuition that questions could be asked in many ways, 
~\citeauthor{Yao2018vae}~\shortcite{Yao2018vae} used conditional-VAE to 
increase the diversity of questions. More recently, models with 
auxiliary feature information~\cite{HarrisonW18} helped improve 
the question quality. Structure question generation aims at 
converting structured data such as triples in knowledge graph to questions. 
~\citeauthor{SerbanGGACCB16}~\shortcite{SerbanGGACCB16} proposed a model to generate factoid questions from knowledge base triples.  None of the above work
considered using parse tree structures to aid question generation process,
which is the focus of this paper.

Sequential RNN model takes sentence as a sequence of words, 
ignoring the syntactic information. In order to utilize
such syntactic information with sequential information, 
~\citeauthor{tai2015improved}~\shortcite{tai2015improved} proposed Tree-LSTM to 
encode the binary parse tree recursively in a bottom-up fashion to 
classify sentiment. In text generation task, 
\citeauthor{eriguchi2016tree}~\shortcite{eriguchi2016tree} 
proposed a tree-to-sequence model with attention mechanism to do 
machine translation and 
~\citeauthor{liang2018automatic}~\shortcite{liang2018automatic} proposed a 
tree-to-sequence model which could handle arbitrary trees, 
to do code comment generation. Our work is inspired by these previous
attempts and we are first to adapt structure encoded neural models to
textual question generations.

\bibliographystyle{aaai}
\bibliography{reference}

\end{document}
