\section{Related Work}

\KZ{First discuss AAAI 2012 and EMNLP 2011, their pros and cons and how
we stack up with them. Then discuss other less similar work. Finally 
applications that can benefit from this work, eg. QA, etc.}

% map natural language to knowledge base
Recently, more advanced approaches have been proposed that attempt to bridge the gap between natural language and knowledge base (KB) using a paraphrasing point of view. Paraphrasing has been proven useful in areas like question answering (QA) \cite{harabagiu2006methods}, relation extraction \cite{romano2006investigating}, machine translation and so on. For instance, \citeauthor{berant2014semantic} \shortcite{berant2014semantic} attacked semantic parsing by mapping natural language utterances into logical forms to be executed on a KB using a paraphrase model and furthermore improved QA performance. Graph-based representations \cite{reddy2014large} is usually used in exploiting structural and conceptual similarity between natural language and KB. Similar works \cite{fader2013paraphrase,berant2013semantic,kwiatkowski2013scaling} also apply paraphrasing techniques into question answering system. \textit{VELVET} \cite{zhang2012ontological} learned a relation extractor after paraphrasing between different ontology.

% mapping natural language relations to knowledge base schemas
Specifically, we consider the paraphrasing problem as mapping human specified relations to complex knowledge base schema graphs. Similar work like \textit{VELVET} \cite{zhang2012ontological} is based on a procedure of ontology mapping which also focus on the relations. Given a user-specified relation, ontology mapping actually generates a complex SQL expression over types and relations on the KB's schema using Markov Logic Network model and inference algorithm. Our work is also related to \citeauthor{lao2011random} \shortcite{lao2011random}, which aims to complete the imperfectly extracted knowledge base \textit{NELL} \cite{carlson2010toward} by predicting all concept $b$ which potentially have the relation $R(a, b)$ given a concept $a$. It used a random walk path finding algorithm to inference new relation instances by mapping the target \textit{NELL} relation to a join of several basic relations. In our distinctive work, we perform a breadth-first search to construct the skeleton of a specific relation schema which is similar as the path finding procedure in the previous works. Beyond relation path, we use a depth-first search to further add more information attributes to the relation path generated in the first step and transform it into a more specific graph form, under the guidance of the MDL principle. As for concrete mapping format,
MapOnto \cite{an2006discovering} used Horn clauses when produces mapping rules between two schemas. Others \cite{zhang2012ontological,} generates complex SQL queries consisting of operations like join, union, project and select as mappings.

% graph search in KB
On the part of schema graph search, \citeauthor{zou2014natural} \shortcite{zou2014natural} interpret a natural language question as a semantic query graph where each vertex represented an argument and each edge is associated with a relation phrase. Comparing to them, our schema graph is more complicated since we divide edges into three categories. Other graph search work like 
