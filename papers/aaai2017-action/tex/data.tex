\section{Data preparation}

We assume online textual news consists of title and body, where
the title discusses the outline of the subject, the body discusses
the details. Based on this assumption, the abstraction nouns are
likely to be found in news title and the action instances should
follow accordingly in the news body. To build our corpus
we used an open source python library Scrapy to crawl the Internet for textual news articles.
The result contains around 5 million news in various domains and styles.

In order to extract action instances from news body, we first need to
parse the news body into Penn trees and Postag each word in both
the titles and bodies. Then we design a set of rules to decide
which syntactical components are kept. Note that sometimes there might be ambiguity when it comes to
deciding which layers of modifiers should be kept.
 For example, in sentence ``He eats delicious hot dogs'', we want to keep
``hot'' as in ``hot dogs'' to be the object( otherwise there would be significant semantic error ),
but ``delicious'' and ``hot'' are effectively regarded as the same Part-of-speech by most NLP tools, hence we
will need external knowledge such as WordNet and Probase to help us make syntactical decision on semantic level.
Specifically, we use a sliding window to gradually increase the size of the argument, and check knowledge
base if the augmented argument is a valid entity, if so we would keep going on, otherwise stop at the last valid
object.

After we obtained the action instances, we would need an action argument concept database to map those action instances
into what we call action classes. In this paper we used the result of \cite{gong2015representing} as this database, which
includes 1,770 unique verbs and their argument concepts. The data is organized in action classes, such as ``company buy
company'', ``event destroy place'' etc.

During action class matching, we compare the subject instance and object instance with the corresponding argument concepts.
Sometimes it is possible that one action instance might match multiple action classes, in such cases we match the instance
to action class with the most specific argument concept, as measured in the hyponym number provided by Probase. For example,
``Mosquito bites him'' could be matched to both ``animal bite human'' and ``insect bite human''. Since animal subsumes insect,
we match the instance to the later one as it is semantically more accurate.

Next, we need to obtain a noun concept dictionary from which we could choose action concepts. The ideal dictionary should contain
nouns that are generally used to describe a particular kind of actions, or closely related to some actions. Since such requirement
is pretty fuzzy, we tried several different methods to generate such dictionary. The main sources we use are: 1). Noun discovered
in news titles. 2). WordNet lexical info. 3). Probase's isA relations. 4). Wiktionary top 10000 frequent English words.

We then build our noun dictionary $D$. Here the main sources we use are Wordnet,Probase and Wikitionary.

WordNet provides a special field called lexical info for most words, which is a human labeled coarse-grained category of words.
For example, for word ``theft'' the lexical info is \emph{noun.act}, for ``ceremony'' it is \emph{noun.event}.
We can use these information to generate an initial noun pool. In this stage Probase could be used in two ways, firstly it
also can be used as a pool generator. We can build the noun pool by manually designing
a small set of terms that has desired nouns as hyponyms, such terms include "activity",
"process", "event" etc. Secondly probase can be used as a filter. We check how
many instances a noun could have and use it as a measurement to decide whether it
is abstract noun or concrete instances. Probase also provides the typicality of
the hypernym-hyponym relations which we could use too. Wikitionary is an online
dictionary, we found a list of top 10000 popular English word there, hoping to
solve the problem that some words in Wordnet or Probase are too obsolete.
Overall we have 4 different methods to generate the dictionary:
\begin{enumerate}
\item Use top 10000 English word as initial pool and filter by Probase
\item Use Wordnet noun iterator, combined with lexical info to build the initial pool and filter by Probase
\item Use Probase to generate the initial pool and filter by Probase
\item Use noun discovered in news titles and filter by Probase
\end{enumerate}

After experiments, we observed the last method provides the best dictionary for action-noun map.
