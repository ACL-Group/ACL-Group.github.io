\subsection{Candidate Schema Generation}
\label{sec:candgen}

We propose a search algorithm to collect 
candidate schemas from training relation instances.
The intuition is that we first find suitable skeletons 
as a starting point, and then recursively add constraints on 
previous schemas, producing more specific candidate schemas.
As a preprocess step, we link all $e_1$ and $e_2$ into the knowledge base.
Some Open IE systems have done this step in 
advance~\cite{gabrilovich2013facc1,nakashole2012patty},
and existing entity linking tool are also 
available~\cite{zuo2014bel,cucerzan2007large}.
Entity linking is not the focus of our work.
%7 sentences introducing bfs
%1. recap
%As outlined in \secref{sec:problem}, a skeleton is a path of KB 
%predicates which connects target variables $x_1$ and $x_2$.
%2. basic: bfs

Now given a list of linked entity pairs as the training data of a relation,
we use breadth-first search 
to find all suitable skeletons that connect each entity pair in KB.
%3. problem of connection
%Due to various predicates and popular entities existed, a relation 
%instance could be linked through a large number of different skeletons.
%4. what's meaningless rep.
%Most of natural language relations are short phrases, a skeleton 
%with too many predicates is meaningless, and is less likely 
We limit the maximum length of the skeletons to be $\tau$.
%6. why use minimal coverage
The percentage of input relation instances covered by a schema $S$ 
is called the {\em support} of $S$, or $sup(S)$.
To ensure the quality of the retrieved skeleton, we also require a schema
to support at least $\gamma$ percentage of training data, that is,
$sup(S) \ge \gamma$. 
This limit helps the system filter out noisy schemas. 

%Also note that we always focus on well-descriptive skeletons, 
%rather than some occasional paths covering only a few entity pairs.
%7. say in detail
%In formal, we define another threshold $cov$ as the minimum 
%percentage of entity pairs among all positive instances covered 
%by a skeleton.
% Comment: we could describe the searching process in detail, like Matt's style "more formally, blabla..."


%18 sentences: bfs basic, search space limit, budget, diversity
%1. general speak
After candidate skeletons are produced, we deploy depth-first search on
each skeleton to obtain more specific schemas.
At each step of the search, we attempt to add an edge to any one node
on the skeleton, which generates a more specific schema $S'$.
A new edge is accepted if $sup(S') \ge \gamma$.
%the support of the new schema is larger than $\gamma$. 
This process continues recursively until
no new schemas can be found.
%3-6: basic limit on schema constraints
%3. why need limitation
%The searching space is a tree structure which grows exponentially,
%making the exhaustive searching intractable on a huge knowlege base.
%4. how to fix the search size
Each additional edge on a variable node acts like a constraint on the variable.
In practice, multiple constraints on the same variable seldom make sense. 
Therefore, we require that at most one edge is added to any node 
on the skeleton.
%5. the intuition behind
%As mentioned before, natural language relations are always short
%phrases, which gives us the point that it's less likely to infer
%a comfortable structure for a relation with multiple restriction 
%imposed on a single element, and our restriction just follows 
%this intuition.
%6. the effect of limitation
Consequently, the maximal depth of the searching tree is $\tau+1$.

%7-11: budget base (why, budget+prune, criteria, how to prune, diversity)
%7. why need budget
Unfortunately, each node in a skeleton may be attached hundreds of
different predicates in a large KB. The overall search space, though
bounded by the constant depth, is still large.
%8. introduce budget+pruning
Inspired by beam search algorithm\cite{ney1992improvements}, 
we introduce a fixed size 
priority queue to store the set of candidate schemas for each relation.
Our goal is to fill this queue (an operational budget) 
with relatively higher quality
schemas. We simply use the support of each schema on the input instances
as the quality or priority of the schema. The idea is that a better schema
should cover more instances. Also since as we attach more edges to the schema,
the support monotonically decreases, we can use this 
monotonicity property to prune the search
space. At each step of the search, we enumerate all the possible new schemas
(with one new edge) and insert the best among them to the priority queue if the
queue is not full. If the queue is full, we compare the support of the 
schema to be inserted with the worst schema on the queue. 
If the current schema is better than the worst schema on the queue, 
the worst schema is replaced and
the search continued from the current best schema. 
Otherwise, we prune the search space and backtrack.
\KZ{The above description might be hard to understand without a simple
pseudo-code.}

%while pruning strategies will be used to reduce 
%searching space so that poor candidates could be ignored.
%9. what's the criteria
%To this end, we use the number of instances covered by 
%a schema as the criteria to approximately measure its quality.
%%10. explanation of the criteria
%The reason is two-fold: we aim to keep those descriptive schemas in 
%the output candidates, since we output a bunch of schemas instead
%of only a few, we don't need a rather precise quality measurement,
%the idea that better schemas cover more positive instances is
%reasonable enough for our task; 
%and the size of coverages would never increase when the search goes
%deeper, which leads to a simple but effective pruning strategy.
%
%11-15. formal describe
%Now we explain the searching step in formal.
%% [A simple pseudo code is available]
%The beginning state of the searching is one skeleton, we enumerate
%all the constraints which are allowed to add on, each constraint 
%maps to a more specific schema.
%Then new schemas are ranked over their coverages by descending order,
%and we sequentially continue recursive searching on those schemas.
%When the searching state comes to a new schema $s_0$, we keep this 
%schema if there has enough room to keep candidates; 
%otherwise, we pick the schema $s_1$ which has the smallest 
%coverage among all kept schemas and compare their coverage.
%If $s_0$ has a larger coverage, then $s_1$ is discarded, we keep 
%$s_0$ and search deeper; otherwise, the current schema $s_0$ is 
%pruned, and we backtrace the searching process immediately.
%Finally, the output candidates are those schemas been kept when
%the searching is over.

%16-18. diversity
Finally, we would like the generated set of candidate schemas
to be diverse and not all similar to each other.
%The diversity of output schemas plays an important rule in the 
%learning parts.
%If most candidates are the same and only differ from one or two 
%constraints, we are actually wasting budgets because it contains
%much redundant information.
We thus maintain multiple priority queues, one for each skeleton.
The size of each queue is proportional to the support of the corresponding
skeleton.
\KZ{Is there a bound on the number of skeletons generated from the
BFS?}
