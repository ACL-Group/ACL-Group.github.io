\section{Conclusion}
This paper mines the equivalence between natural language relations and structured knowledge known as schemas.
It generalizes the simple path representation by adding constraints along the path and support more complex relations.
Experiments show that schema representation is able to describe the concrete and precise semantic meaning,
it significantly improves the results on knowledge base completion when complex relations are involved,
and performs as well as the state-of-the-art methods on ordinary relations.
%To the best of our knowledge, this is the first attempt to directly model complex NL relations in knowledge bases.
By analyzing paraphrase results in detail, we learned that in this data-driven approach,
the result increases in proportional to the size of input, but it's sensitive to the noisy pairs in relation instances.
In the future of this research, we aim to design a self-adaptive algorithm to detect and filter noisy data
from relation instances, and leverage the structural knowledge to improve performance of QA systems.

%Future direction of this research includes schema inference from
%small data sets, efficient generation of
%more complex schemas, including comparative and aggregative constraints, 
%and the exploration of other features during
%schema weighting.
%
