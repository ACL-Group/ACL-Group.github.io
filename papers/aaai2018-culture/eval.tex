\section{Proposed Tasks and Evaluation}
\label{sec:eval}
In this section, we evaluate our framework on two tasks requiring cross-cultural differences/similarities computation: i) mining cross-cultural differences in named
entities, ii) finding the most similar words for slang terms across languages. Following subsections first discuss some
preliminary setup, then present our experiments for the two tasks.

\subsection{Experiment Setup}
\label{sec:prelim}
Prior to evaluating \textit{SocVec} with our two proposed tasks, we need to first pre-process the social media copora, perform entity linking, then train mono-lingual word embeddings, and finally collect the bilingual lexicon for common words and
social word vocabularies, which contain opinion and sentiment related words. 

\textit{Social Media Corpora}~~
%As the name of our model suggests, microblog corpora are extensively used for both our model and the two tasks (MCDNE and BLEIS).
%One highlight about this dataset is that it contains not only normal microblogs, but also those deleted or censored by force of authority. It enables our model to better look into the true and thorough opinions from netizens in China unhindered by censorship.
The English Twitter corpus is from Archive Team's Twitter stream 
grab\footnote{{\url{https://archive.org/details/twitterstream}}}.
The Sina Weibo corpus comes from Open Weiboscope 
Data Access\footnote{{\url{http://weiboscope.jmsc.hku.hk/datazip/}}}~\cite{fu2013assessing}.
Both corpora cover the whole year of 2012. 
We then downsample each corpus to 100 million messages, each 
containing at least 10 characters, normalize the text~\cite{han2012automatically}, lemmatize the text~\cite{manning2014stanford} and use 
LTP~\cite{che2010ltp} to do Chinese word segmentation.

\textit{Entity Linking and Monolingual Word Vectors}~~
After preprocessing the corpora, we first do entity linking.
%link all named entities to the corresponding concepts in Wikipedia. 
For the Twitter corpus, we use Wikifier~\cite{cheng2013relational,ratinov2011local}, a widely used
entity linker. 
Because no suitable Chinese entity linking tool is available, 
we implement our own tool that is optimized for high precision. 
This tool prefers to link an entity with a surface form that appears
more frequently in our corpus. 
%We utilize context information for selecting entity candidates. 
%That is, only when a surface form is an exact match, or the candidate has been 
%linked in the surrounding text before, or satisfies the occurrence 
%frequency criterion, will it be linked by our tool. Also, we only focus on entities that have both English and Chinese 
%Wikipedia pages.
%We argue that such precision oriented approach is sufficiently good for our tasks, 
%because even if an entity is not recognized, it can still be captured as a normal
%word and contribute to the semantics of other terms.
%Due to the lack of existing state-of-the-art tools for Chinese entity linking and the fact that other tools pursue the balance of precision and recall and tend to generate too many links for our task requiring high precision. Thus we devised our own entity linking method suitable for this particular task.
%
%Basically, we obtain the set of possible surface forms for an entity by collecting all anchor texts of the entity in the
%Wikipedia corpus. In addition, we leverage a redirect system
%of Wikipedia and merge all entities that redirect to each other
%as one entity.
%We can also compute anchor-entity linking frequency
%from the Wikipedia corpus.
%Our entity linking algorithm starts by looking for potential
%anchors in the plain text corpus. 
%We adopt a longest match strategy here that prefers longer anchors. This is because we assume longer anchors are more reliable and bring about higher precision. 
%In Entity linking, we aim for high precision rather than recall because even if an entity is not recognized in the text, its constituent words will still be captured later in the ordinary word vector space and contribute to the semantics of other entities or words.
%After this process we turned our original bilingual corpora into annotated ones with all the Named Entities linked to a shared common entity identifier, the Wikipedia title with both English and Chinese page for a given entity. This bridges an inter-language link for named entities that we would like to mine cultural differences on afterwards.
%
Next we use Word2Vec~\cite{Mikolov2013distributed} to train English and Chinese word embedding respectively.

\textit{Bilingual Lexicon}~~
\label{sec:blpre}
Our bilingual lexicon is collected from Microsoft Translator\footnote{{\url{http://www.bing.com/translator/api/Dictionary/Lookup?from=en&to=zh-CHS&text=<input_word>}}}, which translates English words to multiple
Chinese words with confidence scores. 
\footnote{All named entities and slang terms used
	in the following experiments are excluded from this bilingual lexicon.}


%between Twitter/Weibo frequent words and the  
%In order to bridge the two cross-lingual word vector spaces for using cooperatively, we require a bilingual lexicon that maps the words between two languages.
%Our bilingual lexicon is built in an one-way manner, i.e, from English words to translated Chinese words. 
%We first derive the intersection of the word set from our Twitter corpus with a publicly available word count list\footnote{http://norvig.com/ngrams/count\_1w100k.txt} with 100,000 most popular words, which is our English Tweets common word list with over 20,000 words. 
%Using Bing Translate API\footnote{http://www.bing.com/translator}, we translate those common words into Chinese. As each English word can be translated into multiple Chinese
%words, and each Chinese word can be translated into multiple English words, with translation confidence ranging from 0 - 1 and sums to 1, this phase generates a many-to-many translation mapping, or bilingual lexicon.

\textit{Social Word Vocabulary}~~
%As stated in previous sections, in order to capture the sociolinguistic properties and features in social media texts, we need another set of lexicons which represents the meanings that are crucial to our goal and tasks.
\label{sec:sv}
Our social word vocabularies come from 
Empath~\cite{fast2016empath} and OpinionFinder~\cite{choi2005identifying} 
for English, and TextMind~\cite{gao2013developing} for Chinese.
Empath is similar to LIWC~\cite{pennebaker2001linguistic},
but with more words and more categories and publicly available. 
%but agrees with LIWC on many categories they share. 
%It's based on a combination of word embedding models, 
%knowledge bases and crowdsourcing, wherein 
We manually select 91 categories of words that are 
relevant to human perception and psychological processes. 
OpinionFinder consists of words relevant to opinion and sentiment. 
TextMind is a Chinese counterpart for Empath.
%, which is similar to LIWC for analysis the preferences and degrees of different categories in text with an emphasize on Chinese characteristics.
%All of these three vocabularies are used on a combinatorial basis 
%acting as a parameter to our \textit{SocVec} model.
%
In summary, we obtain 3343 words from Empath,  3861 words from OpinionFinder, 
and 5574 unique words in total. 

\subsection{Mining Cross-cultural Differences of Named Entity}
\label{sec:mcdne}
This task is to discover and quantify cross-cultural differences of concerns towards name entities. 
We first explain how we obtain the ground truth from human annotators, then present several baseline methods to this problem and finally 
show our experiment results in detail.

\subsubsection{Ground Truth}
\label{sec:mcdne_truth}
Harris~\shortcite{harris1954distributional} states that the meaning of 
words is evidenced by the contexts they occur with. 
Likewise, in this work, we assume that the cultural properties of an entity 
can be captured by the terms they co-occur with in large text corpus. 
Thus, for each named entity, we present four human annotators\footnote{All four annotators are native Chinese speakers but bilingual. Two of them lived in the US extensively.} with two lists of 20 most co-occurred words with 
the named entity, from Twitter and Weibo respectively. 
We select 700 named entities for annotators to label, which
are the most frequently mentioned both in Twitter and Weibo. 
Annotators are instructed to rate the relatedness between the 
two word lists with one of following labels: ``very different'', 
``different'', ``hard to say'',  ``similar'' and 
``very similar''.\footnote{Our annotators are educated with many 
selected examples and thus have shared understanding of the five-level 
labels. Ranking based annotation method demands annotators to look 
at 40+40 words for the two terms in two languages before a decision can
be made, which is more expensive and harder to administer in our opinion.}

We then map the labels to numerical scores from 1 to 5
and use the average scores from the annotators as the ground truth 
for score ranking and binary classification.
For the binary classification problem, 
an entity is considered culturally similar 
if the score is larger than 3.0, and culturally different otherwise.
The inter-annotator agreement is 0.672 by Cohen's kappa coefficient, 
suggesting substantial correlation, according to the Wikipedia entry
of Cohen's kappa.
%\BL{ (0.531 without hanyuan)}

\subsubsection{Baselines and Our Method}

We propose five baseline methods. The first three
\emph{distribution}-based, while the next two 
are \emph{transformation}-based. 
Distribution-based methods compare the lists of surrounding
English and Chinese terms, denoted as $L_E$ and $L_C$, 
by computing the cross-lingual relatedness between the two lists, 
though different baselines differ by the
selection of words and the way similarity is computed.
Transformation-based methods compute the vector representation 
in English and Chinese corpus respectively, and
then trains a transformation.
% of words, known $L_E$ and $L_C$. The differences of these methods are the selecting method of terms and %the computation method of two word lists.  ii) the second type of baseline methods are first obtain the %comparable vectorial representation of the English title and Chinese title of the given entity, and then just %calculate the similarity between two comparable vectors.

\textbf{Bilingual Lexicon Jaccard Similarity (BL-JS)}
%	The $L_E$ and $L_C$ of both BL-JS and WN-WUP  are the same as the lists that annotators judge.
BL-JS uses the bilingual lexicon to translate $L_E$  to a Chinese word list 
$L_E^*$ as a medium and then calculates the Jaccard Similarity between 
$L_E^*$ and $L_C$ as $J_{EC}$. Similarly, we can compute $J_{CE}$. 
Finally, we compute $\frac{J_{EC}+J_{CE}}{2}$ as the cross-cultural similarity 
of this given name entity.
 
	\textbf {WordNet Wu-Palmer Similarity (WN-WUP)} Instead of using 
the bilingual lexicon and Jaccard Similarity, WN-WUP uses Open Multilingual 
Wordnet~\cite{wang2013building,bond2013linking} to calculate the average 
similarity of two lists of words from different languages.
	
	\textbf {Word Embedding based Jaccard Similarity (EM-JS)} EM-JS is 
very similar to BL-JS, except that its $L_E$ and $L_C$ are generated by 
ranking the similarities between the name of entities and all English words 
and Chinese words respectively. 

	\textbf {Linear Transformation (LTrans)}
	We follow the steps in Mikolov et al.~\shortcite{Mikolov:2013tp} 
to train a transformation matrix between \textit{EnVec} and \textit{CnVec}, 
using 3000 translation pairs with confidence of 1.0 in the bilingual lexicon. 
Given a named entity, this solution simply calculates cosine similarity 
between the vector of its English name and the \textit{transformed} vector 
of its Chinese name. 
	
	\textbf {Bilingual Lexicon Space (BLex)}
	This baseline is similar to \textit{SocVec} but it does not 
utilize socio-linguistic vocabularies and simply uses the bilingual lexicon
as the BSL.

\textbf{{Our SocVec-based method}} Given a named entity with its English and 
Chinese name, we simply compute the similarity between their 
\textit{SocVec}s as its cross-cultural difference score. 

\subsubsection{Experimental Results}

For qualitative evaluation, \tabref{tab:mcdne_res_4} shows some of 
the most culturally different entities obtained by our method. 
The hot and trending topics on Twitter and Weibo are 
manually summarized to help explain the cultural difference. 
All listed entities have large divergence on concerns, 
thus reflecting cross-cultural differences.
\begin{table*}[th!]
	\footnotesize
	\centering
	\caption{{Selected culturally different named entities, with Twitter and Weibo's trending topics manually summarized}}
	\begin{tabular}{|L{1.5cm}|L{5cm}|L{8cm}|}
		\hline
		\textbf{Entity} & \textbf{Twitter topics} & \textbf{Weibo topics}
		\\ \hline
		Maldives & coup, president Nasheed quit, political crisis & holiday, travel, honeymoon, paradise, beach \\ \hline
		Nagoya & tour, concert, travel, attractive, Osaka & Mayor Takashi Kawamura, Nanjing Massacre, denial of history\\  \hline
%		Quebec & Conservative Party, Liberal Party, politicians, prime minister, power failure & travel, autumn, maples, study abroad, immigration, independence   \\ \hline
%		Philippines & gunman attack, police, quake, tsunami & South China Sea, sovereignty dispute, confrontation, protest  \\ \hline
		Yao Ming & NBA, Chinese, good player, Asian  & patriotism, collective values, Jeremy Lin, Liu Xiang, Chinese Law maker, gold medal superstar   \\ \hline
		University of Southern California & college football, baseball, Stanford, Alabama, win, lose & top study abroad destination, Chinese student murdered, scholars, economics, Sino American politics \\ \hline
	\end{tabular}
	\label{tab:mcdne_res_4}
\end{table*}

\begin{table}[th]
	\small
	\centering
	\caption{{Comparison of Different Methods}}
	\begin{tabular}{|l|c|c|c|}
		\hline
		\textbf{Method} & \textbf{Spearman} & \textbf{Pearson}  & \textbf{MAP} \\ \hline\hline
		BL-JS& 0.276 & 0.265 & 0.644   \\ \hline
		WN-WUP  & 0.335 & 0.349 & 0.677 \\ \hline
		EM-JS & 0.221 & 0.210  & 0.571\\ \hline
		LTrans& 0.366 & 0.385  & 0.644  \\ \hline
		BLex& 0.596 & 0.595  & 0.765 \\ \hline\hline
		SocVec:opn& 0.668 & 0.662   & \textbf{0.834} \\ \hline
		SocVec:all& \textbf{0.676} & \textbf{0.671}  & \textbf{0.834}\\ \hline
	\end{tabular}
	\label{tab:mcdne_res_1}
\end{table}
\begin{table}[th]
	\centering
	\small
	\caption{{Evaluation of Different Similarity Functions}}
	\label{tab:mcdne_res_2}
	\begin{tabular}{|l|c|c|c|}
		\hline
		\textbf{Similarity} & \textbf{Spearman} & \textbf{Pearson}   & \textbf{MAP} \\ \hline\hline
		PCorr. & 0.631 & 0.625 & 0.806\\ \hline
		L1 + M & 0.666 & 0.656 & 0.824 \\  \hline
		Cos & \textbf{0.676} & 0.669 & \textbf{0.834} \\ \hline
		L2 + E & \textbf{0.676} & \textbf{0.671} & \textbf{0.834} \\ \hline
	\end{tabular}
\end{table}

\begin{table}[th]
	\centering
	\small
	\caption{{Evaluation of Different Pseudo-word Generators}}
	\begin{tabular}{|l|c|c|c|}
		\hline
		\textbf{Generator} & \textbf{Spearman} & \textbf{Pearson}   & \textbf{MAP} \\ \hline \hline
		Max. & 0.413 & 0.401 & 0.726\\ \hline
		Avg. & 0.667 & 0.625 & 0.831\\ \hline
		W.Avg. & 0.671 & 0.660 & 0.832 \\  \hline
		Top & \textbf{0.676} & \textbf{0.671} & \textbf{0.834} \\ \hline
	\end{tabular}
	\label{tab:mcdne_res_3}
\end{table}

In~\tabref{tab:mcdne_res_1}, we evaluate the baseline methods and 
our approach with three metrics: Spearman and 
Pearson correlation on the ranking problem, and Mean Average Precision (MAP)
on the classification problem (see \secref{sec:mcdne_truth}). The \textit{BSL} of \textit{SocVec:opn} uses only OpinionFinder as English socio-linguistic vocabulary, while \textit{SocVec:all} uses the union of Emapth and OpinionFinder vocabularies.\footnote{
Having tuned the  parameters, we use the best parameters for the \textit{SocVec:opn} method and \textit{SocVec:all} method: 
monolingual word vectors are trained with 5-word context window and 150 dimensions;
choosing cosine similarity as the \textit{sim} function to compute the similarity within the \textit{\socvec}~space;
using ``\textit{Top}'' pseudo-word generator
(see \secref{sec:model}).
} Results show that \textit{SocVec} models perform the best and 
using the union of vocabularies is better.

We also evaluate the effect of four different similarity options in 
\textit{\socvec}, namely, Pearson Correlation Coefficient 
(\textit{PCorr}.), L1-normalized Manhattan distance (\textit{L1+M}), 
Cosine Similarity (\textit{Cos}) and  L2-normalized Euclidean distance (\textit{L2+E}).
%It is mathematically proved that \textit{L2+E} is identical to \textit{Cosine} in ranking.
From~\tabref{tab:mcdne_res_2}, we conclude that among these four options, \textit{Cos} and \textit{L2+E} perform the best. 
%Although it is mathematically proved that \textit{L2+E} is identical to \textit{Cosine} in ranking, we can find that 
\tabref{tab:mcdne_res_3} shows effect of using four different 
pseudo-word generator functions (see~\secref{sec:pg}), from which we can infer that ``\textit{Top}'' generator function performs best for 
it reduces the noise brought by the less possible translation pairs. 



\section{Task 2: Finding most similar words for slang terms across languages}
\label{sec:bleis}
%In this section, we evaluate our model on the second task, which   
%In this section, we first introduce the ground truth and baseline methods for comparison. Then, we analyze the experimental results quantitatively and qualitatively.
\textbf{Task Description:} This task aims to find the most similar English words in terms of meaning and sentiment of a given Chinese slang term, or vice versa. The input is a list of English (or Chinese) slang terms of interest and two monolingual social media corpora; the output is a list of Chinese (or English) word sets with respect to each input slang term.
Simply put, for each given slang term, we aim to find a set of the words (in the other language) that are most similar to it and thus can help people understand it across languages.
We propose using Average Cosine Similarity (see~\secref{sec:exp}) to evaluate each method's output with ground truth word sets (presented below).

\subsection{Ground Truth}
\textbf{Slang Terms}
We collect our Chinese slang terms from an online Chinese slang glossary\footnote{\scriptsize{\url{https://www.chinasmack.com/glossary}}} consisting of 200 popular slang terms with English explanations. 
For English, we resort to a slang word list from OnlineSlangDictionary\footnote{\scriptsize{\url{http://onlineslangdictionary.com/word-list/}}} with explanations and downsample the list to 200 terms.
%To evaluate the performance of our model, 
%we propose to build the ground truth based on above-mentioned explanations from the glossary and slang dictionary. 
%Since the subtle and latent semantics of slang are too difficult to translate without losing any information, exact translation are always missing.

\noindent
\textbf{Truth Word Sets}
For each Chinese slang term, its truth English word set is hand picked from the English explanation in the glossary. 
For example, we construct the ground truth target terms for 
the Chinese slang term ``二百五'' by manually labeling words related to its meaning in the glossary:
~\\ \vspace{-5pt}
\begin{description}
	\item[二百五:] A \textbf{\textit{foolish}} person who is lacking in sense but still \textbf{\textit{stubborn}}, \textbf{\textit{rude}}, and \textbf{\textit{impetuous}}.
\end{description}

\noindent
Similarly, for each English slang term, its  Chinese word sets are the translation of the words hand picked
from its English explanation.
%Different methods should produce a list of translation terms 
%as similar as possible to the ground truth target terms.

%and then computing the average similarity between the source term and the target terms is a better approach to evaluate a bilingual slang lexicon induction system.



%what are the good word translation that best preserve and convey the meaning, sentiment tendency and usage context ot the original slang. 
%However most of the slangs possess most subtle senses that are related to native culture background, general characteristic and language style of netizens in different language worlds. 
%Thus it is really difficult, if not impossible, to find a exact word in another language that carries the exact same meaning with the original slang. 
%Because of this, our ground truth does not pursue exact Internet slang translation from one language to another's corresponding slang, since most likely such slang does not exist yet. 
%Our aim for the task is using IV (in-vocabulary) normal related words in target language to describe and translate the OOV (out-of-vocabulary) slang words in another language, which is more viable and feasible, and easier for people in different culture/language to understand not just literal meaning but the deeper buried and more subtle sense and context it conveys.
%
%Following this goal, we could build our ground truth based on filtered glossary. For Chinese slangs, we tokenize and lemmatize the definition sentences in English and manually remove the stop words that does not contribute to the meaning of a specific slang, left with a list of English words that have either same meaning or high relatedness to the Chinese slang.
%The same process is applied to English slang glossary as well, with the only difference is that due to the lack of direct Chinese definitions, we have to use Google Translate to translate the definition sentences to Chinese and human annotators tokenize, filter and paraphrase the definitions into Chinese word lists, resulting in the ground truth of slangs in the same format as the Chinese ones.
%Note that we do not manually add word translations by ourselves, we only delete irrelevant words for later evaluation, thus lowering the human error, cross-cultural and bilingual requirement to the minimum.

\begin{table*}[th!]
	\scriptsize
	\centering
	\caption{\small Slang Translation Examples \vspace{-10pt}}
	\begin{tabular}{L{1.2cm}|L{4.7cm}|L{1.7cm}|L{1.7cm}|L{1.7cm}|L{2.8cm}}
		\textbf{Slang} & \textbf{Explanation} & \textbf{Google}& \textbf{Bing}& \textbf{Baidu} & \textbf{Ours} \\ \hline 
		浮云 &something as ephemeral and unimportant as ``passing clouds''& clouds& nothing& floating clouds & nothingness, illusion \\ \hline
		水军 &``water army'', people paid to slander competitors on the Internet and to help shape public opinion& Water army& Navy& Navy & propaganda, complicit, fraudulent\\ \hline
		%		城管 & ``City administrators'', who enforce city regulations, with poor reputation as being corrupt and violent, best known for physically bullying illegal street peddlers & urban management& urban management& urban management & terrorist, rioting, threaten\\ \hline \hline
		floozy & a woman with a reputation for promiscuity & N/A&劣根性 (depravity)&荡妇(slut)&骚货(slut),妖精(promiscuous)\\ \hline
		fruitcake& a crazy person, someone who is completely insane & 水果蛋糕 \quad(fruit cake)&水果蛋糕 \qquad(fruit cake)&水果蛋糕 \quad(fruit cake)& 怪诞(bizarre),厌烦(annoying)\\ \hline
		%		nonce &  A person convicted (or simply guilty) of sexual crimes, especially pedophilia. Or a common British insult regardless of the tendencies of the person &随机数 (random numbers)&杜撰 (fabricate)&杜撰 (fabricate) & 伤风败俗(immoral),十恶不赦(extremely evil),畜类(beast),令人发指(heinous)\\ \hline
	\end{tabular}
	\label{tab:bleis_3}
\end{table*}
\subsection{Baseline and Our Methods} 
We propose two types of baseline methods for this task. 
The first type is based on well-known {\em on-line translators}, 
namely Google (Gg), 
Bing (Bi) and Baidu (Bd).\footnote{Experiments using them are done in August, 2017.}  
%With our test set's slang as input, we retrieve the output of translation. 
Another baseline method for Chinese is  CC-CEDICT\footnote{\scriptsize{\url{https://cc-cedict.org/wiki/}}} (CC), an on-line public-domain Chinese-English dictionary, which is constantly updated with popular slang terms. 

Considering situations where many slang terms have literal meaning as well, it may be unfair to retrieve target terms from such on-line translators by solely inputing slang terms without slang contexts. 
Thus, we utilize example slang-meaning sentences from some websites (mainly from Urban 
Dictionary\footnote{\scriptsize{\url{http://www.urbandictionary.com/}}}) 
%as input to them, so that the translators have a greater chance of knowing this is a slang use, rather than an ordinary term. \footnote{Nevertheless, we noticed
%that on-line translators often cannot capture such slang contexts and still produce literal translations.}
The following example using Google Translation shows how we obtain the target translation terms for the slang word ``fruitcake'' (an insane person) from Google Translator:
{\textit{Oh man, you don't want to date that girl. She's always drunk and yelling. She is a total \underline{\textbf{fruitcake}}.}}\footnote{\scriptsize{\url{http://www.englishbaby.com/lessons/4349/slang/fruitcake}}} 
\textit{Google Translation:}
{\small哦, 男人, 你不想约会那个女孩。她总是喝醉了, 大喊大叫。她是一个\underline{\textbf{水果蛋糕}}。}

%Since all possible target terms come from the bilingual lexicon, 
Another lines of baseline methods is ranking based.
We can score each of them in our bilingual lexicon and consider the top K words as the target terms. 
Given a source term to be translated, several such {\em scoring-based baseline methods} are as follows.
Linear Transform (LT), MultiCCA, MultiCluster and Duong method score the candidate target terms by 
computing cosine similarities in their constructed bilingual vector space with the tuned best settings in previous evaluation. 
A more sophisticated baseline (TransBL) leverages the bilingual lexicon: 
for each candidate target term $w$, we first obtain its translations 
$T_w$ back into the source language and then calculate the average word similarities between the source term and $T_w$  as the score of $w$. 
%Now over 20,000 words in the other language of the bilingual lexicon have their corresponding similarity score to the given slang. 
%A word may have multiple possible translation words in the other language. In this case, we choose to take average over all of them in terms of similarity score.
%We then rank the words by their scores and take top 5 words to form a word set, while other online translation baselines directly produce a word set for later comparison with the ground truth word set. 
Our {\em SocVec-based method} (\textbf{SV}) simply calculates the cosine similarities between the source term and each candidate target term within \textit{\socvec} space as scores.


\begin{table}[th!] 
	\scriptsize
	\centering
	\begin{subtable}[h]{\columnwidth}
		\centering
		\begin{tabular}{|ccccc|}
			\hline
			Gg&  Bi& Bd & CC & LT   \\ 
			18.24 &  16.38&  17.11 & 17.38 & 9.14 \\ \hline   
			TransBL& MultiCCA & MultiCluster & Duong   & SV \\ 
			18.13 &  17.29 & 17.47&  20.92& \textbf{23.01}\\ \hline  
		\end{tabular}
		\subcaption{Chinese Slang to English}
	\end{subtable}
	\vfill \hfill
	\begin{subtable}[h]{\columnwidth}
		\centering
		\begin{tabular}{|ccccc|}
			\hline
			Gg&  Bi& Bd &  LT   & TransBL\\ 
			6.40  &   15.96 &  15.44  & 7.32 & 11.43\\ \hline   
			MultiCCA & MultiCluster & Duong   & SV & \\ 
			15.29 & 14.97&  15.13& \textbf{17.31} & \\ \hline  
		\end{tabular}
		\subcaption{English Slang to Chinese}
	\end{subtable}
	\caption{\small ACS Sum Results of Slang Translation \vspace{-5pt}}
	\label{tab:bleis_acs}
\end{table}
\subsection{Experimental Results}
\label{sec:exp}
To quantitatively evaluate our methods, we need to measure similarities between the produced target term set and the ground truth word set. 
Exact-matching Jaccard similarity is too strict to capture valuable relatedness between two word sets.
We argue that average cosine similarity (ACS) between two sets of word vectors is a better metric to evaluate the similarity between two word sets. The following equation illustrates such computation, where $A$ and $B$ are the two word sets, $\mathbf{A_i}$ and $\mathbf{B_j}$ denotes the word vector of the $i^{th}$ word in $A$ and $j^{th}$ word in $B$ respectively. 
The word vectors used in ACS computation is a third-party pre-trained 
embedding\footnote{\scriptsize \url{https://nlp.stanford.edu/projects/glove/}} and thus the ACS computation is fair over different methods.
Table 6 shows the sums of ACS over 200 slang translations. 
\begin{align*}
ACS (A,B)=
{\frac{1}{|A||B|}}{\sum_{i=1}^{|A|}{\sum_{j=1}^{|B|}} \frac{\mathbf{A_i }\cdot \mathbf{B_j}}{\|\mathbf{A_i }\|\|\mathbf{B_j }\|}}
\end{align*}


Experimental results of Chinese and English slang translation in terms of the sum of \textit{ACS} over 200 terms are shown in~\tabref{tab:bleis_acs}.
The performance of on-line translators for slang typically depends on human-set rules and supervised learning on well-annotated parallel corpora, which are rare and costly, especially for social media where slang emerges the most. This could be a possible reason why they do not perform well. 
Linear transformation model is trained on translation pairs with high confidence in the bilingual lexicon, which contains little information about the OOV slang terms and social context on them, which is why LT method performs badly.
\textit{BL} method is competitive because its similarity computations 
are within monolingual semantic spaces and it uses a bilingual lexicon 
to transform, while it loses the information from the related words 
which are not in the bilingual lexicon.
%Experiment results of Chinese and English slang translation in terms of the sum of \textit{ACS} over the translations of 200 slang terms are shown in~\tabref{tab:bleis_acs}.
%The performance of online translators for slang terms typically depends on human-set rules and supervised learning on well-annotated parallel corpora, which are rare and costly, especially for social media where Internet slang emerges the most. 
%This could be a possible reason why they do not perform well. 
%~Linear transformation model is trained on translation pairs with high confidence in the bilingual lexicon, which contains little information about the OOV slang terms and social context on them, which is the reason why  LT method performs badly.
%\textit{BL} method is competitive for its similarity computations are within monolingual word vector spaces and uses a bilingual lexicon to transform, while it loses the information from the related words which are not in the lexicon translation pairs.
Our method (SV) outperforms baselines by directly using the distances in 
our proposed bilingual embeddings~\textit{SocVec}, which proves 
that ~\textit{SocVec} can capture the cross-cultural similarities between terms.
%utilizes comparable English and Chinese social media corpora and 
%encodes the context and usage of a given slang term by computing its similarities with words in the socio-linguistic vocabulary of the source language. Therefore, 
%our model keeps the cross-cultural socio-linguistic features, which is a most important reason why we outperform baselines.
%the best among all the baseline methods. 
%Then, we are able to find the most similar counterparts in the target language by computing the similarity in \textit{SocVec} space through \textit{BSL}. 
%Therefore, our performance is better than the others.    

To qualitatively evaluate our model, in~\tabref{tab:bleis_3}, 
we present several examples of our translations for Chinese and English slang 
terms as well as their explanations from glossaries.
Our results are highly correlated with these explanations and 
capture their core semantics, whereas most online translators just offer 
literal translatation of such slang terms, even with the ample
slang contexts.
% They often offer just literal meanings as translation even with the specific slang context using the example sentences from Urban Dictionary.

Additionally, we take a step forward to directly translate between 
English slang terms and Chinese slang terms by simply filtering out 
ordinary (non-slang) words in the original target term lists. 
Examples are shown in~\tabref{tab:bleis_4}. 
\begin{table}[t]
	\scriptsize
	\centering
	\caption{\small{Slang-to-Slang Translation Examples}\vspace{-10pt}}
	\begin{tabular}{C{1.92cm} C{2.5cm} C{2.0cm}}
		\textbf{Chinese Slang} & \textbf{English Slang} & \textbf{Explanation} \\ \hline
		萌 & adorbz, adorb, adorbs, tweeny, attractiveee & cute, adorable \\ \hline
		二百五 & shithead, stupidit, douchbag & A foolish person\\ \hline
		鸭梨 & antsy, stressy, fidgety, grouchy, badmood & stress, pressure, burden \\ \hline
	\end{tabular}
	\label{tab:bleis_4}
\end{table}

