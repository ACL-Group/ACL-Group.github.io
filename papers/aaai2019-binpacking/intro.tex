\section{Introduction}
\label{sec:intro}

With the vigorous development of e-commerce, shipping and handling costs are
a critical part of the business equation, and the estimated expenditure on
logistics amounts up to about 15\% of China's GDP. 
Reducing the cost of logistics is often the top priority for large businesses. 
There are many ways to help cut down logistic costs, and one of them is reducing
the packing costs. %Methods of reducing logistic costs can range from optimizing inventory levels, to re-charting better shipping networks, to creating better processes, to improving supplier/third party relationships and so on.
Tens of millions of packages are sent to customers every day, 5\% of the which are prepared using plastic wrappers. In order to reduce the cost of plastic packing materials, the companies prefer to pack the items in a way that minimize the wrapping surface area of the stack. 
In this paper, we formalize this real-world scenario into a specific variant 
of the classical three-dimensional bin packing problem (3D-BPP). 
We call this variant 3D flexible bin packing problem (3D-FBPP), which to seek 
the best way of packing a given set of cuboid-shaped items into a variable rectangular bin \KZ{Why variable??} in such a way that the surface area of the bin 
is minimized. 

The 3D-BPP is a classical combinatorial optimization problem which can directly model many industrial applications in logistic and production fields, e.g., 
container and pallet loading, cargo and warehouse management. %During last decades, both exact and heuristic algorithms for this problem have been extensively studied.
It seeks to pack a number of cuboid-shaped items  into a set of homogeneous or heterogeneous bins so that the number of bins used is minimized, or equivalently, the bins’ utilization ratio is maximized. As a strongly NP-hard problem \cite{martello2000three}, the 3D-BPP and its variations have led to extensive studies in the field of operational research (OR) during last decades. While exact algorithm provides optimal solution, it usually needs huge amount of time to solve modern size instances. Heuristic algorithm, which cannot guarantee optimal, is capable to give acceptable solutions with much less computational effort. 
%Exact methods provide the optimal solution but time consuming, whereas the heuristic methods run in limited time and return an acceptable solution. 

%To design an efficient and effective algorithm for 3D-BPP could potentially brings significant operation cost reduction in many real-world applications.

%We formalize this scenario into a new variation of 3D BPP and combine a learning based method with heuristic algorithms to efficiently and effectively solve the problem. 

%so it is a very popular research direction in optimization area. In addition, BPPs have numerous relevant industrial applications. An effective bin packing algorithm means the reduction of computation time, total packing cost and increase in utilization of resources.

%The cost of packing materials is mainly determined by their surface area and this occupies the most part of packing cost. In many real business scenarios such as e-commerce, bins with variable sizes, e.g., PE bag which is made of flexible and soft packing materials, are used to pack items. In this case, our research is engaged in a new type of 3D BPP, of which the objective is to pack all items into a bin with least surface area. 

\KZ{This para is too generic and doesn't serve much purpose here.}
Typically, to achieve good results while maintaining computational 
practicality, traditional exact algorithms and heuristic methods require 
extensive expertise in designing specific search strategies for different 
types of problems. In recent years, %methodologies in machine learning (ML) for decision problems that are typically addressed by OR are mainly found in the areas of reinforcement learning (RL). 
there has been some seminal work on using deep architectures to learn heuristics for combinatorial problems, e.g, Travel Salesman Problem (TSP) \cite{bello2016neural,vinyals2015pointer,kool2018attention}, Vehicle Routing Problem (VRP) \cite{nazari2018deep}. These advances justify the renewed interest in the application of machine learning (ML) to optimization problems. Motivated by these advances, this work uses reinforcement learning (RL) and supervised learning (SL) to parameterize the policy to obtain a stronger heuristic algorithm.

%Combinational optimization problems are usually aimed to minimize or maximize an objective function under a given set of constraints. To transfer the concept to learning based algorithm, the objective function can be set as the reward of RL framework. 
\KZ{In this para, you want to highlight the essential difference between 
3D-BPP and 3D-FBPP. I don't get the difference clearly here. Then you need to 
say why the current solution for 3D-BPP is insufficient for 3D-FBPP.}
In the 3D-FBPP, the smaller objective (i.e. the surface area that can pack all the items of a order) is better, implying a better packing solution. To minimize the objective, a natural way is to decompose the problem into three interdependent decisions \cite{gonccalves2013biased,li2014genetic}: 1) decide the sequence to place these items; 2) decide the place orientation of the selected item; 3) and decide the spatial location. These three decisions can be considered as learning tasks, however, in our proposed method, we concentrated on the first two tasks. Regardless of spatial location, the sequence of packing the items into the bin will influence the orientations of each item and vice versa, so the two tasks are correlated. Meanwhile, with $n$ items, the choice of orientations is $6^n$ and the choice of sequence is $n!$, so the two tasks are unbalanced. Inspired by multi-task learning, %we overcome the limitations of simply outputting the placement sequence of items by adopting a new type of training mode named
we adopt a new type of training mode named 
Multi-task Selected Learning (MTSL) to mitigate the correlation and imbalance mentioned above. MTSL is under the following settings: each subtask and both of them are treated as a training task. In MTSL, we select one kind of the training task at once according to a probability distribution which will decay after several training step. Moreover, through keeping track of the best solution sampled during the search, we find that combining MTSL with sampling at inference time works best in practice.

In this paper, we present a heuristic-like policy learned by a neural model 
and quantitative experiments are designed and conducted to demonstrate 
the effectiveness of this policy. The contributions of this paper are 
summarized below.
\begin{itemize}	
	\item This is a first and successful attempt to define and solve 
	the real-world problem of 3D flexible Bin Packing 
	(Section \ref{sec:problem}).  We collected and will open source a 
	large-scale real-world order dataset (LROD) (Section \ref{sec:data}).
%	\item We use an intra-attention mechanism to tackle the sequential decision problems during the optimization, which considers items that have already been generated by the decoder. \KZ{THis is not refutable.}  
	\item We propose a multi-task framework based on Selected Learning, which can significantly mitigate the correlation and imbalance among training tasks. Based on this framework, packing sequence and orientations can be conducted at the same time. 
	\item By sampling at inference time, we achieves $6.21\%$, $7.80\%$, $8.55\%$ improvement than the greedy heuristic algorithm designed for the 3D-FBPP in BIN8, BIN10 and BIN12.
\end{itemize}

%The rest of the paper is organized as follows:
%\secref{sec:problem} provides the formal definition of new type 3D bin packing problem in our study.
%\secref{sec:data} introduces how we collect our L3DBPD dataset.
%\secref{sec:model} describes the proposed multi-task selected learning architecture.
%Then \secref{sec:implementation} gives the implementation details of our model for reproducibility.
%\secref{sec:eval} presents the experimental results and analysis on L3DBPD data.
%We conduct the detailed comparison with baseline models %\lu{baseline models contains rl ?} 
%including a well-designed algorithm NBPH and a state of art method BRKGA in \cite{gonccalves2013biased} which tackles the fixed-sized 3D BPP.
%\secref{sec:related} provides a literature survey of most related previous work.
%Finally \secref{sec:conclusion} concludes and points out some future research directions.
