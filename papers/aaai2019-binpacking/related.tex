\section{Related Work}
\label{sec:related}

\KZ{can cut this section down to make some space.}
Our work applies the latest Deep Learning technology to solve the 3D-FBPP. In this section, we will go through the main research result of 3D-FBPP and machine learning method in combinatorial optimization.

\subsection{3D bin packing problem}

Bin packing problem is a classical and popular optimization problem and it has attracted great interest of many researchers and some valuable achievements have been obtained. As a strongly NP-hard problem, a lot of research focuses on exact methods \cite{chen1995analytical}, approximation methods \cite{scheithauer1991three} and heuristic methods ( genetic algorithm \cite{gonccalves2013biased},guided local search \cite{faroe2003guided}, extreme point-based heuristics \cite{crainic2008extreme})for 3D BPP. Some variants of BPP from real world are also studied, such as strip packing problem (SPP). In SPP, a given set of cuboid-shaped items should be packed into a given strip orthogonally by minimizing the height of packing. The length and width of the strip is fixed and limited, and the height is infinite (for 2D SPP, the width of strip is fixed and the length is infinite). Different types of algorithms have been proposed to solve the problem, such as exact algorithms in \cite{kenmochi2009exact}, approximation algorithm in \cite{steinberg1997strip}, heuristic algorithm in \cite{bortfeldt2007heuristic} and meta-heuristic algorithms in \cite{bortfeldt2006genetic} and \cite{hopper2001review}.

Our work focus on the 3D-FBPP and its objective is to minimize the surface area of the bin that could place all the items, while the 3D-BPP aims to minimize the number of used bin. As proved in section \ref{sec:problem}, this is a NP-hard problem. Previous methods for solving typical 3D-BPP could not achieve persuasive performance (BRKGA) and we used a new well-designed greedy algorithm (LWSC). We believe that using deep architectures to learn heuristics can obtain a more powerful algorithm.

\subsection{Machine Learning in combinatorial optimization}

Even though machine learning and combinatorial optimization have been studied for decades respectively, there are few investigations about application of machine learning method in combinatorial optimization problems. One research direction is designing hyper-heuristics based on reinforcement learning ideas. An overview of hyper-heuristics is presented in \cite{burke2013hyper}, in which some hyper-heuristics based on learning mechanism are discussed. %In \cite{nareyek2003choosing}, the heuristics selection probability is updated based on non-stationary RL. 
%In addition, various score updating methods have been proposed in the area of hyper-heuristics, such as binary exponential back-off \cite{remde2009binary} and choice function \cite{cowling2000hyperheuristic}.

%Some hyper-heuristics in ~\cite{sutton1998reinforcement} use RL ideas to guide the choice of the heuristics during the search. In ~\cite{nareyek2003choosing}, the heuristics selection probability is updated based on non-stationary RL. To deal with the problem that evolutionary algorithm often neither offers worst-case bounds nor any guarantee of optimality when used to solve individual problems, ~\cite{ross2002hyper} tried to learn a solution process, in which one of various simple non-evolutionary heuristics is chosen to apply to each state of a problem, gradually transforming the problem from its initial state to a solved state. %
%Attention networks are now a standard part of the deep learning toolkit, contributing to impressive results in neural machine translation (Bahdanau et al., 2015; Luong et al., 2015), image captioning (Xu et al., 2015), speech recognition (Chorowski et al., 2015; Chan et al., 2015), question answering (Hermann et al., 2015; Sukhbaatar et al., 2015), abstractive summarization(\cite{paulus2017deep}),and algorithm-learning (Graves et al., 2014; Vinyals et al., 2015), among many other applications (see Cho et al. (2015) for a comprehensive review).%

Recent advances in sequence-to-sequence model \cite{sutskever2014sequence} have motivated the research about neural combinatorial optimization. Attention mechanism, which is used to augment neural networks, contributes a lot in areas such as machine translation (\cite{bahdanau2014neural}) and abstractive summarization \cite{paulus2017deep}. In \cite{paulus2017deep}, intra attention mechanism is proposed to address the repeating phrase problem in abstractive summarization. In \cite{vinyals2015pointer}, a neural network with a specific attention mechanism named Pointer Net was proposed and a supervised learning method is applied to solve the TSP. \cite{bello2016neural} develops a neural combinatorial optimization framework with RL, and some classical problems, such as TSP and Knapsack Problem are solved in this framework. %Besides, \cite{Hu2017Solving} proposes a DRL framework to solve new 3D Bin packing Problem based on effectiveness and generality of the methodology proposed in \cite{bello2016neural}. However,our research is an extension of the previous method in \cite{Hu2017Solving}.


