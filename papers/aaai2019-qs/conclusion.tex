\section{Conclusion and Future Work}
\label{sec:conclusion}

In this paper, we take advantage of external documents to help the query segmentation task. Specifically, for each character, we use its left and right bi-grams to find contexts in external documents. Then we extract character and distance features from these contexts. We propose a attention network to encode contexts of a character and get a vector representation. This vector is added to the common BiLSTM-CRF model to predict labels. Our BiLSTM-CRF(Q+C) achieves 0.049 and 0.023 improvements in F1 value compared with exsiting approaches on dress and bag datasets repectively. These results show that contexts in external documents do contain some boundary information which is valuable to query segmentation task.

The key point of our approach is the contexts from external documents. The quality and quantity of contexts will impact the perfomance of this approach directly. To promise these two requirements of contexts, we argue that the documents should share a similar topic with queries and the number of documents should be enough. In addition to contexts itself, the method to encode them is also important. We currently use the attention mechanism to integrate contexts. In the future, we will explore some new methods to make better use of contexts.
