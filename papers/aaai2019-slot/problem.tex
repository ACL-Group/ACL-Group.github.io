\section{Problem}
\label{sec:problem}
Given an utterance containing a sequence of
words\footnote{We use ``word'' in problem and model description,
	but ``word'' actually means Chinese char in our problem.
And a ``term'' consists of one or several words.} 
$\textbf{w} = (w_1, w_2, ..., w_T)$,
the goal of our problem
is to find a sequence of slot labels $\hat{\textbf{y}} = (y_1, y_2, ..., y_T)$, 
one for each word in the utterance, such that:
\begin{equation*}
	\hat{\textbf{y}} = \mathop{\arg\max}_{\textbf{y}}P(\textbf{y}|\textbf{w}).
\end{equation*}

In this paper we only define a slot filling problem in Dress category domain 
for simplicity.
That means we know in advance the defined category classification (intent).
%Situation will get extremely complicated when we cannot get category classification and do slot filling from scratch.
There are thousands of category classifications in E-commerce domain
and in each category, there can be dozens of properties that 
are totally different.  Performing slot filling with tens of thousands 
slot labels across domains is impractical at this point.
A joint model for category classification and slot filling is 
a future research. 

Including Property Key (\textbf{PK}), Category (\textbf{CG}) and \textbf{O},
there are altogether 29 (57 in the IOB scheme) slot labels in our problem. 
Examples are listed in \tabref{slot-labels}.
Notice that terms such as ``brand'', ``color''
appearing in an utterance
will be labeled as \textbf{PK},
while \textbf{Color} and \textbf{Brand} can be pre-defined slot labels
and will be assigned to terms like
``black'' and ``Nike''.
\begin{table}[htbp]
	\centering
	\scriptsize
	\caption{Slot labels defined in E-commerce Knowledge Base.}
	\begin{tabular}{c|c|c|c}
		\toprule
		Named Entity & PV & PK & CG \\
		\midrule
		Slot Label & \textbf{Color}, \textbf{Brand}, ... & \textbf{PK} & \textbf{CG} \\
		\midrule
		Example Term & black, Nike, ... & brand, color, ... & dress, t-shirt, ... \\
		\bottomrule
	\end{tabular}
	\label{slot-labels}
	\vspace{-10pt}
\end{table}
%Compared to the most well-known dialog state tracking challenge (DSTC) dataset,
%in which there are only 9 slot labels in DSTC1\footnote{\url{http://research.microsoft.com/en-us/events/dstc/}} for bus booking scenario
%and 8 slot labels in DSTC2\footnote{\url{http://camdial.org/~mh521/dstc/}} for restaurant booking scenario,
%the dataset in our problem has larger amount of slot labels.
%Therefore, we call our problem as \emph{Large Scaled Slot Filling} problem.


