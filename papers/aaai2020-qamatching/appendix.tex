\def\year{2020}\relax
%File: formatting-instruction.tex
\documentclass[letterpaper]{article} % DO NOT CHANGE THIS
\usepackage{aaai20}  % DO NOT CHANGE THIS
\usepackage{times}  % DO NOT CHANGE THIS
\usepackage{helvet} % DO NOT CHANGE THIS
\usepackage{courier}  % DO NOT CHANGE THIS
\usepackage[hyphens]{url}  % DO NOT CHANGE THIS
\usepackage{graphicx} % DO NOT CHANGE THIS
\urlstyle{rm} % DO NOT CHANGE THIS
\def\UrlFont{\rm}  % DO NOT CHANGE THIS
\usepackage{graphicx}  % DO NOT CHANGE THIS
\frenchspacing  % DO NOT CHANGE THIS
\setlength{\pdfpagewidth}{8.5in}  % DO NOT CHANGE THIS
\setlength{\pdfpageheight}{11in}  % DO NOT CHANGE THIS
%\nocopyright
%PDF Info Is REQUIRED.
% For /Author, add all authors within the parentheses, separated by commas. No accents or commands.
% For /Title, add Title in Mixed Case. No accents or commands. Retain the parentheses.
\pdfinfo{
	/Title (AAAI Press Formatting Instructions for Authors Using LaTeX -- A Guide)
	/Author (AAAI Press Staff, Pater Patel Schneider, Sunil Issar, J. Scott Penberthy, George Ferguson, Hans Guesgen)
} %Leave this	
% /Title ()
% Put your actual complete title (no codes, scripts, shortcuts, or LaTeX commands) within the parentheses in mixed case
% Leave the space between \Title and the beginning parenthesis alone
% /Author ()
% Put your actual complete list of authors (no codes, scripts, shortcuts, or LaTeX commands) within the parentheses in mixed case. 
% Each author should be only by a comma. If the name contains accents, remove them. If there are any LaTeX commands, 
% remove them. 

% DISALLOWED PACKAGES

\usepackage{array}
\usepackage{soul}
\usepackage{color}
\usepackage[utf8]{inputenc}
\usepackage{amsmath}
\newcommand{\figref}[1]{Figure \ref{#1}}
\newcommand{\eqnref}[1]{Eq. \ref{#1}}
\newcommand{\tabref}[1]{Table \ref{#1}}
\newcommand{\secref}[1]{Section \ref{#1}}
\newcommand{\algoref}[1]{Algorithm \ref{#1}}
\renewcommand\appendix{\setcounter{secnumdepth}{-2}}
\newcommand{\KZ}[1]{\textcolor{red}{Kenny: #1}}
\newcommand{\mx}[1]{\textcolor{green}{Mengxue: #1}}
% the following package is optional:
\usepackage{diagbox}
\usepackage{multirow}
\usepackage{amsthm}
\usepackage{mathtools}
\usepackage{hhline}
\usepackage{booktabs}
\usepackage{makecell}
\usepackage{amssymb}
\usepackage{makecell}
\usepackage{CJKutf8}
% \usepackage{authblk} -- This package is specifically forbidden
% \usepackage{balance} -- This package is specifically forbidden
% \usepackage{caption} -- This package is specifically forbidden
% \usepackage{color (if used in text)
% \usepackage{CJK} -- This package is specifically forbidden
% \usepackage{float} -- This package is specifically forbidden
% \usepackage{flushend} -- This package is specifically forbidden
% \usepackage{fontenc} -- This package is specifically forbidden
% \usepackage{fullpage} -- This package is specifically forbidden
% \usepackage{geometry} -- This package is specifically forbidden
% \usepackage{grffile} -- This package is specifically forbidden
% \usepackage{hyperref} -- This package is specifically forbidden
% \usepackage{navigator} -- This package is specifically forbidden
% (or any other package that embeds links such as navigator or hyperref)
% \indentfirst} -- This package is specifically forbidden
% \layout} -- This package is specifically forbidden
% \multicol} -- This package is specifically forbidden
% \nameref} -- This package is specifically forbidden
% \natbib} -- This package is specifically forbidden -- use the following workaround:
% \usepackage{savetrees} -- This package is specifically forbidden
% \usepackage{setspace} -- This package is specifically forbidden
% \usepackage{stfloats} -- This package is specifically forbidden
% \usepackage{tabu} -- This package is specifically forbidden
% \usepackage{titlesec} -- This package is specifically forbidden
% \usepackage{tocbibind} -- This package is specifically forbidden
% \usepackage{ulem} -- This package is specifically forbidden
% \usepackage{wrapfig} -- This package is specifically forbidden
% DISALLOWED COMMANDS
% \nocopyright -- Your paper will not be published if you use this command
% \addtolength -- This command may not be used
% \balance -- This command may not be used
% \baselinestretch -- Your paper will not be published if you use this command
% \clearpage -- No page breaks of any kind may be used for the final version of your paper
% \columnsep -- This command may not be used
% \newpage -- No page breaks of any kind may be used for the final version of your paper
% \pagebreak -- No page breaks of any kind may be used for the final version of your paperr
% \pagestyle -- This command may not be used
% \tiny -- This is not an acceptable font size.
% \vspace{- -- No negative value may be used in proximity of a caption, figure, table, section, subsection, subsubsection, or reference
% \vskip{- -- No negative value may be used to alter spacing above or below a caption, figure, table, section, subsection, subsubsection, or reference

\setcounter{secnumdepth}{0} %May be changed to 1 or 2 if section numbers are desired.

% The file aaai20.sty is the style file for AAAI Press 
% proceedings, working notes, and technical reports.
%
\setlength\titlebox{2.5in} % If your paper contains an overfull \vbox too high warning at the beginning of the document, use this
% command to correct it. You may not alter the value below 2.5 in
\title{Matching Questions and Answers in Dialogues from Online Forums}
%Your title must be in mixed case, not sentence case. 
% That means all verbs (including short verbs like be, is, using,and go), 
% nouns, adverbs, adjectives should be capitalized, including both words in hyphenated terms, while
% articles, conjunctions, and prepositions are lower case unless they
% directly follow a colon or long dash
%\author{Written by AAAI Press Staff\textsuperscript{\rm 1}\thanks{Primarily Mike Hamilton of the Live Oak Press, LLC, with help from the AAAI Publications Committee}\\ \Large \textbf{AAAI Style Contributions by
%Pater Patel Schneider,} \\ \Large \textbf{Sunil Issar, J. Scott Penberthy, George Ferguson, Hans Guesgen}\\ % All authors must be in the same font size and format. Use \Large and \textbf to achieve this result when breaking a line
%\textsuperscript{\rm 1}Association for the Advancement of Artificial Intelligence\\ %If you have multiple authors and multiple affiliations
% use superscripts in text and roman font to identify them. For example, Sunil Issar,\textsuperscript{\rm 2} J. Scott Penberthy\textsuperscript{\rm 3} George Ferguson,\textsuperscript{\rm 4} Hans Guesgen\textsuperscript{\rm 5}. Note that the comma should be placed BEFORE the superscript for optimum readability
%2275 East Bayshore Road, Suite 160\\
%Palo Alto, California 94303\\
%publications20@aaai.org % email address must be in roman text type, not monospace or sans serif
%}
 \begin{document}

\maketitle

\begin{CJK}{UTF8}{gbsn}
	\section{Supplemental Material}
	\label{sec:appendix}
	
	In this supplementary material, we add additional details supporting the experimental results and case studies.
	
	\subsection{Experimental Results for Ablation tests}
	
	Table \ref{tab:app1} shows the overall performance of four ablations and our full model (HDM). According to the T-test results among three runs, the precision between each two models are not significant different while the recall varies a lot. The full model gets the highest recall and F1-score among all models.
	
	\begin{table}[h]
		\small
		\centering
		\begin{tabular}{p{1cm}<{\centering}p{1cm}<{\centering}ccc}
			\toprule[1.5pt]
			Models &P&R& F1\\
			\midrule[1pt]
			QH&77.54&71.80&74.56\\
			AH&77.50&70.52&73.84\\
			\hline
			NON&75.72&75.91&75.81\\
			ID&76.22&74.76&75.46\\
			\hline
			HDM&76.44&78.44&\textbf{77.43}\\
			\bottomrule[1.5pt]
		\end{tabular}
		\caption{The end-to-end performance of ablation tests.}
		\label{tab:app1}
	\end{table}
	
	Results in Table \ref{tab:app2} reveal that HDM model outperforms the ablations on matching LQAs. QH and AH perform well on SQAs while fail on LQAs. NON and ID sacrifice the accuracy on SQAs a little and make improvements on LQAs. It seems that when the history information is not quite concise and useful, the distance will play an important role on the final results. Besides, improving the accuracy on LQAs always shows a decline of the accuracy on SQAs, which shows the importance of the trade-off between the distance and history features. Although our full model provides a good result on overall performance, how to take better advantage of these two factors is still a challenge.
	
	
	\begin{table}[h]
		\small
		\centering
		\begin{tabular}{p{1.5cm}<{\centering}ccccc}
			\toprule[1.3pt]
			Models &1&2&3&4&$\geq5$\\
			\midrule[1pt]
			QH&96.38&89.98&21.53&21.53&0.79\\
			AH&95.92&87.61&15.81&10.78&4.76\\
			\hline
			NON&95.04&78.66&57.18&37.50&20.04\\
			ID&94.02&81.87&54.50&28.43&11.70\\
			\hline
			HDM &95.99&83.16&59.37&40.44&24.80\\
			\bottomrule[1.3pt]
		\end{tabular}
		\caption{The matching accuracy(\%) of Q-NQ pairs on variable distances for ablation tests.}
		\label{tab:app2}
	\end{table}
	
	\subsection{Example Outputs}
	
	To get a better understanding of the behavior of our models, we include two example outputs in Table \ref{tab:case1}. Both of the cases contain the LQAs with SQAs. The first example is a case where both HYD and HDM predicts QA relations better than RPN. While the second one is a case where both our full model (HDM) and RPN fail, HTY performs best in LQA predictions. As for SQAs, all of the models perform well. However, the DIS model is obviously not capable of matching LQA pairs. It also shows that the distance information sometimes hurts the performance of HDM on matching LQA pairs.
	
	
	%\usepackage{ctex} it changes the row space! what should I do? And translation
	\begin{table*}[h]
		\small
		\centering
		\begin{tabular}{p{1.5cm}<{\centering}cccccc}
			\toprule[1.3pt]
			Ground Truth &RPN&DIS&HTY&HDM&Role&Utterances\\
			\midrule[1.3pt]
			\multicolumn{5}{c}{Q1}&P&\makecell{男,四月。大前天晚上给他试着吃一点蛋黄\\Boy, 4 months. He tried a little yolk yesterday\\ 当天晚上有大便然后到今天为止都\\and shat that night but haven't shat\\还没有大便什么情况????\\ until today what's wrong????}\\
			\hline
			O &O &O &O &O &D&\makecell{你好\\Hello}\\
			\hline
			O &O &O &O &O &P&\makecell{你好\\Hello}\\
			\hline
			\multicolumn{5}{c}{Q2}&D&\makecell{宝宝四个月吗\\Is he four months old}\\
			\hline
			A2&A2&A2&A2&A2&P&\makecell{是的\\Yes}\\
			\hline
			A1&A1&O &A1&A1&D&\makecell{吃的有点早\\Eat too early}\\
			\hline
			A1&O&O &A1&A1&D&\makecell{不建议吃\\Not advise}\\
			\hline
			A1&O&O &A1&A1&D&\makecell{不容易消化哈\\Difficult for digestion}\\
			\midrule[1.3pt]	
			\multicolumn{5}{c}{Q1}&P&\makecell{有什么办法可以减肥呢!\\How can I loose weight!\\太胖了买衣服都难买。烦死了\\It's too difficult to buy clothes. So annoyed}\\
			\hline
			\multicolumn{5}{c}{Q2}&D&\makecell{你多大了,性别\\How old are u, and gender}\\
			\hline
			A2&A2&A2&A2&A2&P&\makecell{男。一八\\Male. Eighteen}\\
			\hline
			\multicolumn{5}{c}{Q3}&D&\makecell{你用过些什么好的方法了\\Any good methods have u tried}\\
			\hline
			A3&A3&A3&A3&A3&P&\makecell{暂时没有哦\\Not yet}\\
			\hline
			A1&A1&O &A1&A1&D&\makecell{如果能坚持锻炼的话,还是可以的\\If you can keep exercising, you will}\\
			\hline
			A1&O &O &O &O &D&\makecell{同时要控制好自己的饮食\\Control ur diet at the same time}\\
			\hline
			A1&O &O &A1&O &D&\makecell{这两点很关键\\These two points are important}\\			
			\bottomrule[1.3pt]
		\end{tabular}
		\caption{Two cases of predictions and human annotations in our dataset.}
		\label{tab:case1}
	\end{table*}
	
	
\end{CJK}



\end{document}
