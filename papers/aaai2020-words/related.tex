\section{Related Work}
\label{sec:related}

There are several studies related to the analysis of vocabulary, including the correlation between frequency and word difficulty, the text-level classification based on vocabulary knowledge,  etc.
%There are several works focusing on the vocabulary knowledge analysis or the text-level classification based on vocabulary knowledge. 
We will discuss the related research in this section.
%.\JQ{rewrite. Besides, it sames that word difficulty is a part of vocabulary knowledge analysis?}

\subsection{Vocabulary Knowledge Analysis}
%\SY{1.The traditional research of linguists and educationists. 2.the research about correlations between word frequency and word difficulty. 3.Some automatic  method to predict word difficulty based on frequency (link).}

Both linguists and educators devoted themselves into the research of vocabulary knowledge. 
There are some previous papers~\cite{koirala2015word,breland1996word,kirkpatrick1949vocabulary} discussing the various features that influence word difficulty. However, most attempts takes the frequency as the only feature for words. It is common sense that there is a negative correlation between the word frequency and word difficulty and it truly performs well in these researches.
%Previously proposed methods have discussed the various features that influence word difficulty.\\
%Most attempts used word frequency as a sole feature.
%There is a negative correlation between the word frequency and word difficulty, as the frequency decreases, the difficulty of word increases~\cite{koirala2015word, breland1996word, kirkpatrick1949vocabulary}. 
%Therefore, frequency can be regarded as an effective measurement for word difficulty estimating.
%\JQ{delete, hide by \%}
Frequency bands are also used to describe the word difficulty distribution.
A typical example is an online language scoring API provided to check the word difficulty by ranking the frequencies in a huge corpus\footnote{\url{https://www.twinword.com/api/language-scoring.php}}. 
% \JQ{what is the global vocabulary set}

Other research talked about the feature combinations that influence word difficulty.
Cesar Koirala~\shortcite{koirala2015word} used the quantity difference between difficult words and easy words to show the function of  word length,  number of syllables and  number of consonant clusters.
Hiebert et al \shortcite{hiebert2019analysis} chose statistic method to discuss the features of words that distinguish students’ performances in various grades.
These features include word frequency, part-of-speech and word morphological family size.
Similarly, spelling rules and morphological features which consist of prefixes and suffixes are also considered in the study of difficulty in Japanese and German \cite{hancke2012readability,nakanishi2012estimating}.

For the morphological features mentioned in previous work, the prefixes and suffixes information has be included in our n-gram feature.
What's more, the n-gram also covers the probability for forming a word.
In this paper, the universal dependency and word embedding features are applied to the recognition of word difficulty for the first time and have achieved remarkable results compared with previous work.

In addition, traditional research mostly focus on theoretical analysis individually, without a comprehensive analysis of all the features. 
They also ignore the importance of developing a effective approach to do the difficulty division task instead of long-term human labors.
%However, tranditional research were more biased towards theoretical analysis or lack of comprehensive feature analysis of words.
%There lacks a computational means to replace long-term human work in predicting words difficulty accurately.

Our method try to solve this bottleneck.
The results of both classification and difficulty ranking tasks show it works in various language environments.
\subsection{Research or Applications based on Vocabulary Knowledge}
%\SY{1.Permance of students on vocabulary study. 2. Text readability analysis based on Vocabulary knowledge.}
The most relevant study of word difficulty is the classification of text readability.
The frequency of words was also valued in text readability analysis~\cite{chen2016characterizing}.
Julia Hancke~\shortcite{hancke2012readability} explored the function of the lexical, syntactic and morphological features in readability classification.

The inspiration of these studies is that the sentence readability cannot be separated from the characteristics of words.
And this is why we add semantic and syntactic features to the analysis of word difficulty.

%What's more,\JQ{Moreover} language models are widely used in word prediction task at sentence level or paragraph level.
%In order to detect the morphological features of words, we propose a novel method to implement an n-gram language model at word level to represent the letter sequence in a word.\JQ{delete? this paragraph is not relevant to related work. This part has been mentioned in the feature design i think}
 