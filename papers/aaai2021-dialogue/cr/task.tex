\section{Task Definition}
\label{sec:task}

Our work aims at identifying the interpersonal relation between interlocutors in dyadic dialogues. 
The types of relationships are pre-defined, annotated as $R=\{R_1, R_2, ..., R_m\}$ where $m$ is the number of relation types.
A number of sessions may happen between the same pair of interlocutor. So, we define the relation classification task in two levels: session-level and pair-level.

Given the $j$-th dialogue session $D_j^i$ between the $i$-th pair of interlocutors,  \textbf{session-level relation classification task} is to inference the most possible relation type for this session, i.e.:
\begin{equation}
R_j^i = \arg\max_{R} f_s(D_j^i)
\end{equation}

Due to the fact that it's quite hard for even human to fabricate the whole story only through one dialogue session, \textbf{pair-level relation classification task} is defined as follows. Given the dialogues between the $i$-th pair of interlocutors denoted as $D^i=(D_1, D_2, ..., D_n)$, pair-level relation classification task is to figure out the most possible relation type for this pair, i.e.:
\begin{equation}
R^i = \arg\max_{R}f_p(D_i) = \arg\max_{R}f_p(D_1, D_2, ..., D_n)
\end{equation}


13-class taxonomy of relationships are covered in our DDRel dataset, 
including {\em child-parent}, {\em child-other family elder}, {\em siblings}, {\em spouse}, {\em lovers}, {\em courtship}, {\em friends}, {\em neighbors}, {\em roommates}, 
{\em workplace superior-subordinate}, {\em colleagues}, {\em opponents} and 
{\em professional contacts},
based on Reis and Sprecher~\cite{reis2009encyclopedia}, 
in which they elaborate on psychological 
and social aspects of various relationships. 
We define these categories by social connections because they make general 
sense in life. Although individual difference exists 
in every real-world case, it was found that such relationship category 
has different expectations, 
special properties (e.g., marriage usually involves sex, 
shared assets and raising children)~\cite{argyle1983sources}, 
distinctive activities (e.g., talking, eating, drinking and 
joint leisure for friendship) and 
rules~\cite{argyle1984rules} of its own, which are agreed across cultures.
Note that this is not an all-round coverage of 
all possible relationships in human society 
and we aim to cover those common
ones in real life which may be of interest in interpersonal 
relationship research.
These fine-grained labels are prepared for possible related future research.


To evaluate the classification ability of the model from coarse-grained to fine-grained, we also clustered our 13 specific 
relations types into 6 classes and 4 classes considering the social field, the seniority 
and the closeness between two speakers. The details of relation types are listed
in Table \ref{table:relationtypes}.