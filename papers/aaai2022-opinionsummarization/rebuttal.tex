\def\year{2022}\relax
%File: formatting-instructions-latex-2022.tex
%release 2022.1
\documentclass[letterpaper]{article} % DO NOT CHANGE THIS
\usepackage{aaai22}  % DO NOT CHANGE THIS
\usepackage{times}  % DO NOT CHANGE THIS
\usepackage{helvet}  % DO NOT CHANGE THIS
\usepackage{courier}  % DO NOT CHANGE THIS
\usepackage[hyphens]{url}  % DO NOT CHANGE THIS
\usepackage{graphicx} % DO NOT CHANGE THIS
\urlstyle{rm} % DO NOT CHANGE THIS
\def\UrlFont{\rm}  % DO NOT CHANGE THIS
\usepackage{natbib}  % DO NOT CHANGE THIS AND DO NOT ADD ANY OPTIONS TO IT
\usepackage{caption} % DO NOT CHANGE THIS AND DO NOT ADD ANY OPTIONS TO IT
\DeclareCaptionStyle{ruled}{labelfont=normalfont,labelsep=colon,strut=off} % DO NOT CHANGE THIS
\frenchspacing  % DO NOT CHANGE THIS
\setlength{\pdfpagewidth}{8.5in}  % DO NOT CHANGE THIS
\setlength{\pdfpageheight}{11in}  % DO NOT CHANGE THIS
%
% These are recommended to typeset algorithms but not required. See the subsubsection on algorithms. Remove them if you don't have algorithms in your paper.
\usepackage{algorithm}
\usepackage{algorithmic}

%ADD
\usepackage{epsfig}
\usepackage{hhline}
\usepackage{booktabs}
\usepackage{multirow}
\usepackage{subfigure}
\usepackage{autobreak}
\usepackage{tablefootnote}
\usepackage{amsmath,amsfonts} %,amssymb,amsthm,amsopn}
\usepackage{epsfig}
\usepackage{booktabs}
\usepackage{diagbox}
\usepackage{array}
\usepackage{multicol}
\usepackage{threeparttable}
\usepackage{epstopdf}
\usepackage{listings}
\usepackage{multirow}
\usepackage{subfigure}
\usepackage{makecell}

\newcommand{\figref}[1]{Figure \ref{#1}}
\newcommand{\eqnref}[1]{Eq. \ref{#1}}
\newcommand{\tabref}[1]{Table \ref{#1}}
\newcommand{\secref}[1]{Section \ref{#1}}
\newcommand{\algoref}[1]{Algorithm \ref{#1}}
\newcommand{\exref}[1]{Example \ref{#1}}
\newcommand{\cut}[1]{}
\newcommand{\tabincell}[2]{\begin{tabular}{@{}#1@{}}#2\end{tabular}}

\newcommand{\KZ}[1]{\textcolor{blue}{Kenny: #1}}
\newcommand{\YZ}[1]{\textcolor{red}{Yizhu: #1}}
\newcommand{\JQ}[1]{\textcolor{green}{JQ: #1}}

%
% These are are recommended to typeset listings but not required. See the subsubsection on listing. Remove this block if you don't have listings in your paper.
\usepackage{newfloat}
\usepackage{listings}
\lstset{%
	basicstyle={\footnotesize\ttfamily},% footnotesize acceptable for monospace
	numbers=left,numberstyle=\footnotesize,xleftmargin=2em,% show line numbers, remove this entire line if you don't want the numbers.
	aboveskip=0pt,belowskip=0pt,%
	showstringspaces=false,tabsize=2,breaklines=true}
\floatstyle{ruled}
\newfloat{listing}{tb}{lst}{}
\floatname{listing}{Listing}
%
%\nocopyright
%
% PDF Info Is REQUIRED.
% For /Title, write your title in Mixed Case.
% Don't use accents or commands. Retain the parentheses.
% For /Author, add all authors within the parentheses,
% separated by commas. No accents, special characters
% or commands are allowed.
% Keep the /TemplateVersion tag as is
\pdfinfo{
/Title (AAAI Press Formatting Instructions for Authors Using LaTeX -- A Guide)
/Author (AAAI Press Staff, Pater Patel Schneider, Sunil Issar, J. Scott Penberthy, George Ferguson, Hans Guesgen, Francisco Cruz, Marc Pujol-Gonzalez)
/TemplateVersion (2022.1)
}

% DISALLOWED PACKAGES
% \usepackage{authblk} -- This package is specifically forbidden
% \usepackage{balance} -- This package is specifically forbidden
% \usepackage{color (if used in text)
% \usepackage{CJK} -- This package is specifically forbidden
% \usepackage{float} -- This package is specifically forbidden
% \usepackage{flushend} -- This package is specifically forbidden
% \usepackage{fontenc} -- This package is specifically forbidden
% \usepackage{fullpage} -- This package is specifically forbidden
% \usepackage{geometry} -- This package is specifically forbidden
% \usepackage{grffile} -- This package is specifically forbidden
% \usepackage{hyperref} -- This package is specifically forbidden
% \usepackage{navigator} -- This package is specifically forbidden
% (or any other package that embeds links such as navigator or hyperref)
% \indentfirst} -- This package is specifically forbidden
% \layout} -- This package is specifically forbidden
% \multicol} -- This package is specifically forbidden
% \nameref} -- This package is specifically forbidden
% \usepackage{savetrees} -- This package is specifically forbidden
% \usepackage{setspace} -- This package is specifically forbidden
% \usepackage{stfloats} -- This package is specifically forbidden
% \usepackage{tabu} -- This package is specifically forbidden
% \usepackage{titlesec} -- This package is specifically forbidden
% \usepackage{tocbibind} -- This package is specifically forbidden
% \usepackage{ulem} -- This package is specifically forbidden
% \usepackage{wrapfig} -- This package is specifically forbidden
% DISALLOWED COMMANDS
% \nocopyright -- Your paper will not be published if you use this command
% \addtolength -- This command may not be used
% \balance -- This command may not be used
% \baselinestretch -- Your paper will not be published if you use this command
% \clearpage -- No page breaks of any kind may be used for the final version of your paper
% \columnsep -- This command may not be used
% \newpage -- No page breaks of any kind may be used for the final version of your paper
% \pagebreak -- No page breaks of any kind may be used for the final version of your paperr
% \pagestyle -- This command may not be used
% \tiny -- This is not an acceptable font size.
% \vspace{- -- No negative value may be used in proximity of a caption, figure, table, section, subsection, subsubsection, or reference
% \vskip{- -- No negative value may be used to alter spacing above or below a caption, figure, table, section, subsection, subsubsection, or reference

\setcounter{secnumdepth}{0} %May be changed to 1 or 2 if section numbers are desired.

% The file aaai22.sty is the style file for AAAI Press
% proceedings, working notes, and technical reports.
%

% Title

% Your title must be in mixed case, not sentence case.
% That means all verbs (including short verbs like be, is, using,and go),
% nouns, adverbs, adjectives should be capitalized, including both words in hyphenated terms, while
% articles, conjunctions, and prepositions are lower case unless they
% directly follow a colon or long dash
\title{AAAI Press Formatting Instructions \\for Authors Using \LaTeX{} --- A Guide}
\author{
    %Authors
    % All authors must be in the same font size and format.
    Written by AAAI Press Staff\textsuperscript{\rm 1}\thanks{With help from the AAAI Publications Committee.}\\
    AAAI Style Contributions by Pater Patel Schneider,
    Sunil Issar,\\
    J. Scott Penberthy,
    George Ferguson,
    Hans Guesgen,
    Francisco Cruz\equalcontrib,
    Marc Pujol-Gonzalez\equalcontrib
}
\affiliations{
    %Afiliations
    \textsuperscript{\rm 1}Association for the Advancement of Artificial Intelligence\\
    % If you have multiple authors and multiple affiliations
    % use superscripts in text and roman font to identify them.
    % For example,

    % Sunil Issar, \textsuperscript{\rm 2}
    % J. Scott Penberthy, \textsuperscript{\rm 3}
    % George Ferguson,\textsuperscript{\rm 4}
    % Hans Guesgen, \textsuperscript{\rm 5}.
    % Note that the comma should be placed BEFORE the superscript for optimum readability

    2275 East Bayshore Road, Suite 160\\
    Palo Alto, California 94303\\
    % email address must be in roman text type, not monospace or sans serif
    publications22@aaai.org
%
% See more examples next
}

%Example, Single Author, ->> remove \iffalse,\fi and place them surrounding AAAI title to use it
\iffalse
\title{My Publication Title --- Single Author}
\author {
    Author Name
}
\affiliations{
    Affiliation\\
    Affiliation Line 2\\
    name@example.com
}
\fi

\iffalse
%Example, Multiple Authors, ->> remove \iffalse,\fi and place them surrounding AAAI title to use it
\title{My Publication Title --- Multiple Authors}
\author {
    % Authors
    First Author Name,\textsuperscript{\rm 1}
    Second Author Name, \textsuperscript{\rm 2}
    Third Author Name \textsuperscript{\rm 1}
}
\affiliations {
    % Affiliations
    \textsuperscript{\rm 1} Affiliation 1\\
    \textsuperscript{\rm 2} Affiliation 2\\
    firstAuthor@affiliation1.com, secondAuthor@affilation2.com, thirdAuthor@affiliation1.com
}
\fi


% REMOVE THIS: bibentry
% This is only needed to show inline citations in the guidelines document. You should not need it and can safely delete it.
\usepackage{bibentry}
% END REMOVE bibentry

\begin{document}

%\maketitle

%\section{General response}
%Thank you very much for your comments. We will explain your concerns point by point.
%All papers submitted for publication by AAAI Press must be accompanied by a valid signed copyright form. They must also contain the AAAI copyright notice at the bottom of the first page of the paper. There are no exceptions to these requirements. If you fail to provide us with a signed copyright form or disable the copyright notice, we will be unable to publish your paper. There are \textbf{no exceptions} to this policy. You will find a PDF version of the AAAI copyright form in the AAAI AuthorKit. Please see the specific instructions for your conference for submission details.

\subsection*{Response to Reviewer 1}

\textbf{Q1\&2}: Easier for readers to understand.
%\textbf{Q1}: Notations and abbreviations. 
%For example, OAs and ISs are referred to $P^e$ and $S^e$. 

\noindent 
\textbf{A1\&2}: 
%We agree with you that we used many notations in this paper. 
%We took OAs and ISs as the abbreviations of Opinion-Aspect Pairs and Implicit Sentences. $P^e$ and $S^e$ are the notations used in the description of approach, which denoted the set of OA {\em pairs} and implicit {\em sentences} extracted from all reviews of an entity $e$.
%To make the notations and abbreviations easier to understand, 
We will revise the annotations
%use $O^e$ and $I^e$ to denote the set of OAs and ISs extracted from all reviews of an entity $e$,
%\smallskip
%\noindent 
%\textbf{Q2}: %It is easier to understand if there are 
%Illustrations of a running example and different models.
%\noindent 
%\textbf{A2}: %We will follow your suggestions. 
%and change Figure 1 in paper to a running example 
%and add the illustration of BAG and BAI to Figure 2 in paper.
and add illustrations of a running example, BAG and BAI to Sec. 2.

%\smallskip
\noindent 
\textbf{Q3}: Evaluation: Human evaluation.

\noindent 
\textbf{A3}: %Human annotators were asked to evaluate the gold summaries and generated summaries by ``considering the consistency with multi-review and informativeness together'' (Section 3.4). 
%We explained the consistency and informativeness for annotators as: 
As per Sec. 3.4, our human evaluation was ``considering the consistency with multi-review and informativeness together''.
The specific instruction includes:
a) how well the sentiment of summary agrees with %the overall of the
multi-review (consistency)?  b) how much useful information %about the product 
does the summary provide (informativeness)?

\cut{
\fbox{
	\parbox{0.9\columnwidth}{
		\small Consistency: how well the sentiment of summary agrees with %the overall of the
		multi-review?
		Informativeness: how much useful information %about the product 
		does the summary provide?
}}
}


%According to your suggestion, 
We follow your suggestion and ask 3 human annotators to evaluate summaries under 5 aspects:
Fluency (Flu), Coherence (Coh), Non-redundancy (NR), Consistency (Cons) and Overall.
%{\em Fluency} (Flu.): the summary is grammatically correct and easy to understand; {\em Coherence} (Coh.): the summary is well organized; {\em Non-redundancy} (Non-red.): there is no unnecessary repetition; {\em Opinion Consistency} (Cons.): the summary reflects common opinions expressed in reviews; {\em Overall}: select the best and the worst summary on your own criteria.
We will include the following table in the revision:
%shown in \tabref{tab:manual}.

\begin{table}[hp]
	\scriptsize
	\centering
	\begin{tabular}{|l|l|c|c|c|c|c|c|}
		\hline
		\bf Data & \bf Model & Flu. & Coh. & NR. & Cons. & Overall & AC \\
		\hline
		\multirow{3}{*}{Yelp}&Gold& 0.34 & 0.49 & 0.41 & 0.35 & 0.31 & -\\
		&TransSum& -0.46 & -0.53 & -0.70& -0.64& -0.48 & 0.36\\
		&MB-B & 0.12 & 0.14 & 0.29 & 0.29 & 0.17 & 0.42 \\
		\hline
		\multirow{3}{*}{Amazon}&Gold& 0.32 & 0.55 & 0.38& 0.44 & 0.32 & -\\
		&TransSum& -0.54 & -0.67 & -0.68 &-0.72 & -0.41 & 0.32 \\
		&MB-B & 0.22 & 0.12 & 0.30 & 0.28 & 0.09 &  0.38\\
		\hline
	\end{tabular}
	%\caption{Human evaluation. %We will add the results of the methods that only using MM OAs or EM OAs as input during training to Table 7 in the paper. 
	%}	
	\label{tab:manual}
\end{table}



%\smallskip
\noindent 
\textbf{Q4}: Evaluation: Aspect coverage (AC).

\noindent 
\textbf{A4}: 
To show it's OK to use a rule-based aspect extraction method as part of the evaluation process,
we ask three annotators to extract aspects from reference summaries as ground truth. 
It turns out the rule-based method can extract 84\% and 88\% of the ground truth aspects from Amazon and Yelp.

We agree that in the evaluation it's better to use human extracted aspects as oracles. 
%Thus, we manually extract the aspects in generated summaries and reference summaries. 
Given a generated summary and its reference summary,
Annotators manually extract aspects from both summaries and compute R-1 recall
between the aspects from generated summary and reference summary. 
The AC scores in the above table
%\tabref{tab:manual} 
are the average of R-1 recall from three annotators. MB-B still performs the best.


%\smallskip
\noindent 
\textbf{Q5}: Evaluation: Ablation study.

\noindent 
\textbf{A5}: 
Table 7 in the paper actually contains ablation study.
%which compares the different ways of using noisy OAs and ISs.
%BAG uses noisy OAs as input, BAI inputs the concatenation of noisy OAs and ISs and MAI uses dual encoder for noisy OAs and ISs. 
As per your request, we conduct further ablation studies on EM OAs and MM OAs for BART-based results in the following table. 
%Due to the limitation of space, 
%Here we just show the BART-based (best) results.
%\begin{itemize}
%	\item EM OAs (R-1/R-2/R-L): Yelp: 33.22/5.54/16.83, Amazon: 30.16/5.24/7.95
%	\item EM OAs (R-1/R-2/R-L): Yelp: 33.22/5.54/16.83, Amazon: 30.16/5.24/7.95
%	\item EM OAs (R-1/R-2/R-L): Yelp: 33.22/5.54/16.83, Amazon: 30.16/5.24/7.95
%\end{itemize}
%BAG performs best in the below table. %\tabref{tab:abla}. 
%The EM OAs is much better than MM OAs, because the OAs of the output summary and MM OAs are mismatched.
We can't use only OAs or ISs of summary as input since the summary is not given during test. 
%\cut{%%%%
\begin{table}[ht]
	\scriptsize
	\centering
	\begin{tabular}{|l|l|m{0.6cm}<{\centering}|c|c|c|c|m{0.6cm}<{\centering}|}
		\hline
		\multicolumn{2}{|c|}{\bf Model} & \multicolumn{3}{c|}{\bf Yelp (R-1/R-2/R-L)} 
& \multicolumn{3}{c|}{\bf Amazon (R-1/R-2/R-L)} \\ %\cline{3-8}
		%\multicolumn{2}{|c|}{}& R-1 & R-2 & R-L & R-1 & R-2 & R-L\\
		\hline
		%only OA & 34.79 & 6.83 & 18.42 & 31.27 & 6.31 & 19.14  \\
		%only IS & 34.79 & 6.83 & 18.42 & 31.27 & 6.31 & 19.14  \\
		\multirow{2}{*}{BART}&EM OAs & 33.22 & 5.54 & 16.83 & 30.16 & 5.24& 17.95\\
		&MM OAs & 20.32 & 3.24 & 11.14 & 19.67& 2.97 & 9.32 \\
%		&BAG & 34.79 & 6.83 & 18.42 & 31.27 & 6.31 & 19.14  \\
		%BAI & 36.27 & 7.94  & 20.60 & 33.96 & 7.23 &20.77 \\
		%MAI & 36.93 & 8.40  & 21.11 & 34.47& 7.57& 21.03\\
		%MB & \bf 37.58 & \bf 8.76 & \bf 21.17 & \bf 35.30 & \bf 7.84 & \bf 21.33  \\
		\hline
	\end{tabular}
	%\caption{Ablation study about EM OAs and MM OAs. %We will add the results of the methods that only using MM OAs or EM OAs as input during training to Table 7 in the paper. 
	%}	
	\label{tab:abla}
\end{table}
%}%%%%%

%\smallskip
\noindent 
\textbf{Q6}: Different Pretrained LMs.

\noindent 
\textbf{A6}: 
We agree that strictly speaking, PlanSum, TransSum and MB-B 
use different pretrained LMs. 
The purpose of Table 3 is not to show the neck-to-neck competition between
the LMs, but rather to give readers a perspective of where we stand among the
SOTA methods. We will remove the classification in the final version.
%Also, Plansum \& TransSum were the previous SOTA methods on the task.
%MB-B can fine-tune the weights of pretrained LM, which is difficult for 
%other two methods because of their model design. 
%It is unfair to compare them with the methods without any pretrained LM.
%So we no longer classify the methods. 
%We will analyze their usage of pretrained LMs.
%the different ways of using pretrained LM.

%\smallskip
\noindent 
\textbf{Q7}: Description of MIN-MINER and syntactic rules.

\noindent 
\textbf{A7}: 
The extract method to extract OAs is not our contribution. We just happen to use MIN-MINER 
and syntactic rules. One can swap them with other methods. 
%when we first mentioned them. 
%Due to the limitation of the space, we will add a URL footnote and provide a detailed method description.
%The use of OA extraction is flexible. 
%The better the extraction method used, the better the results of our approach.

\subsection*{Response to Reviewer 2}
\noindent 
\textbf{Q1}: ROUGE.

\noindent 
\textbf{A1}: In Table 3, only Copycat results used google rouge, while others used
pyrouge  (See footnote 5 and 6 regarding FewSum). 
%we tested FewSum on test sets consistent with other methods , 
We re-evaluated CopyCat by pyrouge (R-1/R-2/R-L): 
28.95/4.80/17.76 (Yelp) and 31.84/5.79/20.00 (Amazon), which are smaller than those in Table 3,
which will be updated in the final version.


%\smallskip
\noindent 
\textbf{Q2}: Human evaluation.

\noindent 
\textbf{A2}: 
Refer to A3 in Response to Reviewer 1.
%According to the suggestion, we follow Copycat for human evaluation (Refer to A2 in Response to Reviewer 1).

%\smallskip
\noindent
\textbf{Q3}: Deeper semantic-level similarities ... more abstractive.

\noindent 
\textbf{A3}: We agree with this. But such methods will bring more noise. For example, OA pairs
that share opinions but different in aspects will also have high similarity.

%\smallskip
\noindent
\textbf{Q4}: Parameters.

\noindent 
\textbf{A4}: The initialized parameters that are in MAI but not in BAG are the parameters 
of BART-large. The parameters of two encoders are not shared.

%\smallskip
\noindent
\textbf{Q4}: Aggregate the information of OAs using LSTM.

\noindent 
\textbf{A4}: It's a good idea to prefix a special token to aggregate all OAs 
and we can definitely try that in the future.
Currently, this part of the design follows (Chen \& Yang 2020) and is not our main contribution.

\subsection*{Response to Reviewer 3}
\textbf{Q1}: Related work.

\noindent 
\textbf{A1}: These works (Elsahar et al. 2021; Tian, Yu, and Jiang 2019; Gerani et al. 2014; Ma et al. 2020; Fabbri et al. 2019) are not mentioned in Sec. 1. 

%\smallskip
\noindent
\textbf{Q2}: Example in Table 8.

\noindent 
\textbf{A2}: We will add more analysis like this:
``Compared with gold summary, the sentence in generated summaries are not coherent enough 
and the coreference is not clear. For example, the sentences in MB output 
on food are incoherent and the `it' in the last sentence denotes the restaurant. 
The last sentence of MB output is inferred by adding ISs.''

%\smallskip
\noindent
\textbf{Q3}: Sec. 2 and other typos.

\noindent 
\textbf{A3}: We will revise the Sec. 2 
%(Refer to A1 and A2 in Response to Reviewer 1) 
and correct the typos.

\subsection*{Response to Reviewer 4}


%\noindent
%\textbf{Q1}: Novelty.
%
%\noindent 
%\textbf{A1}: As far as we know, we are the first to build noisy OAs and noisy ISs according to OAs and ISs in the sampled summary to simulate multi-reviews. 
%
%\smallskip
\noindent
\textbf{Q1}: Conflicting views.

\noindent 
\textbf{A1}: The conflicting views are possible but very unlikely, because
our noisy OAs and ISs are sampled by their semantic distance to the summary.
Majority of the extracted OAs/ISs are consistent with the summary (see Sec. 2.1).  
%In this paper, we first sample a review as summary. 
%Then, we extract the OAs and ISs from this summary and other reviews. 
%We next sample noisy OAs and ISs from other reviews 
%according to the similarity between OAs and ISs of other reviews and sampled summary, 
%which ensures that the the noisy OAs and ISs is consistent with the summary. Besides, it is rare that the sampled summary is inconsistent with all other reviews.

%\smallskip
\noindent
\textbf{Q2}: Implicit information.

\noindent 
\textbf{A2}: Experiments show that non-IS review sentences have few words left (about $10\%$)
with OAs and stop-words removed. 
Human evaluation shows that $<10\%$ of these
sentences contain useful residual information. 
%We also sample 100 reviews from Yelp and Amazon respectively and ask human annotators to record the sentences with loss of information after removed OAs. The percentage of such sentences are $9.1\%$ (Yelp)
%and $9.3\%$ (Amazon), which shows that the sentences from which OAs can be extracted contain very little implicit information.

%\smallskip
\noindent
\textbf{Q3}: Test samples.

\noindent 
\textbf{A3}: Our method performs consistently well on Rotten Tomatoes dataset,  
as shown in following table.
\begin{table}[h!]
	\scriptsize
	\centering
	\begin{tabular}{|l|m{0.6cm}<{\centering}|c|c|}
		\hline
		%\multirow{2}{*}{\bf Model }& \multicolumn{3}{c|}{\bf Rotten Tomatoes} \\ \cline{2-4}
	    \bf Rotton Tomatoes&R-1 & R-2 & R-L \\
		\hline
		PlanSum & 21.77 & 6.18  & 16.98 \\
		TransSum &25.34& 8.62& 18.35 \\
		MB-B & \bf 26.00 & \bf 9.07 & \bf 18.92   \\
		\hline
	\end{tabular}
	%\caption{The evaluation results on Rotten Tomatoes.
	%}	\label{tab:rt}
\end{table}

\nobibliography{aaai22}
\end{document}

