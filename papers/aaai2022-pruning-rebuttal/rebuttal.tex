\def\year{2022}\relax
%File: formatting-instructions-latex-2022.tex
%release 2022.1
\documentclass[letterpaper]{article} % DO NOT CHANGE THIS
\usepackage{aaai22}  % DO NOT CHANGE THIS
\usepackage{times}  % DO NOT CHANGE THIS
\usepackage{helvet}  % DO NOT CHANGE THIS
\usepackage{courier}  % DO NOT CHANGE THIS
\usepackage[hyphens]{url}  % DO NOT CHANGE THIS
\usepackage{graphicx} % DO NOT CHANGE THIS
\urlstyle{rm} % DO NOT CHANGE THIS
\def\UrlFont{\rm}  % DO NOT CHANGE THIS
\usepackage{natbib}  % DO NOT CHANGE THIS AND DO NOT ADD ANY OPTIONS TO IT
\usepackage{caption} % DO NOT CHANGE THIS AND DO NOT ADD ANY OPTIONS TO IT
\DeclareCaptionStyle{ruled}{labelfont=normalfont,labelsep=colon,strut=off} % DO NOT CHANGE THIS
\frenchspacing  % DO NOT CHANGE THIS
\setlength{\pdfpagewidth}{8.5in}  % DO NOT CHANGE THIS
\setlength{\pdfpageheight}{11in}  % DO NOT CHANGE THIS
%
% These are recommended to typeset algorithms but not required. See the subsubsection on algorithms. Remove them if you don't have algorithms in your paper.
\usepackage{algorithm}
\usepackage{algorithmic}

%
% These are are recommended to typeset listings but not required. See the subsubsection on listing. Remove this block if you don't have listings in your paper.
\usepackage{newfloat}
\usepackage{listings}
\lstset{%
	basicstyle={\footnotesize\ttfamily},% footnotesize acceptable for monospace
	numbers=left,numberstyle=\footnotesize,xleftmargin=2em,% show line numbers, remove this entire line if you don't want the numbers.
	aboveskip=0pt,belowskip=0pt,%
	showstringspaces=false,tabsize=2,breaklines=true}
\floatstyle{ruled}
\newfloat{listing}{tb}{lst}{}
\floatname{listing}{Listing}
%
%\nocopyright
%
% PDF Info Is REQUIRED.
% For /Title, write your title in Mixed Case.
% Don't use accents or commands. Retain the parentheses.
% For /Author, add all authors within the parentheses,
% separated by commas. No accents, special characters
% or commands are allowed.
% Keep the /TemplateVersion tag as is

% DISALLOWED PACKAGES
% \usepackage{authblk} -- This package is specifically forbidden
% \usepackage{balance} -- This package is specifically forbidden
\usepackage{color} %(if used in text)
% \usepackage{CJK} -- This package is specifically forbidden
% \usepackage{float} -- This package is specifically forbidden
% \usepackage{flushend} -- This package is specifically forbidden
% \usepackage{fontenc} -- This package is specifically forbidden
% \usepackage{fullpage} -- This package is specifically forbidden
% \usepackage{geometry} -- This package is specifically forbidden
% \usepackage{grffile} -- This package is specifically forbidden
% \usepackage{hyperref} -- This package is specifically forbidden
% \usepackage{navigator} -- This package is specifically forbidden
% (or any other package that embeds links such as navigator or hyperref)
% \indentfirst} -- This package is specifically forbidden
% \layout} -- This package is specifically forbidden
% \multicol} -- This package is specifically forbidden
% \nameref} -- This package is specifically forbidden
% \usepackage{savetrees} -- This package is specifically forbidden
% \usepackage{setspace} -- This package is specifically forbidden
% \usepackage{stfloats} -- This package is specifically forbidden
% \usepackage{tabu} -- This package is specifically forbidden
% \usepackage{titlesec} -- This package is specifically forbidden
% \usepackage{tocbibind} -- This package is specifically forbidden
% \usepackage{ulem} -- This package is specifically forbidden
% \usepackage{wrapfig} -- This package is specifically forbidden
% DISALLOWED COMMANDS
% \nocopyright -- Your paper will not be published if you use this command
% \addtolength -- This command may not be used
% \balance -- This command may not be used
% \baselinestretch -- Your paper will not be published if you use this command
% \clearpage -- No page breaks of any kind may be used for the final version of your paper
% \columnsep -- This command may not be used
% \newpage -- No page breaks of any kind may be used for the final version of your paper
% \pagebreak -- No page breaks of any kind may be used for the final version of your paperr
% \pagestyle -- This command may not be used
% \tiny -- This is not an acceptable font size.
% \vspace{- -- No negative value may be used in proximity of a caption, figure, table, section, subsection, subsubsection, or reference
% \vskip{- -- No negative value may be used to alter spacing above or below a caption, figure, table, section, subsection, subsubsection, or reference

\setcounter{secnumdepth}{0} %May be changed to 1 or 2 if section numbers are desired.

% The file aaai22.sty is the style file for AAAI Press
% proceedings, working notes, and technical reports.
%

% Title

% Your title must be in mixed case, not sentence case.
% That means all verbs (including short verbs like be, is, using,and go),
% nouns, adverbs, adjectives should be capitalized, including both words in hyphenated terms, while
% articles, conjunctions, and prepositions are lower case unless they
% directly follow a colon or long dash

%Example, Single Author, ->> remove \iffalse,\fi and place them surrounding AAAI title to use it

% REMOVE THIS: bibentry
% This is only needed to show inline citations in the guidelines document. You should not need it and can safely delete it.
\usepackage{bibentry}
% END REMOVE bibentry
\newcommand{\KZ}[1]{\textcolor{blue}{Kenny: #1}}
\begin{document}



% Using the \centering command instead of \begin{center} ... \end{center} will save space
% Positioning your figure at the top of the page will save space and make the paper more readable
% Using 0.95\columnwidth in conjunction with the
%\KZ{In the following, if you are going to follow any of the suggestions
%by the reviewer, u can say ``it's a very good idea that ...'' or 
%``it is true that...''. Even if u are not going to follow the advise, as long
%as the comment has some merit, u can paise the reviewers in some appropriate
%way first, like flattering them. The purpose is not be too confrontational
%but be more consultational.}

\section*{Reviewer \#1}

\noindent
\textbf{Q1:`` For pruning layer selection, ... not examining lower layers ($l_b<6$)?''}

\noindent
\textbf{R1:} Recent works~\cite{bert,rogers2020primer} have shown that lower layers~($<6$) of PLMs capture basic linguistic knowledge like constituency syntax, chunking, and word order, while the semantic knowledge is generally learned in higher layers. The commonsense relational knowledge we focus on in this work is one kind of semantic knowledge. Thus, we choose the upper layers of PLMs for pruning.

\noindent
\textbf{Q2:`` In Table 2, ... which doesn’t make sense?''}

\noindent
\textbf{R2:} We have carefully checked our implementation, including different tokenization between WordPiece~(used by DistilBERT, BERT, and MPNet) and BPE~(used by RoBERTa) and optimization setting. However, we observe that RoBERTa consistently shows worse results. We posit that the large vocabulary size~(50,265) of RoBERTa causes the training process to memorize surface co-occurrence patterns on this small training set of C-LAMA.

\noindent
\textbf{Q3:`` From Figure 2~(right), $l_b=6$ ... ''}

\noindent
\textbf{R3:} Setting $l_b$ to 6 removes more weights and produce more specialized subnetwork for a specific commonsense relation,  compared to $l_b>6$. However, it shows slightly worse performance on multi-relation scenario. We show the results of $l_b=6$ because it shows the best trade-off.

\noindent
\textbf{Q4:`` In Table 3, it is surprising  ... ''}

\noindent
\textbf{R4: } We haven't tried this setting, but it's an interesting direction that we would like to explore in the future.

\noindent
\textbf{Q5:`` In Table 3, it is also surprising  ... ''}

\noindent
\textbf{R5: } DistilBERT is less over-parametrized than other larger PLMs. After pruning under the same sparsity regime, its subnetworks have the least amount of remaining weights, making them more specialized for different relations 
thus giving good results on KBC. QA tasks have more requirements beyond 
single commonsense relation, and the larger over-parametrization of models 
like BERT is preferred.

We show one case of KBC in R1 to Reviewer \#3. The novel triple extraction part~(Section 3.2) involves human evaluation. More error analysis on multi-relation cases will be added in the final version. 
%\KZ{If you are putting anything in the final version, u
%better say what error analysis in more details.}

\section*{Reviewer \#2}
Thanks for your feedback. We would like to reiterate the contribution and 
novelty of this paper: a) we are the first to ask the question of whether one can transform a general PLM into dedicated relation-specific knowledge models; 
b) we propose a novel pruning framework to show that we can; and 
c) we conduct in-depth analysis~(both qualitatively and quantitatively) 
and comprehensive experiments~(12 datasets in total) to show 
that the extracted subnetworks can be leveraged for better generalization via a simple yet effective strategy.

\section*{Reviewer \#3}
\noindent
\textbf{Q1:`` It would be helpful to see some qualitative ...''}

\noindent
\textbf{R1:} One example on link prediction: For a triple $<$\textit{classroom}, AtLocation, ?$>$ from KBC test set, the top-3 predicted by original DistilBERT: \textit{home}, \textit{school}, \textit{night} and those by pruned one: \textit{school}, \textit{college}, \textit{university}. Some examples of extracted novel triples are listed in Appendix. We will include more multi-relation examples in Appendix.

\noindent
\textbf{Q2:`` are there any ... consistently modeled than others''}

\noindent
\textbf{R2:} Yes there are. The most popular relations ranked by the number of appearances in the experiments in Section 3.3 are: IsA, CausesDesire, Desires, MotivatedByGoal, which are mostly about \textit{causal} knowledge. Details about optimal relation sets for each task are provided in Appendix.

\noindent
\textbf{Q3:`` It is mentioned that the code and pruned  ...''}

\noindent
\textbf{R3:} The URL was purposely anonymized for blind
review. Our code and models have already been uploaded to the real URL 
and we will show it in the final version.

\section*{Reviewer \#4}
\noindent
\textbf{Q1:`` The method  ... a way of task-specific compression.''}

\noindent
\textbf{R1:} We would like to clarify that we are not doing task-specific compression. Task-specific compression is to produce different sparse networks for different downstream tasks like was done in the reference provided by the reviewer. While in our study, our research question is whether we can transform a general PLM into dedicated knowledge models that inherit different relational commonsense knowledge from pretraining. This transformation should not introduce new knowledge~(hence new parameters). We propose an effective pruning framework to identify subnetworks as the hidden knowledge models for various PLMs in Section 3.1. We then examine the knowledge transfer ability~(Section 3.2 and 3.3) of these subnetworks on \textit{real downstream tasks} by either zero-shot or standard fine-tuning.

\noindent
\textbf{Q2:`` It's non-surprising that the general LM can be compressed or fine-tuned for a specific task.''}

\noindent
\textbf{R2:} Please refer to R1 for our clarification on this.

\noindent
\textbf{Q3:`` The comparisons between original and pruned in Figure 6 and Table 6 are not very meaningful, if we regard pruning as one way of fine-tuning .''}

\noindent
\textbf{R3:} Figure 6 and Table 6 show that our identified relation-specific subnetworks provide more task-relevant prior knowledge hence delivering better performance on those tasks.

\noindent
\textbf{Q4:`` How about the comparisons ...  parameter-efficient-fine-tuning methods~(such as adapter)?''}

\noindent
\textbf{R4:} Parameter-efficient methods like Adapter and Prefix-Tuning are motivated by alleviating catastrophic forgetting issues and high computational cost in standard fine-tuning, and they both introduce additional parameters. Please refer to R1 for a detailed explanation.
% Use \bibliography{yourbibfile} instead or the References section will not appear in your paper
\bibliography{rebuttalbib}
% \bibliographystyle{aaai22}

\end{document}
