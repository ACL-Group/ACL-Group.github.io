\def\year{2022}\relax
%File: formatting-instructions-latex-2022.tex
%release 2022.1
\documentclass[letterpaper]{article} % DO NOT CHANGE THIS
\usepackage{aaai22}  % DO NOT CHANGE THIS
\usepackage{times}  % DO NOT CHANGE THIS
\usepackage{helvet}  % DO NOT CHANGE THIS
\usepackage{courier}  % DO NOT CHANGE THIS
\usepackage[hyphens]{url}  % DO NOT CHANGE THIS
\usepackage{graphicx} % DO NOT CHANGE THIS
\urlstyle{rm} % DO NOT CHANGE THIS
\def\UrlFont{\rm}  % DO NOT CHANGE THIS
\usepackage{natbib}  % DO NOT CHANGE THIS AND DO NOT ADD ANY OPTIONS TO IT
\usepackage{caption} % DO NOT CHANGE THIS AND DO NOT ADD ANY OPTIONS TO IT
\DeclareCaptionStyle{ruled}{labelfont=normalfont,labelsep=colon,strut=off} % DO NOT CHANGE THIS
\frenchspacing  % DO NOT CHANGE THIS
\setlength{\pdfpagewidth}{8.5in}  % DO NOT CHANGE THIS
\setlength{\pdfpageheight}{11in}  % DO NOT CHANGE THIS
\usepackage{color}
%
% These are recommended to typeset algorithms but not required. See the subsubsection on algorithms. Remove them if you don't have algorithms in your paper.
\usepackage{algorithm}
\usepackage{algorithmic}

%
% These are are recommended to typeset listings but not required. See the subsubsection on listing. Remove this block if you don't have listings in your paper.
\usepackage{newfloat}
\usepackage{listings}
\lstset{%
	basicstyle={\footnotesize\ttfamily},% footnotesize acceptable for monospace
	numbers=left,numberstyle=\footnotesize,xleftmargin=2em,% show line numbers, remove this entire line if you don't want the numbers.
	aboveskip=0pt,belowskip=0pt,%
	showstringspaces=false,tabsize=2,breaklines=true}
\floatstyle{ruled}
\newfloat{listing}{tb}{lst}{}
\floatname{listing}{Listing}
\usepackage{makecell}
\setcounter{secnumdepth}{2}

\begin{document}
% \linenumbers

\section{Depression Templates}

Here we provide the templates in detail. We mainly use a combination of 3 direct depression descriptions and the 21 indirect symptoms derived from BDI-II (Table \ref{tab:BDI-II}). We also experimented other well-known depression scales like HDRS (Table \ref{tab:Hamilton}), CES-D (Table \ref{tab:CES-D}) and PHQ-9 (Table \ref{tab:PHQ-9}). The original scales usually contain different descriptions under the same dimension to distinguish different level of intensity or frequency. However, we find that current sentence representations have difficulty in capturing such nuanced differences. We thus condense the descriptions of each dimension into one general template (A few may have more, if there are significant intra-dimension difference).

\begin{table*}[htbp]
  \centering
  \begin{tabular}{l|l}
  \hline
  Dimension & Template \\
  \hline
  Feeling Depressed  &  I feel depressed. \\
  Diagnosis &  I am diagnosed with depression. \\
  Treatment &  I am treating my depression. \\
  \hline
  Sadness & I feel sad.  \\
  Pessimism & I am discouraged about my future.  \\
  Past Failure & I always fail. \\
  Loss of Pleasure & I don't get pleasure from things. \\
  Guilty Feelings & I feel quite guilty. \\
  Punishment Feelings & I expected to be punished. \\
  Self-Dislike & I am disappointed in myself. \\
  Self-Criticalness & I always criticize myself for my faults. \\
  Suicidal Thoughts or Wishes & I have thoughts of killing myself. \\
  Crying & I always cry. \\
  Agitation & I am hard to stay still. \\
  Loss of Interest & It's hard to get interested in things. \\
  Indecisiveness & I have trouble making decisions. \\
  Worthlessness & I feel worthless. \\
  Loss of Energy & I don't have energy to do things. \\
  Changes in Sleeping Pattern & I have changes in my sleeping pattern. \\
  Irritability & I am always irritable. \\
  Changes in Appetite & I have changes in my appetite. \\
  Concentration Difficulty & I feel hard to concentrate on things. \\
  Tiredness  & I am too tired to do things. \\
  Loss of Interest in Sex & I have lost my interest in sex. \\
  \hline
  \end{tabular}
  \caption{\label{tab:BDI-II} The main templates and their corresponding dimensions we used in our experiments, including 3 direct depression descriptions and 21 indirect symptoms derived from BDI-II \citep{beck1996beck}. }
\end{table*}

\begin{table*}[htbp]
  \centering
  \begin{tabular}{|l|}
    \hline
    I have depressed mood. \\
    I always feel sad. \\
    I feel hopeless. \\
    I feel helpless. \\
    I find myself worthless. \\
    I have feelings of guilty. \\
    I always let people down. \\
    I feel like I should be punished. \\
    I think life is not worth living. \\
    I have thoughts of killing myself. \\
    I tried to suicide. \\
    I have difficulty falling asleep. \\
    I feel restless. \\
    I always wake up during the night. \\
    I have lost my interest in many things. \\
    I decrease time spent in my job. \\
    I find it difficult to concentrate on things. \\
    I can not stay still. \\
    I always worry about small things. \\
    I am irritable. \\
    I feel anxiety. \\
    I have a bad appetite. \\
    I am easy to be tired. \\
    I have less interest in sex. \\
    I suffers from menstrual disturbances. \\
    I worry about my health. \\
    I lose weight dramatically. \\
    \hline
  \end{tabular}
  \caption{\label{tab:Hamilton} The templates adapted from the HDRS depression scale \citep{hamilton1986hamilton}. }
\end{table*}

\begin{table*}[htbp]
  \centering
  \begin{tabular}{|l|}
    \hline
    I am bothered by things that usually don't bother me. \\
    I do not feel like eating. \\
    My appetite is poor. \\
    I feel that I could not shake off the blues even with help from my family or friends. \\
    I am not just as good as other people. \\
    I have trouble keeping my mind on what I am doing. \\
    I feel depressed. \\
    I feel that everything I did was an effort. \\
    I feel hopeless about the future. \\
    I thought my life had been a failure. \\
    I feel fearful. \\
    My sleep is restless. \\
    I am unhappy. \\
    I talk less than usual. \\
    I feel lonely. \\
    I think people are unfriendly. \\
    It's difficult for me to enjoy life. \\
    I had crying spells. \\
    I feel sad. \\
    I feel that people dislike me. \\
    I could not get 'going'. \\
    \hline
  \end{tabular}
  \caption{\label{tab:CES-D} The templates adapted from the CES-D depression scale \citep{lewinsohn1997center}. }
\end{table*}

\begin{table*}[htbp]
  \centering
  \begin{tabular}{|l|}
    \hline
    I have little interest in doing things. \\
    I have little pleasure in doing things. \\
    I always feel down. \\
    I always depressed. \\
    I always hopeless. \\
    I have trouble falling asleep. \\
    I sleep too much. \\
    I feel tired. \\
    I have little energy. \\
    My appetite is poor. \\
    I cannot stop overeating. \\
    I feel bad about myself. \\
    I think myself a failure. \\
    I have let other people down. \\
    I have trouble concentrating on things. \\
    I move much slower than before. \\
    I speak much slower than before. \\
    I have been moving around a lot more than usual. \\
    I think that I would be better off dead. \\
    I have thoughts of hurting myself. \\
    \hline
  \end{tabular}
  \caption{\label{tab:PHQ-9} The templates adapted from the PHQ-9 depression scale \citep{kroenke2001phq}. }
\end{table*}

\bibliography{aaai22}

\end{document}