\section{Conclusion}
\label{sec:conclusion}
In this paper, we define a new problem of discovering the linguistic influence of 
the host on the sounds of their pet dogs. Experiments have shown that there is a 
significant difference in audio frequency and speed between the voice of dogs 
in the Japanese language environment and the English environment. 
Specifically, English dogs bark at a lower pitch than Japanese dogs, while Japanese dogs bark
faster than English ones. The phenomena can be observed in humans as well.

We sourced our data from YouTube, which, though noisy, ensures the quantity and variety and has 
rarely been covered in previous studies. The fact that we removed quite a lot of data from
the raw videos is due to the rigorous pipeline we developed for quality purposes. 
Eventually, our EJShibaVoice dataset, which contains a large number of Shiba Inu sound clips 
and their corresponding host speech under various scenes will facilitate future research in 
this field.   
%\JY{rewrite, more detailed findings}

Future direction can be a larger dataset for a more general investigation, i.e. more breeds and 
more sound clips, as well as more various language environments in addition to Japanese 
and English in this study. Note that the pipeline we proposed is language-agnostic and hence 
can be applied to other studies with similar purposes. 

%\par limitations
%\MYW{add a few words about the dataset itself, e.g. the stats about it, the largest dog barking dataset with great variety in location and activity}
