\section{Problem}
\label{sec:problem}
% To understand lexical semantics of dogs with the help of YouTube videos, 
% we seek to address the following technical problems.

Our goal is to understand the lexical semantics of dogs and explore the minimal semantic unit.  
We seek to address the following technical problems: 
\begin{enumerate}
	\item Do dogs use consistent vocal patterns to signify certain meanings?
	\item How to compute the correlation between vocal expressions with possible factors that give rise to different certain meanings?
%How to detemine minimal semantic unit for Shiba Inu dog?\MY{how to compute the correlation between vocal expressions with possible factors that give rise to different certain meanings}

%Given an online video, how to extract valuable information to analyse the semantics of dogs? 
%\MY{do dogs use consisstent vocal patterns to signify certain meanings}
%<<<<<<< .mine
%	\item Given a sequence of dog vocal sounds from a Youtube video, segment it into a sequence of distinct ``words'', similar to what we do to human languages. Further more, divide ``words'' into ``subwords'' and transcribe them into IPA symbols. 
	%\item Given a word with corresponding subwords and its position in the video, extract the context, i.e.,	the location and activity of the dog at that point.

\end{enumerate}

To answer these questions, we need to classify distinct sound types, which are defined as ``words'' and we further phonetically transcribe these words, which are signified as subwords in Section \ref{sec:divide}. Regarding contextual information to uncover the semantics, we define a diversed and comprehensive list for location and activity and utilize respective extraction methods in Section \ref{sec:infer_context}.


%in Section \ref{sec:divide} we propose the method for word segmentation and subwords extraction, and in Section \ref{sec:infer_context}, we introduce the method for context inference. Based on the data we got, we explore the semantics by analyzing the data distribution to uncover potential patterns of dog sounds. \MY{rewrite this paragraph, you still focus on what you did instead of why you did it. To answer..., we need to classify distinct sound types, which are defined as ``words'' and we further phonetically transcribe these words, which are signified as subwords. Regarding contextual information to uncover the semantics, we define diversed and comprehensive list for location and activity and utilize respective extraction method...  }
% To answer this question, we begin by curating a dataset of Shiba Inu videos sourced from YouTube. Our objective is to construct a dataset comprising triplets, each consisting of a segment word of a dog sound audio clip, accompanied by its corresponding location and activity (as depicted in Figure 2). Initially, we focus on extracting the ``words" emitted by the dogs. To ensure the high quality of the audio clips, we segment the dog audio clips into minimal units and eliminate any unwanted noise. Following this, we employ classification techniques to categorize these short units into predefined word types. We define dog surrounding context as a combination of the dog's location and activity. Determining them in YouTube dog videos poses a significant challenge because of the unique video shooting perspective and shooting ways including shaking and scene transition. To overcome this, we establish a robust pipeline that leverages timestamps from the
% ``word'' audio clips with notable accuracy. Then we can analyze the 
% triplets across different contextual scenarios.

