\section{Related Work}
\label{sec:related}
%  \CH{Para1. Previous study that trying to understand dogs. Wants to show that many topics have been explored, but our study is still different and novel. Mainly introduce these studies: study about context related acoustic features, understanding dogs emotion, some studies that apply visual tools on dog study}

It has long been difficult to understand what animals are trying to express. Since we can't speak to them directly, just trying to understand their language becomes an explorable target. Early studies have contributed to our understanding of dogs: dog vocalizations in different context~\cite{molnar2009dogs, robbins2000vocal}, emotion recognition through vocal cues~\cite{pongracz2006acoustic}, and the development of image and video analysis techniques for pet understanding~\cite{mao2023pet}. These studies give us a basic understanding of dog sounds. However, they either only conduct testing experiments, or only studied images and sound signals without mining the relationship between them. Our study does lexical analysis and connects it with the goal of understanding dogs.

% \CH{Para3. Other video datasets about dogs. Want to show that there are also some studies trying to use video modality, a brief introduction to these datasets and what tasks they are used for.}

In our communication with dogs, visual signals complement our understanding through sound signals. There are some interesting datasets that encourage visual tasks which help us understand dogs, such as first-person videos from a camera on the back of dogs for activity classification~\cite{iwashita2014first}, videos with skeletons labeled which help to detect poses~\cite{cao2019cross} and a collection of videos about different animal behavious~\cite{ng2022animal}. These studies have greatly broadened the methods for animal action recognition, but there is a lack of a dataset for dog activities in domestic scenes.


% \CH{Para4. If there are similar studies about other animals}

In addition to dogs, some animals are social animals and have a lot of interspecies interactions and it is an interesting topic about how they communicate with each other. Cetaceans~\cite{bermant2019deep}, elephants~\cite{rossman2020contagious} have a high degree of social complexity, and acoustic features can be used for detection or analyzing responses to other signals. Voice of birds~\cite{koh2019bird, salamon2017fusing, adavanne2017stacked} also brings information that can be used to detect birds or classify breeds. However, it is expensive to record sound and the restricted experiment environment restricts the diversity of data. Our data-driven research pipeline takes advantage of the vast amount of data available on the Internet to build a scalable and diverse dataset.
%\MY{so social?} 

% \CH{Para5. Utilities like panns audioset... Introduce them briefly to those who did not used them before.}

Also, it is of vital importance to make sure the boundary of the words are precise and the background is clear. AudioSet~\cite{gemmeke2017audio} consists of hundreds of audio event classes with human-labeled sound clips and further study~\cite{hershey2021benefit} collected frame-wise labels for a portion of the AudioSet to improve the detection performance. Based on that large-scale audio event dataset, PANNs~\cite{kong2020panns}, including several models for sound event detection are pre-trained. Previous research also tries to decipher different dog sounds ~\cite{web2023sounds, web2018yip}. Our work proposes the definition of words and develops word segment methods based on these foundation works. 

By further splitting the words, we also explore the semantics of subwords and the minimal semantic unit for Shiba Inu dog language. The results show that subwords expressed by IPA vowels do not show a special meaning. This could be attributed to the possibility that IPA vowels are tailored for human language rather than animal communication, yielding the potential need for a broader range of symbols to accurately capture the nuances in vocalization differences.

%This may be because the ipa vowels are applied to humans rather than animals. \MY{this argument seems saying that our method is wrong, rephrase it as that more symbols are needed to represent vocalization differences more precisely}
% \KZ{Put a skeleton here first. What are the topics you are going to cover here?}
