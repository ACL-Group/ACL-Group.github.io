\section{Related Work}
\label{sec:related}
\subsection{Linguistic Structures in Animal Vocalizations}
In animal vocalizations, phones are the fundamental sound units used by different species. Research on chimpanzees reveals that they combine these vocal units into longer, structured sequences to create a versatile vocal communication system \cite{girard2022chimpanzees}. Japanese tits produce distinct alarm calls for different threats and combine these calls into structured sequences, illustrating the complexity of their communication systems \cite{suzuki2021animal}. Similarly, bottlenose dolphins employ unique signature whistles, characterized by specific frequency modulation patterns, to broadcast their identity \cite{janik2013identifying}. Additionally, Egyptian fruit bats use their vocalizations to convey detailed information about the emitter, context, and addressee \cite{prat2016everyday}. These examples highlight that structures similar to linguistic phones and their combinations are present in animal communication, reflecting the complexity and sophistication of these systems across different species.


\subsection{Phoneme Discovery in Human Languages}
Since there are no written letters associated with canine language, it is reminiscent of the documentation of unwritten human languages when we’re trying to do the phonemic analysis of canine language. Generally, there are two types of languages documentation, which are extremely challenging. One is considered as the documentation of extinct languages without any speech resources, while the other is the documentation of unwritten languages with very few language consultants. Our phonemic analysis of canine language is inspired by the documentation of unwritten human languages.

Phonemic analysis is a fundamental part of the description and documentation of a language, which is primarily concerned with identifying the contrastive sounds~\cite{kempton2014discovering}. Since the process of a phonemic analysis involves looking for evidence of contrast between every possible pair of sounds, which is often very time-consuming, we designed an automated and reiterative algorithm to utilize one of most effective methods, minimal pairs. Minimal pairs are two different words that differ in exactly one sound in the same location and considered as the only method to conclusively establish contrast between sounds~\cite{hayes2011intro}.

Previously, to find the putative minimal pairs (with noisy data) in a ``word list'' of unwritten and undocumented Kua-nsi language data, \citet{kempton2014discovering} used a program called Minpair written in C. For the input, the program takes a wordlist with a column containing a phonetic transcription and a column containing a word identifier in English. They also used several algorithms applied to the Kua-nsi data, including putative minimal pair (MP), MP counts and MP independent counts. However, their minimal pair algorithms have surprisingly poor performance based on the ROC-AUC evaluation measure. Therefore, a much more innovative method of minimal pair algorithm is highly needed to expedite the procedures related to the phonemic analysis of any unwritten language, including the canine language in our current study. 

