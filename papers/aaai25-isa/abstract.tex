\begin{abstract}
    % \MY{One sentence is still missing, consider saying something like, There's a difference between images on their semantic complexity. For instance, the ``cookiexxx}
    Quantifying image complexity at the entity level is straightforward, but the assessment of semantic complexity has been largely overlooked.
    As a matter of fact, there are differences in semantic complexity across images.
    For example, the ``Cookie Theft'' picture is widely used to assess human language and cognitive abilities. 
    Compared to most images, it contains richer semantics, allowing it to tell a vivid and engaging story.
    % However, more Cookie Theft-like images are needed for people from different cultural backgrounds and people from different eras etc.
    There is a need for more images like ``Cookie Theft'' to cater to people from different cultural backgrounds and eras.
    % \MY{Assessing the semantic complexity requires human experts and empirical evidences. Automatic evaluation of how semantically rich a picture is will not only benefit researchers in human cognition but also improving vision language models, as images with limited semantics are becoming less challenging for these models.}
    % Additionally, semantically rich images can benefit the development of vision models, as images with limited semantics are becoming less challenging for these models. 
    % Unfortunately, images with high semantic complexity are relatively rare. 
    % \KZ{Why do we need more of such pics?} 
    % Therefore, automatically assessing the semantic complexity of an image to identify more semantically rich images has become a valuable research problem. \MY{can be merged as I suggested}
    Additionally, semantically rich images can benefit the development of vision models, as images with limited semantics are becoming less challenging for these models. 
    Assessing the semantic complexity requires human experts and empirical evidence. 
    Automatic evaluation of how semantically rich an image is will benefit not only researchers in human cognition but also AI models.
    In response, we propose the Image Semantic Assessment (ISA) task to address this problem. 
    We introduce the first ISA dataset and a novel method that leverages language to solve this vision problem. 
    Experiments on our dataset demonstrate the effectiveness of our approach\footnote{Our data and code will be released at: \url{https://anonymous.4open.science/r/ISA}.}.
    % \MY{I think you should use another sentence to describe your contribution in the abstract, a dataset and a model.}
\end{abstract}