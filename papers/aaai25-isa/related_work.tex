\section{Related Work}

\paragraph{Image Quality Assessment}
Image Quality Assessment (IQA) is a task to assess the quality of images.
It mainly concerns various types of distortions introduced in stages of the visual communication systems.
Over the past two decades, the rapid growth of visual media has led to the development of numerous IQA methods~\cite{zhai2020perceptual}.
Some IQA datasets including TID2013~\cite{tid2013}, KonIQ-10k~\cite{koniq10k}, SPAQ~\cite{Fang_2020_CVPR} and PaQ-2-PiQ~\cite{ying2020patches} etc. are proposed.
With the development of Multi-modality Large Language Models (MLLMs), \citet{wu2024qbench} propose Q-Bench to assess the abilities of MLLMs on low-level visual perception and understanding, which plays significant roles in IQA.
The difference between IQA task and our proposed ISA task is that our ISA task focuses on analyzing the semantic content of an image rather than its quality.


\paragraph{Image Aesthetics Assessment}
Different from IQA, Image Aesthetics Assessment (IAA) task assesses the aesthetics of an image from the perspective of its content.
% Typical IAA task aims at computationally distinguishing high-quality photos from low-quality ones 
Typical IAA task seeks to computationally assess quality of photos based on photographic rules~\cite{deng2017image}.
Several IAA datasets are proposed, for example, the Photo.net dataset~\cite{2011Photonet}, the DPChallenge dataset~\cite{2008DPChallenge}, and the TAD66K dataset~\cite{ijcai2022p132} etc.
As the development of image style transfer and AI painting, Artistic Image Aesthetic Assessment (AIAA) task is proposed to automatically evaluate artwork aesthetics~\cite{2015JenAesthetics, fekete2022vienna, Yi_2023_CVPR}.
The difference between IAA and ISA task is that ISA assesses images based on their semantic richness.
% and narrative quality


\paragraph{Image Complexity Assessment}

Image Complexity Assessment (ICA) task is proposed to assess intricacy contained within an image~\cite{visc2009}.
It measures the richness of details and diversity within the image~\cite{Snodgrass1980ASS}.
The SAVOIAS dataset~\cite{saraee2020visual} was created with more than 1,000 images and ground truth labels for the IC analysis.
\citet{ic9600} built the first large-scale IC dataset with 9,600 annotated images IC9600 dataset and proposed a baseline model called ICNet.
Compared to IQA and IAA, ICA task is more relevant to ISA task. 
The Entity Richness Scoring in ISA is basically the same as ICA task. 
The significant difference is that ISA mainly focuses on a higher semantic level, instead of simply assessing complexity at the entity level.
