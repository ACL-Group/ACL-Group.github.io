\section{Conclusion}
In this paper, we recognized the importance of reaching out to depressed users on social media to provide better
community support. We designed a pipeline for three rounds of interaction with depressed users on Chinese Sina
Weibo, and conducted a series of experiments to find out what's the best way of breaking the ice with these depressed users online.
%we are the first to put forward the idea that chat bots can actively and real-timely interact with depressed users of social media. Based on this, we designed and deployed a real-time chat robot platform, which can monitor depressed users on Weibo and get their personal information and post information. Through the analysis of Weibo data set, we find people in need, depressed users and posts to be commented, and use our strategy to break the ice with them. 
Experiment results show that depressed users are more willing to interact with humans, 
and specifically those who have suffered from depression but recovered. 
Moreover, the icebreaker needs to be tailored to the depressed user post. 
%In addition, we also made some preliminary observation and analysis on the data set. 
Finally, the success rate of ice-breaking has some correlations with the depressed user demographics.

%This paper only studies how to break the ice with depressed users on social media. For future work, our chatbot can play the role of cured depressed users and continue to interact with depressed users in depth, so as to achieve the effect of intervening them.