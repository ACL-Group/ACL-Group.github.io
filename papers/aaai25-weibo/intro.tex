\section{Introduction}
\label{sec:intro}
%抑郁症很普遍,会带来严重的社会影响。
%\KZ{One general comment: I notice that you tend to miss a space after a punctuation such as a period. This is common for Chinese people who are not used to inserting spaces. But every punctuation in English is like a word, and words need to be separated with spaces.}
As a common mental health disorder, depression brings a significant burden to individuals, families, and the entire society. According to a WHO study~\footnote{\url{https://www.who.int/news-room/fact-sheets/detail/depression}}, there are 322 million people estimated to suffer from depression, equivalent to 4.4\% of the global population. Nearly half of the in-risk individuals live in the South-East Asia (27\%) and Western Pacific region (27\%) including China and India~\cite{tadesse2019detection}.
%\footnote{https://www.who.int/news-room/fact-sheets/detail/depression}
%\SY{the url is not correct? it seems to be our own github repo, we cannot provide this non-anonymous link}
%\MY{I deleted many things and rewrite the intro part, pls check. you can consider merging the problem section into intro}
%In many countries, depression is still underestimated and under-diagnosed, and it is not treated properly, which may lead to serious self-awareness problems.
%抑郁症患者会选择在社交媒体上需求安慰。
%\KZ{The next three paras are too detailed. Especially the para about chatbot. No need to talk about each work one by one. You can make a general statement and cite all of them together. As long as you cite, you are not ignorant of the work.}
%Social media has increasingly become a popular way for people to exchange information in real time. 
%With the convenience of smartphone apps, social media platforms link individuals to vast networks of friends, information, and support.This connectivity offers a way to measure mental well-being while also presenting opportunities for both beneficial and harmful interactions.
Due to the social stigma surrounding depression, many individuals are reluctant to seek professional help. Instead, they often turn to social media platforms such as Twitter and Facebook~\cite{lau2020android, park2013perception} for emotional support, self-expression, and peer interaction. \textcolor{purple}{(Yanyi: Some psychological research has shown that individuals with depression can alleviate anxiety, tension, stress, and various other negative emotional states by engaging with other users in online communities~\cite{Setoyama2011, Lu2021}.)}
%While non-depressed individuals perceived social media as a source of information, depressed individuals perceived it as a platform of social awareness and emotional support.  
%Some research~\cite{park2013perception} show that social media can help people with depression to view themselves more objectively, and people with depression may get comfort from revealing life events and communicating with others through social media. 
%Social media provides a safe space for users with depression, so that they can get support and express their feelings with the help of virtual platforms, thus reducing their loneliness.\SY{The first two paragraphs are too long.}

%为什么会选择微博:微博是中国最大的社交媒体,被视为中国的Twitter;且在中国境内无法访问Twitter和Facebook。
%\MY{you should say that there are multiple social media platforms first, and then elaborate on why we focus on chinese and weibo associated}


%Each user's emotional state can be expressed through their posts on Weibo~\cite{li2015attitudes}. 
%The research on using social media to investigate depression in China is still in its early stages. However, these two websites cannot be used to study depression in China, because they are inaccessible in Chinese Mainland, so we choose Weibo as the social media platform for research.
%\KZ{This last sentence seems to be at an odd place, nothing to do social media.. Every sentence within a para must be well organized.} 
%\KZ{What's the purpose of this para? Are you saying social media is a good venue to reach out to depressed people and help them? I don't quite catch that in here.} 
%Social media has become a hot topic in discussions about mental health, attracting attention from scholars, the media,  
%and the public.  

%聊天机器人的作用,列举出现有的机器人在减轻抑郁症状上的例子
%\MY{This paragraph is too long and detailed, readers will get lost, usually for an introduction, we have 3-4 paragraph structure, each standing for a leading point to your arguments and motivation. Here the main point is that existing evidence supports the effectiveness of chatbots, however proactive chat strategy, in particular with Mandarin depressed patients are less investigated. Then it can lead to your contribution part.} 

%In recent years, chatbots have been implemented in a wide range of sectors, including customer service, and their utilization has been expanding in the medical domain, especially in the realm of mental health~\cite{ahmed2023chatbot}.  
 
%Studies have found that some people like to talk anonymously with chatbots, which enables chatbots to detect diseases and provide some medical support~\cite{lucas2014s}. \citet{lee2023exploring} pointed out that the interaction between depressed users and chatbots can effectively alleviate the social shame of mental illness.
%Compared with a real psychologist, people tend to express more negative and extreme emotions to a chatbot~\cite{miner2016conversational}.  
With the advancement of numerous efficacious chatbots~\cite{ahmed2023chatbot}, human-machine interaction has emerged as a viable alternative to traditional human-human interactions. Chatbots can positively impact the mental health of people of all ages, including relieving pain and increasing positive emotions~\cite{chen2018social, kabacinska2021socially, miner2016conversational}. A few pioneering studies on social media interactive chatbots~\cite{kaywan2023early, chenbt5153} further suggest their effectiveness in relieving depressive symptoms.
%There are also some chatbots combined with social media.
%\citet{kaywan2023early} integrate DEPRA chat bots with Facebook Messenger, so that participants can interact with chat bots through social media and share their answers. The results show that the participants' responses indicate high satisfaction. \citet{chenbt5153} develop a Weibo 
%based depression detection system and believes that ordinary depression has great potential to be cured and improve their condition, especially compared to those who commit suicide.
However, these chatbots communicate with patients passively, requiring social media users to initiate the conversation. Consequently, some patients may lack the channels to discover these chatbots, leading to missed opportunities for communication. Additionally, certain patients with depression may be introverted or feel ashamed to express their emotions, further reducing the chances for chatbot intervention. \textcolor{purple}{(Yanyi: To address this issue, we focused on developing proactive chatbots. Proactive chatbots on social media can identify depressive symptoms in users, provide timely support, and detect problems earlier than passive systems that wait for users to seek help, thereby improving treatment effectiveness. Previous research on "Tree Hole Users" ~\cite{Jing2020} demonstrated that such proactive outreach on social media can alleviate the stigma of seeking mental health support by offering anonymous and non-invasive means, catering to users' need for discreet emotional expression.)}
%Research has found that some people are more comfortable with anonymous conversations chatbots\KZ{cite there? Is it more comfortable with bots or just anonymously online?}, which enables the chatbots to detect illnesses and provide some medical support~\cite{lucas2014s}.\KZ{But our research result indicate otherwise? That depressed user actually prefer human thanbots? How are we gonna say this?}

%\KZ{is there any previous research on interaction strategies on Twitter/FB or Weibo?
%Some citations will be needed here. I think there's plenty of research on depression on weibo. So you have
%to rephrase it in another way.} 
%Therefore, we have chosen to use chatbots to communicate with depressed users on Weibo.
%\KZ{The reason for targeting Weibo is not strong enough.}



In this work, we propose the idea of proactively seeking out depressed users who need help. We developed a user identification protocol to screen potential depressed users and attempt to break the ice with them through \textbf{initiative} communication and connection. This approach helps users open up, promotes the treatment process, and ultimately improves their mental health.
%\MY{The motivation here that other people use English platforms while we use weibo is not strong enough. In your intro, you should list evidences and arguments that pave a way for your own contribution. Why ice-breaking is important? What's its relation with depression treatment? Why we are intrested in comparing chatbots vs. human? Please make these clear in the intro part.}
Our main contributions are:
%\KZ{You can highlight some keywords below with boldface.}
\begin{itemize}
    \item We are the first to propose the idea of allowing chatbots to \textbf{proactively interact} with depressed users on social media, providing them with opportunities for emotional support and psychological counseling, helping them alleviate symptoms and enhance psychological resilience.
    \item Using our user identification and post selection strategy, we collected a new social media depression dataset consisting of 2,193 users. We annotated their post content into four categories and provide ice-breaking responses with \textit{8 interaction roles} $\times$ \textit{4 emotional support strategies}.
     \item We conducted a series of comparative experiments to understand the interaction preferences of depressed users on social media. Our findings indicate that depressed users are more willing to communicate with humans rather than chatbots. The most effective interaction strategy varies depending on the category of post content. Interestingly, regardless of whether they are communicating with chatbots or humans, depressed users are more willing to interact with individuals who have recovered from depression. %\MY{this final finding is not very informative, do you have anything more interesting?}.
    %\item We design and deploy a real-time chatbot platform that can obtain personal information and post information of users in the Weibo Depression Super Topic Community, identify users with depression tendencies, and identify ice-breaking points, attempting to initiate conversations with depressed users. 
%\KZ{If this is a contribution, then you wanna show how efficient and effective this system is: how many new target users have you found, how much data you have collected, etc. But I'm not sure if this will backfire. We need to discuss this today!} \MY{I don't think we really developed a platform..it's still two-staged actions, you generate responses on chatgpt, and then send out posts on weibo. I think your second contribution should be collecting a dataset and analysing different content types and identify strategies for responses. then the current 2nd contribution can be moved to 3rd.}
\end{itemize}