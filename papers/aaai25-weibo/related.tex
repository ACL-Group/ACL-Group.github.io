\section{Related Work}
\label{sec:related}
% \LC{Para1. 现有的文章大多是关于抑郁症检测和情感分析的}
\paragraph{Depression analysis and detection}
Researchers have been analyzing the online behaviors of depressed users for a long time.
\citet{park2012depressive} explore the use of language to describe depression by using real-time emotions 
obtained from Twitter users.~\citet{saha2021sentiment} use machine learning algorithm to analyze the emotions of posts and comments on social media such as Twitter and Facebook.~\citet{wang2024explainable} uses llms to predict whether social media users suffer from depression by using a large number of articles provided by Reddit users.
\ZT{~\cite{hill2009helping} examines the skills for providing assistance to depressed individuals as a conversational partner.}
~\citet{zhang2022symptom} build the first annotated corpus PsySym for multi-symptom recognition of mental illness, and realize the task of multi-symptom recognition through a new annotation framework.
These depression detection efforts demonstrated that it is possible to analyze massive depressed users on 
social media. 


\paragraph{Chatbot for Mental Health}
Chatbots have been used in depression research for purposes such as therapy, training, and 
screening~\cite{abd2019overview}. Chatbots have played a certain role in treating depression, 
especially in providing emotional support, listening and guidance. Some studies have shown that communicating 
with chatbots can help depressed patients reduce loneliness, stress and anxiety, and provide a low-pressure 
and readily available support. Chatbots are widely used among minors, adults, and the elderly.~\citet{viduani2023assessing} developed a novel chatbot IDEABot based on WhatsApp, 
aiming to help collect emotional data of teenagers in depth.~\citet{he2022mental} show that the mental health chatbot (XiaoE) based on cognitive behavioral therapy (CBT) provides easy access and self-directed mental health help for young people with depressive symptoms.~\citet{ryu2020simple} designed and developed a mental health chatbot, Yeonheebot, to reduce the anxiety and depression of the elderly.
Mental health chatbot is a new digital technology, which can provide full-automatic intervention for depressed users. 
However, although these chat bots can help depressed users to relieve symptoms, many depressed patients will not take the initiative to seek help.



