%\vspace{-5pt}
\section{Conclusion}
%\vspace{-5pt}
We present the \socvec\ method to compute cross-cultural differences and similarities, and evaluate it on two novel tasks about mining cross-cultural differences in named entities and computing cross-cultural similarities in slang terms.
Through extensive experiments, we demonstrate that the proposed lightweight yet effective method outperforms a number of baselines, and can be useful in 
translation applications and cross-cultural studies in 
computational social science. Future directions include: 1) mining cross-cultural differences in general concepts other than names and slang, 2) merging the mined knowledge into existing knowledge bases, and 3) applying the \socvec\ in downstream tasks like machine translation.\footnote{We will make our code and data available at \url{https://github.com/adapt-sjtu/socvec}.}

%
% that the cultural properties and social elements of a term (including both entity names and slang terms) can be effectively embedded by its similarities to social words with the help of our proposed 
%\textit{SocVec}, which enables the comparison between two incompatible 
%monolingual semantic spaces. 
%Our proposed framework can be valuable assistance to cross-cultural 
%social studies, acting as a building block for 
%computing such cross-cultural differences and similarities. 
%The two novel tasks with datasets as well as benchmark results also benefit 
%further study in related topics of computational social science.
%, such as detection of cross-cultural differences in named entities and extraction of a bilingual lexicon for Internet slangs.
