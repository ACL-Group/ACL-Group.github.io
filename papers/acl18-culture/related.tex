\section{Related Work}
\label{sec:related}
%Our work uses a bilingual 
%lexicon, socio-linguistic vocabularies and comparable documents.
%all of which are publicly available and easy to obtain. 
%Actually there are other interesting culture related applications such as the
%two tasks presented in this paper, which require special treatment from 
%sociolinquistic point of view.
%However, none of existing methods worked on it specifically. 

Although social media messages have been essential resources for research in computational social science, most works based on them only focus on a single culture and language ~\cite{Petrovic2010StreamingFS,Paul2011YouAW,Rosenthal2015ICA,Wang2015ThatsSA,Zhang2015ContextawareEM,DBLP:conf/aclnut/LinXLZ17}.
Cross-cultural studies have been conducted on the basis of a questionnaire-based approach for many years.
There are only a few of such studies using NLP techniques.

\citet{nakasaki2009visualizing} present a framework to visualize the cross-cultural differences in concerns in multilingual blogs collected with a topic keyword.  
\citet{elahi2012examination} show that cross-cultural analysis through language in social media data is effective, especially using emotion terms as culture features, but the work is restricted in monolingual analysis and a single domain (love and relationship).
\citet{Garimella2016IdentifyingCD} investigate the cross-cultural 
differences in word usages between Australian and American English through 
their proposed ``socio-linguistic features'' (similar to our social words) in a supervised way. 
With the data of social network structures and user interactions, \citet{Garimella2016QuantifyingCI} study how to quantify the controversy of topics within a culture and language.
\citet{Gutirrez2016DetectingCD} propose an approach to detect differences of word usage in the cross-lingual topics of multilingual topic modeling results.
To the best of our knowledge, our work for Task 1 is among the first to mine and quantify the cross-cultural differences in concerns about named entities across different languages.
%While their supervised model is not generalizable to  cross-lingual differences and differences of named entities. 
%Their results show that socio-linguistic vocabulary 
%are essential in cross-cultural analysis of text. 
%However, their research, which uses traditional topic modeling and SVM classifier,
%cannot be applied directly to cross-lingual tasks. To our knowledge, 
%we are the first to focus on cross-cultural differences on named entities and 
%to propose an approach to conduct cross-lingual cross-cultural social studies 
%through vector representation of words. 


Existing research on slang mainly focuses on automatic 
discovering of slang terms~\cite{elsahar2014a} and normalization of noisy texts ~\cite{han2012automatically} as well as slang formation. 
\citet{ni2017learning} are among the first to propose an automatic 
supervised framework to mono-lingually explain slang terms using 
external resources.
However, research on automatic translation or cross-lingually explanation for slang terms is missing from the literature. 
Our work in Task 2 fills the gap by computing cross-cultural similarities 
with our bilingual word representations (\textit{SocVec}) in an 
unsupervised way.
We believe this application is useful in machine translation for social media~\cite{wangling:acl2013}.

Many existing cross-lingual word embedding models
rely on expensive parallel corpora with word or sentence 
alignments~\cite{klementiev2012inducing,kovcisky2014learning}.
% a 
%supervised model to learn a transformation matrix between two monolingual 
%vector spaces~\cite{Mikolov:2013tp}. 
These works often aim to improve the performance on monolingual tasks and cross-lingual model transfer for document classification, which does not require cross-cultural signals. 
We position our work in a broader context of ``monolingual mapping'' based cross-lingual word embedding models in the survey of ~\citet{DBLP:journals/corr/Ruder17}. 
The \socvec\ uses only lexicon resource and maps monolingual vector spaces into a common high-dimensional third space by incorporating social words as pivot, where orthogonality is 
approximated by setting clear meaning to each dimension 
of the \textit{SocVec} space.

%However, they fail to  capture socio-linguistic information.  
