\section{Related Work}
Classifying relations between entities in a certain sentence plays a key role in NLP applications and thus has been a hot research topic recently.
Feature-based methods~\cite{sem} and neural network techniques~\cite{socher2011semi, ebrahimi2015chain} are most common. 
%\citeauthor{}(2011) design a RNN method using the constituency parse tree. 
%\citeauthor{} (2015) narrows down the attention to shortest dependency path(SDP) of given sentences. 
\citeauthor{xu2015classifying}~(2015) introduce multi-channel SDP-based LSTM model to classify relations incooperating several different kinds of information of a sentence improved by \citeauthor{xu2016improved}~(2016), which performed best on SemEval-2010 Task 8 and is one of our baseline methods. 

The most related work to ours is the extraction of
visual commonsense knowledge by \citeauthor{yatskar2016stating}~(2016). 
This work learns the textual representation of seven types of fine-grained 
visual relations using textual caption for the image in MS-COCO dataset~\cite{lin2014microsoft}.
%such as ``touches'', ``above'' and ``disconnected from'' 
%by jointly modeling the relative position of the 80 kinds of objects in 300,000 images
%and the textual caption for the image in MS-COCO dataset\cite{lin2014microsoft}\cite{lin2014microsoft}. 
%The authors generalized their extracted knowledge using WordNet. 
%Their resource not scalable for its expensive human labor, 
%we propose a framework to use large text which is scalable and involves more real world description. 
Another important related work is from \citeauthor{li2016commonsense}~(2016), which enriches
several popular relations in \textit{ConceptNet} with little textual information from
real large corpora. However, \lnear~relation was not studied in this work, while this relation is extremely scarce in \textit{ConceptNet} and has its own distinctiveness.
% Besides, this work only seeks to
%add more edges into a knowledge graph,  
