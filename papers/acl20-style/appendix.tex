\documentclass[11pt,a4paper]{article}
%\usepackage[hyperref]{emnlp2020}
\usepackage{times}
\usepackage{latexsym}
\usepackage{graphicx}
\usepackage{amsmath}
\usepackage{amsfonts}
%\usepackage{float}
\usepackage{multirow}
%\usepackage{diagbox}
\usepackage[linesnumbered, boxed, ruled]{algorithm2e}
\newcommand{\KZ}[1]{\textcolor{blue}{Kenny: #1}}
\renewcommand\arraystretch{1.2}
\setlength\parskip{0.1\baselineskip}
\setlength{\textfloatsep}{0.5cm}
% This is not strictly necessary, and may be commented out,
% but it will improve the layout of the manuscript,
% and will typically save some space.
\usepackage{microtype}
\newcommand{\secref}[1]{Section \ref{#1}}
\newcommand{\figref}[1]{Figure \ref{#1}}
\newcommand{\eqnref}[1]{Eq. (\ref{#1})}
\newcommand{\tabref}[1]{Table \ref{#1}}
\newcommand{\exref}[1]{Example \ref{#1}}
\newcommand{\cut}[1]{}

\usepackage{tikz}
\usepackage{geometry}
\usetikzlibrary{automata,positioning}
\geometry{left=2.0cm, right=2.0cm, top=2.5cm, bottom=2.5cm}


\title{ST$^2$: Small-data Text Style Transfer via Multi-task Meta-Learning}

\author{Xiwen Chen, Kenny Q. Zhu \\
	Advanced Data and Programming Technologies Lab \\
	Shanghai Jiao Tong University \\
	\texttt{\{victoria-x@sjtu, kzhu@cs.sjtu\}.edu.cn} \\
}
\date{}


\begin{document}

\appendix
\section{Data Characteristics}
\label{ap:datastats}

Table \ref{tb:datastats} shows the vocabulary size, average sentence length, number of adjectives and Flesch readability of our two datasets. For each dataset, we only selected a specific pair of styles, which are Michael R. Katz/Richard Pevear for LT, and simple/standard Wikipedia for GSD. (Flesch readability ranges from 0 to 100, with higher meaning easier to read.)

\begin{table*}[th]\footnotesize
	\centering
	\begin{tabular}{c|cccc}
		& Vocab Size & Avg. Length & \#. of Adj. & Flesch \\
		\hline
		LT & 12589 / 13643 & 19.8 / 21.3 & 15338 / 13623 & 65.1 / 66.2 \\
		GSD & 18124 / 20112 & 11.1 / 11.8 & 10692 / 12070 & 58.5 / 52.3
	\end{tabular}
	\caption{Dataset characteristics.}\label{tb:datastats}
\end{table*}

\section{More Generated Samples}

Table \ref{tb:qualmore} lists some more samples that are randomly selected from the generated sentences of all models.

\begin{table*}[th]\footnotesize
	\centering
	\begin{tabular}{c|c}
		\hline
		\textbf{Original Sentence} (Yelp positive) & \emph{the staff is welcoming and professional .} \\
		\hline
		Template & \emph{the staff is welcoming and professional .} \\
		CrossAlign & \emph{glad glad glad} \\
		CrossAlign (pretrained) & \emph{the staff is welcoming and professional .} \\
		DeleteRetrieve & \emph{the staff is a time .} \\
		DualRL & \emph{less expensive have working .} \\
		VAE & \emph{the staff is rude and rude} \\
		VAE (pretrained) & \emph{the staff is extremely welcoming and professional .} \\
		\hline
		ST$^2$-CrossAlign (ours) & \emph{the staff is friendly and unprofessional} \\
		ST$^2$-VAE (ours) & \emph{the staff are rude and unprofessional .} \\
		\hline
		\hline
		\textbf{Original Sentence} (Yelp negative) & \emph{these people do not care about patients at all !} \\
		\hline
		Template & \emph{these people wonderful about patients at all !} \\
		CrossAlign & \emph{glad glad glad} \\
		CrossAlign (pretrained) & \emph{these people do not care about patients at all !} \\
		DeleteRetrieve & \emph{i was n't be a a appointment and i have .} \\
		DualRL & \emph{and just like that it was over and i was .} \\
		VAE & \emph{these people do not care about patients or doctors} \\
		VAE (pretrained) & \emph{these guys do n't care about the patients at time} \\
		\hline
		ST$^2$-CrossAlign (ours) & \emph{these people do not satisfied at all !} \\
		ST$^2$-VAE (ours) & \emph{i was so happy and i did n't consent} \\
		\hline
	\end{tabular}
	\caption{Randomly selected sample outputs for the Yelp positive/negative review dataset.}\label{tb:qualmore}
\end{table*}


\end{document}