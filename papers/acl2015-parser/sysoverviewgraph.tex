% System Overview
% By Bean
\documentclass{article}

\usepackage{tikz}
\usetikzlibrary{shapes,arrows,shadows,calc}
\usepackage{amsmath,bm,times}
\usepackage{verbatim}
\thispagestyle{empty}
\begin{document}
% Define the layers
\pgfdeclarelayer{background}
\pgfsetlayers{background,main}

% Define block styles used later
\tikzstyle{input}=[draw, fill=blue!20, text width=8em,
    text centered, minimum height=2em,drop shadow]
\tikzstyle{output}=[draw, fill=green!20, text width=8em,
    text centered, minimum height=2em,drop shadow]
\tikzstyle{intermediate}=[draw,dashed, fill=blue!20, text width=8em,
    text centered, minimum height=2em,drop shadow]
\tikzstyle{sys} = [draw, text width=10em, fill=red!20,text centered,
    minimum height=3em, rounded corners, drop shadow]

% Define distances for bordering
\def\blockdist{2.3}
\def\edgedist{2.5}

\begin{tikzpicture}
    \path (0,0) node (oracle) [intermediate] {Oracle Sequence};
    \path (oracle.center)+(0,-2) node (trainset) [input] {Train Treebank};
    \path (oracle.east)+(4,0) node (seqpredictor) [sys] {Sequence Predictor};
    \path (trainset.east)+(4,0) node (headmapper) [sys] {Head Mapper};
    \path (oracle.east)+(9.5,0) node(testset) [input] {Test Sentence};
    \path (trainset.east)+(9.5,0) node(output) [output] {Output Parse};

    \path [draw, ->,line width=0.8pt] (trainset.north) --  node[right] {\textcircled{\small{1}} preprocess} (oracle.south);
    \path [draw, ->,line width=0.8pt] (oracle.east) --  node[above] {\textcircled{\small{2}} train} (seqpredictor.west);
    \path (headmapper.west)+(0.1,0.005) node (temp1) {};
    %\path [draw, ->,line width=0.8pt] (oracle.east)+(0,-0.1) --  node[right] {\textcircled{\small{3}} train} (temp1);
    \path [draw, ->,line width=0.8pt] (oracle.east)+(0,-0.1) --  node[right=-0.5,above=0.5,rotate=-42] {\textcircled{\small{3}} train} (temp1);
    \path [draw, ->,line width=0.8pt] (trainset.east) --  node[below] {\textcircled{\small{3}} train} (headmapper.west);
    \path [draw, ->,line width=0.8pt] (seqpredictor.south) --  node[right] {\textcircled{\small{5}} sequence} (headmapper.north);
    \path [draw, ->,line width=0.8pt] (testset.west) --  node[above] {\textcircled{\small{4}} query} (seqpredictor.east);
    \path (headmapper.east)+(-0.1,0.005) node (temp2) {};
    %\path [draw, ->,line width=0.8pt] (testset.west)+(0,-0.1) --  node[right] {\textcircled{\small{5}} query} (temp2);
    \path [draw, ->,line width=0.8pt] (testset.west)+(0,-0.1) --  node[left=0.5,above=0.5,rotate=42] {\textcircled{\small{5}} query} (temp2);
    \path [draw, ->,line width=0.8pt] (headmapper.east) --  node[below] {\textcircled{\small{6}} decode} (output.west);
    \path (headmapper.south)+(5,-0.5) node(parse) {\textbf{Parse Process}};
    \path (seqpredictor.north)+(-5,0.5) node(train) {\textbf{Train Process}};

    \begin{pgfonlayer}{background}
        %fill parse process
        \path (seqpredictor.west)+(-0.3,0.85) node (temp3) {};
        \path (parse.south -| output.east)+(0.35,-0.25) node (temp4) {};

        \path[fill=yellow!20,rounded corners, draw=black!50, dashed]
            (temp3) rectangle (temp4);
        %fill train process
        \path (train.north -| oracle.west)+(-0.35,0.25) node (temp5) {};
        %\path (oracle.west)+(-0.3,1.5) node (temp5) {};
        %\path (train.east -| headmapper.south)+(4.6,-0.35) node (temp6) {};
        \path (headmapper.east)+(0.3,-0.9) node (temp6) {};
        \path[fill=yellow!20,rounded corners, draw=black!50, dashed]
            (temp5) rectangle (temp6);
        \path[rounded corners, draw=black!50, dashed]
            (temp3) rectangle (temp4);


    \end{pgfonlayer}
\end{tikzpicture}

\end{document}
