\subsection{Entity Linking}

% 11 sentences

% main: tfidf for words on freebase
% based on overlap sim.
% pseudo code?
%


% 0. what are we going to do ?
% Given a relation tuple, we are going to find representative Freebase entities
% which stand for the arguments.
% 1. Formally definition
Formally, given $arg=(w_1,w_2,...,w_m)$, which contains $m$ words, we are going
to find a best Freebase entity standing for this argument, making relation tuples
to linked tuples, represented by $ltup=\langle ent1,\ rel,\ ent2 \rangle$.

% 2. ??? Talk about Freebase
% 3. each mid in FB has one or more entity names.
% example, China and PRC.
Each entity in Freebase has one or more aliases, forming $AList$. The default alias is the name of this entity.
For example, the entity \textit{m.02\_286} has the name ``New York City'' and other aliases
such as ``The Big Apple'' ``NYC'' and ``Empire City''.
% 4. leverage multiple namees to build an invert index.
We aim to support fuzzy matching between arguments and entity aliases,
so we take all the aliases into consideration, and build an inverted index
pointing words to all aliases that appear.

% 5. stop word set is used, and use the idf score to weight words.
Different words in one alias cannot be treated equally. Intuitively, a word
is more important if it occurs in fewer aliases, and vice versa.
Based on the inverted index, we use inverted document frequency score to
approximately model the weight of a word:
\begin{equation}
idf(w)=1\ /\ log(|\{alias : w \in alias\}|)
\end{equation}

\noindent
Besides, stop words are removed from aliases, treating their idf scores as 0.
% 6. matching rule: intersect >= N - 1, weighted overlap score >= threshold

In order to measure the probability of fuzzy matching from an argument to an alias,
we introduce the weighted overlap score:
\begin{equation}
overlap(arg, alias) = \frac {\sum\limits_{w \in arg \cap alias} idf(w)} {\sum\limits_{w \in arg \cup alias} idf(w)}
\end{equation}

\noindent
We merge all the aliases of an entity, producing the fuzzy matching score to entity level:
\begin{equation}
\begin{aligned}
score&(arg, ent) = \\
&\max\limits_{alias \in AList(ent)} overlap(arg, alias)
\end{aligned}
\end{equation}

For one argument having $n$ words (stop words are removed), we keep entities which has at least one alias
matching $n-1$ words in the argument, and has a matching score larger than a threshold, $\tau$.
% we can tune the threshold
% we can use formula to show the weighted score, that is intersect / union, weighed.
% 7. multiple matching, select the one with best wScore.
Once more than one entity is kept, we match the argument to the entity with highest matching score,
% 8. tie breaker: count occurrence in freebase relations.
if there still has a tie, the most popular entity is selected. The popularity of an entity is calculated
by counting number of relations it has in Freebase.
% 9. SUTime is used to map years and datetime.
In addition, we use SUTime \cite{chang2012sutime} to recognize dates, and directly
map these arguments to a pre-defined virtual entity (Freebase has \textit{type.datetime}
representing dates and times, but it doesn't contain any specified entities).
% check other papers, learn how to introduce FB without too much words.
% 10. discard non-match to guarantee accuracy of linking.
If one argument fails to link to any entity, the corresponding relation tuple is discarded.
% Ranking Method May Change?
%   use wScore threshold to filter entities
%   then sorting by interLen, then popularity ??? (maybe we can have a try afterwards)
%


%\begin{table}[htbp]
%	\centering
%	\caption{Syntactic Transform Rules}
%	\begin{tabular}{|l|l|}
%		%\toprule
%        \whline
%		Category & Pattern Template \\
%		%\midrule
%        \hline
%        % Continuous Tense & \{$adv_1$\} \textbf{be} \{$adv_2$\} verb:VBG \{text\}
%        %                  & $verb_{lem}$ \{phrase\} \\
%		% Participle Tense & \{$adv_1$\} \textbf{have} \{$adv_2$\} verb:VBN \{text\}
%        %                  & $verb_{lem}$ \{phrase\} \\
%        % Participle + Passive & \{$adv_1$\} \textbf{have} \{$adv_2$\} been \{text\}
%        %                      & is \{text\} \\
%        Continuous Tense & \textbf{be} verb:VBG \{phrase\} \\
%		Participle Tense & \textbf{have} verb:VBN \{text\} \\
%        Future Tense & \textbf{will}/\textbf{shall} verb:VB \{pharse\} \\
%                     & \textbf{be} going to verb:VB \{phrase\} \\
%		%\bottomrule
%        \whline
%	\end{tabular}%
%	\label{tab:synt rules}%
%\end{table}
