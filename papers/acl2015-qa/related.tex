\section{Related Work}
The {\tt ReVerb} OpenIE system \cite{fader2011identifying} identified a large number of
binary relation textual patterns from the Web using syntactic and lexical constraints, and then extracted pairs of arguments for each
relation phrase. Then a logistic regression classifier is used to compute a confidence score for each tuple. The large pattern collections {\tt ReVerb} provided benefited many information retrieval related problems like Question Answering (QA) systems \cite{berant2013semantic,cai2013large}. {\tt NELL} \cite{carlson2010toward} is anothver similar system, where the relations are from a closed class, but the names of arguments are open. Comparing to {\tt ReVerb}'s constraints \cite{fader2011identifying}, {\tt PATTY} \cite{nakashole2012patty} can learn arbitrary patterns, and constructed a {\tt WordNet}-style taxonomy of binary relations. However, neither of them exploited deep syntactic analysis or clustering techniques. \newcite{moro2013integrating} presented a semantic network called {\tt WiSeNet} which clustered synonymous relational phrases into relation synsets, and assigned semantic classes to the arguments of these synsets using deep syntactic and semantic techniques.
\newcite{wang2014semantic} also proposed an effective multi-level clustering procedure in grouping synonymous relations.
However, our objective is more focused on conceptualizing relations through computing selectional preference on both sides of arguments.

Selectional preference learning is a broadly applicable NLP task, which usually focus on word-to-class relations. Previous work like \newcite{ritter2010latent} proposed a topic model method {\tt LDA-SP} for selectional preference to compute the probability of a binary relation taking both sides of specific arguments. \newcite{seaghdha2010latent} also proposed a series of LDA-style models using several grammatical relations. These works leans a distribution over topics for each relation while grouping related arguments into these topics in the mean time. Other works performed selectional preference in different corpus \cite{agirre2002integrating,judea2012concept} like {\tt Wikipedia} and {\tt WordNet} also inspired our work.

In our distinctive work, we try to give a ranking list of preferred argument types which are linked in {\tt Freebase} \cite{bollacker2008freebase} for each relation in {\tt ReVerb}.

% \KQ{Check "Semantic Parsing on Freebase from Question-Answer Pairs", one part mentioned typed relations} 