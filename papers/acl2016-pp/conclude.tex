\section{Conclusion}
This paper mines the equivalence between natural language relations
and structured knowledge known as schemas. It generalizes
the simple path representation by adding constraints on variables
along the path, to support more complex relations.
The experiments show that the schema representation
significantly improves results on knowledge base completion and QA,
when complex NL relations are involved, and performs as well as the
state-of-the-art methods on ordinary relations.
To the best of our knowledge, this is the first attempt
to model complex NL relations in knowledge bases.
Future direction of this research includes schema inference from
small data sets, efficient generation of
more complex schemas, including comparative and aggregative constraints, 
and the exploration of other features during
schema weighting.

