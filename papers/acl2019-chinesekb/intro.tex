\section{Introduction}
\label{sec:intro}
%\KZ{Instead of using ``translator'', which may have other meanings, consider
%using ``bilingual lexicon'' throughput the paper.}

Over the past decade, great efforts have been made to construct knowledge graphs manually or automatically to facilitate natural language understanding and artificial intelligence.
However, most of these efforts were spent on the English language. 
For example,
\con~\cite{Speer2012a} is a large multilingual common sense knowledge graph with 32,755,210 edges and more than 40 relations, 
which is constructed by crowdsourcing. There are 11,459,670 edges containing at least one English concept and only 1,169,238 
edges containing at least one Chinese concept, accounting for only 1/10 of the former\footnote{These statistics come from the latest \con 5.6}.
\pro \cite{Wu2012} is a universal and probabilistic taxonomic knowledge graph consisting of 20,757,545 ``IsA'' pairs, 
which are extracted automatically from a corpus of 1.68 billion English web pages. 
Linguistic resources, especially knowledge graphs for other languages, such as Chinese, are scarce. 
Therefore, alleviating this imbalance problem is valuable and urgent.

One straightforward way to solve this data resource imbalance problem is to construct 
Chinese knowledge graphs in the same way as building English knowledge graphs, 
but this is impractical. 
For example, if we want to build a taxonomic knowledge graph like \pro, 
we probably need billions of Chinese web pages, which is difficult to obtain. 
Instead of building such a universal taxonomic knowledge graph, some people have 
already tried to build the taxonomic knowledge graphs from online-encyclopedia, 
such as CN-Probase \cite{Xu2017} and zhishi.me \cite{Niu2011}. 
However, they suffer from two critical problems. 
First, their concept spaces are always limited by the original input data, 
while the source texts of Probase 
are more universal, freer, and richer than these encyclopedic texts, see the example in Section \ref{sec:evaluation}. 
Second, they have no probabilistic characteristic, which is very useful in some applications \cite{Cui2016,Song-ijcai-2011}.
Similarly, if we want to construct the Chinese common sense knowledge graph like \con, we also need to spend lots of human efforts, which is too expensive to implement.


In this paper, we propose a method to quickly obtain the corresponding Chinese knowledge 
graphs of two well-known datasets with high quality.  
We choose these two datasets \con and \pro because they are 
fundamental and useful in providing machines the ability of common sense reasoning 
and conceptualization. 
For example, common sense knowledge has been used in many downstream tasks, such as textual entailment \cite{dagan2010recognizing,bowman2015large} and visual recognition tasks \cite{zhu2014reasoning}.
Taxonomic knowledge has also been leveraged in many downstream applications, 
such as taxonomy keyword search \cite{ding2012optimizing}, 
semantic web search \cite{wang2010toward}, 
short text understanding \cite{song2011short}, 
and web table understanding \cite{wang2012understanding}.  
Generally, our approach includes two steps. 
First, we translate these English datasets into Chinese with the existing machine translator\footnote{Here, we use \url{https://translate.google.com}}.  
However, such a direct translation can not handle the non-trivial word sense disambiguation properly. 
For example, (``date'', IsA, ``fruit''), (``can'', AtLocation, ``shelf'') will be 
incorrectly translated into (``日期/dateline'', IsA, ``水果/fruit''), 
(``可以/could'', AtLocation, ``货架/shelf''), respectively. 
The reason may be that almost all present translators are fundamentally based on statistics and they prefer word sense pairs that appear at a higher co-occurrence frequency.
To demonstrate this, we search the Chinese word sense pair (``时间/dateline'', ``水果/fruit'') in Google, showing 197,000,000 results, while search pair (``枣/jujube'', ``水果/fruit''), showing only 15,200,000 results, accounting for less than 1/10 of the former. 
It explains why translators prefer to incorrectly translate pair ( ``date'', ``fruit'') into (``时间/dateline'', ``水果/fruit''), which has a higher co-occurrence frequency.
Second, to handle this word sense disambiguation problem better, 
we introduce a semantic-based method to revise the disambiguation errors to improve the translation quality.
Basically, we try to find the most semantically similar Chinese word sense pair, 
using word embedding \cite{Mikolov_nips_2013}, as the translation result, see detail in Section \ref{sec:approach}. 
In order to test the quality of our knowledge graphs(\zhcon and \zhpro), we evaluate their coverages and accuracies in Section \ref{sec:evaluation}.