\section{Problem Formulation}
In few-shot relation classification, there is a training set $D_{train}$ and a test set $D_{test}$. Each instance in both sets can be represented as a triple $(s,e,r)$, where $s$ is a sentence of length $T$, $e=(e_1, e_2)$ is the entities and $r$ is the semantic relation between $e1$ and $e2$ conveyed by $s$.
$D_{train}$ and $D_{test}$ have disjoint relation set, i.e if a relation appears in a triple in the training set, it must not appear in any triples in the test set and vice versa.
$D_{test}$ is further split into a support set $D_{test-support}$ and a query set $D_{test-query}$. A model is asked to predict the relations of instances in $D_{test-query}$ given $D_{test-support}$ and $D_{train}$.
In a $N$ way $K$ shot few-shot relation classification scenario, $D_{test-support}$ contains $N$ relations and $K$ instances for each relation. Both $N$ and $K$ are supposed to be small(e.g 5 way 1 shot, 10 way 5 shot). Few-shot relation classification is hard because of too few instances in $D_{test-support}$(totally $N\times K$ instances).
The task becomes tougher if the size of $D_{train}$ is also limited.