
\begin{abstract}
% Teaching deep learning models commonsense knowledge is a crucial step towards building human-level
% artificial intelligence. Recent benchmarks access model's ability of applying commonsense knowledge
% from \textit{generative} or \textit{discriminative} point of views(e.g., $\alpha$\textit{NLG} and $\alpha$\textit{NLI}
% from $\mathcal{ART}$ benchmark~\citep{Bhagavatula2020Abductive}). We argue that commonsense generation task require the same ability as its corresponding commonsense
% reasoning tasks, that is applying the underlying commonsense knowledge under the given context.
% However, there is still little work that combine these two tasks together. To bridge this gap, 
% we proposed $\mathcal{COSER}$\footnote{Code available in \url{https://Anonymous.com}}, a commonsense 
% generation model instructed by expert model with commonsense knowledge. We fullfill this intuition by representing
% the optimal desired generative model as an Energy-Based Model and then perform training through a
% distributional gradient policy algorithm. 
% Experiment results have empirically demonstrated the effectiveness of our proposed model. We also conduct
% experiments to analyze how the expert model affect and improve the generations.

Abductive Commonsense Reasoning($\mathcal{ART}$) is a benchmark that investigates model's ability on
inferencing the most plausible explanation within the given context. 
$\mathcal{ART}$ consists of two datasets,
$\alpha$\textit{NLG} and $\alpha$\textit{NLI}, that challenge models from \textit{generative} 
and \textit{discriminative} settings respectively. Despite the fact that both of the datasets 
investigate the same ability, existing work solves them independently. In this work, 
we address $\alpha$\textit{NLG} in a teacher-student setting by getting help 
from another well-trained model on $\alpha$\textit{NLI}. 
We fullfill this intuition by representing the desired optimal generation model 
as an Energy-Based Model and training it using a reinforcement learning algorithm.
Experiment results have demonstrated the effectiveness and feasibility of our 
model\footnote{Code available in https://Anonymous.com}.

\end{abstract}
