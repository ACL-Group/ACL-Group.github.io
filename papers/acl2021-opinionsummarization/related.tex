\section{Related Work}
\label{sec:related}

Multi-document summarization~\cite{abs-2011-04843} is used for generating an informative summary of multiple topic-related texts, including 
%summarization on 
news\cite{FabbriLSLR19}, emails~\cite{ZajicDL08}, Wikipedia articles generation~\cite{LiuSPGSKS18} and so on. Opinion summarization~\cite{GeraniMCNN14} is also a typical multi-document summarization problem
It aims at generating 
%a concise and coherent 
a summary covering the salient opinions %for a product or service 
of multiple reviews. It inherently has a special focus on aspects of the product or service, making it different from other tasks.

Opinion summarization suffers from a lack of training pairs. 
Some work~\cite{MeanSum19} uses auto-encoders to train the model 
by reconstruction loss and similarity loss. 
Others construct synthetic datasets for supervised training. 
The most intuitive way 
%proposed by some approaches
~\cite{Copycat20, Fewshot20} is to 
regard one review for a product as the summary 
and take the rest as input. 
However, the quality of such dataset is limited by the biased reviews, 
which cannot be summarized from others. 
So, further works~\cite{Plansum20} take the nearest neighbors as inputs based on review representations.
Adding noise to the sampled summary and taking the disturbed ones as the input reivews is another way to generate synthetic datsets. \citet{Denoise20} adopted two noising aspects: the segment noising at the token and chunk level, and document noising by replacing the whole review by a similar one. 
Different from their simple replacing, removing and inserting operations, 
our proposed approach takes advantage of the characteristics of reviews. 
We add noise for opinion-aspect pairs, and yield better performances.

Although above mentioned work mainly focused on data construction and ignored the characteristics of reviews, aspects and opinions are quite important for opinion summarization~\cite{MukherjeePVGBG20}.  \citet{AngelidisL18} tried to extract aspect-specific opinions by out-of-review knowledge. 
\citet{TianY019} classified the words into three types, including aspect, opinion and context and predict the work type as a first step. Some work~\cite{OpiDig20} also uses opinion-aspect phrases for filtering information in multiple reviews and transform the task into single document summarization. Besides, previous work~\cite{Plansum20} shows that project each review into aspect and sentiment distributions can also help. However, all of these work neglect that some detailed other information are left after the opinion-aspect extraction. Thus, besides opinion-aspect pairs noising, we add implicit sentences noising, leading to more comprehensive summaries.  
