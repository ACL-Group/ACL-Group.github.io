\section*{Limitations}
Some limitations of the proposed methods remain to be addressed 
in future research. 

First, our experiment settings assume that the desired attributes are available for generation, which would require a separate dialogue policy to decide the 
attribute label provided to the model. Therefore, our model cannot be 
directly applied to end-to-end dialogue models, and may also be affected by 
the potential error propagation from the dialogue policy model. Since the intended use of DASC is to serve as a component of pipeline-style dialogue systems, these common issues in such systems are out of the scope of this work. 
% \KZ{This is only useful in the scenario that the attribute actually changes
% its value dynamically. We can argue that this is out of the scope of
% this work, and that our work can server as a component in this kinda of
% systems.}

Moreover, we require attribute-annotated dataset to train the model. Therefore, we may not be able to train an effective model in scenarios where attribute labels are scarce or hard to solicit.

Last but not least, DASC is not directly applicable for controllable generation with free text as control signal, such as persona descriptions \cite{zhang2018personalizing}, which might limit its application range, though we may simply combine DASC with other techniques like concatenating the descriptions to achieve this goal, which will require further explorations. 
