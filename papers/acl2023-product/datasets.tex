\section{Dynamic Multi-Domain Datasets}
\subsection{Static Multi-domain Datasets}
We select 3 business lines from our e-commerce platform: 
QuickDelivery (QD, targeting fast delivery), BargainHunters (BH, targeting low price), FreshGrocery (FG, targeting fresh vegetables). 
These data instances 
% including product titles and category labels 
are collected from the real-world business, where the product titles are mostly assigned by sellers from the platform and the category labels stem from three pre-defined business taxonomies. 
% Both product titles and category labels are in Chinese. 
We recruit experienced annotators to manually classify the products $X_i$ into assorted categories $y_i$, with 1\% sampling to guarantee annotation accuracy. 
Data groups with over 95\% accuracy in quality checking are used in our final datasets. Meanwhile, $X_i$ is tagged with concepts $\{\lambda_k\}$ following the Appendix~\ref{sec:datasetdetails}.
% \footnote{We attach a subset of sampled data in the data appendix, and full data would be prepared and released once our work is published.}
% Examples in \tabref{tb:exa} are translated from Chinese.

\begin{table}[th]
\small
\centering
% \setlength{\tabcolsep}{5.2pt}
\begin{threeparttable}[b]
  \caption{Statistics for multi-domain datasets}
  \label{tb:dataset}
  \begin{tabular}{l|cccc}
    \toprule
    Dataset & \# training  & \# test  & \# classes & depth  \\
    \midrule
    \small{QD} & 99k & 11k & 1987 & 3   \\
    \small{BH} & 31k & 5k & 2632 & 4  \\
    \small{FG} & 28k & 3k & 1065 & 4  \\
    \bottomrule
  \end{tabular}
  \begin{tablenotes}
    \item[1] \# classes: the total distinct leaf nodes.
    \item[2] depth: the depth of categorical taxonomy tree.
  \end{tablenotes}
  \end{threeparttable}
\end{table}

% \begin{table}[th]
%   % \setlength{\tabcolsep}{2.5pt}
%   \caption{Examples from the three Datasets}
%   \label{tb:exa}
%   \centering
%   \begin{tabular}{c|c|c}
%     \toprule
%      Data & Product title  & Taxonomy path \\
%      \midrule
%      QD & \tabincell{c}{\textit{Towel gourd 1 pcs} \\ \textit{\& soy bean 150g}} & \tabincell{c}{\small{Vegetable} $\rightarrow$ \small{Mixed Product} \\ $\rightarrow$  \small{Vegetables mixture}}\\
%      \midrule
%      BH & \tabincell{c}{\textit{Fresh bamboo shoots} \\ \textit{(dig from mountains)}} & \tabincell{c}{\small{Vegetable/Fruit} $\rightarrow$ \small{Vegetable} \\$\rightarrow$ \small{Tubers} $\rightarrow$ \small{Bamboo}}\\
%      \midrule
%      FG & \tabincell{c}{\textit{Butter leaf lettuce 100g}} & \tabincell{c}{\small{Fresh} $\rightarrow$ \small{Vegetable} $\rightarrow$ \\ \small{Leaf} $\rightarrow$ \small{Lettuce}}\\
%     \bottomrule
%   \end{tabular}
% \end{table}

Statistics of 
% these 3 product categorization 
three datasets are listed in \tabref{tb:dataset}. 
% which reveals the large scale of classes in each taxonomy, 
% and different taxonomy trees may vary in depths, which poses a bigger challenge for multi-domain knowledge sharing. 
Each sample in the three datasets has exactly one ground truth category.
Varied class numbers and hierarchy depths of different taxonomies pose bigger challenges for multi-domain knowledge sharing.

\subsection{Dynamic Test Set}
To verify the generalizability of $\mathsf{TaLR}$ on zero-shot scenarios, we further construct two \textbf{taxonomy evolving} derivatives of the QD test set. 
(ii) QD-$integrate$: 
% the original category nodes are renamed into new ones. There are 1371 samples in this subset, with 127 classes before and \TODO{127} classes after renaming.
During a production business adjustment, 127 classes in the original taxonomy are integrated or replaced by similar categories, which affects 1371 samples in the original test set to form this subset. 
% Products map their original categories with integrated ones.
(i) QD-$divide$: 
22 category nodes from the original QD taxonomy are divided into two or more nodes.
% to simulate the deletion of old nodes and increment of new nodes. 
495 samples in the original test set suffer from this evolution.
% , and we assign them with suitable new categories within divided ones using some heuristics. 

\subsection{Meta Concept Set}
\label{sec: dataset}
Beyond the category labels, each product title is associated with a list of meta concepts from a set $\mathcal{M}$ including over $30k$ entities covering the most fine-grained concepts in product titles. 
% $\mathcal{M}$ is constructed in semi-supervised manners with hybrid sources, like recognized name entities, queries from users, or rules from experts.
The tagging step $X \rightarrow \{\lambda_1, \lambda_2, ... \lambda_k\}$ is accomplished by an industrial Label Tagging System that exploits heterogeneous approaches.
% including text sequence labeling, classification, literal matching and some expert-defined rules. 
Details are in Appendix~\ref{sec:datasetdetails}.
