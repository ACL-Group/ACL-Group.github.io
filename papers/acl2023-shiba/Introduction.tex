\section{Introduction}
\label{sec:intro}

It has long been an interesting interdisciplinary scientific challenge to 
understand the languages of animals. Dogs, who are arguably the best friends
of humans, have drawn particular attention. Learning what dogs want to 
express has broad and profound significance, such as for understanding 
biological evolution\cite{pongracz2017modeling}, for applying their languages 
to information technology, or sometimes just for satisfying the
our curiosity.

%这里插一个引用关于“voice is one of the main communication ways of dogs"的论文
Dog voices, which are their chief means of communication, have been
studied previously. It has been shown that dogs can recognize the scenes 
and express their understandings of the outer world as well as their 
inner states by their voices~\cite{molnar2008classification, hantke2018my}. 
\KZ{Rephrase: These works, however, have certain limitations on 
their methods and datasets 
for people to apply their results into the task of understanding dogs, 
or in other words, match barks of dogs with their corresponding meanings.} 
On the one hand, previous research treats this task as a simply 
classification problems, which means that a piece of audio containing barks 
will be straightforwardly sent into one model to get a class label along a
particular dimension such as emotion (happy or sad). 
The potential linguistic patterns beyond the dog's vocal expressions 
were totally ignored. 
On the other hand, the datasets they used were collected 
by recording the voices of dogs under certain controlled environments.
Such methodology is costly in practice, and the data thus produced is
is limited in size and variety. \KZ{Give some evidence, in terms of
size and variety?}


\KZ{include an intro picture to show our aim.}

Even though it is still highly debatable whether animals, or dogs in this
case, have languages at all, in this paper, we take the approach of treating
the voices of dogs as a kind of rare human language. 
\KZ{Rephrase this: Between the audios and meanings, 
we can get texts as the bridge.} 
In this paper, we provide with a dataset of symbolic transcripts of Shiba Inu 
dog barks~\footnote{Here we refer to ``barks'' in its broadest sense, which
includes any vocal expressions coming from a dog.} called \textit{ShibaScript}, 
which ameliorates some of the aforementioned challenges. 
We pick Shiba Inu as the subject because it is a popular breed around the world
and there is a large amount of their videos on the Web. 
In the meantime, we have done preliminary syntax analysis on this dataset. 
We believe that it can bring much support for further research in 
this pioneering field.

\subsection*{The ShibaScripts Dataset}
\KZ{I think no need to have subsections in the intro to save space.}
ShibaScripts is our dataset which contains barks of X different Shiba Inu dogs as well as the transcripts with timestamps of these barks. These X dogs are respectively from X Japanese families who post these dogs' videos on YouTube. The dataset has a total length of X.X hours, transcripts of X.X sentences and X.X words. The syllables in the transcripts are in total 9 types.



This dataset is believed to help those who are interested in learning what dogs want to expression.

\subsection*{Contributions}

Our contributions lie in three aspects: (1) we introduce a reusable framework for transcribing animal voices from social media like YouTube; (2) we release a novel dataset for other researchers to make use of it on their own researches; (3) we have done some preliminary statistical findings from the dataset.
