\section{Analysis}

We present preliminary statistical findings from ShibaScript, including lexical analysis and transcribing accuracy evaluation.
% \KZ{rewrite this preamble, it's irrelevant now: To give a detailed picture about our dataset ShibaScript and also show several statistical syntax results on it, we will explain our analysis below.}

% This is the analysis on this dataset. Inlcuding its distribution and some further analysis inlcuding bigrams.

% \subsection{Source Information}
% \label{sec:sourceinformation}
% The 16 dogs come from 16 different Japanese user accounts at YouTube. From the videos uploaded from the specific user and the captions attached to these videos, we can get to know much about the growing up environments of dogs, which will influence the expressions of these dogs to some degree. %The environments of them are in \ref{tab:sourceinformation}.
%这是狗的来源的信息,包含地区、是否家养、之类的

% \begin{table}[th]
% \centering
% \begin{tabular}{c|c|c|c}
% \hline
% \textbf{Dog Index} & \textbf{Color} & \textbf{Other Pets} & \textbf{Area}\\\hline
% 0 & yellow & \\
% 1 & yellow & \\
% 2 & yellow & \\
% 3 & yellow & \\
% 4 & yellow & \\
% 5 & yellow & \\
% 6 & yellow & \\
% 7 & yellow & \\
% 8 & yellow & \\
% 9 & yellow & True \\
% 10 & yellow & \\
% 11 & yellow & \\
% 12 & yellow & \\
% 13 & yellow & \\
% 14 & yellow & \\
% 15 & black & \\\hline
% \end{tabular}
% \caption{The source information of dogs in ShibaScript. Color means the fur color of the dog. Other Pets means whether the dog is kept with a family with other animals. Area means the living locations of the family.}
% \label{tab:sourceinformation}
% \end{table}



% \begin{table}
% \centering
% \begin{tabular}{c|c|c|c|c}
% \hline
% \multirow{2}{*}{\textbf{DogID}} & \multicolumn{2}{c|}{\textbf{Sentence}} & \multicolumn{2}{c}{\textbf{Word}}  \\
% \cline{2-5}
% {} & \textbf{Num} & \textbf{Length(s)} & \textbf{Num} & \textbf{Length(s)} \\
% \cline{1-5}
% 0 & 346 & 1107.67 & 577 & 363.36\\
% 1 &  158 & 514.24 & 241 & 129.00\\
% 2 & 553 & 1643.00 & 847 & 469.20\\
% 3 &  55 & 171.69 & 123 & 65.52\\
% 4 & 56 & 224.00 & 107 & 94.08\\
% 5 &  115 & 374.58 & 217 & 98.52\\
% 6 & 52 & 157.00 & 77 & 47.72\\
% 7 & 40 & 135.00 & 65 & 27.44\\
% 8 &  135 & 566.03 & 316 & 143.68\\
% 9 & 255 & 795.00 & 408 & 157.08\\
% 10 & 1188 & 4306.00 & 2029 & 1372.52\\
% 11& 188 & 570.94 & 320 & 203.20\\
% 12  & 130 & 562.28 & 257 & 147.00\\
% 13  & 993 & 2930.19 & 1719 & 749.72\\
% 14  & 118 & 350.11 & 324 & 144.76\\
% 15  & 87 & 299.88 & 134 & 101.24 \\\hline
% sum & \textbf{4469} & \textbf{14707.61} & 7761 & 4314.04\\\hline
% \end{tabular}
% \caption{The basic statistical information of ShibaScript.}
% \label{tab:datasetinformation}
% \end{table}




\subsection{Lexical Analysis}
During the transcribing, there are in total 11 types of tokens, in which 9 types are phonetic symbols~(\tabref{tab:alphabet}), the other two are short pauses between words and long pauses between sentences. 

Similar to humans, the length of these tokens contain ample information. The exact lengths of tokens are kept in ShibaScript for concrete analysis. Because the long pauses are largely determined by the scene at that time, the numerical analysis of it will not be included here. 


\begin{table}[th]
\centering
\small
\begin{tabular}{c|c|c}
\hline
\textbf{Symbol} & \textbf{Mean len (s)} & \textbf{Variance (s)}\\
\hline
\verb|[au]| & 0.35 & 0.022 \\
\verb|[a]| & 0.35 & 0.017 \\
\verb|[^]| & 0.34 & 0.017\\
\verb|[u:]| & 0.45 & 0.054\\
\verb|[u]| & 0.35 & 0.030\\
\verb|[i]| & 0.33 & 0.020\\
\verb|[k]| &  0.24 & 0.009\\
\verb|[(w)au]| & 0.34 & 0.018\\
\verb|[en]| & 0.36 & 0.032\\
\verb|short pause| & 0.57 & 0.335\\
\hline
\end{tabular}
\caption{The mean and variance of the duration of 9 phonetic symbols 
and short pauses between words.}
\label{tab:tokenanalysis}
\end{table}

The mean and variance of each token length can be seen in~\tabref{tab:tokenanalysis}. In which we find that almost every phonetic symbol has a similar length of 0.35s or so. Except for the phonetic symbol \verb|[u:]|, which is a prolonged sound owning an average length of 0.45s. While phonetic symbol \verb|[k]| is a relatively short-lived symbol, only having 0.24s average length.

Considering the monogram~(\figref{fig:monogram}) of ShibaScript, we can find that the most frequent symbol is \verb|[en]|, which reaches to 3478 times in ShibaScript, the following two are \verb|[au]| and \verb|[a]|, which reaches  1981 and 2011 times respectively. One of the reasons why \verb|[en]| exceeds much, which is counterintuitive, is that symbols such as \verb|[a]|, \verb|[au]|, \verb|[(w)au]| are divided up. The least frequent symbol is \verb|[k]|, which only reaches 15 times. This is because dogs seldom produce air-sounds like \verb|[k]|.

At the same time, we can find that these phonetic symbols exist in multiple dogs' sounds, showing that these 9 symbols are universal.

\begin{figure}[th]
\centering
\scalebox{0.32}{\includegraphics{monogram.pdf}}
\caption{The occurrences of each monogram. The blue bars show the occurrences across the whole dataset of each monogram in ShibaScript, the green lines show the numbers of dogs producing the symbols, from 1 to 16.}%\KZ{Remove ``Monogram Stats'' from the pic, since
% you already talk about it in the caption.}
\label{fig:monogram}
\end{figure}

%静音片段分析
%unigram, bigram分析
After analyzing the monogram, we come to find the relationship between symbols, as well as the bigram~(\figref{fig:bigram}) of ShibaScript. Among these bigrams, several appear extremely frequently. It shows a possibility that they are associated with some common semantic meanings. We will dive into that in the future works. Due to space constraints, the detailed information of bigram is shown in~\secref{sec:appendixc}.

% 1 
\begin{figure}[th]
\centering
\scalebox{0.32}{\includegraphics{bigram.pdf}}
\caption{The occurrences of each bigram. The blue bars show the occurrences across the whole dataset of each monogram in ShibaScript, the green lines show the numbers of dogs producing the symbols, from 1 to 16.}
\label{fig:bigram}
\end{figure}




\subsection{Accuracy of Transcription}
In this paper, we discover the certain phonetic pattern of Shiba Inu dogs and assign a vocal dictionary of 9 symbols, which is a first-step trial in this area. To better evaluate the phonetic symbols set as well as the integral accuracy of our transcribing, we have done an evaluation test on these two aspects. The evaluation metric is 5-level Mean Opinion Score~\cite{viswanathan2005measuring}. Three raters will give scores to either one syllable or one word according to~\tabref{tab:ratestandard}. %\MY{How many raters?}


\begin{table}[th]
\centering
\small
\begin{tabular}{c|l}
\hline
\textbf{Score} & \textbf{Description} \\
\hline
5 & The label exactly matches up.\\
4 & Some difference exists between the\\
{}& label and the sound. Humans are s- \\
  & ometimes hard to distinguish.\\
3 & Difference exists between the label\\
  & and the sound. Humans can tell the \\
  & difference immediately.\\
2 & Although the label is obviously wr-\\
{}& ong, there is similarity between t-\\
{}& he label and the sound.\\
1 & The label is totally wrong. \\
% 5 - Excellent 完全一致,失真程度不可察觉
% 4 - Good 有轻微不一致,失真程度略可察觉,如a和ao, u和u:,(w)au 和au,人耳有时也会难以分辨其中差异
% 3 - Fair 一般,失真程度可察觉,人耳可以轻松分辨出二者差异
% 2 - Poor 差,失真程度尚可接受,可以找到一丝相似
% 1 - Bad 很差,失真程度难以接受, 完全不同
\hline
\end{tabular}
\caption{The evaluation metric of rating, which is similar to MOS in speech synthesis evaluation metric.}
\label{tab:ratestandard}
\end{table}

\subsubsection{Phonetic Symbol Accuracy Evaluation}
\label{sec:phone_eva}
%在音素层面进行的evaluation
For each syllable category, we select 50 syllables randomly. The rating result is in~\figref{fig:phoneticaccuracy}. The Fleiss Kappa~\cite{kilicc2015kappa} between three annotators is 0.609.

\begin{figure}[th]
\centering
\scalebox{0.30}{\includegraphics{phoneticaccuracy.pdf}}
\caption{The evaluation result of 9 phonetic symbols. }
\label{fig:phoneticaccuracy}
\end{figure}

% \KZ{Fonts too small, and the image is not clear. Every image must be critical clear after ampification. Use vector images (not bitmap) whenever possible.}

\subsubsection{Word Accuracy Evaluation}
%在word层面进行的evaluation
For the word accuracy evaluation, we select 30 words for each dog randomly and find the same person who rates for phonetic symbols to score for them. The rating result is in~\figref{fig:wordaccuracy}. The Fleiss Kappa between three annotators is 0.516.

\begin{figure}[th]
\centering
\scalebox{0.29}{\includegraphics{wordaccuracy.pdf}}
\caption{The evaluation result of words for 16 different dogs.}
\label{fig:wordaccuracy}
\end{figure}
