\section{Related Work}
\label{sec:related}
% {what do u mean? reached to a rounded system like human languages}
Early works on understanding animal communications have never reached a point of maturity, which have direct connections between their vocal or literal expressions and their meanings. In these works, researchers attempted to interpret animals in a certain aspect through classifications. Among animals, dogs are popular as research subjects.  Considering their vocal expressions, these researches can be divided into mainly three kinds: activity understanding~\cite{ide2021rescue, ehsani2018let, molnar2008classification}, emotion understanding~\cite{hantke2018my, paladini2020bark} and individual understanding~\cite{larranaga2015comparing}. 

The situation above comes from two reasons. The first is that we are short of ample dataset related to the expressions of dogs, and the second reason is that we have never mastered, or seldom investigated the language patterns of dogs.

%dataset
In some datasets~\cite{parkhi2012cats, iwashita2014first, abu2016youtube} related to visual information of dogs, abundant data was collected from the Internet, which saved the cost and made the data extensible. Compared to that, previous vocal-related datasets depended on manual recordings, which limits the context and costs a lot. Given this, a thought is that we can utilize data on the Internet when collecting vocal-related data if we design a systematic process to extract useful fragments.

%classification
% \KZ{Need to cite some more papers here.
% The references section is a bit short.}
In the meantime, previous research adopted a straight-forward classification method, thus lacked enough investigation into the potential sound patterns of dogs. While lexical analysis~\cite{yule2022study} is the fundamental step for language processing, another thought is that we can set up an own  ``alphabet'' for dogs and transcribe barks of dogs into readable tokens for further research.