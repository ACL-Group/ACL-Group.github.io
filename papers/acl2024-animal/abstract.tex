\begin{abstract}

This paper delves into the pioneering exploration of potential communication patterns within dog vocalizations and transcends traditional linguistic analysis barriers, which heavily relies on human priori knowledge on limited datasets to find sound units in dog vocalization. 
We present a self-supervised approach with HuBERT, enabling the accurate classification of phoneme labels and the identification of vocal patterns that suggest a rudimentary vocabulary within dog vocalizations. Our findings indicate a significant acoustic consistency in these identified canine vocabulary, covering the entirety of observed dog vocalization sequences. 
%We use this approach to undercover phonemes and mine vocabulary of dogs. 
We further develop a web-based dog vocalization labeling system. This system can highlight phoneme n-grams, present in the vocabulary, in the dog audio uploaded by users.
%This approach can be simply applied to find other non-human language sound units and is valuable for further research on dog language understanding.

\end{abstract}