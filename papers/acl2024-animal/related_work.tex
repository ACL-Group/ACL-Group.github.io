\section{Related Work}

To decode a dog's language, it is necessary to analyze its basic sound units, linguistic structure, lexicon, meaning, etc. 
The past few years have seen a surge of interest in using machine learning (ML) methods for studying the behavior of nonhuman animals~\citep{rutz2023using}. 
Much of the past work has mostly focused on dog behavior~\citep{ide2021rescue,ehsani2018let} and the meaning of dog sounds~\citep{molnar2008classification,hantke2018my,larranaga2015comparing,hantke2018my,pongracz2006acoustic}. 
Most of them only classify the audio of the dog sounds in multiple categories, including activities, contexts, emotions, ages, etc.
They did not study the sound units of the dog's language.

The above work likewise illustrates the existence of multiple distinct sound units in dog language. Many of the species that appear to use only a handful of basic call types may turn out to possess rich vocal repertoires~\citep{rutz2023using}. Many works demonstrate the diversity of animal sounds~\citep{paladini2020bark,robbins2000vocal,bermant2019deep}. \citet{huang2023transcribing} and \citet{wang2023towards} did a fine-grained study of dog sound units. They directly using a priori knowledge of human language may be inapplicable.
We use HuBERT~\citep{hsu2021hubert}, a self-supervised approach, to find the sound units of dog language, and future research will find the meaning of sound units.
