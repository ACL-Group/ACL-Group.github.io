\begin{abstract}

Whether animal vocal communications constitute a ``language,'' which is
characterized by fixed linguistic patterns is a long-standing scientific 
curiosity. In this paper, we present a web-based interactive
canine language lexical analysis system which automatically processes 
dog vocalizations using HuBERT,
transcribes them into sequences of distinct phonemes, and further segments
them into a sequence of words from a vocabulary discovered by statistic 
analysis.  These vocalizations are either extracted from YouTube videos
or uploaded by the users.
%dogs as our research subjects due to their close relationship with human society and the ease with which humans can infer their behavior. Through denoising, dog vocalization extraction, and self-supervised learning-based sound representation on a dataset obtained from YouTube, we ultimately obtained 50 dog bark phonemes through clustering algorithms. Furthermore, we introduced a plausibility score to rank phoneme combinations, resulting in a vocabulary most resembling words. 
Further the system allows to visualize dog ``sentences,'' in terms of 
the audio spectralgrams and the transcripts in both phonemes and words. 
%For easy understanding, the system also present the video fragment from which the sentence is extracted. 
This system is a first step toward canine language inference and understanding,
and can be used as a platform for verification of canine language processing 
algorithms as well as an annotation tool for understanding the semantics of
canine lexicons.  
\end{abstract}
