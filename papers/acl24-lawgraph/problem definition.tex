\begin{table}[]
    \centering
    \scriptsize

    
    \caption{A Prison Term Prediction example from CAIL2018~\cite{DBLP:journals/corr/abs-1807-02478}.}
    \begin{tabular}{p{0.2\columnwidth}p{0.7\columnwidth}}
    \toprule
       Description  &  Example\\
       \hline
       Case Fact  &  The defendant, Mr Yu, under the clear awareness of his inability to repay, impersonated as "Mr. Zhang" and signed a "Personal Consumption Trust Loan and Service Contract" with the China Foreign Economic and Trade Trust Co., Ltd., fraudulently obtaining a loan of RMB 40,000, which he later spent.\\
       Prison Term & 8 months\\
       \bottomrule
    \end{tabular}
    \label{tab: case example}
    \vspace{-1em}
\end{table}

\section{Problem Definition}

PTP aims to forecast the prison sentences for given legal cases accurately. Specifically, for a crime $c$, the training dataset $\mathcal{D}_{\text{train}}^{c}$ consists of $N$ pairs of documents and their ground truth prison terms, denoted as ${(F_1^{c}, T_1^{c}),(F_2^{c}, T_2^{c}), \ldots, (F_N^{c}, T_N^{c})}$. Here, $F_i^{c}$ represents the set of words describing facts from the $i$-th case, i.e., $F_i^{c}=\{w_{i1}^{c},\ldots,w_{i|F_i|}^{c}\}$, and $T_i^{c}$ is the prison term for that case. Table~\ref{tab: case example} shows an example of case facts and prison terms.
% Realistic Imprisonment Term Prediction (RITP) aims to predict the penalty time for each input case. Specifically, for a specific charge $C$, let $\mathcal{D}_{\text{train}}^{C}=\{(F_1^{C}, T_1^{C}),(F_2^{C}, T_2^{C}), \ldots,$ $(F_N^{C}, T_N^{C})\}$ be a dataset comprising $N$ documents and their corresponding penalty term, where $F_i$ and $T_i$ represent the $i$-th case fact and its imprisonment terms, respectively. Each case fact is represented as a set of words. 

For a specific crime $c$, the ideal result of PTP is that 
% the predicted prison terms $\hat{T}_i^{c}$ for each input case $F_i^{c}$ to minimize 
the sum of point-wise differences between the ground truth ${T}_i^{c}$ and the predicted prison terms $\hat{T}_i^{c}$, i.e., $\sum_{i=1}^N d({T}_i^{c}, \hat{T}_i^{c})$, is close to zero. Here, $d(\cdot,\cdot)$ is a distance function. For simplicity, we omit crime $c$ from all variables and equations in the remaining sections.

% For a specific charge $c$, the learning objective of PTP is to train a predictor to predict an imprisonment term $\hat{T}_i^{c}$ for each input case $F_i^{c}$ to minimize the sum of point-wise differences between the ground truth ${T}_i^{c}$ and the prediction $\hat{T}_i^{c}$, i.e., $\sum_{i=1}^N d({T}_i^{c}, \hat{T}_i^{c})$. Here, $d(\cdot,\cdot)$ is a distance function. For simplicity, we omit charge $c$ from all variables and equations in the remaining sections.

% Most previous studies regard PTP as a classification problem and employ cross-entropy as the distance function $d(\cdot,\cdot)$ in the learning objective~\cite{feng-etal-2022-legal,ML-LJP}. 
% However, this formulation overlooks the ordinal nature of penalty terms, treating two misclassifications as identical even when one is much closer to the ground truth value. Motivated by this, we regard PTP as a regression problem and use Mean Square Error (MSE) as the distance function. 



% However, the classification does not align with the judicial process, where the penalty term is related to the severity of unlawful actions. In reality, there is a huge difference in the severity between ten and fifteen years, which the previous works think the two terms belong to the same class. The classification problem setting violates the principle of \textit{lex talionis} (also known as the proportionality principle), which mandates that punishments should be appropriately matched to the nature of the offense, avoiding any discrepancy in severity~\cite{https://doi.org/10.1111/}. 
% To be accurate, in this paper, we design it as a regression problem and use Mean Square Error (MSE) as the distance function instead of the classification and cross-entropy. Besides, in the evaluation, we keep a classification metric to assess whether the prediction results align with the original statute's class of penalty. More details are discussed in Section \ref{es}.

% \textbf{Misaligned Problem Formulation}. Most previous works model ITP task as a classification problem~, merely predicting prison time on a coarse level. These works 