\section{Detailed Prompts}

\subsection{Psychotherapy CoT Prompts}
\label{apd:thought_prompt}
Here are the prompts mentioned in Section \ref{sec:cot_generation} for generating psychotherapy analysis of seeker's situation. For each turn, we concurrently focus on three different psychotherapy approaches and generate content tailored to the characteristics of each approach separately.

\begin{tcolorbox}[title = {Prompt for CBT CoP}]
\small
    You are an experienced psychologist. Now, I will provide you with a history of a psychological counseling dialogue. Please analyze the seeker's situation from the perspective of  Cognitive Behavioural Therapy, focusing mainly on the seeker's last statement. Please strictly follow the format below and keep it as concise as possible.
    \\
    
    [Cognitive Behavioural Therapy Analysis]

    *Event: <text>

    *Cognition: <text>

    *Behavior: <text>

    *Belief: <text>
\end{tcolorbox}

\begin{tcolorbox}[title = {Prompt for PCT CoP}]
\small
    You are an experienced psychologist. Now, I will provide you with a history of a psychological counseling dialogue. Please analyze the seeker's situation from the perspective of Person-Centered Therapy, focusing mainly on the seeker's last statement. Please strictly follow the format below.
    \\
    
    [Person-Centered Therapy Analysis]

    *Emotion: <text>

    *Self-Awareness: <text>
\end{tcolorbox}

\begin{tcolorbox}[title = {Prompt for SFBT CoP}]
\small
    You are an experienced psychologist. Now, I will provide you with a history of a psychological counseling dialogue. Please analyze the seeker's situation from the perspective of Solution-Focused Brief Therapy, focusing mainly on the seeker's last statement. Please strictly follow the format below and keep it as concise as possible.
    \\
    
    [Solution-Focused Brief Therapy Analysis]
    
    Seeker's State Assessment:
    
    *Goal: <Text>
    
    *Resource: <Text>
    
    *Exception: <Text>
    
    *Action: <Text>
\end{tcolorbox}



% \begin{tcolorbox}[title = {Prompt for SFT}]
% \small
% You will play the role of a mental health counselor. Your task is to generate a response to the seeker's last utterance.
% \end{tcolorbox}

\subsection{\textit{ChatGPT} Baseline Prompt}
\label{apd:baseline_prompt}
The following is the prompt we used for the \textit{ChatGPT} baseline for response generation.

\begin{tcolorbox}[title = {Prompt for \textit{ChatGPT} baseline}]
\small
    Please generate a response to the seeker's last sentence based on the context of the conversation.
    
    Here is the context:
    <context>
    \\
    
    Please respond to the seeker's last sentence coherently and smoothly as a counselor, maintaining a gentle attitude, avoiding repetition of previous remarks, and keeping it concise. Please strictly adhere to the following format for the output:
    
    counselor: <response>
\end{tcolorbox}

\subsection{Empathy Analysis Prompt}
As mentioned in Section \ref{sec:empathy}, we utilize this empathy analysis prompt to instruct GPT-4 to rate seven sets of responses generated by \textit{PsyMix}, along with all other baseline and ablation models, for each context individually. Within the prompt, we furnish comprehensive rating instructions for three dimensions of empathy and the preceding dialogue history.
\begin{figure*}
\begin{tcolorbox}[title = {Prompt for Empathy Analysis}]
\small
You are an expert in psychology. I will provide you with a history of a psychological counseling dialogue and need you to evaluate the empathetic ability of the counselor portrayed in generating responses.

Here are the scoring criteria. The evaluation of empathetic ability will be scored around three dimensions: emotional feedback, understanding, and exploration. Each dimension is set to a score of 1-3, where 1 represents the worst and 3 represents the best. Different responses can have the same scores, but there should be differentiation as much as possible.
\\

\textbf{Emotional Feedback} mainly reflects the warmth, sympathy, and concern expressed in the counselor's replies.
\begin{itemize}
    \item 1 point: No emotional feedback provided.
    \item 2 points: Expresses support but does not explicitly indicate emotions (e.g., everything will get better).
    \item 3 points: Shows empathy towards the seeker, specifically indicating emotions (e.g., I feel sorry for you).
\end{itemize}

\medskip

\textbf{Understanding} refers to the counselor inferring the seeker's feelings and experiences and expressing understanding.
\begin{itemize}
    \item 1 point: No expression of understanding.
    \item 2 points: Expresses understanding but without specific content (e.g., I understand how you feel).
    \item 3 points: Accurately and specifically indicates inferred content (e.g., you must have been very sad at that time) or shares similar experiences (e.g., I sometimes feel very anxious too).
\end{itemize}

\textbf{Exploration} refers to the counselor expressing interest in the seeker's experiences and feelings and gently probing.
\begin{itemize}
    \item 1 point: No interest expressed in the seeker's reply.
    \item 2 points: Expresses interest but in a general manner (e.g., what happened?).
    \item 3 points: Expresses a specific desire to explore some aspect of the seeker's experience (e.g., do you feel lonely now?).
\end{itemize}
Taking into account emotional feedback, understanding, and exploration, please score the response and explain the reasons. The output format is:
\ \\
Scoring Reasons: [Reasons]; 

Emotional Feedback: [Score]; 

Understanding: [Score]; 

Exploration: [Score]; 
\end{tcolorbox}

\end{figure*}