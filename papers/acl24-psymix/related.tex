\section{Related Works}
\label{sec:related}

Empathy holds significant importance in conversations, particularly in scenarios involving mental health support~\cite{liu-etal-2021-towards}. 
Some approaches~\cite{Saha2022Motivational,zheng-etal-2023-augesc} focus on implementing counseling strategies, such as Hill's Helping skill~\cite{Hill2009Helping} or strategies from Motivational Interviewing~\cite{Miller1995MI}, to make responses appear more empathetic. However, these strategies often fall short in fully grasping the client's situation. This phenomenon has been highlighted in previous study~\cite{lee-etal-2023-empathy, Sharma2021Empathy}. Merely using patterned statements like ``I can understand your feelings'', regardless of context, can result in high scores in current empathy benchmarks~\cite{sharma-etal-2020-computational}, which is not reflective of genuine empathy.
Some prior approaches have started to pay attention to understanding the client's perspective. They commonly propose a taxonomy of user behavior, where before generating responses, the dialog system identifies and categorizes the user's emotions~\cite{rashkin-etal-2019-towards}, behaviors~\cite{qiu2023psychat}, or intentions~\cite{Su2023EmpatheticDG}. Moreover, some works~\cite{Li2020KnowledgeBF} integrate external common-sense knowledge to better comprehend clients. However, this approach typically tackles only one aspect, especially emotion, and the categorization outcomes are bound by the taxonomy or knowledge graph, thus restricting scalability.