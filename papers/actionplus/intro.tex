\section{Introduction}
Today majority of the search services are keyword based,
which means the right information is hard to find if the query is
not properly formulated. The ability for search engine to understand
user's search intent is still very limited.
For example, when someone searches "country invade Iraq", 
the searching results might be limited to pages with 
high frequency of the keywords "country", "invade" and "Iraq". 
Other pages that don't contain the explicit occurrences of these
keywords are excluded, even though they may contain words 
such as ``war'' or ``conflicts'', which conveys similar idea. 
Our goal in this paper is to build a search engine that 
is ``aware'' of such semantic equivalence between an action concept
(defined as a verb with abstract arguments)
and a noun concept, and thus provides better search experience.

%\KZ{What are the challenges in discovering the semantic equivalence?
%Any previous attempts on this? summarization, wsd? 
%need more refs here.}

The main difficulties of this mapping lie in the ambiguity nature of human 
languages, and the external knowledge required for conceptualization. FrameNet \cite{baker1998berkeley}
aims to recognize variations of semantically similar short texts. But it does not generate
a human readable representation of the target sentence. Semantic Role Labeling(SRL)
\cite{palmer2005proposition}aims to label the semantic meaning of the arguments of a verb. For example, ``Joe chopped the watermelon with
a giant knife.'' Here Joe is the agent, watermelon is the patient and knife is the instrument.
But in our work we aim to find out the concept of each argument, so in this example
Joe would be a person, watermelon would be a fruit and knife would be a tool or weapon.
But the roles it could find are usually too generalized and limited, which makes it hard
for variations and abstraction generations.

Similar attempts to understand the user's query include classification methods that 
map the query phrase into semantic classes.\cite{demartini2013crowdq,arguello2009sources}
For example attaching semantic taggings for the query phrase based on user's click history.
\cite{manshadi2009semantic}
Other researches\cite{pasca2008weakly} try to extract the important attributes of the query, for example if 
the query is ``Drug'', it will consider keywords such as ``side effect'',``cost''.

%The core component of this project is to mine these semantically equivalent relationships,
%which we describe as the Action-noun mapping. In our ideal mapping, each action concepts
%have multiple noun abstractions which are semantically close to each other. For example,
%for the action concept "company buy company", the possible noun abstractions are "acquisition",
%"purchase", "merge" etc.
%

In this paper, we approach the problem as if the noun concept is a
summary of the verb action. We believe the best place to look for 
such summaries is from news articles. News articles are written
formally, with a title and usually an overview sentence at the 
beginning. The title and the overview essentially summarize the whole
story. Since news stories typically are of narrative nature, there is
plenty of actions embedded in the body of the main content, which is
where we can extract action concepts. 
Based on the previous work of \cite{gong2015representing}, 
we are able to get the possible arguments of a given verb. 
For example the subject of the verb ``invade'' could
be ``country,'' ``group,'' and ``person.''  
The object of ``invade'' could be ``country'', ``area'', etc. 
With the help of an IsA taxonomy, we know that country subsumes
instances such as iraq and US. Consequently
we can recognize the action concepts from the news text.
The key challenge here is
to map the correct nouns from the title and overview to the suitable
actions from the body of the news.
 

%We apply this algorithm on our news corpus to get around 
%10000 unique verbs and around 250000 actions. 
%%We chose 50 actions for demonstration.
%Next we should find a noun concept dictionary as our abstraction inventory. This could be
%done in multiple ways, here we introduce four different methods. All of them use Wordnet and
%Probase as source for lexical information and knowledge. For example, we can predefine a small set of
%general concepts such as "activity", "event", "process" etc, to query their instances as our
%noun concept candidates. We also make extensive use of Wordnet's lexical file info to get similar effect.
%At this point we use around 150 unique noun concepts for evaluation.

Once we extracted the action concepts in news body, 
and noun concepts in news title. 
We aggregate this information from many news articles into
a tri-partite graph containing the action concepts,
the news and the noun concepts. Using retrieval techniques 
such as TFIDF, we can compute the most related noun abstractions
for each action concepts, for example "war" is the most related 
noun concept for the action "country invade country". Because the result shows "war" is likely to be mentioned in those
news titles where the news bodies contain the instance of "country invade country".

The key contributions of this paper: 
%\KZ{Itemize the key contribution of this work.}
\begin{itemize}
\item a noun concept dictionary containing nouns that could be used as abstractions for actions;
\item action-Noun mapping that maps an action concept triple to one or more
    noun abstractions that are semantically similar;
\item a news search engine which features action related queries. 
\end{itemize}

