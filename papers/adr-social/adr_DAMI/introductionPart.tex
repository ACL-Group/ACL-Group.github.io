\section{Introduction}
\label{intro}
Determination of adverse drug reactions (ADR) is an important part of pharmaceutical research and drug development. Pre-marketing clinical trials are limited by the number of participants, the length of the study and the underlying economic burden for both the pharmaceutical companies and the patients. Several recent researches try to predict the potential ADR of drug by using the drug chemical structures, protein targets or therapeutic indications during the drug development cycle\citep{scheiber2009mapping, xie2009drug, yamanishi2012drug, wang2014exploring, xiao2017adverse}. Some of the new adverse reactions to a drug are learned only when the drug is used in a wide spectrum of patients, with varied ethnicity, underlying diseases and a range of concomitant medication, in a post-launch setting. Furthermore, some reactions take a long time to develop a process which goes well beyond the pre-marketing development cycles of the drugs. For example, Vioxx, developed by Merck \& Co, was approved by the FDA in May 1999 as a nonsteroidal anti-inflammatory drug to treat osteoarthritis, acute pain and dysmenorrhea. However, other Merck \& Co sponsored studies, which were concluded or commenced after the drug was launched, indicated that it was associated with elevated risk of cardiovascular complications \citep{bombardier2000comparison,bresalier2005cardiovascular}. In September of 2004, Merck withdrew Vioxx from the market because of concerns about increased risk of heart attack and stroke associated with long-term, high-dosage use. An FDA study estimated that Vioxx could have caused up to 140, 000 cases of serious heart disease in the US since 1999 \citep{graham2005risk}.  Regulatory authorities and pharmaceutical companies make tremendous effort in avoiding such incidences by conducting post-launch Phase IV clinical trials. In the United States, drug companies spend up to \$12,000 per patient in Phase IV clinical trials, with an average of \$5,856 \footnote{https://www.cuttingedgeinfo.com/2011/us-phase-iv-budgets/}. Conducting such studies in an ``\textit{in silico}'' fashion, i.e., collecting ADRs from pre-existing data sources, has become a valid complement, if not an attractive alternative, to costly Phase IV studies.

Recent years saw a growing research interest in mining adverse drug reactions from various data sources. Data sources can be divided into structured data and unstructured text data, and the approaches differ. Structured data primarily includes official adverse event reports collected by health authorities~\citep{harpaz2010statistical,harpaz2012novel,hahn2012mining,gurulingappa2013automatic} such as
FDA. These reports are relatively easy to process due to their strict conformance to the adverse event reporting standards. However, the quantity of such reports is limited due to the complex procedure of submitting reports and patients' unawareness of spontaneous reporting systems. Unstructured data so far includes biomedical literature, clinical notes or medical records, and online health discussions. These data sources pose more processing challenges because signals are embedded in natural language, which is inherently ambiguous and noisy. Biomedical literatures such as scientific papers are comparatively easier to mine \citep{wang2011drug,yang2012automatic} since the medication and adverse reaction are referred to by their formal names. However, the information therein is not up-to-date and is sometimes biased. Clinical resources were targeted using various methods, such as text mining for identifying ADRs from medicine uses \citep{warrer2012using}, rule-based methods to extract side effects from clinical narratives \citep{sohn2011drug} and retrospective medication orders along with inpatient laboratory results to identify ADRs \citep{liu2013azdrugminer}. Privacy concerns and access restrictions are the biggest obstacles for its wide adoption. Compared to the above data sources, online social media, especially health discussion forums, provide the most comprehensive and timely information about medication use experiences. The large volume, colloquial use of natural language, spelling and grammatical errors are some of the major challenges in mining ADRs from such data sources. 

Existing methods for social media text mining can be categorized into lexicon-based methods, statistical methods, rule-based method, advanced NLP and neural network. Most prior studies \citep{leaman2010towards,yang2012detecting,benton2011identifying,wu2013exploiting,yates2013ADRTrace,liu2014identifying,jiang2013discovering,freifeld2014digital,yeleswarapu2014pipeline} focused on expanding lexicons to find ADRs in text. In these lexicon-based methods, due to the novel adverse reaction phrases on websites, they could not recognize non-regular ADRs that are not contained in the lexicon. Besides, they suffer from poor approximate string matching caused by misspelled words. Some researchers instead utilized statistical \citep{li2011medical,wu2012early,liu2013azdrugminer}, rule (pattern) based methods \citep{nikfarjam2011pattern,benton2011identifying,karimi2011drug,yang2012detecting}; When it comes to NLP techniques, common approaches used Support Vector Machine(SVM) and Conditional Random Field(CRF) to detect ADR from social media\citep{sharif2014detecting,sarker2015portable,jonnagaddala2016binary,nikfarjam2015pharmacovigilance}. They always consider different features such as N-grams, POS tags, negation, sentiment word, polarity and etc. These methods could offer a reasonable accuracy, however they are built with supervised training and require large volume of data during the learning process which requires a tremendous amount of manual effort. Various architectures of neural network have also been researched for the detection of ADRs. People have tried convolutional neural network\citep{lee2017adverse}, recurrent neural network\citep{cocos2017deep} or combine them together\citep{huynh2016adverse}. Moreover, attention mechanism and CRF are sometimes added into the architecture to improve the performance of system\citep{pandey2017improving}. 

Although there is substantial previous research on ADRs extraction from English online forums, very limited research was done on Chinese data. To the best of our knowledge, this paper is the first attempt to mine ADRs from two popular Chinese social media sites, namely Xunyiwenyao \footnote{http://club.xywy.com/} and 
Haodaifu \footnote{http://www.haodf.com/}. Xunyiwenyao and Haodaifu are both online public forums for health-related discussions. 
We have also attempted to use the data from Weibo\footnote{http://weibo.com} which is a Chinese microblogging website. However, very few Weibo messages contain a drug and an ADR at the same time, and most of the messages are noisy. 
For example, among all the messages we crawled from Weibo, 7734 messages 
mentioned Betaloc, but only 1323 of these also contain an ADR. 
After viewing these messages, only 36\% of them are really experience reports 
from the patients who have taken that medicine.
In consequence, we only use the the data from ``Xunyiwenyao'' and ``Haodaifu'' 
in this paper to discover the potential ADRs.

Herein, we propose a semi-supervised learning framework requiring very little manual annotations for mining ADRs from Chinese social media. As an alternative to the methods described above, we build a list of commonly misspelled drug names and extend the customized lexicon with colloquial words and adjective modifiers, in order to address the problem of irregular ADR terms and typos. We also focus on distinguishing between indications and ADRs by training a binary classifier, using the SVM model. To train the classifier, we introduce an automatic labeling algorithm to generate large amount of training data.
