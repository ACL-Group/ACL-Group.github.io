\begin{abstract}
情感分析与观点挖掘作为自然语言处理的一个子领域,相比其他领域如信息挖掘,更着重对于自然语言中观点的表达与理解的研究。文档中观点的自动总结是情感分析与观点挖掘中很重要的一个任务。由于网络上用户评论的数目急剧增加,而我们又没有一个统一的总结这些观点的框架和格式,人们现在很关系如何能够自动的从没有结构的数据中挖掘出有结构的知识。其中一种很实用的总结的格式被是围绕方面的观点总结,这种格式常见于TripAdvisor.com等网站,用户除了写文字以外,还会对酒店等产品的各个方面进行打分。在本文中,我们关注如何能够自动的为一个产品找到这些合适的方面,使得我们可以围绕这些方面来总结用户的观点,这个任务称为方面词提取。由于互联网上产品的种类繁多,为每一类产品人工的制定这些方面词是不可行的。而另一方面我们又有大量的无标注的用户评论数据,适于使用非监督机器学习的方法。在本文中,我们提出一种非监督的多阶段聚类框架来解决方面词提取这一问题,其中使用到聚类、话题模型和深度学习及神经网络在自然语言上的应用。我们提出的模型可以很容易的应用到任何领域的数据上,而且不需要任何标注,具有很好的可扩展性。在试验中,我们的模型取得了比其他单阶段方法更好的表现。我们还会展示如何扩展我们的模型,通过搭配情感预测实现一个完整的评论总结。对比诸多电商网站现有的评论总结形式,我们的方法可以提供更清晰、更易比较、更忠实于用户观点的信息。

\keywords{自然语言处理,情感分析,观点挖掘,文档总结,话题模型,深度学习,神经网络,基于方面的观点挖掘}
\end{abstract}
