%%==================================================
%% chapter02.tex for SJTU Bachelor Thesis
%% version: 0.5.2
%% Encoding: UTF-8
%%==================================================

\chapter{一些 \LaTeX 排版的例子}
\label{chap:example}

\section{数学排版的例子}
\label{sec:matheq}

\subsection{公式排版}
\label{sec:eqformat}

这里有举一个长公式排版的例子,来自\href{http://www.tex.ac.uk/tex-archive/info/math/voss/mathmode/Mathmode.pdf}{《Math mode》}:

\begin {multline}
  \frac {1}{2}\Delta (f_{ij}f^{ij})=
  2\left (\sum _{i<j}\chi _{ij}(\sigma _{i}-
    \sigma _{j}) ^{2}+ f^{ij}\nabla _{j}\nabla _{i}(\Delta f)+\right .\\
  \left .+\nabla _{k}f_{ij}\nabla ^{k}f^{ij}+
    f^{ij}f^{k}\left [2\nabla _{i}R_{jk}-
      \nabla _{k}R_{ij}\right ]\vphantom {\sum _{i<j}}\right )
\end{multline}

\subsubsection{一个四级标题}
\label{sec:depth4}

这是全文唯一的一个四级标题。在这部分中将演示可伸长符号(箭头、等号的例子)的例子,以及如何在可伸长的符号上标注。在\href{http://zhou63.ahut.edu.cn/latex/ctexfaq.pdf}{《CTeX常见问题集》}中也由类似的介绍。
首先需要在diss.tex导言区引入如下的内容:

\begin{lstlisting}[language={TeX}, caption={插入导言区的内容}]
  \makeatletter
  \def\ExtendSymbol#1#2#3#4#5{\ext@arrow 0099{\arrowfill@#1#2#3}{#4}{#5}}
  \def\RightExtendSymbol#1#2#3#4#5{\ext@arrow 0359{\arrowfill@#1#2#3}{#4}{#5}}
  \def\LeftExtendSymbol#1#2#3#4#5{\ext@arrow 6095{\arrowfill@#1#2#3}{#4}{#5}}
  \makeatother
  
  \newcommand\myRightarrow[2][]{\RightExtendSymbol{=}{=}{\Rightarrow}{#1}{#2}}
  \newcommand\myLeftarrow[2][]{\LeftExtendSymbol{\Leftarrow}{=}{=}{#1}{#2}}
  \newcommand\myBioarrow[2][]{\ExtendSymbol{\Leftarrow}{=}{\Rightarrow}{#1}{#2}}
  \newcommand\myLongEqual[2][]{\ExtendSymbol{=}{=}{=}{#1}{#2}}
\end{lstlisting}

然后,在正文插入如代码\ref{mathextend}所示的内容。效果如下:

\begin{lstlisting}[language={TeX}, caption={可伸长的符号},label=mathextend,float]
  \begin{eqnarray}
    f(x) & \myBioarrow{A=B}  & B \\
    & \myLongEqual{A=B} & B \\
    & \myLeftarrow[A=B^2]{B=A^2} & B \nonumber \\
    & \myRightarrow{B^2=A^2} & B
  \end{eqnarray}
\end{lstlisting}

\begin{displaymath}
    A \xleftarrow{n=0} B \xrightarrow[LongLongLongLong]{n>0} C 
\end{displaymath}

\begin{eqnarray}
  f(x) & \myBioarrow{A=B}  & B \\
  & \myLongEqual{A=B} & B \\
  & \myLeftarrow[A=B^2]{B=A^2} & B \nonumber \\
  & \myRightarrow{B^2=A^2} & B
\end{eqnarray}

又如:

\begin{align}
  \label{eq:none}
  & I(X_3;X_4)-I(X_3;X_4|X_1)-I(X_3;X_4|X_2) \nonumber \\
  \myLongEqual{a)}\, & [I(X_3;X_4)-I(X_3;X_4|X_1)]-I(X_3;X_4|\tilde{X}_2) \\
  \myLongEqual[\rule{0.28cm}{0cm}]{}\, & I(X_1;X_3;X_4)-I(X_3;X_4|\tilde{X}_2)
\end{align}


\subsection{定理环境}

模板中定义了丰富的定理环境
algo(算法),thm(定理),lem(引理),prop(命题),cor(推论),defn(定义),conj(猜想),exmp(例),rem(注),case(情形),
bthm(断言定理),blem(断言引理),bprop(断言命题),bcor(断言推论)。
amsmath还提供了一个proof(证明)的环境。
这里举一个``定理''和``证明''的例子。
\begin{thm}[留数定理]
\label{thm:res}
  假设$U$是复平面上的一个单连通开子集,$a_1,\ldots,a_n$是复平面上有限个点,$f$是定义在$U\backslash \{a_1,\ldots,a_n\}$上的全纯函数,
  如果$\gamma$是一条把$a_1,\ldots,a_n$包围起来的可求长曲线,但不经过任何一个$a_k$,并且其起点与终点重合,那么:

  \begin{equation}
    \label{eq:res}
    \ointop_{\gamma}f(z)\,\mathrm{d}z = 2\uppi\mathbf{i}\sum^n_{k=1}\mathrm{I}(\gamma,a_k)\mathrm{Res}(f,a_k)
  \end{equation}

  如果$\gamma$是若尔当曲线,那么$\mathrm{I}(\gamma, a_k)=1$,因此:

  \begin{equation}
    \label{eq:resthm}
    \ointop_{\gamma}f(z)\,\mathrm{d}z = 2\uppi\mathbf{i}\sum^n_{k=1}\mathrm{Res}(f,a_k)
  \end{equation}

      % \oint_\gamma f(z)\, dz = 2\pi i \sum_{k=1}^n \mathrm{Res}(f, a_k ). 

  在这里,$\mathrm{Res}(f, a_k)$表示$f$在点$a_k$的留数,$\mathrm{I}(\gamma,a_k)$表示$\gamma$关于点$a_k$的卷绕数。
  卷绕数是一个整数,它描述了曲线$\gamma$绕过点$a_k$的次数。如果$\gamma$依逆时针方向绕着$a_k$移动,卷绕数就是一个正数,
  如果$\gamma$根本不绕过$a_k$,卷绕数就是零。

  定理\ref{thm:res}的证明。
  
  \begin{proof}
    首先,由……

    其次,……

    所以……
  \end{proof}
\end{thm}

上面的公式例子中,有一些细节希望大家注意。微分号d应该使用``直立体'',也就是用mathrm包围起来。
并且,微分号和被积函数之间应该有一段小间隔,可以插入\verb+\,+得到。
斜体的$d$通常只作为一般变量。
i,j作为虚数单位时,也应该使用``直立体'',为了明显,还加上了粗体,例如\verb+\mathbf{i}+。斜体$i,j$通常用作表示``序号''。
其他字母在表示常量时,也推荐使用``直立体'',譬如,圆周率$\uppi$(需要upgreek宏包),自然对数的底$\mathrm{e}$。


\section{向文档中插入图像}
\label{sec:insertimage}

\subsection{支持的图片格式}
\label{sec:imageformat}

\XeTeX 可以很方便地插入PDF、EPS、PNG、JPG格式的图片。

插入PNG/JPG的例子如\ref{fig:SRR}所示。
这两个水平并列放置的图共享一个``图标题''(table caption),没有各自的小标题。

\begin{figure}[!htp]
  \centering
  \includegraphics[width=0.3\textwidth]{chap2/testpng}
  \hspace{1cm}
  \includegraphics[width=0.3\textwidth]{chap2/testjpg}
  \bicaption[fig:SRR]{这里将出现在插图索引中}{中文题图}{Fig}{English caption}
\end{figure}

这里还有插入eps图像和pdf图像的例子,如图\ref{fig:pdfeps}。这里将EPS和PDF图片作为子图插入,每个子图有自己的小标题。并列子图的功能是使用subfigure宏包提供的。

\begin{figure}
  \centering
  \subfigure[EPS Figure]{
    \label{fig:epspdf:a} %% label for first subfigure
    \includegraphics[width=0.3\textwidth]{chap2/testeps}}
  \hspace{1in}
  \subfigure[PDF Figure]{
    \label{fig:epspdf:b} %% label for second subfigure
    \includegraphics[angle=-90,origin=br,width=0.3\textwidth]{chap2/testpdf.pdf}}
  \bicaption[fig:pdfeps]{插入eps图像和pdf图像}{插入eps和pdf的例子}{Fig}{An EPS and PDF demo}
\end{figure}

更多关于 \LaTeX 插图的例子可以参考\href{http://www.cs.duke.edu/junhu/Graphics3.pdf}{《\LaTeX 插图指南》}。

\subsection{长标题的换行}
\label{sec:longcaption}

图\ref{fig:longcaptionbad}和图\ref{fig:longcaptiongood}都有比较长图标题,通过对比发现,图\ref{fig:longcaptiongood}的换行效果更好一些。
其中使用了minipage环境来限制整个浮动题的宽度。

\begin{figure}[!htp]
 \centering
 \includegraphics[angle=-90,origin=br,width=4cm]{chap2/testpdf.pdf}
 \bicaption[fig:longcaptionbad]{这里将出现在插图索引}{海交通大学是我国历史最悠久的高等学府之一,是教育部直属、教育部与上海市共建的全国重点大学.}{Fig}{Where there is a will, there is a way.}
\end{figure}


  \begin{figure}[!hbp]
    \centering
    \begin{minipage}[b]{0.6\textwidth}
      \captionstyle{\centering}
      \centering
      \includegraphics[angle=-90,origin=br,width=4cm]{chap2/testpdf.pdf}
      \bicaption[fig:longcaptiongood]{这里将出现在插图索引}{海交通大学是我国历史最悠久的高等学府之一,是教育部直属、教育部与上海市共建的全国重点大学.}{Fig}{Where there is a will, there is a way.}
    \end{minipage}     
  \end{figure}

  
\section{表格的例子}
\label{sec:tab}

这一节给出的是一些表格的例子,如表\ref{tab:firstone}所示。

\begin{table}[!hpb]
  \centering
  \bicaption[tab:firstone]{指向一个表格的表目录索引}{一个颇为标准的三线表格\footnotemark[1]}{Table}{A Table}
  \begin{tabular}{@{}llr@{}} \toprule
    \multicolumn{2}{c}{Item} \\ \cmidrule(r){1-2}
    Animal & Description & Price (\$)\\ \midrule
    Gnat & per gram & 13.65 \\
    & each & 0.01 \\
    Gnu & stuffed & 92.50 \\
    Emu & stuffed & 33.33 \\
    Armadillo & frozen & 8.99 \\ \bottomrule
  \end{tabular}
\end{table}
\footnotetext[1]{这个例子来自\href{http://www.ctan.org/tex-archive/macros/latex/contrib/booktabs/booktabs.pdf}{《Publication quality tables in LATEX》}(booktabs宏包的文档)。这也是一个在表格中使用脚注的例子,请留意与threeparttable实现的效果有何不同。}

下面一个是一个更复杂的表格,用threeparttable实现带有脚注的表格,如表\ref{tab:footnote}。

\begin{table}[!htpb]
  \bicaption[tab:footnote]{出现在表目录的标题}{一个带有脚注的表格的例子}{Table}{A Table with footnotes}
  \centering
  \begin{threeparttable}[b]
     \begin{tabular}{ccd{4}cccc}
      \toprule
      \multirow{2}{6mm}{total}&\multicolumn{2}{c}{20\tnote{1}} & \multicolumn{2}{c}{40} &  \multicolumn{2}{c}{60}\\
      \cmidrule(lr){2-3}\cmidrule(lr){4-5}\cmidrule(lr){6-7}
      &www & k & www & k & www & k \\
      \midrule
      &$\underset{(2.12)}{4.22}$ & 120.0140\tnote{2} & 333.15 & 0.0411 & 444.99 & 0.1387 \\
      &168.6123 & 10.86 & 255.37 & 0.0353 & 376.14 & 0.1058 \\
      &6.761    & 0.007 & 235.37 & 0.0267 & 348.66 & 0.1010 \\
      \bottomrule
    \end{tabular}
    \begin{tablenotes}
    \item [1] the first note.% or \item [a]
    \item [2] the second note.% or \item [b]
    \end{tablenotes}
  \end{threeparttable}
\end{table}

\section{参考文献管理}

参考文献的管理是这个学位论文模板又一个有趣的地方。

\subsection{将参考文献的内容与表现分离}

这个论文模板使用BibTeX处理参考文献,这又是一个``内容''与``表现形式''分离的极好例子
\footnote{当然,你也可以手动编参考文献item,直接插入文档中。}。
参考文献的``内容''就是reference文件夹下的chap\textit{xx}.bib,参考文献的元数据(名称、作者、出处等)以一定的格式保存在这些纯文本文件中。
.bib文件也可以理解为参考文献的``数据库'',正文中所有引用的参考文件条目都会从这些文件中``析出''。
控制参考文献条目``表现形式''(格式)的是.bst文件。.bst文件定义了参考文献风格,使用不同的参考文献风格能将同一个参考文献条目输出成不同的格式。
当然,一个文档只能使用一个参考文献风格。本模板使用的是国标GBT7714风格的参考文献。

BibTeX的工作过程是这样的:
BibTeX读取.aux(第一次运行latex得到的)看看你引用了什么参考文献条目,
然后到.bib中找相关条目的信息,
最后根据.bst的格式要求将参考文献条目格式化输出,写到.bbl文件中。
在运行latex将.bbl插入文档之前,你可以用文本编辑器打开它,做一些小的修改。
你会发现,.bbl的格式和你自己手动写item很相似,它已经被赋予了一定的``表现形式''。

.bib数据库中的参考文献条目可以手动编写,也可以在google的学术搜索中找到。
各大数据库\footnote{譬如SCOPUS, IEEE, OSA等。}也支持将参考文献信息导出为.bib,省时省力。
以Google学术搜索为例:进入\url{http://scholar.google.com},在``学术搜索设置''中,将``文献管理软件''设为``显示导入BibTeX''的连接,保存退出。
然后学术搜索找到文献下会有``导出到BibTeX''连接,点击后Firefox会打开新的标签页,出现类似代码\ref{googlescholar}所示的内容
\footnote{展示这些.bib条目使用了listings宏包,因为listings宏包协调中文的能力很糟糕,所以读者在查看模板的这部分源代码时会看到一些非常麻烦的东西。并且,直接将源代码的这部分内容复制到.bib中可能还会出错。我的建议是:这部分内容留意PDF就足够了。}。
请注意,这个条目离``规范''还有一些距离。

  \begin{lstlisting}[caption={从Google Scholar找到的,但并不规范的.bib条目}, label=googlescholar, float, escapeinside="", numbers=none]
    @phdthesis{"白2008信用风险传染模型和信用衍生品的定价",
      title={{"信用风险传染模型和信用衍生品的定价"}},
      author={"白云芬"},
      year={2008},
      school={"上海交通大学"}
    } 
  \end{lstlisting}

  上面的.bib条目的``名字''\cndash{}``白2008信用风险传染模型和信用衍生品的定价'',包含ASCII以外的字符,BibTeX无法处理;
  条目还缺少了address域,这样编译出来的结果会出现``地址不详'';
  并且,条目还缺少language域,BibTeX需要language域来判断是否是中文参考文献。
  将上面的条目修正(改英文名、增加address和language域),复制到本地的.bib文件中就可以了。
  显然,这里描述的是参考文献的内容,而不是表现形式。

  \begin{lstlisting}[caption={一个符合规范的.bib条目}, label=itemok, float, escapeinside="", numbers=none]
    @phdthesis{bai2008,
      title={{"信用风险传染模型和信用衍生品的定价"}},
      author={"白云芬"},
      year={2008},
      language={zh},
      address={"上海"},
      school={"上海交通大学"}
    } 
  \end{lstlisting}

由于中英文参考文献处理起来有差异,所以需要在参考文献中标注是否是中文文献。
确切地说,BibTeX并不具有区分中英文参考文献的``智能'',这种智慧的来源是.bst文——它定义了处理参考文献的规则。
GBT7714-2005NLang.bst中规定:.bib中的条目,如果条目的``language''域非空,就被认为是中文文献,否则被认为是英文文献。
例如,刚才的文献,就会被认为是中文参考文献,采取一些针对中文的处理方式。

最后,这个条目被bibtex处理后,赋予了一定的``表现形式'',在.bbl文件中以下面的样子出现。
你还可以对它进行小的修改。
再次运行latex之后,它将被插入到文档中。

\begin{lstlisting}[caption={.bbl中被格式化之后的条目}, escapeinside="", numbers=none]
\bibitem["白云芬(2008)"]{bai2008}
  \textsc{"白云芬"}.
  \newblock {"信用风险传染模型和信用衍生品的定价"}[D].
  \newblock "上海: 上海交通大学, 2008."
\end{lstlisting}

另外,
.bst文件书写起来非常繁杂\footnote{可以参考\href{http://ftp.ctex.org/mirrors/CTAN/info/bibtex/tamethebeast/ttb_en.pdf}{《Tame The BeaST》}。},书写符合GBT7714标准的.bst文件更是一项浩大的工程。
因此,当大家为漂亮、标准的参考文献列表感到满意时,应该对GBT7714-2005NLang.bst的作者充满谢意。
作者在CTeX BBS发的帖子,请看
\href{http://bbs.ctex.org/viewthread.php?tid=33571&highlight=\%B2\%CE\%BF\%BC\%CE\%C4\%CF\%D7\%2BGB}{文后参考文献著录规则 GB/T 7714-2005}。
关于GB/T 7714-2005标准本身,请看\href{http://bbs.ctex.org/viewthread.php?tid=33571&highlight=GB\%2B\%B2\%CE\%BF\%BC\%CE\%C4\%CF\%D7}{这里}。

.bib是“参考文献的内容”,而控制参考文献表现(格式)的是.bst文件,本模板附带的是GBT7714-2005NLang.bst。

\subsection{在正文中引用参考文献}

参考文献可以分章节管理,只需要在主文件中的参考文献中都包含进去就可以,如\verb+\bibliography{chap1,chap2,...}+。

正文中引用参考文献时,用\verb+\upcite{key1,key2,key3...}+可以产生“上标引用的参考文献”,
如\upcite{Meta_CN,chen2007act,DPMG}。
使用\verb+\cite{key1,key2,key3...}+则可以产生水平引用的参考文献,例如\cite{JohnD,zhubajie,IEEE-1363}。
请看下面的例子,将会穿插使用水平的和上标的参考文献:关于书的\cite{Meta_CN,JohnD,IEEE-1363},关于期刊的\upcite{chen2007act,chen2007ewi},
会议论文\cite{DPMG,kocher99,cnproceed},
硕士学位论文\cite{zhubajie,metamori2004},博士学位论文\upcite{shaheshang,FistSystem01,bai2008},标准文件\cite{IEEE-1363},技术报告\upcite{NPB2},电子文献\cite{xiaoyu2001, CHRISTINE1998},用户手册\cite{RManual}。

最后总结一些注意事项:
\begin{itemize}
\item 参考文献只有在正文中被引用了,才会在最后的参考文献列表中出现;
\item 参考文献``数据库文件''.bib是纯文本文件,请使用UTF-8编码,不要使用GBK编码;
\item 参考文献条目中通过language域是否为空判断是否是中文文献;
\item 参考文献条目同样有“内容”和“表现形式”之分,这种可控性是BibTeX带来的。
\end{itemize}


\subsection{参考文献管理器}

参考文献数据库.bib虽然是纯文本的,可以用任意的文本编辑器查看,但总有人喜欢一个找一个``可视化''地查看每一条参考文献。
我想\href{http://jabref.sourceforge.net/}{JabRef}应该是个很不错的选择。
这是一个Java写的程序,需要JRE才能运行。就测试情况来看,很幸运,JabRef可以顺利打开GBK编码的.bib文件。
但是,打开UTF--8编码的.bib源文件过程中总会崩溃,原因不得而知。
由于我们的.bib文件使用的是UTF-8编码,所以JabRef暂时不可用。

提到参考文献管理器,不得不提到另一个广被使用的软件——\href{http://www.endnote.com/}{EndNote}。
EndNote可以导入.bib文件,却不能导出.bib,只能导出.bbl——被格式化的.bib。
看来,EndNote和Word配合得更好一些。


\section{用listings插入源代码}

原先ctexbook文档类和listings宏包配合使用时,代码在换页时会出现莫名其妙的错误,后来经高人指点,顺利解决了。
感兴趣的话,可以看看\href{http://bbs.ctex.org/viewthread.php?tid=53451}{这里}。
这里给使用listings宏包插入源代码的例子,这里是一段C代码。
另外,listings宏包真可谓博大精深,可以实现各种复杂、漂亮的效果,想要进一步学习的同学,可以参考
\href{http://mirror.ctan.org/macros/latex/contrib/listings/listings.pdf}{listings宏包手册}。

\begin{lstlisting}[language={C}, caption={一段C源代码}]
#include <stdio.h>
#include <unistd.h>
#include <sys/types.h>
#include <sys/wait.h>

int main() {
  pid_t pid;

  switch ((pid = fork())) {
  case -1:
    printf("fork failed\n");
    break;
  case 0:
    /* child calls exec */
    execl("/bin/ls", "ls", "-l", (char*)0);
    printf("execl failed\n");
    break;
  default:
    /* parent uses wait to suspend execution until child finishes */
    wait((int*)0);
    printf("is completed\n");
    break;
  }

  return 0;
}
\end{lstlisting}

再给一个插入MATLAB代码的例子,感谢daisying站友提供的代码。

\begin{lstlisting}[language={matlab}, caption={一段MATLAB源代码}]
function paper1
r=0.05;
n=100;
T=1;
X=1;
v0=0.8;
sigma=sqrt(0.08);
deltat=T/n;
for i=1:n
    t(i)=i*deltat;
    w(i)=random('norm',0,t(i),1);
end
for i=1:n
    alpha(i)=0.39;
end
for i=1:n
    temp=0;
    for k=1:i
        temp=temp+alpha(k);
    end
    B(i)=exp(r*t(i));
    BB(i)=B(i)*exp(temp*deltat);
    BBB(i)=exp(-r*(T-t(i)));
end
for i=1:n
    s0(i)=X*BBB(i);
    v(i)=v0*exp((r-0.5*sigma^2)*t(i)+sigma*w(i));
    for j=i+1:n
        D=X*BBB(j);
        d1=(log(v(i)/D)+(r+sigma^2/2)*(t(j)-t(i)))/(sigma*sqrt(t(j)-t(i)));
        d2=d1-(sigma*sqrt(t(j)-t(i)));
        ppp(i,j)=D*exp(-r*(t(j)-t(i)))*(1-cdf('normal',d2,0,1))-v(i)*(1-cdf('n
ormal',d1,0,1));
    end
end
for i=1:n
    s1(i)=0;
    for j=i+1:n
        s1(i)=s1(i)+BB(j)^(-1)*alpha(j)*deltat*(X*BBB(j)-B(j)/B(i)*ppp(i,j));
    end
    s2(i)=0;
    for j=1:n
        s2(i)=s2(i)+alpha(j);
    end
    s2(i)=X*exp(-r*T-s2(i)*deltat);
    s(i)=BB(i)*(s1(i)+s2(i));
end
plot(s)
hold on;
plot(s0);
\end{lstlisting}
