\chapter{Introduction}

Opinon is at the center of human thinking: expressing opinions is a crucial component of our communication with the world; understanding opinions is a huge part our information gathering behaviour. Alongside our logical and mathematical thinking about what we deem as facts and science, opinons build up our sentimental and perceptual world. The study of how opinions are formed and exchanged is a very important task of psychological research, and the study how people express and interpret opinons is a great challange in the linguistics study, which, if thoroughly understood, will surely lead to a better understanding of human beings itself. With today's rapid development in natural language processing (NLP) and machine learning techniques, people are trying to approach such problem, alongside a lot of other linguistic questions, with the power of computer. 

Besides the computational methods developed by computational linguistics researchers, we also have the help of the internet users across the world - the data they produced and shared publicly boosted thousands of new applications that uses machine learning. The amount of online textual content, which is a main part of user-generated data, is growing exponentially. In addition to the billions of books now digitally available online, we also have new resources including social network, discussion boards, even personal emails and messages. For the researchers, it's exciting to have this amount of data for research purpose, but for normal users, it's overwhelming to face so much information in everyday life. One cannot effectively process the information that the internet has to offer, so it's the task of researchers to develop new tools that will help the users effectively gather the information they need without having to browse too many web pages. 

Besides this need for tools that assists users, there is also a linguistic motivation for diving into the big data. Enabling the machines to understand human language has always been the ``holy grail" of natural language processing research. Now researchers are actively exploring the possibilities of building systems with the help of big data that would exceed the performance of traditional methods. The importance of pursuing in this direction is absolutely in no question, and during this process many useful byproducts will emerge. But we also note that there is a very practical goal at hand for natural language processing, that is, to assist people, to help us better communicate and get the information we need, to make our life easier with the data, instead of more chaotic. Machine learning methods boosted by big data have already proved their effectiveness in helping people gather factual information with applications like web search engines and online encyclopedias. Now we can use simple queries to retrieve thousands of related web pages, sorted by their relativety, in just several milliseconds. In the next section, we show that there is a clear demand for such system aiming at subjective information.

% The applications mentioned above are built arround factual information, how can we do about subjective information? How can we let people query other's opinions? When some international issue takes place, how can we help a user understand what others think when millions are speaking out their minds freely online? When a user is comparing products online, how can we select the most useful product reviews so that we can make the decision making process easier and less stressful? This kind of tasks, which is mainly concerned with people's opinions and other subjective information, is the focus of sentiment analysis (also called opinion mining).

\section{The Demand for Subjective Information}

What other people think helps shape our own opinions. Acquiring other people's opinions has always been a natural and important component of our decision making process. Long before the emergence of internet and its applications, people often consult their friends and relatives for things like restaurant recommendations, personal experiences with products and services, and their reasons for supporting a certain presidential candidate. People also referred to newspapers and magazines for product reviews by experts or critics to decide which refrigerator to buy. But the internet allows us seek opinions from a much larger group of people - people that we don't know personally and people that are not professional critics. And conversely, more and more people are getting used to sharing their personal opinions with others online. Results from two surveys \cite{conrad2007opinion,horrigan2008online} of over 2000 american adults demonstrated the activeness of internet users seek others' opinion:

\begin{itemize}
	\item 81\% of the internet users have done product research online
    \item 32\% have rated products or services via online rating system
    \item 30\% have posted comments or reviews online
    \item 73\% of the readers of online reviews find them to have a significant influence in on their decisions when choosing restaurants, hotels and various other services
\end{itemize}
	
People seek others' opinions not just for deciding which product or service to purchase, the need for political information is another important motivation. In a survey \cite{rainie2007election} of over 2500 american adults, Rainie and Horrigan studied the 2006 campaign internet users, which is defined as those who gathered information about the 2006 elections online and exchanged opinions via email.  Among this group, which is 31\% of Americans - over 60 million people,

\begin{itemize}
    \item 28\% said that the major reason for these online activities was to get perspectives from \emph{within} their community
    \item 34\% said that the major reason was to get perspectives from \emph{outside} their community
    \item 27\% looked online for the endorsements or ratings of external organizations
    \item 28\% said that most of the websites they used \emph{share} their point of view
    \item 29\% said that most of the websites they used \emph{challenge} their point of view
    \item 8\% posted their own political commentary online
\end{itemize}

The users' tendency of seeking out what others think on the internet is merely one reason behind the arising interest in developing new systems that focuses on opinions. While a majority of the users reported positive experiences with online product research, 58\% also reported that some information is missing, confusing and overwhelming. There is also a clear need for systems that help people deal with subjective information more effectively. 

Because the online reviews have a large influence on consumers, they are attracting more and more attention from the producers and vendors. It is possible for them to carry out product image tracking, user requirements analysis and even public relation operations more effectively and accurately.  
Many large companies are starting to realize that the user reviews can cause enormous influence in shaping the opinions of other consumers. This can further influence the consumers' purchase decision and the brands' advocacy.
Companies can respond to the user reviews through social media monitoring and analysis, and modifying their marketing strategy, brand positioning, and product development accordingly.

But this kind of tools requires new technologies. The amount of social media content is growing exponentially everyday, therefore traditional methods like clipping services and field agents simply can't keep up with the pace. 
Thus besides individual consumers, the large companies are also the audience for sentiment analysis systsmes, because they are eager to understand how consumers perceive their products and services.
Sentiment analysis syetems can also be a powerful weapon for politicians. With such systems, they can seek what the people think about an issue within a geographical area, easily determine whether the majority of them like a particular policy or not. 
Since during elections, there is a tight correlation between the opinions people express towards the presidential candidates and the final votes, such systems can even be used for predicting the results of the elections.
To develop such systems, we need technologies similar to what is required for the analysis of product reviews. Unfortunately no current method is accurate and reliable enough to be used in real-life applications, such as providing guide for a campaign. But the challenge itself and what could be acheived in the future is very exciting.

\section{An Example of Sentiment Analysis System}

Sentiment analysis and opinion mining is an important research subfield of natural language processing which focuses on identifying and extracting subjective information from natural language data. Due to the intrinstic ambiguity of human language, it is very difficult for machines to understand the semantics of sentences. It is especially challenging when it comes to opinions, as the expression of opinions is particularly complicated, with language techniques we use such as sarcasm and analogy. Clearifying opinion expressions and making them more intelligible is a main part of sentiment analysis research. Anouther important part is to provide structured summarizations for a large amount of unstructured data, such tasks include product review summarization, classification of tweets, and opinion search engine. 

To illustrate some of the main challenges we face in the research of sentiment analysis, let us consider a concrete example of building a search engine for opinions. As explained above, such system would be of great help to both indivisual internet users and big companies or political parties. The development of such system would involve attacking the following problems.

\begin{itemize}
	\item Ideally, this opinon search engine can be integrated with our current factual search engines and form a general-purpose search engine, then we need to determine whether a user is asking for subjective information or factual information. This would require us to do a classification on the the queries, or even determine which part of each query points to subjective information.
	\item For each document, we need to determine if it's topically related to the query, and then we need to highlight the part in which the opinons that the user cares about are expressed. This can be very challenging if our system is to handle information from different sources, varying from well-formatted reviews from Amazon.com and TripAdvisor.com, to loosely structured texts like blog posts, tweets, or even personal messages. Some sources can even be grammatically inaccurate.
	\item Once we have the relavant document, we still need to study the opinions at different levels of granularity: we need to determine the overall sentiment towards a topic, and also specific opinions regarding particular issues or some aspect of a product or service. Also, we need to determine the correct target, and attribute the opinons to the correct opinion holder especially when there are quotations. 
	\item Finally the system has to present the information to the users in a intelligible way - a summarization. It it important that the summarization is accurate and concise, so that it can really serve the users by saving their time and effort. The summarization is likely to consists the following components:
    \begin{itemize}
        \item Identification of the opinion holder
	    \item Subject or target of the opinion
	    \item Ratings or scores on different aspects of the subject
	    \item Highlighted central opinion sentences
	    \item Supporting facts and quotations
    \end{itemize}
\end{itemize}

\section{A Brief History of Sentiment Analysis}

In 1994, under the influence of the writings of the literary theorist Banfield, \cite{wiebe1999development} centered the idea of subjectivity around that of private states, that is, states not open to objective observation or verification as defined by \cite{quirk1985comprehensive}.  Although general opinions, sentiment, emotions all fall under this definition, the typical research that is described as subjectivity analysis is the classification task that seperates opinion-rich language from objective language. The term \emph{opinion mining} first appeared in \cite{dave2003mining}, in which the author claimed that the ideal opinion mining tool would ``process a set of search results for a given item, generating a list of product attributes (quality, features, etc.) and aggregating opinions about each of them (poor, mixed, good)". Much of the subsequent opinion mining research follows this description in its emphasis on the extraction and analysis on various aspects of a given object. The term opinion mining is interpreted more broadly in \cite{liu2007web} to include many different forms on a wider spectrum of data, and the type of anlysis originally described in \cite{dave2003mining} is now called \emph{aspect-based opinion mining}. In \cite{das2001yahoo,tong2001operational}, the term \emph{sentiment analysis} is used in reference to the automatic analysis of evaluative text and tracking of the predicted judgments, with specific applications in market sentiment analysis. The two paper were published in 2001, and since then people start to realize the potential research challenges and great opportunities in real-life applications that sentiment analysis might offer, hundreds of subsequent papers were published on this subject. There are several important factors behind this trend, and the three most important ones are:

\begin{itemize}
    \item The successful application of machine learning in natural language processing and information retrieval
    \item The rapid growth of the amount of data that can boost learning algorithms due to the eplosion of the internet, especially the appearance of e-commerace websites that gather reviews, and social networks that let people express their opinions
    \item People start to realize the importance of enabling the machines to deal with human opinions, the benifit of sentiment analysis systems as a result of such progress, and finally the intellectual and commercial value of such systems
\end{itemize}

\section{Challenges}

Sentiment analysis has drawn an increasing interest due to its potential of assisting both indiviual users and organizations. But besides this practical benefit, the intellectual challenges and new problems it posts are also attractive to the research community.

Take text classification as an example, we will talk about an important difference between sentiment analysis and factual information analysis. Traditionally, text classification tasks aim to categorize documents by their topics. The set of topics largely depends on the data and the number of topics might vary from only two (e.g. fiction vs non-fiction) to hundreds (scientific research fields). By contrast, in sentiment analysis, the value we try to measure can be generalized across domains - we want to predict positive v.s. negative, or a score. This distinction might make classification problems in sentiment analysis appear deceptively simpler than their counterparts in factual information analysis. This is not true - and there are two main reasons.

The first reason is that, it is non-trivial for humans to determine the sentiment of a document. Let's consider product reviews for example: a typical user review often mentions both pros and cons about a product, so there is not an accurate sentiment word that could summarize the whole review, especially when the reviewer does not talk about his/her overall impression explicitly. This is why the review-gathering websites often ask the users to rate the products in various aspects. Another reason is that our intuitives may not be accurate and reliable. The way we perceive a piece of sentiment information is a lot different from how computers perceive it from a statistical viewpoint. In \cite{pang2002thumbs}, the authors asked two persons to propose a set of keywords for determining the sentiment of movie reviews; the prediction using the proposed keyword sets is then compared with using another set gathered by simple statistical methods; the results showed that humans performed worse - 58\% and 64\% from the two participants, 69\% from the computer. On one hand, this experiment demonstrates the necessity and effectiveness of statistical methods, on the other hand it reminds us of the gap between the insights learned from the big data and our perceptions. At the end of the day, we are the audience of such systems, if they cannot learn insights that benefits our purpose, or cannot present those insights in a why that is coherent with our perception, the effectiveness is severely damaged, and we can be misled. The reason behind this the intrinstic ambiguity in human language and vagueness in human perception. In research, this gap between real-life performance and statistical evaluation metrics used in experiments might harm the practicability of statistical methods.

\section{Tasks}

To build a sentiment analysis system like the sentiment search engine we described above, people formulated various tasks to approach the problem in a divide-and-conqur fashion. 
% The tasks can be roughly seperated into two categories: classification problems and extraction problems.

\subsection{Classification and Extraction}

A large portion of sentiment analysis research falls in the category of sentiment classification, which focuses on the polarity or degree of positivity of the language. The most typical task is predicting whether a sentence expresses positive or negative sentiment, but many other tasks of sentiment-related classification / regression / ranking fall in this category. Examples include predicting whether a sentence / document:

\begin{itemize}
    \item Expresses positive or negative sentiment
    \item Is supportive of some issue
    \item Conveys a good news or bad news
    \item Agrees with another opinion
\end{itemize}

Some other classification tasks have more of a detection flavor, including the detection of:

\begin{itemize}
    \item Strong subjectivity
    \item Sexism or racism
    \item Sarcasm and irony
    \item Humor
    \item Fakeness (typically fake review detection)
\end{itemize}

The tasks above are all classification task, there are also extraction tasks, including the extraction of:

\begin{itemize}
    \item Opinion target
    \item Opinion holder
    \item Supporting facts of a claim
\end{itemize}

Joint topic and sentiment analysis is the combination of factual and sentiment analysis. Since the expression of opinion is highly domain- and context-dependent, this kind of research aims for a better scalability across different domains. One typical example is the joint inference of aspect and rating from product reviews, whose goal is to extract the aspect and sentiment value simultaneously.

\subsection{Summarization}

There is a clear connection between single document summarization and topic extraction, since the extracted information can serve as a summarization of the document. This connection also exists for sentiment-oriented analysis. For extracting sentiment information and generating summarizations, there has been many methods, including:
\begin{itemize}
    \item Select one sentence that best summarizes the authors opinion
    \item Find similar text snippets and merge them into shorter snippets
    \item Track the sentiment flow and choose the sentences at the flow local extrema
\end{itemize}

As an extension to single document summarization, people are also interested in summarizing multiple documents. One of the most popular and important task of multi-document summarization in sentiment analysis is aspect-based summarization. This form of summarization is often seen in website like TripAdivisor.com. It divides a single object into multiple aspects and summarize people's opinion arround these aspects.
