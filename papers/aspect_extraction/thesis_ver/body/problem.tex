\chapter{Problem Analysis}
\label{chapter:problem}

The main problem we want to tackle in this paper is \emph{aspect extraction} for aspect-based sentiment analysis for review summarization. The input is a set of review documents about a certain object (it can be a set of reviews about all laptops, or a particular model). The output is a set of words can capture the most important aspects of the object. In this section, we will analyze the characteristics of the data used for this task and how they make it difficult.

Product and service consumers express their opinions towards their purchase through posting product reviews. These reviews are dense of objective information and are appropriate material for sentiment analysis. Some websites that focus on one particular product or service, like TripAdivisor.com, often hand-craft a list of aspect for the reviewers to rate on. Review from these websites contain not only a written paragraph, but also a structured summarization of the reviewer's opinion based on the aspects. But for general e-commerace websites like Amazon.com, which could have millions of different products, it is impossible to manually assign aspects to each product type. Due to this difficulty, most e-commerace websites cannot provide a structured summarization of user reviews. Without a proper summarization, it is much harder for consumers to make purchase decisions based on user reviews. Hence, researchers are encouraged to look for methods that can automatically generate a structured summarization from unstructured user reviews, but without hand-crafting the set of aspects or having the users to rate them. This process is called \emph{aspect-based sentiment analysis}, and as a part of it, determining the set of aspects is called \emph{aspect extraction}.

This kind of automatically generated summarization can benefit not only the potential consumers by providing clear information, but also the reviewers and retailers:

\begin{itemize}
    \item For reviewers, the ratings can be automatically generated from their written paragraph, without having them to explicitly rating the aspects. This can avoid the inconsistency between reviews from different users: for the same product, a satisfied user might give a 5-star while another might only give 4-star. Furthermore, when the aspect words need to be updated due to changes of the product, the reviewers don't have to re-rate the products.
    \item For retailers, they don't have to manually select the set of aspects, which saves a lot of time and effort. Secondly, the automatically generated aspects will reflect the users's opinions, not what the retailers deem as important. As mentioned in the previous chapter, hand-crafted features might not be accurate or effective. Also, the aspect set can change over time when more and more reviews come in.
\end{itemize}

However, since the users have the freedom to write whatever they want, we cannot assume anything about the format of the reviews, so the data can be very noisy. Those reviews can come in many forms, which only adds to the difficulty of analyzing it. Here we talk about the main factors of user reviews that makes aspect-based sentiment analysis difficult.

\begin{itemize}
    \item \textbf{The shift of topics.} In product reviews the topics tend to shift very quickly from sentence to sentence, without smooth transitions. For example it is common to see sentences like the following in hotel reviews: ``The rooms are very clean. Breakfast is nice and there is also a nice restaurant right next door." The two sentences actually expressed the reviewer's opinion on two aspects of hotel: room and food; they are both about the same product, hotel, however there is a clear distinction in the words they used. Since products have many aspects and reviews are usually not very long, it's common that two consecutive sentences talk about very different topics.
    \item \textbf{The mix of opinion and context.} In product reviews people often explain their opinion by sharing personal experiences. For example it is common to see sentences like the following in hotel reviews: ``The staffs are very helpful. The lady at the checkin table answered all our questions about paris very patiently." The two sentences both expressed positive sentiment towards the service aspect of a hotel, but the opinion is expressed in two different ways. In the first sentence the sentiment is expressed directly, with a key word ``staff" mentioned explicity; however in the second sentence this opinion is expressed by giving a specific personal experience, which takes common knowledge to comprehend accurately.
\end{itemize}

These two characteristics are the key factors that make analyzing product reviews difficult. Due to the quick-shifting topics, it is necessary for us to determine which part of the review talk about the same aspect and we need to break a review down to pieces. Due to the mix of opinion and context, we need common knowledge to understand what are the implicitly mentioned aspect in the review. We also need to focus on the opinion-rich sentences, instead of personal experiences that are not focused on any aspect.
