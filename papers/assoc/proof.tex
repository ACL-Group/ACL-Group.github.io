\documentclass[letterpaper]{article}
\newtheorem{lemma}{Lemma}

\begin{document}

\begin{lemma}
Each term $t$ in $T_0$ appear in exactly one vertex of $G$, and no two vertices share an identical concept $c$.
\end{lemma}

Let $C_0$ be the set of all Wikipedia concepts. Consider a bipartite undirected graph $G_0$ with the two disjoint sets being $T_0$ and $C_0$. In $G_0$, there exists an edge between term $t$ and concept $c$ if and only if $t$ is defined by $c$. It's obvious that for every term $t\in T_0$, $t$ belongs to exactly one connected component of $G_0$. Similarly, for every concept $c\in C_0$, $c$ belongs to exactly one connected component of $G_0$.

Since every vertex $v$ of $G$, as defined by Algorithm 1, is obviously a connected component of $G_0$, each $t$ or $c$ belongs to at most one $v$ of $G$.

Since every connected component of $G_0$ with at least one $t$ is obviously a vertex $v$ of $G$, each $t$ belongs to at least one $v$ of $G$.

To put them together, each $t$ belongs to exactly one $v$ of $G$ and each $c$ belongs to at most one $v$ of $G$, which is exactly the lemma.

\end{document}