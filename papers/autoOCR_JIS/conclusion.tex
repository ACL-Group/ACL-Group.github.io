\section{Conclusion}
\label{sec:conclude}
% \KZ{Conclusion can mention what are our main results.
% Need to be a bit more detailed. Too short right now.}
We have proposed a new declarative 
data description language for describing and extracting 
structured text data from medical images. 
Our new language makes use of both data type and 
spatial constraints to automatically 
generate parsers for information extraction from 
images. Although there are errors in the OCR results, 
scoring strategies in the parsers make them capable of 
tolerating errors. 
% enabling one to extract the correct information. 
% With the proposed incremental correction framework, 
We further improve the accuracy of the 
extracted information with the proposed incremental correction framework. 
Our evaluation results validated 
that our proposed methods outperform the other three 
existing solutions on our real life dataset. 
% By designing fuzzy matching strategies, our system 
% finds the optimal results within the OCR results which 
% may be full of errors and noises. 
Furthermore, our evaluation of the manual correction framework indicates that 
by using a suitable error correction recommendation strategy, 
we can correct more and important errors with limited number of 
manual corrections. In sum, our system can be used to automatically extract 
information from a large quantity of medical images with simple 
descriptions and make efficient use of human power. 
Although the dataset used in this paper are exclusively
medical images, which is the starting point of our research, 
this framework can conceivably benefit 
information extraction tasks targetting other types of
semi-structured images.

% We propose to design a declarative data description 
% language for descripting and extracting information from 
% medical images. Our new language make use of the spatial 
% and data constraints in medical images, which can be 
% used to automatically generate parsers for 
% information extraction from these images. 
% We also propose an incremental correction framework 
% to make use of user corrections. 
% Our evaluation results validated that our 
% proposed methods outperform other solutions on 
% our real life dataset. 
