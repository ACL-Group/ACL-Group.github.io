%syntax and examples
\subsection{Expression}
The syntax of ODL is formally described in 
\figref{fig:syntax}. An example 
description for \figref{fig:ecgexample} is show in 
\figref{fig:description}. 

% \newsavebox{\absfalign}
% \begin{lrbox}{\absfalign}
% % \fontsize{6pt}{7pt}\selectfont
% \begin{align*}
% \text{int} ::= ~&[-+]?[0-9]+ \\
% \text{float} ::= ~&[-+]?[0-9]*.[0-9]+ \\
% \text{num} ::= ~&int|float\\
% \text{len} ::= ~&num ~~ (pixel|cm) \\
% \text{coord} ::= ~&\langle len_1, ~~ len_2, ~~ len_3, ~~ len_4\rangle \\
% \text{bop} ::= ~&+|-|*|/|=|!=|<|>|<=|>=\\
% % \text{datatype} ::=
% % ~&Oint(int, int)\\
% % |~& Ofloat(int, int, int, int)\\
% % |~& Ostring(string)\\
% \text{Value v} ::=
% ~&() \\
% |~& int \\
% |~& float \\
% |~& string \\
% |~& len \\
% |~& coord \\
% % |~& {v_1, ~~ ..., ~ v_n}\\
% %|~& \{v_1 | v_2 | ... | v_n\}\\
% |~& \{v_1, ..., v_n\} \\
% \end{align*}
% \end{lrbox}

% \newsavebox{\abssalign}
% \begin{lrbox}{\abssalign}
% % \fontsize{6pt}{7pt}\selectfont
% \begin{align*}
% \text{Expression e} ::=
% % |~& datatype ~~ x \tag{variable}\label{syntax:variable}\\
% ~&c \tag{constant}\label{syntax:constant}\\
% |~& x \tag{name}\label{syntax:name}\\
% % |~& nop ~~ e\\
% |~& e_0(e_1, e_2, ..., e_n) \tag{constraints} \label{syntax:constraints}\\
% % |~& \lambda x.e \tag{funciton}\label{syntax:function}\\
% % |~& e_1 ~~ e_2 \tag{apply}\label{syntax:apple}\\
% |~& hskip ~~e \tag{horizontal skip}\label{syntax:hskip}\\
% |~& vskip ~~e \tag{vertical skip}\label{syntax:vskip}\\
% |~& \{e_1 | e_2 | ... | e_n\} \tag{union}\label{syntax:union}\\
% % |~& \{e_1, ~~ ..., ~~ e_n\}\\
% % |~& e.i\\
% |~& \{e_1, ..., e_n\} \tag{struct}\label{syntax:struct}\\
% |~& e ~~ list \tag{list}\label{syntax:list}\\
% % |~& e_1[e_2] \tag{list element}\label{syntax:listele}\\
% |~& e ~~ as ~~ x \tag{blinding}\label{syntax:blinding}\\
% |~& e_1 ~~ bop ~~ e_2 \tag{binary operation}\label{syntax:bop}\\
% \end{align*}
% \end{lrbox}

% \begin{figure}[h]
% % \centering
% % \subfloat{\parbox{0.5\textwidth}{
% % {\usebox\absfalign}
% \begin{align*}
% \text{int} ::= ~&[-+]?[0-9]+ \\
% \text{float} ::= ~&[-+]?[0-9]*.[0-9]+ \\
% \text{num} ::= ~&int|float\\
% \text{len} ::= ~&num ~~ (pixel|cm) \\
% \text{coor} ::= ~&\langle len_1, ~~ len_2, ~~len_3, ~~ len_4\rangle \\
% \text{bop} ::=
% ~&+|-|*|/\\
% |~& =|!=\\
% |~& <|>|<=|>=\\
% % \text{datatype} ::=
% % ~&Oint(int, int)\\
% % |~& Ofloat(int, int, int, int)\\
% % |~& Ostring(string)\\
% \text{v} ::=
% ~&() \\
% |~& int \\
% |~& float \\
% |~& string \\
% |~& len \\
% |~& coor \\
% % |~& {v_1, ~~ ..., ~ v_n}\\
% %|~& \{v_1 | v_2 | ... | v_n\}\\
% |~& \{v_1, ..., v_n\} \\
% % a = b\\
% % \end{align*}
% % }}
% % \end{subfloat}
% % \hfill
% % \subfloat{
% % \subfloat{\parbox{0.5\textwidth}{
% % {\usebox\abssalign}
% % \begin{align*}
% \text{e} ::=
% % |~& datatype ~~ x \tag{variable}\label{syntax:variable}\\
% ~&c \tag{constant}\label{syntax:constant}\\
% |~& x \tag{name}\label{syntax:name}\\
% % |~& nop ~~ e\\
% |~& e_0(e_1, e_2, ..., e_n) \tag{constraints} \label{syntax:constraints}\\
% % |~& \lambda x.e \tag{funciton}\label{syntax:function}\\
% % |~& e_1 ~~ e_2 \tag{apply}\label{syntax:apple}\\
% |~& hskip ~~e \tag{horizontal skip}\label{syntax:hskip}\\
% |~& vskip ~~e \tag{vertical skip}\label{syntax:vskip}\\
% |~& \{e_1 | e_2 | ... | e_n\} \tag{union}\label{syntax:union}\\
% % |~& \{e_1, ~~ ..., ~~ e_n\}\\
% % |~& e.i\\
% |~& \{e_1, ..., e_n\} \tag{struct}\label{syntax:struct}\\
% |~& e ~~ list \tag{list}\label{syntax:list}\\
% % |~& e_1[e_2] \tag{list element}\label{syntax:listele}\\
% |~& e ~~ as ~~ x \tag{blinding}\label{syntax:blinding}\\
% |~& e_1 ~~ bop ~~ e_2 \tag{binary operation}\label{syntax:bop}\\
% % c = d\\
% % \text{Expression e} ::=
% % % |~& datatype ~~ x \tag{variable}\label{syntax:variable}\\
% % ~&c\\
% % |~& x\\
% % % |~& nop ~~ e\\
% % |~& e_0(e_1, e_2, ..., e_n)\\
% % % |~& \lambda x.e \tag{funciton}\label{syntax:function}\\
% % % |~& e_1 ~~ e_2 \tag{apply}\label{syntax:apple}\\
% % |~& hskip ~~e\\
% % |~& vskip ~~e\\
% % |~& \{e_1 | e_2 | ... | e_n\}\\
% % % |~& \{e_1, ~~ ..., ~~ e_n\}\\
% % % |~& e.i\\
% % |~& \{e_1, ..., e_n\}\\
% % |~& e ~~ list\\
% % % |~& e_1[e_2] \tag{list element}\label{syntax:listele}\\
% % |~& e ~~ as ~~ x\\
% % |~& e_1 ~~ bop ~~ e_2\\
% \end{align*}
% % }}
% % \end{subfloat}
% \caption{Syntax}
% \label{fig:syntax}
% \end{figure}

\begin{figure*}[!ht]
%\small
%\setlength{\abovecaptionskip}{0.cm}
%\setlength{\belowcaptionskip}{-0.cm}
%\scalebox{0.5}{
\begin{minipage}{0.8\columnwidth}
\begin{align*}
%\text{int} ::=~& \land-?\backslash d+\$\\
%\text{float} ::=~& \land(-?\backslash d+)(.\backslash d+)?\$\\
\text{num} ::=~& int|float\\
\text{len} ::=~& num  (pixel|l|w) ~~ | ~~ \backslash s ~~ | ~~ \backslash n\\
\text{coor} ::=~& \langle len_1, len_2, len_3, len_4\rangle\\
\text{bop} ::=
~& +|-|*|/|\%\\
|~&=|!=|<|>|<=|>=\\
% \text{datatype} ::=
% ~&Oint(int, int)\\
% |~& Ofloat(int, int, int, int)\\
% |~& Ostring(string)\\
\text{c} ::=~& ()~ |~ int~ |~ float~ |~ string \\
% |~& len \\
% |~& coor \\
% |~& {v_1, ~~ ..., ~ v_n}\\
%|~& \{v_1 | v_2 | ... | v_n\}\\
%|~& \{v_1, ..., v_n\} \\
\end{align*}
\end{minipage}
%\scalebox{0.5}{
\begin{minipage}{0.8\columnwidth}
\begin{align*}
\text{e} ::=
% |~& datatype ~~ x \tag{variable}\label{syntax:variable}\\
~& c \tag{constant}\label{syntax:constant}\\
|~& x \tag{variable}\label{syntax:name}\\
% |~& nop ~~ e\\
% |~& \lambda x.e \tag{funciton}\label{syntax:function}\\
% |~& e_1 ~~ e_2 \tag{apply}\label{syntax:apple}\\
|~& hskip ~~len \tag{horizontal skip}\label{syntax:hskip}\\
|~& vskip ~~len \tag{vertical skip}\label{syntax:vskip}\\
|~& \{e_1 | e_2 | ... | e_n\} \tag{union}\label{syntax:union}\\
|~& \{e_1, ..., e_n\} \tag{struct}\label{syntax:struct}\\
|~& e ~~ list \tag{list}\label{syntax:list}\\
% |~& e_1[e_2] \tag{list element}\label{syntax:listele}\\
|~& e ~~ as ~~ x \tag{binding}\label{syntax:binding}\\
|~& e_1 ~~ bop ~~ e_2 \tag{binary operation}\label{syntax:bop}\\
|~& e_0(e_1, e_2, ..., e_n) \tag{constraint} \label{syntax:constraints}\\
\end{align*}
\end{minipage}
\caption{Syntax of OCR description language.}
\label{fig:syntax}

\end{figure*}

% \KZ{Make \figref{fig:syntax} double column and more compact.}


\begin{figure}[h]
\begin{eqnarray*}
&&\{``Vent.'', ``rate'', valueVent_{(``int'')}, (hskip ``s''), ``bpm'', (vskip ``n'')\} ~~ as ~~ triples\\
&&triples_{(imageWidth/10, default, imageWidth/2, default)} ~~ as ~~ description\\
\end{eqnarray*}
\caption{Description}\label{fig:description}
\end{figure}

% blinding and entry
\subsubsection*{Blinding}
Each line in the example description is a blinding expression, which 
means we assign a name for each expression. 
We use the key word \emph{description} as the signal for the 
entry. 
% struct expression and sequence
\subsubsection*{Struct}
In this example, we want to describe \emph{triples}, which is a struct type data. 
A struct type data means all sub descriptions of the struct appear sequentially on 
the image. The default sequence we used for describe the physical layout 
is from left to right and from top to bottom. 
% hskip and vskip expression
\subsubsection*{Spatial Function}
However, there are two violations: 
applying horizontal skip function and vertical skip function with negative 
number. These functions are used to indicate that there are some spaces or irrelevant 
data we want to skip in the description. They can be applied with the spatial 
value. 
Spatial values different from normal values in that they are to describe space 
information. 
They can be the combination of number and spatial unit or some rough description, 
like using ``s'' to indicate a small space or ``n'' to indicate a new line. 
When the skip functions applied with negative spatial values, it means 
the position of the cursor should be moved backward. 
% base type constant and variable
\subsubsection*{Constant and Variable}
In the example, we use some base type expression as the sub expressions of  
\emph{triples}. Constant expressions are used when we can determine the exact  
value. Variable expressions are defined by providing additional information, like 
type constarin, range constarin or precision constarin (for floating point). 
They are used when we know some constarins for the data instead of the exact value. 
By combining the constant expression and the variable expression, ODL have the ability 
to play the role of describing data format by using constant expressions for 
the common values for images sharing the same format, while using variables expressions 
for changeable values for individual images. 
% coordinate information
\subsubsection*{Additional Information}
In order to narrow down the search area for extracting information, 
we use coordinates for the corners of the rough searching area in the image. 
Additional information expression is used for providing such constrains by 
listing them in the subscript expression. 