\section{Robust ODL Parser}
\label{sec:parsing}

In this section, we focus on the ODL parser,
which is the core of the entire information extraction system.
% we focus on the process of structured textual information extraction,
% which is the kernel part of the entire system.
We first describe the semantics that the parser is used for generating
parsing trees based on the ODL specification.
Afterwards, we describe the detail of the fuzzy matching strategy
and the automatic correction model.

\subsection{Semantics of the ODL Parser}
% 1. in general, what's the parsing process
Referring to \figref{fig:running-parsing-tree},
the parsing process evaluates the entire data expression of ODL
into hierarchical texts organized by the parsing tree,
defined as $T = (node, [T_1, ..., T_n])$,
where $T_i$ is the $i$-th sub-tree of $T$.
The leaf nodes of $T$ are $(e, text)$ pairs representing the alignment
between some primitive expression and its corresponding text,
and non-leaf nodes are always the variable names $x$.

% 2. recursive process
Since the entire expression has a complex structure,
the evaluation is conducted recursively:
it first evaluates each sub-expression, and then composes multiple parsing trees
into a large one.
So the semantics of the parser consists of two parts:
the evaluation rules for non-compositional expressions
(constants, primitive variables and spatial expressions),
and the inductive evaluation rules for compositional expressions.

% talk about basic
% \subsubsection{Evaluation Rules for Non-compositional Expressions}
The most basic rule is for evaluating constants (or primitive variables):
given the OCR data $D$ and the constant (or primitive variable) $e$ in ODL,
searching all possible alignments between $e$ and some text box $d \in D$.
During the whole parsing process, the parser maintains an environment
$E = (x_0, y_0, x_1, y_1, x_{cr}, y_{cr})$,
which contains the coordinates of the searching area,
as well as a cursor ($x_{cr}$, $y_{cr}$).
The cursor is a reference point, indicating the rough position
that the desired text box resides.
By default, the searching area is the whole image,
and the cursor is at the top-left corner.
The relative layout in ODL is represented in a
left-to-right then top-to-bottom manner,
thus the text box $d$ becomes a candidate alignment of $e$,
if it's within the searching area,
and not located in the top-left side of the cursor.
The following $InBound$ function defines such rule:
\begin{equation}
  \begin{aligned}
      InBound(D, E) = \{d \in D | &
      E.x_0 \leq d.x_0 \leq d.x_1 \leq E.x_1 \\
      \land &
      E.y_0 \leq d.y_0 \leq d.y_1 \leq E.y_1 \\
      \land & \lnot
      ( d.x_1 \leq E.x_{cr} \land d.y_1 \leq E.y_{cr} )
      \}. \\
  \end{aligned}
\end{equation}

With the environment $E$ provided,
the parser enumerates all candidate text boxes,
and use a boolean match function $Match(e, d; E)$ to
judge whether an alignment exists between $d$ and
the constant (or primitive variable) $e$.
Intuitively, $d$ is a valid match of $e$,
if its text is close to the constraints of $e$,
and it's located near the cursor ($E.x_{cr}, E.y_{cr}$).
For a better flow of explanation, the formal definition of the match function
will be given in the next section.
If matches, a new parsing tree $T=((e, d.text), [])$ is generated,
which has a single node without any children.
Besides, both $D$ and $E$ need to be changed, so that the parser
can work on the alignment of the subsequent expressions in ODL.
Following the relative layout of expressions,
the cursor is moved to the top-right corner of the box:
$E' = Move(E, d)$, and $d$ is removed from $D$,
as one text box can be aligned at most once: $D' = D - \{d\}$.
\equref{equ:semantic-operations} defines a series of functions
that changes the environment, which are used in different evaluation rules.
\begin{equation}
  \begin{aligned}
    Move(E, d) = & (E.x_0, E.y_0, E.x_1, E.y_1, d.x_1, d.y_0), \\
    Hskip(E, len) = & (E.x_0, E.y_0, E.x_1, E.y_1, E.x_{cr}+len, E.y_{cr}), \\
    Vskip(E, len) = & (E.x_0, E.y_0, E.x_1, E.y_1, E.x_0, E.y_{cr}+len), \\
    Restrict(coor) = & (coor.x_0, coor.y_0, coor.x_1, coor.y_1, coor.x_0, coor.y_0). \\
  \end{aligned}
  \label{equ:semantic-operations}
\end{equation}

The tuple $(T, E, D)$ is called a \textit{parsing state},
which records the partial parsing tree,
the environment information and the remaining text boxes.
Since there exist multiple candidate text boxes for alignment,
given $E$ and $D$, $e$ will be evaluated into a set of parsing states,
written in the following judgment form:
\begin{equation}
  E,D;e \Downarrow \bigcup_{i} \{(T_i, E'_i, D'_i)\}.
\end{equation}

% give rules for basic
Based on the definition of the environment, match function and parsing state,
\figref{fig:semantics-kangqi} lists all the evaluation rules.
The rules \ref{rule:empty}, \ref{rule:c1} and \ref{rule:c2} are used for
evaluating the constants or primitive variables in a recursive form,
where $r$ stands for a list of parsing states.
The parser generates new parsing trees based on whether $d$ and $e$ matches.
In addition, the parser can simply skip the text box $d$ and try to find
alignments from remaining OCR data $D-\{d\}$.

The rules \ref{rule:hskip} and \ref{rule:vskip}
are used for evaluating the spatial expressions.
Intuitively, the $hskip$ and $vskip$ expression doesn't match any text boxes,
but indicating the rough size of spacings.
Therefore, both rules merely move the cursor horizontally or vertically,
using $Hskip$ or $Vskip$ function defined in \equref{equ:semantic-operations}.

% talk about compositional
Now we introduce the evaluation rules for compositional expressions,
and focus on how the output parsing states are composed from
the multiple sub-states.
The rule \ref{rule:coor} is used for evaluating spatial constraints $e(coor)$.
The given coordinates $coor$ restrict the searching area of alignments
in the image, thus both the environment and the available OCR data are modified
when evaluating $e$.
The $Restrict$ function sets the new environment based on $coor$,
and the OCR data in the output parsing state contains
all text boxes outside the searching area ($D-D'$),
as well as unused boxes inside it ($D''_i$).
There doesn't have a evaluation rule for value constraints,
but such constraints will be used in the match function.

The last 5 rules in \figref{fig:semantics-kangqi} are used for evaluating
\textit{union}, \textit{struct} and \textit{list} expressions.
The rules \ref{rule:wrap1} and \ref{rule:wrap2}
construct the hierarchical structure of parsing trees,
where the root node $x$ is the identifier of union/struct/list expressions;
\ref{rule:union} simply combines parsing states from all its branches;
\ref{rule:struct} binds the parsing tree of $e_1$ to the larger tree of
the remaining parts;
\ref{rule:list} is similar with \ref{rule:struct}, considering that
a list equals to a struct with unlimited expressions.

\begin{figure*}[ht!]
% \[
%   {\rm Judgment~ Form:}~~ E,D;e \Downarrow (D';parse\_tree)list
%   \label{semantic:judegement}
% \]
% where E is environment, D and D' are input\_data, C records the position of the cursor.
\begin{gather*}
  \tag{\sc E-Empty}\label{rule:empty}
  \frac
  {InBound(D,E)=\emptyset}
  {E,D;e \Downarrow \{(((e, Nil), []), E, D)\}}\\
  \tag{\sc E-PRIM1}\label{rule:c1}
  \frac
  {d \in InBound(D,E), Match(e, d; E)=True, E'=Move(E, d) ~~ E',D-\{d\};e \Downarrow r}
  {E,D;e \Downarrow r \cup \{ (((e, d.text), []), E', D-\{d\} ) \} }\\
  \tag{\sc E-PRIM2}\label{rule:c2}
  \frac
  {d \in InBound(D,E), Match(e, d; E)=False, E'=Move(E, d) ~~ E',D-\{d\};e \Downarrow r}
  {E,D;e \Downarrow r \cup \{ (((e, Nil), []), E', D-\{d\} ) \} }\\
  \tag{\sc E-Hskip}\label{rule:hskip}
  \frac
  {E'=Hskip(E,len)}
  {E,D;hskip\ len \Downarrow \{(Nil,E',D)\}   }\\
  \tag{\sc E-Vskip}\label{rule:vskip}
  \frac
  {E'=Vskip(E,len)}
  {E,D;vskip\ len \Downarrow \{(Nil,E',D)\}   }\\
  \tag{\sc E-Coor}\label{rule:coor}
  \frac
  {E'=Restrict(coor), D'=InBound(D,E') ~~ E',D';e \Downarrow \bigcup_{i} \{(T_i, E''_i, D''_i)\}}
  {E,D;e(coor) \Downarrow \bigcup_{i} \{(T_i, (E.x_0, E.y_0, E.x_1, E.y_1, E''_i.x_{cr}, E''_i.y_{cr}), D-D'+D''_i)\}}\\
  \tag{\sc E-Wrap1}\label{rule:wrap1}
  \frac
  {E,D;e \Downarrow \bigcup_i \{(T_i, E'_i, D'_i)\}}
  {E,D;\{e\}\ as\ x \Downarrow \bigcup_i \{((x, [T_i]), E'_i, D'_i)\}} \\
  \tag{\sc E-Wrap2}\label{rule:wrap2}
  \frac
  {E,D;e \Downarrow \bigcup_i \{(T_i, E'_i, D'_i)\}}
  {E,D;e\ list\ as\ x \Downarrow \bigcup_i \{((x, [T_i]), E'_i, D'_i)\}} \\
  \tag{\sc E-Union}\label{rule:union}
  \frac
  {E,D;\{e_1\}\ as\ x \Downarrow r_1 ~~ E,D;\{e_2|..|e_n\}\ as\ x \Downarrow r_2}
  {E,D;\{e_1|e_2|...|e_n\}\ as\ x \Downarrow r_1 \cup r_2}\\
  \tag{\sc E-Struct}\label{rule:struct}
  \frac
  {E,D;e_1 \Downarrow \bigcup_i \{(T_i, E'_i, D'_i)\} ~~
  \forall i: E'_i,D'_i;\{e_2,...,e_n\}\ as\ x \Downarrow
  \bigcup_j \{((x, [T_{2ij},...,T_{nij}]), E''_{ij}, D''_{ij})\}}
  {E,D;\{e_1,e_2,...e_n\}\ as\ x \Downarrow
  \bigcup_{ij} \{((x, [T_i, T_{2ij},...,T_{nij}]), E''_{ij}, D''_{ij})\}}\\  \tag{\sc E-List}\label{rule:list}
  \frac
  {E,D;e_1 \Downarrow \bigcup_i \{(T_i, E'_i, D'_i)\} ~~
  \forall i: E'_i,D'_i;e\ list\ as\ x \Downarrow
  \bigcup_j \{((x, \textbf{\textit{T}}_{ij}^{(child)}), E''_{ij}, D''_{ij})\}}
  {E,D;e\ list\ as\ x \Downarrow
  \bigcup_{ij} \{((x, [T_i] + \textbf{\textit{T}}_{ij}^{(child)}), E''_{ij}, D''_{ij})\}}\\
  % \tag{\sc E-List}\label{rule:list}
  % \frac
  % {E,D;e \Downarrow (E',D';parse\_tree)list ~~~~ }
  % {E,D;e ~~ list \Downarrow }\\
\end{gather*}
\caption{Evaluation rules of the ODL parser.}
\label{fig:semantics-kangqi}
\end{figure*}



\subsection{Fuzzy Match Function}
% 1. the parser works fuzzily.. (emmm)
% 2. two meanings: data: allows inexact match (15o, vcnt rule)
%                  spatial: not assigned precise boundary to every c/v
% 3. To tolerate the errors and noises,
%    we define functions measuring the distance of matching
%    at both data and spatial level, named xxx and yyy.
% 4. Match function are built based on them.
The key feature of the parsing process is the robustness:
rather than conducting exact match,
the parser tolerates the OCR recognition errors,
and the slight layout variances between images of the same format.
In order to measure the degree of fuzzy matching
between primitive expressions and text boxes,
we define penalty functions at both value and spatial level,
then based on that, we give the formal definition of the $Match$ function.
% We first discuss
% the parser can align under inexact match,
% with no precise boundary.
% Instead, using a fuzzy match function, which consider
% both data and spatial fitness between text box and expressions.
% We discuss these two points, and then give a formal definition of match function.
% Due to the limitations of OCR techniques, data derived from images is not 100\%
% correct. To tolerate the errors and noises in the OCR results, a scoring
% policy is proposed to take consideration of the constraints and OCR results.


\paragraph{Penalty Function for Value Constraints}
The value penalty function $F_v(e, text)$ measures the penalty of
aligning some text with the primitive expression $e$,
i.e., either constants or primitive variables.
For the constant $c$, since the desired value is fixed, the penalty score
is simply derived from the edit distance (Levenshtein distance)
between two texts,
which measures the minimal number of edits required to change one text to
the other.
For the primitive variable $x$ of the integer and floating point type,
the value constraint is put into use.
Referring to the value constraints $x(int, v_{min}, v_{max})$ and
$x(float, length, precision, v_{min}, v_{max})$,
edit distances are calculated between the text data and each numerical value,
satisfying the restrictions of value range, length or precision,
and the smallest edit distance is picked as the error score.
The complete form of the value penalty $F_v(e, text)$ is defined as follows:
\begin{equation}
  \begin{aligned}
    F_v(c,text)= & EditDist(c, text), \\
    F_v(x(int,v_{min},v_{max}),text)= &
    \min_i\{EditDist(i,text)\ |\ i \in Z, v_{min} \leq i \leq v_{max}\}, \\
    F_v(x(float,l,p,v_{min},v_{max}),text)= &
    \min_i\{EditDist(i,text)\ |\ i \in R', v_{min} \leq i \leq v_{max}\}, \\
  \end{aligned}
\end{equation}
\noindent
where $R'$ is the set of all real numbers in length $l$ and precision $p$.
For example, the penalty score (edit distance)
between the \textit{constant} ``Vent. rate'' and the text ``Vcnt. rule'' is 3.
As another example, the penalty score between $x(int, 60, 100)$ the text ``53''
is 1, since for all desired integers between 60 and 100,
``63'' holds the minimum edit distance with ``53'', which is 1.

\paragraph{Penalty Function for Spatial Layout}
Recap that in the parsing process, based on the left-to-right and top-to-bottom
relative layout between expressions embedded in ODL,
the cursor $(E.x_{cr}, E.y_{cr})$ maintains a rough reference position
that the desired text box resides.
That's to say, the closer a text box $d$ to the cursor,
the higher confidence to align with the current expression.
Therefore, the spatial penalty score $F_s(d, E)$ measures the spatial distance
between the cursor and the top-left corner of box, calculated in L$_1$-norm:
% TODO: unit in len / width.
\begin{equation}
	F_s(d, E) = |d.x_0 - E.x_{cr}| + |d.y_0 - E.y_{cr}|.
\end{equation}

Now we formally define the match function $Match(e, d; E)$.
The function returns a boolean value for whether the text box $d$
can be aligned to the primitive expression $e$.
Given the above penalty functions at both value and spatial view,
the $Match$ function is defined as the weighted sum of two scores,
controlled by an output threshold $\tau$:
\begin{equation}
Match(e, d; E) =
\begin{cases}
\text{T}& \text{$F_v(e, d.text)+k\cdot F_s(d, E) < \tau$}\\
\text{F}& \text{otherwise}
\end{cases}
.
\label{equ:match}
\end{equation}
\noindent
where $k$ and $\tau$ are hyperparameters of the system.
Finally, for picking the best parsing tree
from multiple parsing states of the entire image expression,
the textual error score of each primitive expression equally contributes
to the final score of the whole parsing tree $T$.
We define the overall score of $T$ as follows:
\begin{equation}
  score(T) = \sum_{(e,text) \in leaf(T)} F_v(e, text).
  \label{equ:overall}
\end{equation}


\subsection{Automatic Correction Module}
% 1. what's the correction model
The correction module aims at automatically detecting and correcting
error texts during the parsing process.
For example, the parser not only detects the error of matching the text ``15o''
to the variable $p1$ in \figref{fig:running-parsing-tree},
but also tries to correct the error text into ``150'' on-the-fly.
We first explain the automatic correction in the parsing process,
then discuss the incremental generation of the correction model.

% 2. formally define M, S
\subsubsection{Correction Model}
The correction model $M$ is a set of correction strategies $S$.
Each correction strategy $S$ is a probabilistic distribution of replacements
from the original string $ori$ to multiple candidate strings $dst$:
\begin{equation}
  S = \{(ori, dst_1, p_1), ..., (ori, dst_m, p_m)\}.
\end{equation}
\noindent
A concrete correction strategy, for example,
$S = \{(\text{``o''}, \text{``o''}, 0.6), (\text{``o''}, \text{``0''}, 0.3), (\text{``o''}, \text{``O''}, 0.1)\}$ indicates that given the character ``o''
there's a 60\% possibility that no correction is needed,
30\% possibility to replace into ``0'',
and another 10\% possibility to replace into ``O''.
Since the most frequent error of OCR recognition happens at character level,
all the original strings are short phrases (1,2,3-letter-gram).
In addition, we define $rep(text, ori, dst)$ as the result of
replacing all occurrences of $ori$ with $dst$ in the string $text$.

% 3. correction model-based scoring.
% define rep, es_with_prob, improved induction rule.
Given the correction model $M$,
the parser is able to vary the input text
such that a lower penalty for value constraints results.
Intuitively, the value penalty between ``15o'' and $x(int)$
is always 1 without correction.
It will become smaller than 1 when the correction model is applied,
due to a possibility of varying ``15o'' to ``150''.
More specifically, the relaxed version of value penalty $F'_v$
is the minimum expectation score of different texts
after being replaced by some strategy $S$:
\begin{equation}
F'_v(e, text; M)=\min_{S \in M}\{
  \sum_{(dst_i, p_i) \in S} p_i \cdot F_v(e, text'_i)~ |~ text'_i=rep(text, ori, dst_i)
\}.
\end{equation}

By changing $F_v$ in \equref{equ:match} into $F'_v$,
the matching function $Match(e, d; E, M)$ is able to make better judgments
with the help of the correction model.
Once some $d$ and $e$ matches (\figref{fig:semantics-kangqi}, \ref{rule:c1}),
We record the best strategy $S^\star$ which reaches the minimum value penalty,
and generate the corresponding parsing tree $T_i = ((e, d.text'_i), [])$
for each variant of string replacement by $S^\star$.
For example, (p1, ``15o''), (p1, ``150'') and (p1, ``15O'')
are all valid parsing trees.
The best parsing result will be picked by the overall scoring function
defined in \equref{equ:overall}.
In this case, the system is able to find the correct matching results
from those candidate variants.


\subsubsection{Generation of Correction Model}
% 4. how the correction model is incrementally updated.
% what's the beginning,
The model is generated in a hybrid approach.
The initial correction model is generated from the result of OCR engine.
For each text box, the OCR engine outputs top-K candidate texts, although
only the best one is displayed in the raw OCR result.
Regarding each $i$-th candidate as the replacement of the best text,
the system counts all different ($ori$, $dst$) substring replacements.
For example, given the replacement from ``15o'' to ``150'',
distinct substring replacements are:
(``1'', ``1''), (``5'', ``5''), (``o'', ``0''),
(``15'', ``15''), (``5o'', ``50'') and (``15o'', ``150'').
For those images of the same format,
the system counts all occurrences of substring replacements from every image,
and builds the initial correction model $M$,
where the probabilities are calculated by normalization over occurrences.

The correction model can be updated by incremental human corrections.
Once the user manually changes some error texts into correct ones,
the system obtains new substring replacements ($ori$, $dst$)
and re-calculates all the probabilities in $M$.
Since the human labeled texts are ground truth texts,
all occurrences of substring replacements derived from human correction
are multiplied by a weight factor $w$,
so as to make larger contributions to the probability values in $M$.




% Model initiates by OCR software.
%
% Incremental learnt by human.
%
% Also: prompt users with most frequent error,
% associated with primitive / consts across multiple images.
%
% (random / frequent: leave to experimental settings)
%
% % \begin{enumerate}
% % \item incremental learning model, features;
% % \item How do we figure out which error should be corrected first;
% % \item What will happen after an error been corrected and
% % why the process is incremental.
% \label{sec:incremental}
% The correction model mentioned in \secref{sec:corrmodel}
% is an incremental learning model because
% the model is incrementally changed according to the corrections that
% humans make. The design of the correction model adheres to the scoring
% policy in \secref{sec:score}.
%
% \paragraph{Initial Model}
% % \KZ{Correct all the backquotes!}
% Before the results been corrected by human,
% the initial model is generated using the candidate results of the OCR
% engine. For example, for ``QRS'' in the example image, based on the OCR
% engine, the most reliable result is ``ORS''. Other top candidates are
% ``QPS'', ``QRS'' and ``OPS''. So we can learn from them that ``OR'' can be
% corrected as ``QP'', ``QR'' or ``OP''. These three candidates
% will be added into the initial model.
%
% To calculate the probabilities of the correction candidates in the initial
% model, we count the occurrence of each correction. In the example, the
% probabilities for ``OR'' corrected as ``OR'', ``QP'', ``QR'' or ``OP'' are equal
% since such corrections only happened once.
%
% % \[
% % P(newStr|oriStr) = \frac{occurrence ~of~ ()}{\sum_{tar \in all} occurrence ~of~ C(oriStr)=tar}
% % \]
%
% \paragraph{Training From Human Correction}
% After generating the parsing results using the initial model, we have
% made full use of the OCR engine. To correct the remaining errors,
% human input is needed. The incremental learning model is also suitable
% for learning from human correction.
%
% For example, if a human corrects the error result ``1o.o'' to ``10.0'',
% we can learn from it that for ``o''s in the OCR results, it's possible that
% they should be corrected as ``0''s. So the correction strategy
% for ``o'' is modified and ``0" is the new correct candidate.
% We also calculate the occurrence
% of different human correction for the probability calculation.
%
% \paragraph{Application of the Model}
% The model for correction is used in the scoring policy in the
% fuzzy parser. As shown in \secref{sec:score}, for each
% description, our system will consider all the potential
% results based on the correction strategies in the model.
% For the description ``QRS'' and the most confident results
% of OCR ``ORS'', we will try all the strategies in the model
% and consider both whether the corrected result satisfies the
% description and whether the correction strategy is
% feasible. In this example, since the four correction strategies
% are the same probability to happen, we choose ``QR'' as the correction
% result for ``OR'',
% which has the lowest error score based on the description.
%
% \subsubsection{Manual Correction Policy}
% In this section, we describe the policies for recommending
% errors to be manual corrected. When making use of human correction
% we find that some errors will have a greater impact on
% accuracy if they are corrected. The reason is some similar errors
% occur frequently. Which errors are recommended to a user
% for correction will affect the accuracy and the
% number of corrections that the user has to made.
%
% \paragraph{Random}
% The baseline for correction recommendation is random
% recommendation. Based on the parsing results, we can randomly
% recommend the errors we found for humans to correct.
%
% % \subsubsection*{Most Frequently Error Type}
%
% \paragraph{Most Frequent Error Description Elements}
% Another more efficient way to perform manual correction is to
% recommend the description
% elements that contain the most frequent errors. For a set of images
% in similar formats and the corresponding ODL descriptions,
% we find out which elements in the description are more likely
% to be parsed with errors. For those elements, similar errors
% are more likely to happen since the descriptions for them are the
% same. In this way, our recommendations can be more
% accurate than
% a random recommendation.
