\section{Syntax of OCR Description Language}
\label{sec:syntax}

In this section, we introduce the OCR description language,
and explain what kind of structured data that ODL is able to describe.
The abstract syntax of ODL is formally described in \figref{fig:syntax}.
ODL is able to describe both structured data and
spatial layouts of textual information within the medical image.
In the following parts, we will discuss the ODL syntax and its type system in more detail.
% \newsavebox{\absfalign}
% \begin{lrbox}{\absfalign}
% % \fontsize{6pt}{7pt}\selectfont
% \begin{align*}
% \text{int} ::= ~&[-+]?[0-9]+ \\
% \text{float} ::= ~&[-+]?[0-9]*.[0-9]+ \\
% \text{num} ::= ~&int|float\\
% \text{len} ::= ~&num ~~ (pixel|cm) \\
% \text{coord} ::= ~&\langle len_1, ~~ len_2, ~~ len_3, ~~ len_4\rangle \\
% \text{bop} ::= ~&+|-|*|/|=|!=|<|>|<=|>=\\
% % \text{datatype} ::=
% % ~&Oint(int, int)\\
% % |~& Ofloat(int, int, int, int)\\
% % |~& Ostring(string)\\
% \text{Value v} ::=
% ~&() \\
% |~& int \\
% |~& float \\
% |~& string \\
% |~& len \\
% |~& coord \\
% % |~& {v_1, ~~ ..., ~ v_n}\\
% %|~& \{v_1 | v_2 | ... | v_n\}\\
% |~& \{v_1, ..., v_n\} \\
% \end{align*}
% \end{lrbox}

% \newsavebox{\abssalign}
% \begin{lrbox}{\abssalign}
% % \fontsize{6pt}{7pt}\selectfont
% \begin{align*}
% \text{Expression e} ::=
% % |~& datatype ~~ x \tag{variable}\label{syntax:variable}\\
% ~&c \tag{constant}\label{syntax:constant}\\
% |~& x \tag{name}\label{syntax:name}\\
% % |~& nop ~~ e\\
% |~& e_0(e_1, e_2, ..., e_n) \tag{constraints} \label{syntax:constraints}\\
% % |~& \lambda x.e \tag{funciton}\label{syntax:function}\\
% % |~& e_1 ~~ e_2 \tag{apply}\label{syntax:apple}\\
% |~& hskip ~~e \tag{horizontal skip}\label{syntax:hskip}\\
% |~& vskip ~~e \tag{vertical skip}\label{syntax:vskip}\\
% |~& \{e_1 | e_2 | ... | e_n\} \tag{union}\label{syntax:union}\\
% % |~& \{e_1, ~~ ..., ~~ e_n\}\\
% % |~& e.i\\
% |~& \{e_1, ..., e_n\} \tag{struct}\label{syntax:struct}\\
% |~& e ~~ list \tag{list}\label{syntax:list}\\
% % |~& e_1[e_2] \tag{list element}\label{syntax:listele}\\
% |~& e ~~ as ~~ x \tag{blinding}\label{syntax:blinding}\\
% |~& e_1 ~~ bop ~~ e_2 \tag{binary operation}\label{syntax:bop}\\
% \end{align*}
% \end{lrbox}

% \begin{figure}[h]
% % \centering
% % \subfloat{\parbox{0.5\textwidth}{
% % {\usebox\absfalign}
% \begin{align*}
% \text{int} ::= ~&[-+]?[0-9]+ \\
% \text{float} ::= ~&[-+]?[0-9]*.[0-9]+ \\
% \text{num} ::= ~&int|float\\
% \text{len} ::= ~&num ~~ (pixel|cm) \\
% \text{coor} ::= ~&\langle len_1, ~~ len_2, ~~len_3, ~~ len_4\rangle \\
% \text{bop} ::=
% ~&+|-|*|/\\
% |~& =|!=\\
% |~& <|>|<=|>=\\
% % \text{datatype} ::=
% % ~&Oint(int, int)\\
% % |~& Ofloat(int, int, int, int)\\
% % |~& Ostring(string)\\
% \text{v} ::=
% ~&() \\
% |~& int \\
% |~& float \\
% |~& string \\
% |~& len \\
% |~& coor \\
% % |~& {v_1, ~~ ..., ~ v_n}\\
% %|~& \{v_1 | v_2 | ... | v_n\}\\
% |~& \{v_1, ..., v_n\} \\
% % a = b\\
% % \end{align*}
% % }}
% % \end{subfloat}
% % \hfill
% % \subfloat{
% % \subfloat{\parbox{0.5\textwidth}{
% % {\usebox\abssalign}
% % \begin{align*}
% \text{e} ::=
% % |~& datatype ~~ x \tag{variable}\label{syntax:variable}\\
% ~&c \tag{constant}\label{syntax:constant}\\
% |~& x \tag{name}\label{syntax:name}\\
% % |~& nop ~~ e\\
% |~& e_0(e_1, e_2, ..., e_n) \tag{constraints} \label{syntax:constraints}\\
% % |~& \lambda x.e \tag{funciton}\label{syntax:function}\\
% % |~& e_1 ~~ e_2 \tag{apply}\label{syntax:apple}\\
% |~& hskip ~~e \tag{horizontal skip}\label{syntax:hskip}\\
% |~& vskip ~~e \tag{vertical skip}\label{syntax:vskip}\\
% |~& \{e_1 | e_2 | ... | e_n\} \tag{union}\label{syntax:union}\\
% % |~& \{e_1, ~~ ..., ~~ e_n\}\\
% % |~& e.i\\
% |~& \{e_1, ..., e_n\} \tag{struct}\label{syntax:struct}\\
% |~& e ~~ list \tag{list}\label{syntax:list}\\
% % |~& e_1[e_2] \tag{list element}\label{syntax:listele}\\
% |~& e ~~ as ~~ x \tag{blinding}\label{syntax:blinding}\\
% |~& e_1 ~~ bop ~~ e_2 \tag{binary operation}\label{syntax:bop}\\
% % c = d\\
% % \text{Expression e} ::=
% % % |~& datatype ~~ x \tag{variable}\label{syntax:variable}\\
% % ~&c\\
% % |~& x\\
% % % |~& nop ~~ e\\
% % |~& e_0(e_1, e_2, ..., e_n)\\
% % % |~& \lambda x.e \tag{funciton}\label{syntax:function}\\
% % % |~& e_1 ~~ e_2 \tag{apply}\label{syntax:apple}\\
% % |~& hskip ~~e\\
% % |~& vskip ~~e\\
% % |~& \{e_1 | e_2 | ... | e_n\}\\
% % % |~& \{e_1, ~~ ..., ~~ e_n\}\\
% % % |~& e.i\\
% % |~& \{e_1, ..., e_n\}\\
% % |~& e ~~ list\\
% % % |~& e_1[e_2] \tag{list element}\label{syntax:listele}\\
% % |~& e ~~ as ~~ x\\
% % |~& e_1 ~~ bop ~~ e_2\\
% \end{align*}
% % }}
% % \end{subfloat}
% \caption{Syntax}
% \label{fig:syntax}
% \end{figure}

\begin{figure*}[!ht]
%\small
%\setlength{\abovecaptionskip}{0.cm}
%\setlength{\belowcaptionskip}{-0.cm}
%\scalebox{0.5}{
\begin{minipage}{0.8\columnwidth}
\begin{align*}
%\text{int} ::=~& \land-?\backslash d+\$\\
%\text{float} ::=~& \land(-?\backslash d+)(.\backslash d+)?\$\\
\text{num} ::=~& int|float\\
\text{len} ::=~& num  (pixel|l|w) ~~ | ~~ \backslash s ~~ | ~~ \backslash n\\
\text{coor} ::=~& \langle len_1, len_2, len_3, len_4\rangle\\
\text{bop} ::=
~& +|-|*|/|\%\\
|~&=|!=|<|>|<=|>=\\
% \text{datatype} ::=
% ~&Oint(int, int)\\
% |~& Ofloat(int, int, int, int)\\
% |~& Ostring(string)\\
\text{c} ::=~& ()~ |~ int~ |~ float~ |~ string \\
% |~& len \\
% |~& coor \\
% |~& {v_1, ~~ ..., ~ v_n}\\
%|~& \{v_1 | v_2 | ... | v_n\}\\
%|~& \{v_1, ..., v_n\} \\
\end{align*}
\end{minipage}
%\scalebox{0.5}{
\begin{minipage}{0.8\columnwidth}
\begin{align*}
\text{e} ::=
% |~& datatype ~~ x \tag{variable}\label{syntax:variable}\\
~& c \tag{constant}\label{syntax:constant}\\
|~& x \tag{variable}\label{syntax:name}\\
% |~& nop ~~ e\\
% |~& \lambda x.e \tag{funciton}\label{syntax:function}\\
% |~& e_1 ~~ e_2 \tag{apply}\label{syntax:apple}\\
|~& hskip ~~len \tag{horizontal skip}\label{syntax:hskip}\\
|~& vskip ~~len \tag{vertical skip}\label{syntax:vskip}\\
|~& \{e_1 | e_2 | ... | e_n\} \tag{union}\label{syntax:union}\\
|~& \{e_1, ..., e_n\} \tag{struct}\label{syntax:struct}\\
|~& e ~~ list \tag{list}\label{syntax:list}\\
% |~& e_1[e_2] \tag{list element}\label{syntax:listele}\\
|~& e ~~ as ~~ x \tag{binding}\label{syntax:binding}\\
|~& e_1 ~~ bop ~~ e_2 \tag{binary operation}\label{syntax:bop}\\
|~& e_0(e_1, e_2, ..., e_n) \tag{constraint} \label{syntax:constraints}\\
\end{align*}
\end{minipage}
\caption{Syntax of OCR description language.}
\label{fig:syntax}

\end{figure*}

% \KZ{Make \figref{fig:syntax} double column and more compact.}



\subsection{Primitive Expressions}
According to the abstract syntax, we start from the most primitive expressions
that ODL can describe: \textit{constant} and \textit{variable}.
Constants represent fixed-valued strings or numerical values to be recognized from the image.
For example in \figref{fig:running-odl-abstract}, ``Vent. rate'' and ``mm/mV''
are constant expressions, as they always occur in all ECGs of the same format.
% Constants are usually markers or delimiters in semi-structured data formats.
On the other hand, primitive variables represent numerical (int or float) values
varied in different images.
In \figref{fig:running-odl-abstract}, the variable $vr$ represents the
numerical value of ``Vent. rate''.%, which varies by different patients.
%Variables usually have a numeric type (int or float),
%and sometimes users know the constraints of its data field.
It's worth mentioning that,
the reason we specially define constants in ODL is to
enrich the relative layout information between different expressions,
so that the parser can locate the variables in the image more accurately.

\subsection{Spatial Expressions}
Spatial description expressions include \textit{hskip len} and \textit{vskip len}.
% These two expressions explicitly encode horizontal and vertical
% spacing information in the images, and we mainly explain the former one.
As shown in \figref{fig:running-ecg}, there has a large horizontal margin
between ``Vent. rate'' and ``63 bpm'', hence we can use \textit{hskip len} to
explicitly and approximately describe such horizontal margin
between the previous and next expressions.
% we can explain why needed here, because not all places have hskip/vskip.
The size of margin is determined by $len$, which has two parts:
the length value and its measurement unit.
Units can be the absolute pixel, or what are more encouraged,
the percentage of width ($w$) or height ($h$) of the image.
Besides, \textbackslash t stands for a special margin size,
which equals to the average width of 4 Latin characters.
We can easily estimate this value via the width of each text box
in the raw OCR results.
Similarly, \textit{vskip len} explicitly describes the vertical margin
between expressions, and \textbackslash n stands for the average height
of one Latin character.

\subsection{Composition}
Compositions are compound expressions constructed from other expressions.
These include \textit{union}, \textit{struct}, \textit{list}, \textit{binding}, \textit{bop} and \textit{constraint}.

The first three compositions define more structured and
complex type expressions.
The \textit{union} expression means there exists
multiple potential data or spatial expressions.
For example, the union description of $month\_str$ is the enumeration
of all abbreviations of different months.
The \textit{struct} expression is used to describe
an expression with multiple sub-expressions.
All sub-expressions must be described sequentially,
following the left-to-right then top-to-bottom manner.
As another example in \figref{fig:running-odl-surface},
the struct description of $triple\_t$ contains
the constant attribute name, spacing,
variable value and constant unit listed from left to right.
The \textit{list} expression indicates that a sequence of similar data or
the same spatial expressions should be applied multiple times.
In the example, \textit{vskip \textbackslash n list} represents several
blank lines between the data of interpretation and parameters in the images.

The function of \textit{binding} is to give a variable name \textit{x} to
the composition expression \textit{e}, so that each expression has an identifier
in the output parsing tree.
The function of \textit{bop} supports basic binary operations between numerical
values in the ODL.
These two expressions are designed to simplify the description of ODL.

Finally, the \textit{constraint} expression in ODL consists of two categories:
value constraints and spatial constraints.
Value constraints can be applied to the primitive variable,
indicating its type and value range, if the user knows in prior.
Since variables are usually numerical, two value constraints are available:
$x(int, v_{min}, v_{max})$ and
$x(float, length, precision, v_{min}, v_{max})$.
For example in \figref{fig:running-odl-abstract},
$day(int, 1, 31)$ constrains the variable to be an integer ranging from 1 to 31;
$p2(float, 3, 1)$ constrains the variable to be a floating number in length 3
and precision 1, without range limits.
Spatial constraints have the form $e(coor)$,
which can be applied to any expression,
and restrict the areas that corresponding data resides in the image.
Such positions in ODL are represented by $coor$, the 4-tuple of
left, top, right, bottom coordinates.
As shown in \figref{fig:running-odl-abstract}, spatial constraints
are applied to several structs: $time$, $tri$ and $inter$.
These constraints are rather rough and large,
as users are encouraged to give larger spatial constraints
if they are not so sure of the exact bounding boxes.


\subsection{Type System}

% In ODL, we can assign a types to every expression.
% We set up some rules determining types for expressions,
% which make up a complete type system.
The inductive typing rules of ODL is shown in \figref{fig:typingrule}.
% The type system of ODL is shown in \figref{fig:type} and \figref{fig:typingrule}.
% \figref{fig:type} shows the basic principles for determining types and
% \figref{fig:typingrule} extends it by providing inductive typing rules.
\textit{T-VARIABLE} indicates that the type of \textit{variable} is
based on the type of name in the typing context.
\textit{T-INT ARITH}, \textit{T-INT REL}, \textit{T-FLOAT ARITH} and
\textit{T-FLOAT REL} indicate that the two expressions of the \textit{bop}
are of the same type (int or float), and the final type of the \textit{bop}
expression is based on the binary operation.
\textit{T-CONSTRAINT} indicates that the final type of the \textit{constraints} expression is always the same as the original expression $e_0$.
\textit{T-HSKIP} and \textit{T-VSKIP} indicate that the spatial parameter $e$
is of the \textit{len} type, and the final types of these two expressions
should be \textit{unit}.
The last three of the typing rules are for
the \textit{union}, \textit{struct} and \textit{list} expressions, respectively.



\begin{figure}[ht!]
\centering
\tiny
\begin{minipage}{0.45\columnwidth}
\centering
\begin{align*}
  \tag{T-VARIABLE}
  &\frac
  {\Gamma(x)=t}
  {\Gamma \vdash x:t}\\
  \tag{T-INT ARITH}
  &\frac
  {\Gamma \vdash e_1:int ~~ \Gamma \vdash e_2:int ~~ bop \in \{+,-,*,/,\%\}}
  % bop \in \{+,-,*,/,\%, =, !=, >, <, <=, >=\}
  {\Gamma \vdash e_1 ~~ bop ~~ e_2 :int} \\
  \tag{T-INT REL}
  &\frac
  {\Gamma \vdash e_1:int ~~ \Gamma \vdash e_2:int ~~ bop \in \{=, !=, <, >, <=, >=\}}
  % bop \in \{+,-,*,/,\%, =, !=, >, <, <=, >=\}
  {\Gamma \vdash e_1 ~~ bop ~~ e_2 :bool} \\
  \tag{T-FLOAT ARITH}
  &\frac
  {\Gamma \vdash e_1:float ~~ \Gamma \vdash e_2:float ~~ bop \in \{+,-,*,/,\%\}}
  % bop \in \{+,-,*,/,\%, =, !=, >, <, <=, >=\}
  {\Gamma \vdash e_1 ~~ bop ~~ e_2 :float}\\
  \tag{T-FLOAT REL}
  &\frac
  {\Gamma \vdash e_1:float ~~ \Gamma \vdash e_2:float ~~ bop \in \{=, !=, <, >, <=, >=\}}
  % bop \in \{+,-,*,/,\%, =, !=, >, <, <=, >=\}
  {\Gamma \vdash e_1 ~~ bop ~~ e_2 :float}\\
  \tag{T-CONSTRAINT}
  &\frac
  {\Gamma \vdash e_0:t_0}
  {\Gamma \vdash e_0(e_1, e_2, ..., e_n):t_0}\\
  % \tag{T-FUNCTION}
  % &\frac
  % {\Gamma[x:t_1] \vdash e:t_2}
  % {\Gamma \vdash fn ~~ x=> e:t_1 \rightarrow t_2} \\
  \end{align*}
 \end{minipage}
%\hfill
% \begin{minipage}{0.45\columnwidth}
%\begin{align*}
%  \tag{T-HSKIP}
%  &\frac
%  {\Gamma \vdash e:len}
%  {\Gamma \vdash hskip ~~ e:unit}\\
%  \tag{T-VSKIP}
%  &\frac
%  {\Gamma \vdash e:len}
%  {\Gamma \vdash vskip ~~ e:unit}\\
%  \tag{T-UNION}
%  &\frac
%  {for ~~ each ~~ i ~~ \Gamma \vdash e_i:t_i }
%  {\Gamma \vdash \{e_1|...|e_n\}:t_1+...+t_n}\\
%  \tag{T-STRUCT}
%  &\frac
%  {for ~~ each ~~ i ~~ \Gamma \vdash e_i:t_i }
%  {\Gamma \vdash \{e_1, ..., e_n\}:t_1*...*t_n}\\
%  % \tag{T-UNION}
%  % &\frac
%  % {for ~~ each ~~ i ~~ \Gamma \vdash e_i:t_i }
%  % {\Gamma \vdash \{e_1|...|e_n\}:t_1+...+t_n}\\
%  \tag{T-LIST}
%  &\frac
%  {\Gamma \vdash e:t}
%  {\Gamma \vdash e~~list:t~~list}\\
%  % \tag{T-LIST ELEMENT}
%  % &\frac
%  % {\Gamma \vdash e_1:t~~list ~~ \Gamma \vdash e_2:int}
%  % {\Gamma \vdash e_1[e_2]:t}\\
%\end{align*}
%\end{minipage}
\caption{Selected Typing Rules}\label{fig:typingrule}
\end{figure}


% \textit{unit} type are for the
% expression of \textit{()} value. \textit{int}, \textit{float} and \textit{float} are types
% of the expressions that can be parsed into these type of values. \textit{len} is
% the type of \textit{len} value. The rest four kinds of types are the combination of
% all the types. \textit{$\langle len ~~ len, ~~ len, ~~ len\rangle$} is the type for
% \textit{coord}. \textit{$\{t_1 + t_2 + ... + t_n\}$} is the type for \textit{union} expression, which
% means the type of the value of \textit{union} can be any type of the components of the expression.
% \textit{$\{l_1 : t_1, ~~ ..., ~~ l_n : t_n\}$} is the type for \textit{struct}. All the types of
% the subexpressions in \textit{struct} are recorded. Finally, \textit{t ~~ list} is the type for
% \textit{list}. \textit{list} is made up of a sequence of the expressions in the same type.
% In \figref{fig:typingrule}, the detailed typing rules are described. Other than the types
% of the expressions, \textit{T-INT ARITH}, \textit{T-INT REL}, \textit{T-FLOAT ARITH} and \textit{T-FLOAT REL}
% indicate that the two expressions of the \textit{bop} shown be in the same type, int or float, and
% the final type of the \textit{bop} expression is based on the binary operation. \textit{T-CONSTRAINS}
% indicates that the types of the expressions in the constraints have nothing to do with the
% final type of the \textit{constraints} expression.
