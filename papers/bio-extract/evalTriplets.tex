\subsection{Evaluation of Tuples}
\label{EvalTuples}

%After candidate tuples are extracted by patterns, \emph{PPISnowball} use two totally different approaches to evaluate the quality of each tuple. Next, we will briefly introduce their evaluation principles as follows:\\

Since each tuple can be generated from a number of patterns, that is to say, to quantify the goodness of a certain tuple involves associating with confidences of those patterns. Consider a candidate tuple \emph{T} and a set of patterns \emph{P=\{$P_{i}$\}} that generate such tuple \emph{T}. Assuming for the moment we have already calculated confidence score for each pattern $P_{i}$, called \emph{Conf($P_{i}$)}, then we define the confidence of candidate tuple \emph{T} to be:

\begin{equation}
\begin{aligned}
Conf(T)=1-\prod_{i=0}^{|P|}(1-Conf(P_{i})))
\end{aligned}
\end{equation}

\emph{where P=\{$P_{i}$\} is the set of patterns that facilitate generating tuple \emph{T}}\\

%(2) There is an assumption that a valid tuple of proteins, which possesses an interaction between each other, are biologically related tightly. Based on that, we calculate the biological distance between two proteins directly by mapping proteins to Gene Ontology(GO) (http://www.geneontology.org).The GO project provides an ontology of defined terms representing gene product properties and is structured as a directed acyclic graph\cite{geneontology04}. Therefore the PPI relationship can be somewhat validated by the closeness of distance between two proteins in that directed acyclic graph. Here we use GoSim package (http://cran.r-project.org/web/packages/GOSim/index.html).This package implements several functions useful for computing similarities between GO terms and gene products based on their GO annotation \cite{DBLP:journals/bmcbi/FrohlichSPB07}. The value of similarity is between 0 and 1, we can regard it as sort of confidence.

After calculating the confidence of candidate tuples, \emph{PPISnowball} discards all tuples with low confidence. Just like the negative effect of wrong patterns, those tuples with low confidence might cause \emph{PPISnowball} system to extract erroneous patterns in next iteration and in turn generate more "bad" tuples, resulting in the deterioration of the performance of \emph{PPISnowball} system. So we throw away tuples of which the confidence value is below some prespecified threshold $\mathcal {T}_{t}$, after that, we add the remaining survivors to seed positive tuples for next iteration.
