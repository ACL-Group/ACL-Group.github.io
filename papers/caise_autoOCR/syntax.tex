\subsection{Syntax}
\label{sec:syntax}
% Describe the syntax of the description language.
% \begin{enumerate}
% \item Syntax for the description language, including CFG and type 
% system;
% \item Emphasize the description about spatial information (differences with pads) and constrains;
% \item Description examples.
% \end{enumerate}

%In this section, we briefly describe what an ODL data description looks like by presenting some illustrative examples. 
%
% \newsavebox{\absfalign}
% \begin{lrbox}{\absfalign}
% % \fontsize{6pt}{7pt}\selectfont
% \begin{align*}
% \text{int} ::= ~&[-+]?[0-9]+ \\
% \text{float} ::= ~&[-+]?[0-9]*.[0-9]+ \\
% \text{num} ::= ~&int|float\\
% \text{len} ::= ~&num ~~ (pixel|cm) \\
% \text{coord} ::= ~&\langle len_1, ~~ len_2, ~~ len_3, ~~ len_4\rangle \\
% \text{bop} ::= ~&+|-|*|/|=|!=|<|>|<=|>=\\
% % \text{datatype} ::=
% % ~&Oint(int, int)\\
% % |~& Ofloat(int, int, int, int)\\
% % |~& Ostring(string)\\
% \text{Value v} ::=
% ~&() \\
% |~& int \\
% |~& float \\
% |~& string \\
% |~& len \\
% |~& coord \\
% % |~& {v_1, ~~ ..., ~ v_n}\\
% %|~& \{v_1 | v_2 | ... | v_n\}\\
% |~& \{v_1, ..., v_n\} \\
% \end{align*}
% \end{lrbox}

% \newsavebox{\abssalign}
% \begin{lrbox}{\abssalign}
% % \fontsize{6pt}{7pt}\selectfont
% \begin{align*}
% \text{Expression e} ::=
% % |~& datatype ~~ x \tag{variable}\label{syntax:variable}\\
% ~&c \tag{constant}\label{syntax:constant}\\
% |~& x \tag{name}\label{syntax:name}\\
% % |~& nop ~~ e\\
% |~& e_0(e_1, e_2, ..., e_n) \tag{constraints} \label{syntax:constraints}\\
% % |~& \lambda x.e \tag{funciton}\label{syntax:function}\\
% % |~& e_1 ~~ e_2 \tag{apply}\label{syntax:apple}\\
% |~& hskip ~~e \tag{horizontal skip}\label{syntax:hskip}\\
% |~& vskip ~~e \tag{vertical skip}\label{syntax:vskip}\\
% |~& \{e_1 | e_2 | ... | e_n\} \tag{union}\label{syntax:union}\\
% % |~& \{e_1, ~~ ..., ~~ e_n\}\\
% % |~& e.i\\
% |~& \{e_1, ..., e_n\} \tag{struct}\label{syntax:struct}\\
% |~& e ~~ list \tag{list}\label{syntax:list}\\
% % |~& e_1[e_2] \tag{list element}\label{syntax:listele}\\
% |~& e ~~ as ~~ x \tag{blinding}\label{syntax:blinding}\\
% |~& e_1 ~~ bop ~~ e_2 \tag{binary operation}\label{syntax:bop}\\
% \end{align*}
% \end{lrbox}

% \begin{figure}[h]
% % \centering
% % \subfloat{\parbox{0.5\textwidth}{
% % {\usebox\absfalign}
% \begin{align*}
% \text{int} ::= ~&[-+]?[0-9]+ \\
% \text{float} ::= ~&[-+]?[0-9]*.[0-9]+ \\
% \text{num} ::= ~&int|float\\
% \text{len} ::= ~&num ~~ (pixel|cm) \\
% \text{coor} ::= ~&\langle len_1, ~~ len_2, ~~len_3, ~~ len_4\rangle \\
% \text{bop} ::=
% ~&+|-|*|/\\
% |~& =|!=\\
% |~& <|>|<=|>=\\
% % \text{datatype} ::=
% % ~&Oint(int, int)\\
% % |~& Ofloat(int, int, int, int)\\
% % |~& Ostring(string)\\
% \text{v} ::=
% ~&() \\
% |~& int \\
% |~& float \\
% |~& string \\
% |~& len \\
% |~& coor \\
% % |~& {v_1, ~~ ..., ~ v_n}\\
% %|~& \{v_1 | v_2 | ... | v_n\}\\
% |~& \{v_1, ..., v_n\} \\
% % a = b\\
% % \end{align*}
% % }}
% % \end{subfloat}
% % \hfill
% % \subfloat{
% % \subfloat{\parbox{0.5\textwidth}{
% % {\usebox\abssalign}
% % \begin{align*}
% \text{e} ::=
% % |~& datatype ~~ x \tag{variable}\label{syntax:variable}\\
% ~&c \tag{constant}\label{syntax:constant}\\
% |~& x \tag{name}\label{syntax:name}\\
% % |~& nop ~~ e\\
% |~& e_0(e_1, e_2, ..., e_n) \tag{constraints} \label{syntax:constraints}\\
% % |~& \lambda x.e \tag{funciton}\label{syntax:function}\\
% % |~& e_1 ~~ e_2 \tag{apply}\label{syntax:apple}\\
% |~& hskip ~~e \tag{horizontal skip}\label{syntax:hskip}\\
% |~& vskip ~~e \tag{vertical skip}\label{syntax:vskip}\\
% |~& \{e_1 | e_2 | ... | e_n\} \tag{union}\label{syntax:union}\\
% % |~& \{e_1, ~~ ..., ~~ e_n\}\\
% % |~& e.i\\
% |~& \{e_1, ..., e_n\} \tag{struct}\label{syntax:struct}\\
% |~& e ~~ list \tag{list}\label{syntax:list}\\
% % |~& e_1[e_2] \tag{list element}\label{syntax:listele}\\
% |~& e ~~ as ~~ x \tag{blinding}\label{syntax:blinding}\\
% |~& e_1 ~~ bop ~~ e_2 \tag{binary operation}\label{syntax:bop}\\
% % c = d\\
% % \text{Expression e} ::=
% % % |~& datatype ~~ x \tag{variable}\label{syntax:variable}\\
% % ~&c\\
% % |~& x\\
% % % |~& nop ~~ e\\
% % |~& e_0(e_1, e_2, ..., e_n)\\
% % % |~& \lambda x.e \tag{funciton}\label{syntax:function}\\
% % % |~& e_1 ~~ e_2 \tag{apply}\label{syntax:apple}\\
% % |~& hskip ~~e\\
% % |~& vskip ~~e\\
% % |~& \{e_1 | e_2 | ... | e_n\}\\
% % % |~& \{e_1, ~~ ..., ~~ e_n\}\\
% % % |~& e.i\\
% % |~& \{e_1, ..., e_n\}\\
% % |~& e ~~ list\\
% % % |~& e_1[e_2] \tag{list element}\label{syntax:listele}\\
% % |~& e ~~ as ~~ x\\
% % |~& e_1 ~~ bop ~~ e_2\\
% \end{align*}
% % }}
% % \end{subfloat}
% \caption{Syntax}
% \label{fig:syntax}
% \end{figure}

\begin{figure*}[!ht]
%\small
%\setlength{\abovecaptionskip}{0.cm}
%\setlength{\belowcaptionskip}{-0.cm}
%\scalebox{0.5}{
\begin{minipage}{0.8\columnwidth}
\begin{align*}
%\text{int} ::=~& \land-?\backslash d+\$\\
%\text{float} ::=~& \land(-?\backslash d+)(.\backslash d+)?\$\\
\text{num} ::=~& int|float\\
\text{len} ::=~& num  (pixel|l|w) ~~ | ~~ \backslash s ~~ | ~~ \backslash n\\
\text{coor} ::=~& \langle len_1, len_2, len_3, len_4\rangle\\
\text{bop} ::=
~& +|-|*|/|\%\\
|~&=|!=|<|>|<=|>=\\
% \text{datatype} ::=
% ~&Oint(int, int)\\
% |~& Ofloat(int, int, int, int)\\
% |~& Ostring(string)\\
\text{c} ::=~& ()~ |~ int~ |~ float~ |~ string \\
% |~& len \\
% |~& coor \\
% |~& {v_1, ~~ ..., ~ v_n}\\
%|~& \{v_1 | v_2 | ... | v_n\}\\
%|~& \{v_1, ..., v_n\} \\
\end{align*}
\end{minipage}
%\scalebox{0.5}{
\begin{minipage}{0.8\columnwidth}
\begin{align*}
\text{e} ::=
% |~& datatype ~~ x \tag{variable}\label{syntax:variable}\\
~& c \tag{constant}\label{syntax:constant}\\
|~& x \tag{variable}\label{syntax:name}\\
% |~& nop ~~ e\\
% |~& \lambda x.e \tag{funciton}\label{syntax:function}\\
% |~& e_1 ~~ e_2 \tag{apply}\label{syntax:apple}\\
|~& hskip ~~len \tag{horizontal skip}\label{syntax:hskip}\\
|~& vskip ~~len \tag{vertical skip}\label{syntax:vskip}\\
|~& \{e_1 | e_2 | ... | e_n\} \tag{union}\label{syntax:union}\\
|~& \{e_1, ..., e_n\} \tag{struct}\label{syntax:struct}\\
|~& e ~~ list \tag{list}\label{syntax:list}\\
% |~& e_1[e_2] \tag{list element}\label{syntax:listele}\\
|~& e ~~ as ~~ x \tag{binding}\label{syntax:binding}\\
|~& e_1 ~~ bop ~~ e_2 \tag{binary operation}\label{syntax:bop}\\
|~& e_0(e_1, e_2, ..., e_n) \tag{constraint} \label{syntax:constraints}\\
\end{align*}
\end{minipage}
\caption{Syntax of OCR description language.}
\label{fig:syntax}

\end{figure*}

% \KZ{Make \figref{fig:syntax} double column and more compact.}


%\subsubsection{Example}
%The examples we are using are shown in \secref{sec:appro}. Those are the ECG images in the real life 
%and we want to extract the time, the values for different tests, the interperation report of the ECG and the detail parameters in drawing the ECG. The ODL description for \figref{fig:ecgexample1} and \figref{fig:ecgexample2} is shown in \figref{fig:absdes} by using the abstract syntax. The out layer of the whole description is a struct expression, which combines the subdescription of four kinds of information together. 
%
% In this section, a concert example will be used to illustrate how to provide a image form description in ODL to extract information from images sharing the same form.

% For a sample ECG image \figref{fig:ecgexample}, we can write ODL description shown in \figref{fig:description}. To extract useful information from images sharing the same form. In this case, we want to extract the values for thoses four attributes on ECG. Different images will have the same name for the attributes in similar position if using similar form. So we can write the description shown in fig.
% Each line of the description is a naming expression to simply the description and the name description is used as the key word for entry of the description. 
% The out layer of the description is a struct expression provide with coordinates information. We set the coordinates of the box to avoid distributing by noises else where. The coordinates is just estimated number to make the description easy to write. In the struct named triples, we describe all important data on images and some important spational relation between them. Since thoses attributes have the form of key name, value, and optional unit. Since value differs between different images, we use variable expressions with constrains like normal range, and type. Each triples are in different line, so we use vertical skip function to indicate it. Horizontal skip function also used in order to provide detail information about layout to improve the accuracy of results.

%\subsubsection{Syntax and Type System}
% \KZ{There are still many, many typos and grammatical errors through out 
% this section and the following sections. Run a spell checker, such as aspell
% and read the sections through very carefully.}
The abstract syntax of ODL is formally described in \figref{fig:syntax}.
%\JL{
The first six types in \figref{fig:syntax} are basic types. 
%}
ODL can be used to describe both semi-structured data and 
spatial information for the physical layout. Next we 
will discuss the 
syntax and type system in more detail. 

\subsubsection{Values}

% \KZ{Discuss what types of values are there, sand it relationship with the 
% expression. Basically expression can be parsed into values. Constants
% can also be any of the values. Need to explain len, coord, etc. There's no
% such thing as $v_1$ | $v_2$ | ..| $v_n$. I removed it.}

% \KZ{Is it coor or coord? Make it consistent thruout the paper.}

Basically, all expressions in ODL can be parsed into values. Values 
in ODL can take the form of extracted textual information or the spatial parameters. 
%\JL{
	We can see that a value can take on many different shapes 
in \figref{fig:syntax}. For example, {\em ()} can be regarded as a value.
{\em int}, 
{\em float} and {\em string} can also be regarded as values.
%} 
{\em len} and {\em coor} are two special vlaue types in ODL. They indicate 
the values of spatial parameters. {\em len} contains two parameters: the  
length and the unit for the spatial description. {\em coor} {indicate  values of those
coordinates, and it contains four {\em len} values representing 
the coordinates of the top left corner and the bottom right corner of 
the region of interest (see it in \figref{fig:syntax}). The last definition for value  {\em $\{v_1, ..., v_n\}$} indicates that a struct which contains many values can also be defined as a value. 

\subsubsection{Primitive Expression}
Primitive expressions include {\em constant} and {\em name} , which are the first two definitions for expressoin.
%The idea of ODL is to describe the data information on images 
%based on the data type. 
%This idea comes from PADS, which uses types to describe ad hoc data \cite{fisher2011pads}.
%{\em Constant} and {\em name} are the basic data description 
%expressions in ODL. 
Constants represent fixed-valued strings or numerical values to be recognized from the 
image. In \figref{fig:absdes}, the key name ``Vent. rate'' and the unit 
``bpm'' for the medical test are constant expressions 
%in all ECGs of the same
%format. Constants are usually markers or delimiters in semi-structured
%data formats.

%For ECGs in the same format, the key name for the test value is always 
%``Vent. rate'' and the unit for the test value is always ``bpm''. 
On the other hand, if a data field has variable values,
but we know the data type or the constraints of the data field,
(e.g., a numeric type), we can use a {\em name}. 
In \figref{fig:absdes}, the value for ``Vent. rate'' is a variable by
the patient and their different medical tests, we use a name $x$ to represent that. 
In addition, we know that the value for ``Vent. rate'' should be an integer 
between 60 and 100, so we combine {\em name} and {\em constraints} 
to describe the whole field. 
%With the help of {\em constant} and {\em name}, ODL can be used for 
%describing a data form by using {\em constant} for shared data 
%information and {\em name} for different data on individual images.

\subsubsection{Spatial Expression}
Spatial description expressions include % {\em function}, 
{\em hskip e} and {\em vskip e}. 
These are essential constructs for ODL to describe spaces in the images. 
For ODL, an image document can be considered the combination of 
one or more text areas, or text boxes. In \figref{fig:ecgexample2}, 
there are four separate text boxes on the image. 
%In this case, we designed ODL with expressions for describing 
%this spatial information. 
In ODL, the text boxes are described by using the coordinates of 
the corners of the box. This coordinates information also serves to define 
constraints. Text boxes can be specified if we want 
to describe the layout on large scale. 
% \KZ{What's the diff between information box and text box?} 
{\em hskip e} and {\em vskip e} can be used for describing 
approximate distances horizontally or vertically. 
%So in ODL, rough spatial description can also be used to indicate 
%that there is a small space or there is a large space. 
%\JL{
%The detailed process of choosing final results with the 
%help of spatial description will be discussed later 
%in section \ref{sec:semantics}. 
%}
% Fuzzy matching the layout description with the real input 
% will be discussed later in section \ref{sec:semantics}. 
Since spatial information is usually described together with the 
measurement unit, like pixel or centimeter, 
the spatial description expression in ODL can only 
be generated with ``len'' expressions. 
In the example, the coordinates of the upper left corner and 
the bottom right corner for four main parts of the image are outlined in the 
constraints and the symbol ``\_'' means default, which means the 
coordinates of the upper left corner and the bottom right corner of the 
whole document. {\em hskip $\backslash$s} indicates that there are some spaces 
between the key ``Vent. rate'' and the value $x$. And {\em vskip $\backslash$s} 
indicates that there are some vertical spaces between information about 
interpretation and parameters. 
% \KZ{Why not also $vskip~\backslash s$?}

\subsubsection{Composition}
Composition expressions are compound expressions constructed from other
expressions. These include 
{\em union}, {\em struct}, {\em list}, 
{\em constraints}, {\em binding} and {\em bop}.

%With the help of the two kinds of expressions before, we can describe what's on the image and where are they. We combine these two parts in ODL by using complex type expressions. In the example, {\em struct}, {\em union} and {\em list} is used. 
{\em Struct} is used to describe a text box by using both data description 
expressions and spatial description expressions. 
In {\em struct}, all sub-expressions are described in a sequence, 
and the default sequence is from left to right and from top to bottom. In the struct description for ``tuples'', key, value and unit are listed from left to right. 
%Exceptions happen when the space skip function is applied to a negative number, which means moving the cursor backwards. 
{\em Union} means there are many potential data or potential spatial operations to consider. The union description for ``month'' is the enumeration of all abbreviations of the names of the months.
{\em List} is used to indicate that a sequence of similar data or the same spatial operations should be applied multiple times. In the example, {\em list} is used to indicate that there are many blank lines between descriptions for interpretation and parameters.
{\em Constraints} can  
be set based on both data description expressions and 
spatial description expressions. In ODL, constraints can be put on 
% primitive expressions or composition expressions. 
data or a text box. 
% \KZ{What's the difference between data and a text box?}
%Detailed constraints on data can be about the 
%type of the data or the value of the data, while constraints for a
%text box are coordinates that specify the spatial boundaries of the text box. 
Other expressions like {\em binding} and {\em bop} are designed to 
simplify the description and make ODL more expressive. 
The function of {\em binding} is to give a name {\em x} to 
the expression {\em e} so that we can reuse the parsing result of {\em e} 
in the latter description. 
% \KZ{Can say a bit more about what is the purpose of binding.}

% \figref{fig:ecgexample} gives an example ECG report. An ECG report 
% contains values for different attributes. 
% Those values are important for doctors to know the physical condition 
% of patients and make decisions about the treatment. 
% For most of the medical images, values of different test results 
% usually appear in the similar key-value pair format. 
% In this case, we propose ODL data description language to describe the 
% structured information on medical images easily and extract them efficiently.

% \begin{figure}[h]
% \begin{eqnarray*}
% &&\{``Vent.", ``rate", Oint ~~ valueVent, (hskip ``s"), ``bpm", (vskip ``n")\} ~~ as ~~ triples\\
% &&triples_{(imageWidth/10, default, imageWidth/2, default)} ~~ as ~~ description\\
% \end{eqnarray*}
% \caption{Description}\label{fig:description}
% \end{figure}
% % ?general description, features
% ODL uses a type system and spatial description to 
% describe data on medical images. 
% Base types describe atomic pieces of data or represent 
% spatial information and complex types describe how 
% to build compund descriptions from simpler ones. ODL 
% also use additional information to provide constrains 
% based on doctors' knowledge. 
% Besides the description of the data itself, 
% ODL provides spatial description expression to allow users 
% to descrbe the physical layout of the data on the image. 


% %syntax and examples
\subsection{Expression}
The syntax of ODL is formally described in 
\figref{fig:syntax}. An example 
description for \figref{fig:ecgexample} is show in 
\figref{fig:description}. 

% \newsavebox{\absfalign}
% \begin{lrbox}{\absfalign}
% % \fontsize{6pt}{7pt}\selectfont
% \begin{align*}
% \text{int} ::= ~&[-+]?[0-9]+ \\
% \text{float} ::= ~&[-+]?[0-9]*.[0-9]+ \\
% \text{num} ::= ~&int|float\\
% \text{len} ::= ~&num ~~ (pixel|cm) \\
% \text{coord} ::= ~&\langle len_1, ~~ len_2, ~~ len_3, ~~ len_4\rangle \\
% \text{bop} ::= ~&+|-|*|/|=|!=|<|>|<=|>=\\
% % \text{datatype} ::=
% % ~&Oint(int, int)\\
% % |~& Ofloat(int, int, int, int)\\
% % |~& Ostring(string)\\
% \text{Value v} ::=
% ~&() \\
% |~& int \\
% |~& float \\
% |~& string \\
% |~& len \\
% |~& coord \\
% % |~& {v_1, ~~ ..., ~ v_n}\\
% %|~& \{v_1 | v_2 | ... | v_n\}\\
% |~& \{v_1, ..., v_n\} \\
% \end{align*}
% \end{lrbox}

% \newsavebox{\abssalign}
% \begin{lrbox}{\abssalign}
% % \fontsize{6pt}{7pt}\selectfont
% \begin{align*}
% \text{Expression e} ::=
% % |~& datatype ~~ x \tag{variable}\label{syntax:variable}\\
% ~&c \tag{constant}\label{syntax:constant}\\
% |~& x \tag{name}\label{syntax:name}\\
% % |~& nop ~~ e\\
% |~& e_0(e_1, e_2, ..., e_n) \tag{constraints} \label{syntax:constraints}\\
% % |~& \lambda x.e \tag{funciton}\label{syntax:function}\\
% % |~& e_1 ~~ e_2 \tag{apply}\label{syntax:apple}\\
% |~& hskip ~~e \tag{horizontal skip}\label{syntax:hskip}\\
% |~& vskip ~~e \tag{vertical skip}\label{syntax:vskip}\\
% |~& \{e_1 | e_2 | ... | e_n\} \tag{union}\label{syntax:union}\\
% % |~& \{e_1, ~~ ..., ~~ e_n\}\\
% % |~& e.i\\
% |~& \{e_1, ..., e_n\} \tag{struct}\label{syntax:struct}\\
% |~& e ~~ list \tag{list}\label{syntax:list}\\
% % |~& e_1[e_2] \tag{list element}\label{syntax:listele}\\
% |~& e ~~ as ~~ x \tag{blinding}\label{syntax:blinding}\\
% |~& e_1 ~~ bop ~~ e_2 \tag{binary operation}\label{syntax:bop}\\
% \end{align*}
% \end{lrbox}

% \begin{figure}[h]
% % \centering
% % \subfloat{\parbox{0.5\textwidth}{
% % {\usebox\absfalign}
% \begin{align*}
% \text{int} ::= ~&[-+]?[0-9]+ \\
% \text{float} ::= ~&[-+]?[0-9]*.[0-9]+ \\
% \text{num} ::= ~&int|float\\
% \text{len} ::= ~&num ~~ (pixel|cm) \\
% \text{coor} ::= ~&\langle len_1, ~~ len_2, ~~len_3, ~~ len_4\rangle \\
% \text{bop} ::=
% ~&+|-|*|/\\
% |~& =|!=\\
% |~& <|>|<=|>=\\
% % \text{datatype} ::=
% % ~&Oint(int, int)\\
% % |~& Ofloat(int, int, int, int)\\
% % |~& Ostring(string)\\
% \text{v} ::=
% ~&() \\
% |~& int \\
% |~& float \\
% |~& string \\
% |~& len \\
% |~& coor \\
% % |~& {v_1, ~~ ..., ~ v_n}\\
% %|~& \{v_1 | v_2 | ... | v_n\}\\
% |~& \{v_1, ..., v_n\} \\
% % a = b\\
% % \end{align*}
% % }}
% % \end{subfloat}
% % \hfill
% % \subfloat{
% % \subfloat{\parbox{0.5\textwidth}{
% % {\usebox\abssalign}
% % \begin{align*}
% \text{e} ::=
% % |~& datatype ~~ x \tag{variable}\label{syntax:variable}\\
% ~&c \tag{constant}\label{syntax:constant}\\
% |~& x \tag{name}\label{syntax:name}\\
% % |~& nop ~~ e\\
% |~& e_0(e_1, e_2, ..., e_n) \tag{constraints} \label{syntax:constraints}\\
% % |~& \lambda x.e \tag{funciton}\label{syntax:function}\\
% % |~& e_1 ~~ e_2 \tag{apply}\label{syntax:apple}\\
% |~& hskip ~~e \tag{horizontal skip}\label{syntax:hskip}\\
% |~& vskip ~~e \tag{vertical skip}\label{syntax:vskip}\\
% |~& \{e_1 | e_2 | ... | e_n\} \tag{union}\label{syntax:union}\\
% % |~& \{e_1, ~~ ..., ~~ e_n\}\\
% % |~& e.i\\
% |~& \{e_1, ..., e_n\} \tag{struct}\label{syntax:struct}\\
% |~& e ~~ list \tag{list}\label{syntax:list}\\
% % |~& e_1[e_2] \tag{list element}\label{syntax:listele}\\
% |~& e ~~ as ~~ x \tag{blinding}\label{syntax:blinding}\\
% |~& e_1 ~~ bop ~~ e_2 \tag{binary operation}\label{syntax:bop}\\
% % c = d\\
% % \text{Expression e} ::=
% % % |~& datatype ~~ x \tag{variable}\label{syntax:variable}\\
% % ~&c\\
% % |~& x\\
% % % |~& nop ~~ e\\
% % |~& e_0(e_1, e_2, ..., e_n)\\
% % % |~& \lambda x.e \tag{funciton}\label{syntax:function}\\
% % % |~& e_1 ~~ e_2 \tag{apply}\label{syntax:apple}\\
% % |~& hskip ~~e\\
% % |~& vskip ~~e\\
% % |~& \{e_1 | e_2 | ... | e_n\}\\
% % % |~& \{e_1, ~~ ..., ~~ e_n\}\\
% % % |~& e.i\\
% % |~& \{e_1, ..., e_n\}\\
% % |~& e ~~ list\\
% % % |~& e_1[e_2] \tag{list element}\label{syntax:listele}\\
% % |~& e ~~ as ~~ x\\
% % |~& e_1 ~~ bop ~~ e_2\\
% \end{align*}
% % }}
% % \end{subfloat}
% \caption{Syntax}
% \label{fig:syntax}
% \end{figure}

\begin{figure*}[!ht]
%\small
%\setlength{\abovecaptionskip}{0.cm}
%\setlength{\belowcaptionskip}{-0.cm}
%\scalebox{0.5}{
\begin{minipage}{0.8\columnwidth}
\begin{align*}
%\text{int} ::=~& \land-?\backslash d+\$\\
%\text{float} ::=~& \land(-?\backslash d+)(.\backslash d+)?\$\\
\text{num} ::=~& int|float\\
\text{len} ::=~& num  (pixel|l|w) ~~ | ~~ \backslash s ~~ | ~~ \backslash n\\
\text{coor} ::=~& \langle len_1, len_2, len_3, len_4\rangle\\
\text{bop} ::=
~& +|-|*|/|\%\\
|~&=|!=|<|>|<=|>=\\
% \text{datatype} ::=
% ~&Oint(int, int)\\
% |~& Ofloat(int, int, int, int)\\
% |~& Ostring(string)\\
\text{c} ::=~& ()~ |~ int~ |~ float~ |~ string \\
% |~& len \\
% |~& coor \\
% |~& {v_1, ~~ ..., ~ v_n}\\
%|~& \{v_1 | v_2 | ... | v_n\}\\
%|~& \{v_1, ..., v_n\} \\
\end{align*}
\end{minipage}
%\scalebox{0.5}{
\begin{minipage}{0.8\columnwidth}
\begin{align*}
\text{e} ::=
% |~& datatype ~~ x \tag{variable}\label{syntax:variable}\\
~& c \tag{constant}\label{syntax:constant}\\
|~& x \tag{variable}\label{syntax:name}\\
% |~& nop ~~ e\\
% |~& \lambda x.e \tag{funciton}\label{syntax:function}\\
% |~& e_1 ~~ e_2 \tag{apply}\label{syntax:apple}\\
|~& hskip ~~len \tag{horizontal skip}\label{syntax:hskip}\\
|~& vskip ~~len \tag{vertical skip}\label{syntax:vskip}\\
|~& \{e_1 | e_2 | ... | e_n\} \tag{union}\label{syntax:union}\\
|~& \{e_1, ..., e_n\} \tag{struct}\label{syntax:struct}\\
|~& e ~~ list \tag{list}\label{syntax:list}\\
% |~& e_1[e_2] \tag{list element}\label{syntax:listele}\\
|~& e ~~ as ~~ x \tag{binding}\label{syntax:binding}\\
|~& e_1 ~~ bop ~~ e_2 \tag{binary operation}\label{syntax:bop}\\
|~& e_0(e_1, e_2, ..., e_n) \tag{constraint} \label{syntax:constraints}\\
\end{align*}
\end{minipage}
\caption{Syntax of OCR description language.}
\label{fig:syntax}

\end{figure*}

% \KZ{Make \figref{fig:syntax} double column and more compact.}


\begin{figure}[h]
\begin{eqnarray*}
&&\{``Vent.'', ``rate'', valueVent_{(``int'')}, (hskip ``s''), ``bpm'', (vskip ``n'')\} ~~ as ~~ triples\\
&&triples_{(imageWidth/10, default, imageWidth/2, default)} ~~ as ~~ description\\
\end{eqnarray*}
\caption{Description}\label{fig:description}
\end{figure}

% blinding and entry
\subsubsection*{Blinding}
Each line in the example description is a blinding expression, which 
means we assign a name for each expression. 
We use the key word \emph{description} as the signal for the 
entry. 
% struct expression and sequence
\subsubsection*{Struct}
In this example, we want to describe \emph{triples}, which is a struct type data. 
A struct type data means all sub descriptions of the struct appear sequentially on 
the image. The default sequence we used for describe the physical layout 
is from left to right and from top to bottom. 
% hskip and vskip expression
\subsubsection*{Spatial Function}
However, there are two violations: 
applying horizontal skip function and vertical skip function with negative 
number. These functions are used to indicate that there are some spaces or irrelevant 
data we want to skip in the description. They can be applied with the spatial 
value. 
Spatial values different from normal values in that they are to describe space 
information. 
They can be the combination of number and spatial unit or some rough description, 
like using ``s'' to indicate a small space or ``n'' to indicate a new line. 
When the skip functions applied with negative spatial values, it means 
the position of the cursor should be moved backward. 
% base type constant and variable
\subsubsection*{Constant and Variable}
In the example, we use some base type expression as the sub expressions of  
\emph{triples}. Constant expressions are used when we can determine the exact  
value. Variable expressions are defined by providing additional information, like 
type constarin, range constarin or precision constarin (for floating point). 
They are used when we know some constarins for the data instead of the exact value. 
By combining the constant expression and the variable expression, ODL have the ability 
to play the role of describing data format by using constant expressions for 
the common values for images sharing the same format, while using variables expressions 
for changeable values for individual images. 
% coordinate information
\subsubsection*{Additional Information}
In order to narrow down the search area for extracting information, 
we use coordinates for the corners of the rough searching area in the image. 
Additional information expression is used for providing such constrains by 
listing them in the subscript expression. 
% abstract syntax and type
% example

\subsubsection{Type System}


\begin{figure}[ht!]
\centering
\tiny
\begin{minipage}{0.45\columnwidth}
\centering
\begin{align*}
  \tag{T-VARIABLE}
  &\frac
  {\Gamma(x)=t}
  {\Gamma \vdash x:t}\\
  \tag{T-INT ARITH}
  &\frac
  {\Gamma \vdash e_1:int ~~ \Gamma \vdash e_2:int ~~ bop \in \{+,-,*,/,\%\}}
  % bop \in \{+,-,*,/,\%, =, !=, >, <, <=, >=\}
  {\Gamma \vdash e_1 ~~ bop ~~ e_2 :int} \\
  \tag{T-INT REL}
  &\frac
  {\Gamma \vdash e_1:int ~~ \Gamma \vdash e_2:int ~~ bop \in \{=, !=, <, >, <=, >=\}}
  % bop \in \{+,-,*,/,\%, =, !=, >, <, <=, >=\}
  {\Gamma \vdash e_1 ~~ bop ~~ e_2 :bool} \\
  \tag{T-FLOAT ARITH}
  &\frac
  {\Gamma \vdash e_1:float ~~ \Gamma \vdash e_2:float ~~ bop \in \{+,-,*,/,\%\}}
  % bop \in \{+,-,*,/,\%, =, !=, >, <, <=, >=\}
  {\Gamma \vdash e_1 ~~ bop ~~ e_2 :float}\\
  \tag{T-FLOAT REL}
  &\frac
  {\Gamma \vdash e_1:float ~~ \Gamma \vdash e_2:float ~~ bop \in \{=, !=, <, >, <=, >=\}}
  % bop \in \{+,-,*,/,\%, =, !=, >, <, <=, >=\}
  {\Gamma \vdash e_1 ~~ bop ~~ e_2 :float}\\
  \tag{T-CONSTRAINT}
  &\frac
  {\Gamma \vdash e_0:t_0}
  {\Gamma \vdash e_0(e_1, e_2, ..., e_n):t_0}\\
  % \tag{T-FUNCTION}
  % &\frac
  % {\Gamma[x:t_1] \vdash e:t_2}
  % {\Gamma \vdash fn ~~ x=> e:t_1 \rightarrow t_2} \\
  \end{align*}
 \end{minipage}
%\hfill
% \begin{minipage}{0.45\columnwidth}
%\begin{align*}
%  \tag{T-HSKIP}
%  &\frac
%  {\Gamma \vdash e:len}
%  {\Gamma \vdash hskip ~~ e:unit}\\
%  \tag{T-VSKIP}
%  &\frac
%  {\Gamma \vdash e:len}
%  {\Gamma \vdash vskip ~~ e:unit}\\
%  \tag{T-UNION}
%  &\frac
%  {for ~~ each ~~ i ~~ \Gamma \vdash e_i:t_i }
%  {\Gamma \vdash \{e_1|...|e_n\}:t_1+...+t_n}\\
%  \tag{T-STRUCT}
%  &\frac
%  {for ~~ each ~~ i ~~ \Gamma \vdash e_i:t_i }
%  {\Gamma \vdash \{e_1, ..., e_n\}:t_1*...*t_n}\\
%  % \tag{T-UNION}
%  % &\frac
%  % {for ~~ each ~~ i ~~ \Gamma \vdash e_i:t_i }
%  % {\Gamma \vdash \{e_1|...|e_n\}:t_1+...+t_n}\\
%  \tag{T-LIST}
%  &\frac
%  {\Gamma \vdash e:t}
%  {\Gamma \vdash e~~list:t~~list}\\
%  % \tag{T-LIST ELEMENT}
%  % &\frac
%  % {\Gamma \vdash e_1:t~~list ~~ \Gamma \vdash e_2:int}
%  % {\Gamma \vdash e_1[e_2]:t}\\
%\end{align*}
%\end{minipage}
\caption{Selected Typing Rules}\label{fig:typingrule}
\end{figure}


In ODL, we can assign a type to every value and expression. We set up some rules  determining types for values and expressions, which make up a complete type system.Some selected parts of type system of ODL is shown in \figref{fig:type} and \figref{fig:typingrule}. \figref{fig:type} shows the basic principles for detemining types and \figref{fig:typingrule} extends it by giving some of judgement forms which can be seen as inductive typing rules.
{\em T-VARIABLE} indicates that the type of {\em name} expression is based on 
the type of name in the typing context. {\em T-INT ARITH}, {\em T-INT REL}, 
{\em T-FLOAT ARITH} and {\em T-FLOAT REL} 
indicate that the two expressions of the {\em bop} are of the same type, int or float, and 
the final type of the {\em bop} expression is based on the binary operation. 
{\em T-CONSTRAINT} indicates that the types of the expressions 
in the constraints have nothing to do with the 
final type of the {\em constraints} expression, which is the same as the original 
expression {\em $e_0$}. 
%{\em T-HSKIP} and {\em T-VSKIP} are rules for two spatial expressions. 
%The expression {\em e} in {\em horizontal skip} and {\em vertical skip} 
%should have spatial parameters and be of the {\em len} type. The type of 
%these two expressions should be {\em unit}. The last three of all the typing rules are for 
%the composition expressions. {\em T-STRUCT} indicates that the type 
%of {\em struct} is the combination of the types of all sub-expressions. 
%{\em T-UNION} indicates that the type of {\em union} can be 
%any type of the sub-expressions. {\em T-LIST} indicates that the type of 
%{\em list} is a sequence of the same type. 
% {\em unit} type are for the 
% expression of {\em ()} value. {\em int}, {\em float} and {\em float} are types 
% of the expressions that can be parsed into these type of values. {\em len} is 
% the type of {\em len} value. The rest four kinds of types are the combination of 
% all the types. {\em $\langle len ~~ len, ~~ len, ~~ len\rangle$} is the type for 
% {\em coord}. {\em $\{t_1 + t_2 + ... + t_n\}$} is the type for {\em union} expression, which 
% means the type of the value of {\em union} can be any type of the components of the expression. 
% {\em $\{l_1 : t_1, ~~ ..., ~~ l_n : t_n\}$} is the type for {\em struct}. All the types of 
% the subexpressions in {\em struct} are recorded. Finally, {\em t ~~ list} is the type for 
% {\em list}. {\em list} is made up of a sequence of the expressions in the same type. 
% In \figref{fig:typingrule}, the detailed typing rules are described. Other than the types 
% of the expressions, {\em T-INT ARITH}, {\em T-INT REL}, {\em T-FLOAT ARITH} and {\em T-FLOAT REL} 
% indicate that the two expressions of the {\em bop} shown be in the same type, int or float, and 
% the final type of the {\em bop} expression is based on the binary operation. {\em T-CONSTRAINS} 
% indicates that the types of the expressions in the constraints have nothing to do with the 
% final type of the {\em constranints} expression.

