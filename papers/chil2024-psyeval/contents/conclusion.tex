% !TEX root = ../main.tex

\section{Conclusion}
%\KZ{Don't repeat the abstract. Say what important lessons we learned from the above.}
%Our proposed PsyEval benchmark distinguishes itself by offering a more targeted and comprehensive evaluation, specifically designed for mental health-related tasks. PsyEval goes beyond assessing basic understanding or response safety, delving into the complexities unique to mental health. It recognizes that symptoms of mental disorders are subtle, subjective, and highly individualized, requiring a level of expertise, empathy, and emergency response awareness that is not addressed in other benchmarks. This includes understanding nuanced emotional states, detecting subtle signs of mental distress, and providing safe, empathetic interactions. PsyEval, therefore, fills a critical gap in the evaluation of LLMs, positioning itself as a necessary tool for advancing LLMs in the nuanced field of mental health, and setting a new standard for benchmarks in this domain.

%The findings underscore a significant room for improvement in addressing mental health-related questions. Notably, GPT-4 is the only model that achieves a satisfactory performance on the mental health QA task within PsyEval. However, even with GPT-4, there is evident room for advancement. The models exhibit poor performance in tasks predicting multiple diseases from social media posts and forecasting depression severity in simulated doctor-patient dialogues. Additionally, the models face challenges in understanding empathy and ensuring conversation safety in mental health counseling scenarios.

%These results underscore the imperative for further advancements in tailoring language models for optimal performance in mental health applications. PsyEval serves as a valuable tool for assessing and guiding LLM development, emphasizing the need to refine models' comprehension of nuanced mental health contexts, empathy, and conversational safety. Beyond being an evaluative metric, PsyEval acts as a catalyst for innovation in LLMs within the critical domain of mental health.
%We present PsyEval, a comprehensive benchmark spanning three dimensions with six sub-tasks, meticulously designed for a nuanced evaluation of LLMs in mental health contexts. Through extensive experimentation and in-depth PsyEval analysis, we illuminate the performance of various large models.

Our PsyEval benchmark offers a specialized and thorough evaluation for mental health-related tasks, addressing the nuanced complexities of mental health that other benchmarks overlook. It assesses subtle, subjective symptoms of mental disorders, requiring expertise, empathy, and emergency response awareness, including the recognition of nuanced emotional states and providing safe, empathetic interactions. PsyEval thus bridges a vital gap in evaluating LLMs in mental health, setting a new standard in this area.

The results underscore the pressing need for improvement in tasks related to mental health. GPT-4 stands out as the only model that exhibits satisfactory performance in PsyEval's mental health QA task; however, it still demonstrates further potential for development. These models perform suboptimally in tasks such as predicting multiple disorders from social media posts and assessing the severity of depression through simulated doctor-patient dialogues. While they demonstrate Fluency and Coherence comparable to human levels in mental health counseling, ensuring safe outputs, there remains a significant gap in terms of empathy compared to human performance.