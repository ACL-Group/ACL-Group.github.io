\documentclass[sigconf, authordraft, anonymous]{acmart}

\usepackage{booktabs} % For formal tables
\usepackage{graphicx}
\usepackage{epsfig}
\usepackage{amsmath}
\usepackage{float}
\usepackage{array}
\restylefloat{table}
\usepackage{ragged2e}
\usepackage[font=small]{caption}
\newcolumntype{P}[1]{>{\RaggedRight\hspace{0pt}}p{#1}}
\newcolumntype{L}[1]{>{\raggedright\let\newline\\\arraybackslash\hspace{0pt}}m{#1}}
\newcolumntype{C}[1]{>{\centering\let\newline\\\arraybackslash\hspace{0pt}}m{#1}}
\newcolumntype{R}[1]{>{\raggedleft\let\newline\\\arraybackslash\hspace{0pt}}m{#1}}

%\setlist[itemize]{leftmargin=*}

\newcommand{\secref}[1]{Section \ref{#1}}
\newcommand{\figref}[1]{Figure \ref{#1}}
\newcommand{\eqnref}[1]{Eq. (\ref{#1})}
\newcommand{\tabref}[1]{Table \ref{#1}}
\newcommand{\exref}[1]{Example \ref{#1}}
\newcommand{\argmin}{\operatornamewithlimits{argmin}}
\newcommand{\argmax}{\operatornamewithlimits{argmax}}

\newcommand{\cut}[1]{}
\newcommand{\lnear}{\textsc{LocatedNear}}

\newcommand{\BL}[1]{\textcolor{blue}{Bill: #1}}
\newcommand{\HY}[1]{\textcolor{red}{Hanyuan: #1}}
\newcommand{\KZ}[1]{\textcolor{green}{Kenny: #1}}
\newcommand{\SH}[1]{\textcolor{green}{Seung: #1}}
\newcommand{\FX}[1]{\textcolor{blue}{Frank: #1}}


% Copyright
%\setcopyright{none}
%\setcopyright{acmcopyright}
%\setcopyright{acmlicensed}
%\setcopyright{rightsretained}
%\setcopyright{usgov}
%\setcopyright{usgovmixed}
%\setcopyright{cagov}
%\setcopyright{cagovmixed}


% DOI
%\acmDOI{10.475/123_4}

% ISBN
%\acmISBN{123-4567-24-567/08/06}

%Conference
\acmConference[CIKM'17]{The 26th ACM International Conference on Information and Knowledge Management}{November 2017}{Pan Pacific Singapore} 
\acmYear{2017}
\copyrightyear{2017}
%
%\acmPrice{15.00}


\begin{document}
\title{Enriching \lnear\ Relationship in ConceptNet \\ from Large Text Corpora}
%\titlenote{Produces the permission block, and
%  copyright information}
%
%\subtitlenote{The full version of the author's guide is available as
%  \texttt{acmart.pdf} document}


%\author{Frank F. Xu}
%\authornote{Both authors contributed equally to the paper.}
%\affiliation{%
%	\institution{Shanghai Jiao Tong University}
%}
%\email{frankxu@sjtu.edu.cn}
%
%\author{Bill Y. Lin*}
%\affiliation{%
%	\institution{Shanghai Jiao Tong University}
%}
%\email{yuchenlin@sjtu.edu.cn}
%
%\author{Kenny Q. Zhu}
%\affiliation{%
%	\institution{Shanghai Jiao Tong University}
%}
%\email{kzhu@cs.sjtu.edu.cn}
%
%% The default list of authors is too long for headers}
%\renewcommand{\shortauthors}{F. Xu et al.}
%

\begin{abstract}
\lnear~relation describes two typically co-located objects, which
is a useful commonsense knowledge. 
%Nevertheless, existing commonsense knowledge bases, such as ConceptNet, 
%contain only a limited number of such relations contributed 
%by human annotators. 
We propose to automatically extract such relation through
a sentence level classification problem and the aggregation of
instances detected from large amount of sentences.
%two tasks to this end. One is sentence level
%LocatedNear relation classification and the other is LocatedNear relation 
%extraction which aims at extracting pairs of objects that are commonly
%co-located in real world. 
To enable research of these tasks, we release two benchmark datasets, 
one containing 5,000 annotated sentences extracted from the Guttenberg corpus; 
the other containing 500 pairs of physical objects and whether they are 
commonly located nearby.
We also propose some baseline methods for the tasks and compare the results with
a state-of-the-art general-purpose relation classifier.
\end{abstract}

%
% The code below should be generated by the tool at
% http://dl.acm.org/ccs.cfm
% Please copy and paste the code instead of the example below. 
%
%\begin{CCSXML}
%<ccs2012>
% <concept>
%  <concept_id>10010520.10010553.10010562</concept_id>
%  <concept_desc>Computer systems organization~Embedded systems</concept_desc>
%  <concept_significance>500</concept_significance>
% </concept>
% <concept>
%  <concept_id>10010520.10010575.10010755</concept_id>
%  <concept_desc>Computer systems organization~Redundancy</concept_desc>
%  <concept_significance>300</concept_significance>
% </concept>
% <concept>
%  <concept_id>10010520.10010553.10010554</concept_id>
%  <concept_desc>Computer systems organization~Robotics</concept_desc>
%  <concept_significance>100</concept_significance>
% </concept>
% <concept>
%  <concept_id>10003033.10003083.10003095</concept_id>
%  <concept_desc>Networks~Network reliability</concept_desc>
%  <concept_significance>100</concept_significance>
% </concept>
%</ccs2012>  
%\end{CCSXML}
%
%\ccsdesc[500]{Computer systems organization~Embedded systems}
%\ccsdesc[300]{Computer systems organization~Redundancy}
%\ccsdesc{Computer systems organization~Robotics}
%\ccsdesc[100]{Networks~Network reliability}
%
%
%\keywords{ACM proceedings, \LaTeX, text tagging}
%

\maketitle

\section{Introduction}

Protein$-$protein interactions (PPIs) are of central importance for the majority of biological functions, such as signal transduction, metabolic pathways, molecular dynamics, and protein networks\cite{Hoffmann.Krallinger.ea:2005}, for they serve as the most fundamental building blocks of the entire interacademic systems of any organisms. Collecting data on pairwise interaction relationships is essential for multiple purpose, including identification of modules with certain functionality\cite{Spirin.Mirny.03}, mapping diseases to dominated genes\cite{Ideker.Sharan.08}, and after all, understanding wholistic metabolic/genetic networks from a system biology perspective.

A lot of databases have been built to store protein and genetic interactions from major model organism species and are available in various standardized formats, such as MINT\cite{Zanzoni.Montecchi-Palazzi.ea:2002}, BIND\cite{Bader.ea:2003}, BIOGRID\cite{DBLP:journals/nar/StarkBRBBT06}, etc. Among those mainstream databases, the data largely rely on voluntary reports by scientists or researchers, besides, comprehensive curation efforts become indispensable for the sake of accuracy. However, the amount of biology-related literatures with respect to protein interactions grows explosively and thus make it either impossible or impractical to manually detect PPI information anymore.

Considering huge amount of PPI information with great wealth hidden in published papers, in recent years, numerous mining techniques have been proposed that aim to extract PPI information automatically from free text, especially machine learning, information retrieval, and natural language processing\cite{DBLP:journals/bib/WinnenburgWPDS08}.These approaches can be roughly categorized into three classes: co$-$occurrence, rule$-$based, and machine learning. 

Co$-$occurrence is the approach with most simplicity and naivete. Just as its name implies, this method intends to find out pairs of proteins that co-occur in the same context. The scope of "same context" ranges from phrase, sentence, paragraph to whole abstract, even document. The underlying assumption is that whenever two proteins are mentioned together by authors, chances are high that there is some kind of relationship between them. However, however, in-context closeness even semantic relation does not necessarily represent actual biological interaction. As a consequence, a large fraction of candidate pairs are mismatched inevitably, causing a high recall but low precision.

The second approach is rule-based extraction, in other words, pattern matching. There are many types of rules, most of them concern natural language processing (NLP). One way is to specify hand-crafted regular expressions before hand, which mostly lean on language usage preference. Besides, by using full or partial (shallow) parsing strategies, more information would be acquired, such as part-of-speech taggers, local dependencies between syntactic components, context-free grammar\cite{DBLP:journals/bioinformatics/TemkinG03}, and full sentence structure. Compared to co$-$occurrence, rule-based approach enjoy better precision but much lower recall. In addition, since the rules are usually derived from training data, that is to say, the improper choice of training data would be significantly lethal, therefore quality of extraction is invariably instable and may not applicable to other data.

The third and most commonly used approach use machine learning techniques, in this case, the task to extract protein$-$protein interactions turns out to be a binary classification problem. Each protein pairs are represented along with a set of features, which is associated with their context, then a well$-$defined classifier gives the answer whether the candidate protein pairs is classified to be qualified PPI. (TO BE FURTHER FILLED!!!)

In this paper, we introduce a general bootstrapping framework for Protein$-$protein interaction extraction from natural text.Our method differs from most of the previous works in three aspects:

(1)The extraction process is driven by only tiny fraction of training data, which are regarded as seed data. In each round, it would derive reliable patterns automatically from seed data, then extract more positive PPI pairs consequently, what's more, the seed data would be augmented by the newly extracted results with high confidence.

(2)multiple graph kernel. 

(3)various evaluation.





\section{Proposed Feature-based Transfer Learning Models}
Obtaining the above features for link and time,
we first apply several classic machine learning models for regression that are trained on source areas with above-explored spatiotemporal features and then simply predict traffic condition on target areas as a test.
Afterwards, we present a novel transfer learning approach named CTMP.

\textit{Notations:} a prediction query instance about a link $l$ at the time $t$ is denoted as $(l,t)$; 
the spatial feature vector for the link $l$ is denoted as  $s_l$, and the temporal feature vector of the time $t$ being predicted on the link $l$ is denoted as  $t_l$.
The ground truth of the speed is denoted as $v_l[t]$.
\subsection{Linear Regression Model}
Linear regression (LR) is an approach for modeling the relationship between a scalar dependent variable $y$ and one or more explanatory variables denoted $\mathbf x$. 
%The linear regression method is a typical supervised learning method since $\mathbf x$ and $y$ are all known and the relationships are modeled using linear predictor functions whose unknown model parameters are estimated from the data. 
The model can be expressed in the following form:
$y = \mathbf{w}^T \mathbf{x} + b$ ,where $w$ and $b$ are the parameters we should learn in order to optimize a particular loss. 
%Many methods can be adopted, such as the gradient descent method, to find the proper parameter of the linear model.
In our case, we regard the $\mathbf x$ for each query instance as the concatenation of the spatial feature $s_l$ and the temporal feature $t_l$. 
Thus, we have $\mathbf{x} = [s_l;t_l]$.



\subsection{Neural Network Model}
%Neural network is a kind of computational model widely used in 
%machine learning, computer science and other research disciplines. 
%Recent research indicates that traditional machine learning methods are not sufficiently capable of extracting suitable features and capturing the non-linear nature of complex tasks. 
%Neural network models are presented as a remedy. 
%Fig. \ref{fig:nn} shows the structure of a typical three-layer neural network. 
%Each connection between (two) neurons can transmit an unidirectional signal with an activating strength that varies with the strength of the connection. 
%As a result, a typical four-layer neural network model can approximate most non-linear functions.
%
We adopt a four-layer neural network model (MLP) with following structure: 
the size of input layer is equal to the dimensionality of $\mathbf{x}$; 
the second layer contains half of it; 
the third is half of the second layer; 
the final output layer contains only one neuron.
The model (NN) takes the same input as before and output a continuous value by the last output layer which is the predicted speed.
%If the combined incoming signals (from potentially many transmitting neurons) are strong enough, the receiving neuron activates and propagates a signal to downstream neurons connected to it.

%\begin{figure}[th!]
%	\centering
%	\includegraphics[width=0.3\textwidth]{figures/Colored_neural_network.png}
%	\caption{Structure of a typical neural network}
%	\label{fig:nn}
%\end{figure}

%Moreover, a threshold may govern each connection and neuron, such that the signal must exceed the limit before propagating. 
%Back propagation is the use of forward stimulation to modify connection weights. 
%Training typically requires several thousand cycles of interaction.
%
%In our work, we ...

\subsection{Support Vector Regression Model}
Support vector regression (SVR) depends only on a subset of the training data, 
because the cost function for building the model ignores any training data close to the model prediction. 
Training the original SVR means solving following optimization problem, where ${\displaystyle \mathbf{x_{i}}}$ is a training sample with target value ${\displaystyle y_{i}}$:
\begin{align*}
\min ~ {\displaystyle {\frac {1}{2}}\|\mathbf{w}\|^{2}}, ~~
\text{subject to} ~{\displaystyle {\begin{cases}y_{i}-\langle \mathbf{w},\mathbf{x_{i}}\rangle -b\leq \varepsilon \\\langle \mathbf{w},\mathbf{x_{i}}\rangle +b-y_{i}\leq \varepsilon \end{cases}}}
\end{align*}

%The inner product plus intercept ${\displaystyle \langle w,x_{i}\rangle +b} $
%is the prediction for that sample,
%and ${\displaystyle \varepsilon }$  
%is a free parameter that serves as a threshold: 
%all predictions have to be within an ${\displaystyle \varepsilon }$ 
%range of the true predictions. Slack variables are usually added into the above to allow for errors and to allow approximation in the case the above problem is infeasible.


%To predict the future traffic speed, we input spatial and temporal features as x and the true traffic speed as y for training. 
%After obtaining the $w$ and $b$, we use this model to predict traffic speed of test data and compare the prediction with true data. 

%\subsection{CTMP: A Clustering-based Transfer Model }
We introduce our novel \textbf{C}lustering-based \textbf{T}ransfer \textbf{M}odel for \textbf{P}rediction  (CTMP), which first clusters links in both source and target areas based on their spatial features and then do time series based prediction for the target links based on neighboring source links with historical data.

\subsection{Intuition Behind the CTMP}
Our intuition behind CTMP is that given a link in target areas with spatial features, we can first find the most similar links in source areas and then leverage the source data to predict the speed of links in target areas.

The assumption here is that links with similar spatial features should also share similar traffic patterns.
However, simply clustering road links based on spatial features performs not very well in practice, because not all the features are equally important and the importances cannot be obtained in such an unsupervised way.
Therefore, we incorporate a regularization term in the distance metric for feature reduction and selection.\footnote{CTMP model can be seen as a combination of clustering and Nadaraya-Watson kernel regression.}

\subsection{Clustering with Regularized Distance Metric}
%\textit{Notations:} a prediction query instance about a link $l$ at the time $t$ is denoted as $(l,t)$; 
%the spatial feature vector for $l$ is denoted as  $s_l$, and the temporal feature vector of the time $t$ being predicted on the link $l$ is denoted as  $t_l$.
%The ground truth of the speed is denoted as $v_l[t]$.
We use the $s_i$ and $s_j$ to denote two spatial feature vectors of any two links $i$ and $j$ respectively.
We capture the distance between the two feature vectors by computing $\text{s\_dis}(i,j) = 1-\cos(s_i,s_j)$.
To regularize the time series similarities between two links, we add a regularization term $\text{t\_dis}(i,j)$, which has multiple options.
A desirable option is the DTW \cite{} similarities between the weekly HAM traffic speed series of the two links.
Thus, the total distance between two links can be regarded as follows, where $\lambda$ is a hyper parameter to control the weight of temporal distance:
$\text{dis}(i,j) =  \text{s\_dis}(i,j) + \lambda \text{t\_dis}(i,j) $.

With such supervision in the source area data,
we can use K-means as our clustering algorithm.
For each query instance $(l,t)$\footnote{The link $l$ has no historical data in the transfer scenario.}, we first find the closest $k$ neighboring source links with historical data $\{l_1,...l_k\}$.
We compute all the distances between them and the target link $l$ respectively, and obtain the set of spatial feature distances $\{\text{dis}(l,l_1),...,\text{dis}(l,l_k)\}$.
Also, we can get the predicted typical traffic speed for such neighboring links based on existing time-series models  at the time $t$: $\{y(l_1,t),...,y(l_k,t)\}$.
Finally, we can compute the predicted result for the query instance $(l,t)$ is:
$$ y(l,t) = \sum_{i=1}^k \left( \frac{\text{dis}(l,l_i)}{\sum_{j=1}^k \text{dis}(l,l_j)} 
y(l_i,t)
\right)$$ 

%\BL{add a figure to illustrate this novel model}

\subsection{Dataset}
\label{sec:data}
To train and evaluate our approach of detecting false rumors, a labeled
data set is needed.
We collect a set of known false rumors from Sina community management center \cite{website:Manage},
which deals with reporting
of issues including various misinformation which we regard as
certified false rumors.
% consists of all kinds of fields, so the diversity of rumors is guaranteed.
There are 11466 reported false rumors between 2012/05/28 and 2014/04/11.
%in the result publication category at that time excluding the those without links for the original message webpages.
%Above rumor data are captured directly from Sina Weibo's mobile website.
Since a rumor must have sufficient circulation, we only keep those false
rumors that have at least 100 reposts, which leaves us with 2601 false rumors
up to 2014/04/11.
%and all their reposting information by the time of being captured excluding abnormal original message links.
%Some previous works \cite{yang2012automatic} also made use of
%Sina Weibo's official acount to collect false rumors.
%But at that time, Sina Weibo had not constructed the community management
%and there was an only official false rumor busting account in
%Sina Weibo posting some identified misinformation to public.
%One problem is that the false rumor reports posted by the account
%had no links to the original messages so they needed to
%construct queries manually to find out the original messages.
Sina Weibo API provides interfaces to capture the information of
original messages as well as their repost messages.
From Sina Weibo API, we captured the post time, post client
and content of 2601 false rumors along with all their reposts.

In the real world, the number of false rumors on Sina Weibo
is much smaller than the number of normal messages (1 out of 9 or less).
Thus a ``dummy'' classifier that rules all messages as normal messages
will achieve a very high accuracy (above 90\%) on real-world data.
To avoid this problem, we construct a data set with roughly equal number of
false rumors and normal messages. Most studies in the past also use
data sets which are either 50-50 split
\cite{castillo2011information,jin2013epidemiological}
or close to that \cite{yang2012automatic,qazvinian2011rumor}.
Thus, we randomly select 5000 other Weibo original messages
which are not proved to be false as well as their reposts
using the Sina Weibo API. Then, we manually filtered out messages with fewer than 100 reposts as well as false rumors to form a set of 2536 normal messages.
%To make non-rumors be in accord with rumors, we also only select the tweets that have 100 reposts at least here. The profiles of users involved are included in the data captured from API. Afterwards, we labeled 2844 pieces of non-rumors from them manually.
Each message or repost contains links to the author profile
information such as age, gender, number of followers and friends,
and can be crawled using the Weibo API.
%
%Sina Weibo provides API to capture a user's information but the speed is too slow because of frequency restriction. So we capture the original poster's information through Sina Weibo API and the other users' information directly from their homepages on Sine Weibo's mobile website.
%

At the end of this phase, our labeled data set 
\footnote{The labeled data set of the original messages (without reposts)
is available at \url{http://adapt.seiee.sjtu.edu.cn/~kzhu/rumor/}.}
consists of 2601 false rumors, 2536 normal messages
and with 4 million distinct users involved in these messages. Of these
500 false rumors and 500 other messages (called small data set) are used for
SVM parameter tuning while
the rest (called big data set) are used for end-to-end cross validation.


\section{Experiment}
In this section, we experiment on different NLG tasks. We first present the experimental setup on different tasks. Then, we show the quantitative and qualitative results together with comprehensive analysis and ablation studies.

\subsection{Implementation Details}
We evaluate the newly proposed ICL strategy on five commonly-researched natural language generation tasks: reading comprehension, dialogue summarization, style transfer, question generation and news summarization. Details on the task description, the strong baseline, corresponding  dataset, evaluation metrics and key hyper-parameters for each task are presented as follows.

\begin{table*}[th]
	\scriptsize
	\centering
	\begin{tabular}{lp{1.1cm}rrrcccc}
		\hline
		Task & Dataset & \#Train & \#Val & \#Test & Input & Output & Avg & Std\\
		\hline
		Reading Comprehension & DREAM & 6,116 & 2,040 & 2,041 & ``Q:''+ question + dialogue & answer & 5.59 & 2.61\\
		Dialogue Summarization & SAMSum & 14,732 & 818 & 819 & dialogue & summary  & 24.99 & 13.06\\
		Style Transfer & Shakespeare & 36,790 & 2,436 & 2,924 & original/modern  & modern/original  & 11.63 & 8.19 \\
		Question Generation & SQuAD1.1 & 75,722 & 10,570 & 11,877 & passage + [SEP] + answer & question & 13.09 & 4.27 \\
		News Summarization & CNNDM & 287,227& 13,368& 11,490 & document & summary & 70.97 & 29.59\\ 
		\hline
	\end{tabular}
	\caption{A summary of tasks and datasets. \#Train, \#Val and \#Test refers to the number of samples in the corresponding dataset. Avg and Std are the statistics for the number of output tokens. ``+'' refers to the concatenation operation.}
	\label{tab:taskdata}
\end{table*}

\textbf{Reading comprehension} is the task that answering questions about a piece of text. We use the DREAM dataset~\cite{sun2019dream} where questions are about corresponding dialogues and the answer is a complete sentence in natural language. We neglect the negative choices in the original dataset and formulate it as a NLG task. We adopt the pre-trained language model BART~\cite{lewis2020bart} as the baseline, where the input is a concatenation of a question and the corresponding dialogue made up of speakers and utterances. 
We experiment with  transformers\footnote{\url{https://github.com/huggingface/transformers}} based on the publically available ``facebook/bart-large'' checkpoint \footnote{\url{https://huggingface.co/facebook/bart-large}}.
%The preceding BART model is also adopted as the baseline, whereas the input is a concatenation of question and a dialogue.
The generated answers are evaluated by BLEU scores\footnote{The BLEU-1/2/3/4 scores are computed according the Google's implementation(\url{https://github.com/tensorflow/nmt/blob/master/nmt/scripts/bleu.py}).}~\cite{papineni2002bleu} widely used for QA systems, together with Meteor and Rouge-L F1 as mentioned above. The parameters are also the same as dialogue summarization, except that the early-stop is activated if there is no improvement on the perplexity of the validation set. 


\textbf{Dialogue summarization} is to generate a concise summary covering the salient information in the input dialogue. The preceding model BART has shown to be a strong baseline for this task, where only the dialogue is concatenated into a single sequence as the input. We experiment with  %transformers\footnote{\url{https://github.com/huggingface/transformers}} based on the publically available ``facebook/bart-large'' checkpoint \footnote{\url{https://huggingface.co/facebook/bart-large}} and 
SAMSum dataset\footnote{\url{https://arxiv.org/src/1911.12237v2/anc/corpus.7z}}~\cite{gliwa2019samsum} for daily-chat dialogues. 
The generated summaries are evaluated by comparing with the reference through evaluation metrics, including Rouge-1/2/L F1 scores\footnote{\url{https://github.com/pltrdy/files2rouge}}~\cite{lin2004rouge}, Meteor~\cite{banerjee2005meteor} and BertScore F1\footnote{Both Meteor and BertScore are calculated by SummEval(\url{https://github.com/Yale-LILY/SummEval}), and the latter one is based on the default bert-base-uncased model.}. We evaluate the model on the validation set after each training epoch and the early-stop patience will be added 1 if there is no improvement according to the Rouge-2 F1 score. The training process terminates when the early-stop patience equals or is larger than 3.  During the inference, the minimum and maximum output length is set to 5 and 100 respectively, with no\_repeat\_ngram\_size=3, length\_penalty=1.0 and num\_beams=4.


% The answer is either a span of words in the original text or a complete sentence in natural language.
\textbf{Style transfer} preserves the semantic meaning of a given sentence while modifies it's style, such as positive to negative, formal to informal, etc.
We adopt the Shakespeare author imitation dataset~\cite{xu2012paraphrasing}, containing William Shakespeare's original plays and corresponding modernized versions. Krishna el al.~\shortcite{krishna2020reformulating} proposed to do unsupervised style transfer by training paraphrase models based on the GPT-2 language model~\cite{radford2019language}. We re-implemented their approach STRAT\footnote{\url{https://github.com/martiansideofthemoon/style-transfer-paraphrase}} and evaluated with the provided script. Evaluation metrics includes 
transfer accuracy(ACC), semantic similarity(SIM), Fluency(FL) and two aggregation metrics, i.e., geometric averaging(GM) and their newly introduced $J(\cdot)$ metric. The hyper-parameter $hp$ equaling 0.0, 0.6 or 0.9  in Table~\ref{tab:end2endst} is the sampling parameter for trades off between ACC and SIM in their approach. 
In the training stage, we evaluate the model after updating every 500 steps. The perplexity on the validation set is used to activate the early-stop which equals 3. The inference is done as default.
 
\textbf{Question generation}~\cite{zhou2017neural} aims at generating a question given an input document and its corresponding answer span. SQuAD 1.1~\cite{rajpurkar2016squad} is generally used for evaluation. We adopt the data split as in \cite{du2017learning} and fine-tune the pre-trained UniLM~\cite{dong2019unified} as the strong baseline according to their official implementation\footnote{\url{https://github.com/microsoft/unilm/tree/master/unilm-v1}}. Generated questions are evaluated by metrics including BLEU-1/2/3/4, Meteor and Rouge-L with the provided scripts. The model is evaluated every 1000 steps and the early-stop equaling 3 is associated with the perplexity on the validation set. Other parameters are unchanged following the official guideline.

\textbf{News summarization} differs from dialogue summarization where the input is a document instead of a dialogue. We adopt the same strong baseline BART and evaluation metrics as dialogue summarization. Experiments are done with CNNDM dataset~\cite{HermannKGEKSB15} consisting of news articles and multi-sentence summaries\footnote{\url{https://github.com/pytorch/fairseq/blob/main/examples/bart/README.summarization.md}}. The model is evaluated every 2000 steps and the early-stop equaling 3 is associated with the Rouge-2 on the validation set. During the inference, the minimum and maximum output length is set to 45 and 140 respectively, with no\_repeat\_ngram\_size=3, length\_penalty=2.0 and num\_beams=4.
%\footnote{Inference parameters are borrowed from \url{https://github.com/pytorch/fairseq/blob/main/examples/bart/summarize.py}}

The summary of each task is listed in Table~\ref{tab:taskdata}. For fair comparisons, we re-implemented baselines following the above instructions on our machine. On top of the above baselines, we further arm them with the ICL strategy according to the Algorithm~\ref{alg:picl}. The settings of newly introduce Start and Stride are specified and discussed in following sub-sections. All of our experiments are done on a single RTX 3090 or a single RTX 2080Ti with 24G and 11G GPU memory respectively.
%and the result are averaged over three runs.


 
\subsection{Automatic Evaluations on Different Tasks}
\label{sec:taskperformances}

We compare our approach with the vanilla models mentioned above and the approach from~\citet{liang-etal-2021-token-wise} as baselines.
The performances on different NLG tasks are shown in Table~\ref{tab:end2end}. 
These tasks not only focus on solving different problems, but also has various amount of training data as well
as reference output lengths as shown
Table~\ref{tab:taskdata}.
Besides, the basic model are also different, including BART, GPT-2 and UniLM. 
Our new training strategy achieves significantly improvements among different tasks on most evaluation metrics, which shows that our method not only works well, but also has strong generalization abilities.

We explain the some specific results as follows:

(1) Our training strategy boosts the performances of the original STRAT with different $hp$ in the style transfer task. GM and J are two comprehensive evaluation metrics, with our approach topping the ranks with significant improvements.

(2) TCL generally performs poorly on tasks
with more training data. For example, it failed on question generation without any improvements over the vanilla model under the same parameter setting, while ICL still 
logs gains. This is mainly due to two reasons.
First, because the nature of TCL is data augmentation which is more effective in low-resource settings,
when training data is abundant, it becomes less useful. 
Second, the way they calculate the loss as sub-sequence generation better suites paraphrasing tasks, such as machine translation tested in their paper, as the order of 
the corresponding tokens between input and output 
are almost the same. Learning such forward mapping can 
be regarded as a kind of ``easy-to-hard'' 
in these limited scenarios.
However, this doesn't hold true for other tasks, 
such as summarization and question generation. 
Therefore, we didn't further test it on CNNDM since
CNNDM has the large amount of training data among
the five.

(3) For news summarization, Rouge-1 scores (precision, recall) for the baseline and our method on CNNDM are (38.16, 52.72) and (40.84, 49.23) correspondingly. Our method made substantial improvements on the precision with a compromise on the recall. 
The meteor score based on the unigram precision and recall emphasizes more on the recall than the Rouge-1 F1. As a result, it drops while Rouge-1 F1 increases. Overall, our method still outperforms BART on this task, especially on F1 scores of Rouge-2 and Rouge-L.




\begin{table}[th]
	\small
	\centering
	\begin{subtable}{\linewidth}
		\scriptsize
		\centering
		\begin{tabular}{lcccccc}
			\hline
			{Method} & {B1} & {B2} & {B3} & {B4} & {Met} & {RL}\\
			\hline
			w/o CL &  32.03 & 16.01 & 8.77 & \textbf{4.80} & 19.84 & 38.89\\
			TCL & 32.53 & 16.25 & 8.52 &4.67 &19.88 & 39.65 \\
			ICL &  \underline{\textbf{33.99}} & \underline{\textbf{17.43}} & \underline{\textbf{9.18 }}& 4.64 & \textbf{20.60} & \textbf{40.78}\\

			\hline
		\end{tabular}
		\caption{Reading Comprehension}
		\label{tab:end2endrc}
	\end{subtable}
	\\[5pt]
	\begin{subtable}{\linewidth}
		\scriptsize
		\centering
		\begin{tabular}{lccccc}
			\hline
			{Method} & {R1} & {R2} & {RL} & {Met} & {BertS} \\
			\hline
			%BART & 52.60&27.00 &42.10 &- & - \\
			w/o CL & 51.88 & 27.30 & 42.77 & 24.75 & 71.38 \\
			TCL  & 52.33 & 27.80 & \textbf{43.91} & 24.59 & 71.77 \\
			ICL & \underline{\textbf{53.07}} & \underline{\textbf{28.23}} & {43.83} & \underline{\textbf{26.12}}& \underline{\textbf{72.17}} \\
			
			\hline
		\end{tabular}
		\caption{Dialogue Summarization}
		\label{tab:end2endds}
	\end{subtable}
	\\[5pt]
	\begin{subtable}{\linewidth}
		\scriptsize
		\centering
		\begin{tabular}{lcccccc}
			
			\hline
			{Method}&$hp$ &  {ACC} & {SIM} & {FL} & {GM} & {J}\\
			\hline
			%\multirow{3}{*}{STRAT}& 0.0 & 71.70 & \textbf{56.40} & 85.20 & 70.10 & 34.70 \\
			%& 0.6 & 75.70 & 53.70 & 82.70 & 69.50 & 33.50 \\
			%& 0.9 & 79.80 & 47.60 & 71.70 & 64.80 & 27.50 \\
			%\hline
			\multirow{3}{*}{w/o CL}& 0.0 & 70.49 & 55.70 & 85.98 & 69.63& 33.72 \\
			& 0.6 &75.31 & 53.46 & 82.56 & 69.27& 33.30\\
			& 0.9 & 78.76 & 47.38 & 74.42 &65.24 & 27.88\\
						\hline
			\multirow{3}{*}{TCL } & 0.0 & 70.31 & \textbf{55.95} &\textbf{87.24} &  70.01& 34.71 \\
			& 0.6 & 74.79 & 53.14 & 82.56 & 68.97 & 33.21 \\
			& 0.9 & 79.41 & 46.88 & 71.92 &64.45 & 26.92 \\
			\hline
			\multirow{3}{*}{ICL}& 0.0 & \underline{73.72} & 55.91 & 86.30 & \underline{\textbf{70.60}} &\underline{\textbf{35.81}}\\
			& 0.6 & 77.26 & \underline{53.80} & \underline{83.87} & \underline{70.38} & 34.64\\
			& 0.9 & \textbf{79.65} & 48.16 & 76.06 & 66.32 & 29.03\\

			\hline
		\end{tabular}
		\caption{Style Transfer.}
		\label{tab:end2endst}
	\end{subtable}
	\\[5pt]
	\begin{subtable}{\linewidth}
		\scriptsize
		\centering
		\begin{tabular}{lcccccc}
			\hline
			{Method} & {B1} & {B2} & {B3} & {B4} & {Met} & {RL}\\
			\hline
			w/o CL & \textbf{50.38} & 35.67 & 27.24 & 21.36 & 24.40 & 50.67 \\
			TCL &\textbf{50.38} & 35.67 & 27.24 & 21.36 & 24.40 & 50.67\\
			ICL &  50.18 & \textbf{35.72} & \textbf{27.36} & \textbf{21.54} & \textbf{24.57} & \underline{\textbf{51.09}} \\
			\hline
		\end{tabular}
		\caption{Question Generation}
		\label{tab:end2endqg}
	\end{subtable}
		\\[5pt]
	\begin{subtable}{\linewidth}
		\scriptsize
		\centering
		\begin{tabular}{lccccc}
			\hline
			{Method} & {R1} & {R2} & {RL} & {Met} & {BertS}\\
			\hline
			%BART &  \\
			w/o CL &  43.07 & 20.01 & 35.94 & \textbf{21.44} & 63.72 \\
			TCL & - & -&- &- &- \\
			ICL & \textbf{43.39} & \underline{\textbf{20.55}} & \underline{\textbf{36.63}} & 19.68 & \textbf{64.05}\\
			\hline
		\end{tabular}
		\caption{News Summarization}
		\label{tab:end2endns}
	\end{subtable}
	\caption{Performances on different NLG tasks. ICL represents the models trained with our ICL algorithm. TCL refers to the previous work from~\cite{liang-etal-2021-token-wise}. Scores underlined are statistically significantly better than both re-implemented baselines with $p<0.05$ according to t-test. }	
	\label{tab:end2end}
\end{table}


\subsection{Human Evaluations}

To further prove the improvement of ICL, we hired three proficient English speakers for human evaluation. 20 samples from the test set of each task are randomly selected, ignoring the ones with totally same generations among three models, including the vanilla model, TCL and ICL. The original input, reference output and three generations are shown to annotators together, while the order of three generations are unknown and different among samples. 3-point Likert Scale is adopted for scoring for each generation~\cite{gliwa2019samsum}, where [1, 3, 5] represent 
excellent, moderate and disappointing results 
respectively. The average scores and agreements 
among the annotators are shown in 
Table~\ref{tab:humaneval}.

The Fleiss Kappa on the first four tasks indicates the fair to moderate agreements. It shows the promising improvement of ICL over the vanilla model and TCL especially on DREAM, SAMSum, and SQuAD1.1, which is consistent with the conclusion based on automatic metrics.
Although the agreement on style transfer is fair, 
our annotators without Shakespeare background 
tend to give low scores to all outputs.
Therefore, the absolute improvement is 
only $0.04$ compared to both baselines.
%This mainly due to the indistinguishable styles between
%Shakespeare’s plays with are quite different from modern languages. 
Besides, the poor agreement on CNNDM reflects the 
diverse concerns of summarization from different 
annotators. Without more specific instructions, they 
tends to focus more on the content coverage instead 
of checking the detailed facts. This is also 
consistent with the higher Meteor scores of the 
vanilla model over ICL.

\begin{table}[th]
	\scriptsize
	\centering
	\begin{tabular}{l|ccc|c}
		\hline
		{Datasets} & {w/o CL} & {TCL} & {ICL} & {Agreement}  \\
		\hline
		DREAM  &3.07 & 2.50&3.20 &0.48 \\
		SAMSum &2.97 &3.57 &3.97 &0.40 \\
		Shakespeare &2.23 &2.23 & 2.27&0.32 \\
		SQuAD1.1 &3.43 & 3.43 &3.77 &0.35 \\
		CNNDM & 3.45 &- &3.40 &0.11 \\
	%	\hline
	%	overall & & & &\\
		\hline
	\end{tabular}
	\caption{Human evaluations. The agreement is calculated by Fleiss Kappa.}
	\label{tab:humaneval}
\end{table}




%Following Liu et al.\shortcite{liu2021competence}'s work, we asked annotators to comparing the performance between our generated results and baselines by choosing from ``Better, Tie, Worse''. 
%The counts for each choice are shown in Table~\cite{}, where the Fleiss Kappa among annotators is ??.

%Analysis





%\subsection{Analysis on Variable Generation Lengths}

%Teacher forcing, which predicts each token given the reference summary tokens during training and given the previous generated tokens during inference, leads to the exposure bias problem for NLG tasks.
%Since ICL starts the training process by predicting the last few tokens of outputs and gradually calculates the loss based on more tokens when the model is stronger, we hypothesis that it can alleviate the exposure bias for training Seq2Seq models to some extent.
%As stated in~\cite{pang2020text}, the output quality tends to degrade as the output length increase with the exposure bias.
%So, we divided the test set of each task according to the length of the generated output into 4 buckets and randomly picked 20 samples in each buckets for both the corresponding baselines and our approach. Each generation is annotated by 5 point Likert Scale, where 1 is the worst and 5 is the best. 

%The trends of performances on variable generation lengths are in Figure~\ref{}.



%\section{Related Work}
This section surveys previous works on question generation and tree encoding
respectively.

Text question generation has attracted the attention 
after the work of ~\citeauthor{du2017learning}~\shortcite{du2017learning}, who uses deep seq2seq model 
to generate questions from a raw text paragraph. 
Before that, text question generation relied heavily on hand-craft 
question patterns~\cite{HeilmanS10,LabutovBV15,MostowC09} which is time and 
labor consuming. 

However, this pure seq2seq model is not focused and 
has no control over part in the paragraph to generate question. 
~\citeauthor{zhou2017neural}~\shortcite{zhou2017neural} proposed to encode 
key phrase information using binary indicators to generate 
key-aware questions and they assumes the answer to be key phrase. 
Considering key phrase (answer) is unavailable in reality, 
~\citeauthor{SubramanianWYT17}~\shortcite{SubramanianWYT17} applied 
a two-stage approach. First, key phrases are extracted by 
pointer network~\cite{ptrnet}. Second, 
key phrases are encoded in the same way as 
Zhou et al. With the intuition that questions could be asked in many ways, 
~\citeauthor{Yao2018vae}~\shortcite{Yao2018vae} used conditional-VAE to 
increase the diversity of questions. More recently, models with 
auxiliary feature information~\cite{HarrisonW18} helped improve 
the question quality. Structure question generation aims at 
converting structured data such as triples in knowledge graph to questions. 
~\citeauthor{SerbanGGACCB16}~\shortcite{SerbanGGACCB16} proposed a model to generate factoid questions from knowledge base triples.  None of the above work
considered using parse tree structures to aid question generation process,
which is the focus of this paper.

Sequential RNN model takes sentence as a sequence of words, 
ignoring the syntactic information. In order to utilize
such syntactic information with sequential information, 
~\citeauthor{tai2015improved}~\shortcite{tai2015improved} proposed Tree-LSTM to 
encode the binary parse tree recursively in a bottom-up fashion to 
classify sentiment. In text generation task, 
\citeauthor{eriguchi2016tree}~\shortcite{eriguchi2016tree} 
proposed a tree-to-sequence model with attention mechanism to do 
machine translation and 
~\citeauthor{liang2018automatic}~\shortcite{liang2018automatic} proposed a 
tree-to-sequence model which could handle arbitrary trees, 
to do code comment generation. Our work is inspired by these previous
attempts and we are first to adapt structure encoded neural models to
textual question generations.

\section{Conclusion}
\label{sec:conclude}
In this work,
we propose a new data creation method to generate
 a semi-structured synthetic training data for 
opinion summarization,
which is known for lacking training data.
\cut{We showed that by extracting an aspect-opinion pairs and 
implicit sentences from multiple reviews
first and then synthesizing them into semi-structured data, we achieve
better performance on opinion summarization.}
%\KZ{It is critical to show in your experiments that the proposed
%synthetic data is better than other possible alternatives.}, 
We also designed an aspect-guided model with opinion-aspect pair encoder and implicit sentence encoder.
The results showed that
the proposed model can make full use of semi-structured data
and generate high-quality summaries.





\bibliographystyle{ACM-Reference-Format}
\bibliography{sigproc} 

\end{document}
