\section{Experiments}
\label{sec:experiments}

In this section, we present the datasets and baseline models first. Then we introduce our setup 
and finally analyze the experimental results.

\subsection{Datasets}
\label{sec:dataset}

\subsubsection{Superficial Patterns}
\label{sec:patterns}
We choose three datasets, MNLI, SNLI and FEVER, for experiment since 
errors have been identified and analyzed in these dataset~\cite{} and 
we can easily detect the superficial patterns from the errors. 
As shown in~\tabref{tab:cues}, the errors can be roughly divided into 2 aspects: 
\YZ{These two aspects are not clear.}
due to superficial patterns like negation, word overlap, length mismatch 
grammaticality and require higher comprehension and reasoning capacity, like 
models are insensitive to antonym and numbers. 
Similar to MNLI but without fine-grained partition \YZ{The MNLI without fine-grained partition?}
, SNLI and FEVER tasks also suffer 
from superficial patterns like unbalance distribution of n-gram~\cite{} and
cross-n-gram tokens (negation preference is also caused by unbalance distribution). 
In these spurious \YZ{superficial?} patterns, length mismatch and 
grammaticality are easier to detect and revise. Thus, we pay more attention on 
how to fight against the spurious features taken by unbalanced tokens and word overlap. 
Another evidence for the existence of spurious features is hypothesis-only classifiers
achieves mostly above 60\%, far above the majority
baseline (33.3\%).


\begin{table}[]
\centering
\scriptsize
\setlength{\tabcolsep}{0.23mm}{
\begin{tabular}{|l|c|l|l|l|c|}
\hline
\multicolumn{2}{|c|}{Superficial pattern}                          & \multicolumn{1}{c|}{Definition}                                                                             & \multicolumn{1}{c|}{Premise}        & \multicolumn{1}{c|}{Hypothesis}        & Error Type \\ \hline
\multirow{3}{*}{Overlap} & \multicolumn{1}{l|}{Lexical overlap} &\tabincell{l}{Assume that a premise entails all \\hypotheses constructed from words \\in the premise.}                         & The doctor was paid by the actor.   & The doctor paid the actor.             & C$\rightarrow$E        \\ \cline{2-6} 
                         & \multicolumn{1}{l|}{Subsequence}     & \tabincell{l}{Assume that a premise entails all \\of its contiguous subsequences.  }                                         & The doctor near the actor danced.   & The actor danced.                      & N$\rightarrow$E        \\ \cline{2-6} 
                         & \multicolumn{1}{l|}{Constituent}     & \tabincell{l}{Assume that a premise entails all \\complete subtrees in its parse tree. }                                     & If the artist slept, the actor ran. & The artist slept.                      & N$\rightarrow$E        \\ \hline
Negation                 & -                                    & \tabincell{l}{Strong negation words (“no”, “not”) \\cause the model to predict contradiction.}                           & The doctor was paid by the actor.   & The doctor did not like the actor.     & N$\rightarrow$C        \\ \hline
Antonym                  & -                                    & \tabincell{l}{Premise-hypothesis pairs containing \\antonyms are  not detected as \\contradiction by the model.}               & The doctor liked the actor.         & The doctor hated the actor.            & C$\rightarrow$E        \\ \hline
Numerical Reasoning      & -                                    & \tabincell{l}{Model is unable to perform reasoning \\involving numbers or quantifiers for \\correct relation prediction.}      & The actor paid the doctor \$300.    & The doctor was paid no more than \$350 & E$\rightarrow$C        \\ \hline
Length Mismatch          & -                                    & \tabincell{l}{The premise is much longer than \\the hypothesis and this information \\is used to distract models. }            & The doctor danced...(50 words).     & The doctor slept.                      & C$\rightarrow$E        \\ \hline
Grammaticality           & -                                    & \tabincell{l}{The premise or the hypothesis \\is ill-formed because of spelling \\errors or incorrect subject-verb \\agreement.} & The doctor rean the actor sang.     & The doctor danced.                     & N$\rightarrow$C        \\ \hline
\end{tabular}}
	\caption{The definition and examples of superficial patterns which are found in MNLI. 
	The Error Type is from the gold label to predicted label. E=entailment, C=Contradiction and N=Neutral}
	\label{tab:cues}
\end{table}

\subsubsection{Adversarial Test data}
To evaluate models on a benchmark without spurious features, 
we choose to use adversarial test data which
 evaluates the models' performance at semantic level, independent shallow superficial patterns. 
Models are trained on the given datasets and tested on 
the adversarial test data with zero-shot to
 elucidate capabilities and limitation of models.
 
For MNLI, we consider Stress Test~\cite{} and HANs~\cite{} as adversarial 
test datasets. Stress Test is an evaluation method that helps to examine whether the
models can predict \YZ{predict what?} at semantic level. 
They \YZ{'They' denotes stress test? Maybe 'It'?} created a test
set that is constructed following a variety of different rules,
including competence test set (antonym, numerical reasoning),
distraction test set (with three strategies: word overlap,
negation, and length mismatch). 
HANs focus on overlap issues and 
generate test samples with 3 logic rules. 
We use Adversarial-NLI~\cite{} as the test set for SNLI. 
This test set is collected by an iterative, adversarial human-and-model-in-the-loop 
solution relying on different models and context situations. 
R1 used a BERT-Large model~\cite{} trained on a concatenation of SNLI and MNLI
\YZ{trained on the dataset including SNLI and MNLI?}. 
R2 used a more powerful RoBERTa model. R3 selected a more diverse set of contexts
in order to explore robustness under domain transfer. 

For FEVER, we follow \YZ{author}~\shortcite{} and manually produces a synthetic pair that holds the same relation (e.g.
SUPPORTS or REFUTES) but expressing a different fact \YZ{One pair holds multiple relations?} which means a certain claim 
corresponding to two different evidence and two different labels.
This dataset can guarantee that if only watch claim (hypothesis-only), we 
can not always choose the right alternatives.

\subsection{Baseline Models}
\YZ{You use present tense to introduce existing models at this section but use past tense at last section. Are R1 and others existing methods?}
\textbf{BERT} applies the bidirectional training of Transformer\YZ{cite}(Vaswani et al., 2017), 
a popular attention model, to language modeling. 
\YZ{BERT is a popular attention model, which applies ...}
It is trained on BooksCorpus (Zhu et al., 2015) and English Wikipedia in two unsupervised
tasks, i.e., Masked LM (MLM) and Next Sentence Prediction (NSP). During fine-tuning, the final
hidden vector corresponding to the first input token ([CLS]) is used as the aggregate representation
followed by an extra fully connected layer to compute the score.

\textbf{RoBERTa} is an improved pre-training procedure of BERT. 
RoBERTa removes the Next Sentence Prediction (NSP) task from 
BERT’s pre-training \YZ{BERT is pre-training model?} and introduces dynamic masking so that the 
masked token changes during training. 
Larger batch-training sizes are more useful in the training procedure.

For these two pre-training models, we use the code of PyTorch-Transformers of Hugging Face to implement them on SNLI, MNLI and FEVER. We take batch size as 64 and fine-tune for 3 epochs. The maximum input sequence length for all
models is 128. In this paper, we use the basic version of above models. 

\subsection{Experimental Result}
We apply our three kinds of augmentation data on three tasks: SNLI, MNLI and FEVER. 
The number of balance instances is fixed which have 
been introduced in~\secref{sec:approach}. 
For hypothesis-only and overlap noise data, we try different proportions 
of noise data for training models. The data unit (100\%) is defined as
the smallest number of examples with a certain label among all labeled examples. 
For example, the training data for FEVER is consist of 80,035 \textit{supported} examples, 
29,775 \textit{refuted} examples and 35,639 \textit{not enough information} examples. 
The unit FEVER is 29,775.
We trained the models with original dataset augmented by four different cases (i.e. when
0\%/50\%/100\%/200\%/300\% of a unit).
To show the effectiveness of our noise data augmentation methods, 
each model will test both on original test (or validation 
dataset) and adversarial dataset. 


\subsubsection{Original Results}

The performance of all tested models on the three datasets 
is presented in \tabref{tab:original}.
The performance of RoBerta  and BERT training with noise data 
is consistent with the original ones and even slightly improved. 
BERT with 100\% hypothesis-only noise data get 0.75 point improvement on MNLI 
and increase 0.41 point with 300\% overlap noise data on SNLI. 
RoBERTa achieves 88.32\% accuracy on FEVER which is about 0.5\% point 
higher than original dataset training. Though these improvement is small, it at least shows 
these noise augmentation methods are harmless for the pretraining models.

%astText is better than random guess, showing that word correlation could be
%used to help improve performance to some extent. It is difficult for Bi-LSTM to converge on this
%dataset. Transformer-based pre-training models have relatively good performance, close to the performance
%of graduate students. However, we find that these models only perform well on EASY set
%with around 70% accuracy, showing these models have an outstanding ability to capture the biases
%of the dataset, but they perform poorly on HARD set with only around 30% accuracy. In contrast,
%humans can still keep good performance on HARD set. We notice the difference in testing accuracy
%performed by graduate students on EASY and HARD set, but this could be due to the small number
%of students participated in the experiments. Therefore, we say humans perform relatively consistent
%on both biased and non-biased dataset

\begin{table}[]
	\centering
	\scriptsize
\begin{tabular}{|c|c|c|c|c|c|c|c|c|c|c|c|c|c|}
\hline
\multirow{2}{*}{Dataset} & \multirow{2}{*}{Model}   & \multirow{2}{*}{Test data} & \multirow{2}{*}{Original} & \multirow{2}{*}{O-HO} & \multirow{2}{*}{Balance Noise} & \multicolumn{4}{c|}{Hypothesis-only Noise}                        & \multicolumn{4}{c|}{Overlap Noise}                                \\ \cline{7-14} 
                         &                          &                            &                           &                              &                                & 50\%           & 100\%          & 200\%          & 300\%          & 50\%           & 100\%          & 200\%          & 300\%          \\ \hline
\multirow{4}{*}{MNLI}    & \multirow{2}{*}{BERT}    & Matched                    & 83.42                     & 60                           & \textbf{83.65}                 & \textbf{83.84} & \textbf{84.16} & 83.37          & \textbf{83.83} & \textbf{83.79} & \textbf{83.69} & \textbf{83.99} & \textbf{83.87} \\ \cline{3-14} 
                         &                          & Mismatched                 & 84.06                     & 58.84                        & 83.69                          & 83.93          & \textbf{84.14} & 84.05          & \textbf{84.11} & 83.82          & \textbf{84.28} & \textbf{84.2}  & \textbf{84.12} \\ \cline{2-14} 
                         & \multirow{2}{*}{RoBERTa} & Matched                    & 87.47                     & 61.16                        & 87.36                          & 87.32          & 87.15          & 87.28          & 87.2           & 87.38          & 87.45          & \textbf{87.56} & 87.41          \\ \cline{3-14} 
                         &                          & Mismatched                 & \textbf{87.59}            & 60.03                        & 87.23                          & 87.17          & 87.11          & 87.36          & 87.08          & 87.31          & 87.15          & 87.29          & 87.21          \\ \hline
\multirow{2}{*}{SNLI}    & BERT                     & Test                       & 90.42                     & 70.89                        & \textbf{90.48}                 & \textbf{90.73} & \textbf{90.48} & \textbf{90.52} & \textbf{90.54} & \textbf{90.65} & \textbf{90.44} & \textbf{90.46} & \textbf{90.83} \\ \cline{2-14} 
                         & RoBERTa                  & Test                       & 91.61                     & 70.82                        & 91.03                          & 91.59          & \textbf{91.72} & 91.43          & 91.52          & \textbf{91.63} & 91.34          & 91.46          & 91.09          \\ \hline
\multirow{2}{*}{FEVER}   & BERT                     & Dev                        & 85.86                     & 68.87                        & 85.78                          & \textbf{86.25} & \textbf{86.21} & \textbf{86.29} & 85.47          & \textbf{86.19} & \textbf{86.38} & 85.74          & 85.74          \\ \cline{2-14} 
                         & RoBERTa                  & Dev                        & 87.83                     & 67.88                        & 87.36                          & \textbf{88.03} & 87.65          & 87.45          & 87.18          & \textbf{88.32} & 87.61          & 87.75          & 87.37          \\ \hline
\end{tabular}
	\caption{Train models with three datasets and different noise data then test 
	on original test data or validation data(dev). O-HO=original hypothesis-only}
	\label{tab:original}
\end{table}




\subsubsection{Adversarial Test Results}

\textbf{HANs} 

By looking into the Ratio rows of~\tabref{tab:HANS}, we can observe
that different noise augmentation methods with different proportions 
have diverse effects on models.
\textit{Entailed} passive examples include the instances with \textit{entailment} 
relation, versus \textit{non-entailed}. 
We can see models trained on original MNLI dataset are more likely 
to choose \textit{entailment} relation on \textit{entailed} and \textit{non-entailed} examples. 
They are confused on this new test data which was designed for overlap problems 
with about 17\% on RoBERTa and 30\% on BERT accuracy drop 
(overall score on HANs compared with original test data) . 
Any noise data augmentation method have consistent performance with 
original ones on \textit{entailed} approaching 100\% accuracy. 
For the performance on \textit{non-entailed} examples which has wide gap 
with entailed examples, the results of BERT training with noise data all have 
improvement on these examples. The main contribution comes from the 
the better performance on \textit{lexical overlap}. It is predictive because 
we design noise data with lexical disturb which doesn't include subsequence or 
constituent information. The overall score exceed 20\% improvement with balance noise, 
5\%, 100\%, 200\%  hypothesis-only noise and 100\% overlap noise. The results indicate 
our noise data augmentation methods can help BERT learn more semantic knowledge 
and more robust which can not be easily mislead by overlap spurious features.

RoBERTa has relatively better reasoning ability in this area with original training data. 
The performance gap on \textit{entailed} and \textit{non-entailed} examples are smaller. 
However there is still space for improvement with noise augmentation data. 
The lexical overlap score can also raise above 10\% by adding 300\% hypothesis-only noise data 
or 50\% overlap noise data. Training with 50\% overlap noise data can get 
the best performance on overall test in HANs with 5\% up-grading (The improvement 
score are based on ratio). 





\begin{table}[]
\centering
\scriptsize
\begin{tabular}{|c|c|c|c|c|c|c|c|c|c|c|c|c|}
\hline
\multirow{2}{*}{Model}    & \multicolumn{2}{c|}{\multirow{2}{*}{Test Type}} & \multirow{2}{*}{Original} & \multirow{2}{*}{Balance Noise} & \multicolumn{4}{c|}{Hypothesis Noise} & \multicolumn{4}{c|}{Overlap Noise} \\ \cline{6-13} 
                          & \multicolumn{2}{c|}{}                           &                           &                                & 50\%    & 100\%   & 200\%   & 300\%   & 50\%    & 100\%  & 200\%  & 300\%  \\ \hline
\multirow{14}{*}{BERT}    & \multirow{6}{*}{Entailed}     & Lexical Overlap & 97.34                     & \bf 98.56                          & \bf 98.66   & \bf 99.38   & 95.82   & \bf 98.48   & \bf 99.3    &\bf 98.36  & \bf 98.54  & \bf 97.74  \\ \cline{3-13} 
                          &                               & Ratio           & 100\%                     & \bf 101\%                          & \bf 101\%   & \bf 102\%   & 98\%    & \bf 101\%   & \bf 102\%   &\bf 101\%  & \bf 101\%  & 100\%  \\ \cline{3-13} 
                          &                               & Subsequence     & 99.34                     & 99.2                           & \bf 99.94   & \bf 99.82   & 99.22   & \bf 99.9    & \bf 99.92   & \bf 99.38  & \bf 99.52  & \bf 99.78  \\ \cline{3-13} 
                          &                               & Ratio           & 100\%                     & 100\%                          & \bf 101\%   & 100\%   & 100\%   & \bf 101\%   & \bf 101\%   & 100\%  & 100\%  & 100\%  \\ \cline{3-13} 
                          &                               & Constituent     & 99.28                     & 98.96                          & \bf 99.46   & 98.78   & 98      & \bf 99.4    & 98.32   & 98.44  & 98.8   & 98.64  \\ \cline{3-13} 
                          &                               & Ratio           & 100\%                     & 100\%                          & 100\%   & 99\%    & 99\%    & 100\%   & 99\%    & 99\%   & 100\%  & 99\%   \\ \cline{2-13} 
                          & \multirow{6}{*}{Non-Entailed} & Lexical Overlap & 28.12                     & \bf 52.26                          & \bf 46.66   & \bf 42.42   & \bf 55.52   & 25.52   & \bf 29.66   &\bf 51.06  & \bf 36.22  & \bf 42.18  \\ \cline{3-13} 
                          &                               & Ratio           & 100\%                     & \bf 186\%                          & \bf 166\%   & \bf 151\%   & \bf 197\%   & 91\%    & \bf 105\%   & \bf 182\%  & \bf 129\%  & \bf 150\%  \\ \cline{3-13} 
                          &                               & Subsequence     & 8.88                      & \bf 10.74                          & 5.68    & 7.00    & \bf 12.00   & 5.92    & 5.86    & \bf 9.42   & 5.46   & 6.46   \\ \cline{3-13} 
                          &                               & Ratio           & 100\%                     & \bf 121\%                          & 64\%    & 79\%    & \bf 135\%   & 67\%    & 66\%    & \bf 106\%  & 61\%   & 73\%   \\ \cline{3-13} 
                          &                               & Constituent     & 6.08                      & 6.00                           & 4.20    & \bf 11.60   & \bf 17.82   & \bf 11.95   & 5.50    & 3.86   & 5.16   & \bf 9.22   \\ \cline{3-13} 
                          &                               & Ratio           & 100\%                     & 99\%                           & 69\%    & \bf 191\%   & \bf 293\%   & \bf 197\%   & 90\%    & 63\%   & 85\%   & \bf 152\%  \\ \cline{2-13} 
                          & \multirow{2}{*}{Overall}      & -               & 56.51                     & \bf 68.04                          & \bf 67.97   & \bf 67.95   & \bf 70.52   & \bf 59.60   & \bf 59.47   & \bf 68.76  & \bf 64.40  & \bf 66.55  \\ \cline{3-13} 
                          &                               & Ratio           & 100\%                     & \bf 120\%                          & \bf 120\%   & \bf 120\%   & \bf 125\%   & \bf 105\%   & \bf 105\%   & \bf 122\%  & \bf 114\%  & \bf 118\%  \\ \hline
\multirow{14}{*}{RoBERTa} & \multirow{6}{*}{Entailed}     & Lexical Overlap & 98.84                     & 98.28                          & 98.62   & 98.22   & \bf 99.16   & 98.44   & 98.44   & 97.9   & \bf 99.08  & 97.7   \\ \cline{3-13} 
                          &                               & Ratio           & 100\%                     & 99\%                           & 100\%   & 99\%    & 100\%   & 100\%   & 100\%   & 99\%   & 100\%  & 99\%   \\ \cline{3-13} 
                          &                               & Subsequence     & 99.96                     & 99.96                          & \bf 100     & \bf 99.98   & 99.88   & 99.64   & 99.74   & 99.82  & \bf 99.98  & 99.86  \\ \cline{3-13} 
                          &                               & Ratio           & 100\%                     & 100\%                          & 100\%   & 100\%   & 100\%   & 100\%   & 100\%   & 100\%  & 100\%  & 100\%  \\ \cline{3-13} 
                          &                               & Constituent     & 99                        & 99                             & 98.9    & \bf 99.5    & \bf 99.4    & 98.16   & 98.86   & \bf 99.36  & \bf 99.24  & 98.96  \\ \cline{3-13} 
                          &                               & Ratio           & 100\%                     & 100\%                          & 100\%   & \bf 101\%   & 100\%   & 99\%    & 100\%   & 100\%  & 100\%  & 100\%  \\ \cline{2-13} 
                          & \multirow{6}{*}{Non-Entailed} & Lexical Overlap & 67.18                     & \bf 72.40                          & \bf 72.52   & \bf 71.60   & 65.40   & \bf 78.10   & \bf 75.38   & \bf 70.52  & 64.12  & \bf 73.34  \\ \cline{3-13} 
                          &                               & Ratio           & 100\%                     & \bf 108\%                          & \bf 108\%   & \bf 107\%   & 97\%    & \bf 116\%   & \bf 112\%   & \bf 105\%  & 95\%   & \bf 109\%  \\ \cline{3-13} 
                          &                               & Subsequence     & 32.92                     & 30.84                          & 28.22   & 21.84   & 28.08   & \bf 33.04   & 31.02   & 25.70  & 27.12  & 30.72  \\ \cline{3-13} 
                          &                               & Ratio           & 100\%                     & 94\%                           & 86\%    & 66\%    & 85\%    & 100\%   & 94\%    & 78\%   & 82\%   & 93\%   \\ \cline{3-13} 
                          &                               & Constituent     & 31.00                     & \bf 35.76                          & \bf 42.38   & 23.18   & \bf 32.68   & 28.60   & \bf 45.04   & \bf 33.70  & \bf 33.80  & \bf 32.7   \\ \cline{3-13} 
                          &                               & Ratio           & 100\%                     & \bf 115\%                          & \bf 137\%   & 75\%    & \bf 105\%   & 92\%    & \bf 145\%   & \bf 109\%  & \bf 109\%  & \bf 105\%  \\ \cline{2-13} 
                          & \multirow{2}{*}{Overall}      & -               & 71.48                     & \bf 72.71                          & \bf 73.44   & 69.05   & 70.77   & \bf 72.66   & \bf 74.75   & 71.17  & 70.56  & \bf 72.21  \\ \cline{3-13} 
                          &                               & Ratio           & 100\%                     & \bf 102\%                          & \bf 103\%   & 97\%    & 99\%    & \bf 102\%   & \bf 105\%   & 100\%  & 99\%   & \bf 101\%  \\ \hline
\end{tabular}
\caption{Train models with MNLI dataset and different noise data then test on HANS. 
Ratio denotes the ratio of results over the case on original training data.}
	\label{tab:HANS}
\end{table}


\noindent\textbf{Stress Test} 

To know
if these methods can weaken spurious features in original dataset 
and make retrained models
more robust to better capacity, we also test the retrained models on Stress Test, 
Which consists of 6 aspects corresponding to MNLI error 
types in error analysis. In~\tabref{tab:mnli-stress}, we show the fine-grained results 
on each aspect. Similar to the performance on HANs, BERT have a larger drop when 
tests on harder adversarial test data. The main weakness are \textit{Antonym}, 
\textit{Negation}, \textit{ Word Overlap} and \textit{Numerical Reasoning}. 
We can find that any noise augmentation method can improve the performance of 
\textit{ Word Overlap} which indicates our methods can avoid this kind of superficial
patterns and may focus on other higher level features. BERT training with 
300\% hypothesis-only noise data raise about 1.5\% on \textit{Negation} and 2.6\% 
on \textit{Numerical Reasoning}. 50\% overlap noise data also take 1\% improvement 
on  \textit{Numerical Reasoning} and 3\% on \textit{ Word Overlap}. Balance noise 
seems not benefit more except for \textit{ Word Overlap}. All noise types can not 
improve the result on \textit{Antonym} of BERT.
Just as we discuss in~\secref{sec:patterns}, we design noise data 
to weaken the influence of superficial patterns for models. So we may can not 
improve the models on every aspect. And we wonder to study on this in our future work.

The result for RoBERTa in~\tabref{tab:mnli-stress} is mostly consistent with BERT but the 
most suitable noise data quantity is different. 200\% hypothesis-only noise and 200\% overlap noise 
take a better performance. The result on \textit{Antonym} is also improved. It may caused by the difference of 
models.

\begin{table}[]
	\centering
	\scriptsize
\begin{tabular}{|c|l|c|c|c|c|c|c|c|c|c|c|}
\hline
\multirow{2}{*}{Model}                          & \multirow{2}{*}{Stress Test} & \multirow{2}{*}{Original} & \multirow{2}{*}{Balance Noise} & \multicolumn{4}{c|}{Hypothesis-only Noise}                        & \multicolumn{4}{c|}{Overlap Noise}                                \\ \cline{5-12} 
                                                &                              &                           &                                & 50\%           & 100\%          & 200\%          & 300\%          & 50\%           & 100\%          & 200\%          & 300\%          \\ \hline
\multirow{11}{*}{BERT}                          & Antonym                      & \textbf{54.77}            & 53.1                  & 52.78 & 53.49 & 52.97& 54.51 & 53.61 & 51.51          & 54.51 & 54.26 \\ \cline{2-12} 
                                                & Antonym(mm)                  & \textbf{50.11}            & 47.23                 & 46.31 & 45.79 & 46.25 & 47.29 & 46.94 & 44.46          & 47.58 & 49.13 \\ \cline{2-12} 
                                                & Negation                     & 55.1                      & \textbf{55.45}                 & \textbf{56.23} & \textbf{55.4}  & \textbf{55.74} & \textbf{57.3}  & \textbf{55.38} & \textbf{55.21} & \textbf{55.65} & \textbf{55.44} \\ \cline{2-12} 
                                                & Negation(mm)                 & 55.64                     & \textbf{55.88}                 & \textbf{56.37} & \textbf{55.89} & 55.59 & \textbf{57.1}  & 55.32 &55.64 & \textbf{55.94} & \textbf{55.87} \\ \cline{2-12} 
                                                & Length Mismatch              & 80.86                     & 80.34                 & \textbf{81.23} & 80.65          & 80.83          & \textbf{81.19} & \textbf{80.89} & 80.82          & 80.86 & \textbf{81.09} \\ \cline{2-12} 
                                                & Length Mismatch(mm)          & 81.86                     & 80.92                          & \textbf{81.89} & 81.61 & 81.7  & \textbf{81.87} & \textbf{81.99} & \textbf{82.04} & \textbf{82.12} & \textbf{82.21} \\ \cline{2-12} 
                                                & Spelling Error               & 78.17                     & \textbf{78.3}                  & \textbf{78.21} & \textbf{78.32} & 77.59          & \textbf{78.5}  & \textbf{78.25} & \textbf{78.5}  & \textbf{78.45} & 77.6           \\ \cline{2-12} 
                                                & Spelling Error(mm)           & 78.43                     & \textbf{78.49}                 & 78.35 & 78.07 & 77.8           & 78.29          & \textbf{78.7}  & \textbf{78.58} & \textbf{78.72} & 77.74          \\ \cline{2-12} 
                                                & Word Overlap                 & 59                        & \textbf{59.93}                 & \textbf{59.9}  & \textbf{59.56} & 57.95 & \textbf{61.79} & \textbf{62.18} & 58.65 & \textbf{60.04} & \textbf{61.72} \\ \cline{2-12} 
                                                & Word Overlap(mm)             & 58.16                     & \textbf{59.6}                  & \textbf{59.63} & \textbf{59.76} & 57.23 & \textbf{61.31} & \textbf{61.52} & \textbf{58.81} & \textbf{59.6}  & \textbf{61.7}  \\ \cline{2-12} 
                                                & Numerical Reasoning          & 32.91                     & 30.12                 & 32.29          & \textbf{35.12} & \textbf{35.07} & \textbf{35.49} & \textbf{33.87} & 32.2  & 32.53 & 29.1           \\ \hline
\multicolumn{1}{|l|}{\multirow{11}{*}{RoBERTa}} & Antonym                      & 65.02                     & \textbf{67.07}                 & \textbf{68.1}  & \textbf{66.82} & \textbf{68.8}  & \textbf{66.23} & \textbf{67.71} & 64.64          & \textbf{69.25} & \textbf{69.7}  \\ \cline{2-12} 
\multicolumn{1}{|l|}{}                          & Antonym(mm)                  & 60.61                     & \textbf{62.8}                  & \textbf{62.86} & \textbf{61.3}  & \textbf{63.73} & \textbf{61.65} & \textbf{62}    & 59.34          & \textbf{64.07} & \textbf{64.19} \\ \cline{2-12} 
\multicolumn{1}{|l|}{}                          & Negation                     & 55.54                     & \textbf{55.61}                 & \textbf{56.72} & \textbf{56.92} & \textbf{56.42} & \textbf{56.61} & \textbf{56.7}  & \textbf{56.59} & \textbf{56.8}  & \textbf{58.05} \\ \cline{2-12} 
\multicolumn{1}{|l|}{}                          & Negation(mm)                 & 55.5                      & \textbf{55.89}                 & \textbf{56.92} & \textbf{56.77} & \textbf{57.05} & \textbf{56.77} & \textbf{57.03} & \textbf{56.88} & \textbf{56.95} & \textbf{58.23} \\ \cline{2-12} 
\multicolumn{1}{|l|}{}                          & Length Mismatch              & 85.31                     & \textbf{85.36}                 & 85.19          & 85.05          & 84.93          & 85.05          & 85.04          & 84.89          & 85.11          & 85.22          \\ \cline{2-12} 
\multicolumn{1}{|l|}{}                          & Length Mismatch(mm)          & 85.3                      & 85.27                          & 85.23          & \textbf{85.67} & \textbf{85.54} & 85.22          & 85.23          & \textbf{85.4}  & \textbf{85.31} & \textbf{85.59} \\ \cline{2-12} 
\multicolumn{1}{|l|}{}                          & Spelling Error               & 83.07                     & 82.8                           & \textbf{83.17} & 83.43          & 82.95          & \textbf{83.14} & 83.06 & 83.06          & \textbf{83.31} & 82.8           \\ \cline{2-12} 
\multicolumn{1}{|l|}{}                          & Spelling Error(mm)           & 83.09                     & 83.06                          & \textbf{83.29} & \textbf{83.62} & 82.91          & 83.07          & 83.09          & \textbf{83.25} & \textbf{83.38} & 82.87          \\ \cline{2-12} 
\multicolumn{1}{|l|}{}                          & Word Overlap                 & 61.96                     & \textbf{62.63}                 & \textbf{64.01} & \textbf{62.41} & \textbf{64.46} & \textbf{63.82} & \textbf{63.03} & \textbf{63.14} & \textbf{64.29} & \textbf{63.92} \\ \cline{2-12} 
\multicolumn{1}{|l|}{}                          & Word Overlap(mm)             & 60.05                     & \textbf{60.71}                 & \textbf{62}    & \textbf{61.18} & \textbf{62.52} & \textbf{61.74} & \textbf{61}    & \textbf{61.82} & \textbf{62.53} & \textbf{61.59} \\ \cline{2-12} 
\multicolumn{1}{|l|}{}                          & Numerical Reasoning          & 56.92                     & 56.51                 & 54.88          & 56.41          & \textbf{57.47} & 52.7           & 54.57          & \textbf{57.06} & \textbf{58.46} & 49.47          \\ \hline
\end{tabular}
	\caption{Train on MNLI and noise augmentation data then test on Stress Test}
	\label{tab:mnli-stress}
\end{table}

\noindent\textbf{Diagnostic NLI Test} 

We perform a zero-shot evaluation of the two models. Adversarial NLI allows to test for transfer
capabilities on three test datasets (R1, R2 and R3).
On each of the benchmarks above, the model trained on the noise augmentation 
data outperforms
the model trained on the original data on challenging examples in the HANs benchmark and 
most aspects in Stress Test, which
targets models purely relying on lexical and syntactic cues. 
%Our methods perform better
Similarly, we retrain models with SNLI and noise data and test 
on the instances in NLI-Diagnostics that require logical reasoning and commonsense knowledge and 
don't on specific aspects. 
In~\tabref{tab:snli-fever-stress}, we can find that BERT only have a 
relatively small improvement on overall test for performing badly on R3. 
RoBERTa training with 300\% hypothesis noise data can get 1.7\% 
improvement and 1.3\% improvement with 200\% overlap noise data. 
This result can not identify our methods are useless, but we may can only 
improve the models on some easy aspect, like overlap. 

\noindent\textbf{Generated FEVER Test} 

We show the result of BERT and RoBERTa models which retrain on FEVER and 
augmentation noise data and test on Generated FEVER Test in~\tabref{tab:snli-fever-stress}. 
We also compare with a re-weight  BERT model, which propose to balance the distribution by
 re-weighting the weight of each instance. However it doesn't perform well on the new test dataset. 
 RoBERTa training with 100\% overlap noise data can get the best result. BERT get 1.7\% and 1.3\% 
 improvement on 200\% hypothesis-only noise data and 200\% overlap noise data respectively.
 
\begin{table}[]
\centering
	\scriptsize
\begin{tabular}{|c|c|c|c|c|c|c|c|c|c|c|c|c|}
\hline
\multirow{2}{*}{Dataset} & \multirow{2}{*}{Model}   & \multirow{2}{*}{Test} & \multirow{2}{*}{Original} & \multirow{2}{*}{Balance Noise} & \multicolumn{4}{c|}{Hypothesis-only Noise}                        & \multicolumn{4}{c|}{Overlap Noise}                                \\ \cline{6-13} 
                         &                          &                       &                           &                                & 50\%           & 100\%          & 200\%          & 300\%          & 50\%           & 100\%          & 200\%          & 300\%          \\ \hline
\multirow{8}{*}{SNLI}    & \multirow{4}{*}{BERT}    & R1                    & 26.3                      & 25.2                           & \textbf{27.6}  & \textbf{27.9}  & \textbf{27.6}  & \textbf{26.6}  & 26             & 26.1           & \textbf{27.3}  & \textbf{26.8}  \\ \cline{3-13} 
                         &                          & R2                    & 29.1                      & \textbf{29.5}                  & 29             & \textbf{29.9}  & \textbf{29.9}  & 28.7           & \textbf{29.3}  & \textbf{29.5}  & \textbf{30.8}  & \textbf{29.9}  \\ \cline{3-13} 
                         &                          & R3                    & \textbf{33.42}            & 31.92                          & 31.75          & 32             & 31.42          & 32.42          & 32.67          & 32.58          & 32.25          & \textbf{33.42} \\ \cline{3-13} 
                         &                          & Overall               & 29.85                     & 29.06                          & 29.59          & \textbf{30.06} & 29.75          & 29.48          & 29.58          & 29.54          & \textbf{30.23} & \textbf{30.18} \\ \cline{2-13} 
                         & \multirow{4}{*}{RoBERTa} & R1                    & 33.2                      & 31.6                           & \textbf{33.8}  & \textbf{33.9}  & 33.2           & \textbf{34.1}  & 32.1           & 32.2           & \textbf{33.9}  & 31.1           \\ \cline{3-13} 
                         &                          & R2                    & 28.4                      & \textbf{32.6}                  & \textbf{30.8}  & \textbf{30.2}  & \textbf{28.9}  & \textbf{32.6}  & \textbf{31.5}  & \textbf{31.6}  & \textbf{31.1}  & \textbf{31.7}  \\ \cline{3-13} 
                         &                          & R3                    & 31.58                     & \textbf{32.25}                 & 30             & \textbf{31.75} & 31.08          & 31.5           & 31.5           & \textbf{31.91} & \textbf{32.17} & 30.41          \\ \cline{3-13} 
                         &                          & Overall               & 31.09                     & \textbf{32.13}                 & \textbf{31.52} & \textbf{32.01} & \textbf{31.14} & \textbf{32.77} & \textbf{31.71} & \textbf{31.91} & \textbf{32.40} & 31.06          \\ \hline
\multirow{3}{*}{FEVER}   & BERT                     & G\_dev                & 65.06                     & 64.75                          & \textbf{65.59} & 64.64          & \textbf{66.84} & 63.28          & \textbf{65.69} & \textbf{65.69} & \textbf{66.32} & \textbf{66.21} \\ \cline{2-13} 
                         & RoBERTa                  & G\_dev                & 69.46                     & \textbf{69.56}                 & 69.14          & 69.04          & 69.45          & 68.41          & 69.14          & \textbf{69.97} & 69.14          & \textbf{69.56} \\ \cline{2-13} 
                         & Reweight BERT            & G\_dev                & 61.6                      & -                              & -              & -              & -              & -              & -              & -              & -              & -              \\ \hline
\end{tabular}
	\caption{Model results on Diagnostic NLI Test and Generated FEVER Test}
	\label{tab:snli-fever-stress}
\end{table}

\subsection{Discussion}

\textbf{Is there a global proper noise 
could be used for all models and datasets?}: 
No. First, for datasets with related superficial patterns,
we generate corresponding noise. If there isn't any superficial patterns, we 
may not have to use the noise data augmentation methods. 
Thus the noise data 
type depend on the spurious issues of datasets.
%problems in a dataset, we may need to use more noise data to weaken the 
%influence of patterns. 
Second, the proportion relies on different models. 
Because different models have their corresponding reasoning ability and to \YZ{and to?} different extent 
that takes advantage of spurious features. If a model pays more attention on spurious features 
and has the ability to learn more complex features, more noise data can be applied. 
If a model pays more attention on spurious features but does not have the ability to learn more 
semantic features, like some simple models, the noise data may make it performance worse. And if 
a model is robust enough, it may don't need any noise data.



%\noindent\textbf{Does noise data augmentation works for all models?} 


