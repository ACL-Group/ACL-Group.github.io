\section{Conclusion}
\label{sec:conclusion}
In this paper, we propose a task of computationally inferring inter-personal
relationships between two speakers in a dialogue. 
We build a new dataset with dialogues from movie scripts for this task. 
%With reference to humam's inference process, 
We identify and extract multiple indicative features of dialogues from texts. 
Logistic Regression with those features achieved a $78.49\%$ accuracy on 
our test set, which significantly outperforms other baselines,
and is close to the $81.69\%$ human performance. 
Analysing indicative power of each feature, we find that lexical features, 
namely address terms and frequencies
of selected words that are unevenly distributed across 2 classes(d-BOW), 
play important role in classification. Syntactic structures and 
linguistic features (e.g. entrainment), however, are not as informative,
either used alone or together with lexical features, despite that their 
distribution in the data also showed disparity.

%Beyond this specific task, we believe our dataset, 
%the identified features and methods can also be applied to 
%research on language phenomenon (e.g., entrainment),  dialogue intentions, 
%emotions and more specific characteristics of relationships 
%revealed by language (e.g., whether the relationship is equal 
%or dominated by one side).  
