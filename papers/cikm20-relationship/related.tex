\section{Related Work}
\label{se:related}

To the best of our knowledge, there is no previous research on the same task,
while related work exists in relationship mining.
Previous work have focused on
identifying connection or measuring familiarity by counting
how many times two people are involved in one action~\cite{relation-act}
or conversation~\cite{relation-conver}.
Chaturvedi et al. have done much work in identifying relationships
from books or narrative summary~\cite{rel-mining-1,rel-mining-2,rel-mining-3}.
They either model relationships simply as a bipolar variable (friend/enemy), or use
unsupervised ways to cluster those with similar patterns. However, they all use
narrative texts such as plot summary in their work, where relationship information
is more explicitly stated. Another related task is role recognition in certain scenes,
such as meetings~\cite{role1} and online health communities~\cite{role2}. Such work
usually deploys either or both of carefully selected linguistic patterns and social network
features(such as different measures of social distance). Since we are studying
dyadic dialogues, network-based features are not applicable, but we refer a
lot to the methodology of linguistic feature engineering in our work.
%Different from all above, in this paper we focus on linguistic cues
%in dialogues that are indicative of relationships.

