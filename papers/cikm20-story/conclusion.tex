\section{Conclusion}
%To understand a story, it is sufficient to represent the story by its key
%commonsense concepts and events, preserving the informative words.
%discarding other unimportant words. 
%It is also important to capture the relations between these concepts and
%events across the sentences so that inference can be made following
%the story line. 

%We consider the investigation of
%explicit representation of commonsense relations in the story to allow
%better interpretability of predication result as future work. 
Predicting story ending is a challenging task in artificial intelligence.
To understand a story, it is sufficient to represent the story by its key
commonsense concepts and events, preserving the informative words.
%discarding other unimportant words. 
It is also important to capture the relations between these concepts and
events across the sentences so that inference can be made following
the story line. 
Our approach well integrated the ideas of main information extraction and 
structured knowledge incorporation and get better performance with automatically 
generated unbiased dataset. From the results we can find that 
predicting story ending is still a challenging task 
in artificial intelligence with little high quality data.
%achieve the 
%current best performance on an unbiased dataset. 
We consider to generate higher quality datasets for training as future work.