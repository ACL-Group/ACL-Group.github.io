\section{Related Work}
\label{sec:related}

Traditional multi-label classification models which transform the problem to a series of binary classification problems like ``one-vs-rest'' SVMs \cite{rifkin2004defense} usually have a dramatic performance drop when the label set has a large size and a skewed frequency distribution. As \cite{liu2005support} stated, the extremely rare categories introduced by the skewed distribution would make the SVM classifier unacceptable. Therefore when dealing with real world data, models which had good performances in traditional benchmark datasets are not satisfying. 

Therefore researchers have turned to generative approaches based on LDA \cite{blei2003latent} for document classification, while several adaptation have been made such that this model could be used in a supervised context \cite{mimno2012topic, blei2007supervised, ramage2009labeled, ramage2009clustering}, where given a multi-labeled corpus the word-label distribution and document-label distribution could be inferred. Ramage et al. proposed Labeled-LDA (L-LDA)\cite{ramage2009labeled} trying to establish a one-to-one correspondence between topics the original LDA learned and the labels which are provided manually. They did this by restricting the document-topic distribution for each document such that only topics specified by the document labels could be counted, while all others are masked. In this way, every word in this particular document could only be assigned topics corresponding to the labels, and the topic-label correspondence becomes obvious. On the other hand MM-LDA\cite{ramage2009clustering} could also be used for multi-label classification as it's not constrained to one label per document, but their learned topics don't correspond with the label set.

Models describing the dependencies among tags are usually another kind of extensions of LDA such as \cite{blei2006correlated, li2006pachinko, blei2003hierarchical, ghamrawi2005collective}. Ghamrawi and McCallum \cite{ghamrawi2005collective} proposed a CRF model which could be used for multi-label classification,  while the following work \cite{druck2007semi} utilized unlabeled data such the model became a semi-supervised learning one. On the other hand \cite{zhang2010multi} used a hybrid generative-discriminative approach where separate classifiers have been trained in a Bayesian network and accumulated in the topological order.