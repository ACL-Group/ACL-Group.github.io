\section{Related Work}
\label{sec:related}

We start by discussing previous work that extracts causal relation term
pairs from open domain text, and briefly mention the general task of relation
extraction. Then we present various past attempts to solve the
commonsense causal reasoning problem. Common ingredients in these
approaches are word association or similarity measures, which are
discussed last.

\subsection{Causal Relation Extraction}
Causal relation recognition and extraction
%are important for many NLP applications, such as
%text understanding, summarization and
can be seen as a pre-process step of commonsense causal reasoning.
%Previous work on causal relation extraction is relatively sparse.
It naturally boils down to a binary classification problem of
causal/non-causal relations. Existing approaches focus on developing
hand-coded patterns and linguistic features and learning the
classifier to recognize and extract causal relation for their
systems. Moreover, previous work are specific with the type of
causal pairs and extracted either noun-noun, verb-verb or verb-noun
causation.

%The existing approaches use hand-coded and domain-specific\ZY{?} patterns
%to extract causal knowledge.
Girju et al.
\cite{girju2003automatic} were the first to work on causal relation discovery
between nominals. They semi-automatically extracted causal cues, but only
extracted noun category features for the head noun. Chang et al.
\cite{ChangC04} developed an unsupervised method and
utilized lexical pairs and cues contained in noun phrases as
features to identify causality between them. Both of them ignored
how the text spans surrounding the causal cue
affects the semantics.  Our causal relation extraction
step, instead,  benefits from these contexts and
constructs a much larger and more powerful causal network.
Blanco et al. \cite{blanco2008causal} used
different patterns to detect the causation in long sentences that
contain clauses.

%We proposed numeric features based on that,
%and get better results.
Do et al. \cite{do2011minimally} introduced a form of association
metric into causal relation extraction. They mainly focus on
detecting causality between verbs and also worked with verb-noun
causality in which nouns are drawn from a small predefined list.
They used discourse connectives and similarity distribution to
identify event causality between predicate, not noun phrases, but
achieved a F1-score around 0.47. And most recently, Riaz et al.
\cite{riaz2014recognizing} focus on noun-verb causality detection
and extraction. None of the previous work except for Do's provides a
causality metric, hence cannot be extended to commonsense causal
reasoning task. Do's work is also of limited help because it only
measures causality strength between verbs. Our CausalNet are
constructed out of all types of words in web corpus, and as a result
the framework on top of it can model the causal strength between
arbitrary text units.

\subsection{General Relation Extraction}
Besides causal relations, much work has been done on extracting many
other types of relations from text, e.g., hyponymy (isA)
\cite{Etzioni:Web,12MSRA:Probase}; meronymy (part-whole)
\cite{GirjuBM06}, metaphor \cite{LiZW13}, relatedness
\cite{Zhang15:Assoc} as well as general relations
\cite{Banko:TextRunner,S:YAGO,S:YAGO2,fader2011identifying,NakasholeWS12}.
Relation extraction generally involves identifying the target terms
or entities in text and then annotating the relations properly.
Previous approaches are either supervised or semi-supervised.

Supervised approaches usually treat the extraction as a
classification problem, where the input is the sentence with marked
target entities/terms, and the output is the classification into one
of the predefined relations or none. Marking the entities often
relies on syntactic patterns or named entity recognition. These
approaches require labeled data and hence cannot be easily extended
to new types of relations. They also make heavy use of NLP tools
such as POS tagger and dependency parser which are all error-prone.
Semi-supervised method often starts with a seed set of entity pairs,
and uses a bootstrapping strategy to accumulate more pairs either by
gradually discovering contextual patterns that represent the target
relation \cite{Etzioni:Web}, or by using a fixed set of strong
patterns and some logical rules to determine the plausibility of a
pair in each iteration \cite{12MSRA:Probase}.

In contrast, our extraction of causal pairs is completely
unsupervised. This allows us to harness a web-scale evidences though
with noises, which we eliminate using statistical evidences.

\subsection{Commonsense Causal Reasoning}

The causal knowledge which encodes the causal implications of
actions and events is crucial for commonsense reasoning problems
listed in \cite{WinNT}. In other words, causal reasoning is a
central problem for commonsense reasoning. Causal knowledge
acquisition is an important first step of commonsense causal
reasoning, which allows us to predict future events, make decisions
and plan actions to achieve goals. However, such knowledge
acquisition is a difficult task since causality is encoded in
various complex contexts. Commonsense causal reasoning is thus a
grand challenge in artificial intelligence. Earlier attempts on the
problem were largely linguistic, for example, developing formal
theories to capture temporal or logical properties in causal
entailment \cite{LascaridesAO92,lascarides:asher:1993a}. These
approaches were not effective due to the difficulty in handcrafting
the theories for board-ranging open domain reasoning.

Recently, the NLP community has explored knowledge based approaches
and shown substantial potential. Such knowledge bases are structural
representations of the general world knowledge. Early approaches for
construction of such knowledge bases were mostly hand-coded
\cite{lenat1995cyc}. Another approach toward this goal is to accrue
commonsense knowledge through crowd-sourcing. A prominent example
along this line is the Open Mind Common Sense (OMCS) project by MIT
\cite{singh2002open}. Some of the knowledge such as ``Causes'' and
``effect of'' relation in the ConceptNet \cite{liu2004commonsense},
which is a sub-project under OMCS, can be used to identify causal
discourse in COPA task. However, such human curated knowledge
suffers from scalability bottleneck. In fact, the ConceptNet is only
a fraction of our causal network by size after 15 years of community
efforts. And recently, several automatic, data-driven methods have
been attempted to acquire commonsense
knowledge\cite{schubert2002can, gordon2010learning,
gordon2010mining, akbikweltmodell}.
%Commonsense knowledge base is a structural representation of the general world
% knowledge.
These approaches focused on the acquisition of general
worldly knowledge expressed as factoids, and not causality knowledge
per se. Hence their coverage of causality knowledge is very limited.

More successful efforts are centered around using correlational
statistics \cite{gordon2012copa} such as pointwise mutual
information (PMI) between unigrams (words) or bigrams from large
text corpora \cite{Mihalcea2006:CKM}. Corpora attempted include LDC
gigaword news corpus \cite{goodwin2012utdhlt}, Gutenberg e-books
\cite{roemmele2011choice}, personal stories from Weblogs
\cite{gordon2011commonsense} and Wikipedia text
\cite{jabeen2014exploiting}.
%Previous research shows that the type of information source has a significant impact on the accuracy of such knowledge based approach.
This paper falls into this category
of research, but instead proposed to compute a generalized PMI
measure \cite{Washtell09:CWW} not from the plain text corpus but
from a causal relation graph induced from large web text. In
addition, instead of fixing the target language units in the
discourse sentences to either word or n-gram, we dynamically
construct events which contain internal causality information and
make use of these multi-word events in the computation of the final
causal strength between two sentences.

Commonsense causal reasoning requires considerable general causal
knowledge which involves the proper modeling of causal dependency amongst text.
We investigate the potential and effectiveness of large-scale data driven
approaches for commonsense causal knowledge base construction.
CausalNet can be seen as a large graph-based representation of common sense
causal knowledge and can provide relatively reasonable causal strength
between terms.

\subsection{Word Association Metrics}
Our generalized causality is inspired from association strength
measure proposed by Wettler et al.~\cite{Wettler:1993} and Washtell~\cite{Washtell09:CWW},
which introduced parameter $\alpha$ being $0.66$ and $0.5$ respectively.
Our causality strength considers both directions of the causality
relation. Causality strength is similar to
association strength to some degree, since association between term pair $(u,v)$ which also asymmetrical treated $u$ and $v$. We introduced an generalized
formula by introducing $\alpha$ and $\beta$ following the same intuition
for discounting high frequency causes and effects respectively.
%(Mention SCI?)
%parameter $\alpha$ PMI is lexical order based.

%SCI has a param of 0.5.
