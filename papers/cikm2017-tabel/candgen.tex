\subsection{Candidate Entity Generation}
\label{sec:candgen}
To generate a candidate entity table, we have to generate candidate English 
entities for each non-English mention at first. Without a reliable 
non-English knowledge base as the bridge, we use translation tools to 
produce a set of possible translations for a given non-English mention. 
Afterwards, we use several heuristic rules to obtain candidate English 
entities. The set of candidate entities consist of:  1) exact match of 
any mention translation; 2) anchor entities of any mention translation in 
knowledge base; 3) fuzzy match (e.g. edit distance) of any mention translation. 
Take the Chinese mention ``疑犯追踪'' as an example. It will be translated 
into ``person of interest'' or  ``suspect tracking'', depending on 
what translation tool to use. The corresponding candidate entity set 
would contain entities like ``person of interest'', 
``person of interest (tv series)'' and ``suspect (1987 film)''.
