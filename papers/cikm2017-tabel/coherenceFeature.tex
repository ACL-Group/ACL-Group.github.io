\subsection{Coherence Feature}
\label{sec:coherence}

The previous two features try to encode the similarity or compatibility of mention table and entity table. Besides, if we look at a correctly linked table, the inner structure of that table is also valuable.
The intuition is that entities in the same column (row) tend to own the same type, or we can say that their embeddings are very close. Therefore, we propose a third feature, which captures this correlations among entities of same column. 

We calculate the dimension-wise variance for all entity embeddings in the same column to get a coherence embedding for that column. Then we average them among all columns to get the coherence feature for an entity table. This feature represents how closely connected all the entities with the same column, which would help entity disambiguate.
For an entity table $T_E$ of size $R\times C$, we get calculate coherence feature as follows.

\begin{equation}
\label{eqn:coherence}
f_{coh}(T_E) = \frac{1}{C} \sum_{j} var(\{v_{\textbf{e}_{ij}}, i\in[0, R], \langle i,j \rangle \in P\})
\end{equation}

Where $var$ is a function to calculate dimension-wise variance for a bunch of vectors.