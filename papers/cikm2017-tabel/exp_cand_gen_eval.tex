\subsection{Evaluation of Candidate Generation}
\label{sec:exp-cand-gen-eval}

%In this part, we evaluate the quality of candidate entity generation.

In this part, we investigate the translated English mentions from Chinese table inputs.
As described in \secref{sec:impl-detail}, English mentions is derived from multiple resources.
Compare with different combination of resources, we evaluate the quality by
measuring the proportion of cells that the correct entity appears in the top-$n$ candidates
(Hits@$n$).

From the results in \tabref{tab:cand-gen-quality}, we observe that
ensembling multiple translation resources is able to discover more correct entities 
without bringing too many noisy candidates.
Besides, the English mentions generated by Pinyin is a complementary to
those generated from pure translation methods.

\begin{table}[ht]
    \centering
    \caption{Hits@$n$ results on candidate entity generation}
    \label{tab:cand-gen-quality}
    \begin{tabular} {c|ccc}
        Resources   &   n=1     &   n=5     &   n=10    \\
        \hline
        Google      &   0.463   &   0.585   &   0.596   \\
        Baidu       &   0.542   &   0.669   &   0.684   \\
        Tencent     &   0.394   &   0.510   &   0.522   \\
        All Trans.  &   \textbf{0.558}   &   0.705   &   0.723   \\
        \hline
        Pinyin      &   0.046	&   0.052   &   0.053   \\
        \hline
        Trans. + Pinyin & \textbf{0.558}  & \textbf{0.708}  &  \textbf{0.726}   \\
    \end{tabular}
\end{table}


