\section{Data Collection}
\label{sec:data}

To the best of our knowledge, available large scale 3D Bin Packing order dataset is very rare, especially with real-sized items. For that purpose, we create a dataset on 3D bin packing problem of real-world E-commerce platform. This L3DBPD consists of two parts: the customer order data is collected from Taobao\footnote{\url{https://www.taobao.com/}} E-commerce platform and item size data (with length, width and height) is collected from Cainiao\footnote{\url{https://www.cainiao.com/}} Logistics platform. We randomly sampled 150,000 training data and 150,000 testing data for customer orders with 8, 10 and 12 items, which are named as BIN8, BIN10 and BIN12 respectively. We use one line to demonstrate one customer order.

\begin{table*}[!h]
	\small
	\centering
	\caption{Examples of L3DBPD. }
	\label{dataset table}
	\begin{tabular}{|c| c | c | c | c | c | c | c | c | c |}
		\hline
		Order ID & item1  & item2 & item3 & item4  & item5 & item6 & item7  & item8  &  baseline \\  \hline
		1   & (140,50,180) & (100,70,60)  & (170,150,40) & (130,70,40) & (190,150,20)  & (190,150,20) & (240,200,160) &
		(160,170,50)  & 432600  \\ \hline
	\end{tabular}
\end{table*}


Take Table \ref{dataset table} as an example of BIN8, the first and last columns indicate the order ID of a customer order and the heuristic baseline of this BPP, the contents of the remaining 8 columns respectively shows the length, width and height of each items that belong to this order.
We believe this dataset will contribute to the future research of 3D bin packing problem. \footnote{The data will be published soon after accepted.}
