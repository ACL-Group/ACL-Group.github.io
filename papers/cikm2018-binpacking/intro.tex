\section{Introduction}
\label{sec:intro}

%
Bin packing problem (BPP) is a classical and important optimization problem in logistic and production systems. There are many variants of BPP and the most meaningful and challenging one is 3D BPP, in which a number of cuboid-shaped items with different sizes should be packed into bins orthogonally. For vanilla 3D BPP, the size and cost of bins are fixed and the objective is to minimize the number of bins used, i.e., to minimize the total cost. BPP is a typical and interesting combinatorial optimization problem and is strongly NP-hard \cite{coffman1980performance}, so it is a very popular research direction in optimization area. In addition, BPPs have numerous relevant industrial applications. An effective bin packing algorithm means the reduction of computation time, total packing cost and increase in utilization of resources.

The cost of packing materials is mainly determined by their surface area and this occupies the most part of packing cost. In many real business scenarios such as e-commerce, bins with variable sizes, e.g., PE bag which is made of flexible and soft packing materials, are used to pack items. In this case, our research is engaged in a new type of 3D BPP, of which the objective is to pack all items into a bin with least surface area. 

Due to the difficulty of obtaining optimal solutions of BPPs, many researchers have proposed various approximation or heuristic algorithms. To achieve good results and guarantee performance, heuristic algorithms have to be designed specifically for different types of problems or situations, therefore still rely on human created heuristic rules. In recent years, artificial intelligence, especially deep reinforcement learning (DRL), has received intense research and achieved amazing results in many fields \cite{mnih2013playing, silver2016mastering}. In addition, DRL method has shown huge potential to solve combinatorial optimization problems \cite{vinyals2015pointer,bello2016neural}. Previous
DRL-based method that attacks this new type of 3D BPP \cite{Hu2017Solving}
relies heavily on the output of the heuristic algorithm because the network just 
produces the sequence of packing items. 

As the surface area is determined by three kind of factors: the sequence, spatial locations and orientations, each kind of decision can be treated as a task. Regardless of spatial locations, the sequence of putting the items into the bin will influence the orientations of each item and vice versa, so the two tasks are correlated. Meanwhile, with $n$ items, the choice of orientations is $6^n$ and the choice of sequence is $n!$, so the two tasks are unbalanced.
%\lu{why tightly coupled} 
Inspired by multi-task learning, we overcome the limitations of simply outputting the placement sequence of items by adopting a new type of training mode named 
Selected Learning (SL) to solve the new type of 3D BPP. %\kz{consider rephrase the previous sentence. Now sure what u mean by each task is tightly coupled and unbalanced.} 
In Selected Learning, we prefer to select one kind of the loss function at each batch than to train the loss of all tasks at once. 

%\kz{rephrase the following sentence:}
Combinational optimization problems are to minimize or maximize the objective under some constraints. The objective can be set as the reward of RL framework. In this new type of 3D BPP, the smaller the objective, the better. Through keeping track of the best solution sampled during the search, we find that combining Selected Learning with sampling at inference time works best in practice.

In this paper, we present a neural model that achieves state-of-the-art results on the proposed Large-scale 3D Bin Packing Problem Dataset (L3DBPD) and 
quantitative experiments are designed and conducted to demonstrate effectiveness of this method. The contributions of this paper are summarized below.
\begin{itemize}
	\item This is the first attempt to define and solve the real-world problem of New Type 3D Bin Packing (Section \ref{sec:problem}).
		We collect and will open source a large scale 3D Bin Packing order dataset (Section \ref{sec:data}).
	\item We use an intra-attention mechanism to solve the combinatorial optimization problem, which considers items that have already been generated by the decoder.  
	\item We propose a multi-task framework based on Selected Learning, generating packing sequence and orientations at the same time.
	\item By sampling at inference time, we achieves $6.21\%$, $7.80\%$, $8.55\%$ improvement than the new 3D bin packing heuristic algorithm (NBPH) (see Appendix \ref{NBPH}) for BIN8, BIN10 and BIN12.
\end{itemize}

%The rest of the paper is organized as follows:
%\secref{sec:problem} provides the formal definition of new type 3D bin packing problem in our study.
%\secref{sec:data} introduces how we collect our L3DBPD dataset.
%\secref{sec:model} describes the proposed multi-task selected learning architecture.
%Then \secref{sec:implementation} gives the implementation details of our model for reproducibility.
%\secref{sec:eval} presents the experimental results and analysis on L3DBPD data.
%We conduct the detailed comparison with baseline models %\lu{baseline models contains rl ?} 
%including a well-designed algorithm NBPH and a state of art method BRKGA in \cite{gonccalves2013biased} which tackles the fixed-sized 3D BPP.
%\secref{sec:related} provides a literature survey of most related previous work.
%Finally \secref{sec:conclusion} concludes and points out some future research directions.
