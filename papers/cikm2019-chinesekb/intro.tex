\section{Introduction}
\label{sec:intro}
%\KZ{Shrink the first three paras.}
Over the past decade, great efforts have been made to construct knowledge graphs manually or automatically to facilitate artificial intelligence.
However, most of these efforts were spent on the English language. For example, \con~\cite{Speer2012a} is a large multilingual common sense knowledge graph based on crowdsourcing. In \con, there are 2,975,240 edges containing both English concepts, while only 438,307 edges contain both Chinese concepts, which accounts for only around 1/7 of the former.\footnote{These statistics come from the latest \con 5.6.} Besides, \pro \cite{Wu2012} is a universal probabilistic taxonomic knowledge graph consisting of 13,949,064 ``IsA'' pairs, which are extracted automatically from 1.68 billion English web pages. 
These two datasets are crucial to artificial intelligence research because they can provide machines with the ability of common sense reasoning and conceptualization, which are fundamental to many downstream tasks. For example, common sense knowledge has been used in tasks, such as textual entailment \cite{dagan2010recognizing,bowman2015large} and visual recognition tasks \cite{zhu2014reasoning}, etc., and taxonomic knowledge has also been leveraged in applications, such as taxonomy keyword search \cite{ding2012optimizing}, semantic web search \cite{wang2010toward}, short text understanding \cite{song2011short}, and web table understanding \cite{wang2012understanding}, etc. 

Although there exist some Chinese taxonomic knowledge graphs built from online-encyclopedia, such as CN-Probase \cite{Xu2017} and zhishi.me \cite{Niu2011}, they have two serious problems in use. First, since their source is not as universal as \pro, they are informative on specific topics, while lacking coverage of diverse topics, which is fundamental in some explicit topic model application, such as short text understanding \cite{song2011short}. 
Second, they have no probabilistic characteristic, which is crucial in some applications \cite{Cui2016,Song-ijcai-2011}. 
Meanwhile, as far as we know, there is no dedicated Chinese common sense knowledge graph.
%Therefore, alleviating this resource imbalance problem is valuable and urgent for Chinese NLP research. 
Moreover, it is impractical to construct Chinese common sense knowledge graph from scratch in the same way as it is built in English, because it will consume a large number of human efforts and time, which are unaffordable for most researchers. Similarly, building a Chinese universal probabilistic taxonomic knowledge graph such as \pro will need billions of high-quality Chinese web pages, Chinese information extraction technology with high precision, and high-quality auxiliary Chinese language resources, which are unavailable currently \cite{Wang2015}. 

%\KZ{The problem you are solving is translating ConceptNet and Probase into
%Chinese. What are the key challenges? Why existing methods cannot solve them
%effectively? Why your solution can solve them better. 
%You need to focus on these questions and expand this para.}
To tackle these challenges, we propose a simple but effective method to quickly obtain the corresponding Chinese knowledge graphs of (\pro, \con) with high quality, consuming only a little time and a few resources.
Generally, our approach includes two steps. First, we translate the less ambiguous part of English datasets into Chinese with the existing machine translator.  
However, existing translators are powerless when facing some extremely intractable word sense disambiguations. Typically, the distribution of these word senses is more skewed, with a few more common word senses and a long tail of rarer word senses. For example, (``date'', IsA, ``fruit'') and (``ball'', AtLocation, ``ballroom'') will be incorrectly translated into (``日期'', IsA, ``水果'') and (``球'', AtLocation, ``舞厅''), respectively, even if we send the deep contextualized sentences into the translator, such as ``date is fruit.'' and ``the ball is at the ballroom.''. 
%Specifically, this problem is mostly caused when the contextualized word uses the tail word sense. 
Hence, second, to further handle this problem, we introduce the following revision step to tackle the rest part of English dataset, which is more ambiguous.
The intuition is that linguistic items with similar distributions have similar meanings \cite{harris1954distri}.
For example, it is easy to recognize the rare sense ``枣/date'' rather than the common sense ``日期/date'' as the translation of ``date'' in the triple (``date'', IsA, ``fruit''), since ``枣/date'' shares more similar Chinese semantic distribution, calculated by word embedding \cite{Mikolov_nips_2013}, than ``日期/date'' together with ``水果/date'', see detail in Section \ref{sec:approach}. 
Last, we merge the translation results from both steps as the final result and evaluate its coverage and accuracy in Section \ref{sec:evaluation}.

Our contributions in this paper are: First, we propose a simple but effective method based on the translation to construct the knowledge graph quickly and effectively. Second, the experiments show the effect of translation is satisfactory. Third, based on this method, we release two valuable Chinese datasets: \zhcon and \zhpro. To the best of our knowledge, the former is the first dedicated Chinese common sense knowledge graph and the latter can substantially enrich existing Chinese taxonomic knowledge graphs. 
