\section{Conclusion}
\label{sec:conclusion}

In this paper,
we point out that one of the biggest challenges in current e-commerce solutions
is that they are not directly driven by user needs, 
which, however, are precisely the ultimate goal of e-commerce platform try to satisfy.
To tackle it, we introduce a specially designed e-commerce knowledge graph practiced in Taobao, trying to conceptualize user needs as various shopping scenarios, also known as e-commerce concepts. 
We further proposed a deep attentive inference model to intuitively infer those concepts accurately.
On our real-world e-commerce dataset, the proposed model achieved state-of-the-art performance against several strong baselines.
After applying to online recommender system, great gain regarding both accuracy and novelty are achieved. A real user survey is conducted to demonstrate such user needs inference actually improves user satisfaction.
More importantly, we believe that the idea of conceptualizing and inferring user needs can be applied to more e-commerce applications. In the future, we will continuously explore various possibilities of ``user-needs driven'' e-commerce.

\section{Acknowledgement}
\label{sec:ack}
We deeply thank Peng Wang, Peng Yu and Guli Lin for supporting the online experiments in this paper.