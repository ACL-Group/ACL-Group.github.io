\section{Related Work}
\label{sec:related}

%We introduce related work from three perspectives:
%1) KGs in open-domain and e-commerce. 2) knowledge-aware recommendation. 3) attention mechanism.

%\KZ{Is there any work on conceptualization on KG?}

%\subsection{Knowledge Graphs}
%\XS{Need to rewrite}
%Introducing rich information from external knowledge graphs (KGs) to various downstream applications such as text understanding, question answering has made achievements in recent years.
User needs in e-commerce, are not formally defined previously.
Hierarchical categories and browse nodes \footnote{\url{https://www.browsenodes.com/}} are ways of managing billions of items in e-commerce platforms
and are usually used to represent user needs or interests \cite{zhou2018deep, zhou2018deep2, feng2019deep}.
However, we argue that user needs are far broader than categories or browse nodes. Imaging a user who is planning an outdoor barbecue, or who is concerned with how to get rid of a raccoon in his garden.
They have a situation or problem but do not know what products can help.
Therefore, tree-like structures such as category tree and browse nodes are not enough to represent those user needs.
E-commerce concept net, on the other hand, is a graph structure where each node can be expressed by terms from domains including ``Incident'' and ``Function'' and is able to represent various shopping needs.
%Great human efforts have been devoted to the construction of open domain KGs such as Freebase \cite{bollacker2008freebase} and DBpedia \cite{auer2007dbpedia}, 
%which typically describe specific facts rather than inconsistent concepts appearing in natural language text. 
%ConceptNet \cite{speer2012representing} tries to include common sense knowledge by recognizing informal relations between concepts, where the concepts could be the conceptualization of any human knowledge such as ``games with a purpose'' appearing in free texts.
%Our e-commerce concept net is entirely focusing on conceptualizing user needs in e-commerce.
Inspired by the construction of open-domain KGs such as Freebase \cite{bollacker2008freebase} and DBpedia \cite{auer2007dbpedia}, different kinds of KGs in e-commerce are constructed
%, in order to tackle different challenges in e-commerce scenario.
%Generally, KGs in e-commerce 
to describe relations among users, items and item attributes \cite{catherine2017explainable,ai2018learning}.
One example is the ``Product Knowledge Graph'' \footnote{\url{https://blog.aboutamazon.com/innovation/making-search-easier}} of Amazon. Their KG mainly supports semantic search, aiming to help users search for products that fit their need with search queries like ``items for picnic''. The major difference is that they never conceptualize user needs as explicit nodes in KG as we do.
In comparison, our e-commerce concept net introduces a new node to explicitly represent user needs. Besides, it becomes possible to link our e-commerce KG to open-domain KGs through the concept vocabulary mentioned in \secref{sec:ecn}, making our concept net even more powerful.

%\subsection{Knowledge-aware Recommendation}
Prior efforts on integrating KG into recommendation can be categorized into path-based and embedding-based.
Path-based methods are closely related to heterogeneous information network (HIN) perspective. They treat KG as HINs and extract meta-path based features to enhance recommendations \cite{zhao2017meta,gao2018recommendation}. 
Meta-path can be further extended to Meta-graph \cite{zhao2017meta} to model more complicated structure.
KPRN \cite{wang2018explainable} claims that relation name is important and should be encoded into path representation, while MCRec \cite{hu2018leveraging} only consider entities.
Embedding-based methods, mainly adopt the idea of knowledge graph embedding (KGE) \cite{wang2014knowledge,lin2015learning}, to bring extra information from pre-trained embeddings of entities and relations. However, this direction of work has not been proven scalable or effective in large dataset. 
RippleNet \cite{wang2018ripplenet} combines the ideas of above two types of methods, and proposes to simulate propagation of user preference over the KG.
We do not select RippleNet as our comparison,
due to the reason that RippleNet explore paths freely and relies heavily on KGE, leading to an unacceptable computational overhead in industry dataset, which is extremely large.
%Therefore, we mainly follow path based methods, and further propose CptInfer which can handle informative input in industry scenario.


%\subsection{Attention Mechanism}
%Attention mechanism is first proposed in machine translation \cite{bahdanau2014neural},
%under the intuition that different parts of input should be treated differently facing different target.
%and the idea is soon transplanted to various areas including recommendation \cite{wang2017dynamic}.
%In most cases, attention mechanism only focuses on a list of documents and one single query, including the co-attention mechanism proposed in MCRec \cite{hu2018leveraging}, which is a two-step process and unable to handle mutual influences among multiple lists. ABCNN \cite{yin2016abcnn} first proposes an attention matrix to model mutual influence between two sentences (lists of words) simultaneously. 
%We further propose an attention cube to tackle the problem with even more informative inputs in our scenario.


