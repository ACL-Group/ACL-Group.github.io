\section{Conclusion}
In this paper, we propose a few-shot learning framework for relation classification that aims to (1) extract intra-support knowledge by classifying both support and query instances, and (2) bring external implicit knowledge from cross-domain corpus by task enrichment.
%extract information within support instances and using data augmentation.
Our framework is particularly powerful when only small amount of training data is available. 
%MME aims to extract underlying information brought within support sentences. In MME, a supplementary classifier is adopted to classify the relation of each support sentence, and participates in the back propagation progress by a fast-slow learner strategy. We also propose a way of aggregating cross-domain data into the training process. This helps the model to learn underlying cross-domain knowledge. These two innovations are especially useful under circumstances where extremely few data is available.
Additionally, we construct our own dataset, the TinyRel-CM dataset, a Chinese few-shot relation classification dataset in medical domain. The small training data size and highly similar 
relation classes make the TinyRel-CM dataset a challenging task.
As for future work, we intend to futher investigate whether the proposed framework is able to handle zero-shot learning tasks.
%Different from previous released few-shot relation classification datasets, the TinyRel-CM dataset contains extremely limited labeled data and provides a much harder few-shot learning task. Our proposed MME framework and data augmentation method achieves state-of-the-art accuracy on the TinyRel-CM dataset and competitive results on the FewRel dataset.
%As for future work, we aim to extract abundant relation classes and instances from massive open-source data with our proposed framework to construct a knowledge graph in medical domain .

\section*{Acknowledgement}
This work was partially supported by NSFC grant 91646205, SJTU Medicine-Engineering Cross-disciplinary Research Scheme and SJTU-Leyan Joint Research Scheme.

