\section{Problem Formulation}
%\KZ{Don't use the term constraint so much: constraint is usually a more complex mathematical
%equation or inequality. Here it's just simply a limitation.} 
We add limitation on the size of training data compared with previous few-shot relation classification tasks.
Just like in conventional few-shot relation classification, there is a training set
$D_{\rm{train}}$ and a test set $D_{\rm{test}}$.
Each instance in both sets can be represented as a triple $(s,e,r)$,
where $s$ is a sentence of length $T$, $e=(e_1, e_2)$ is the head and tail entities and
$r$ is the semantic relation between $e_1$ and $e_2$ conveyed by $s$.
$r \in R$, where $R=\{r_1,...,r_N\}$ is the set of all candidate relation classes.
$D_{\rm{train}}$ and $D_{\rm{test}}$ have disjoint relation sets, i.e.,
if a relation $r$ appears in a triple of the training set, it must not appear
in any triples of the test set and vice versa.
$D_{\rm{test}}$ is further split into a support set $D_{\rm{test\text{-}s}}$
and a query set $D_{\rm{test\text{-}q}}$. The problem is to predict
the classes of instances in $D_{\rm{test\text{-}q}}$ given
$D_{\rm{test\text{-}s}}$ and $D_{\rm{train}}$. While no restrictions are lied on how to use $D_{\rm{train}}$, it is conventionally splited into a support set and a query set to train models.
In a $N$-way $K$-shot %few-shot relation classification 
scenario,
$D_{\rm{test\text{-}s}}$ contains $N$ relation classes and $K$ instances
for each class. Both $N$ and $K$ are supposed to be small (e.g., 5-way 1-shot, 10-way 5-shot).
Particularly, we limit the size of training data (i.e.,  $D_{\rm{train}}$ is also small).
The difficulty of 
%few-shot relation classification 
the task lies in not only the small size of
$D_{\rm{test\text{-}s}}$ (totally $N\times K$ instances) but also the small training data size.

%Basing on few-shot relation classification task, we put forward the challenge of few-shot relation classification
%\emph{under limited amount of training data} (i.e., the size of $D_{\rm{train}}$ is also small).
%The small training data size makes the task much harder. %The task becomes harder if the size of $D_{\rm{train}}$ is also small, and
