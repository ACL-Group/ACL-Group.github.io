\section{Problem Statement}
\label{sec:problem}
The problem of automatic slogan generation is formulated
as follows: Given a topic represented as a sequence of characters $x = (x_1, x_2, ..., x_n) \in \mathcal{X} $ and an item preference $p$ which consists of a set of preference categorized items associated to $x$ (see  \defref{item_preference}), the goal is to generate a textual slogan $y = (y_1, y_2, ..., y_m) \in \mathcal{Y}$, which properly describes the topic $x$ under the item preference $p$.

%categorized items from each affiliated preference category as \emph{item preferences} 


\theoremstyle{definition}
\begin{definition}
	\label{item_preference}
	\textbf{Item Preference}
	Each item preference $p$ associated with the topic $x$ consists of a set of categorized items 
	$I=\{I_1, I_2, ..., I_u\}$ which are randomly sampled from the preference product category associated with $x$.
%	Different item preferences of a specific topic are expected to imply different focuses or selling-points according to their item members. 
	Formally, $p$ is the concatenation of the character sequences of the titles of $I$'s items.
\end{definition}

The systems are expected to use item preferences for the selling point exploration and generate attractive slogans accordingly.
Next, we attack the problem of slogan generation. 
%We present the workflow of our method in \figref{fig:flow}.
We propose a semantics-enhanced model (SSG) which enriches the deep semantic representations with external knowledge and facilitate the slogan generation in e-commerce.


