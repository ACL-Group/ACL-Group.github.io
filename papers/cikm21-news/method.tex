\section{Approach}
\label{sec:approach}
We first give some preliminary definitions, then present our base model, before
presenting our framework to incorporate implicit positive and negative feedback.

\label{sec:method}
\subsection{Preliminaries}
In the typical session-based setting, given the prefix sequence of the session, denoted as $S_u=(n_1, n_2,...,n_t)$, 
where $n_i$ is the article id, our goal is 
to predict $n_{t+1}$ article that the target user $u$ is most likely to click next. 
Following \citet{liu2018stamp}, an $N\times d_n$ 
item embedding matrix, where $d_n$ is the embedding dimensionality and $N$ is the number of
all candidate items, provides article $n_i$'s embedding vector 
as $\mathbf{x}_i$. Then methods like RNN~\cite{hidasi2018recurrent}, GNN~\cite{wu2019session, pan2020star}, 
or attention-based approaches~\cite{kang_self-attentive_2018, liu2018stamp} can be used to 
encode the session vector $\mathbf{s}_u$ from the sequence 
${(\mathbf{x}_1, \mathbf{x}_2, ..., \mathbf{x}_t)}$, which represents the user's history preferences.
Meanwhile, the same item embedding matrix can be also regarded as $N$ encoded candidates $[\mathbf{x}_1, \mathbf{x}_2,.., \mathbf{x}_N]$.
For $u$, the cosine similarity score $\hat{y_i}^u$ between the session representation and the article $i$ is calculated by the inner product of the session vector and the candidate news vector:
\begin{equation}
    \hat{y_i}^u = \mathbf{x}_i^T\mathbf{s}_u, 
\end{equation}
$\hat{y_i}^u$ is normalized by softmax to be the probability of article $i$ being clicked next in the session. 
The cross-entropy is usually used to compute loss:
\begin{equation}
    \mathcal{L}_1 = - \frac{1}{|S|}\sum_{S_u \in S}\sum_{i=1}^K ( y_i^u \log(\hat{y_i}^u) + (1-y_i^u)\log(1-\hat{y_i}^u)),
\end{equation}
where $S$ is the whole training sessions, $y_i^u=1$ if $i$ is indeed the next-clicked articles in $S_u$ and $y_i^u=0$ otherwise.

\subsection{Content-aware Recommendation (CAR)}
In order to recommend new and emerging articles,
our starting point is a basic content-aware recommendation model. 
We use pre-trained word embeddings like Word2Vec to encode the content, specifically, we average the $d_c$-dimensional word vectors in news title to represent the 
topic-oriented content of articles. Once we get the content vector $\mathbf{x_c}_i$ of 
article $n_i$, we concatenate $\mathbf{x_c}_i$ and $\mathbf{x_n}_i$ to represent 
the article as $\mathbf{x_{nc}}_i$. 
To model the varying user preference to the articles in the same session, we adopt a simple attention network, using the weighted sum of the input vector sequence to encode the whole session.

We define $\alpha_i$, the attention weight of $i$-th articles $n_i$ in session $S_u$ as:
\begin{equation}
    \alpha_i = W_0 \times \sigma (W_1 \times \mathbf{x_{nc}}_i +  b_1),
\end{equation}
where $W_0\in \mathbb{R}^{1 \times d_n}, W_1 \in \mathbb{R}^{d_n\times (d_n+d_c)}$ 
are weighting parameters, $\mathbf{x_{nc}}_i$ is the article vector, 
and $b_1\in \mathbb{R}^{d_n}$ is a bias. $\alpha_i$ will be normalized by softmax.

Finally, $\mathbf{s}_u$ of session $S_u$ is defined as the weighted sum:
\begin{equation}
    \label{eq:final_repre}
    \mathbf{x_s} = \sum_{i=1}^{|S_u|} \alpha_i \mathbf{x_{nc}}_i.
\end{equation}

\subsection{Utilize Positive Feedback}
\label{sec:positive feedback}
Our implicit positive feedback takes the form of the \textit{active time} interval that 
a user spent on each article after clicking on it. If the user spends a short time 
in an article, it's probably because the user is fooled by the title but actually doesn't like 
the article~\cite{lu_quality_2019}. If the active time is not explicitly available, 
it can be estimated by the time interval between the user's two consecutive clicks. 
For the different and continuous active time, we bucketize them into $k$ distinct categories, 
and each category shares the same embedding vector $\mathbf{ta}_i$, representing a level of positive feedback. We feed this vector into the attention computation as extra click-level 
feedback. Now, $\alpha_i$ is modified to:
\begin{equation}
    \alpha_i = W_0 \times \sigma (W_1 \times \mathbf{x_{nc}}_i + W_2 \times \mathbf{ta}_i + b_1),
\end{equation}
where $W_2 \in \mathbb{R}^{d_n \times d_t}$ is projection parameters that map active time embeddings with $d_t$ dimension into another dimension space. The final session vector $\mathbf{x_s}$ still follows
\eqnref{eq:final_repre}.

\subsection{Joint Learning with Negative Feedback}
\label{sec:negative feedback}
The most straight-forward and widely adopted negative sampling strategy is the random sampling 
from a non-clicked set of items, or from a global buffer with the last $N$ 
clicks~\cite{moreira_news_2018}. 
The major problem of randomly sampled items is that these items might be completely unrelated to 
the user, posing too little challenge for the model to learn. 

While reading news, a user scrolls up and down the news stream, 
and the articles that are exposed to the user collectively 
form an impression list $Imp_u$. We take unclicked articles in $Imp_u$ as more informative negative signals than other candidates.
As we discussed before, since the impression list is not always explicitly available, 
we assume an article is more likely to be in $Imp_u$ if it was published nearby an article that has
been clicked by $u$.
We aim to minimize the cosine score between $\mathbf{s}_u$ and the negative samples 
$\mathbf{v}_i\in Ne_u$, where $Ne_u\subseteq Imp_u$ is the set of negative samples for session $S_u$, 
and the final loss is: 

\begin{equation}
    \label{eq:loss}
    \begin{split}
        \mathcal{L}_2 = - \frac{1}{|S|}\sum_{S_u \in S}\sum_{i=1}^K &( y_i^u \log(\hat{y_i}^u) + (1-y_i^u)\log(1-\hat{y_i}^u) \\
        & +  \lambda \mathbbm{1}(\mathbf{v}_i \in Ne_u) \log(\sigma(1-\mathbf{v}_i^T\mathbf{s}_u))),
    \end{split}
\end{equation}
where $\mathbbm{1}(\cdot)$ returns 1 if the expression is true, $\lambda$ is 
the weighting parameter of loss from negative articles. We jointly optimize these two losses 
with Adam optimizer. 
