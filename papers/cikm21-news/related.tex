\section{Related Work}
In this section, we discuss works in the area of news recommendation, session-based
recommendation which targets not only news but other sequential applications, and
finally recommendation systems that exploit temporal information.

\subsection{News Recommendation}
First, the news recommendation task can be formulated as conventional recommendation tasks, the account of a user is reserved and articles are recommended based on users' long term click history. Some works using factorization machines, such as Wide\& Deep~\cite{cheng2016wide} and deepFM~\cite{guodeepfm2017}, can be adopted in news recommendation, and others use well-designed graph network to represent the target user and the clicked article~\cite{hu2020graph, ge2020graph}, and in both situations the relation of items and users are well exploited. Unfortunately, in real-time scenarios, new articles and anonymous users emerge, causing severe cold-start problem. Then if we want to capture users' preference within session and recommend articles with their several interactions as input, this kind of approach with the static user-item matrix is not suitable. Some propose incremental matrix factorization algorithm based on classic MF algorithm by adding a new user or a new article to the matrices with a random initialization~\cite{al2018adaptive}, and others apply meta-learning which aims to train a model that can rapidly adapt to a new task with a few examples~\cite{lee_melu:_2019}, but do not solve the problem fundamentally.

Second, some news recommendation systems use clicked articles to represent a user, which can be adaptive to anonymous users. Some of them encode the article text with fine-grained attention mechanism~\cite{zhu2019dan, wu_neural_2019-1, wu2019npa, wang2020fine}, some consider the relation between the dwell time of the user and satisfaction of the user~\cite{wu2020CPRS}, and others use the knowledge graph of entities in the news title as affiliated information~\cite{wang2018dkn, wang_ripplenet:_2018}. They mainly focus on the textual feature of articles in order to aggregate users' preference while pay less attention to the click behavior. Although they can be applied for anonymous users by replacing long term history clicks with articles within the session when fetch user representations, challenges are that they cannot take full advantage of the textual information due to the limited interactions and the overload of training cannot be avoided. Besides, they evaluate their methods by classifying the real clicked article and several constructed distractors from impression list, and this is not consistent with the real recommendation scenario, where the recommender retrieves top-K recommendation lists from all candidates.

For the rest of works, one uses the information of how frequent user returns to help improve recommendation~\cite{zheng2018drn}, another work jointly models click/unclick/dislike as explicit/implicit feedback~\cite{xie2020deep}, and others excavate the quality of news articles~\cite{lu_quality_2019} or the backtracking behavior as the user feedback~\cite{smadja_understanding_2019}.
\subsection{Session-based Recommendation}
Many online recommender systems are proposed to deal with the session-based scenarios~\cite{epure_recommending_2017,zhou_variational_2019}, where the user interaction information is limited and items are increasingly generated. Some traditional content-based or hybrid recommendation approaches are transferred to the session-based recommendation~\cite{sottocornola2018session}.  They combine session-based CF similarity and content-based similarity to recommend top-K similar articles. Some combine a variant of MF adapted to the incremental nature of data streams and a topic drift modeling module~\cite{al2018adaptive}. As for deep learning methods, Recurrent Neural Network (RNN), Long Short-Term Memory (LSTM) and Graph Neural Network (GNN) possess properties that make them attractive for sequence modeling of user sessions~\cite{guo_streaming_2019,hidasi2015session,wang2019modeling,moreira_news_2018,wu2019session}. Further, a hybrid encoder with an attention mechanism is introduced to model the sequential behavior of users~\cite{li2017neural,liu2018stamp,xu2019time,song_islf_2019,zhang_feature-level_2019}. 
Besides, many sequential recommendation systems~\cite{pereira2019online,xu2019graph} on music listening, e-commerce purchasing and games playing construct assorted RNN-related architectures (e.g, Recurrent Convolutional Neural Network (RCNN)~\cite{xu_recurrent_2019}, Gated Recurrent Unit (GRU)~\cite{hidasi2018recurrent}, Hierarchical Gating Network (HGN)~\cite{xiao2019hierarchical,ma2019hierarchical}), showing RNN's high capacity to modeling user shift preference. On the other hand, some reproduce the result of these recent algorithmic proposals, showing that many of them do not consistently outperform a well-tuned non-neural linear ranking method~\cite{dacrema_are_2019,ludewig_performance_2019}.

Although sequential modeling naturally takes the preference shifting into account, the implicit user feedback and content\&temporal information are neglected. When sampling negative articles, an adaptive negative sampling method based on Generative Adversarial Nets (GAN) is proposed~\cite{wang_neural_2018}. Beyond that, few works pay attention to the implicit meaning of negative samples. Randomly sampling from such continuously increasing and high-volume news articles might be fast but won't be effective enough.

\subsection{The Use of Temporal Information}
Sequence and Time Aware Neighborhood (STAN)~\cite{garg2019sequence} takes vanilla SKNN as its special case. They build static time decay functions for three factors: the position of an item in the current session, recency of a past session w.r.t. to the current session, and position of a recommendable item in a neighboring session. This approach can be regarded as rule-based SKNN, with exponential decay function, and the experiment result on e-commerce website even outperforms some deep-learning based approaches. In the deep learning model, some works design different temporal kernel functions for different consumption scenarios~\cite{wang2020make, wu2020deja}. However, the decay function of news articles is fixed, which may undermine the ability to model user's short-term preference towards different articles. Dwell time is considered in~\cite{wu2020CPRS}, but they do not measure the same dwell time at different time of the day differently. A time-interval-based GRU is proposed to model user session-level representations~\cite{lei_tissa_2019}, and some work~\cite{rakkappan2019context,xu2019time,wu_recommender_2019} treat the time feature of interactions as a temporal context, while they fail to consider the publish/click/active time in the different dimension.
