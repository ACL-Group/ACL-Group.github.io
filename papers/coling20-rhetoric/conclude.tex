\section{Conclusion and Future Work}
\label{conclude}

In this work, we presented the task of clarifying questions in the dialogue and measured how to ask better quality clarification questions. Also, we collected a large-scale Chinese free-style dialogue dataset in the field of medical consultation, called D-P QA. We used BERT to detect clarification trigger and used seq2seq to generate questions. In both of these steps, we devised six ways of selecting context to model dialogue context. Finally, our automatic evaluation, manual evaluation and preliminary analysis showed that the quality of the generated clarification question is best when taking the length of the current dialogue context into consideration. However, there are still generalization problems in the results of question generation, which can lead to inefficient dialogue. That is something we need to think about.
