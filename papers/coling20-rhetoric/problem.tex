\section{Problem}
\label{problem}

In this section, we describe in detail the components of the consultation 
dialogue. The consultation dialogue consists of two speakers $A$ and $B$. 
We use \emph{p} to denote sentences for what $A$ says and \emph{q} to 
denote sentences from the $B$. The two speakers take turns talking with 
each other. A consultation dialogue can be represented as an alternating 
sequence of utterances by the two speakers:  
$p_{1}, q_{1}, p_{2}, q_{2}, \ldots, p_{i}, q_{i}, ...$, 
where $p_i$ and $q_i$ are utterances
by each of the two speakers. Note that each utterance may contain one or more sentences. Assuming that the existing conversation is $p_{1}, q_{1}, p_{2}, q_{2}, \cdots, p_{n}$ and the next sentence is $q_{n}$ from B, n=$1, 2, \cdots, i$.  The task is to first determine whether $q_{n}$ should be
a clarification question; and second, if $q_{n}$ is a clarification question, 
generate that question $q_{n}$. In the context of consultation dialogues,
$q_i$ is an utterance by the expert, which is more likely to ask a 
clarification question.
