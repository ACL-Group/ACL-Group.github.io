\section{Conclusion}
%We implement a prototype system for a novel
%sequence-based non-projective dependency parsing framework
%and get a competitive result on non-projective treebanks.
%Further, we could promote the system by introducing distributed
%representation to explore more complex features and applying searching
%methods to make up for the greedy parsing process.
We develop a novel sequence-based dependency parsing
framework. It shows promising results despite of an unoptimized
implementation.
The key idea is that a good parsing sequence can be predetermined
and can contribute to good parsing accuracy and substantial
speedup. Although only a few simple approaches are attempted
to train the sequence predictor, the framework allows the integration
of better and more advanced models,
which may lead to results closer to an upper bound 93.59\% \footnote{Take the sequence inferred the oracle actions from MaltParser as both training sequence and parsing sequence and define only first and second order features in MSTParser for head mapper to get this raw bound} for the WSJ test set.
%We have to emphasis that the key idea of this paper is to
%propose the sequence-based framework, instead of just combining graph-based and
%transition-based methods.
\cut{We currently predict sequence by Malt action classifier,
which offers a best result among our preliminary attempts. }

Even though the current classifier based sequence predictor
produces better results among our preliminary attempts,
the parsing accuracy is limited by the
rather localized or even incorrect sequence order produced.
%which is the main cause of not reaching the upper bound(93.59\%).
More importantly, we discovered that the parsing accuracy is very
sensitive to the quality of parsing sequence. Future work
can be focused on developing better sequence predictors that
outperform this classifier based method.

Graph-based methods spend most of the time extracting features.
Some work attempted to save time by displaying arc
filter~\cite{bergsma2010fast,rush2012vine}. We can incorporate some of these
techniques to speed up the parsing. Furthermore, Beam search works well in a
left-to-right head attaching. We can also adapt beam search to
our framework so as to relax its strictly greedy nature.
