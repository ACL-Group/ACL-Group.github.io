\section{Related Work}

\subsection{Chinese word segmentation}

Statistical Chinese word segmentation has been studied for decades. ~\newcite{xue2003chinese} was the first to treat it as a sequence tagging problem, using a maximum entropy model. ~\newcite{Peng:2004:CSN:1220355.1220436} achieved better results by using a conditional random field model ~\cite{lafferty2001conditional}. This method has been followed by many other works ~\cite{DBLP:conf/paclic/ZhaoHLL06,Sun:2012tl}.

Recently, neural network models have been applied on CWS. These methods use automatically derived features from neural network instead of hand-crafted discrete features. \newcite{DBLP:conf/emnlp/ZhengCX13} first adopted neural network architecture to CWS. ~\newcite{DBLP:conf/emnlp/ChenQZLH15} used Long short-term memory(LSTM) to capture long term dependency. ~\newcite{DBLP:conf/acl/ChenQZH15} proposed a gated recursive neural network (GRNN) to incorporate context information. In this paper, we adopt Bidirectional LSTM-CRF Models~\cite{DBLP:journals/corr/HuangXY15} as our base model.

% CWS is a preliminary step for Chinese natural language processing. It has long been treated as a sequence tagging problem since ~\cite{xue2003chinese}. Supervised learning methods are used, including maximum entropy ~\cite{low2005maximum}, conditional random fields ~\cite{lafferty2001conditional,Peng:2004:CSN:1220355.1220436,DBLP:conf/paclic/ZhaoHLL06}. These methods depend heavily on hand-crafted features. Recently, neural networks have been applied on CWS tasks. \newcite{DBLP:conf/emnlp/ZhengCX13} first introduced the neural network architecture to CWS task. Later, different variants of RNN and score functions are developed to improve the performance~\cite{DBLP:conf/acl/PeiGC14,DBLP:conf/emnlp/ChenQZLH15,DBLP:conf/acl/ChenQZH15,DBLP:journals/corr/CaiZ16,DBLP:conf/acl/CaiZZXWH17}. 
 % \newcite{DBLP:conf/emnlp/ChenQZLH15} adopted the long short-term memory(LSTM) to keep long dependency and avoided the limit of window size of local context. \newcite{DBLP:conf/acl/ChenQZH15} proposed to employ a gated recursive neural network to incorporate the complicated combinations of the context characters. \newcite{DBLP:journals/corr/CaiZ16} employed a factory to produce word representation given governed characters and proposed sentence-level likelihood evaluation system for CWS. ~\cite{DBLP:conf/acl/CaiZZXWH17} proposed a greedy neural word segmenter with balanced word and character embedding.
 % Besides, joint CWS with part-of-speech tagging was proven to improve both tasks~\cite{DBLP:journals/corr/ChenQH16a,DBLP:conf/ijcai/ChenQH17}. Also, the heterogeneous annotating problem was discussed~\cite{DBLP:conf/emnlp/QiuZH13,DBLP:conf/acl/ChenSQH17}.
\subsection{Transfer Learning}

Transfer learning distills knowledge from source domain and helps target domain to achieve a higher performance~\cite{Pan:2010:STL:1850483.1850545}. In feature-based models, many transfer approached have been studied, including instance transfer~\cite{DBLP:conf/acl/JiangZ07,DBLP:conf/icml/LiaoXC05}, feature representation transfer~\cite{DBLP:conf/nips/ArgyriouEP06,DBLP:conf/nips/ArgyriouMPY07}, parameter transfer\cite{DBLP:conf/icml/LawrenceP04,DBLP:conf/nips/BonillaCW07} and relation knowledge transfer\cite{DBLP:conf/aaai/MihalkovaHM07,Mihalkova09transferlearning}. 

Recently, the transferability of neural networks is also studied. For example, ~\cite{DBLP:journals/corr/MouMYLXZJ16} studied two methods (INIT, MULT) on NLP applications. \newcite{DBLP:journals/corr/PengD16a} proposed to use domain mask and linear projection upon multi-task learning (MTL)~\cite{DBLP:journals/corr/Long015a}. In this paper, we follow MTL and extend the framework with a novel loss function.

% However, there's little study on transfer learning for neural networks. ~\cite{DBLP:journals/corr/MouMYLXZJ16} used intuitive methods (INIT, MULT) to study the transferability of neural networks on NLP applications. \newcite{DBLP:journals/corr/PengD16a} proposed to use domain mask and linear projection upon multi-task learning~\cite{DBLP:journals/corr/Long015a}.
