\section{Evaluation}
\label{sec:eval}
\subsection{Datasets}
CNN/Daily Mail~\cite{HermannKGEKSB15}
\footnote{\url{https://cs.nyu.edu/~kcho/DMQA/}} 
is a popular summarization dataset, 
which contains news articles paired with summaries.
There are 286,817 training pairs,
13,368 validation pairs and 11,487 test pairs.
We follow \cite{SeeLM17} in data preprocessing and use 
the non-anonymized version.


\subsection{Model Parameters and Evaluation Metrics}
\label{sec:expset}
In this experiments
~\footnote{
All of the data and source code
can be downloaded from http://Anonymous.},
all the competing models contain $8$ convolutional layers in
both encoders and decoders, with kernel width of $3$.
For each convolutional layer, 
we set the hidden state size to $512$ and the embedding size to $256$.
To alleviate overfitting,
we apply a \textit{dropout} ($p=0.2$) layer to 
all convolutional and fully connected layers.
We take gradient method with gradient clipping $0.1$, momentum $0.99$,
and initial learning rate $0.2$.
Training terminates when learning rate $\le 10e$-$5$.
Beam size $b=5$ at test time.
We set the threshold $sz$ to $3$, 
because nearly $90\%$ 
of sections are with length$>=$3.
We set $n$ (Equation (\ref{eq:s})) to $5$,
since less than $5\%$ of reference summaries have
the LCS of less than $5$.

We use the following evaluation metrics:
\itemsep0em
\begin{itemize}

\item \textbf{ROUGE} scores (F1) include ROUGE-1 (R-1), ROUGE-2 (R-2) and
ROUGE-L(R-L)~\cite{rouge-a-package-for-automatic-evaluation-of-summaries}.

\item \textbf{Repeatedness} (Rep) 
includes N-gram repeatedness, sentence repeatedness
and total repeatedness.
\textbf{N-gram} or \textbf{sentence repeatedness} is the percentage of repeated N-grams 
or sentences in a summary.
%\begin{equation}
%\small Rep = \frac{n}{N}
%\end{equation}
%where $n$ is the number of repeated N-grams or sentences, 
%$N$ is the total number of N-grams or sentences in a summary.
We use \textit{sim} in \eqnref{eq:s} to
identify repeated sentences.
\textbf{Total repeatedness} (Algorithm \ref{alg:red}) is a comprehensive score
that unifies word-level and sentence-level repeatedness.
%It is not computed by N-gram repeatedness score 
%and sentence repeatedness score.
\begin{minipage}{10cm}
%\setlength\parindent{3em}
\begin{algorithm}[H]
\scriptsize
\caption{Calculation of Total Repeatedness}
\label{alg:red}
\textbf{Input}: a sentence set $s = {s_{1}, s_{2},...,s_{n}}$\\
%\textbf{Parameter}: Optional list of parameters\\
\textbf{Output}: Total repeatedness percentage $p$
\begin{algorithmic}[1] %[1] enables line numbers
\STATE Let $total$ be the sum of lengths of the sentences in $s$.
\STATE $n \leftarrow total$, $overlap \leftarrow 0$
\WHILE{$n \geq 3$}
\STATE The lengths of LCS between two sentences from $s$ comprise a length set $L$.
\STATE $n \leftarrow \max(L)$.
\STATE Find a substring $b$ with length $n$ that appears most frequently in $s$.
\STATE Let $k$ be the frequency that $b$ appears in $s$.
\STATE $overlap \leftarrow overlap + k\cdot n$
\STATE Remove every appearance of substring $b$ from sentences in $s$.
\ENDWHILE
\STATE $p \leftarrow overlap/total$
\STATE \textbf{return $p$} 
\end{algorithmic}
\end{algorithm}
\end{minipage}


\item \textbf{Readability} (Readable) is a human evaluation,
which reflects the fluency and readability of the summary.
We randomly sample 300 summaries generated by each model
and educate human annotators to assess each summary
from three independent perspectives: 
(1) Informative: How informative the summary is? 
(2) Coherent: How coherent (between sentences) the summary is? 
(3) Fluent: How grammatical and factual the sentences of a summary are? 
Readability score will be judged on the following 5-point scale:
Very Poor (1.0), Poor (2.0), Barely Acceptable (3.0), Good (4.0) and Very Good (5.0).
The Cohen's Kappa coefficient between them is $0.78$, 
indicating agreement. Here we use the average annotation score.
\end{itemize}

%We use \textit{readability} to complement ROUGE scores 
%Since \cite{YaoWX17} showed that the standard 
%ROUGE scores cannot capture grammatical or factual errors. 

\subsection{Baselines}
Our goal is to evaluate the
effectiveness of our repetition reduction technique.
We choose to implement 
%all existing repetition reduction techniques (\tabref{tab:baselines}) on top of vanilla CNN seq2seq model.
all existing repetition reduction techniques on top of vanilla CNN seq2seq model.
Because the vanilla CNN seq2seq model is fast and enjoys the best accuracy among
the other vanilla RNN seq2seq models 
%such as 
%RNN seq2seq model and LSTM seq2seq model
~\cite{bai2018empirical,gehring2017convs2s}.

\cut{%%%%%%%%%%%
\begin{table}[th]
	\caption{Baselines}
	\centering
	\begin{tabular}{|l|l|}
		\hline
		\textbf{Abbrev.} & \textbf{Description} \\ \hline
		\textbf{CNN} &  Convolutional seq2seq model~\cite{gehring2017convs2s} \\
		\hline
		\textbf{ITA} &  Intra-temporal attention~\cite{NallapatiZSGX16} \\
		\hline
	%	\textbf{ITDA} & \tabincell{l}{Intra-temporal attention and intra-decoder attention\\ \cite{PaulusXS17,FanGA18}}\\
		\textbf{ITDA} & \tabincell{l}{Intra-temporal attention and intra-decoder attention~\cite{PaulusXS17,FanGA18}}\\
		\hline
	    \textbf{COV}	& Coverage mechanism~\cite{SeeLM17}\\
		\hline
	    \textbf{COVP}	& Coverage penalty~\cite{GehrmannDR18}\\
		\hline
	    \textbf{SCL}	& Semantic cohesion loss~\cite{elikyilmazBHC18}\\
		\hline
        \textbf{TRI} & Trigram decoder~\cite{PaulusXS17} \\
		\hline
	\end{tabular}
	\label{tab:baselines}
\end{table}
}%%%%%%%%%%%%


We did not implement the repetition reduction methods 
on top of the seq2seq models with higher ROUGE scores,
because the effectiveness of the repetition reduction is not necessarily
reflected in the ROUGE~\cite{YaoWX17}
As shown in \tabref{tab:compete_exp}, 
after reducing repetition, the summary becomes better
but the ROUGE score is not improved. 
Therefore our evaluation mainly compares
the effectiveness of different repetition reduction techniques
in terms of all four metrics above.
As we known, ROUGE is not very good at evaluating abstractive summarization
and the room for improvement on the ROUGE scores are very limited.
If the repetition reduction methods 
were applied on top of the models with higher ROUGE scores, 
the differences in ROUGE scores by these repetition reduction techniques will be
indistinguishable and complicate the analysis. 
Hence, in this work, 
we construct seven baselines 
%(\tabref{tab:baselines})
by converting
repetition reduction techniques developed on RNN seq2seq models to their
counterparts on vanilla CNN seq2seq models,
to be fair.
%The baselines are listed in \tabref{tab:baselines}.
The 7 baselines are:
(1) CNN seq2seq model~\cite{gehring2017convs2s}
(2) ITA: Intra-temporal attention~\cite{NallapatiZSGX16}
(3) ITDA: Intra-temporal attention and intra-decoder attention~\cite{PaulusXS17,FanGA18}
(4) COV: Coverage mechanism~\cite{SeeLM17}
(5) COVP: Coverage penalty~\cite{GehrmannDR18}
(6) SCL: Semantic cohesion loss~\cite{elikyilmazBHC18}
(7) TRI: Trigram decoder~\cite{PaulusXS17}.
\begin{table}[th!]
\scriptsize
\begin{center}
\caption{Example of generated summaries}
\begin{tabular}{|l|l|c|}%{|p{7cm}|rl|}
  \hline & \bf summary & \bf R-2 \\
  \hline \bf Reference & timberlake and jessica biel welcome son silas randall timberlake. 
  the couple announced the pregnancy in january . & - \\
  \hline \textbf{COV} & \tabincell{l}{timberlake and jessica biel announced the pregnancy in january. 
       the couple announced the pregnancy in january.} & 0.60 \\
  \hline \textbf{ATTF+SBD} & \tabincell{l}{the couple announced the arrival of their son, silas randall timberlake. 
       the couple announced the pregnancy in january. \\ it is the first baby for both.} & 0.52 \\
  \hline
\end{tabular}
\label{tab:compete_exp}
\end{center}
\end{table}



\subsection{Results}
\label{sec:result}

\textbf{Accuracy.} As shown in \tabref{tab:eval_main}, 
our model (ATTF+SBD) outperforms all the baselines in ROUGE scores, 
indicating we are able to generate more
accurate summaries. 

\begin{table*}[th]
\scriptsize
	\centering
	\begin{tabular}{|c|ccccccc|cccccc|}
		\hline
	            & CNN  & ITA & ITDA & COV & COVP & SCL & ATTF & TRI* & SBD-b1* & SBD-b2* & SBD *& ATTF+TRI* & ATTF+SBD* \\
		\hline
		R-1 & 34.33 & 34.30 & 34.62 & 35.85 & 34.53 & 35.13 & \bf 36.32 & 36.81 & 34.24 & 35.88 & 37.19 & 37.33 &\bf 37.69 \\
		R-2 & 14.25 & 14.20 & 14.52 & 14.81 & 14.41 & 14.61 & \bf 15.38 & 15.47 & 14.33 & 14.83 & 15.45 & 16.65 & \bf 17.02\\
		R-L & 35.68 & 35.67 & 35.94 & 35.96 & 35.81 & 35.93 & \bf 36.09 & 36.00 & 34.75 & 35.15 & 36.03 & 36.30 & \bf 36.47\\
		\hline
	\end{tabular}
	\caption{ROUGE scores on CNN/Daily Mail. The models with * do corresponding operations at test.
We take t-test 
as our significance test.
All p-values between baselines and ATTF+SBD are less than 0.05,
which means that the difference of the similarity results is significant. 
	}
	\label{tab:eval_main}
\end{table*}

Without any special operations at testing, 
our ATTF model achieves the highest score on ROUGE, showing
its effectiveness in improving summary quality.
%ATTF is effective to improve the summarization quality of basic CNN seq2seq models.
Models with SBD or TRI at testing
are more effective than the basic CNN seq2seq model,
because more information is involved in summary generation 
as a by-product of repetition reduction.
Compared with its two variants, SBD is a little slower 
but has higher ROUGE scores, reflecting its advantage due to
better choices taken globally.
Therefore, 
we use SBD as our backtracking decoder in the following experiments. 
The number of explored candidate hypothesis, up to a point of
repetition, is less than 30 tokens.
The ROUGE score of SBD is higher than TRI on R-1 and R-L, but lower on R-2. 
The reason is that R-2 and R-L respectively evaluate
bigram-overlap and longest common sequence between the reference
summary and generated summary. This is in line with different techniques 
in SBD and TRI, the former promoting the diversity of sentences and 
the latter promoting that of trigrams.
SBD has higher ROUGE scores than ATTF, 
because the summaries from
SBD do not have the repetition caused by attending to similar sentences in source.
Unlike ATTF, 
SBD cannot obtain the ability to attend to different POIs through training.
In \tabref{tab:src_rep}, the sentences in SBD are not repetitive, 
but summarized from the same POI.
The summaries may lose important information when only using SBD.
The readability score of SBD is lower than ATTF in \tabref{tab:eval_repe}.

\begin{table}[th!]
\scriptsize
\begin{center}
\caption{Repeatedness scores (\%) and Readability scores on CNN/Daily Mail dataset.
The ``Gold'' denotes reference summaries, which are the most readable.
%By default, the readability score of reference summaries is judged to be 5.0. 
%The summaries of models with * do corresponding operations at test.
} 
            \begin{tabular}{|c|c|ccccccc|cccc|}
                \hline
                    & Gold & CNN  & ITA & ITDA & COV & COVP & SCL & ATTF & TRI* & SBD* & ATTF+TRI* & ATTF+SBD* \\
                \hline
                1-gram & 33.79 & 56.25 & 54.44 & 51.18 & 42.18 & 52.46 & 52.23 & \bf 34.98 & 31.91 & \bf 29.88 & 32.0 & 30.83 \\
                2-gram & 2.98 & 36.55 & 34.76 & 30.64 & 16.77 & 32.11 & 34.08 & \bf 8.16 & 3.17 & \bf 2.84 & 2.94 & 3.71 \\
                3-gram & 0.43 & 32.62 & 31.10 & 27.14 & 12.95 & 28.59 & 30.58 & \bf 5.11 & \bf 0.0 & 0.40 & \bf 0.0 & 0.74 \\
                Sent & 3.98 & 49.45 & 48.34 & 42.96 & 14.52 & 25.52 & 27.58 & \bf 6.69 & \bf 0.0 & 3.47 & \bf 0.0 & 3.44 \\
                \hline
                Total-Rep & 0.51 & 18.86 & 17.94 & 15.62 & 7.77 & 16.46 & 17.65 & \bf 3.27 & \bf 0.0 & 0.44 & \bf 0.0 & 0.80 \\
                \hline
                Readable & - & 2.97 & 3.23 & 3.54 & 3.72 & 3.80 & 3.75 & \bf 4.53 & 3.75 & 4.13 & 3.88 & \bf 4.67 \\
                \hline
            \end{tabular}
\label{tab:eval_repe}
\end{center}
\end{table}


For models that tackle repetition both at training and test time, 
ATTF+SBD outperforms ATTF+TRI.
SBD works in synergy with ATTF, and they together process 
information with \textit{section/segment} as a unit.
%The ROUGE scores of ATTF+SBD are lower than
%those of SOTA models 
%because rather than reducing repetition, the SOTA models use 
%other structural tricks to enhance ROUGE scores 
%such as pointer-generator and reinforcement learning.
%These structures are orthogonal 
%to our attention filters
%and we expect them to work just as well on our model if applied.
%\cite{FanGA18} shows that these structural tricks can work just as well on seq2seq models if applied.
%Aftering adding ATTF+SBD, R-2 is increased by 1.82.
%In those SOTA model, R-2 is increased by no more than 1.6 after adding other baselines alone.
ATTF+SBD scores higher ROUGE than the other baselines, 
demonstrating its power to  reduce 
repetition and generate more accurate summaries.
Besides, as shown in \tabref{tab:compete_exp}, the quality of a summary cannot be evaluated by
ROUGE scores alone.
%The R-2 scores of above example: COV 0.60, ATTF+SBD 0.52.
ATTF+SBD obviously produces a better, logically more consistent summary despite 
a lower ROUGE score.  
Due to variable nature of abstractive summarization, ROUGE is
not the optimal evaluation metric.
Repeatedness and Readability score, 
in our opinion, are important complementary metrics to ROUGE scores.  


\cut{%%%%%%%%%%%%%%
\begin{table}[th!]
\begin{center}
\subtable[The models without operations at test.]{
	    \begin{tabular}{|c|c|ccccccc|}
		\hline
	            & Gold & CNN  & ITA & ITDA & COV & COVP & SCL & ATTF \\
		\hline
		1-gram & 33.79 & 56.25 & 54.44 & 51.18 & 42.18 & 52.46 & 52.23 & \bf 34.98 \\
		2-gram & 2.98 & 36.55 & 34.76 & 30.64 & 16.77 & 32.11 & 34.08 & \bf 8.16 \\
		3-gram & 0.43 & 32.62 & 31.10 & 27.14 & 12.95 & 28.59 & 30.58 & \bf 5.11 \\
		4-gram & 0.12 & 30.18 & 28.85 & 25.04 & 11.17 & 26.48 & 28.34 & \bf 4.19 \\
		Sent & 0.08 & 37.04 & 35.79 & 31.46 & 13.98 & 24.63 & 26.38 & \bf 3.56 \\
		\hline
		Total-Rep & 0.51 & 18.86 & 17.94 & 15.62 & 7.77 & 16.46 & 17.65 & \bf 3.27 \\
		\hline
		Readable & 1.0 & 0.65 & 0.75 & 0.76 & 0.80 & 0.76 & 0.75 & \bf 0.86 \\
		\hline
	    \end{tabular}
        }
\qquad
\subtable[The models with operations at test.]{
        %\begin{tabular}{lcccccccc}
    	\begin{tabular}{|c|c|cccc|}
		\hline
	            & Gold & TRI* & SBD* & ATTF+TRI* & ATTF+SBD* \\
		\hline
		1-gram & 33.79 & 31.91 & \bf 29.88 & 32.0 & 30.83 \\
		2-gram & 2.98 & 3.17 & \bf 2.84 & 2.94 & 3.71 \\
		3-gram & 0.43 & \bf 0.0 & 0.40 & \bf 0.0 & 0.74 \\
		4-gram & 0.12 & \bf 0.0 & 0.06 & \bf 0.0 & 0.13 \\
		Sent & 0.08 & \bf 0.0 & \bf 0.0 & \bf 0.0 & \bf 0.0 \\
		\hline
		Total-Rep & 0.51 & \bf 0.0 & 0.44 & \bf 0.0 & 0.80 \\
		\hline
		Readable & 1.0 & 0.75 & 0.81 & 0.77 & \bf 0.93 \\
		\hline
	    \end{tabular}
        }
\caption{Repeatedness scores (\%) and Readability scores on CNN/Daily Mail dataset.}
\label{tab:eval_repe}
\end{center}
\end{table}
}%%%%%%%

\textbf{Repeatedness.}
To demonstrate the effectiveness of ATTF and SBD in reducing repetition, 
we compare \textit{repeatedness} (\tabref{tab:eval_repe}) 
of generated summaries.
%The lower repeatedness reflects larger ability of reducing repetition.
Lower repeatedness 
means the model is more capable of reducing repetition.
In \tabref{tab:eval_repe}, Gold row shows the repeatedness scores of
reference summaries. ATTF achieves the lowest
score among all baselines without any operations at test time. 
%It denotes that our model has ability to remember 
%the summarized part of source document by segments in summary.  
%Compared with summaries generated by 
As shown in \tabref{tab:example}, \tabref{tab:strong_methods} and \figref{fig:attn_maps},
baseline models suffer from severe repetition problem because they attend to the same POIs 
of the source document, whereas 
ATTF attends to different POIs and generates summaries 
such as this:

\fbox{
\small
\parbox{0.9\columnwidth}{
\textbf{ATTF}: manchester city are rivalling manchester united and arsenal for defender dayot 
pamecano. the 16-year-old joined in the january transfer window only for 
him to opt to stay in france.
}}

\begin{figure}[th!]
\centering
\subfigure[ITA]{
%\includegraphics[width=0.16\linewidth]{mapITA}
\includegraphics[width=0.13\linewidth]{mapITA}
}
\quad
\subfigure[ITDA]{
\includegraphics[width=0.13\linewidth]{mapITDA}
}
\quad
\subfigure[COV]{
\includegraphics[width=0.13\linewidth]{mapCOV}
}
\quad
\subfigure[COVP]{
\includegraphics[width=0.13\linewidth]{mapCOVP}
}
\quad
\subfigure[SCL]{
\includegraphics[width=0.13\linewidth]{mapSCL}
}
\quad
\subfigure[ATTF]{
\includegraphics[width=0.13\linewidth]{map2}
}
\caption{Attention distribution of summaries for the source in \tabref{tab:example}}
\label{fig:attn_maps}
\end{figure}

Compared with the Gold standard,
ATTF still generates some repetitive sentences,
due to similar sentences in source
such as \exref{ex:repeatsrc}.
%The result of summarizing that document using ATTF and its local attention map are
The summary generated by ATTF and its local attention are
shown in \tabref{tab:src_rep}.
%and \figref{fig:attn_map3}.
Also, SBD further reduces the repetition when combined with ATTF. 
%which demonstrates its effectiveness.

\begin{table}[th!]
\scriptsize
\begin{center}
\caption{Summaries generated from \exref{ex:repeatsrc}.The parts of Source and Reference in same color denote local attention.}
\begin{tabular}{|l|}%{|p{7cm}|rl|}
\hline \textbf{Source:} ..fixture in the league on saturday sees leaders chelsea welcome 
       manchester united ... \color{red}{chelsea midfielder oriol romeu, currently on loan at}\\
	   \color{red}{stuttgart}, \color{black}{...} \color{green}{romeu is currently on a season-long loan at bundesliga side stuttgart}.\\
\hline \textbf{Reference:} oriol romeu is on a season-long loan at stuttgart from chelsea . 
       the spanish midfielder predicts the scores in saturday 's matches . oriol goes \\
	   head-to-head with sportsmail 's martin keown .\\
\hline \textbf{ATTF:} chelsea beat manchester united on saturday . \color{red}{\textit{oriol romeu is currently
       on a season-long loan at stuttgart}}. \color{green}{\textit{oriol romeu is currently on a season-long}} \\
	   \textit{\color{green}{loan at bundesliga side stuttgart}.}\\
\hline \textbf{SBD:} chelsea beat manchester united on saturday . chelsea face manchester 
       united in the premier league . \\ 
\hline \textbf{ATTF+SBD:} chelsea face manchester united in the premier league on saturday . 
       oriol romeu is currently on loan at stuttgart . \\
\hline
\end{tabular}
\label{tab:src_rep}
\end{center}
\end{table}

As shown in \tabref{tab:eval_repe}, TRI has the lowest total repeatedness score.
It does not generate any repetitive N-grams (N$>$2) and sentences 
because TRI prevents the generation of the same trigrams during testing.
But as the Gold row shows, reference summaries do have some natural repetition.
%As reference summaries are human written, 
Therefore we evaluate the correlation of repeatedness distribution between
generated summaries and reference summaries (\tabref{tab:eval_repcor}).
Our proposed models perform best,
which indicates that ATTF and SBD are more capable of producing summaries with a natural level of repeatedness.
%It indicates that ATTF and SBD are more capable of producing summaries with a natural level of repeatedness.
%missing POIs and repetition in source documents.

\begin{table}[th!]
\scriptsize
        \centering
        \caption{Repeatedness correlation between generated summaries and Gold summaries.}
        \begin{tabular}{|l|ccccc|}
                \hline
                     & ATTF & TRI & SBD & ATTF+TRI & ATTF+SBD \\
                \hline
                pearson & \bf 1.0 & 1.0 & \bf 1.0 & 1.0 & \bf 1.0\\
			    spearman & \bf 1.0& 0.89 &\bf 1.0 & 0.86 &\bf 1.0\\
				kendall's tau & \bf 1.0 & 0.84 & \bf 1.0 & 0.84 & \bf 1.0 \\
                \hline
        \end{tabular}
        \label{tab:eval_repcor}
\end{table}

%\begin{figure}[th!]
%\centering
%\includegraphics[width=0.5\columnwidth]{map3}
%\caption{Attention distribution for ATTF in \tabref{tab:src_rep}}
%\label{fig:attn_map3}
%\end{figure}
	
\textbf{Readability.}
As shown in \tabref{tab:eval_repe}, 
the models with ATTF achieve the
%ATTF achieves the
highest readability score among all baselines, 
which means ATTF is more readable.
%without any operations during test.
TRI achieves the best score on repeatedness, 
% which represents the repetition over N-gram and sentences,
but lower readability score than other models.
Also, the readability of ATTF drops after adding TRI.
%All of our models score higher than $0.80$. 
SBD enhances performance of CNN and ATTF by reducing the repetitive unreadable sentences. 
ATTF+SBD scores highest on readability.
\cut{%%%%%%%%%
\begin{table}[th!]
	\centering
	\begin{tabular}{|l|c|c|c|}
		\hline
		     & pearson  & spearman & kendall's tau \\
		\hline
		TRI* & 1.0 & 0.894 & 0.837  \\
		SBD* & 1.0 & 1.0 & 1.0 \\
		ATTF+TRI* & 1.0 & 0.894 & 0.837 \\
		ATTF+SBD* & \bf 1.0 & \bf 1.0 & \bf 1.0 \\
		\hline
	\end{tabular}
    \caption{Repeatedness correlation between generated summaries and Gold summaries.}
	\label{tab:eval_repcor}
\end{table}
}%%%


\textbf{Speed.} 
As shown in \tabref{tab:eval_main} and \tabref{tab:eval_speed}, 
SBD is the best sentence-level backtracking decoder.
Compared with SBD-b1 and SBD-b2,
SBD logs higher ROUGE scores without losing much on speed. 
We compare the speed of our model to RNN~\cite{SeeLM17} and FastRNN~\cite{P18-1063}
which used K40. 
We perform experiments on GTX-1080ti and scale the speed 
reported for the RNN methods,
since GTX-1080ti is twice as fast as K40~\cite{gehring2017convs2s}.
Our model can 
generate summaries much faster than previous RNN seq2seq models.
\begin{table}[th!]
\scriptsize
\centering
\caption{Testing time and speed of generation.}
\begin{tabular}{|l|cc|cccccc|}
\hline
Model & RNN & FastRNN & CNN & SBD-b1 & SBD-b2 & SBD & ATTF & ATTF+SBD \\
\hline
Time (s) & 21600 & 3600 & 346.1 & 412.8 & 843.5 & 912.8 & 1332 & 1832.3 \\
summaries/s & 0.48 & 2.92 & 30.36 & 25.46 & 12.16 & 11.51 & 7.89 & 5.74 \\ 
tokens/s & 29.60 & 219.60 & 1343.46 & 1126.38& 551.24 & 493.68 & 349.00 & 253.77 \\
\hline
\end{tabular}
\label{tab:eval_speed}
\end{table}


%FastRNN is also fast because 

%\subsection{Significance Test on ROUGE scores}
\cut{%%%%%%%%%%%%%
\begin{table}[th!]
\begin{center}
\caption{p-value of significance test between 
our best proposed model (ATTF+SBD) and baselines on ROUGE scores}
		\begin{tabular}{|l|c|c|c|}
		\hline
		Model &   R-1 & R-2 & R-L \\
		\hline
		CNN &  2.32e-35 & 6.34e-48 & 3.68e-10 \\
		ITA &  6.14e-34 & 2.12e-48 & 5.67e-12 \\
		ITDA & 2.76e-32 & 4.52e-44 & 3.94e-12 \\
		COV	& 4.14e-30 & 4.61e-50 & 7.12e-12 \\
		COVP & 2.51e-32 & 3.17e-41 & 2.15e-10 \\
		SCL	& 3.11e-32 & 3.29e-44 & 3.43e-15 \\
		TRI & 5.25e-30 & 1.33e-43 & 3.67e-12 \\
		\hline
		\end{tabular}
\label{tab:ttest}
\end{center}
\end{table}

\textbf{Significance Test.} We use significance test to prove that the ROUGE scores in \tabref{tab:eval_main} is reliable.
We take t-test 
\cite{loukina2014automatic,albert2017exploring}
as our significance test to
measure that the ROUGE scores between our proposed approach (ATTF+SBD) and each baseline are significant or not. 
All p-values between 7 baselines and ATTF+SBD are less than 0.05,
which means that the difference of the similarity results is significant. 
}%%%%%%
				

%Overall, the summaries generated by sequence-to-sequence models with attention mechanism always contain repetition.  
%Through our observations, there are two reasons for repetition in abstractive summarization.
%One is that the traditional attention mechanisms attend to the same location in source document at decoding.
%The other is that the attention mechanism attend to the repetitive sentences in different locations in source document. 
%As shown in \figref{fig:attn_maps} and \tabref{tab:src_rep},
%our proposed ATTF and SBD effectively solve above two problems.  
%The higher ROUGE scores (\tabref{tab:eval_main}) of our model means that
%the summaries generated by our model are more similar to their corresponding reference summaries.
%The natural-level repeatedness and higher readability score (\tabref{tab:eval_repe}) of our model means 
%that our model can produce summaries with higher quality.
%Thus, our model can improve the reading speed and accuracy of reading comprehension.


