\section{Conclusion}
\label{sec:conclusion}

In this paper,
we point out that the biggest challenge of current recommender systems
is that they are not directly driven by user needs, 
which, however, are precisely the ultimate goal of recommender systems try to satisfy.
To tackle it, we introduce a specially designed e-commerce knowledge graph practiced in Taobao, trying to conceptualize user needs as various shopping scenarios, also known as e-commerce concepts. 
We further proposed a deep interpretable inference model to intuitively infer those concepts accurately.
On our real-world e-commerce dataset, the proposed model achieved state-of-the-art performance against several strong baselines.
After applying to online recommender system, great gain regarding both accuracy and novelty are achieved, which ultimately improved user satisfaction.
More importantly, we believe that the idea of conceptualizing and inferring user needs can also be applied to e-commerce search engine and advertising system as well. In the future, we will continuously explore various possibilities of ``user-needs driven'' e-commerce.