\section{Introduction}
%Knowledge discovery is a hot research topic in computer community
%\cite{WuLWZ12,HoffartSBW13,Jamieson-Patent}. Usually user's data are stored
%in database or in raw file with their original format. How to mine these data to find useful information
%is quite interesting and important. Knowledge is mined from raw data. In many scenarios it's much more efficient
%to use knowledge than the raw data.
%
Knowledge is the high level, structured abstraction from basic facts.
Recently, there has been many attempts to extract and construct knowledge
from unstructured text data\cite{WuLWZ12,HoffartSBW13}.
In this paper, we present a Chinese association knowledge-base,
which is extracted from three Chinese encyclopedias,
Chinese Wikipedia \cite{ch-wiki}, Baidu Baike
\cite{baidu-baike}, Hudong Baike \cite{hudong-baike}.
The articles in these three encyclopedias are used to discover associations
between two terms, such as ``party'' and ``beer.''
The relationship between these two terms can be explained as a conditional
probability $p(party|beer)$, or the likelihood
that one thinks of ``party'' when mentioning ``beer.''
The probability is derived generally from the sentence-level co-occurrence
within the three corpora and the co-occurrences between the title and the body
of the encyclopedia articles. We treat these two kinds of co-occurrences
separately because we believe they carry different aspects of the association.
The whole knowledge base can be viewed as a weighted directed network where
Chinese terms are nodes and directed edges are the associations.
%The sentence level and title-body level co-occurrence score is used to derive the relatedness between
%two Chinese terms such as Shanghai and Beijing. When displaying the related concepts for a certain term we use
%a clustered manner which makes the demonstration more friendly to end user.
%
%Especially, we build a Chinese name association knowledge-base. This knowledge-base is also concluded from
%the three encyclopedia- Chinese Wikipedia, Baidu Baike, Hudong,Baike. The entities in this knowledge base is
%the name of both Chinese and foreign person really and fictional person. For example, the terms will like
%Mao ZeDong and Deng Xiaoping. The relationship between two terms in this knowledge-base is also the probability
%that you will consider personA when you think about personB. For example, there are two persons in this knowledge-base
%Mao ZeDong and Deng Xiaoping. The knowledge-base then will give the relationship between Mao Zedong and Deng Xiaoping
%as $P(Deng Xiaoping|Mao Zedong)$, that is, the probability you will consider Deng Xiaoping when there is an Mao Zedong
%in your mind.

This comprehensive, open-domain association network can be
used in many scenarios \cite{JiangLD12,TangWSS12}.
One of them is {\em cross-domain} recommendations in e-commerce.
User $A$ can be represented by a list of commodities or services he
or she has purchased, i.e., $t_1, \ldots, t_n$. We can then carry out two kinds
of recommendation. For content-based recommendation, we can calculate
the likelihood of the user to purchase the next product $t_{n+1}$ as
$p(t_{n+1} | t_1, \ldots, t_n)$, which maybe approximated by $p(t_{n+1} | t_1)$,
$p(t_{n+1}, t_2)$, etc.
For collaborative filtering recommendation, we can calculate the
resemblance between user $A$ and user $B$ by connecting the items purchased
by $A$ and $B$ through the association network.

%User purchase behavior will be modeled as a list of terms related to
%the commodities that he/she has purchased. The commodity is also
%modeled as a list of terms. Then the recommendation system could use the association knowledge-base to infer the
%probability that the user will purchase commodity A with the user profile B, that is, $P(A|B)$.
%This probability will finally be represented as a product of the scores in the association knowledge-base.
%With the help of our association knowledge-base, the diversity and novelty of the recommendation will be significantly improved.

Next we introduce the three corpora,
then outline the construction of the knowledge base,
and finally explain the demo setup.
% which includes both
%an interface to the association knowledge-base and a prototype
%recommendation system.
