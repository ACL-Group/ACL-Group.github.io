\section{The Association Knowledge Base}
Our input dataset consists of 8 million of synsets. 
We first compute the association relationship between any two synsets and 
then generate association knowledge-base.

%\subsection{Relatedness in Association Knowledge Base}
The association between two concepts or synsets in association knowledge-base is computed from two kinds of co-occurrence. The first is sentence-level co-occurrence, which means two terms co-occur in a sentence in any of our documents. The second is known as title-body co-occurrence, which means one term appears 
in another term's description page. Given two synsets A and B,
the probability of $P(B|A)$ is defined as follows:
\begin{equation}
P(B|A)=\frac{CoOccur_{s\textrm{-}level}(A,B)+CoOccur_A(B)}{\sum_{i=1}^nCoOccur_{s\textrm{-}level}(A,S_i)+length(A)}
\end{equation}
Here $CoOccur_A(B)$ represents the frequency of B appearing on A's page, and $length(A)$ means the total length of A's page, while $S_i$ can be any synset that co-occurs with A in a sentence.

%\subsection{Related Terms Clustering}
For each synset, we can get a list of related synsets ordered by the probability score described above, and those related synsets can usually be of great variety. While sometimes we need a special kind of synsets from the results to achieve a special goal, so we do clustering on those synsets. And since each term has several category labels on its encyclopedia page, we collect those category information, cluster synsets that have common category labels together, and represent each cluster with those labels. If cluster $A$ has category labels $c_1$ and $c_2$, while cluster $B$ has label $c_3$, and synset $s$ has labels $c_1$,$c_2$,$c_3$ and$c_4$, then we compute a relatedness score between $s$ and $A$,$B$ as $R(s,A)=P(c_1|s)+P(c_2|s)$, and $R(s,B)=P(c_3|s)$, and assign $s$ to the cluster with a higher relatedness score with $s$. After clustering, we can get each cluster which is also an ordered synsets list by its category labels.

%\subsection{Person Knowledge Base}
%The difference between general knowledge-base and person knowledge-base is that the terms in the person knowledge-base
%are Chinese names. We first use information in article to identify Chinese names. Then the identified Chinese name are
%used as vocabulary to form knowledge base. The algorithm for computing the score is same as the algorithm for computing
%general knowledge base.
%
%The purpose of the person knowledge base is to felicitate queries which need the relatedness between two names. The names could be ancient Chinese name, translated name, names in novel. Other than general knowledge base, the person
%knowledge base have more accurate relatedness between two names.
