\section{Problem Definition}
\label{sec:problem}

Our work aims at identifying the response turns to a question turn in a multi-turn,
two-party dialogue. Given a dialogue sequence with $T$ turns: 
$$[(R_1,L_1,U_1),(R_2,L_2,U_2),...,(R_T,L_T,U_T)]$$ 
where $R$ denotes the role, identifying which party utters
the turn. $L\in\{Q, NQ\}$, and $U$ is a sequence of words in natural language. 


Our job is to match each $(Q, U_i)$ with corresponding $(NQ, U_j)$, where:

\begin{equation}
\begin{aligned}
j>i&\quad 1\leq i,j\leq T\\
R_i&\not=R_j\\
\end{aligned}
\end{equation}

The {\em distance} of a Q-NQ pair ($U_i$, $U_j$) is $j-i$. We define
the {\em history} as the turns 
$\{U_{i+1},U_{i+2}...,U_{j-1}\}$ which are located between the Q and NQ. The intuition will be explained in Section \ref{sec:mutual attention}.

Recent work by He et al.~\cite{he2019learning} considers a slightly different 
QA alignment problem where one answer can be matched with multiple questions. 
However, in this paper, we assume that if a question is asked repeatedly, the answer should be matched to 
the nearest question and all earlier ones are disregarded. 
In our definition, a Q can match nothing (U9) or several NQs (U2). From the viewpoint of a NQ, it is 
either matched, or not matched, with a Q (such as U7). When a NQ is matched to a Q, it is 
considered as an {\em answer} (A). Otherwise, it's considered as others(O).

