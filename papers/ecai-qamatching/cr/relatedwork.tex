\section{Related Work}
\label{sec:relatedwork}

 
Detection of QA pairs from online discussions has been widely researched these years. Shrestha and Mckeown~\cite{shrestha2004detection} learned rules using Ripper for detecting QA pairs in email conversations. Ding et al.~\cite{ding2008using}, Kim et al.~\cite{kim2010tagging} and Catherine et al.~\cite{catherine2012does} applied the supervised learning method including conditional random field and support vector machine. Cong et al.~\cite{cong2008finding} proposed an unsupervised method combining graph knowledge to solve the task. Catherine et al.~\cite{catherine2013semi} proposed semi-supervised approaches which require little training data. He et al.~\cite{he2019learning} used the pointer network to find QA pairs in Chinese customer service. 
However, the tasks mentioned above are all different from ours. 
We identify QA pairs from two-party dialogues on online discussion forum, 
and focus especially on long-distance QA pairs. Besides, our 
dialogue is constrained between two roles who can both utter questions and  answers.


There exists several methods on other tasks which can be adapted to 
our QA matching problem. Feature-based method is popular for solving 
many NLP problems. In the work of Ding et al.~\cite{ding2008using}, 
Wang et al.~\cite{wang2010modeling} and 
Du et al.~\cite{du2017discovering}, 
they examined lexical and semantic features in two sentences 
for QA matching. However, the features such as common question words 
and roles have already been explicitly annotated in our data. 
Besides, other features such as special word occurrence or time stamp are 
unavailable here. According to the data, we considered the distance as 
the most important feature and implemented this feature-based method as one 
baseline. Recent researches using deep neural networks have increased a lot. 
He and Lin~\cite{he2016pairwise} and 
Liu et al.~\cite{liu2016modelling} used the sentence pair interaction 
approach which takes word alignment and interactions between the sentence 
pair into account. Attention mechanism was also added for performance 
improvement~\cite{rocktaschel2015reasoning,wang2016learning,chen2017enhanced}. 
We also use word alignment and interactions to calculate the QA similarity. 
Specially, we adopt attention mechanism to solve the LQA cases.

There are other kinds of alignment problems such as temporal sequences alignment. Video-text alignment is one of the temporal assignment or sequence alignment problems. Previous work automatically provided a time (frame) stamp for every sentence to align the two modalities such as \cite{bojanowski2015weakly} and \cite{dogan2018neural}. Bojanowski et al.~\cite{bojanowski2015weakly} extended prior work by including the alignment of actions with verbs and aligned text with complex videos. Dynamic time warping (DTW) is anothor algorithm for measuring similarity between two temporal sequences. It's also widely used in video-text alignment task~\cite{dogan2018neural}, 
speech recognition task~\cite{vintsyuk1968speech}.



