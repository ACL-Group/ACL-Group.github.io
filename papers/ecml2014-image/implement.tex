%\section{Implementation}
%\label{sec:implement}
%In the section, we show some implementation details of our system.
%We first tune the threshold of the tri-stage clustering algorithm,
%then, we discuss integrating visual features to our system.

\subsection{Threshold of Tri-stage Clustering}
The tri-stage clustering (TSC) algorithm is based on HAC\_CC
algorithm. Similar to traditional HAC algorithm, HAC\_CC has
a threshold to control the granularity of the clustering
result. We tune different threshold $\tau_t$ of HAC\_CC on a training
data collected from top 100 images of 10 different queries.
Cluster labels are assigned to each image by human judges.
\figref{fig:thtri} shows the clustering result on
different thresholds of HAC\_CC.
We prefer to choose a threshold which can ensure high purity, F1 and NMI
at the same time. NMI reaches a peak value at $\tau_t=0.15$. 
At this threshold, the purity is significantly higher than when
$\tau_t=0.1$ and F1 score is relatively high, too.
Consequently, in this system, we set $\tau_t$ to be 0.15.

\begin{figure}[th]
\centerline{\psfig{figure=tht.eps,width=0.5\columnwidth}}
	\caption{Clustering Result on Different $\tau_t$}
	\label{fig:thtri}
\end{figure}

%\subsection{Combining Visual Features}
%Because visual features are complementary and may not necessary
%represent the true semantics of the image, in our framework,
%clustering by visual features is biased toward high purity,
%%to make sure further clustering based on visual features is reliable.
%We experiment on different visual clustering threshold $\tau_v$ on
%the same data set as $\tau_t$.
%As shown in \figref{fig:thv}, using $\tau_v$ = 0.6, we can obtain
%an acceptable result of purity higher than 0.9.
%\begin{figure}[th]
%	\caption{Clustering result on different $\tau_v$}
%	\label{fig:thv}
%\end{figure}


