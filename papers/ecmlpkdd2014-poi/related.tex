\section{Related work}
\label{RelatedWork}
Ye et al's work is most related to our work. The first part of their features comes from explicit pattern, including total number of check-ins, total number of unique visitors, maximum number of check-ins by a single visitor, distribution of check-in time in a week and distribution of check-in time in 24-hour scale. We found these features very useful, too, and the reasons were clearly described in \cite{yemao}. However, the relatedness score, which is the second part of their features, is too noisy for our data set from Foursquare. The relatedness score was conducted from two undirected bipartite graph applying Random Walk: the User-Place graph representing user's check-in record, and the Time-Place graph indicating the time of check-ins at POIs. We believe the main reason comes from dataset difference. Ye et al's dataset comes from whrrl, they filter out users who have less than 40 check-ins, and further filter out users whose check-in places' entropy larger than 0.5. However, our dataset from Foursquare don't have so many frequent users, we didn't do such filtering thus there's no clear relatedness between POIs visited by the same user; and the number of check-ins for one POI may not sufficient to indicate the places' normal pattern, thus the relatedness indicated by time also not convincing. According to our experiment on our dataset, adding the relatedness score would not gain better result, on the contrary the result get worse. Therefore, we only compare with previous work by Mao Y. et al by comparing to explicit pattern. In fact, we add spatial features to the explicit pattern to show improvement, so we call explicit pattern as BASE in our experiment.

Some work use spatial features in their work to improve performance in recommendation task (\cite{LocationZheng}), Zheng et al. introduce categories distribution in a settled size rectangle to measure the functionality of certain area. To emphasize the importance of uncommon kinds of POIs, for example schools, among ubiquitous kinds of POIs, such as restaurants. In their recommendation task, they showed a smaller region size would result in better performance, however, in our work, we did a step further and found out that we should set different size for different categories. In \cite{LocationPlacer}, Krumm et al. aim at solving a similar task with us, labeling the POI a category, however their data is unobtainable which is diary data from government studies. The data in fact records people's daily trajectory, thus exposes large of privacy with long time monitoring. In their work they propose to use category distribution within different diameters and category distance as features to predict a tag. However, according to our experiments, using all the diameters as features would introduce redundancy and choosing one appropriate distance would be a wiser choice. And we tried on different data processing on the category distance feature, and finding out that a log operation or twice log operation will further benefit the prediction. Overall, both of the above-mentioned work utilize the spatial features in a casual way, which limit the effectiveness of spatial features. In our work, we focusing on analyzing spatial features respectively related to different categories, which enable much more delicate use on spatial features.

Along with the popularity of smart phone with GPS sensors, many interesting works (\cite{MDCconditional,phoneImageAudio,topicmodelMDC1,MDCdescriptive}) focusing on utilizing smart phone data emerges. Semantic annotation of place, dedicated Task 1 in Nokia Mobile Data Challenge, attracts many attention(\cite{MDCconditional,topicmodelMDC1,MDCdescriptive}), in which the data provides both the phone's sensors' data (e.g. GPS sensor, Wifi sensor, bluetooth sensor) and the specific phone status data (e.g. charging status, calling or messaging, phone-silent setting). And \cite{phoneImageAudio} explores other interesting features from smart phone including images and audio clips. Again these data needs fully monitoring, thus is hard to obtain and cause privacy problems. Other than smart phone data source, in \cite{interestprofile}, Montoliu et al. utilize words users post on online social networks, for example micro-blogging, to build user preferences and interests, then generate descriptive tags for POIs according to groups of users' check-ins and interests. These tags are mainly for semantic understanding rather than classification.

There are also interesting works focusing directly on user behaviour itself rather than utilizing it on tagging or classification. \cite{userbehavior} fully analyzed users' check-in behavior based on frequent users on Foursquare, and getting interesting conclusion from the perspective of user as well as from POIs. However, POIs covered by frequent users is only a small part comparing to all, thus delicate features based on these conclusions lack discrimination for most of the POIs, not to mention the POIs which lost even general statistic characteristics because of small quantity of check-ins.

