\section{BaseLine Algorithm}
The baseline algorithm to finish the splitting task is content-based algorithm. Given the program viewing sequence, content-based algorithm
generates similarity matrix whose entry is the similarity between the two programs considered. The algorithm then uses an unsupervised
clustering algorithm to cluster the programs. Here, we use hierarchical agglomerative clustering algorithm as the underlying algorithm.
After clustering, we could split the sequence into different users. However, we still don't know how many users behind this sequence.

Usually, there are less than 5 persons in a family. So we guess there are $k(1 \leq k \leq 5)$ members in front of TV and force HAC\cite{DBLP:conf/icpr/Gil-GarciaBP06}
algorithm to generate $k$ clusters. We compare the confidence of the clustering result with regard to different $k$ value. Then, we select the most
confident clustering result as splitting result and the corresponding $k$ as the number of members in front of TV.
Pseudocode for content-based algorithm is depicted in Algorithm \ref{code:1}

\begin{algorithm}[htb]
\caption{Content-based Algorithm}
\label{code:1}
\begin{algorithmic}[1]
\STATE $sim[][] = double[progSize][progSize]$
\STATE $userSize = -Infinity$
\STATE $maxPurity = -Infinity$
\STATE $listOfClusters = NULL$
\FORALL{$i \in candidateClustersSize$}
\FORALL{$prog1 \in progList$}
\FORALL{$prog2 \in progList$}
\STATE $sim[indexof(prog1)][indexof(prog2)] = contentSimilarity(prog1,prog2)$
\ENDFOR
\ENDFOR
\STATE $clusters = HAC(sim[][])$
\STATE $purity = calcPurity(clusters)$
\IF{$maxPurity < purity$}
\STATE $userSize = i$
\STATE $maxPurity = purity$
\STATE $listOfClusters = clusters$
\ENDIF
\ENDFOR
\end{algorithmic}
\end{algorithm}

However, there are two main shortcomings of content-based algorithm. First of all, for a single user, he may have multiple interests. For example,
a doctor may both like scientific program and news report. However, hierarchical agglomerative clustering algorithm tends to cluster scientific
program and news report to different clusters. Thus the algorithm loses a lot of accuracy. Secondly, two different users may share a common interest.
For example, a doctor and a lawyer may both like news report. But hierarchical agglomerative clustering algorithm usually clusters all news reports into
a single cluster. Thus, the algorithm couldn't distinguish the news report belonging to doctor or lawyer.
