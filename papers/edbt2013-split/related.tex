\section{Related Word}
Time-series analysis is an active area of research and relates to our watching sequence splitting task.

Discovering sequential patterns was first introduced in \cite{DBLP:conf/icde/AgrawalS95} and \cite{DBLP:conf/edbt/SrikantA96}. Especially in \cite{DBLP:conf/icde/AgrawalS95},
the author proposes three algorithms to mine sequential patterns in transaction database. The mined patterns is a maximal sequence with user-specified minimum support.

Jiong Yang et al.\cite{JYMINING} proposes a method to calculate asynchronous periodic pattern that may be present only within a subsequence and whose occurrences may be
shifted due to disturbance.

The surprising sequential pattern discovery is proposed in\cite{DBLP:conf/kdd/YangWY01}. In this paper, the author focus on mining surprising periodic patterns in a sequence of events. The concept of information gain is proposed to measure the overall degree of surprise of the pattern within a data sequence.

Bettini et al.\cite{Discover_Freq} proposed an algorithm to discover temporal patterns in time sequennces. The paper introduces event structures that have temporal constraints
with multiple granularities, defines the pattern-discover problem with these structures, and studies effective algorithms to solve it. The basic components of the algorithm
includes timed automata with granularities and a number of heuristics.

Xianping Ge et al.\cite{DBLP:conf/kdd/GeS00} proposed a novel and flexible approach based on segmental semi-Markov model to automatically detect specific patterns or shapes
in time-series data. The pattern of interest is modeled as a K-state segmental hidden Markov model where each state is responsible for the generation of a component of the
overall shape using a state-based regression function.

Jiawei Han et al.\cite{DBLP:conf/kdd/HanGY98} developed an efficiet method for mining multiple-level segment-wise periodicity in time-related database by exploring data cube,
bit-array, and the apriori mining techniques.

Valery Guranlnik et al.\cite{Guralnik:1999:EDT:312129.312190} proposed an event detection approach from time series data. The proposed methods uses  an iterative algorithm
that fits a model to a time segment, and uses a likelihood criterion to determine if the segment should be partition further. Meanwhile, the technique is independent of
regression and model selection methods. Experimental results show that the proposed method is more robust than using visual inspection.

An efficient incremental algorithm for identifying distinctive subsequences in multivariate, real-valued time series is described and evaluated in \cite{Oates:1999:IDS:312129.312268}.
The application of this algorithm includes financial time series gathered prior to significant declines or advances in the stock market, time series produced by the monitors in
an intensive care unit for patients who die, and traces of the behavior of unauthorized users of computer systems.

In \cite{754913}, the author proposed an suite of methods for mining partial periodicity in time series database. Partial periodicity, which associates periodic
behavior with only a subset of all the time points, is less restrictive than full periodicity and thus covers a broad class of applications.
