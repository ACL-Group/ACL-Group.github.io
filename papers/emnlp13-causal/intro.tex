\section{Introduction}
\label{sec:intro}

Automatic identification of semantic relations in text is an
important task in many natural language processing applications,
such as question answering (QA), information extraction (IE),
information retrieval (IR), etc.
Causal relation is one of the most common and important semantic relations
in these applications. For example, Q is a causal question for
QA system by Girju et al.\cite{girju2003automatic},
and the causation module can help the system
improve the answer from A-1 to A-2.
\begin{description}
\item[Q] What are the \emph{effects} of acid rain?
\item[A-1] Projects, reports, and information about the effects of acid rain.
\item[A-2] \emph{Acid rain} is known to \emph{contributes to} the \emph{corrosion of metals}.
\end{description}

In this work, we aim to detect and extract causal relations
between noun phrases from open domain text. Given a text corpus,
we first identify the candidate noun phrases by matching the text with
causal cue patterns\cite{GirjuM02}. The causal patterns around
the noun phrases and their statistical information provide good causal
association metric. Next, we extract four numeric features based on
cue patterns and associativity to help identify causal relations.
We then develop supervised and unsupervised methods to classify causal pairs
using these features. The main contributions of this paper are summarized
below:
\begin{itemize}
\item We propose four numeric features (Section \ref{sec:features})
to represent causality between noun
phrases. These features demonstrated superior effectiveness over
the features proposed in previous work
\cite{girju2003automatic,chang2006incremental}.
\item We developed both supervised and unsupervised learning algorithms (Section
\ref{sec:learning}) which utilize the four features to extract and classify
causal pairs. Our algorithms, especially the supervised algorithm,
achieve better accuracy on the entire Wikipedia corpus
by significant margin (see Section \ref{sec:eval})
than the previous state-of-the-art approaches
reported in Section \ref{sec:related}.
\end{itemize}

