\section{Previous Work}
\label{sec:related}
Previous work on causal relation extraction is relatively sparse,
especially on noun-phrase causality discovery. The existing approaches
use hand-coded and domain-specific patterns to extract causal knowledge.
Girju et al. first focused on casual knowledge discovery between nominals
\cite{girju2003automatic}. They semi-automatically extracted causal cue, but only extracted noun category features for the head noun. Chang et al. developed an unsupervised method and utility lexical pairs and cue contained in noun phrases as features to identify causality between them \cite{chang2006incremental}. Both of them ignored how the remaining causal text span between noun phrases effects the semantics. We proposed numeric features based on that, and get a better result.
Blanco et al. used different patterns to detect the causation in sentences
that contain clauses \cite{blanco2008causal}. And most recently, Do et al.
\cite{do2011minimally}
first introduced a form of association metric into causal relation extraction.
They used discourse connectives and similarity distribution to identify
event causality between predicate, not noun phrases,
and achieved an F1-score around 0.47.

%% If the paper is produced by a printer, make sure that the quality
%% of the output is dark enough to photocopy well.  It may be necessary
%% to have your laser printer adjusted for this purpose.  Papers that are too
%% faint to reproduce well may not be included.

%% {\bf Do not print page numbers on the manuscript.}  Write them lightly
%% on the back of each page in the upper left corner along with the
%% (first) author's name.
