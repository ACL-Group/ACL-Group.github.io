\section{Preliminary Results}
\label{sec:eval}
We present some preliminary results on Wikipedia articles from different domains.

\subsection{Data Set}

We use the full set of Wikipedia articles as our original corpus. After the preprocessing and filter step in Section 2, we obtained 5754 candidates. Then, we manually labeled 200 instances as ``causal'' (103 instances) or ``non-causal'' (97 instances). We take 150 out of 200 instances as the training set, and take the other 50 instances as the test set. The complete set of training and test
data used in this paper can be found at \url{http://adapt.seiee.sjtu.edu.cn/~jessie/causal/dataset.txt}.

\subsection{Causality Detection Accuracy}

We use manually labeled data set for evaluation. Table 1 shows the comparison of Sup+Logistic and Sup+SA approach against with previous work \cite{girju2003automatic} and \cite{chang2006incremental}.


\begin{table}
\small
\begin{center}
\begin{tabular}{|l|r|r|r|}
\hline \bf Model & \bf Precision & \bf Recall & \bf F-Score \\ \hline

\bf Sup + SA & 0.715 & 0.713 & 0.714\\
\bf Sup + Logistic & {\bf 0.797} & {\bf 0.76} & {\bf 0.778} \\
\bf unSup & 0.630 & 0.723 & 0.673 \\
\bf Girju & 0.691 & 0.667 & 0.665 \\
\bf Chang & 0.666 & 0.667 & 0.666 \\
\hline
\end{tabular}
\end{center}
\caption{\label{table:accuracy} The results of different models. }
\end{table}

We see the Sup+Logistic model gives the best performance. The Girju's approach only extracted noun category feature for head noun, and missed some semantic meanings for noun phrase. For example, it cut noun phrase pair \pair{inbreeding\_depression}{population\_bottleneck}
into \pair{depression}{bottleneck}. It loses important semantic information, because the cut pairs don't have the causal meaning any more. Chang's method fixed this problem by using lexical probability feature of noun phrases, but still worked weak on its cue probability feature. Our approaches consider the real noun phrase and extract more effective features for candidates which insufficiently express the causal association metrics as described in Section \ref{sec:features}.

Our Sup+SA model also can give a rank of extracted causal pairs and help improve the QA system better. The list of highly ranked causal pairs are shown in Table 2.

\begin{table}
\small
\begin{center}
\begin{tabular}{|l|}
\hline \bf  Causal Pairs \\ \hline

\pair{monoxide}{incomplete\_combustion} \\
\pair{nuclear\_holocaust}{world\_war\_iii} \\
\pair{epstein-barr\_virus\_infection}{cancer} \\
\pair{mud\_volcano}{earthquake\_zone} \\
\pair{inbreeding\_depression}{population\_bottleneck} \\
\pair{pesticide}{air\_pollution} \\
\pair{population}{environmental\_stress} \\
\pair{anxiety}{destructive\_behavior} \\
\pair{hyperbilirubinemia}{red\_blood\_cell\_destruction} \\
\pair{colic}{premature\_death} \\

\hline
\end{tabular}
\end{center}
\caption{\label{tab:examples} The list of highly ranked causal pairs. }
\end{table}
