\section{Problem Definition}
\label{sec:problem}
   The technical problem that is waiting to be conquered is defined as follow: given a corpus C with sentences $s_i$, where C=$\{s_1,s_2,...,s_m\}$ and a set of seeds T with seeds $t_j$, where T = $\{t_{1},t_{2},...,t_n\}$.  In general, each $t_j$ is a pair of arguments$(t_j^1,t_j^2)$, where each argument is represented as a form of string. For any pair $t_j$, find all sentences $s_i$ in C that: these sentences either contain words that exactly matches strings in $t_j$ or contain words that fussily (either lexically contain similar word tokens, semantically share the same meaning or syntactically share the same structure) matches strings in $t_j$. 
   
   \eve{semantically entails one another. give examples}
 
To clearly understand this problem, an example will be shown as follow. Consider we have a corpus only contains two sentences: ``John is going to spend time enjoying his holiday. Going to cinema is a good relaxing way for him.'' and a set of seeds \{(watch movies, perfect choice to relax),  (John, spend wonderful time enjoying vocation), (John, stay at home)\}. By using the first seed, we can apply fussy match to find the second sentence ``Going to cinema is a good relaxing way for him'' (semantically share same meaning). The second seed matches the first sentence, since they share the same pattern ``spend * time doing *" and they are semantically similar. The last seed doesn't match any sentence. 
 
 On the way dealing with the technical problem, we expect these challenges: a) Given a seed, how to find matching sentences under the premise of moderate accuracy ? b) Consider the size of corpus, how to reduce the cost of matching ? 