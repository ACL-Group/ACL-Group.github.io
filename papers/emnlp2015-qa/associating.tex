\subsection{Selectional Association}
% 18 sents.
% 1. startup
Given a relation group, the step of selectional association produces
a ranked list of relation schemas with association scores.
% 2. simultaneously process, showing the result
%The schema is represented as type pairs $\langle t1, t2 \rangle$, with its association score.
% 3. given example ?
Take ``play\ in'' as an example, the ideal schemas will contain the pair
$\langle actor,\ film \rangle$ and
$\langle athlete,\ sports\ league \rangle$.
% 4.
% 5. Freebase type index, putting entity into types
%
%We define $Tlist(ent)$ as all types for entity $ent$, which can be
%extracted by scanning \textit{type.object.type} relation in Freebase.
%% 6. Example of type list?
%For instance, ``New York City'' contains coarse-grained type \textit{location.location},
%fine-grained types \textit{location.administrative\_division},
%\textit{location.citytown} and other types like \textit{business.business\_location}.

% 6. type set to 1
Each linked tuple $\langle ent_1,\ rel,\ ent_2 \rangle$ supports the type pair $\langle t_1,\ t_2 \rangle$
where $t_1 \in Tlist(ent_1)$ and $t_2 \in Tlist(ent_2)$. We treat these pairs equally, since it's not trivial to
tell which type is more related to the argument given the relation tuple as context.
% 7. forming type pairs
% 8. example type pairs
% 9. count p(r, t)
Combining all tuples in one group, we define the support tuples of a type pair $tp$
in a group (use the representative pattern $r$ to stand for the group):
\begin{equation}
\begin{aligned}
sup(r,\ &\langle t1, t2 \rangle) = \\
        &\{ltup : t1 \in ent1,\ t2 \in ent2 \}
\end{aligned}
\end{equation}
% 10. formula of p(r, t)
\noindent
The joint probability of relation and type pairs is defined as the normalized number of supports:
\begin{equation}
p(r, tp) = |sup(r, tp)|\ /\ \sum\limits_{r', tp'} |sup(r', tp')|
\end{equation}

A simple intuition is to rank schemas by the number of supports.
Since one entity belongs to multiple types, relation schemas with general types
will be ranked higher. For instance, the schema $\langle person,\ location \rangle$
has more supports than $\langle deceased\ person,\ location \rangle$ for the relation ``\textit{X die in Y}''.
However, the former schema also occurs in many other relations, which lowers the preference
to the relation ``\textit{die in}'', while the latter one shows a more concrete representation.


% 1. Example: general case is not good
% 2. p(r, tp) as a baseline scoring function


% 11. main score function: Mutual Information between relation and type pairs
Therefore, the preference score of a schema cannot be measured by using joint probability alone.
We use mutual information to measure the preference between type pair and relation:
% MI(r, t) = p(r, t) / (p(r, *) * p(*, t))
\begin{equation}
MI(r, tp) = p(r, tp) \log \frac {p(r, tp)}{p(r) p(tp)}
\end{equation}


% 12. get score(r, t)
Finally, for each relation group, we collect all the possible type pairs with
their mutual information as the selectional association score.
We filter pairs with score equal or less than 0, and the remaining type pairs
are the possible schemas for the relation group.
Besides, $p(r,\ tp)$ is taken as the baseline scoring function, the comparing result is shown
in Section 4.

% 13.
% 14. 
