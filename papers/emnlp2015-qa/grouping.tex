\subsection{Relation Grouping}
% 8 sents.
% 1. group tuples together, give definition
%A relation group consists of linked tuples sharing similar relation patterns,
%along with a representative pattern.
% 2. same & syntactically similar rel. patterns will be in a group, no overlapping.
%Each linked tuple belongs to one unique group.
In the step of relation grouping, linked tuples with similar relation patterns form a group.
Each linked tuple belongs to one unique group.

% 1. what is similar ?

% 3. algorithm: syntactic rules to convert tense,
% mainly focus on 3 tense: will/should/must be, be -ing, participle

%We define syntactically equivalence between two relation patterns, as both of them can be converted
%into the same simple pattern by a list of transformations.
%Every relation pattern in one group is equivalent with each other.
The idea is to simplify relation patterns by syntactic transformations.
If two patterns share the same simplified pattern, we treat them as
being equivalent and put them into one group.
First, since adjectives, adverbs and modal verbs can hardly change the
type distribution of arguments in a relation,
we remove these words from a pattern.
Second, many relations from Open IE contain verbs, which come in
different tenses. We transform all tenses into present tense.
In addition, passive voice in a pattern, if any, is kept in
the transformed pattern.
% 4. Create a table, showing the rules to find them.
%The detail of syntactic rules is shown in Table 1.
%\KQ{refer to Liang et al., 2014 to build the rule table, containing continuous, participle, be-the-name-of
%and passive form}
% For example, a --> b
% check liang's 14 paper to learn the representation of tables.
% 5. use stanford parser to tokenize & postag.
% 6. representative relation: present tense
A simple example below shows a group of relations:
\begin{center}
    $\langle$X, \textit{resign from}, Y$\rangle$
    
    $\langle$X, \textit{had resigned from}, Y$\rangle$
    
    $\langle$X, \textit{finally resigned from}, Y$\rangle$
\end{center}

All linked tuples with the same simplified pattern form a group.
This pattern is selected as the representative pattern, like the pattern \textit{``resign from''}
in the above example.
