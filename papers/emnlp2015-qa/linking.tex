\subsection{Entity Linking}

% 11 sentences

% main: tfidf for words on freebase
% based on overlap sim.
% pseudo code?
%


% 0. what are we going to do ?
% Given a relation tuple, we are going to find representative Freebase entities
% which stand for the arguments.
% 1. Formally definition
In the entity linking step, by matching arguments to entities in the knowledge base,
each relation tuple is transformed into linked tuples,
$ltup=\langle e_1,\ rel,\ e_2 \rangle$, with linking scores.

% 2. ??? Talk about Freebase
% 3. each mid in FB has one or more entity names.
% example, China and PRC.



%Each entity in Freebase has one or more aliases. The default one is the name of this entity.
%For example, the entity \textit{m.02\_286} has the name ``New York City'' and other aliases
%such as ``The Big Apple'' ``NYC'' and ``Empire City''.
% 4. leverage multiple namees to build an invert index.
We aim to support fuzzy matching between arguments and entity aliases,
so we take all the aliases into consideration, and build an inverted index from words to aliases.
% 5. stop word set is used, and use the idf score to weight words.
Different words in one alias cannot be treated equally. Intuitively, a word
is more important if it occurs in fewer aliases ($n$), and vice versa.
Based on the inverted index, we use inverted document frequency score to
approximately model the weight of a word $w$:
\begin{equation}
idf(w)=1\ /\log(|\{n : w \in n\}|)
\end{equation}

Besides, stop words are removed from aliases, treating their idf scores as 0.
% 6. matching rule: intersect >= N - 1, weighted overlap score >= threshold
In order to measure the probability of fuzzy matching from an argument ($a$) to an alias ($n$),
we introduce the weighted overlap score:
\begin{equation}
overlap(a, n) = \frac {\sum\limits_{w \in a \cap n} idf(w)} {\sum\limits_{w \in a \cup n} idf(w)}
\end{equation}

We merge all the aliases of an entity together to producing a similarity score 
of fuzzy matching between an entity and an argument:
\begin{equation}
\begin{aligned}
sim(e, a) = \max\limits_{n \in Alist(e)} overlap(a, n)
\end{aligned}
\end{equation}

% \KQ{TODO: Two Strategies: Just choose the best separately, or FB 2-hop connectable}

In order to control the quality of candidate entities,
for an argument having $m$ words (with stop words removed),
we only keep entities that have at least one alias
matching $m-1$ words in the argument, and have a similarity score larger than a threshold, $\tau$.
% we can tune the threshold
% we can use formula to show the weighted score, that is intersect / union, weighed.
% 7. multiple matching, select the one with best wScore.
% 8. tie breaker: count occurrence in freebase relations.
%if there still has a tie, the most popular entity is selected. The popularity of an entity is calculated
%by counting number of relations it has in Freebase.
With similarity score computed, we generate 10 best entity candidates 
respectively for both the subject and the object of $rel$.

Next, we model the joint similarity score ($F$) of the relation tuple
$\langle a_1,rel,a_2\rangle$ with each entity pair combination $\langle e_1,e_2\rangle$ in two ways.
One is a \textbf{naive method} which only considers the similarity
between arguments and corresponding entities:
\begin{equation} \label{eqn:naive}
\begin{aligned}
F(a_1, e_1, &a_2, e_2, rel) = \\
            &sim(e_1, a_1) \times sim(e_2, a_2).
\end{aligned}
\end{equation}

The other method takes predicate paths between $e_1$ and $e_2$
into consideration.
Let $\vec{w}$ be the word vector of $rel$,
and $\vec{p}$ be a path of predicates connecting $e_1$ and $e_2$ in at most
2 hops. Here we say two entities $e_1$ and $e_2$ are connected in 1 hop,
if there exists a predicate $p$, such that $p(e_1, e_2)$ (or $p(e_2, e_1)$) is in the knowledge base.

Similarly, $e_1$ and $e_2$ are connected in 2 hops,
if there exists two predicates $p_1, p_2$ and a transition entity $e'$,
such that $p_1(e_1, e')$ (or $p_1(e', e_1)$) and $p_2(e', e_2)$ (or $p_2(e_2, e')$) are in the knowledge base.
 
We hence define the relatedness between $\vec{p}$ and $\vec{w}$ 
in the form of a conditional probability according to the Naive Bayes model:
\begin{equation}
\begin{aligned}
    P(\vec{p}\, |\, \vec{w}) & \approx \prod\nolimits_p P(p\, |\, \vec{w})  \\
                        & \propto \prod\nolimits_p P(p) \prod\nolimits_w P(w\, |\, p),
\end{aligned}
\end{equation}
\noindent
and we follow the IBM alignment Model 1~\cite{yao2014information} to calculate the conditional probability
between predicates and relation words $P(\vec{p}\, |\, \vec{w})$.
Based on the information above, we define a richer joint similarity score, 
considering all valid
paths between $e_1$ and $e_2$:
\begin{equation} \label{eqn:full}
\begin{aligned}
F(a_1, e_1,a_2,&e_2,rel) = sim(e_1,\, a_1) \times\\
               &sim(e_2,\, a_2) \times
		\sum\nolimits_{\vec{p}} P(\vec{p}\,| \vec{w}).
\end{aligned}
\end{equation}

Due to the multiplications, the value of $P(\vec{p}\,| \vec{w})$ varies a lot 
among different entity pair candidates. The large deviation makes 
$P(\vec{p}\,| \vec{w})$ the most important term in \eqnref{eqn:full}, 
especially in the case when none of predicate paths
are similar enough to the relation words.
Therefore, we trust the factor of $P(\vec{p}\,| \vec{w})$ only when 
there exists a similar predicate path.
In practice, we use a threshold $\rho$ to control whether to 
use \eqnref{eqn:full} or \eqnref{eqn:naive}.
We call this an \textbf{ensemble method}.
For each case of entity linking, if there exists one candidate entity pair satisfying $P(\vec{p}\, |\, \vec{w}) > \rho$,
we use the ensemble method, otherwise we fall back to the naive method for the current case.

%\footnote{The popularity of an entity is calculated by counting number of relations it has in the taxonomy.}
%The other strategy (SIM) selects all the $\langle ent1,\ ent2 \rangle$ pairs simultaneously,
%where $ent1$ is reachable from $ent2$ in the taxonomy by 1-hop or 2-hop relation.
%Experimental results on these two strategies are shown in Section 4.


% 9. SUTime is used to map years and datetime.

% check other papers, learn how to introduce FB without too much words.
% 10. discard non-match to guarantee accuracy of linking.
% If one argument fails to link to any entity, the corresponding relation tuple is discarded.
% Ranking Method May Change?
%   use wScore threshold to filter entities
%   then sorting by interLen, then popularity ??? (maybe we can have a try afterwards)
%


%\begin{table}[htbp]
%	\centering
%	\caption{Syntactic Transform Rules}
%	\begin{tabular}{|l|l|}
%		%\toprule
%        \whline
%		Category & Pattern Template \\
%		%\midrule
%        \hline
%        % Continuous Tense & \{$adv_1$\} \textbf{be} \{$adv_2$\} verb:VBG \{text\}
%        %                  & $verb_{lem}$ \{phrase\} \\
%		% Participle Tense & \{$adv_1$\} \textbf{have} \{$adv_2$\} verb:VBN \{text\}
%        %                  & $verb_{lem}$ \{phrase\} \\
%        % Participle + Passive & \{$adv_1$\} \textbf{have} \{$adv_2$\} been \{text\}
%        %                      & is \{text\} \\
%        Continuous Tense & \textbf{be} verb:VBG \{phrase\} \\
%		Participle Tense & \textbf{have} verb:VBN \{text\} \\
%        Future Tense & \textbf{will}/\textbf{shall} verb:VB \{pharse\} \\
%                     & \textbf{be} going to verb:VB \{phrase\} \\
%		%\bottomrule
%        \whline
%	\end{tabular}%
%	\label{tab:synt rules}%
%\end{table}
