%
% File acl2015.tex
%
% Contact: car@ir.hit.edu.cn, gdzhou@suda.edu.cn
%%
%% Based on the style files for ACL-2014, which were, in turn,
%% Based on the style files for ACL-2013, which were, in turn,
%% Based on the style files for ACL-2012, which were, in turn,
%% based on the style files for ACL-2011, which were, in turn,
%% based on the style files for ACL-2010, which were, in turn,
%% based on the style files for ACL-IJCNLP-2009, which were, in turn,
%% based on the style files for EACL-2009 and IJCNLP-2008...

%% Based on the style files for EACL 2006 by
%%e.agirre@ehu.es or Sergi.Balari@uab.es
%% and that of ACL 08 by Joakim Nivre and Noah Smith

\documentclass[11pt]{article}
\usepackage{acl2015}
\usepackage{amsmath,amsfonts}
\usepackage{times}
\usepackage{url}
\usepackage{color}
\usepackage{latexsym}
\usepackage{epsfig}
\usepackage{graphicx}
%\usepackage[pdftex]{graphicx}
%\DeclareGraphicsExtensions{.pdf,.jpeg,.png}
%\usepackage{epstopdf}
\usepackage{booktabs}
\usepackage{diagbox}
\usepackage{array}
\usepackage{multicol}
\usepackage{threeparttable}

\newcommand{\figref}[1]{Figure \ref{#1}}
\newcommand{\tabref}[1]{Table \ref{#1}}
\newcommand{\eqnref}[1]{Eq. (\ref{#1})}

\newcolumntype{I}{!{\vrule width 1pt}}
\newlength\savedwidth
\newcommand\whline{\noalign{\global\savedwidth\arrayrulewidth
                            \global\arrayrulewidth 1pt}%
                   \hline
                   \noalign{\global\arrayrulewidth\savedwidth}}
\newlength{\Oldarrayrulewidth}
\newcommand{\Cline}[2]{%
  \noalign{\global\setlength{\Oldarrayrulewidth}{\arrayrulewidth}}%
  \noalign{\global\setlength{\arrayrulewidth}{#1}}\cline{#2}%
  \noalign{\global\setlength{\arrayrulewidth}{\Oldarrayrulewidth}}}

\newcommand{\KZ}[1]{\textcolor{blue}{Kenny: #1}}
\newcommand{\KQ}[1]{\textcolor{red}{Kangqi: #1}}
\newcommand{\XS}[1]{\textcolor{red}{Xusheng: #1}}

%\setlength\titlebox{5cm}

% You can expand the titlebox if you need extra space
% to show all the authors. Please do not make the titlebox
% smaller than 5cm (the original size); we will check this
% in the camera-ready version and ask you to change it back.


\title{Inferring Binary Relation Schemas for Open Information Extraction}

%\author{Kangqi Luo \and Xusheng Luo \and Yang Zhan \and Kenny Q. Zhu  \\
%  Shanghai Jiao Tong University \\
%  {\tt \{luokangqi, kzhu\}@sjtu.edu.cn} \\
%  }

\author{Kangqi Luo$^1$ \and Xusheng Luo$^2$ \and Kenny Q. Zhu$^3$\\
  Department of Computer Science \& Engineering \\
  Shanghai Jiao Tong University, Shanghai, China \\
  {\small \tt $^1$luokangqi@sjtu.edu.cn $^2$freefish\_6174@sjtu.edu.cn $^3$kzhu@cs.sjtu.edu.cn} \\}


\date{}

\begin{document}
\maketitle
\begin{abstract}

% model the selectional preference of ReVerb relations to FB taxonomy.
% model both left and right side, and combination.
% one relation can have several type pairs
% we need to make use of FB type hierarchy
% we need to talk about relation combining
This paper presents a framework to model the semantic representation
of binary relations produced by open information extraction systems.
% to Freebase type taxonomy.
% By inferring preferred types simultaneously on two arguments for one
% binary relation, RvSp shows the different interpretations of one binary
% relation, and rank these interpretations by preference score.
For each binary relation, we infer a set of preferred types on the two
arguments simultaneously,
and generate a ranked list of type pairs which we call schemas.
All inferred types are drawn from the Freebase type taxonomy,
which are human readable. Our system collects 171,168 binary
relations from ReVerb, and is able to produce
top-ranking relation schemas with a mean reciprocal rank of 0.337.

% possible scenario

\end{abstract}

\section{Introduction}

Protein$-$protein interactions (PPIs) are of central importance for the majority of biological functions, such as signal transduction, metabolic pathways, molecular dynamics, and protein networks\cite{Hoffmann.Krallinger.ea:2005}, for they serve as the most fundamental building blocks of the entire interacademic systems of any organisms. Collecting data on pairwise interaction relationships is essential for multiple purpose, including identification of modules with certain functionality\cite{Spirin.Mirny.03}, mapping diseases to dominated genes\cite{Ideker.Sharan.08}, and after all, understanding wholistic metabolic/genetic networks from a system biology perspective.

A lot of databases have been built to store protein and genetic interactions from major model organism species and are available in various standardized formats, such as MINT\cite{Zanzoni.Montecchi-Palazzi.ea:2002}, BIND\cite{Bader.ea:2003}, BIOGRID\cite{DBLP:journals/nar/StarkBRBBT06}, etc. Among those mainstream databases, the data largely rely on voluntary reports by scientists or researchers, besides, comprehensive curation efforts become indispensable for the sake of accuracy. However, the amount of biology-related literatures with respect to protein interactions grows explosively and thus make it either impossible or impractical to manually detect PPI information anymore.

Considering huge amount of PPI information with great wealth hidden in published papers, in recent years, numerous mining techniques have been proposed that aim to extract PPI information automatically from free text, especially machine learning, information retrieval, and natural language processing\cite{DBLP:journals/bib/WinnenburgWPDS08}.These approaches can be roughly categorized into three classes: co$-$occurrence, rule$-$based, and machine learning. 

Co$-$occurrence is the approach with most simplicity and naivete. Just as its name implies, this method intends to find out pairs of proteins that co-occur in the same context. The scope of "same context" ranges from phrase, sentence, paragraph to whole abstract, even document. The underlying assumption is that whenever two proteins are mentioned together by authors, chances are high that there is some kind of relationship between them. However, however, in-context closeness even semantic relation does not necessarily represent actual biological interaction. As a consequence, a large fraction of candidate pairs are mismatched inevitably, causing a high recall but low precision.

The second approach is rule-based extraction, in other words, pattern matching. There are many types of rules, most of them concern natural language processing (NLP). One way is to specify hand-crafted regular expressions before hand, which mostly lean on language usage preference. Besides, by using full or partial (shallow) parsing strategies, more information would be acquired, such as part-of-speech taggers, local dependencies between syntactic components, context-free grammar\cite{DBLP:journals/bioinformatics/TemkinG03}, and full sentence structure. Compared to co$-$occurrence, rule-based approach enjoy better precision but much lower recall. In addition, since the rules are usually derived from training data, that is to say, the improper choice of training data would be significantly lethal, therefore quality of extraction is invariably instable and may not applicable to other data.

The third and most commonly used approach use machine learning techniques, in this case, the task to extract protein$-$protein interactions turns out to be a binary classification problem. Each protein pairs are represented along with a set of features, which is associated with their context, then a well$-$defined classifier gives the answer whether the candidate protein pairs is classified to be qualified PPI. (TO BE FURTHER FILLED!!!)

In this paper, we introduce a general bootstrapping framework for Protein$-$protein interaction extraction from natural text.Our method differs from most of the previous works in three aspects:

(1)The extraction process is driven by only tiny fraction of training data, which are regarded as seed data. In each round, it would derive reliable patterns automatically from seed data, then extract more positive PPI pairs consequently, what's more, the seed data would be augmented by the newly extracted results with high confidence.

(2)multiple graph kernel. 

(3)various evaluation.





\section{Problem Definition}
\label{sec:problem}

In this section we formally define the problem of short title extraction.
A char is a single Chinese or English character.
A segmented word (or term) $x$ is a sequence of several chars such as 
``Nike'' or ``牛仔裤''(jean).
A product title, denoted as $X$, is a sequence of words $\{x_1, x_2, ..., x_n\}$.
Let $Y$ be a sequence of labels $\{y_1, y_2, ..., y_n\}$ over $X$, where $y_i \in \{0, 1\}$.
The corresponding short title is a subsequence of $X$, denoted as $S = \{x_i\}$, 
where $y_i = 1$ and $|S| \le n$.

%we are interesting in obtaining a short title which can represent the most important information about the product.

We regard short title extraction task as a sequence classification problem.
Each word is sequentially visited in the original product title order
and a binary decision is made.
We do this by scoring each word $x_i$ within $X$ and predicting a label $y_i \in \{0, 1\}$, 
indicating whether the word should or should not be included in the short title $S$.
As we apply supervised training, the objective is to maximize the likelihood of all word labels
$Y=\{y_1,y_2,...,y_n\}$, given the input product title $X$ and model parameters $\theta$:
\begin{equation}
\label{eqn:problem}
\log{p(Y|X,\theta)}=\sum_{i=1}^{n}{\log{p(y_i|X,\theta)}}.
\end{equation}

%Our problem is different from Sequece Labelling problem, as ...

%In a more restrictive scenario, the number of words $m$ in the short title is strictly limited, where $m$ is some fixed number and $m \le \sum_{i=1}^{n} len(x_i)$. $len(x_i)$ is the number of words (chars) in term $x_i$.



\section{Framework}
%\BF{workflow figure to show the framework}
%\begin{multicols}{2}
%\begin{figure*}
%\centering
%\includegraphics[width=2\columnwidth]{sysoverviewgrapheps.eps}
%\caption{Framework wrokflow} \label{fig:workflow}
%\end{figure*}
%\end{multicols}
%\KZ{In the framework, say ``Output Parse'' instead of ``Output File.''}
The general architecture of the our parser is shown in \figref{fig:workflow}
and is divided into training phase and parsing phase.
%We take training treebank as input, which carries the
%essential information (we only use FORM and POSTAG) and
%gold dependency parses.

\begin{figure}[th]
\centering
\epsfig{file=sysoverviewgrapheps.eps, width=\columnwidth}
\caption{Sequence Based Parser Framework}
\label{fig:workflow}
\end{figure}

{\bf Training:} The preprocessing step generates oracle sequences
from the gold standard parse trees. Only the word forms and the POS tags 
in these parse trees are used. Here, we assume that a child node is
easier to process than its parent node and it is supposed to be attached
before its parent. \footnote{By this rule, multiple gold sequences
can be generated from one dependency tree. In this paper, when a parent node
has multiple children, we generate the sequence by a left-to-right order.}
%\KZ{Which one do we use or do we use all of them?}
%\footnote{
%For example, a bottom-up, breadth-first traversal of the gold parse tree or oracle transition
%process order from Malt Parser are both gold sequences.}
%and further discussion is deferred to Section 4.
%\TJ{maybe they will ask which one is the best; needs some explanations here}
We then train respectively a graph-based head mapper (a.k.a. decoder)
from the gold sequences and the gold parses, and a sequence predictor
from the gold sequences.

{\bf Parsing:} Given an input sentence, the sequence predictor
outputs a feasible decoding sequence, which is a permutation of
the words in the input. For each word in this sequence,
the head mapper returns its best head word according to a scoring function
while employing a cycle detection mechanism.
The process continues until all words in the sentence have found their
heads.
%(except manually introduce ROOT node in dependency parsing).
%For a sentence with $N$ words, the final result consists of ($N+1$) nodes
%and constructed $N$ arcs.
The procedure guarantees to produce a tree structure eventually.
\cut{
We implemented a simple version of this framework,
%and released the source code as well as the evaluation data\footnote{\urlstyle{same}\url{https://github.com/littlebeanfang/BeanParser}}.
%To reproduce the experiments refered in this paper, all our data and related commands are offered in the compressed file.
%\BF{add the data download source}
%\KZ{Besides the open-source system, create an online demo using default model
%and allow users to type in
%a sentence to have it parsed.}
and built an online demo\footnote{\urlstyle{same}\url{http://202.120.38.146/BeanParser}} to show parses of eight languages with the model
trained in our experiment.}

In the current implementation, we generate the decoding sequence by
{\em stackproj} algorithm~\cite{nivre2009non} in
malt parser and scorer-based greedy head mapper.
%\KZ{Consider rephrase this sentence.
%What does graph-based head mapper have to do with sequence?}
%The training and testing data are both in
%CoNLL format~\footnote{http://ilk.uvt.nl/conll/}.



%\section{Demonstration}
% how to evaluate?

% random picking relation, labeling true or false?
% the count precision@x?

% check: 1. GY's paper
%        2. ritter's paper

% Check their page size
% write about our size.
% and compare the version of MI-Equal, MI-Uniform, TfIdf-Equal, TfIdf-Uniform
%%In the demonstration part, we first introduce the experimental setup.
%%Secondly we evaluate the accuracy of relation type inferring.
%%Then we present our web interface of RvSp system, and finally
%%we provide some example relations with the inferred argument types.

\section{Evaluation}
% Add Freebase dump citation
%\KZ{First, say a bit about the ReVerb dataset, and the specifics of
%Freebase. Then say something about our implementation details in the
%3 steps, such as parser we used, etc. What about the numbers in the table?}


Freebase \cite{bollacker2008freebase} is a collaboratively generated knowledge base,
which contains more than 40 million entities, and more than 1,700 real types
\footnote{Freebase types are identified by type id, for example, $sports.pro\_athlete$ stands for ``professional athlete''.}.
In our experiment, We use the 16 Feb. 2014 dump of Freebase as the knowledge
base.

ReVerb \cite{fader2011identifying} is an Open IE system
which aims to extract verb based relation instances from web corpora.
The release ReVerb dataset contains more than 14 millions of relation tuples with high quality.
%Each entity belongs to at least one type.
%When compared with other knowledge bases, Freebase has a much greater focus on named entities than {\tt WordNet}.
%Besides, the type hierarchy of {\tt Yago} is too fine-grained, which is not suitable for schema inferring.
%Considering aspects mentioned above, we adapt Freebase as our knowledge base in our work.
%The input ReVerb dataset is released by Lin et al.\shortcite{lin2012entity}, containing 3 millions of relation tuples with high quality.
We observed that in ReVerb, some argument is unlikely to be an entity in Freebase, for example:

$\langle Metro\ Manila,\ consists\ of,\ \textbf{12 cities}\rangle$,

\noindent
where the object argument is not an entity but a type. Since types are usually represented by lowercase common words,
we remove the tuple if one argument is lowercase, or if it is made up
completely of common words in WordNet.
In addition, because date/time such as ``Jan. 16th, 1981''
often occurs in the object argument while Freebase does not have any
such specific dates as entities,
we use SUTime \cite{chang2012sutime} to recognize dates as an virtual entity.
After cleaning, the system collects 3,234,208 tuples and
171,168 relation groups.

% Talk about entity linking.
%We make the following parameter settings by empirics:
The following parameters are tuned using a development set:
$\tau = 0.667$,
$\epsilon=0.6$, $\lambda = 5\%$ and $\rho = e^{-50}$.
For relation grouping, we use Stanford Parser \cite{klein2003accurate}
to perform POS tagging, lemmatizing and parsing on relations.
%1. data come from Rv 3M
%2. Freebase.
% set tau to be 0.667 as the empirical value
%2. lowercased are removed
%3. remaining xxx relations, and xxx tuples.
% Important part, how to evaluate ?
%All the data sets involved in the evaluation are available at
%\url{http://202.120.38.146/schema/}.


We first evaluate the results of entity linking.
We randomly pick 200 relation instances from ReVerb, and manually
labeled arguments with Freebase entities.
For both naive and ensemble strategy, we evaluate the precision, recall, F1 and MRR score on the labeled set.
An output entity pair is correct, if and only if both arguments
are correctly linked. Experimental results are listed in \tabref{tab:linking_result}.

%We assigned 3 human annotators to judge whether both arguments are linked to correct entities.
%We don't have a gold set for entity linking, but we assume that each unlinked relation tuple corresponds to a linked tuple.
%Therefore, we can approximate the recall of entity linking as:
%\begin{equation}
%recall\ =\ \frac {precision * \#Linked\ Tuples} {\#Relation\ Tuples}
%\end{equation}
%For each strategy, the total number of linked tuples, precision, recall and F1 are listed in \tabref{tab:linking_result}.

\begin{table}[ht]
\small
	\centering
	\caption{Entity Linking Result}
	\begin{tabular}{Ic|c|c|c|cI}
		%\toprule
        \whline
		Strategy & P & R & F1 & MRR \\
        \whline
        Naive    & 0.371 & 0.327 & 0.348 & 0.377 \\
        \hline
        Ensemble & 0.386 & 0.340 & 0.361 & 0.381 \\
        \whline
	\end{tabular}%
	\label{tab:linking_result}%
\end{table}

For the evaluation of relation schema, we first randomly pick
50 binary relations with support larger than 500 from the system.
For each relation, we selected top 100 type pairs with the largest
support, as what we evaluated.
We assigned 3 human annotators to label the fitness score of type
pair for the relation. The labeled score ranges from 0 to 3.
Then we merge these 3 label sets, forming 50 gold standard rankings.
When evaluating a relation schema list from our system,
we calculate the MRR score~\cite{liu2009learning} by the top schemas in the gold rankings.

For comparison, we use Pointwise Mutual Information~\cite{church1990word}
as our baseline model, which is
used in other selectional preference tasks \cite{resnik1996selectional}.
We define the association score between relation and type pair as:
\begin{equation}
PMI(r, tp) = p(r, tp) \log \frac {p(r, tp)}{p(r, *) p(*, tp)}
\end{equation}
Where $p(r, tp)$ is the joint probability of relation and type pair in the whole linked
tuple set, and $*$ stands for any relations or type pairs.


\tabref{tab:precision} shows the MRR scores by using both baseline model (PMI) and our approach.
As the result shows, our approach improves the MRR score by 10.1\%.
%proving that RvSp can find enough correct type pairs.

\begin{table}[ht]
\small
	\centering
	\caption{End-to-end Schema Inference Results}
	\begin{tabular}{Ic|c|c|c|cI}
		%\toprule
        \whline
		Approach & MRR Score \\
        \whline
        PMI Baseline & 0.306 \\
        \hline
        Our Approach & 0.337 \\
        \whline
	\end{tabular}%
	\label{tab:precision}%
\end{table}

%\begin{figure%}[htp]
%\centering \scalebox{0.6}{\includegraphics{eval.eps}}
%%\epsfig{file=figure1-cropped.eps, width=2\columnwidth}
%%\scalebox{0.35}
%\caption{Average precision at different ranks.}
%\label{fig:precision}
%\end{figure}

% We randomly sample K relations, use 3 annotators to annotate whether a type pair is true or not.
% count precision@px

%\subsection{Web Interface}
%In addition, we set up a website \footnote{http://202.120.38.146/rvsp} for users to query the schemas of a binary relation.
%Users can search for type pairs by providing the binary relation alone, or the relation with the type of either arg1 or arg2.
%The interface will output the ranked list of schemas satisfying the input constraint along with its support instances.
%Before querying, the interface will transform the relation pattern, using the method introduced in section 4.

%Due to argument types in RvSp is recognized by Feebase type id, which doesn't match its name exactly, we provide typing suggestion in the web interface, %making users easily enter Freebase types.
%Users can browse Freebase website \footnote{http://www.freebase.com} for detail information about type id.

%
%\\
%\\
%
%\begin{figure}[ht]
%\centering
%\epsfig{file=cropped-demo1.eps, width=0.6\columnwidth, angle=270}
%\caption{Query Interface}
%\label{fig:demo1}
%\end{figure}
%
%
%\figref{fig:demo1} shows the result page.
%User can click ``page up'' and ``page down'' to check more results.
%Besides, for each relation schema, user can click ``detail'' link too check all its support tuples.
%The schema details are shown in \figref{fig:demo2}.
%\\
%\\
%
%\begin{figure}[ht]
%\centering
%\epsfig{file=cropped-demo2.eps, width=0.6\columnwidth, angle=270}
%\caption{Schema Details}
%\label{fig:demo2}
%\end{figure}
%

Finally, \tabref{tab:sample_relation} shows some example binary relations,
and their schemas inferred by our system.  We can see that
with a well-defined type hierarchy, our system is able to extract both
coarse-grained and fine-grained type information from entities,
resulting in a informative type lists.

%
%\begin{table*}[htbp]
%	\centering
%	\caption{Sample Relation Schemas}
%	\begin{tabular}{Ic|l|lI}
%		%\toprule
%        \whline
%		Relation & Arg1 Type & Arg2 Type \\
%        \whline
%        & book.author & book.book \\
%        & book.author & book.written\_work \\
%        be the writer of & tv.tv\_writer & award.award\_nominated\_work \\
%        & people.person & book.book \\
%        & people.person & book.written\_work  \\
%        \hline
%        & fictional\_universe.fictional\_character & tv.tv\_actor  \\
%        & fictional\_universe.fictional\_character & film.actor  \\
%        be play by & fictional\_universe.fictional\_character & people.person  \\
%        & fictional\_universe.fictional\_character & influence.influence\_node  \\
%        & people.person & tv.tv\_actor  \\
%        \hline
%        & organization.organization\_founder & organization, organization \\
%        & people.person & organization, organization \\
%        found & people.deceased\_person & organization, organization \\
%        & organization.organization\_founder & business.business\_operation \\
%        & organization.organization\_founder & business.employer \\
%        \whline
%	\end{tabular}%
%	\label{tab:sample_relation}%
%\end{table*}
\begin{table}[ht]
\small
	\centering
	\caption{Sample Relation Schemas}
	\begin{tabular}{Ic|lI}
		%\toprule
        \whline
		Relation & Top Schemas \\
        \whline
        & $\langle location,\ location \rangle$ \\
        be find at & $\langle employer,\ location \rangle$ \\
        & $\langle organization,\ location \rangle$\\
        \hline
        & $\langle person,\ tv\ program \rangle$\\
        appear on & $\langle person,\ nominated\ work \rangle$\\
        & $\langle person,\ winning\ work \rangle$\\
        \hline
        & $\langle person,\ nominated\ work \rangle$\\
        be the writer of & $\langle person,\ film \rangle$\\
        & $\langle person,\ book\ subject \rangle$\\
        \whline
	\end{tabular}%
	\label{tab:sample_relation}%
\end{table}


%\section{Related Work}
This section surveys previous works on question generation and tree encoding
respectively.

Text question generation has attracted the attention 
after the work of ~\citeauthor{du2017learning}~\shortcite{du2017learning}, who uses deep seq2seq model 
to generate questions from a raw text paragraph. 
Before that, text question generation relied heavily on hand-craft 
question patterns~\cite{HeilmanS10,LabutovBV15,MostowC09} which is time and 
labor consuming. 

However, this pure seq2seq model is not focused and 
has no control over part in the paragraph to generate question. 
~\citeauthor{zhou2017neural}~\shortcite{zhou2017neural} proposed to encode 
key phrase information using binary indicators to generate 
key-aware questions and they assumes the answer to be key phrase. 
Considering key phrase (answer) is unavailable in reality, 
~\citeauthor{SubramanianWYT17}~\shortcite{SubramanianWYT17} applied 
a two-stage approach. First, key phrases are extracted by 
pointer network~\cite{ptrnet}. Second, 
key phrases are encoded in the same way as 
Zhou et al. With the intuition that questions could be asked in many ways, 
~\citeauthor{Yao2018vae}~\shortcite{Yao2018vae} used conditional-VAE to 
increase the diversity of questions. More recently, models with 
auxiliary feature information~\cite{HarrisonW18} helped improve 
the question quality. Structure question generation aims at 
converting structured data such as triples in knowledge graph to questions. 
~\citeauthor{SerbanGGACCB16}~\shortcite{SerbanGGACCB16} proposed a model to generate factoid questions from knowledge base triples.  None of the above work
considered using parse tree structures to aid question generation process,
which is the focus of this paper.

Sequential RNN model takes sentence as a sequence of words, 
ignoring the syntactic information. In order to utilize
such syntactic information with sequential information, 
~\citeauthor{tai2015improved}~\shortcite{tai2015improved} proposed Tree-LSTM to 
encode the binary parse tree recursively in a bottom-up fashion to 
classify sentiment. In text generation task, 
\citeauthor{eriguchi2016tree}~\shortcite{eriguchi2016tree} 
proposed a tree-to-sequence model with attention mechanism to do 
machine translation and 
~\citeauthor{liang2018automatic}~\shortcite{liang2018automatic} proposed a 
tree-to-sequence model which could handle arbitrary trees, 
to do code comment generation. Our work is inspired by these previous
attempts and we are first to adapt structure encoded neural models to
textual question generations.

\section{Conclusion}
We implement a novel sequence-based dependency parsing
framework which takes advantage of high order features 
in parsing history. 
%We can also adapt beam search to this framework so as to
%relax the strictly greedy nature. Vine pruning\cite{rush2012vine} could
%be incorporated to speed up the parsing.
More importantly, we discovered that the parsing accuracy is very sensitive to
the quality of parsing sequence. Future work can be focused on
developing better sequence predictors that outperform Malt action classifier.
Furthermore, we use two sets of features for sequence predictor and
head mapper right now. A unified set of features between these two components
are worth exploring.
%Besides, better sequence predicting method and unified feature
%representation of two components are worth exploring.
%
%Though we currently get a not bad result,
%the sequence predictor still needs more exploration.
%According to our experiment, slightly changes
%on the sequence can lead to a fatal decline on accuracy. Ensuring the match degree of training sequence and testing
%sequence demands a high quality of sequence predictor.
%
%Further, the features in our current implementation are not expanded and well tuned yet  and we are free to define high order features to make use of parsing history. Our framework is flexible to merge other technics to enhance the performance. Introducing beam could make up for our greedy decoder and improve our accuracy. Vine pruning\cite{rush2012vine} could speed up parsing process. Besides, better sequence predicting method and unified feature representation of two components are worth exploring.


\section*{Acknowledgement}
Kenny Q. Zhu is the contact author and was supported by
NSFC grant 61373031 and NSFC-NRF Joint Research Program No. 61411140247.

\bibliographystyle{acl}
\bibliography{qa}

\end{document}
