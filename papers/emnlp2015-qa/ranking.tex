\subsection{Schema Ranking}

Given a relation group, the step of schema ranking produces a ranked list of relation schemas with two constraints.
Take ``play\ in'' as an example, the ideal schemas will contain the pair
$\langle actor,\ film \rangle$ and
$\langle athlete,\ sports\ league \rangle$

Each linked tuple $\langle e_1,\ rel,\ e_2 \rangle$ supports the type pair $\langle t_1,\ t_2 \rangle$
where $(e_1, t_1),\; (e_2, t_2) \in IsA$ in the knowledge base. 
We treat these pairs equally, since it's not trivial to
tell which type is more related to the argument given the relation tuple as context.
Combining all tuples in one group, we define the support of a type pair $tp$
in a group (using the representative pattern $r$ to stand for the group):
\begin{equation}
\begin{aligned}
sup_{r}(&\langle t_1, t_2 \rangle) = \\
        &\{(e_1, t_1) \in IsA,\; (e_2, t_2) \in IsA \}
\end{aligned}
\end{equation}

A simple intuition is to rank schemas by the size of the support.
Since one entity belongs to multiple types, relation schemas with general types
will be ranked higher.
However, two different schemas may share the same support.
For instance, given the relation ``\textit{X die in Y}'', 
suppose Open IE extractions and entity linking step returns correct results,
the schema $\langle person,\ location \rangle$ and $\langle deceased\ person,\ location \rangle$ have identical supports.
The latter one shows a more concrete representation of the relation, because
\textit{deceased person} covers small entities than \textit{person} in the knowledge base.

Therefore, the schemas cannot be ranked by using the support alone.
Next, we aim to extract the subsumption relations between types in the knowledge base,
building the taxonomy of types.

We first define all entities in $t$ as
\begin{equation}
cover(t) = \{e\; |\; (e, t) \in IsA\}.
\end{equation}
Intuitively, type $t_1$ is subsumed in $t_2$, 
if all entities in $t_1$ also belong to $t_2$,
that is, $cover(t_1) \subseteq cover(t_2)$. This uses the idea of
\textbf{strict set inclusion}.
For example, we can learn that the type \textit{person} subsumes types such as
\textit{actor}, \textit{politician} and \textit{deceased person}.

However, strict set inclusion doesn't always hold in the knowledge base.
For example, entities in type \textit{award winner} are mostly \textit{person}, but there still has some
organizations in it. The strict method fails to find the subsumption relation between \textit{award winner}
and \textit{person}, while this subsumption actually holds with a large confidence. 
%
%The ranked list should be adjusted by the IsA relations between the types in
%knowledge base $K$.
%Each type $t$ in $K$ is subsumed by a set of entities (a.k.a. the
%support set). Next we will write $Supp$ to denote the set of entities
%it covers.
%The main difficulty in generating type subsumptions is that the
%support sets may be incomplete \KZ{Why incomplete and isn't type subsumption
%already included in the ISA relation between types?},
%\XS{We can consider ISA relation not explicit between types, thus we need the support set to
%determine the subsumptions between types.}
%thus making strict set inclusion computation impossible.
%\KZ{The rest of this section needs to be rewritten to be consistent
%with the problem definition section. You can't use $S$, which means a set
%of extracted triples, and you can't mention Freebase!}

To resolve this problem, we use a \textbf{relaxed set inclusion},
where the set $cover(t_1)$ can be a subset of another set $cover(t_2)$ to a certain degree.
We define the degree of the subsumption as the ratio between the number 
of entities in the two sets: 
\begin{equation}
deg(t_1 \subseteq t_2) = \frac{\left|cover(t_1) \cap cover(t_2)\right|} {\left|cover(t_1)\right|}.
\end{equation}
%$deg(Supp_1\subseteq Supp_2)=\left|Supp_1 \cap Supp_2\right| / \left|Supp_1\right|$.
If $deg(t_1 \subseteq t_2) > \epsilon$,
then $t_1$ is subsumed by $t_2$, and $\epsilon$ is a confidence parameter 
determined by weight tuning.
By scanning all types in the knowledge base, all subsumption relations 
with enough confidence are extracted, forming our type taxonomy.

With a type hierarchy computed by above relaxed set inclusion, 
we can define a schema 
$\langle t_1,\ t_2\rangle$ subsumes another schema
$\langle t_3,\ t_4\rangle$ if 
i) $t_1$ subsumes $t_3$ and $t_2$ subsumes $t_4$;
ii) $t_1$ subsumes $t_3$ and $t_2 = t_4$; or
iii) $t_2$ subsumes $t_4$ and $t_1 = t_3$.

If a schema (type pair) $tp_1$ subsumes another schema $tp_2$, 
and their supports ($|sup_{r}(tp)|$) are approximately equal, 
we give the more specific schema $tp_2$ a higher rank in the output list.
Here two supports are roughly equal if:
\begin{equation}
\frac{\left|sup_{r}(tp_1)\right|-\left|sup_{r}(tp_2)\right|} {max(\left|sup_{r}(tp_1)\right|,\ \left|sup_{r}(tp_2)\right|)} < \lambda
\end{equation}
\noindent
Where $\lambda$ is a threshold determined in the experiments.
