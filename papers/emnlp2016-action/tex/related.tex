\section{Related work}
\subsection{Argument conceptualization}
As mentioned earlier, in our definition the most important components for an action are the arguments of the predicate.
Argument conceptualization tries to abstract the subject and object arguments of a verb by categorizing them into set of noun concepts.
For example, for the verb ``accept'', the set of subject noun concepts might contain ``person'', ``community'' and
the set of object noun concepts might contain ``document'', ``payment''.

Earlier efforts of such tasks including Semantic role labeling (\cite{gildea2002automatic} and
\cite{palmer2005proposition}), which labels the semantic meaning of the arguments of the verb, for example
in the sentence ``He killed the enemy with a rifle.'' ``He'' is the agent, ``enemy'' is the patient and ``rifle'' is the instrument.
However the choices of such role labels are rather limited and could be further expanded through Knowledge graphs. Especially in our
application we hope the role to be more specific than it is in Semantic role labeling.

Selectional constraints studies what are appropriate arguments for a particular verb.
\cite{resnik1996selectional} proposed class-based selectional preference which decides if a class of terms is
prefered to a verb. For example ``water'' is more prefered than ``table'' for the verb ``drink''. The main problem with this
method is it does not consider overlap between classes which results in very similar classes.

\cite{gong2015representing} proposed a data-driven approach which models the conceptualization problem as
finding $k$-clique with maximum combined weights. They managed to build up an inventory of arguments concepts for
more than 1,700 unique verbs. Our work is a further development of their project by converting action concepts into
noun concepts.

\subsection{Short text understanding}
Our conceptualization is closely related to short text understanding in that the final result of our work is a minimal abstraction of the
whole sentence. FrameNet project \cite{baker1998berkeley} annotates sentences with semantic frames and using frame elements to represent
the semantic meaning of the sentences. The drawbacks of FrameNet include its requirements for manual labeling, and its distance from
real world human readable explanations.

Probase \cite{wu2012probase} is a probabilistic taxonomy for text understanding, which incorporates open-domain web knowledge and automatically
infers the probabilistic taxonomy.  Probase is rich in the quantity of entities and subsumption relations among them. Our work makes extensive
use of its Knowledge graph as the abundance of such open-domain knowledge is crucial in our concept-instance pair identification.

\subsection{Relational Phrase}
\cite{nakashole2012patty} proposed a collection which contains semantically typed relational patterns, called \emph{relational phrase} such as \emph{<singer> [prp] sings <song>} with POS tag [prp] and organised them into a sparse hierarchical structure. The HARPY\cite{grycner2014harpy} project extended PATTY and reduced the sparseness of its relational hierarchy by disambiguating and aligning relational phrases with WordNet, but it suffers from low precision. The RELLY \cite{grycner2015relly} project integrated statistical and semantic information from knowledge bases like YAGO \cite{suchanek2007yago}, WordNet, PATTY, and HARPY, and created a large hypernymy graph of relational phrases with high coverage and precision.

Our \emph{action class} is quite similar to relational phrase, but with a higher abstraction. These projects focus on building a structural graph among relational phrases while our project can be used to extend the graph by relating the phrases to nouns. 


\subsection{Semantic search}
Traditional search engines usually only consider the lexical level match of query keywords. But there is a gap between human understanding and
machine interpretation. For example if one searches ``Iraq war'' the search engine is likely to neglect those related pages that do not
contain the keywords, but maybe instead they contain something like ``U.S. troops occupied Baghdad''. Semantic search aims at understanding
the real intent of the user behind the query term.\cite{john2012semantic}
It involves use of ontology information and other real world knowledge to match query on a semantic level.

Some of the existing attempts includes translating the natural language input by the user into formal queries on ontology database.\cite{bonino2004ontology}
Such efforts have made substantial progress but also endure the drawback of domain-specific restrictions. The ontology provided is usually
limited to a certain domain.

Other researchers also tried to build classification methods to map the query phrase into semantic classes.
 \cite{demartini2013crowdq} proposed a crowdsourced method to generate query templates based on user log-mining.
 \cite{arguello2009sources} tries to predict the relevant subdomain of the given query by examining evidences such as
previous search log and corpora representatives.
