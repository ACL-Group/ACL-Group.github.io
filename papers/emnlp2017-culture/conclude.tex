\section{Conclusion}
In this paper, we conclude that the cultural properties and usages of a term (including named entities and slang terms) can be effectively represented by its similarities to socio-linguistic words. 
Building a bilingual socio-linguistic lexicon enables two incomparable 
monolingual semantic spaces to be comparable. 
Our proposed framework can be a valuable assistance to cross-cultural 
social studies by acting as a building block for 
computing such cross-cultural differences and similarities. 
%, such as detection of cross-cultural differences in named entities and extraction of a bilingual lexicon for Internet slangs.
