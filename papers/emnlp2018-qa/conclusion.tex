\section{Conclusion}

%In this work, we propose a semantic parsing based approach to handle complex KBQA task.
%aims at  handle complex questions on KBQA in SP + NN.
To the best of our knowledge, this is the first work to handle
complex KBQA task by explicitly encoding the complete semantics of a complex query graph
using neural networks.
We studied different methods to further improve the performance,
mainly leveraging dependency parse and the ensemble method for linking enrichment.
%encode represent question
%and predicate sequences, and leveraging dependency parse information
%and   ensemble  entity linking  enrichment method.
%to improve the 
%And we present a 
Our model becomes the state-of-the-art on ComplexQuestions dataset,
and produces competitive results on other simple question based datasets.
%compQ : 42.8 \% signifiantly higher than
%simple: SimpQ and WebQ outperforming lot of works
Possible future work includes supporting more complex semantics like implicit time constraints.

%complex questions in different categories,
%such as questions implicit time constraints and syntactically simple
%but semantically complex questions.
%understanding implicit time constraints within in a question,
%and 
%future work: understand vague constraints,
%implicit time constraints.
