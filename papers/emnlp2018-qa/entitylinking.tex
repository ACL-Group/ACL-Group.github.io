\subsection{Entity Linking}
\label{sec:entitylinking}

%1. S-MART only
%2. S-MART ensemble
%   how 
%   regression

Entity linking, the task of linking focus mentions in questions to entities in knowledge base, is one of the most crucial components in KBQA system. In our model, we have two versions of entity linker: one is S-MART~\cite{yang2015s} entity linking only, which is one of the most popular entity linkers in many previous KBQA models, such as ~\cite{yih2015semantic}~\cite{bao2016constraint}~\cite{yu2017improved}
; the other is our enhanced entity linker based on S-MART.

S-MART is a entity linker trained from Twitter data, which aims to identify and link name mentions to entities in a knowledge base. It has proved to have excellent performance in many knowledge acquisition tasks, including KBQA systems. However, it also has some shortcomings. Specifically, S-MART is not extendable and its lexicon is too small to cover Complex questions.

Therefore, to address this problem, we proposer our enhanced version entity linking method based on S-MART. First, we have dumped the Wikipedia data, preprocessed it to get its statistical information: ["link probability", "word jaccard index", "wiki popularity", "n-grams jaccard", "wiki popularity", "fb popularity"] and use n-grams method to generate raw linking results. Second, we selected the intersect set of S-MART linking results with our Wikipedia lexicon linking results and use it to train a 2 layer linear regression model, with which we can transform our results into the form as S-MART. What’s more, we add a new feature named “source” indicating each line comes from S-MART, Wikipedia or Bothinto the results, which can be used in Candidate generation network to assign weights. 
