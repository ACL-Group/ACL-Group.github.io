\documentclass{article}
\usepackage{amsmath}
\usepackage{algorithm}  
\usepackage{algorithmic}
\begin{document}
\section{an executable Example}
\label{sec:example}
A tiny executable example to illustrate causal reasoning with uncertainty is as follows:
Suppose we have a rule base with only rule (1), Zh-Probase and Zh-ConceptNet.
First, given a query sentence: ``Thailand suffered an earthquake attack".
Second, extract the quintuple event: ('',Thailand,suffer,earthquake,attack).
Third, search for the related rules and facts. We can get the only rule and related facts: isA(thailand,country), isA(earthquake,disaster), atLocation(rice,thailand), isA(rice,product).
Fourth, convert them into standard Prolog code.\\

\begin{algorithm}[htb]
	\begin{algorithmic}[1]
		\STATE isA(thailand,country).
		\STATE isA(earthquake,disaster).
		\STATE isA(rice,product).
		\STATE atLocation(rice,thailand).
		\STATE isconfident(X):-X $>$ 0.3. (here, the threshold is 0.3)
		\STATE rise(\_,\_,\_,\_,[],\_,[]).
		\STATE suffer(-, X, Y, attack, Es, C, Cs):-rise(Z, price, -, -, EIs, CI, CIs), isA(X, country), isA(Y, disaster), isA(Z, product), atLocation(Z, X), is(CI, 0.84*C), isconfident(CI), append(EIs, [rise(Z, price, -, -)], Es), append(CIs, [0.84], Cs).
	\end{algorithmic}
\end{algorithm}	

Line 5,6 and `append' predicate in Line 7 are auxiliary code that helps to run in Prolog.
Fifth, query ``suffer(-,thailand,earthquake,attack,Es,1,Cs)." in SWI-Prolog.
Get predicted events with confidences:\\
Es=[rise(rice,price,-,-)],\\
Cs=[0.84];

Thus, we can predict the price of rice will increase with confidence of 0.84 if Thailand suffered an earthquake attack.
Furthermore, if we search for these two related facts: isA(rubber, product),atLocation(rubber, thailand), we can also predict that the price of rubber will rise.

\end{document}