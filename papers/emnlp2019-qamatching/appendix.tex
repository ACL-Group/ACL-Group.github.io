%
% File emnlp2019.tex
%
%% Based on the style files for ACL 2019, which were
%% Based on the style files for EMNLP 2018, which were
%% Based on the style files for ACL 2018, which were
%% Based on the style files for ACL-2015, with some improvements
%%  taken from the NAACL-2016 style
%% Based on the style files for ACL-2014, which were, in turn,
%% based on ACL-2013, ACL-2012, ACL-2011, ACL-2010, ACL-IJCNLP-2009,
%% EACL-2009, IJCNLP-2008...
%% Based on the style files for EACL 2006 by 
%%e.agirre@ehu.es or Sergi.Balari@uab.es
%% and that of ACL 08 by Joakim Nivre and Noah Smith

\documentclass[11pt,a4paper]{article}
\usepackage[hyperref]{emnlp-ijcnlp-2019}
\usepackage{times}
\usepackage{latexsym}

\usepackage{url}


\usepackage{array}
% Use the postscript times font!
\usepackage{times}
\usepackage{soul}
\usepackage{url}
\usepackage{color}
%\usepackage[hidelinks]{hyperref}
\usepackage[utf8]{inputenc}
%\usepackage[small]{caption}
\usepackage{graphicx}
\usepackage{amsmath}

\newcommand{\figref}[1]{Figure \ref{#1}}
\newcommand{\eqnref}[1]{Eq. \ref{#1}}
\newcommand{\tabref}[1]{Table \ref{#1}}
\newcommand{\secref}[1]{Section \ref{#1}}
\newcommand{\algoref}[1]{Algorithm \ref{#1}}
\renewcommand\appendix{\setcounter{secnumdepth}{-2}}


\newcommand{\KZ}[1]{\textcolor{red}{Kenny: #1}}
\newcommand{\mx}[1]{\textcolor{green}{Mengxue: #1}}
% the following package is optional:
%\usepackage{latexsym} 
\usepackage{diagbox}
\usepackage{multirow}
\usepackage{amsthm}
\usepackage{mathtools}
\usepackage{hhline}
\usepackage{booktabs}
%\usepackage{ctex}
\usepackage{makecell}
\usepackage{CJKutf8}



%\aclfinalcopy % Uncomment this line for the final submission

%\setlength\titlebox{5cm}
% You can expand the titlebox if you need extra space
% to show all the authors. Please do not make the titlebox
% smaller than 5cm (the original size); we will check this
% in the camera-ready version and ask you to change it back.

\newcommand\BibTeX{B{\sc ib}\TeX}
\newcommand\confname{EMNLP-IJCNLP 2019}
\newcommand\conforg{SIGDAT}

\title{Matching Questions and Answers in Dialogues from Online Forums}

\author{First Author \\
	Affiliation / Address line 1 \\
	Affiliation / Address line 2 \\
	Affiliation / Address line 3 \\
	{\tt email@domain} \\\And
	Second Author \\
	Affiliation / Address line 1 \\
	Affiliation / Address line 2 \\
	Affiliation / Address line 3 \\
	{\tt email@domain} \\}

\date{}

\begin{document}
	\maketitle
\begin{CJK}{UTF8}{gbsn}
\section{Supplemental Material}
\label{sec:appendix}

In this supplementary material, we add additional details supporting the experimental results and case studies.

\subsection{Experimental Results for Ablation tests}

Table \ref{tab:app1} shows the overall performance of four ablations and our full model (HDM). According to the T-test results among three runs, the precision between each two models are not significant different while the recall varies a lot. The full model gets the highest recall and F1-score among all models.

\begin{table}[h]
	\small
	\centering
	\begin{tabular}{p{1cm}<{\centering}p{1cm}<{\centering}ccc}
		\toprule[1.5pt]
		Models &P&R& F1\\
		\midrule[1pt]
		QH&77.54&71.80&74.56\\
		AH&77.50&70.52&73.84\\
		\hline
		NON&75.72&75.91&75.81\\
		ID&76.22&74.76&75.46\\
		\hline
		HDM&76.44&78.44&\textbf{77.43}\\
		\bottomrule[1.5pt]
	\end{tabular}
	\caption{The end-to-end performance of ablation tests.}
	\label{tab:app1}
\end{table}

Results in Table \ref{tab:app2} reveal that HDM model outperforms the ablations on matching LQAs. QH and AH perform well on SQAs while fail on LQAs. NON and ID sacrifice the accuracy on SQAs a little and make improvements on LQAs. It seems that when the history information is not quite concise and useful, the distance will play an important role on the final results. Besides, improving the accuracy on LQAs always shows a decline of the accuracy on SQAs, which shows the importance of the trade-off between the distance and history features. Although our full model provides a good result on overall performance, how to take better advantage of these two factors is still a challenge.


\begin{table}[h]
	\small
	\centering
	\begin{tabular}{p{1.5cm}<{\centering}ccccc}
		\toprule[1.3pt]
		Models &1&2&3&4&$\geq5$\\
		\midrule[1pt]
		QH&96.38&89.98&21.53&21.53&0.79\\
		AH&95.92&87.61&15.81&10.78&4.76\\
		\hline
		NON&95.04&78.66&57.18&37.50&20.04\\
		ID&94.02&81.87&54.50&28.43&11.70\\
		\hline
		HDM &95.99&83.16&59.37&40.44&24.80\\
		\bottomrule[1.3pt]
	\end{tabular}
	\caption{The matching accuracy(\%) of Q-NQ pairs on variable distances for ablation tests.}
	\label{tab:app2}
\end{table}

\subsection{Example Outputs}

To get a better understanding of the behavior of our models, we include two example outputs in Table \ref{tab:case1}. Both of the cases contain the LQAs with SQAs. The first example is a case where both HYD and HDM predicts QA relations better than RPN. While the second one is a case where both our full model (HDM) and RPN fail, HTY performs best in LQA predictions. As for SQAs, all of the models perform well. However, the DIS model is obviously not capable of matching LQA pairs. It also shows that the distance information sometimes hurts the performance of HDM on matching LQA pairs.


%\usepackage{ctex} it changes the row space! what should I do? And translation
\begin{table*}[h]
	\small
	\centering
	\begin{tabular}{p{1.5cm}<{\centering}cccccc}
		\toprule[1.3pt]
		Ground Truth &RPN&DIS&HTY&HDM&Role&Utterances\\
		\midrule[1.3pt]
		\multicolumn{5}{c}{Q1}&P&\makecell{男,四月。大前天晚上给他试着吃一点蛋黄\\Boy, 4 months. He tried a little yolk yesterday\\ 当天晚上有大便然后到今天为止都\\and shat that night but haven't shat\\还没有大便什么情况????\\ until today what's wrong????}\\
		\hline
		O &O &O &O &O &D&\makecell{你好\\Hello}\\
		\hline
		O &O &O &O &O &P&\makecell{你好\\Hello}\\
		\hline
		\multicolumn{5}{c}{Q2}&D&\makecell{宝宝四个月吗\\Is he four months old}\\
		\hline
		A2&A2&A2&A2&A2&P&\makecell{是的\\Yes}\\
		\hline
		A1&A1&O &A1&A1&D&\makecell{吃的有点早\\Eat too early}\\
		\hline
		A1&O&O &A1&A1&D&\makecell{不建议吃\\Not advise}\\
		\hline
		A1&O&O &A1&A1&D&\makecell{不容易消化哈\\Difficult for digestion}\\
		\midrule[1.3pt]	
		\multicolumn{5}{c}{Q1}&P&\makecell{有什么办法可以减肥呢!\\How can I loose weight!\\太胖了买衣服都难买。烦死了\\It's too difficult to buy clothes. So annoyed}\\
		\hline
		\multicolumn{5}{c}{Q2}&D&\makecell{你多大了,性别\\How old are u, and gender}\\
		\hline
		A2&A2&A2&A2&A2&P&\makecell{男。一八\\Male. Eighteen}\\
		\hline
		\multicolumn{5}{c}{Q3}&D&\makecell{你用过些什么好的方法了\\Any good methods have u tried}\\
		\hline
		A3&A3&A3&A3&A3&P&\makecell{暂时没有哦\\Not yet}\\
		\hline
		A1&A1&O &A1&A1&D&\makecell{如果能坚持锻炼的话,还是可以的\\If you can keep exercising, you will}\\
		\hline
		A1&O &O &O &O &D&\makecell{同时要控制好自己的饮食\\Control ur diet at the same time}\\
		\hline
		A1&O &O &A1&O &D&\makecell{这两点很关键\\These two points are important}\\			
		\bottomrule[1.3pt]
	\end{tabular}
	\caption{Two cases of predictions and human annotations in our dataset.}
	\label{tab:case1}
\end{table*}


\end{CJK}
\end{document}

