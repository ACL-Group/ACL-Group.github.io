\section{Problem Definition}
\label{sec:problem}
%There are two participants in an online health counselling dialgue: a patient and a doctor. In most cases, both participants will raise questions and ask for information from the other. 
Our work aims at identifying QA relations by matching Qs and NQs in two-party dialogues, which can be regarded as a turn matching problem. Given a dialogue sequence with $T$ turns: 
$$[(R_1,L_1,U_1),(R_2,L_2,U_2),...,(R_T,L_T,U_T)]$$ 
where $R$ denotes the role, identifying which party utters
the turn, and $L\in\{Q, NQ\}$. 
%$Q$ and $NQ$ categories all the turns into questions and non-questions. 

Our job is to match each $(Q, U_i)$ with corresponding $(NQ, U_j)$, where:

\begin{equation}
\begin{aligned}
j>i&\quad 1\leq i,j\leq T\\
R_i&\not=R_j\\
\end{aligned}
\end{equation}

The {\em distance} of a Q-NQ pair ($U_i$, $U_j$) is $j-i$. We define
the {\em history} of such a Q-NQ pair as the turns 
$\{U_{i+1},U_{i+2}...,U_{j-1}\}$ which located between the Q and NQ.

Recent work by He et al.~\shortcite{he2019learning} considers a slightly different QA alignment problem where one answer can match multiple questions. We think that if a question is asked repeatedly, the answer should be matched to the closer one and the former ones are considered missed. By our definition, a Q can match nothing (U9) or several NQs (U2). From the viewpoint of a NQ, it is either or not matched with a question (such as U7). When a NQ is matched to a question, it is 
considered as an {\em answer} (A). Otherwise, it's considered as a chit-chat turn
(O).


%Prefix $R$ refers to the role of the turn. The matched NQ will be named ofter $A_i$, while the rest which do not have corresponding Q will be regarded as others($O$).

%\subsection{Challenges}

% they assume that the questions are always raised by one participant and try to align among the utterances from the other to generate QA pairs.

% Utterances alignment in the multi-round conversations is much more difficult besides mixing of different kinds of utterances. 
% %\KZ{I don't see how this is more difficult than the three things you mentioned in the prev para.}
% Two or more questions may be proposed at the same time which causes the order of the answers a mystery. For example, in Figure \ref{fig:sample}, the doctor raises question U2 and U3 continuously, while the patient may answer the latter one first. In the recent work from He, they consider multiple alignment issue which means one answer may align with multiple questions. Different from their problem definition, we don't allow such one-to-many alignments for each non-question utterance. The fragmentation phenomenon of answers and long-range QA pair matching are the main difficulty in our dataset. 

%QA matching in online dialogues is not an easy task.The challenges are as follows:
%\begin{itemize}
%    \item Several questions may be raised continuously instead of solving one by one, and the order of answers may be not the same. For example, in Figure \ref{fig:sample} the doctor ask the question U2 followed by a question U3. The patient can answer U2 first and then U3, or conversely.
%   \item The participants may ignore some of the questions as well. As a result, these questions like U9 has no related answers.
 %  \item For IQA and FQA cases, the distances between Q and NQ are large. This cause the long distance QA matching problem.
   %\item Since the dialgoues come from the real online discussion forum, the utterances are quite noisy and informal with punctuation misuses, wrongly written characters and other informal expressions. 
   
    %\item There also exits some questions can not be answered directly without acquiring more information from the other paricipant. IQA matching exits.  
    %\item Based on above statistics, around 22.67\% questions have more than one answer. It doesn't mean that the question has been answered by the other again and again, but that the complete answers tends to be broken into several pieves. The fragments of a complete answer maybe show up one by one in one's sequential utterances, and may be interrupted by the other participant.

    % This happens when the original question can not be replied directly, so the other participants need to ask for more information and may give the final answer to the original question after many turns.
%\end{itemize}

 %After all, the questions, answers and other utterances are mixed together and cause the great challenges of the task.

