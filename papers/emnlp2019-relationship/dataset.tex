\section{Dataset}
\label{sec:data}
Although there are many currently available dialogue datasets, 
most of them are used for training automatic dialogue robots/systems, 
thus don't have diversity in the speakers personal relationships or 
the relationship labels are hardly available.
Therefore, we build a new dataset 
composed of $6,307$ sessions of dyadic dialogues with labels 
about the personal relationship between the two speakers, 
starting from movie scripts crawled from 
The Internet Movie Script Database (IMSDb).
Despite the focus on family/workplace relationships in this paper,
we will introduce the construction process of the whole dataset
in the hope that it will be helpful for future research.
%~\footnote{\url{https://www.imsdb.com/}}.

\subsection{Movie Dialogues vs. Real-world Dialogues}
Recording real-life dialogues, especially those of various relationships, 
is really hard to be carried out on large scale. Therefore, in this paper,
we instead use movie dialogues, which is 
easy to obtain and process computationally. 
Previous psycholinguistic study has shown that movie dialogues
show similar stylistic futures with spontaneous ones, and it points
out that linguistic analysis of film dialogues promises to be 
rewarding~\cite{overhearer}. Cognitive process of contextualizing
movie dialogues is found to be parallel to understanding real ones~\cite{FSC}.
Livingstone~\shortcite{television} also states in 
the book \textit{Making Sense of Television: The Psychology of 
Audience Interpretation} that ``television contents and real-life experiences, 
despite their physical difference, are perceived through the same, 
or at least, heavily overlapping, interpretative frameworks''. In computational
linguistics, previous study focusing on language entrainments~\cite{cornell-corpus} 
also build their conversation dataset from movie scripts. 
They find that movie dialogues exhibit stylistic convergence 
similar to real-world ones.
We thus reasonably assume movie dialogues are acceptable
substitutions for real ones on the relationship inference task.

\subsection{Dialogue Extraction and Processing}
Initially, we crawl 995 movie scripts from IMSDb, and 941 remain after 
we automatically match the titles with movies in 
IMDb\footnote{\url{https://www.imdb.com/}} and filter out those that 
(1) don't have a match in IMDb; (2) are not in English; 
(3) are very unpopular(measured by number of raters). 
By observing the formats of the scripts and manually defining text patterns, 
we split each script into scenes, extract the sequence of (speaker, utterance) 
pairs for each scene and identify subsequences that meet the following requirements  
as dyadic dialogue sessions: 
(1) two speakers speak alternately without being interrupted 
by a third one; 
(2) each contains at least 3 turns. 
We set this minimum length requirement to make sure 
that two speakers are speaking to each other instead of 
participating in a group discussion. 
Finally, we count total number of turns taken 
between each character pairs and filter out those having 
fewer than 20 turns to 
make sure the relationship between the two characters is significant 
and not as trivial as greetings between strangers. 
This filtering step also helps reduce the cost of labelling 
because more sessions can share the same pair of speakers.

\subsection{Labeling Methodology}
%\subsubsection*{The Taxonomy}

%An important social psychology problem concerned in our task is 
%how to classify personal relationships. A taxonomy of personal 
%relationships has been considered as a prerequisite for 
%understanding their functionalities and 
%social effects~\cite{taxonomy2}. 
%There are many discussions on this, a great deal of 
%which describe relationships by certain 
%characteristics that the researchers believe have social significance, 
%such as attachment, intimacy, communalty and independence~\cite{class-SDT, %encyclopedia}. For example, \citeauthor{class1} 
%classified relationships into ``communal'' ones and ``exchange'' ones by 
%the rules governing the giving and receiving. 
%A problem with those taxonomies is the characteristics 
%are very domain-specific~\cite{taxonomy2}, 
%which means the label-obtaining process would require highly 
%professional annotators, and the results only pertain to
%certain topics and say little about others. 
%Therefore, in this paper we adopt a taxonomy that makes 
%more general sense in life: classify relationships by their 
%relatively objective social connections such as child-and-parent, 
%spouse and co-workers. 
For the annotation process, we define a 13-class taxonomy of relationships, 
including {\em child-parent}, {\em child-other family elder}, {\em siblings}, {\em spouse}, {\em lovers}, {\em courtship}, {\em friends}, {\em neighbors}, {\em roommates}, 
{\em workplace superior-subordinate}, {\em colleges}, {\em opponents} and 
{\em professional contacts},
based on \citeauthor{encyclopaedia}'s book, 
in which they elaborate on psychological 
and social aspects of various relationships. 
we choose categories by social connections because they make general 
sense in life. Although individual difference exists 
in every real-world case, it was found that such relationship category has different expectations, 
special properties (e.g., marriage usually involves sex, 
shared assets and raising children)~\cite{class-property}, 
distinctive activities (e.g., talking, eating, drinking and 
joint leisure for friendship) and 
rules~\cite{class-rules} of its own, which are agreed across cultures.
Note that this is not an all-round coverage of 
all possible relationships in human society and there may be
overlaps between different categories, but we aim to cover those common
ones in real life which may be of interest in interpersonal 
relationship research.
These fine-grained labels are prepared for possible related future research.
To study relationship at a less specific level as in this work, 
our labels can be easily translated to coarser-grained taxonomies 
by combining related categories. For example, \{child-parent, child-other family elder, siblings and spouse\} constitute the family relationship group.

%\subsubsection*{Annotation}
Although personal relationships are not static or mutual exclusive, 
most of them exhibit relative stability over time~\cite{stability}, and
 relationships in movies are usually more clear-cut. 
Therefore, in this paper, we model relationship as a single, 
stable label. Such assumption simplifies our task and 
significantly reduces the workload of labelling, though it introduces 
ambiguity in certain cases such as evolving relationships 
(e.g., courtship $\rightsquigarrow$ lover $\rightsquigarrow$ spouse) or 
concurrent ones that do not usually exist together
(e.g., enemies falling in love). To avoid these situations, 
We require the annotator to only 
assign labels when the relationship is clear, relatively stable and typical.

Our ground truth annotator was provided with the movie title, 
the pair of characters involved in the dialogue, 
movie synopsis from IMDb and Wikipedia for each movie, 
as well as complete access to the Internet, 
and was asked to choose between one out of 13 classes or 
``Not applicable (NA)'' label. 

It took the annotator 100 hours across one and a half months to 
finish the annotation of 300 movies, at a rate of approximately 
$4.07$ minutes per pair. Only $47.11\%$ of the pairs
received a specific label, while others are considered ``not applicable''. 
A detailed statistical description of our current dataset 
along with annotation evaluation is provided in \secref{sec:eval}.
