\section{Related Work}
\label{sec:related}

Multi-document summarization~\cite{abs-2011-04843} is used for generating an informative summary of multiple topic-related texts, such as 
%summarization on 
news\cite{FabbriLSLR19} and emails~\cite{ZajicDL08}.
% Wikipedia articles ~\cite{LiuSPGSKS18} and so on. 
Opinion summarization~\cite{GeraniMCNN14} is a typical multi-document summarization problem,
which can generates
%a concise and coherent 
a summary covering the salient opinions %for a product or service 
of multiple reviews. It inherently has a special focus on aspects of the product or service, making it different from other tasks.

Opinion summarization suffers from a lack of training pairs. 
Some work~\cite{MeanSum19} uses auto-encoder to train the model 
by a reconstruction loss and a similarity loss. 
Others construct synthetic datasets for supervised training. 
The input format of the synthetic datasets can be divided into the textual input and the structured input.
For the textual input,
the most intuitive way 
%proposed by some approaches
\citet{Copycat20, Fewshot20} is to 
regard one review for a product as the summary 
and take all or part of the rest as input. 
%So, further works~
\citet{Plansum20} took the nearest neighbors as inputs based on review representations.
%Adding noise to the sampled summary and taking the disturbed ones as the input reivews is another way to generate synthetic datsets. 
\citet{Denoise20} added noise to the sampled summary from two aspects: the segment noising at the token and chunk level, and document noising by replacing the whole review by a similar one. 
%Different from their simple replacing, removing and inserting operations, 
However, the quality of such datasets is limited by the biased reviews, 
which cannot be summarized from other reviews. 


%Although above mentioned work mainly focused on data construction and ignored the characteristics of reviews, aspects and opinions are quite important for opinion summarization~\cite{MukherjeePVGBG20}. 
For the structured input, as the aspects and opinions are quite important for opinion summarization~\cite{MukherjeePVGBG20},
 \citet{AngelidisL18} tried to extract aspect-specific opinions by out-of-review knowledge. 
%\citet{TianY019} classified words into three types, including aspect, opinion and context and predicted the work type as a first step. 
\citet{OpiDig20} also used opinion-aspect phrases for filtering information in multiple reviews and transformed the task into single document summarization. 
%Besides, previous work~\cite{Plansum20} shows that project each review into aspect and sentiment distributions can also help.
However, all of these works are limited by the accuracy of the opinion-aspect extractor and also
neglect that some detailed other information are left in the sentences without typical opinion-aspect pairs. 

Thus, we propose a method to create a semi-structured synthetic dataset consisting of
opinion-aspect pairs and implicit sentences. 
We construct the noisy opinion-aspect pairs noising and noisy implicit sentences as input, which leads to more comprehensive summaries.  
