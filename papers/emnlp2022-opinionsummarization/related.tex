\section{Related Work}
\label{sec:related}

%Multi-document summarization~\cite{abs-2011-04843} is used for generating an informative summary of multiple topic-related texts, such as news~\cite{FabbriLSLR19} and emails~\cite{ZajicDL08}.
% Wikipedia articles ~\cite{LiuSPGSKS18} and so on. 
Opinion summarization 
%can generate a summary covering the salient opinions
%of multiple reviews. It 
has a special focus on aspects of the product or service, 
making it different from other multi-document tasks, 
such as news summarization~\cite{FabbriLSLR19}.

Opinion summarization suffers from a lack of training pairs. 
Some work~\cite{MeanSum19, Copycat20,tree21} used autoencoder to train the model by 
reconstructing loss or sentence embeddings. 
%\citet{TianY019} classified words into three types, including aspect, opinion and context and predicted the work type as a first step. 
Others create synthetic datasets for supervised training. 
The input format of synthetic datasets is textual or structured.  
For the textual input,
some approaches~\cite{Fewshot20,transsum21}
regarded one review as a summary 
and took all or part of the rest as input. 
\citet{transsum21} computed the distance between the summary and all remaining reviews as weights of review embeddings.
\citet{Plansum20} took the nearest neighbors as inputs based on review representations.
\citet{Denoise20} added noise to the sampled summary from the segment noising and document noising by replacing the whole review with a similar one. 
\citet{prefix21} labeled input reviews and sampled summaries with control tokens and took control tokens as prefixes at decoding.
%which cannot be compared fairly with the methods without external labels.
However, these datasets is limited by biased reviews, 
which cannot be summarized from other reviews. 


%Although above mentioned work mainly focused on data construction and ignored the characteristics of reviews, aspects and opinions are quite important for opinion summarization~\cite{MukherjeePVGBG20}. 
The aspects~\cite{luo2018extra,luo2019knowledge} and 
opinions are quite important 
for opinion summarization.
Some approaches~\cite{AngelidisL18,MukherjeePVGBG20} classified the sentences of reviews into different aspects and collected the most salient sentence of each class as summary.
\citet{TianY019} 
classified words into three types (i.e., aspect, opinion and context) and predicted summary by the probability distribution on these types.
Inspired by these works,
\citet{OpiDig20} extracted opinion-aspect phrases from each review and transformed the task into single document summarization. 
\citet{amplayo-etal-2021-aspect} used predefined aspects to construct synthetic training data and trained a controllable model to generate summaries based on aspects.
However, all of these works 
neglect some other information in the sentences which cannot be explicitly formulated as opinion-aspect pairs. 

Thus, we create a mix-structured synthetic dataset consisting of
opinion-aspect pairs and implicit sentences, which can get more accurate and comprehensive summaries.  
