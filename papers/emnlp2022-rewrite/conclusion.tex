\section{Conclusion}
%\KZ{This conclusion looks more like a summary of the whole paper. Instead, u should focus on the lessons learned and the open questions going forward. Someone rewrite this one?}
In this work, we present a new 2-phase 
%(locating positions to rewrite and filling the blanks) 
framework
which includes locating positions to rewrite
and filling the blanks
for solving Incomplete Utterance Rewriting (IUR) task.
We also propose an LCS based method 
%to generate our training data automatically
to align the original
incomplete sentence with the ground truth utterance to obtain the positions of coreference and ellipsis.
%in the original sentence.
%Using these generated data, we first train the sequence tagging model and try to predict the positions in the sentence that should be rewritten.
%Then we use the fine-tuned T5 model to fill in the predicted positions to rewrite.
%On three public datasets for rewriting task, 
Results show that 
our model performs the best in several metrics. 
%As for time cost, our model is less time-consuming 
%than the strongest baseline 
%(T5-small generative model)
%at prediction stage.
%And the time cost is close to the strongest baseline. 
%which demonstrates the efficiency of our framework.
We also recognize two directions for further research.
First, as the performance of 
our 2-phase framework is often limited by 
the first phase, 
we will try to improve the accuracy of locating rewriting positions.
Second, it will be useful to study
the best way for applying our
rewriting model to 
other downstream NLP tasks.
%We believe that our work will inspire future rewriting tasks.
%and promote the resolution of downstream NLP tasks. The divide and conquer approach is sometimes better than the end-to-end methods.
