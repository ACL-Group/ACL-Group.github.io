\section{Limitations}
\label{sec:limitation}

Our work has some limitations that could be addressed in future research. 

\begin{itemize}
    \item Despite our best efforts in designing the prompt for the doctor chatbot to align with the requirements specified by psychiatrists, the final human evaluation reveals that the full prompt (i.e., \texttt{D1}) does not outperform the results obtained by removing certain parts (i.e., \texttt{D2}, \texttt{D3}) from the prompt in terms of user experience. Although we may not achieve the optimal prompt design, our comprehensive exploration provides valuable insights into what constitutes an professional doctor chatbot, which can serve as a foundation for future works in this domain.
   \item In this study, our focus is on the development and evaluation of doctor and patient chatbots powered by LLMs, specifically targeting depressive disorder. However, it is important to note that in reality, individuals often experience multiple mental disorders concurrently, which introduces additional challenges. For doctor chatbots, simultaneously diagnosing multiple mental disorders requires managing a broader range of possibilities. On the other hand, patient chatbots need to simulate a complex mixture of symptoms, which can be difficult to accurately replicate. We hope to scale our approach to encompass a wider range of mental disorders in the future.
% \item we only involve depression in this work, .... \KZ{This will not
% be a limitation if this paper is only about depression and we be explicit
% about that in the title. In fact this can be plus, cos we can say that
% what's done here for depression can be extended to other disorders.} 
\end{itemize}

% NEWCOMMENT: rare是否定含义,表示几乎没有,应该不是想表达的意思,我改成a few了
% NEWCOMMENT: 这一节太长了,还有表格,太夸张了。这两个表格的内容可以移到补充材料里作为案例分析(case study),相关的现象可以在实验结果部分简单提一提,而不是放在局限这一节
% NEWCOMMENT: limitation还是不建议写太长,用陈老师的话讲,就是,你的工作,审稿人肯定不会比你更了解,不用主动把把柄递到审稿人手里,写一些比较明显但又无伤大雅、无关紧要的内容就好
