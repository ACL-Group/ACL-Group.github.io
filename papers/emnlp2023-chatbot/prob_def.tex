\section{Objectives}
% objectives
\label{sec:objectives}
% In this paper, we ask the following scientific question:
% \begin{quote}
%     ``What constitutes a realistic doctor/patient chatbot in psychiatric outpatient conversations?'' 
% \end{quote}

Given the lack of formal definition for what constitutes a good doctor/patient chatbot in psychiatric diagnosis conversations, we invite five experienced psychiatrists to establish clear objectives and standards, which will guide us throughout the study\footnote{These psychiatrists come from top-ranked national mental health centers. 
Their professional titles and areas of expertise can be found in 
Appendix \ref{apd:psych_info}.}. 
The objectives are derived from the first two phases. In phase 1, the psychiatrists are encouraged to freely express their opinions, allowing us to summarize a set of objectives.  In phase 2, as we iteratively refine the prompts, we consult the psychiatrists for their input and identify any additional objectives to consider based on the performance of chatbot. The final objectives are organized as follows.
% \MY{how do we know that at last stage, this is the final standard? do we have a comment from doctors that the defined objectives are exhaustive and up to application standard?}

% \KZ{The following is a bit weird. Shall we change our title to include depression? Since this paper focuses on bots to diagnose depression only?}
% Considering the variation in diagnosis standards and symptoms for different mental disorders, we concentrate on depressive disorders for this study, while leaving the scaling to include other disorders as future work.

\subsection{Doctor Chatbot}
\label{sec:doc_requirements}
As a doctor chatbot, the primary task is to conduct a professional diagnostic process for the patient and provide an \textbf{accurate diagnosis}. To achieve this, a good doctor chatbot should possess the following three capacities:
% \MY{I don't think empathy-related stuff is only for user experience, it also serves the purpose of building inter-personal relation and trust for the patients, and eased their nerves for the purpose of better revealing truthful feelings and symptoms, hence a more accurate diagnosis}
\begin{itemize}
    \item \textbf{Comprehensiveness:} Inquire about the key symptoms of depression, including sleep, mood, diet, and other relevant aspects that are required for diagnosis, as defined in DSM-5~\cite{american2013diagnostic}.
    \item \textbf{In-depth Questioning:} Conduct thorough questioning (e.g., ask the duration of a symptom) based on patient's responses to gain a more precise evaluation of the symptoms.
    \item \textbf{Empathy:} Demonstrate empathy and provide emotional support to patients to establish trust and ease their nerves. This helps patients feel more comfortable in expressing their genuine feelings and symptoms, leading to a more accurate diagnosis.
    % \item Provide examples to guide patients to articulate their symptoms, rather than just asking, "Do you have any other symptoms?", the latter often causing patients to overlook important symptoms and forget to mention to the doctor.
\end{itemize}
Moreover, to offer patients superior \textbf{user experience}, the doctor chatbot should fluently switch between topics and enhance the efficiency of the conversation, thus preventing patients from feeling bored or disengaged.
\subsection{Patient Chatbot}
\label{sec:pat_requirements}

% After establishing objectives for doctor chatbots, we encountered difficulties when defining the requirements for chatbots that resemble real patients. This is due to the fact that individuals with the same disorder can exhibit significant variations in their manifestations. Moreover, psychiatrists, though experienced, have no firsthand chatting experience with a ``non-patient-like'' chatbot, making it challenging for them to generalize the requirements for a ``patient-like'' chatbot.

% To address this issue, we decide to develop an initial version of the chatbot first. This allows psychiatrists to interact with ``non-patient-like'' examples, which can help them better define the characteristics and behaviors that constitute a ``patient-like'' chatbot. Based on their feedback, we then iterate and update the chatbot accordingly.
% At this phase, we only establish one fundamental requirement for a patient chatbot.

The basic requirement for a patient chatbot is \textbf{honesty}, which entails presenting an accurate and rational description of symptoms in the provided symptom list, without reporting any non-existent ones.

Additionally, to make the chatbot more resemble real patients, psychiatrists also describe some behaviors commonly exhibited by real patients during consultations. 
\begin{itemize}
    \item \textbf{Emotion:} Patients in a depressed mental state may experience emotional fluctuations during the conversation.
    % while the chatbot's presentation of symptoms is too calm and polite. 
    \item \textbf{Expression:} Patients use colloquial expressions when describing symptoms, and may have difficulty expressing themselves clearly. They often talk about their daily life experiences. While current chatbots tend to explicitly list out the symptoms~\cite{Llanos2021Lessons} using formal language, which is too sane and professional for a patient.
    % However, the chatbot tends to use formal language similar to the official diagnostic criteria (DSM-5).
    \item \textbf{Resistance:} Patients may be reluctant to seek help. They may remain silent and refuse to communicate, or downplay their symptoms to avoid being perceived as a burden. 
    % In contrast, the chatbot is overly cooperative, readily acknowledging and providing detailed descriptions of its symptoms.
\end{itemize}



