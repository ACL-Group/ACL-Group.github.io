\section{Limitations}
\label{sec:limitations}

Our work has some limitations that could be addressed in future research. 

\begin{itemize}
    % \item Though we follow data construction convention in previous works and have made efforts to construct the multiple MDD dataset with a focus on high precision, the disease labels produced automatically by pattern matching of self-reported diagnosis without the guidance of psychiatrists can certainly have some errors. Consequently, our model may inherit inevitable bias from the dataset.
    \item Despite the significant performance boosting over the baseline, our proposed PsyEx model still cannot achieve satisfying performance in multi-label setting, especially on rarer diseases like eating disorder (See Table \ref{tab:disease_multi_label}). To tackle this issue brought by the imbalanced data,  we utilize a commonly-used resampling method, which samples equal amount of users with each disease for each batch. However, we find no improvement in the detection performance of these rarer diseases after balanced sampling, indicating that the unsatisfying results aren't just a matter of sparse positive samples. Therefore, we hope to further address this issue in future studies.
    \item Multiple MDD task is still under-explored currently. Many previous works (e.g., \citet{sekulic2019adapting}) only conduct experiments on binary setting (i.e., separately train $D$ models for detecting $D$ mental diseases). Therefore, for comparison under multi-label setting, we can only adopt their hyperparameters on the binary setting, which may not be optimal in some cases. 
\end{itemize}
  
% First, we only use symptom features to screen high risk posts, but  fail to incorporate them into disease detection model. Though we have tried some ways, such as concatenating symptom features with text embedding, but it doesn't work out well. Maybe more clever methods can be proposed to better process these features in future researches.

% Second, the pooling method for risky score in our post screening process is currently simplistic, and can be further refined. There are certainly options beyond max pooling that can select more symptom-relevant posts.

% Finally, our model does not take useful signals from other modalities, such as image and audio, into account. 
