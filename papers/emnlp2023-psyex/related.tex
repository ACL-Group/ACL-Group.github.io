\section{Related Work}

In the literature, substantial efforts have been made to detect a certain mental disease. Some of these works focus on leveraging features like TF-IDF,  LIWC \citep{pennebaker2001linguistic}, and posting patterns \citep{trotzek2018utilizing, losada2016test} for MDD. 
Others apply various deep learning methods \citep{yates2017depression, gui2019cooperative}, as well as the contextualized embedding \citep{ji2021mentalbert, jiang2020detection} to improve the performance of classifiers. However, these methods often fail to generalize well \citep{harrigian2020models} and cannot provide explainable results due to lack of knowledge in the psychiatric domain. 
To tackle these issues, some works \citep{lee2021micromodels, nguyen2022improving} began to utilize symptom features, but they extract symptom features with unsupervised/weakly supervised methods, which isn't so reliable for the downstream MDD task. 
% \KZ{It would be better to say here what is the differences in our way of using symptoms vs. the previous work that did use symptoms.}

Recent years, some works start to detect multiple mental disorders. \citet{cohan2018smhd} proposed a massive Reddit dataset \textit{SMHD} containing 9 mental disorders, followed by many subsequent studies based on this dataset, such as \citet{sekulic2019adapting} and \citet{Zhang2022SymptomIF}. However, these works often directly apply a single-disease model to the multi-disease data (i.e., train the model for $D$ times to obtain the results of $D$ diseases), overlooking the correlation among multiple diseases, and thus fail to perform well on rarer diseases.

With the emergence of large language models (LLMs), there exists a few studies utilizing LLMs for tasks like depression detection on social media~\cite{lamichhane2023evaluation, qin2023read} or developing chatbots for depression diagnosis~\cite{chen2023llmempowered}, but we fail to find any existing work that simultaneously detects multiple mental disorders or for mental health symptom extraction. While we acknowledge LLM's potential in mental health application, it needs further exploration and investigation as our preliminary results indicated that this remains a challenging task.  
%Furthermore, LLMs have limited inference speed and require substantial computational resources, making it challenging to apply them in practical settings.

% \KZ{I think u can include a few sentences about using LLM for detecting
% mental disorders and say why our approach is still relevant and has some
% advantages.} 

% To reduce computation overhead and interference, \citet{zogan2021depressionnet} uses extractive summarization to extract key posts of a user; \citet{zhang2022psychiatric} and \citet{lee2021micromodels} uses psychiatric scales like BDI \cite{beck1996beck}, PHQ-9 \cite{kroenke2001phq} to guide the  screening. However, these methods based on clustering or  semantic similarity may fail to find many posts describing symptoms comparing to symptom identification model \cite{Zhang2022SymptomIF} with supervised training.

