\section{Knowledge Graph based Product Search Relevance Task}
The Knowledge Graph based Product Search Relevance (KGPSR) task focuses on utilizing KG to improve PSR task. PSR task is clarified as follows. Given a query-product pair $<q, p>$, $q$ is user query and $p$ is the name of product. The goal of PSR task is to compute the relevance score of each query-product pair. 

KGPSR task is a simple variant of PSR task. In KGPSR task, $q$ and $p$ are linked on one KG $G=\{V,E\}$, where $V$ is a set of e-commerce concept nodes and $E$ is a set of relationships. A query $q$ usually performs as a concept in KG and a product $p$ is usually linked on the taxonomy. The goal of KGPSR task is to utilize KG as much as possible to benefit PSR task in e-commerce.

% Specifically, products are mainly linked on the product taxonomy and queries are mainly linked on the e-commerce commonsense concept graph.

% $ S $ of $ n $ 

% $ \{<q_{1},p_{1}>,<q_{2},p_{2}>, ... ,<q_{n},p_{n}>\} $,