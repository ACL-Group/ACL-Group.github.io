

\section{Feature Descriptions}


In our experiment, five audio features mfcc, mel spectrogram, GeMAPS, filterbank and PLP are used, which cover physical meanings at different levels\ref{table:featuremeaning}. 

\begin{table}[ht]
	\centering
	\scriptsize
	\begin{tabular}{l|l}
		\toprule
		    Feature & Description    \\
		\midrule
	    MFCC &  A representation of the short-term power spectrum\\
	    & of a sound, based on a linear transform of a log po-\\
	    & wer spectrum on a nonlinear mel scale of frequency.   \\
		Mel Spectrogram &  Based on the nonlinear characteristics of human pe- \\
		& ripheral auditory system. It can reflect the energy of\\
		& different frequency based on the feeling of human. \\
		GeMAPS & The Geneva Minimalistic Acoustic Parameter Set, \\
		& a basic standard acoustic parameter set for various\\
		& areas of automatic voice analysis.  \\
		filterbank & An array of bandpass filters that separates the input\\
		& signal into multiple components, each one carrying \\
		& a single frequency sub-band of the original signal. \\
		PLP & Perceptual Linear Predict ive is a set of coefficients\\
		& of the all-pole polynomial model prediction.\\
		\bottomrule
	\end{tabular}
	\caption{Several features we used and their descriptions}
	\label{table:featuremeaning}
\end{table}


The toolkits we adopt to extract these features are listed as follows:

\paragraph{Shennong Toolkit}
\begin{itemize}
    \item Feature(s): mfcc, PLP, Mel Spectrogram, filterbank
    \item Link: \url{https://github.com/bootphon/shennong}
\end{itemize}


\paragraph{OpenSMILE}
\begin{itemize}
    \item Feature(s): GeMAPS
    \item Link: \url{https://github.com/naxingyu/opensmile}
\end{itemize}