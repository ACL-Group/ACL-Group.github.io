\section{Problem and Dataset}
\label{sec:assumption}
% \KZ{We ask two scientific questions in this paper: 1) do pet dogs from different human language environment bark differently? 2) if so, are their barking related to their host's language in anyway? To answer these questions, we need a dataset that contains pure dog barking voices from at least two different countries which speak distinctly different languages. Why did we pick Shiba in the first place, and why did we pick Japanese and English?}  

In this paper, we ask two scientific questions: 1) do pet dogs from different human language environment bark differently? 2) if so, are their barking related to their host's language in anyway? To answer these questions, we need a dataset that contains pure dog barking voices from at least two different countries which speak distinctly different languages. Considering that Japanese and English are two common languages with much difference, and Shiba Inu dogs are widely kept as pets, they are picked to construct the dataset.


To find out if dogs in different cultures bark differently, we built a dataset \textbf{EJShibaVoice}~\footnote{The data is available at \url{https://anonymous.4open.science/r/Ani-3040/}.} which is composed of different scenes in the English environment and in the Japanese environment.

The scenes in the dataset and the number of samples under them are shown in Table\ref{table:keywords}. The scene is identified by the searching keyword while gaining the data.


\begin{table}[h]
    \setlength\tabcolsep{4pt}
	\centering
	\scriptsize
	\begin{tabular}{c|cccccccc}
		\toprule
		Scene & play & alone & fight & run & stranger & walk & eat & bath         \\
		\midrule
	    Size(En)    & 420 & 208 & 274 & 199 & 325 & 203 & 142 & 303     \\
	    avg.Len(/s)  & 2.91 &  2.92 & 2.87 & 3.06 & 2.79 & 2.96 & 3.03& 2.53          \\

		\midrule
		Size(Ja)      & 420 & 324 & 328 & 247 & 146 & 211 & 321 & 392          \\
		avg.Len(/s)    &  2.80 & 2.68 & 3.17 & 3.12 & 2.64 & 2.81 & 2.64& 2.67\\
		\bottomrule
	\end{tabular}
	\caption{The number of samples and the average audio length in each environment for different scenes.}
	\label{table:keywords}
\end{table}
%\MYW{avg. audio length in seconds should be included in this table as data statistics.}

So in this paper, our method is to find the audios which may contain dog barks at first, then next we extract audio clips of dog barks.

\subsection{Sourcing for Audio Online}
% YouTube
To understand how the culture of the host influence the language of dogs, we need a large number of the dog barks in different scenes in English and Japanese environment. By setting different keywords which describe the scenes, we accessed the videos on YouTube, which has a huge collection of videos under different scenes, with titles explaining what the videos are about. Then we extracted audios from the videos, which may contain dog barks under different scenes.

\subsection{Clean-up of Audio Samples}
% PANNs
Besides dogs barking, there are other background noises in the audio extracted from a video, such as human voices, birds singing, and so on. Thus, we need to cut out the clean barking audio samples so as to exclude noise. 

For splitting barks from other sounds, we use PANNs~\cite{kong2020panns}, a sound event detection model pretrained on Audioset including as much as 527 sound classes such as speech, telephone and animal to detect dog barks and eliminate background interference. Audios are cut out by the time stamps of ``barks'' detection results. While only those clips not covered by ``music'' and ``speech'' are reserved.

% \MYW{PANNS should be cited and use 1 sentence saying that it is a sound event detection model pretrained on Audioset, including as many as 527 sound classes. And you only retain audio segments with only dog barking present} to detect the sound events in one audio. \KZ{Say a little more on what sound events are being classified here.}
