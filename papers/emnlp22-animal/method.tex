%介绍一下我们使用的features,以及scene-specific是个什么东西捏
\section{Method}
To verify our assumptions that dogs from different human language environments bark differently and the barking difference is related to their host's language. We conduct classification-based experiments as follows.
% \KZ{which 2 are they assumptions or hypotheses? two assumptions}
\subsection{Distinguishing Dogs by Host Languages}
We first attempt to classify dog barks of different host language environments. We conduct Logistic Regression on several audio features to distinguish the barks of dogs from each other by their language environments. We select five types of features shown in \tabref{table:feature} which contain different levels of acoustic characteristics. 

%\KZ{Give some refs or brief desc about these features. If there's no space, this can go into appendix.}

\begin{table}[H]
	\scriptsize
	\centering
	\begin{tabular}{l|l|lllll}
		\toprule
		\multicolumn{2}{c|}{Feature}            & mel-spec & mfcc    & GeMAPS   &filterbank &PLP  \\
		\midrule
		\multicolumn{2}{c|}{Dimension}          & 1025        & 12         & 62    & 24 &  13    \\
		\bottomrule
	\end{tabular}
	\caption{Ultilized features and their dimensions.}
	\label{table:feature}
\end{table}

Some scientists have claimed that the barks of dogs are positively correlated with the scene context~\cite{larranaga2015comparing, molnar2008classification}, which is also validated in EJShibaVoice by  a confusion experiment\ref{sec:main}.


\subsection{Prominent Factors Analysis}
% \KZ{You need a bridge sentence to connect from the previous section.} 
Just finding the existence of the differences does not satisfy us. To find the core influence of host culture on dog barks, we tend to find the prominent features. Considering the Logistic Regression:
\begin{align}
    P(y=1|\bm{x}) = \dfrac{1}{1 + e^{\bm{-\omega^Tx}}}
    \label{eq:lr}
\end{align}

When the inputs are normalized, the partial coefficient $\omega$ is able to express the significance of each feature. Logistic regression performs feature selection effectively. The higher the absolute value of the coefficient is, the more significant the corresponding feature is~\cite{feat1}.

As most of the other feature types lack physical meaning, we choose GeMAPS as the source of our prominent factors. GeMAPS is 62-dimension with comprehensive meaning for each dimension. 

The selection is based on these steps: First whichever dimension with the absolute value of coefficient being higher than 0.20 is considered as important. Those in the general scene classification consist of alternative prominent factors $D$. Then we compare $D$ with those important dimensions of specific scenes. Calculate the overlapping. Select those overlapping more than four times.

%remove the afterwards to another word
Furthermore we analyze the relationship between the barks of dogs and the voice of humans on these final prominent factors and find similarities between them~(\secref{sec:prominentfactor}).
% Finally, five prominent factors are adequate for our selection. \KZ{As this is the method section, you don't need to mention five. You can decide that in the experiments.} We analyze the relationship between the barks of dogs and the voice of humans on these final prominent factors and find similarities between them.