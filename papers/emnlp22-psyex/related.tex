\section{Related Work}

% \KZ{no need to have separate related work section. Shorten and integrate the stuff into intro.} There have been lots of works focusing on the detection of a certain disease like depression \citep{de2013predicting, losada2017erisk} or anxiety \citep{Fauziah2020DesignTM}. 

Recent years, many researchers start to detect multiple mental disorders. \citet{cohan2018smhd} proposed a massive Reddit dataset \textit{SMHD} containing 9 mental disorders, followed by many subsequent studies on this dataset, such as \citet{sekulic2019adapting} and \citet{Zhang2022SymptomIF}. 

Much work in this area focuses on leveraging features like TF-IDF,  LIWC \citep{pennebaker2001linguistic}, and posting patterns for MDD \citep{trotzek2018utilizing, losada2016test}. 
Various deep learning methods \citep{yates2017depression, gui2019cooperative}, as well as the contextualized embeddings \citep{ji2021mentalbert, jiang2020detection} are applied and has improved the performance of classifiers. 

To reduce computation overhead and interference, \citet{zogan2021depressionnet} uses extractive summarization to extract key posts of a user; \citet{zhang2022psychiatric} and \citet{lee2021micromodels} uses psychiatric scales like BDI \cite{beck1996beck}, PHQ-9 \cite{kroenke2001phq} to guide the  screening. However, these methods based on clustering or  semantic similarity may fail to find many posts describing symptoms comparing to symptom identification model \cite{Zhang2022SymptomIF} with supervised training.

