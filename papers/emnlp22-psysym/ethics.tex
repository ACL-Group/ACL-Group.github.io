\section*{Ethics Statement}

\paragraph{Annotation} We pay the annotators a fair wage above the minimum requirement.
If workers have any questions or concerns, we will respond to them immediately. 
Since the content involves the expression of mental disease symptoms, we may expect negative effects on the annotators. Therefore, the annotators can freely take breaks or quit the task at anytime. 
We also interviewed some annotators about their feeling after annotation. They only reported slight discomfort at the time of reading sad or frightening posts due to empathy, and they found no long-term negative effects on them.

\paragraph{Application} Mental disease detection can be related to some sensitive topics, so we should be careful with its applications. First, since mental diseases like depression are still not well understood or even stigmatized in many regions, improper usage of MDD techniques may do harm to the users. Moreover, the precision and recall of the algorithm is far from prefect. It may make false/missing diagnoses which can prevent the user from getting proper treatment, but may still be an useful auxiliary tool for those who are unaware of their mental conditions or cannot access mental services. Therefore, the predictions of the model should be carefully re-examined by professionals for a confirmed diagnosis, where the symptom prediction results may facilitate quick inspection when served as the disease-specific summary of the long posting history. 
We will also require the users of PsySym to comply with a data usage agreement to prevent the invasion of privacy or other potential misuses. 
