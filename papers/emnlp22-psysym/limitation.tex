\section*{Limitations}
\label{sec:limitations}
There are some limitations to this study that could be addressed in future research. 

First, although we tried our best to improve the diversity of the annotated sentences with embedding-based retrieval methods that can find symptom expressions without standard keywords. There can still be blind points we can not cover, such as the posts outside mental health related subreddits, and those cannot be found due to the limitations of the retrieval model itself.

Moreover, there are some types of symptoms we are unable to annotate due to the characteristics of our data source. For example, hallucination (a symptom of schizophrenia) usually requires the observations from another person to be identified, and can hardly be found on Reddit where user mainly shares subjective experience. The fact that Reddit is dominated by adult users \citep{gjurkovic2021pandora} also prevents us from finding the typical symptoms of autism and ADHD among children. 

Last but not least, our dataset does not include useful signals from modalities other than text. For instance, the time pattern of posting may also reveal the symptom of insomnia, and the features of the user's ego centric network may show the troubles in his/her social relations \citep{de2013predicting}. The faces and colors of the posted image may also help identify depression \citep{gui2019cooperative}. If videos or sounds can be leveraged, the acoustic features of speech can help recognize the emotions like sadness, fear and anger \citep{Busso2008IEMOCAPIE} for better detection of mental diseases \citep{wu2022climate}. 