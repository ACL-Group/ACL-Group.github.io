\section{Related Work}

\paragraph{Mental Disease Detection} Mental Disease Detection (MDD) from social media is enabled by the users' self disclosure of their diagnosis, or their participation in mental-disease related topics and forums. These proxy signals can be leveraged to automatically label the diagnosed diseases of users for the supervised learning of machine learning algorithms. Early researches mainly focus on the detection of depression \citep{de2013predicting}, and following works further extend the scope to multiple diseases \citep{coppersmith2015adhd,cohan2018smhd}. 

Approaches for MDD can be mainly divided into two types. The first utilizes features like bag-of-words, topic modeling and LIWC \citep{pennebaker2001linguistic} with traditional machine learning algorithms \citep{shen2017depression,trotzek2018utilizing}. These methods can provide word/topic-level interpretability, but they cannot leverage the temporal pattern of the posts. The second type leverage deep neural networks that can encode the posts as a sequence for better temporal modeling \citep{yates2017depression,sekulic2019adapting,gui2019cooperative}. However, these methods work as black-boxes and cannot provide explanations. Recent works have also revealed that both types of methods suffer from the lack of generalizability \citep{harrigian2020models,nguyen2022improving}. 

\paragraph{Symptom Identification} There have been some pioneering attempts to leverage symptom-related features for MDD. \citet{karmen2015screening} uses manually complied lexicon to detect symptoms, and aggregate them into a score for the detection of depression. \citet{lee2021micromodels} and \citet{zhang22risky} leverage the embedding similarity between a post and symptom-related descriptions to decide the presence or risk of a symptom. \citet{nguyen2022improving} uses regular expressions and heuristics to automatically build weakly-supervised training data for symptom identification. These methods have exhibited superior generalizability and interpretability. Nevertheless, the efforts on establishing annotated corpus for symptom identification are still limited \citep{mowery2017understanding}, which may hinder the potential of symptom-assisted MDD methods for leveraging stronger supervised symptom models. 
% Moreover, these works mainly focus on depression, leaving the possible benefits of modeling the shared symptoms of multiple diseases unexplored. We thus aim to build a large-scale, multi-disease annotated dataset for symptom identification to facilitate further research progress. 
