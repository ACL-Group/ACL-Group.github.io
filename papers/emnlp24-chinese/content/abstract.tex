\begin{abstract}
Reconstructing ancient Chinese pronunciation is a challenging task due to the scarcity of phonetic records. 
Different from historical linguistics' comparative approaches, we reformulate this problem into a temporal prediction task with masked language models, digitizing existing phonology rules into ACP (Ancient Chinese Phonology) dataset of 70,943 entries for 17,001 Chinese characters. 
Utilizing this dataset and Chinese character glyph information, our transformer-based model demonstrates superior performance on a series of reconstruction tasks, with or without prior phonological knowledge on the target historical period.
Our work significantly advances the digitization and computational reconstruction of ancient Chinese phonology, providing a more complete and temporally contextualized resource for computational linguistics and historical research. 
The dataset and model training code are publicly available\footnote{\url{https://anonymous.4open.science/r/Ancient-Chinese-Phonology-BA72}}.
% \KZ{Too much details here. Cut some.}
%\KZ{Don't put everything in one file. Create one file for each section: intro.tex, approach.tex, eval.tex, etc.}
\end{abstract}