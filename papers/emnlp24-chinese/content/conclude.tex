\section{Conclusion}
We introduce an extensive ancient Chinese pronunciation dataset with 70,943 entries for 17,001 Chinese characters, alongside an enhanced transformer-based model integrating glyph and temporal information to refine traditional phonological reconstruction results. Our model outperforms traditional methods across various ancient Chinese pronunciation reconstruction tasks with superior accuracy even under low-resource scenarios. Despite the incomplete phonetic data, it maintains high performance for reconstructing and predicting Chinese pronunciations.
We offer a richer, temporally contextualized resource for computational linguistics and historical research. This study lays a strong foundation for future research in phonetic reconstruction and language evolution.