\section{Related Work}
\paragraph{Language models for phonetic reconstruction}
A related task of phonological ancient language reconstruction is proto-word reconstruction, which takes set of words in different contemporary languages as input and the corresponding word in their common ancestral language as result of supervised reconstruction. ~\citet{meloni_ab_2021} and~\citet{akavarapu_cognate_2023} both evaluate neural networks' performance on Romance language family's reconstruction task. ~\citet{kim_transformed_2023} first introduce Transformer architecture into proto-word reconstruction task and outperforms previous models on both Romance and Sinitic dataset. 
While large language models (LLMs) have recently demonstrated exceptional capabilities in understanding and generating contemporary languages, their proficiency in comprehending ancient Chinese, remains inadequate. \citet{zhang-li-2023-large}'s research highlights the limitations of LLMs in handling the complex ancient Chinese phonetic information.
\paragraph{Chinese phonetic dataset}
In terms of Chinese phonetic datasets, current digitization all organize the ancestor language (Middle Tang Chinese) and its daughter languages (modern Chinese dialects) into a cognate set. ~\citet{hou_j_xiandai_2004} first collecte 2,789 cognates of word-wise Chinese dialect pronunciation. ~\citet{chang_wikihan_2022} expand Hou's dataset, organize entries by characters instead of word.
As for chronological phonology dataset in Chinese, existing resources are mainly from studies of historical linguistics. Swedish sinologist Karlgren first put forward the phonological reconstruction of Middle Tang Chinese~\cite{TheReconstructionofAncientChinese}. ~\citet{wang_l_hanyu_2012} provides a comprehensive analysis of Chinese language phonological evolution. However, these sources are not digitized to our knowledge.