\begin{abstract}

This paper attempts to discover communication patterns automatically within dog vocalizations
in a data-driven approach, which breaks the barrier previous approaches that rely
on human prior knowledge on limited data. 
We present a self-supervised approach with HuBERT, enabling the accurate classification 
of phones, and an adaptive grammar induction method that identifies phone sequence patterns 
that suggest a preliminary vocabulary within dog vocalizations. 
Our results show that a subset of this vocabulary has substantial causality relations
with certain canine activities, suggesting signs of stable semantics associated with these
``words''.
%We use this approach to undercover phonemes and mine vocabulary of dogs. 
%We further develop a web-based dog vocalization labeling system. This system can highlight phoneme n-grams, present in the vocabulary, in the dog audio uploaded by users.
%This approach can be simply applied to find other non-human language sound units and is valuable for further research on dog language understanding.

\end{abstract}
