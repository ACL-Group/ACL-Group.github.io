\section{Related Work}

To decode a dog's language, it is necessary to analyze its basic sound units, linguistic structure, lexicon, meaning, etc. 
The past few years have seen a surge of interest in using machine learning (ML) methods for studying the behavior of nonhuman animals~\citep{rutz2023using}. 
Much of the past work has mostly focused on dog behavior~\citep{abzaliev2024towards, ide2021rescue,ehsani2018let} and the meaning of dog sounds~\citep{molnar2008classification,hantke2018my,larranaga2015comparing,hantke2018my,pongracz2006acoustic}. 
Most of them only classify the audio of the dog sounds in multiple categories, including activities, contexts, emotions, ages, etc. They did not study the sound units of the dog's language.

A number of researchers~\citep{hagiwara2023aves, abzaliev2024towards} have shown that self-supervised methods are equally adept at analyzing and characterizing the vocalizations of animals. Specifically, the work by ~\citeposs{hagiwara2024ispa} has utilized HuBERT to establish a phonetic alphabet that transcends species, facilitating the transcription of animal sounds.


The above work likewise illustrates the existence of multiple distinct sound units in dog language. Many of the species that appear to use only a handful of basic call types may turn out to possess rich vocal repertoires~\citep{rutz2023using}. Many works demonstrate the diversity of animal sounds~\citep{paladini2020bark,robbins2000vocal,bermant2019deep}. \citet{huang2023transcribing} and \citet{wang2023towards} did a fine-grained study of dog sound units. But they directly using a priori knowledge of human language may be inapplicable. \citet{hagiwara2024ispa}
and \citet{hagiwara2023aves} also did a fine-grained study of dog sound units. 
But they did not explore the possible meanings of these units.

Conducting unsupervised lexicon discovery on speech without any prior knowledge is 
a very challenging task~\citep{park2007unsupervised}.
\citet{lee2015unsupervised} achieved end-to-end human lexicon discovery from audio by 
jointly training Hidden Markov Models (HMMs)\citep{schwartz1984improved} for phone discovery
and Adaptor Grammar~\citep{johnson2006adaptor} for lexicon discovery.
However, the accuracy of their lexicon discovery remains low. In this paper, we used an 
improved approach to train the parameters for Adaptor Grammar.

