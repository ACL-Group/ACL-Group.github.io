Prison Term Prediction (PTP) models are vital for aiding judicial decision-making. Existing PTP models fall short in leveraging the structural knowledge in legal rules, leading to inaccurate predictions. Furthermore, the scarcity of training data for most criminal charges makes the existing PTP models hard to train. In response to these challenges, we propose to represent structural legal knowledge by \lawgraph{s}. We further propose a lightweight Statute Knowledge Encoder (SKE) to acquire the latent representations of \lawgraph{s} via a specialized message-passing mechanism tailored for law graphs. Experiments on Chinese criminal cases show that our solution outperforms the state-of-the-art by a significant margin across various metrics\footnote{The code used in this paper will be released after paper acceptance.}. 
% \KZ{Cut down the abstract.}

% designed to sift through judicial decisions to identify anomalies, thereby ensuring fairness. We conclude previous research within PTP frameworks, and find misaligned problem formulations, the conflation of diverse crime categories, and data imbalance and scarcity issues. In response, we introduce a novel PTP framework that aligns more closely with realistic legal theories. Further, We present the Statute Knowledge Enhanced (SKE) approach to mitigate the suboptimal performance issue. This methodology transforms unstructured legal texts into a structured format via a Law-Graph. We evulate our proposed method SKE on  Chinese criminal cases and demonstrate that the proposed SKE method improves performance across various metrics, including regression analysis, category accuracy, and correlation with human judicial decisions.\footnote{The code used in this paper will be released after acceptance.}