\section{Details of \lawgraph{} Definition}
\label{app:e}

The node types and edge types are summarized in Table~\ref{apptab:L_node_t} and \ref{apptab:edge} and include some examples in these two tables. 



We set the nodes as generalizations to make the \lawgraph{} definition applicable to as many statutory provisions as possible. Specifically, \textit{Punishment} nodes widely exist in the Criminal Law since sanctions are necessary. \textit{Numerical} nodes are also used in every provision, as statutory sentences are defined by numbers or ranges. For example, in Table~\ref{tab:three}, there are several numerical nodes: \textit{$\geq$ Twenty Thousand Yuan}, \textit{$\leq$ Five Years}, \textit{$\leq$ Ten Years}, and \textit{$\geq$ Ten Years}. \textit{Logic} nodes can be used in any provision as they all contain logical text, such as ``And'' and ``One of ...''. Additionally, we assume all other selected words or phrases from statutory provisions as \textit{keyword} types. Thus, there is no challenge in using this definition for general and widely applicable criminal statutory provisions.

\begin{table}[!h]
    \centering
    % \scriptsize
    \small
    \begin{tabular}{p{0.14\columnwidth}p{0.3\columnwidth}p{0.36\columnwidth}}
    \toprule
        Entity Type & Definition & Example(s)  \\
        \hline
        \textit{keyword} & A law keyword is the description selected from the statutory provisions, it can be either a word or a phrase. & \textit{Defendant; Extremely Huge Amount}\\
        \hline
        \textit{punishment} & A law punishment represents sanction in the legal norm. & \textit{Fixed-Term Imprisonment; Fine; Detention}\\
        \hline
        \textit{numerical} & An law numerical represents a limit on the amount of a description or punishment. & \textit{>= Twenty Thousand Yuan; <= Ten Years; <= Five Hundred Thousand Yuan}\\
        \hline
        \multirow{2}{*}{\textit{Logic}} & \textit{And} denotes the conjunction of all child nodes.  & \textit{AND ${\longleftarrow}$ (1) Fixed Term Imprisonment; (2) Fine.} \\
        \cline{2-3}
        & \textit{Or} denotes the disjunction of the child nodes. & 
        \textit{OR {${\longleftarrow}$} (1) Using Fake Certification Documents; (2) Using Fake Economic Contracts.}\\
        \bottomrule
    \end{tabular}
    \caption{Node types in \lawgraph{}}
    \label{apptab:L_node_t}
    % \vspace{-1.5em}
\end{table}

\begin{table}[!h]
    \centering
    % \scriptsize
    \small
    \begin{tabular}{p{0.1\columnwidth} p{0.4\columnwidth} p{0.3\columnwidth}}
    \toprule
        Relation & Definition & Example\\ 
        \hline
        \textit{Do} & The most general relation in the hypothesis part. It connects illegal conduct and the subject. & \textit{A \raisebox{-0.6ex}{${\stackrel{Do}{\longrightarrow}}$} defraud}\\
        \hline
        % \textit{Effect} & It represents the connection between the object of illegal conduct and the illegal conduct itself. & \textit{stolen \raisebox{-0.6ex}{${\stackrel{Effect}{\longrightarrow}}$} B's money}\\
        % \hline
        \textit{Default} & There is no meaning of the edge. It connect the component nodes of logical nodes & \textit{AND ${\longleftarrow}$ (1) Fixed Term Imprisonment; (2) Fine.}\\
        \hline
        \textit{Satisfy} & The relation connects the hypothesis and sanction part. It is only used when the law mentions it.  &  \textit{Hypothesis \raisebox{-0.6ex}{${\stackrel{Satisfy}{\longrightarrow}}$} Sanction} \\
        \hline
        \textit{NextDo} &
        It connects two conducts, which follow the time order. & \textit{drive \raisebox{-0.6ex}{${\stackrel{NextDo}{\longrightarrow}}$} escape}\\
        \hline
        \textit{LeadTo} &
        It connects two conducts, the second caused by the first. & \textit{escape \raisebox{-0.6ex}{${\stackrel{LeadTo}{\longrightarrow}}$} victim died}\\
        \hline
        \textit{Modify} &
        A is a modifier, which can be a word, phrase, or clause, for B that provides more information. & \textit{defraud \raisebox{-0.6ex}{${\stackrel{Modify}{\longrightarrow}}$} purpose}\\
        \hline
        \textit{Constrain} &
        A is a condition or constrained words or phrases for B. ``A" typically defines or limits the scope of ``B" through specific stipulations or criteria. & \textit{bank \raisebox{-0.6ex}{${\stackrel{Constrain}{\longrightarrow}}$} defraud} \\
        \bottomrule
    \end{tabular}
    \caption{Edge types in \lawgraph{}.}
    \label{apptab:edge}
    % \vspace{-2em}
\end{table}

Each criminal statutory provision contains specific conduct, such as loan fraud in Table~\ref{tab:three}. The \lawgraph{} construction is also related to the key conduct; all the destinations of the edges focus on the conduct. For instance, \textit{Do} is related to the subject of the conduct, \textit{NextDo} connects the sequential conducts following the key conduct, \textit{LeadtDo} highlights the results caused by the key conduct, and \textit{Modify} and \textit{Constrain} introduce the restrictions and modifications of the conduct (including time, etc.). Therefore, there are no limiting factors making the definition suitable only for a small range of provisions.