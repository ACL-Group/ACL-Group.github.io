\section{Details of Interpretability Analysis}
\label{sec:c}
In this section, we show how to use Large Language Models (LLMs) and \lawgraph{} to explain the predicted results and enhance the reliability of the proposed models.

\paragraph{Overview} There are two steps in explaining the decision. First, we need to find the corresponding disposition information through the \lawgraph{s}. According to the design of \lawgraph{} and the Legal Norm theory, several prison term ranges exist in statutory provisions, and each range has a corresponding description of the severity of the crime, which is represented by the disposition subgraphs in \lawgraph{s}. After extracting the related disposition subgraphs, we can utilize LLMs to find the real case facts that align with the disposition. Then, the matched facts become the reasons for the decision. 

\paragraph{Step 1: Disposition sub-graphs matching.} After using the proposed methods to predict results, we can categorize the predicted terms into one of the sanction subgraphs. In the \lawgraph{s}, each sanction subgraph has a corresponding disposition subgraph. The nodes in the related disposition subgraph include the possible legal reasons why the case reached the predicted prison terms. As the \lawgraph{s} are well-structured, it is easy to extract this information.

For instance, if the proposed models predict a loan fraud case with a 6-year imprisonment term, according to the \lawgraph{} of the Crime of Loan Fraud, shown in Figure~\ref{Law-Graph-Case-study}, we can deduce the following information:
\begin{itemize}
    \item There are three sanction subgraphs representing less than five years imprisonment, five to ten years imprisonment, and more than ten years fixed-term imprisonment or life imprisonment.
    \item The sanction subgraph of less than five years imprisonment is linked to a disposition subgraph with ``Large Amount''.
    \item The sanction subgraph of five to ten years imprisonment is linked to a disposition subgraph with ``Huge Amount'' or ``Other Serious Circumstances''.
    \item The sanction subgraph of more than ten years fixed-term imprisonment or life imprisonment is linked to a disposition subgraph with ``Extremely Huge Amount'' or ``Other Especially Serious Circumstances''.
\end{itemize}

As the predicted prison term is 6 years, it aligns with the sanction subgraph of five to ten years imprisonment and the disposition subgraph with ``Huge Amount" or ``Other Serious Circumstances". Therefore, we can infer that this case was decided with a 6-year term because the crime amount was huge or had serious circumstances.

\paragraph{Step 2: LLMs analysis.} After the first step, we know the possible legal reasons for the decision. Then, we can use LLMs to find more detailed decision reasons related to specific cases. We designed the following prompt:
\begin{quote}
    Case Facts: [\textit{Case Describtions}]\\
    Prediction of the sentence duration is [\textit{Predicted Term}] months. The statutory sentencing circumstance involved is [\textit{Disposition Subgraph}]. Please identify the specific situation corresponding to this case. If yes, provide specific details; if no, respond with ``no match''.
\end{quote}

Continuing the example in Step 1, the prompt is filled as follows:
\begin{quote}
    Case Facts: \textit{In October 2012, the defendant in collusion with others, used fake car purchase contracts, down payment receipts, income certificates, and other materials to fraudulently obtain a loan of more than 190,000 RMB from a branch of the Industrial and Commercial Bank of China through a company, under the pretense of arranging a car loan. The loan was used to purchase gold, mobile phones, and other items. Subsequently, the defendant changed personal contact information and did not repay the loan. On April 16, 2013, after the defendant was arrested and brought to justice, defendant's relatives repaid the embezzled funds, amounting to 196,000 RMB.}\footnote{The case is cited from China Judgment Online. All personal information is hidden.}\\
    Prediction of the sentence duration is \textit{72} months. The statutory sentencing circumstance involved is \textit{Huge Amount or Other Serious Circumstances}. Please identify the specific situation corresponding to this case. If yes, provide specific details; if no, respond with ``no match''.
\end{quote}

Then, the prompt can be used for LLM analysis. The output of the LLM is the related fact that aligns the disposition information. Therefore, the highlighted fact can be used as the reason for the decision. For example, the output of the LLM is:
\begin{quote}
    \textit{Yes, the specific situation is: the defendant in collusion with others, used fake materials to defraud the Branch of the Industrial and Commercial Bank of China of a loan amounting to more than 190,000 RMB, which constitutes a "huge amount".}
\end{quote}

Based on this, we can understand why each case's decision is made using our proposed methods. In this example, the defrauded amount exceeds 190,000 RMB, satisfying the "huge amount" criterion in the statutory provision.

\paragraph{Quality Analysis} To further understand the correctness of the proposed reliability analysis, we designed an experiment based on randomly chosen 50 cases for each crime in Section~\ref{es}. Then, we utilized ChatGPT-4 to perform the reliability analysis using the above steps for finding the decision reasons. After completing the analysis, we asked a legal expert to categorize the analysis output as either correct or incorrect. The statistical results show that the proposed analysis solution achieves 96\%, 94\%, and 92\% accuracy for \textit{fraud}, \textit{intentional injury}, and \textit{arson} crimes, respectively. The high accuracy demonstrates that this process for explaining the reasons behind the predicted decisions is effective.

We also provide some examples from the interpretability analysis in Table~\ref{tab:rex}\footnote{The all analyzed cases will be released after acceptance. During the review process, we have uploaded them as supplementary material}.

\begin{table*}[]
    \centering
    \begin{tabular}{p{0.13\columnwidth}|p{1.8\columnwidth}}
    \toprule
    Type &  Content\\
    \midrule
    \multirow{2}{*}{Prompt} &  Case Fact: On August 6, 2014, around 3 PM, the defendant asked his father for money to buy cigarettes. Dissatisfied with the 20 RMB his father gave him, he got into a dispute with him. To vent his frustration, the defendant set fire to his father's house, burning it down. According to the Fengcheng City Price Center, the destroyed house was valued at 10,000 RMB. After the incident, the public security authorities arrested the defendant, and he truthfully confessed to his crime after being brought to justice.\\
    & The predicted sentence is 36.29 months, and the statutory sentencing circumstance involved is that no serious consequences have yet been caused. Please identify the specific circumstance corresponding to this case. If there is one, provide the specific situation; if not, respond with "not match".\\
    Output &  Yes, the specific situation is: The defendant set fire to his father's house out of dissatisfaction with the amount of money his father gave him. Although the house was completely destroyed by the fire, since no casualties or other serious consequences were involved, the case was classified as "no serious consequences have yet been caused".\\
    \midrule
    \multirow{2}{*}{Prompt}  &  Case Fact: In November 2013, the defendant, together with others, defrauded a young man of 24,000 RMB under the pretence of marriage.\\
    & The predicted sentence is 15.29 months, and the statutory sentencing circumstance involved is a relatively large amount of fraud. Please identify the specific circumstance corresponding to this case. If there is one, provide the specific situation; if not, respond with "not match".\\
    Output & Yes, the specific situation is: The defendant defrauded 24,000 RMB under the pretense of marriage, which meets the statutory sentencing circumstance of "large amount". \\
    \midrule
    \multirow{2}{*}{Prompt}  & Case Fact: From August 10 to September 27, 2011, the defendant falsely claimed to be a logistics staff member. Under the pretence of transferring two box trucks owned by the logistics and handling the related vehicle paperwork, the defendant defrauded the victim of 94,000 RMB in four instalments at various locations. The fraud money was not recovered after the incident. \\
    & The predicted sentence is 48.96 months, and the statutory sentencing circumstance involved is a huge amount of fraud or other serious circumstances. Please identify the specific circumstance corresponding to this case. If there is one, provide the specific situation; if not, respond with "not match".\\
    Output & Yes, the specific situation is: The defendant falsely claimed to be a staff member of a logistics company and defrauded the victim of a total of 94,000 RMB under the pretense of transferring box trucks, which meets the "huge amount". \\
    \midrule
    \multirow{2}{*}{Prompt}  &  Case Fact: On September 23, 2013, at around 5 PM, the defendant had a dispute with the victim over a trivial matter in a store in this city. During the ensuing altercation, the defendant injured the victim's left eye with a metal jewelry stand, causing contusions, traumatic iridocyclitis, optic nerve contusions, and retinal contusions in the left eye. Forensic evaluation classified the injuries as second-degree serious injuries. During the incident, the defendant was also injured. On February 13, 2014, the defendant was apprehended by the police.\\
    & The predicted sentence is 58.31 months, and the statutory sentencing circumstance involved is causing serious injury to another person. Please identify the specific circumstance corresponding to this case. If there is one, provide the specific situation; if not, respond with "not match".\\
    Output & Yes, the specific situation is: The defendant had a dispute with the victim over a trivial matter and resorted to violence, causing the defendant multiple serious eye injuries, which forensic evaluation classified as second-degree serious injuries. This case falls under the statutory sentencing circumstance of "causing serious injury to another person". \\
    \bottomrule
    \end{tabular}
    \caption{Examples of Interpretability Analysis.}
    \label{tab:rex}
\end{table*}