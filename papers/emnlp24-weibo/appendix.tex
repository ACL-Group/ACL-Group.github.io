\section{Details of Data Construction }
\label{apd:data_details}
The dataset includes a personal information table for all users, as shown in ~\tabref{tab:User} (key fields), and also includes a posting information table for each user, as shown in ~\tabref{tab:Post} (key fields). The concepts in the ice-breaking point identification flowchart are shown in ~\tabref{tab:concepts_of_ibkp}.
\begin{table}[th]
	\small
	\begin{tabular}{p{0.25\columnwidth}|p{0.6\columnwidth}}
		\toprule
		\textbf{Field name} & \textbf{Description} \\ \midrule
		User\_id & User's ID \\ \midrule
		User\_name & User's name   \\ \midrule
        Gender & User's gender   \\ \midrule
	\end{tabular}
	\caption{User personal information table.}
	\label{tab:User}
\end{table}
\begin{table}[th]
	\small
	\begin{tabular}{p{0.25\columnwidth}|p{0.6\columnwidth}}
		\toprule
		\textbf{Field name} & \textbf{Description} \\ \midrule
		User\_id & User's ID \\ \midrule
		Bid & Post's ID   \\ \midrule
        Text & User's post body content \\ \midrule
        Date & User's posting date   \\ \midrule
        No. of comments & The number of comments under this post   \\ \midrule
        comments & User ID, username, comment content, and comment time for commenting on this post   \\ \midrule
	\end{tabular}
	\caption{Personal History Posting Situation Table.}
	\label{tab:Post}
\end{table}

\begin{table}[th]
	\small
	\begin{tabular}{p{0.3\columnwidth}|p{0.6\columnwidth}}
		\toprule
		\textbf{Concept} & \textbf{Illustrate} \\ \midrule
		num\_comments & Number of comments \\ \midrule
		comment\_id & Commenter's id   \\ \midrule
        comment\_ids & User id set for commenting on this post \\ \midrule
        post\_id & Poster's id   \\ \midrule
        comment\_his\_ids & All comment user id sets of the poster's historical posts   \\ \midrule
        post\_time & User posting time   \\ \midrule
comment\_time & User comment time   \\ \midrule
crawl\_time & User Crawling time for this post   \\ \midrule
Average Post Frequency Interval Time (APFIT) & (Latest Post - First Post Time)/Total Number of Posts   \\ \midrule
	\end{tabular}
	\caption{The concepts in the ice-breaking point identification flowchart.}
	\label{tab:concepts_of_ibkp}
\end{table}

The categories of posts are defined as follows:
\begin{description}
\item[Emotional expression: ] This category mainly focuses on various emotions directly expressed by users in posts, including happiness, depression, anxiety, loneliness and so on. Users can express and release their emotions by crying complaining and venting their emotions without describing the background reasons in detail.
\item[Daily life sharing: ] This kind of posts mainly focus on trivial matters and hobbies in daily life, such as sharing pictures, greetings, games, stars, music and other topics.
\item[Psychological state reflection: ] This category of posts can directly or indirectly express the psychological state of patients with depression, including symptoms of depression, depression, self-denial, suicidal tendency, and talk behavior. Users tend to share their psychological troubles, pains, anxieties, etc. in posts, seeking listening and support to relieve psychological pressure and obtain psychological support.
\item[Asking for help and support: ] This kind of post clearly expresses the need for substantive help and support, including directly asking questions, asking for help, or seeking comfort.
\end{description}
The categories of replies are defined as follows:
\begin{description}
\item[Emotional support and resonance: ] This category focuses on the respondents’understanding, support and resonance of the emotions of the original post author, providing comfort and encouragement, and making the original post author feel understood and concerned.
\item[Suggestions and solutions: ] This kind of reply provides substantive suggestions, solutions or suggestive support, and the respondent provides possible solutions or suggested actions for the problems or troubles raised in the original post.
\item[Emotional talk and communication: ] This category focuses on the respondents’ personal emotional talk and sharing of the emotional content in the original post. Respondents may share similar experiences or emotions and establish emotional connection
and communication with the original post author.
\item[Encouragement and affirmation: ] This category includes respondents conveying encouragement, affirmation and positive emotions to the original post author, trying to improve the emotional state and self-esteem of the original post author.
\item[Discouraging reply: ] Refers to the way of reply that lacks
deep and emotional connection, with sarcasm or negative hints, encourages negative behavior or is indifferent. This way may aggravate negative emotions and is not conducive to mental health.
\end{description}

\section{Implementation Details of Experimental}
\label{apd:exp_details}
This section shows all the experimental details mentioned in the text. For all the models used above, we set hyperparameters according to our experience and follow the existing implementation and previous ones to achieve the best performance without fine-tuning.
 
\subsection{Icebreaker Generation}
\label{apd:icebreaker_ge}
We use gpt-3.5-turbo to automatically generate icebreakers for 400 posts in the chatbot group using four different strategies. The prompts and examples corresponding to the four strategies are shown in~\tabref{tab:prompt}. Taking the role of a counselor as an example.

\begin{table*}[th]
	\small
	\begin{tabular}{p{0.5\columnwidth}|p{1\columnwidth}|p{0.5\columnwidth}}
		\toprule
		\textbf{Strategy} & \textbf{Prompt} & \textbf{Example}\\ \midrule
		Emotional support \( \mbox{\&} \) resonance & Please reply to this depressed user as a consultant. You need to use the following strategy: generate emotional support and resonance with netizens: focus on the respondent's understanding, support, and resonance with the emotional content of the original post author, and provide comfort or encouragement. In such replies, respondents usually express understanding and support for the emotions of the original author, in order to alleviate their negative emotions and make them feel understood and cared for. Your reply should be within 140 words. & I understand that you are currently feeling very low, and depression does make people feel helpless and hopeless. Every day's tears are a heavy burden, but please remember that you are not alone.  \\ \midrule
		Suggestions \( \mbox{\&} \) solutions &  Please reply to this depressed user in the voice of a psychological counselor. You need to use the following strategies: suggestions and solutions: This type of reply provides substantive suggestions, solutions, or suggestive support. In response to the problems or troubles raised in the original post, the respondent provides possible solutions or suggested actions. Your reply should be within 140 words. & Try to find activities that can bring you joy and satisfaction, such as sports, painting, reading, or volunteer activities. Regularly develop plans and communicate with family or friends. Seeking professional psychological counseling is also one of the help. \\ \midrule
        Emotional talk \( \mbox{\&} \) communication & Please reply to this depressed user in the voice of a psychological counselor. You need to use the following strategies: emotional sharing and communication: This category focuses more on the respondent's personal emotional sharing or sharing of the emotional content in the original post, establishing emotional connections and communication. In this type of reply, the respondent may share similar experiences or emotions, establish resonance and emotional connection with the original post author. Your reply is within 140 words. & I understand your current troubles. Depression and invasive thinking can make people feel frustrated and tired. Remember, you are not alone, many people have experienced similar feelings. \\ \midrule
        Encouragement \( \mbox{\&} \) affirmation & Please reply to this depressed user in the tone of an ordinary person. You need to use the following strategies: encouragement and affirmation: This classification includes the respondent conveying encouragement, affirmation, and positive emotions to the original author, attempting to improve the author's emotional state and self-esteem. Your reply is within 140 words & Dear, first of all, I want to affirm that you have bravely expressed your inner troubles. You are not a lonely warrior, we have all experienced ups and downs. Depression is a disease that deserves attention and treatment. Don't be afraid, seeking professional help is a form of courage. Everyone deserves to have mental health, and so do you. May you find inner peace and strength.  \\ \midrule
	\end{tabular}
	\caption{Prompt under four strategies.}
	\label{tab:prompt}
\end{table*}


\subsection{Classification of Posts and Replies}
\label{apd:classify_details}

We selected 1000 posts and 1000 comments from depressed users from Weibo data and conducted multi label classification experiments on them. There are four categories for posts and five categories for replies, see section. We divide these samples in an 8:2 ratio between the training set and the test set. For this model, the pre-trained model we have chosen is ``ernie-3.0-medium-zh'', with a learning rate of 3e-5, batch size of 32, and epoch of 100. Due to the length of the post being too long, we have set the maximum seq length to 128. To prevent overfitting, we have set an early stop. The F1-score for multiple classifiers in post categories is 80.6\%, and the F1-score for multiple classifiers in reply categories is 62.8\%.

\subsection{Depressed Users Detector}
\label{apd:dd_details}
We select 500 users from Weibo data and select their 10 historical posts, concatenating them with semicolon intervals. Depressed users are labeled as positive, while non-depressed users are labeled as negative. The criteria for labeling depressed users and non depressed users are as follows: manually observe historical posts in the obtained user posts. If there are words such as ``diagnosed with depression'' or ``taking medication for depression'', they will be labeled as positive, and in any case, they will be labeled as negative.
%\KZ{What's the criteria for labeling someone depressed or not?} 
After manual annotation, there are 263 positive samples and 237 negative samples. We divide these samples in a ratio of 6:2:2 between training set, testing set, and validation set. For the model, the pre-trained model we have chosen is ``Chinese-L-12-H-768-A-12'', with a learning rate of 3e-5, train-batch-size of 32, eval-batch-size of 8, and num-epoch of 10. Due to the length of the post being too long, we have set the max-seq-length to 512. The results of depression detection model on the test set are that Precision is 83.3\%, Recall is 94.3\%, and F1-score is 88.5\%.

