\subsection{Selectional Association}
% 18 sents.
% 1. startup
Given a relation group, selectional association step produces the list of relation
schema.
% 2. simultaneously process, showing the result
The schema is represented as type pairs $\langle t1, t2 \rangle$, with its association score.
% 3. given example ?
Take $group("play\ in")$ as an example, the ideal type pairs will contain the pair
$\langle sports.pro\_athlete,\ sports.sports\_league \rangle$ and
$\langle film.actor,\ film.film \rangle$.
% 4.
% 5. Freebase type index, putting entity into types

We define $TList(ent)$ as the list of all types for entity $ent$. Type list can be
extracted by scanning \textit{type.object.type} relation in Freebase.
% 6. Example of type list?
For instance, the type list of "New York City" contains coarse-grained type \textit{location.location}, fine-grained type \textit{location.administrative\_division},
\textit{location.citytown} and other types like \textit{business.business\_location}.

% 6. type set to 1
Each linked tuple $\langle ent1,\ rel,\ ent2 \rangle$ will make a contribution to $\langle t1,\ t2 \rangle$
where $t1 \in TList(ent1)$ and $t2 \in TList(ent2)$. We treat these pairs equally, since it's not trivial to
say which type is related to the tuple.
% 7. forming type pairs
% 8. example type pairs
% 9. count p(r, t)
Combining all tuples in one group, we define the support tuples of a type pair in a group:
\begin{equation}
\begin{aligned}
sup(rPat, &\langle t1, t2 \rangle) = \\
&\{ltup : t1 \in ent1,\ t2 \in ent2 \}
\end{aligned}
\end{equation}

% 10. formula of p(r, t)
\noindent
The joint probability of relation and type pairs is defined as:
\begin{equation}
p(r, tp) = |sup(r, tp)|\ /\ \sum\limits_{r', tp'} |sup(r', tp')|,
\end{equation}

\noindent
where $r$ is the representative pattern, and $tp$ stands for the type pair.

% 11. main score function: Mutual Information between relation and type pairs
The relatedness of a type pair to the relation cannot be measured
by using joint probability alone. Thus, we use mutual information to
measure the relatedness between type pair and relation:
% MI(r, t) = p(r, t) / (p(r, *) * p(*, t))
\begin{equation}
MI(r, tp) = p(r, tp) \log \frac {p(r, tp)}{p(r) p(tp)}
\end{equation}

% 12. get score(r, t)
Finally, for each relation group, we collect all the possible type pairs with
their mutual information as the selectional association score.
We filter pairs with score equal or less than 0, and the remaining type pairs
are the possible schemas for the relation group.
% 13.
% 14. 