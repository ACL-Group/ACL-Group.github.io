\subsection{Relation Grouping}
% 8 sents.
% 1. group tuples together, give definition
A relation group consists of a list of linked tuples sharing similar relation patterns,
along with a pattern representing this group, $group(rPat) = \{ltup_1, ltup_2, ..., ltup_k\}$, where
$rPat$ is the representative pattern.
% 2. same & syntactically similar rel. patterns will be in a group, no overlapping.
Each tuple must be contained in only one group.

% 3. algorithm: syntactic rules to convert tense,
% mainly focus on 3 tense: will/should/must be, be -ing, participle
We define syntactically equivalence between two relation patterns, as both of them can be converted
into the same simple pattern by a list of transformations. 
Every relation pattern in one group is equivalent with each other.
Since adverbs and modal verbs are less important
in finding preference of a relation, we firstly remove these words,
and then perform transformation on the remaining relation.
In ReVerb, relations are verb based, the syntactic rules detect different
tenses, such as continuous tense, participle tense, transforming them into present tense.
If the relation has passive voice, it will be kept in the transformed relation.

% 4. Create a table, showing the rules to find them.
%The detail of syntactic rules is shown in Table 1.
%\KQ{refer to Liang et al., 2014 to build the rule table, containing continuous, participle, be-the-name-of
%and passive form}
% For example, a --> b
% check liang's 14 paper to learn the representation of tables.
% 5. use stanford parser to tokenize & postag.
We use Stanford Parser \cite{klein2003accurate} to perform POS tagging, lemmatizing and parsing on relations.
% 6. representative relation: present tense
After transformation, all linked tuples with the same pattern form a group, and this pattern
is selected as representative pattern.
