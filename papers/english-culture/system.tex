\section{System}
In this section, we firstly introduce the overview of our system for relation schema inferring,
then we present the detail of each step in the system.

\subsection{System Overview}
% 3 main work: entity linking, relation merging, sel. pref.
The workflow of RvSp is shown in Figure 1.
% input: reverb relation, total 3 steps (2 sent)
A {\tt ReVerb} binary relation tuple contains one relation and two arguments, $tup = \langle arg1,\ rel,\ arg2 \rangle$, where
the relation is represented by a lexical pattern, and both arguments are phrases.
RvSp takes relation tuples as input, performs entity linking, relation grouping and
selectional association to translate them into type pairs.

% firstly, entity linking (3 sent)
{\bf1.  Entity Linking:} Relation arguments are linked to Freebase entities by string matching.
Each entity in Freebase has a unique identifier called mid. For example, the mid \textit{m.02\_286} represents the entity
``New York City''.

% secondly, relation grouping (2 sent)
{\bf2. Relation Grouping:} Linked tuples sharing similar relation patterns are grouped together.
A relation group consists of all the supporting tuples, along with one
representative relation pattern, which is generated from all the patterns
within the group.

% thirdly, simultaneously SP (3 sent)
% need to refine at the last sentence.
{\bf3. Selectional Association:} Freebase has provided relations for linking entities to type hierarchy.
For each linked tuple in one relation group, argument entities are transformed into types. The selectional
association model considers both types simultaneously, and produce a list of type pairs as different schemas for the relation group.

