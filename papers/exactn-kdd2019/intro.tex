\section{Introduction}
\label{sec:intro}
%1. What is Top-K recommendation problem?
%2. In real-world there are other recommendation systems in form of Exact-K. Give some examples.
%3. The difference between Top-K and Exact-K recommendation.
%4. Top-K recommendation problem is well-studied in IR and our work is the first to pay attention to Exact-K recommendation.
%5. Our contributions
The explosive growth and variety of information (e.g. movies, commodities, news etc.) available on the web frequently overwhelms users,
while Recommender Systems (RS) are valuable means to cope with the information overload problem. 
RS usually provide the target user with a list of items,
which are selected from the overwhelmed candidates
in order to best satisfy his/her current demand.
%The key problem of RS lies in how to generate users' most preferred item list.
In the most traditional scenarios of RS especially on mobiles,
recommended items are shown in a waterfall flow form,
i.e. users should scroll the screen and items will be presented one-by-one.
%\KZ{Not sure what the following means:  
Due to the pressure of QPS (Query-Per-Second) for users interacting with RS servers, it is common to return a large amount of ranked items
(e.g. 50 items in Taobao RS)
based on CTR (Click-Through-Rate) estimation for example\footnote{Here we take CTR as an example, other preference score can also be used, e.g. Movie Rating, CVR (Conversion-Rate) or GMV (Gross-Merchandise-Volume) etc.} and present them from top to bottom.
That is to say we believe the top ranked items take the most chance to be clicked or preferred so that when users scroll the screen and see items top-down,
the overall clicking efficiency can be optimized.
It can be seen as \emph{top-K} recommendation \cite{cremonesi2010performance}, because the ranking of item list is important.
%}
\begin{figure*}[th]
	%\vspace{-10pt}
	\centering
	\epsfig{file=figures/exactk_demo.eps, width=1.75\columnwidth}
	%\vspace{-10pt}
	\caption{Show cases for exact-K recommendation in Taobao and YouTube.}
	\label{fig:exactk_demo}
	%\vspace{-10pt}
\end{figure*}

%\KZ{I think you don't have to start the intro with the top-1 ranking RS, but instead go directly to k-item set recommendation directly. Just give enough
%motivation why this problem is important. Later on, you can say how this
%problem differs from the conventional 1-item recommendation. That is,
%it actually generalizes the more common 1-item recommedation.}
However in many real-world recommendation applications,
%taking the popular RS of \emph{Taobao}\footnote{\url{https://www.taobao.com/}} and \emph{YouTube}\footnote{\url{https://www.youtube.com/}} for example (illustrated in Figure \ref{fig:exactk_demo}),
%system will return exact $K$ items with constraints and show them once all to users.
%That is to say users should not scroll the screen 
%and combination of the $K$ items is shown as a whole \textbf{card}.
%We call it \emph{Exact-K} recommendation problem,
%it's novel and practical,
%the goal of such task is to maximize 
%the chance of clicked or satisfied by a user to the whole card.
%In the most time, various combinations of items in a card may take different chances to be satisfied given a user.
%And meantime items in a card may maintain some \textbf{constraints} between each other to guarantee the user experience of recommendation system
%taking two popular RSs in the homepages of Taobao\footnote{\url{https://www.taobao.com/}} and YouTube\footnote{\url{https://www.youtube.com/}} for example
%(illustrated in Figure \ref{fig:exactk_demo}),
%\KZ{From these two examples, I don't really see the logical relations or
%constraints among the k-items. They don't seem to complement each other. 
%So I can't see the difference between this type of RS and the more 
%traditional 1-item recommendation.}
exact $K$ items are shown once all to the users. 
In other words, users should not scroll the screen and the combination of $K$ items is shown as a whole \textbf{card}. 
Taking two popular RS in the homepages of Taobao
%\footnote{\url{https://www.taobao.com/}}
and YouTube
%\footnote{\url{https://www.youtube.com/}}
for example
(illustrated in Fig. \ref{fig:exactk_demo}), they recommend cards with exact 4 commodities and 6 videos respectively.
%In Taobao,
%categories or titles of these 4 commodities are constrained to be diverse, in order to form a complementary combination.
Note that items in the same card may interact with each other, e.g. in Taobao, co-occurrence of ``hat'' and ``scarf'' performs better than ``shoe'' and ``scarf'', but ``shoe'' and ``scarf'' can be optimal individually.
We call it \emph{exact-K} recommendation, 
whose key challenge is to maximize the chance of the whole card being clicked or satisfied by the target user. 
Meanwhile, items in a card usually maintain some \textbf{constraints} between each other to guarantee the user experience in RS,
e.g. the recommended commodities in E-commerce should have some diversity rather than being all similar for complement consideration.
%\yu{, so the constraints can be difference of categories or smaller similarity of titles
%	between every two items in a card.}
In a word, top-K recommendation can be seen as a ranking optimization problem which assumes that ``better'' items should be put into top positions,
while exact-K recommendation is a (constrained) combinatorial optimization problem
which tries to maximize the joint probability of the set of items in a card.
%(i.e. items may interact between each other in the same card).

%While top-K recommendation is well studied for many years in IR 
%research community, and listwise models are the most related to our problem as they optimize ranking considering the whole items list, but they either target on ranking refinement or don't consider the constraints in ranking list, which will fall into sub optimal towards exact-K recommendation (refer to Sec. \ref{sec:ltr} for more discussions).
%Our work mainly focuses on solving exact-K recommendation problem end-to-end,
%and the main contributions can be summarized as follows.
Top-K recommendation has been well studied for decades in information retrieval (IR) research community. Among them, listwise models are the most related to our problem as they also perform optimization considering the whole item list. However, they either target on ranking refinement or do not consider constraints in the ranking list, which will fall into sub-optimal towards exact-K recommendation (refer to Sec. \ref{sec:ltr} for more discussions). Our work mainly focuses on solving exact-K recommendation problem end-to-end, and its main contributions can be summarized as follows.
\begin{enumerate}[(1)]
\item We take the first step to formally define the exact-K recommendation problem and innovatively reduce it to a Maximal Clique Optimization problem based on graph. %which is NP-hard.
\item %To solve the Maximal Clique Optimization problem,
To solve it, we propose \emph{Graph Attention Networks} (GAttN) with an Encoder-Decoder framework which can end-to-end learn the joint distribution of $K$ items and generate an optimal card containing $K$ items.
Encoder utilizes Multi-head Self-attention to encode the constructed undirected graph into node embeddings considering nodes correlations. Based on the node embeddings, decoder generates a clique consisting of $K$ items with RNN and attention mechanism which can well capture the combinational characteristic of the $K$ items.
Beam search with masking is applied to meet the constraints.
Then we adopt well-designed \emph{Reinforcement Learning from Demonstrations} (RLfD) which combines the advantages in behavior cloning and reinforcement learning, making it sufficient-and-efficient to train GAttN.
\item We conduct extensive experiments on three datasets (two constructed from public MovieLens datasets and one collected from Taobao).
Both quantitative and qualitative analysis justify the effectiveness and rationality of our proposed \emph{GAttN with RLfD} for exact-K recommendation.
Specifically, our method outperforms several strong baselines with significant improvements of 7.7\% and 4.7\% on average in Precision and Hit Ratio respectively. 
%\KZ{This evaluation is on the CTR? I think that is the end-to-end
%5evaluation that's more effective because you proposed a new way of
%recommending items.}
\end{enumerate}
