\section{Related Work}

There has been a large amount of recent work on information extraction from web page, though none that pays attention to automatically extract list items from web page.

MDR \cite{LiuGZ03:MDR} is a novel system proposed to automatically mine all the data records in a given Web page. The method is based on the observation that a group of similar data records being placed in a specific region is reflected in the tag tree by the fact that they are under one parent node. There are three steps in this method. The first one is building a DOM tree for the given web page. The second step is mining data regions in the page using DOM tree and string comparison. The final step is identifying data records from each data region. The method is quite intuitive but it  has some new features such as discovering non-contiguous data records and high efficiency.

Gengxin Miao et al. \cite{MiaoTHSM09:TagPathClustering} introduce a new method for record extraction that captures a list of objects based on a holistic analysis of a Web page. 

Yudong Yang et al. \cite{YangZ01:VisualCues} present a visual cue based approach to the extraction of the semantic structure of HTML documents.
Because web pages are normally composed for viewing in visual web browsers and lack information on semantic structures, this approach brings semantic information to the traditional DOM tree.
This method can be applied to adaptive content delivery system and search engine to extract useful information from the whole web page.

Chia-Hui Chang et al. \cite{ChangL01:IEPAD} introduce IEPAD, a system that automatically discovers extraction rules from Web pages. The motivation of the method is from the observation that useful information in a Web page is often placed in a structure having a particular alignment and order. After encoding token string of web pages, IEPAD system uses PAT tree for pattern discovery. The discovered maximal repeats are further filtered by two measures:regularity and compactness. IEPAD system is quick and effective.

Kristina Lerman et al.\cite{Lerman01:AutomaticData} describe a technique for extracting data from lists and tables. The approach proposed by these article develops a suite of unsupervised learning algorithms that induce the structure of lists by exploiting the regularitities both in the format of the pages and the data contained in them. There are three steps in this method: Extract all data from lists, Indentify columnsand Indentify rows. One limitation of this approach is that it requires several pages to be analyzed before data can be extracted from a single list.

Wrapper induction \cite{AshishK97:WrapperGeneration}\cite{BaumgartnerFG01:Lixto}\cite{MeccaCM01:RoadRunner} is instructive to our work. We present these three typical wrapper induction system here. They talk about the information extraction from three different perspective.

Naveen Ashish et al. \cite{AshishK97:WrapperGeneration} introduce a method for semi-automatically generating wrappers which can be used to provide database-like querying for semi-structured WWW sources. The key idea used by this article is to exploit the formatting information in pages and heuristics from designer.

Robert Baumgartner et al.\cite{BaumgartnerFG01:Lixto} present a new techniques for supervised wrapper generation and automated web information extraction.
The lixto system based on these techniques has three modules. The Interactive Pattern Builder module requires user to provide an extraction pattern from interactive UI and then generates Elog program which is a specification for the actual extraction system.
The Extractor module is the Elog program interpreter. The Extractor module generates a pattern instance base, a data structure encoding the extracted instances as hierarchically ordered trees and string, as its output. Finally the XML generator performs translation from the extracted pattern instance base to XML. The methods and system mentioned by this paper have distinctive features such as it is easy to learn and use,it only depends on the sample web page and so on.

RoadRunner \cite{MeccaCM01:RoadRunner} develops a novel technique to compare HTML pages and automatically generate a wrapper based on their similarities and differences. RoadRunner does not rely on priori knowledge and user interaction and sample web pages.

 