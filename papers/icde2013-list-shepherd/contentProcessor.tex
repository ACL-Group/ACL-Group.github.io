\subsection{Content Processor}
The content processor takes as input a ``top-$k$'' list and
extracts the main entities as well
as their attributes.
%normalized and conceptualized ``top-k list'' to the output.
%It has two major tasks:
Sometimes the text within an HTML text node contains a structure itself, e.g.
``Hamlet By William Shakespear''. The content processor infers the structure of
the text \cite{Fisher08:dirttoshovels} by building a histogram for
all potential separator tokens such as ``By'', ``:'' and ``,'' from all the items
of the ``top-$k$'' list. If we identify a sharp spike in the histogram for a
particular token, then we successfully find a separator token, and we use that
token to separate the text into multiple fields.

It is useful provide names to the extracted attribute values. For example,
we want to infer ``name'', ``image'', and ``Wikipedia link'' as
attribute names from the list in Figure \ref{fig:topscientists}.
To do this, we conceptualize the extracted columns \cite{Song11:Conceptualize},
using Probase and a Bayesian model.
%who utilized Probase \cite{WuLWZ12:Probase} as knowledgebase and
%developed a short text understanding system based on Bayesian model.
In addition, for special columns like indexes, pictures and long paragraphs,
we apply specified rules to conceptualize them.

