\section{Implementation Details}
\label{sec:imp}

To build the CRF model of Title Classifier, we used a training data
set with 4000 positive and 2000 negative samples.
In this data set, all negative and 50\% of positive samples are real
web page titles from a fragment of Bing snapshot $T_{1}$,
while the remaining samples are synthesized (see \ref{sec:titleDataSet}).
To generate POS tags and lemma features,
we used the \emph{Stanford Part-Of-Speech Tagger}~\cite{toutanova2003feature},
which is a maximum-entropy tagger for English.  % In addition, to optimize the
% accuracy of the tagger, we need to filter ill-formatted writing in the
% title and lowercase all the words before generating the model pattern.

The HTML Parser we use is the Winista HtmlParser \cite{winista}. 
%It is a popular HTML parser written in C\#, and provides very high accuracy
%and efficiency.  
We filter out unwanted lists according to a black list of
tags, including \emph{$<$head$>$, $<$link$>$, $<$style$>$, $<$form$>$,
  $<$iframe$>$, and $<$input$>$}.

For the Top-K Ranker, we propose two approaches, which are labeled as
\emph{rule-based} and \emph{learning-based}, respectively in
Section \ref{sec:eval}.
For the \emph{learning-based} ranker, we build a training set as
follows. \ZZX{First, we use the original system with the
\emph{rule-based} ranker to process web pages from Bing fragment $T_{1}$.}
From the result set, we
select 1000 top-$k$ pages from which top-$k$ lists can be correctly
extracted.  The extracted lists are labeled as positive cases,
while the rest of the candidate lists are labeled as negative ones, 
as our basic assumption is that one top-$k$ page only contains one top-$k$
list.  Then training data set thus contains 1000 positive lists and
2000 negative lists.

As for Content Processor, we use the \emph{Stanford Named Entity
Recognizer}~\cite{finkel2005incorporating} to detect date and
location entities.  The recognizer uses a CRF sequence model,
together with well-engineered features for Named Entity Recognition
in English. 
%Both \emph{Stanford Named Entity Recognizer} and 
%\emph{Stanford Part-Of-Speech Tagger} are components of 
%\emph{Stanford CoreNLP}, a state-of-art toolkit for general NLP tasks.
