\section{Problem Definition}
\label{sec:problem}
%Let $s$ be a string which represents a number either in alphabetic or
%numeric form. We define a function $num : string \rightarrow int $
%which converts the string into the corresponding integer.
%
%Let $D$ be the domain of all DOM trees, $N$ be the domain of all DOM nodes.
%We define function $text : N \rightarrow string$ to return the text
%content of a DOM node.
%We define a function $tp : D \times N \rightarrow [N] $ which returns
%the path (list of tag nodes) between the root of a DOM tree
%and a node in the tree.
%
%Given a DOM tree $d \in D$, we define an equivalence class
%\[N_x (d) = \{ n~ |~ n \in leaves (d) \land tp(d, n) = x\}\]
%to be the set of all leave nodes with the same path $x$ to root.
%
%We further define a boolean function
%$isa: string \times string \rightarrow bool$ to test whether
%the first string is an instance of the second string in an is-a taxonomy.

In this section, we formally define the problem of extracting top-k
lists from the web.

Let a web page be a pair ($t$, $d$) where $t$ is the page title, and
$d$ is the HTML body of the page.
%Let the set of all web pages be $P = T \times D$, where
%$T$ is domain of all page titles and $C$ is the domain of
%all DOM tree representations.
A page ($t$, $d$) is a top-$k$ page if:
\begin{enumerate}
\item from title $t$ we can extract a 5-tuple ($k$, $c$, $m$, $t$,
  $l$) where $k$ is a natural number, $c$ is a noun-phrase concept
  defined in a knowledge base such as the one described in
  Section~\ref{sec:prelim}, $m$ is a ranking criterion, $t$ is
  temporal information, $l$ is location information. Note that $k$ and
  $c$ are mandatory, while $m$, $t$, and $l$ are optional.
\item from the page body $d$ we can extract $k$ and only $k$ items
  such that:
\begin{enumerate}
\item each item represents an entity that is an instance of the
  concept $c$ in an is-a  taxonomy;
\item the pairwise syntactic similarity of the $k$ items is greater
than a threshold.
\end{enumerate}
\end{enumerate}

Here, the syntactic similarity is a function that measures the
syntactic (defined for example as entity's position in the DOM tree
of the page) closeness between two terms.

For example, suppose $t$ is ``Twelve Most Interesting Children’s Books in
USA'', we can extract $k = 1$, $c = $ ``children's books'',
$m = $ ``interesting'', $t = null$ and $l = $ ``USA''. If
the body of the page contains exactly 12 similar elements
such as ``Harry Potter'' and ``Alice in Wonderland'',
then we can conclude this is a top-$k$ page.

%\begin{enumerate}
%\item $\exists s_1 \subset t: num(s_1) = k$;
%\item $\exists s_2 \subset t: s_1 \cap s_2 = \emptyset$;
%\item  $\exists x: |N_x(d)| = k$; and
%\item $\forall n \in N_x(d): isa(text(n), s_2)$.
%\end{enumerate}
%
The top-$k$ extraction problem can then be defined as three sub-problems
(in terms of three functions):
\begin{enumerate}
\item {\em Title recognition} $tr : (t, d) \rightarrow (k, c, m, t, l)$
%where $k$ is an integer (number of list items),
%$c$ is a noun-phrase concept, $\alpha$ is an adjective modifier,
%$\tau$ is an optional temporal modifier and $\sigma$ is an optional
%spatial modifier.
\item {\em List extractor} $le : (k, c, d) \rightarrow \mathcal{I}$ where
$\mathcal{I}$ is the set of terms which are instances of
$c$ and $|\mathcal{I}| = k$
\item {\em Content extractor} $cr : (c, d, \mathcal{I}) \rightarrow
(\mathcal{T},~ S)$ where $\mathcal{T}$ is a table of attribute values for
the elements in $\mathcal{I}$ and $S$ is its schema.
\end{enumerate}

%such as $\mbox{tr}(t^+) = (k, \kappa, \alpha,
%\tau, \sigma)$ and $\mbox{le}(c^+) = \mathcal{L}$ and $|\mathcal{L}| = k$.
%Here $k$ is an integer,
%
%We define a general web page as a tuple $p = \langle t, c \rangle$,
%where $t$ is a string representing the page title, and $c$ is the page content in DOM tree representation.
%Then we can define $P = \{ p \}$
%$T = \{ t\}$ and $C = \{c\}$
%
%As is described before, a top-$k$ page is a special case of web page,
%we can define it as $p^{*} = \langle t^{*}, c^{*} \rangle$.
%where $t^{*}$ is the top-$k$ title,
%a \emph{proper} title for a top-$k$ page,
%and $c^{*}$ is the DOM tree that contains the top-$k$ list content.
%Similarly, we can define $P^{*} = \{ p^{*}\}$,
%$T^{*} = \{ t^{*}\}$ and
%$C^{*} = \{ c^{*}\}$.
%
%%Pay attention to the difference between ``top-$k$ titles'' and ``titles of top-$k$ pages''.
%%The set of the latter can be defined as $T^{**} = \{t | \exists p \in P^{*}, t = p.t \}$,
%%which should be a subset of $T^{*}$.
%%This is because a top-$k$ title need not be corresponding to any top-$k$ page
%%(it can even be synthetic and not corresponding to any existing web page).
%
%For any $t^{*}$, we have 5-tuple $\mathcal{T}^{*} =$
%$\langle k$, \emph{concept, criterion, temporal, spacial} $\rangle$.
%where $k$ is a unsigned integer, and other fields are string.
%\emph{Temporal and spacial} and be \emph{null} as some titles miss such information.
%For example, for $t^{*} = $ ``Twelve Most Interesting Children's Book in USA'',
%$\mathcal{T}^{*} = \langle12$, \emph{Children's Book, Most Interesting, null, in USA} $\rangle$.
%More generally, for any title $t$,
%we have $\mathcal{T} = \mathcal{T}^{*} |$ \emph{null},
%where \emph{null} indicates $t \notin T^{*}$.
%
%A top-$k$ list is the structured representation extracted from
%a top-$k$ page content $c^{*}$.
%It can be formally defined as a 2-tuple
%$\mathcal{L}^{*} = \langle$ \emph{meta, table} $\rangle$.
%$\emph{meta} = \langle h_{1}, h_{2}, ..., h_{n} \rangle$, where
%$n$ is number of attributes of each list,
%and $h_{i}$ is the $i$th attribute name (table head).
%$\emph{table} =  \langle e_{1}, e_{2}, ..., e_{k} \rangle$,
%where $k$ is the number of list items,
%and $e_{i}$ is the $i$th list item.
%Each list item can be further defined as a $n$-tuple
%$e_{i} = \langle a_{1}, a_{2}, ..., a_{n} \rangle$
%where $a_{j}$ is the value of the $j$th attribute corresponding to $\emph{meta}$.
%For a general page content $c$, we have $\mathcal{L} = \mathcal{L}^{*} |$ \emph{null},
%where \emph{null} indicates $c \notin C^{*}$.
%
%The top-$k$ extraction problem is to design a system $\mathcal{S}: p \rightarrow r$,
%where the input $p$ is a web page,
%and the output $r = \langle \mathcal{T}, \mathcal{L} \rangle$.
%
%Furthermore, we can divide $\mathcal{S}$ into three subsystem:
%\begin{enumerate}
%  \item \emph{Title recognition $\mathcal{S}_{1}: p.t \rightarrow r.\mathcal{T}$},
%  where the input is a page title and the output $\mathcal{T}$ is the corresponding 5-tuple representation.
%  \item \emph{List extraction $\mathcal{S}_{2}: \langle r.\mathcal{T}, p.c \rangle \rightarrow r.\mathcal{L}.$\emph{table}},
%  where the input is $\mathcal{S}_{1}$'s output and the page content, the output is the content table of the top-$k$ list
%  (without meta data).
%  \item \emph{List content understanding $\mathcal{S}_{3}: r.\mathcal{L}.$\emph{table}$ \rightarrow r.\mathcal{L}.$\emph{meta}},
%  where the input is $\mathcal{S}_{2}$'s output (the content table), the output is the meta data of the top-$k$ list.
%\end{enumerate}


%%% Local Variables:
%%% mode: latex
%%% TeX-master: "paper"
%%% End:
