\subsection{Top-K Ranker}
\label{sec:ranker}

When there are multiple candidate lists,
we select only one of them as the {\em main list}.
Intuitively, the main list is the one that best matches the title.
In Subsubsection \ref{sec:title}, we extract a set of concepts from
the title, and one of them should be the central concept of the top-$k$ list.
Our key idea is that one or more items from the main list should be instances
of one of the concepts extracted from the title. For example, if the title
contains the concept ``scientist'', then the items of the main list should
be {\em instances} of the ``scientist'' concept. The Probase taxonomy provides
large number of concepts and their instances. 
For instance, ``scientist'' concept has 2054 instances in Probase.
%Considering the fact that Probase cannot cover all the instances and
%concepts in the world,
We calculate the score of each candidate list $L$ as:

\[Score(L)= \frac{1}{k} \sum_{n \in L} \frac{LMI(n)}{Len(n)}\]
where $LMI(n)$ is the word count of the longest matched
instance in the text of node $n$,
while $Len(n)$ means the word count of the entire text in node $n$.

If there is a tie in $score(L)$, we prefer the list with the largest
{\em visual area} in the page.
The visual area is estimated by calculating text area
of the candidate list:

\[Area(L)= \sum_{n \in L} (TextLength(n)\times FontSize(n)^2).\]

%After we know the main list, we can also get attribute lists that
%are interleaved with the main list.
