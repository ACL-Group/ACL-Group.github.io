\vfill\eject
\section{Related Work}
\label{sec:related}

\subsection{Topic Model}
\label{sec:relatedtm}
Information Retrieved techonology has been utilized in mining the source code repositories. \cite{kawaguchi2006mudablue} and \cite{kuhn2007semantic} applied LSI to categorize and identify the topics of the source code repositories. \cite{kawaguchi2006mudablue} use source code identifiers but ignore comments. While \cite{kuhn2007semantic} concentrated on clustering software artifacts(like methods and files) as topics, by exploiting source code identifiers and comments. Moreover, they figure out a very clear and nice way call Distribution Map to visulize the result of semantic clusters of the source code in their case studies.

Latent Dirichlet Allocation has been widely used in Nature Language Processing field, and the applications\cite{baldi2008theory, linstead2007mining, lukins2008source, maskeri2008mining} in software artifacts also proved to be successful.

\subsection{Taxonomy}
\label{sec:relatedtax}
Most of the automatic taxonomies are constructed from Wikipedia, as it's the largest hierarchical and up to date corpus in the Internet, which is very friendly accessible.

\cite{liu2012automatic} develops a Bayesian approach to build a hierarchical taxonomy for a given set of keywords using a general purpose knowledgebase and keyword search to supply the required knowledge and context.
\cite{ponzetto2009large} takes a step further to combine the WordNet\cite{miller1995wordnet} and Wikipedia Category system by restructing and integrating Wikipedia based on WordNet.
\cite{ponzetto2007deriving} takes advantage of both connectivity in the page network and lexicosyntactic rules to automatically build a taxonomy from Wikipedia. 
\cite{kotlerman2011support} aims at constructing a domain-specific taxonomy about movie tags.