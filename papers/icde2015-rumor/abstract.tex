\begin{abstract}
This paper studies the problem of automatic detection of false rumors on
Sina Weibo, the popular Chinese microblogging social network. 
Traditional feature-based approaches extract features 
from the false rumor message, its author, as well as
the statistics of its responses to form a flat feature vector. This
ignores the propagation structure of the messages and has not achieved
very good results. We propose a graph-kernel based hybrid SVM classifier
which captures the high-order propagation patterns in addition to
semantic features such as topics and sentiments.
The new model achieves a classification accuracy of 91.3\% on randomly
selected Weibo dataset, significantly higher than 
state-of-the-art approaches.
Moreover, our approach can be applied at the early stage of rumor
propagation and is 88\% confident in detecting an average 
false rumor just 24 hours after the initial broadcast.
\footnote{Kenny Q. Zhu, the corresponding author, is partially 
supported by NSFC Grants 61033002, 61100050 and 61373031, 
and Google Facult Research Award.}
\end{abstract}
