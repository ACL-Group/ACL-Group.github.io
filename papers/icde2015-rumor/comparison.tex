\subsection{End-to-end false rumor detection}
We compare our hybrid SVM classifier
%($\gamma = 2^{-11}, cost = 2^{13}, 2\sigma^2=3, \rho=0.1$)
 with two other state-of-the-art
false rumor detection algorithms\cite{castillo2011information,yang2012automatic}.
Castillo's J48 decision tree is implemented using the 15 best reported
features under WEKA; Yang's SVM classifier was implemented using all 19
reported features except locations of the messages which are not available in
our data. The location feature was shown to be not particularly useful
in Yang's system anyway.
%Two features in Yang which are not used by us are:
%user's avatar which was extracted using face
%detection package xxx and whether username is a person or organization,
%which is done by yyy.
We also train an SVM classifier with only graph kernel
($\beta=1$) as baseline to evaluate the classification performance of
graph kernel (Graph).
Finally, we train an SVM classifier with all features except the graph kernel,
plus 7 simple graph features proposed by Castillo\cite{castillo2011information} 
to examine if our graph kernel is indeed important versus just 
simple graph features extracted from propagation tree (Simple).  
For this experiment, we compute the accuracies,
precisions, recalls and F$_1$ measures by 3-fold cross validation on the
big data set.

\begin{table}[ht]
\centering
\small
\caption{Comparison of different methods}\label{table:result-comparison}
\begin{tabular}{@{}lccccc@{}}
\toprule
\multicolumn{1}{c}{Methods} & Hybrid & Castillo & Yang     & Graph     &Simple \\ \midrule
Accuracy                             & {\bf 0.913} &0.854  & 0.772 & 0.770 &0.856 \\
F precision                          & {\bf 0.905} &0.853  & 0.773 & 0.773 &0.846 \\
F recall                             & {\bf 0.922} &0.854  & 0.776 & 0.763 &0.871 \\
F F$_1$                              & {\bf 0.913} &0.854  & 0.774 & 0.768 &0.859 \\
O precision                          & {\bf 0.920} &0.853  & 0.770 & 0.766 &0.866 \\
O recall                             & {\bf 0.903} &0.854  & 0.768 & 0.776 &0.840 \\
O F$_1$                              & {\bf 0.912} &0.854  & 0.769 & 0.771 &0.853 \\ \bottomrule
\end{tabular}
\end{table}

\tabref{table:result-comparison} shows the result.
Overall, the table demonstrates that our approach outperforms
the other competitions by large margins across all measures.
%The performance of methods proposed by Castillo is between ours and Yang.
%The rumor class F$_1$ is 0.774 for approach of Yang
%versus 0.913 by our method.
The hybrid SVM classifier achieve a higher accuracy than the classifier with simple graph features. This is because hybrid SVM classifier could store much information of the propagation tree while simple graph features lose a lot of 
such structural information.
Besides, the graph kernel alone has
a comparable performance to the baseline of Yang.
%On the rumor class precision measure,
%graph kernel retain a little bit higher value than baseline,
%whereas baseline has a much better result than graph kernel
%on the rumor class recall measure.
These results indicate that propagation tree pattern is a critically
important high-order feature for distinguishing false rumors from others.

