\section{Conclusion}
This paper studies the problem of automatically detecting false rumors on 
the popular Chinese microblogging service, Sina Weibo. 
We develop a graph-kernel-based SVM
classifier which combines the features from the topics of the original message,
the sentiments of the responses, the message propagation patterns, and the
profiles of the users who transmit this message around. Message propagation
patterns have been used as high order features for the first time.
Our results show that the repost patterns of false rumors and others are
very different, which makes the random walk graph kernel very useful
in detecting false rumors.
The combination of random walk kernel and RBF kernel performs better
than each of them alone, as well as recent state-of-the-art approaches,
with an accuracy of 0.913. More importantly, our model can be 
used for the early detection of false rumors. Results show that
our algorithm is almost 90\% confident when detecting false rumors 
just one day after their initial broadcast.
%We also propose some new features and investigate their contributions. We found that the topic of message and the sentiment of the users who repost the message are very helpful in classification. Besides, we shrink the tree by merging normal users and keeping opinion leaders to speed up the calculation of the graph kernel, which increases the accuracy.

%Besides, we propose a method to speed up the calculation of the graph kernel. We shrink the tree by merging normal users and keeping opinion leaders. The result shows that shrinking the tree with a proper ratio will increase the accuracy.
