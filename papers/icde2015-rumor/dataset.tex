\subsection{Dataset}
\label{sec:data}
To train and evaluate our approach of detecting false rumors, a labeled
data set is needed.
We collect a set of known false rumors from Sina community management center \cite{website:Manage},
which deals with reporting
of issues including various misinformation which we regard as
certified false rumors.
% consists of all kinds of fields, so the diversity of rumors is guaranteed.
There are 11466 reported false rumors between 2012/05/28 and 2014/04/11.
%in the result publication category at that time excluding the those without links for the original message webpages.
%Above rumor data are captured directly from Sina Weibo's mobile website.
Since a rumor must have sufficient circulation, we only keep those false
rumors that have at least 100 reposts, which leaves us with 2601 false rumors
up to 2014/04/11.
%and all their reposting information by the time of being captured excluding abnormal original message links.
%Some previous works \cite{yang2012automatic} also made use of
%Sina Weibo's official acount to collect false rumors.
%But at that time, Sina Weibo had not constructed the community management
%and there was an only official false rumor busting account in
%Sina Weibo posting some identified misinformation to public.
%One problem is that the false rumor reports posted by the account
%had no links to the original messages so they needed to
%construct queries manually to find out the original messages.
Sina Weibo API provides interfaces to capture the information of
original messages as well as their repost messages.
From Sina Weibo API, we captured the post time, post client
and content of 2601 false rumors along with all their reposts.

In the real world, the number of false rumors on Sina Weibo
is much smaller than the number of normal messages (1 out of 9 or less).
Thus a ``dummy'' classifier that rules all messages as normal messages
will achieve a very high accuracy (above 90\%) on real-world data.
To avoid this problem, we construct a data set with roughly equal number of
false rumors and normal messages. Most studies in the past also use
data sets which are either 50-50 split
\cite{castillo2011information,jin2013epidemiological}
or close to that \cite{yang2012automatic,qazvinian2011rumor}.
Thus, we randomly select 5000 other Weibo original messages
which are not proved to be false as well as their reposts
using the Sina Weibo API. Then, we manually filtered out messages with fewer than 100 reposts as well as false rumors to form a set of 2536 normal messages.
%To make non-rumors be in accord with rumors, we also only select the tweets that have 100 reposts at least here. The profiles of users involved are included in the data captured from API. Afterwards, we labeled 2844 pieces of non-rumors from them manually.
Each message or repost contains links to the author profile
information such as age, gender, number of followers and friends,
and can be crawled using the Weibo API.
%
%Sina Weibo provides API to capture a user's information but the speed is too slow because of frequency restriction. So we capture the original poster's information through Sina Weibo API and the other users' information directly from their homepages on Sine Weibo's mobile website.
%

At the end of this phase, our labeled data set 
\footnote{The labeled data set of the original messages (without reposts)
is available at \url{http://adapt.seiee.sjtu.edu.cn/~kzhu/rumor/}.}
consists of 2601 false rumors, 2536 normal messages
and with 4 million distinct users involved in these messages. Of these
500 false rumors and 500 other messages (called small data set) are used for
SVM parameter tuning while
the rest (called big data set) are used for end-to-end cross validation.
