\section{Related Work}

\subsection{Relation Classification}
Relation classification or extraction is an important first step for constructing structured knowledge graph in NLP with the popular benchmark datasets such as NTY-10~\cite{RiedelYM10} and the SemEval-2010 dataset~\cite{HendrickxKKNSPP10}. Previous datasets for relation classification focus on figuring out the relation type between two entities in a single sentence, including FewRel~\cite{HanZYWYLS18} and  TACRED~\cite{ZhangZCAM17}.  However, such intra-sentence relation classification has a limitation in real applications and looses nearly 40.7\% of relational facts according to previous research~\cite{SwampillaiS10,VergaSM18,YaoYLHLLLHZS19}.

Inter-sentence relation classification or document-level relation classification has gained more attention in recent years. There are only several small-sized dataset for this task, including a specific-domain dataset PubMed~\cite{LiSJSWLDMWL16} and two distant supervised datasets from Quirk and Poon~\cite{QuirkP17} and Peng et al. ~\cite{PengPQTY17}. To facilitate the research in this area, DocRED~\cite{YaoYLHLLLHZS19} has been proposed as the largest dataset for document-level relation classification. Our task is different from it since we focus on interpersonal relations while person-related entities is only a small component in DocRED. Besides, our task is based on dialogue sessions instead of plain documents and interpersonal relation classification may need inferences beyond session level.

\subsection{Dialogue Datasets}
Dialogue system has been a hot research point in recent years with a rapid growing number of available dialogue datasets. 

Generally, dialogue datasets can be divided into two categories. 
One is the task-oriented dialogue datasets such as Movie Booking Dataset~\cite{LiCLGC17}, CamRest676~\cite{UltesRSVKCBMWGY17} and MultiWOZ~\cite{BudzianowskiWTC18}. These datasets focus on single or multiple targeting domains and are usually labeled with dialogue act information, serving for the slot filling~\cite{LiuWXF20} and dialogue management tasks~\cite{BudzianowskiV19} when building task-oriented dialogue systems.
The other is the open-domain chit-chat datasets such as DailyDialog~\cite{LiSSLCN17}, MELD~\cite{PoriaHMNCM19} and PERSONA-CHAT~\cite{KielaWZDUS18}. The resource of these conversations are usually social media platforms, including Facebook, Twitter, Youtube, and Reddit. Researches on these datasets mainly focus on emotion recognition and emotion interplay among interlocutors, helping chatbots generate more emotionally coherent~\cite{GhosalMPCG19} and persona consistent responses~\cite{ZhengZHM20}.

There are two existing datasets similar to our settings. One dataset is the DialogRE~\cite{YaoYLHLLLHZS19}. It focuses on predicting the relations between two arguments in a dialogue session, where relations between arguments of interlocutors are rare. Also, since all of the 1,788 dialogue sessions are crawled from the transcript of \textit{Friends}, it suffers a limitation of the diversity of scenarios and speakers. Another dataset is MPDD~\cite{ChenHC20}. This dataset contains 4,142 dialogues annotated with speaker-listener interpersonal relation in multi-party dialogues for each utterance, while the relation types in our dataset is not such directional relationships. Besides, both datasets ignore the fact that interlocutors may have multiple sessions which is considered in our task and dataset. Our task is more reasonable with practical social meanings.


