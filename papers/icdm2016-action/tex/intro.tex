\section{Introduction}

Conceptualization is crucial psychological processes in the human mind,
in which an abstraction is formed to describe a number of instances
or entities sharing some common attributes.
%such as knowledge representation, imagination, memory etc. However, it is hard
%to strictly define the term concept as its nature involves deep interpretation of human minds.
%But in general we regard concept
%as a set of entities that share some common properties.
For example ``China'', ``Japan'' are both entities under the concept ``country''.
``China'' is called an instance of ``country'' while ``country'' is the concept
of ``China''. Such concept-instance relationship is also called isA relation.
Previous work \cite{wordnet,probase} has created either manually or automatically
knowledge bases (or taxonomies) that include such isA relation.

Such relations, as fundamental in ontologies and taxonomies, is indispensable when it
comes to understanding of natural language and representation of real world knowledge.
Researchers could benefit from such relation by using them to compare real world objects
in semantic prospective, and categorizing entities that share similar properties.

\KZ{Name a few benefits of such isA taxonomy}
\XY{benefit listed below}
But so far such
knowledge bases are primarily concerned with isA relations between two nouns\footnote{There is a
small amount of isA relation between verbs and between adjectives in WordNet as well.}.
\KZ{Avoid referring to our previous papers as ``we'' since this is double blind
review.}
\KZ{Define informally what is action. Maybe we need a citation here.
Why is it important (and userful) to
conceptualize actions(predicates)? (In terms of NL understanding and information retrieval?)
Give examples that benefit search here.}
%In previous works we managed to build a noun-based Knowledge graph called
%Probase in which the entities are linked via is-a relations.


One of the first steps towards conceptualizing actions to conceptualize
arguments (subject or object for example) of a verb predicate into noun concepts \cite{action}.
With the argument conceptualization it becomes possible to
represent actions on a new level. For example ``Google bought Deepmind'' and
``Microsoft bought Nokia'' would both be conceptualized into ``company buy company''.
\KZ{The essense of argument conceptualization is to disambiguate verbs
and classify their senses. With this, we can assimilate actions with each other, but
how do we know if an action and a noun is similar?}
With this insight we can further develop the
framework by abstracting these action concepts into single noun concepts,
which is what we call action conceptualization. For example,
here ``acquisition'' or ``purchase'' would be good abstractions for ``company buy company''.
In our work we aim at constructing
an action-noun map to reflects such action to noun abstractions.

The main difficulties of this mapping lie in the ambiguity nature of human
languages, and the external knowledge required for conceptualization. FrameNet \cite{baker1998berkeley}
aims to recognize variations of semantically similar short texts. But it does not generate
a human readable representation of the target sentence. Semantic Role Labeling(SRL)
\cite{palmer2005proposition}aims to label the semantic meaning of the arguments of a verb.
For example, ``Joe chopped the watermelon with
a giant knife.'' Here Joe is the agent, watermelon is the patient and knife is the instrument.
But in our work we aim to find out the concept of each argument, so in this example
Joe would be a person, watermelon would be a fruit and knife would be a tool or weapon.
But the roles it could find are usually too generalized and limited, which makes it hard
for variations and abstraction generations.

In this paper, we approach the problem as if the noun concept is a
summary of the verb action. We believe the best place to look for
such summaries is from news articles. News articles are written
formally, with a title and usually an overview sentence at the
beginning. The title and the overview essentially summarize the whole
story. Since news stories typically are of narrative nature, there is
plenty of actions embedded in the body of the main content, which is
where we can extract action concepts.
Based on the previous work of \cite{gong2015representing},
we are able to get the possible arguments of a given verb.
For example the subject of the verb ``invade'' could
be ``country,'' ``group,'' and ``person.''
The object of ``invade'' could be ``country'', ``area'', etc.
With the help of an IsA taxonomy, we know that country subsumes
instances such as iraq and US. Consequently
we can recognize the action concepts from the news text.
The key challenge here is
to map the correct nouns from the title and overview to the suitable
actions from the body of the news.

This paper makes the following contributions:
\begin{itemize}
\item First to study the problem of action conceptualization... blah;
\item developed algorithms to automatically mine action-noun relations from
large body of news articles with accuracy of XXX;
\item demonstrate the usefulness of such knowledge with a prototype
news search engine which outperform the TF-IDF keyword search and
LDA powered topic search.
\end{itemize}
