\section{Introduction}

Source code repositories are the main part of software projects.
They are composed of thousands of related source code files which contribute to the whole project.
Since the software industry has been growing to a indispensable part of the society,
more and more source code repositories are created. We want to create some tools through the analysis
of them.

However, maintaining source code repositories take more effort comparing with writing new files.
Software engineers need to not only read code written by others but also try to find which part to edit.
People are facing with such a big challenge that they are working at low efficiency.
Here are examples from Github in table 1. They are some popular open source projects written in Java.
The number of Java files are showed in the table. We can see that in such kind of big repositories,
there are hundreds of or thousands of source code files. If someone wants to check one function among
them, it is quite a hard work.

\begin{table}[!htp]\label{table:repos}
\caption{Number of Java files}
\centering
\begin{tabular}{|c|c|}
\hline
Project Name & Number of Java Files \\
\hline
RxJava & 545\\
\hline
okhttp & 269\\
\hline
j2objc & 2741\\
\hline
libgdx & 2163\\
\hline
java-design-patterns & 828\\
\hline 
\end{tabular}
\end{table}


Here comes our motivation to design a system to automatically show people the topic of a source code repository.
Besides the function in topic recommendation, people also need to check the similarity of two source code repositories,
to see how much novelty one new projects can make. However, people can change the semantic information in the
files such as rewriting a program with the same structure and different identifiers. So we need a tool to calculate
the similarity of two files or repositories, simultaneously using the semantic information and structural information.

Another function we want to realize is ``code searching''. We can type into a system with some tags, showing the
properties of the source code repositories we wants. Then the system returns people with some repositories or
source code snippets.

A discussion of source code repositories is necessary.
After long-time observation on many source code repositories and snippets from them,
we found the following features of code, which are the sources of how people can understand the code.
\begin{itemize}
\item Semantic information comes from the naming of variables and methods, and the annotation or comments.
Programmers always try to use simple words or phrases as variable and method names, to suggest readers the
function of them. Also much information is included in the annotation and comments which are in the form
of a short paragraph in natural language.
\item Structural information, which is always related with some key words. For example, the parenthesis in C++ or Java
is used to restrict the scope of a statement. Readers can simulate the function of code themselves with the help of
such kind of structural information.
\item Invocation information which means one method calls another one, namely the call graph. In source
code repositories, many methods resort to other functions to finish a task. Such kind of invocation information
is quite useful in code comprehension.
\end{itemize}

Topic models are used to process natural language and provide people the abstract topic of the documents.
Though topic models can solve similar problems, they will produce bad results when adapted to
source code repositories because of the following reasons:
\begin{itemize}
\item Topic models neglect the structural information provided by code;
\item The contribution of semantic information to the meaning of code is exaggerated;
\item They cannot show specific theme of a method or a function.
\end{itemize}

Since topic models work badly, we try to combine the semantic and structural information together
in our model. Allamanis et al\cite{allamanis2015bimodal} had developed a model to combine
source code snippets and natural language and also form a bimodal which can map both from code to natural
language and from natural language to code. Their model were trained to produce some tags for a short
code snippet with the information from structural information in code and the related description in natural
language.

We based on the bimodal by Allamanis et al\cite{allamanis2015bimodal} and added our new algorithm to it.
We noticed that in some natural language model, structural information is not included in and in Allamanis et al's
model, similarly semantic information cannot bring too much contribution. So we combined the two kinds of
information in one model to provide more useful information in code understanding. Also, we tacked on the
invocation information into our model to show the ability to handle multiple methods.

In this paper, we make the following main contributions:
\begin{itemize}
\item Strengthening the ability of the previous model to describe source code, adding semantic information
and invocation information.
\item Transferring the model from short code snippets to large code repositories, which can show more powerful
function in software industry.
\item Showing our results on five different source code repositories and comparing the results with the
previous model. Results show that our model works better than the previous one.
\end{itemize}
