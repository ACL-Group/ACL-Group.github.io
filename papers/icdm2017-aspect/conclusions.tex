\section{Conclusion}

In this paper, we propose our multistage framework ExtRA for extracting aspect terms from user reviews, 
which is beneficial for both qualitative and quantitative review 
summarization and opinion mining for any types of product or service. 
We found that simply performing topic modeling alone does not do
very well with this problem because reviews are highly condense and tend to
switch review aspects quickly within a short text. 
Our unsupervised framework solves the problem by slicing and shuffling
the reviews and then clustering them in sentence level and 
topic level respectively. Finally we design our AspVec to represent 
the semantics of aspects and use it to extract aspect terms by 
computing the similarities.
Extensive experimental results show that this approach
outperforms other the state-of-the-art models.
%Although LSTM-based RNN has been used widely in various 
%sentiment analysis tasks like sentiment prediction, we find out that
%paragraph vector is more effective in our tasks.

As for the future work, we will improve AspVec to better represent 
the semantics of topic clusters. Also, we would generalize 
our framework to other domains, e.g., extracting answer aspects 
from community question answering.
% 
%we first made a general introduction to sentiment analysis, 
%talking about the motivation and potential power of such research; 
%we talked about the tasks in sentiment analysis, including classification, 
%extraction and summarization; we introduced several popular methods for 
%sentiment analysis, including linguistics features, topic model and its 
%variations, and finally deep learning for natural language processing, 
%with an emphasis on language modeling and sentence representation. 
%We then focus on the problem we try to tackle in this paper, that is, 
%aspect extraction for aspect-based review summarization; 
%we explained why this task is difficult and why models like LDA won't work well. 
