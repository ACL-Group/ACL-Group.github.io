\section{Conclusion}

In this paper, we propose to solve the aspect extraction from user reivews, 
which is useful for structured and quantitative review summarization and 
opinion mining for any product or service. Topic modeling alone doesn't do
very well with this problem because reviews are highly condense and tend to
switch topics quickly within themselves. Instead, we proposed a
multistage framework, which clusters the reviews from coarse-grained
sentences to fine-grained words. This is a purely unsupervised method, which
doesn't require any human annotation. Results shows that this approach
outperforms other models, including MG-LDA, which is a popular method
designed specifically for aspect extraction, on 15 different categories of
products and services. Although RNN and LSTM has been used widely in various 
sentiment analysis tasks like sentiment prediction, we find out that
paragraph vector can be more effective when it comes to learning a
representation under unsupervised settings, although paragraph vector
is simpler than RNN and has fewer parameters.

% 
%we first made a general introduction to sentiment analysis, 
%talking about the motivation and potential power of such research; 
%we talked about the tasks in sentiment analysis, including classification, 
%extraction and summarization; we introduced several popular methods for 
%sentiment analysis, including linguistics features, topic model and its 
%variations, and finally deep learning for natural language processing, 
%with an emphasis on language modeling and sentence representation. 
%We then focus on the problem we try to tackle in this paper, that is, 
%aspect extraction for aspect-based review summarization; 
%we explained why this task is difficult and why models like LDA won't work well. 
