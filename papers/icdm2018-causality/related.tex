\section{RELATED WORK}
\label{sec:related}
%Furthermore, our focus on deriving causality knowledge is novel and never explicitly addressed before except for submodule of rule instance extraction, argument generalization. Firstly, event extraction in rule instance extraction has been researched in many approach, supervised method \cite{b1}, But we have no that much annotated data, while unsupervised method \cite{b1} needs kinds of linguistic resources, which is hard to get for Chinese, and translating would make such standard resource reliable, and a easy way to represent the event by nouns and verbs which is easy but limited in their short news headline corpus. Secondly, argument conceptualization is done by \cite{b1}.  \cite{xu2016improved}

causality bas been researched for a long time. \cite{luo2016commonsense} extracts the causal words or phrases from large text corpus and reason between short text. \cite{zhao2017constructing} extracts the causal event represented by verbs and nouns ,and generalize events to construct abstract causality network, and embed it to empower downstream application.
  
Since our focus on mining causality knowledge with a firstly proposed representation scheme is novel and
never explicitly addressed, We would introduce some related works involving each submodule.

\textbf{knowledge representation}
	Now, two directions toward knowledge representation.\cite{li-16} tries neural representation, represent the common sense knowledge with numerical vector. Still the other way is try to use different structure to represent the knowledge, like First Order Logic, Production, Semantic network, FrameNet\cite{baker1998berkeley}. We walk along the direction of symbolic form. 

\textbf{event representation and extraction} supervised unsupervised
\cite{ding2016knowledge} \cite{ding2015deep}

\textbf{conceptualization}
\cite{gong2016representing} tries conceptualize the subject and object of a verb. 


%\textbf{construct knowledge base}
