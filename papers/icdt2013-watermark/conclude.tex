\section{Conclusion}
\label{sec:conclude}

In this paper, we proposed a new blind watermarking scheme for digital vector
road maps. The scheme produces and detects watermarks according to local information
with the help of three secret keys of negligible sizes but without referring to
the original map. The algorithm dynamically partitions a given map according to
road density and inserts one-bit watermarks to one of the least significant bits
of points determined by the secret keys. The amount of distortion brought by
watermarks is arguably much smaller than existing methods.
Our preliminary evaluation shows that this algorithm 
is resilient to massive crop and merge attacks and significantly outperforms
two other state-of-the-art vector map watermarking approaches in terms of
detection accuracy.
%all the existing watermarking approach as far as we know and also proposed a 
%watermarking algirthm which can survive from this kind of attack as well as 
%various kinds of attacks on GIS spatial data sets. 
%We partition the digital map 
%into small blocks by constructing Quadtree structure and insert watermark to each blocks just with
%local information of these blocks. The most important feature of this approach is that
%it can survive ''Merge Attack'', which is an important property that no other watermark 
%approaches guarantee. Finally, we implement the approach, test it under different kinds 
%of attack and make some analysis.
%
%Our future plan is to extend our approach to watermarking general 
%spatial data sets such as protein 3-D structure data.
%This is an important avenue of research because it can prevent 
%online published spatial data sets, including protein data, 
%from being illegally published by others on their own web sites.
