\section{Introduction}\label{sec:intro}
Hardware description language (HDL)\cite{intro:HDL} based IP synthesis has been the
industry standard in recent years.
However, it has a number of problems.
First, Verilog and VHDL are so low level that they are often compared to assembly languages
in terms of programmability.
Second, the simulation and verification
is the most time-consuming step in IP design cycle. Third, while most HDLs offer design libraries of basic building blocks,
these are largely limited to the hardware circuit level, and are often inadequate for programming
large, complex but common algorithms such as those used in cryptography and image processing.
Today, designs for these algorithms require thousands of lines of Verilog code which is
extremely expensive to produce, debug and maintain. Last, because HDLs do not offer the
capability of high level abstraction, it is not easy to reconfigure the functionality of an
existing design. For example, for a given design of an AES algorithm, if the user prefers
to trade die space for speed, a common approach is to unroll a loop a number of times
and execute it in parallel within a clock cycle. Such unrolling cannot be achieved
in Verilog without substantial code change.

To address some of these problems, a range of new solutions have been
proposed. There are two main categories, new and powerful languages to make programming easier and generators to help users to get a circuit with several clicks. SystemC\cite{Grotker-sysC} introduces an event driven simulation kernel and some ability in describing hardware in C. SystemVerilog\cite{Sutherland-systemverilog}, on the other hand, aids hardware designer by raising the abstract level of Verilog with convenient programming constructs and some object-oriented concepts. Spiral\cite{NordinMHP05:Spiral}\cite{SpiralProject} project, which develops systems that generate hardware designs from high level mathematical representations of certain DSP transforms. However, none of the above methods provides
opportunities to users to re-configure 
the circuit to gain time-space tradeoffs.

In our work, a template based interactive generator was explored. Users can seek tradeoffs during the generating process via selecting the number of s-box and mix-column sub-module instantiated in the architecture. 
A new language construction $map$ was proposed to describe circuit in algorithm level that depicts the mathematical computation meaning of the circuit. The $map$ construction can be embedded with Verilog code, so it can be used anywhere as long as there are such circuit structures.
