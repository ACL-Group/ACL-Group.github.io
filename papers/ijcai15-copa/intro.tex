\section{Introduction}
\label{sec:intro}
Commonsense causal reasoning aims at understanding the causal dependency
between concepts or events in our daily life. To illustrate the problem,
we present a question from
Choice of Plausible Alternatives (COPA) evaluation~\cite{gordon2012copa},
which consists of one thousand multiple-choice questions
requiring common causal reasoning to answer correctly.
Specifically, each question is composed of a premise and two alternatives,
where the task is to select the more plausible alternative as
a cause (or effect) of the premise.

%\begin{itemize}
%\item[] Premise: \emph{The runner wore shorts.}. What is the
%cause?
%\item[] Alternative 1: \emph{The forecast predicted a hot day.}
%\item[] Alternative 2: \emph{She planned to run along the beach.}
%\end{itemize}

\begin{itemize}
\item[] Premise: \emph{I knocked on my neighbor's door.} What happened as an
effect?
\item[] Alternative 1: \emph{My neighbor invited me in.}
\item[] Alternative 2: \emph{My neighbor left her house.}
\end{itemize}

Commonsense causal reasoning has been actively studied,
as such understanding is crucial in text understanding, natural
language processing, artificial intelligence and other fields.
From the above example, we can observe that a key challenge is
harvesting common sense causal knowledge
that the action of knocking causes that of invitation.

Existing work can be categorized by how such knowledge is harvested.
First category is data-driven approach of harvesting causality from web corpus.
Best known results in this category leverage
Pointwise Mutual Information (PMI) statistic~\cite{Mihalcea2006:CKM}
between words in the premise and alternative, to identify the pairs with
high correlation.
In our example, we can expect that two words \emph{knock} and \emph{invite}
co-occur frequently in web documents, which indicate a potential causality.

Though PMI has been an effective indicator in prior literature, it suffers from the following limitations:
First, lexical co-occurrence can be a false alarm. In our example,
\emph{door} and \emph{house} are also observed
frequently together, but identifying this pair as causality leads to falsely identifying the second sentence as a
result. This observation suggests that term causality from lexical co-occurrence alone is somewhat noisy and
harvesting from causal lexico-syntactic patterns would avoid collecting
house and neighbor as a causal pair.
Second, co-occurrence is undirectional, while direction is crucial in causality.
COPA task is directional such that our example question can be asked in another direction, that is, asking what is a cause of knocking.
In this direction, \emph{call} can be a strong cause, but it cannot
be a result, though PMI statistic would model
the two directions equally likely.

Second category, pursuing the opposite emphasis of depth in
understanding sentences, seeks to overcome the limitation of
the first approach.
These approaches build on deeper lexico-syntactic analysis of sentences,
to identify knocking and inviting in our examples as
\emph{events}, and determine whether causality between two events hold.
Alternatively, ConceptNet~\cite{HavasiSALAM10} leverages human efforts
to encode causal events as common sense knowledge.
However, these approaches, building on human and heavy analysis,
inherently lack coverage, compared to the first category, which is
reported to outperform the second~\cite{gordon2012copa}.

In contrast, our goal is to pursue both breadth and depth in modeling
commonsense causality.
To pursue breadth, we propose a data-driven approach of harvesting
\emph{term causality network} from a large corpus.
To pursue depth, we conduct lexico-syntactic analysis of the sentences to
extract events and identify events that are strongly causal using the causality network
and other semantic resources such as WordNet~\cite{Miller1995}.
Our approach overcomes the two limitations of existing data-driven approaches,
with the following novel contributions:
\begin{itemize}
\item We harvest term causality network, selectively from causal
lexico-syntactic patterns, effectively pruning out false causality observed
from lexical co-occurrence.
This network encodes both causal direction and strength between terms.
\item We redefine causal strength $u \rightarrow v$ to reflect directions, by combining conditional probability
of $u$ being the cause of the pairwise causality and $v$ being its effect.
\item To quantify causality between phrases, we aggregate term causality
leveraging both syntactic and semantic understanding on the premise
and the alternatives. For syntactic understanding, we parse sentences to
extract \emph{event} from premise and alternatives, consisting of head words
in verb and objects. For semantic understanding, we leverage
semantic knowledge on each term in the event obtained from WordNet,
to properly discount causality from ambiguous terms.
\end{itemize}

We evaluate the strength of our proposed approach using COPA task,
from which ours outperformed all existing results of
state-of-the-arts by achieving $68.8\%$ in accuracy. In addition, we
validate the accuracy of our causality detection  using manually
labeled causal relations from ConceptNet as ground truth.
