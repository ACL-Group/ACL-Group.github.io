\section{Approach}
\label{sec:approach}

Given a natural language relation with its training relation
instances, our inference model first generates candidate 
schemas from its training relation instances, and then constructs a
probabilistic distribution over all the candidates.
Due to the lack of direct $\langle relation,\, schema \rangle$
training data, our learning model is distant supervised, we design
a data-driven function to label each candidate with ``silver'' 
score, and then perform the learning step based on ``silver'' labels.

% talk about model here
Suppose we have $N$ training relations in total.
We define $GEN(r)$ as the generation set of candidate schemas for 
the relation $r$.
The conditional probability of a schema $s \in GEN(r)$ is produced 
by log-linear model:
\begin{equation}
  p(s|r; \vec{w}) = 
    \frac { 
      exp \{ \vec{f} (s, r) \cdot \vec{w} \} 
    } { 
	  \sum\limits_{s' \in GEN(r)} { 
	    exp \{ \vec{f} (s', r) \cdot \vec{w} \} 
	  }
    },
\end{equation}

\noindent
where $\vec{f} (s, r)$ is the feature vector extracted from the
relation and the schema, and $\vec{w}$ represents the vector of
feature weights, which we are going to learn.
As mentioned above, we define the silver labeling function on a 
schema ranging from 0 to 1, which is denoted by $lb(s, r)$.
The silver labeling function approximates the correctness of one
schema, and the goal of training is to minimize the negative 
log-likelihood function of correct schemas over all the candidates.
The formal loss function is shown below:
\begin{equation}
J(\vec{w}) = \lambda \| \vec{w} \|_2^2 \! - \!  
  \frac {1} {N} \! 
  \sum\limits_{i=1}^{N} {
    log \! \sum\limits_{s \in GEN(r_i)} { 
	  lb(s, r) p(s|r; \vec{w}) 
	}
  },
\end{equation}

\noindent
where $\lambda$ is L2 regularization parameter.
In the following sections, we describe generation set, silver 
labeling function and feature functions in detail.


% then 3 step in detail
% 1. candidate generation (BFS + DFS)
\section{Candidate Schema Generation}
\label{sec:candgen}

In the first part, we propose a searching alogrithm to collect 
candidate schemas from training relation instances.
The intuition is that we first find suitable skeletons 
%(please mention the def in problem) 
as a starting point, and then recursively add constraints on 
previous schemas, making candidates more and more specific.

\subsection{Skeleton Retrieval}
% 7 sentences introducing bfs
%1. recap
As outlined in \secref{sec:problem}, a skeleton is a path of KB 
predicates which connects subject and object variable.
%2. basic: bfs
For each positive relation instance, $\langle e_1, e_2 \rangle$, 
we use breadth-first searching algorithm to find all possible 
skeletons which connect them in KB.
%3. problem of connection
Due to various predicates and popular entities existed, a relation 
instance could be linked through a large number of different skeletons.
%4. what's meaningless rep.
Since most of natural language relations are short phrases, a 
skeleton with too many predicates is meaningless, and is less likely 
to be a suitable representation.
%5. solution to filter
In our method, we use a pre-defined parameter $\tau$ to limit the 
searching scope, only skeletons with length no larger than $\tau$ 
are kept, all remaining candidates are filtered out.
%6. why use minimal coverage
Also note that we always focus on well-descriptive skeletons, 
rather than some occasional path that just hits a few entity pairs.
%7. say in detail
In formal, we define another threshold $cov$ as the minimum 
percentage of entity pairs among all positive instances covered 
by a skeleton.
% Comment: we could describe the searching process in detail, like Matt's style "more formally, blabla..."

% 9 sentences explaning ground graph maintain
%1-2. introduce the HP size
How do we get the evidence that one skeleton is more general than 
another?
The training data is limited, and two skeletons may cover the same
positive instances, but a more general skeleton will always cover 
more entity pairs than the other one in the whole knowledge base, 
and the coverage size could be a useful feature in the further 
learning step.
%3. brute force is intractable
We could get the exact coverage of one skeleton by brute force 
searching its ground graphs over the KB, but it's very time consuming.
%4-8. explain in detail 
Here we propose a randomized approach to estimate the coverage size.
Supoose a skeleton has $n$ predicates with $n+1$ variables 
$x_0,\, x_1,\, ...\, x_n$.
Firstly, we only consider the predicate between $x_0$ and $x_1$, i
extracting all the ground graphs (only $e_0,\, e_1$ connected by 
the predicate), and keep a sample list 
\footnote{Dut to memory limit, we set the maximal sampling number 
as 100,000 in the implementation.}
of graphs by randomly picking, in order to saving time and memory.
Next, we follow the predicate between $x_1$ and $x_2$, expanding 
previous ground graphs into 3 entities, and also sample them randomly.
We perform the expansion step iteratively, until all variables in 
the skeletons are processed, resulting in a list of ground graphs 
with $n+1$ entities.
The estimated coverage of the skeleton is the size of distinct 
$\langle e_0,\, e_n\rangle$ pairs in sampled ground graphs, divided
by sampling rates at each iteration.
%9. show example
We show some candidate skeletons for ``attend'' relation, with 
coverages over positive training instances and Freebase, displayed 
in \tabref{tab:bfs-attend}.

\begin{table}[ht]
%\small
	\centering
	\caption{Example of candidate skeletons for ``attend'' relations.
		We show the coverage with percentage over training instances,
		and estimated coverage over Freebase. $\tau$=3, $cov$=10\%.}
	\begin{tabular}{|c|c|c|c|}
		%\toprule
        \hline
		skeleton 	& $cover_{train}$	& $cover_{FB}$ \\
        \hline
        ``p+p''		& 120 (45.2\%)		& 12000  \\
        \hline
        ``p+r+p''	& 130 (56.1\%)		& 12000000 \\
        \hline
	\end{tabular}
	\label{tab:bfs-attend}
\end{table}


\subsection{Schema Generation}
% 18 sentences: bfs basic, search space limit, budget, diversity
%1. general speak
In the second part of generating process, we take all candidate 
skeletons with corresponding ground graphs as input, and explore
more specific schemas.
%2. basic idea: search
The approach we are using here is to perform a depth-first search:
it starts from a skeleton, when coming to a new schema, we attach 
one constraint to a variable on the skeleton recursively, and 
continue searching deeper, until no new schemas could be traversed.
%3-6: basic limit on schema constraints
%3. why need limitation
The searching space is a tree structure which grows exponentially,
making the exhaustive searching intractable on a huge knowlege base.
%4. how to fix the search size
Aiming to collect meaningful schemas in this process, we limit the 
structure of a candidate schema that no more than 1 constraint
is allowed to add on each variable skeleton.
%5. the intuition behind
As mentioned before, natural language relations are always short
phrases, this gives us the point that it's less likely to infer
a comfortable structure for a relation with multiple restriction 
imposed on a single element, and our restriction just follows 
this intuition.
%6. the effect of limitation
Therefore, the maximal depth of the searching tree is $\tau+1$,
which is feasible for our task.

%7-11: budget base (why, budget+prune, criteria, how to prune, diversity)
%7. why need budget
Even though, we still encounter a large searching space, beacuase
there has hundreds of different constraints which can be attached 
to one variable, not to mention their combinations.
%8. introduce budget+pruning
Inspired by beam search algorithm, we introduce a budget over each
relation to control the total number of output candidates, while 
pruning strategies will be used to reduce searching space so that 
poor candidates could be ignored.
%9. what's the criteria
In our data-driven method, we use the number of positive instances 
covered by a schema as the criteria to approximately measure its
quality.
%10. explanation of the criteria
The reason is two-fold: we aim to keep those descriptive schemas in 
the output candidates, since we output a bunch of schemas instead
of only a few, we don't need a rather precise quality measurement,
the idea that better schemas cover more positive instances is
reasonable enough for our task; 
and the size of coverages would never increase when the search goes
deeper, which leads to a simple but effective pruning strategy.

%11-15. formal describe
Now wee explain the searching step in formal.
% [A simple pseudo code is available]
The beginning state of the searching is one skeleton, we enumerate
all the constraints which are allowed to add on, each constraint 
maps to a more specific schema.
Then new schemas are ranked over their coverages by descending order,
and we sequentially continue recursive searching on those schemas.
When the searching state comes to a new schema $s_0$, we keep this 
schema if there has enough room; otherwise, we pick the schema 
$s_1$ which has the smallest coverage among all kept schemas and 
compare their coverage.
If $s_0$ has a larger coverage, then $s_1$ is discarded, we keep 
$s_0$ and search deeper; otherwise, the current schema $s_0$ is 
pruned, and we backtrace the searching process immediately.
Finally, the output candidates are those schemas been kept when
the searching is over.

%16-18. diversity
The diversity of output schemas plays an important rule in the 
learning parts.
If most candidates are the same and only differ from one or two 
constraints, we are actually wasting budgets because it contains
much redundant information.
Since the budget is defined over each relation, we split the whole 
budget into separate parts for each skeleton, where the size of 
budget allocated to each skeleton is determined by the distribution
of coverages over positive instances.


% 2. silver labeling (ratio function)
\section{Silver Labeling Function}
\label{sec:label}

%18 sents.
%1. intro (2 sent)
With a schema $s$, a relation $r$ and its training instances as 
input, the silver labeling function $lb(s, r)$ approximately 
measures the correctness of the schema representing the relation.
The function follows a simple data-driven idea: a schema is more
confident to be correct, if it covers more positive relation 
instances and less negative instances. 
%2. incompleteness (3 sent)
A straightforword function can be derived by just calculate the 
proportion of instances covered in both positive and negative side.
However, the incompleteness fact of KB has not been taken into 
consideration: given a positive instance $\langle e_1, e_2 \rangle$,
where $e_1$ is a rare entity in KB, and its relationship with other
entities are not well connected, even a best schema (annotated by 
human) can't cover this entity pair, should this pair contributes
a negative evdience equally with those popular pairs not covered by
a schema?
The answer is no, so treating all positive and negative instances 
equally is not the best way to produce labeling function.

%3. group by e1 + formula (7 sent)
Now we introduce our solution to handle this problem.
The notations listed below are used in this secton:
\begin{itemize}
  \item $E_1(r)$: the set of distinct $e_1$ in training instances of $r$,
  \item $CV_s(e_1)$: all $e_2$ where $\langle e_1, e_2 \rangle$ is covered by $s$ in KB, 
  \item $PS_r(e_1)$: all $e_2$ where $\langle e_1, e_2 \rangle$ is in positive instances,
  \item $NS_r(e_1)$: all $e_2$ where $\langle e_1, e_2 \rangle$ is in negative instances,
  \item $PC_{sr}(e_1) = PS_r(e_1) \cap CV_s(e_1)$, and
  \item $NC_{sr}(e_1) = NS_r(e_1) \cap CV_s(e_1)$.
\end{itemize}

For positive part of training data, we define the confidence score 
of $e_1$ to the schema $s$ over relation $r$ as 
\eqnref{eqn:scp}:
\begin{equation}
\label{eqn:scp}
  sc_p(e_1, \! s, \! r) \! = \! \left\{
  	\begin{aligned}
	\! 1 / ( 1 \! + \! \ln \frac 
	  {\left| PS_r(e_1) \right|} 
	  {\left| PC_{sr}(e_1) \right|} 
	)    & ~ & PC_{sr}(e_1) \! \neq \! \varnothing  \\
	\! 0 & ~ & CV_s(e_1) \! = \! \varnothing        \\
	\! \frac {1} {
	  1 \! + \! \ln (\left| CV_s(e_1) \right| \! + \! 1)
	} \! - \! 1  & ~ & otherwise    \\
	\end{aligned}
  \right..
\end{equation}

These 3 branches in the formula represents different scenarios.
The first branch shows us that some positive $e_2$ are covered by 
the schema in KB, thus $e_1$ makes a confidence larger than 0:
The confidence reaches 1 if all positive instances are covered, 
and the scores decreases smoothly when more positive instance are
not covered by $s$.
The second branch encounters the problem of incompleteness: 
querying the shcema over $e_1$ returns nothing, without enough 
information, we can't make the statement that $s$ is not correct,
so we put a zero confidence here.
The last branch goes even worse: some $e_2$ are covered in the 
knowledge base, but neither is found in the positives instances.
Though KB incompleteness is still possible in this scenario, 
with more $e_2$ covered, the $e_1$ is more popular, which 
gives us a stronger evidence that $s$ is not correct.
Therefore, the confidence for the schema goes down from 0,
and become smaller when its coverage on $e_1$ goes larger, with
the minimum confidence as -1.

%4. negative side (3 sent)
Then we consider the negative part of training data.
Similar with \eqnref{eqn:scp}, negative relation instances also
provide confidence scores to the schema, 
defined in \eqnref{eqn:scn}:
\begin{equation}
\label{eqn:scn}
  sc_n(e_1, \! s, \! r) \! = \! \left\{
    \begin{aligned}
	\! -1 / ( 1 \! + \! ln \frac
	  {\left| NS_r(e_1) \right|}
	  {\left| NC_{sr}(e_1) \right|}
	)            & ~ & NC_{sr}(e_1) \! \neq \! \varnothing  \\
	\! 0         & ~ & CV_s(e_1) \! = \! \varnothing        \\
	\! 1 \! - \! \frac {1} {
	  1 \! + \! ln (\left| CV_s(e_1) \right| \! + \! 1)
	}            & ~ & otherwise \\
	\end{aligned}
  \right.,
\end{equation}

\noindent
where the first scenario makes the confidence score smaller than 0
(negative instances are covered by $s$); again the second scenario
faces the problem of KB incompleteness and we can't decide the 
quality of schema by this $e_1$; the last scenario shows us a
confidence score larger than 0, since all entity pairs it covered 
on $e_1$ are not negative, and the score increases monotonically 
when the size of coverage increases.

%5. summary  (3 sent)
In summary, each distinct $e_1$ from the training data of $r$ gives
us confidence scores on the side of both positive and negative
instances, ranging from -1 to 1.
Our silver labeling function takes all the confidences and 
averages them into the score from 0 to 1:
\begin{equation}
\begin{aligned}
  lb(s, & r) = 
  \frac {1} {2 \cdot \left| E_1(r) \right|}
  \sum\limits_{e_1 \in E_1(r)} \{ 
    1 + \\
	& \alpha \cdot sc_p(e_1, s, r) 
	+ (1 - \alpha) \cdot sc_n(e_1, s, r)
  \},  \\
\end{aligned}
\end{equation}

\noindent
where $\alpha$ controls the tradeoff between positive and negative
instances.


% 3. learning step (feature selection)
\section{Spatiotemporal Features}
In this section, we discuss the proposed spatial features extracted from OpenStreetMap\footnote{\url{https://www.openstreetmap.com}} and temporal features that act as fundamental components of our proposed transfer learning approach. 
The proposed spatial features capture the traffic-related geographical characteristics for each link in road networks.

\subsection{Basic Information Features} 
An example of extracting the basic information for a particular link is shown in Figure~\ref{fig:basic}.
We have 5 features for representing the basic information of each link: 
\texttt{length}, \texttt{\#begin\_node\_in\_links}, \texttt{\#begin\_node\_out\_links}, \texttt{\#end\_node\_in\_links} and \texttt{\#end\_node\_out\_links}.
For each link, the \texttt{length} is the real distance between the begin and the end node of this link. 
The other 4 features represent the number of \textit{in} and \textit{out} links connected to both nodes of a link, which may provide information about crossroads or one-ways.

\begin{figure}[th]
	\centering
	\includegraphics[width=0.4\textwidth]{figures/basic.pdf}
	\caption{An example for extracting basic information features.}
	\label{fig:basic}
\end{figure}
\subsection{Road Density Features}
Additionally, we believe traffic speed is highly relevant to road density, which can be measured by the number of neighboring nodes and links within the same area.
To be more specific and capture the sensitivity about directions, we compute road density respectively for each end in terms of the density of neighboring node , and the density of neighboring in and out link,  according to three radius (100/300/500m), as shown in Figure~\ref{fig:road_density}.
%The nodes in the range of 100, 300 and 500 meters are marked as blue. The in/out links of these nodes are part of link features.
Consequently, we have $2 \times 3 \times 3 = 18$ road density features in total.
%
\begin{figure}[t]
	\centering
	\includegraphics[width=0.35\textwidth]{figures/roaddensity.pdf}
	\caption{An example of extracting the road density features with three different radius. The origin is begin/end node of a certain link, and each blue circle is a link intersection.}
	\label{fig:road_density}
\end{figure}


\subsection{Categorical POI Density Features}
Points of interest (POI) are specific locations that people may find useful or interesting, such as restaurants, shopping halls, parks, etc.
Since such places are very influential to the traffic, we query nearby POIs\footnote{The 11 major POI types we consider are \{\textit{Eat, Drink, Going Out, Sights \& Museums, Transport, Accommodation, Shopping, Business \& Services, Facilities, Facility, Administrative Areas/Buildings, Natural or Geographical}\}. } 
 for each node with three different radius (100/300/500m) using HERE Places API\footnote{\url{https://developer.here.com/documentation/places/topics/introduction.html
}}. 
Figure~\ref{fig:poi} shows such an example for extract \textit{road density} features and\textit{ POI density features}. 

\begin{figure}[t]
	\centering
	\includegraphics[width=0.35\textwidth]{figures/poi.pdf}
	\caption{{An example for extracting POI Features. Different colors indicate different POI types.}}
	\label{fig:poi}
\end{figure}


\subsection{Temporal Features}
Our temporal feature is simply a distributed representation of the time information.
It is basically a concatenation of several one-hot vectors, 
where each vector represents the \textit{month}, the \textit{day} of a week, the \textit{hour} of a day and whether it is \textit{workday} respectively. 
%To sum up, the total number of spatial and temporal features is 89 and 44 correspondingly.


