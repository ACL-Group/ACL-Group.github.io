\section{Schema Inference}
\label{sec:candidate}
In this section, we propose a searching algorithm to generate output schemas,
with a training model to learn weights for each schema.
% Besides, we also explore the size of hit pairs and all hit pairs in $EP$.
%which satisfies what?
We first define a ``simple schema'' as a schema formed by a path of
solid predicates, connecting from $x_1$ to $x_2$.
%basic framework 5 sents
%1 Intro
Basically, the searching process is made up of three parts.
%2 BFS
In the first part, we generate a list of most simple schema candidates,
which are descriptive over a part of $EP$.
%3 DFS
With simple schema candidates as the starting point, in the second part, we expand schemas by
recursively adding new predicates onto previous schemas, making schemas more and more
specific.
%4 HP maintain
% During the searching process, we maintain ground graphs of each candidate, and
% calculate number of hit pairs that a candidate schema has.
%5 Selection
Finally, we select a finite number of well descriptive schemas from all candidates
as output measured by $cost(EP, S)$, and learn weights for these schemas.

\subsection{Simple Schema Retrieval}
%BFS part (with hit pair gen.)
%1. what is simple path
% At first, we state the ``most simple'' schema as a \textbf{solid} path, that is,
% all vertices in the schema are variables, connecting $x_{subj}$ and $x_{obj}$.
%2. intuition
Intuitively, a simple schema (path) represents a join operation over necessary predicates
in the knowledge graphs, which is a most general schema, and is the skeleton of more complex schemas.

%3-6. where the path comes
In order to find descriptive paths, we use breadth-first search algorithm
to find paths for each entity pair $\langle e_1, e_2 \rangle$.
Due to the existence of various predicates and popular entities,
we need to control the searching scope, in order to avoid retrieving long-path but
meaningless schemas.
The basic idea is to retrieve all simple schemas with the length
of path no large than a certain threshold, $\tau$.

Note that we always focus on well-descriptive schemas, rather than some occasional path that
just hits a few entity pairs. Therefore, we propose a iterative sampling framework to generate
simple schemas.

\begin{algorithm}
\caption{Iterative Simple Schema Retrieval}
\label{alg:bfs}
\textbf{Input}: Positive entity pair $EP=\{\langle e_1^{(i)}, e_2^{(i)} \rangle\}$,
length threshold $\tau$, sample size $sz$.

\textbf{Output}: A set of simple schemas $SchemaSet$, with $GroundMap$ storing ground graphs
in $EP$ for each output schema.
\begin{algorithmic}[1]
\State $SchemaSet \gets \varnothing$
\State $GroundMap \gets \varnothing$
\State $EP\_Pool \gets EP$
\While { $EP\_Pool$ not empty: }
	\State $SampleEP \gets randomSample(EP\_Pool, sz)$
	\State $EP\_Pool \gets EP\_Pool\, \backslash\, SampleEP$
	\For { $\langle e_1, e_2 \rangle$ in $SampleEP$}
		\State $schemas \gets bfs(e_1, e_2, \tau)$
		\State $newSchemas \gets schemas\, \backslash\, SchemaSet$
		\For { $ \langle e_1^{'}, e_2^{'} \rangle $ in $EP\_Pool$ }
			\If { $isHit(e_1^{'}, e_2^{'}, newSchemas)$ }
				\State remove $\langle e_1^{'}, e_2^{'} \rangle$ from $EP\_Pool$
			\EndIf
		\EndFor

		\State $SchemaSet \gets SchamSet + schemas$
	\EndFor
\EndWhile
\For { $schema$ in $SchemaSet$ }
	\State $subgraphs \gets extractGround(schema, EP)$
	\State $GroundMap \gets GroundMap + \langle schema, subgraphs \rangle$
\EndFor
\Return
\end{algorithmic}
\end{algorithm}

Algorithm 1 shows the details of finding simple schemas.
We sample a small number of unprocessed entity pairs for each iteration (line 5),
and find all its schemas by performing breadth-first search (line 8).
Once we find any new schemas, we enumerate each unprocessed entity pair
and judge whether they are hit by new schemas or not (line 11), if so,
we remove this pair from unprocessed set (line 12).
Finally, we collect all ground graph in $EP$ for each picked simple schema (line 15).

The intuition of iteratively sampling is that, for one entity pair,
it's much quicker to judge a schema hits or not (line 11, 15), rather than exploring
all its possible schemas. Based on the random sampling technique, it's
more likely to retrieve schemas with larger coverage. And the removal operation
ensure that every pair is hit by some schema (unless there has no paths
connecting this pair within $\tau$ length).
\begin{table}[ht]
%\small
	\centering
	\caption{Example of retrieved simple schemas for ``hasGrandFather'' relation.
	$\tau$=3, sz=20.}
	\begin{tabular}{|c|c|c|c|}
		%\toprule
        \hline
		schema & $|EP(S)|$ & $|HP(S)|$ & cover ratio \\
        \hline
        ``p+p''	& 0.456 & 0.471 & 0.xxx  \\
        \hline
        ``p+r+p''	& 0.xxx & 0.xxx & 0.xxx \\
        \hline
	\end{tabular}%
	\label{tab:simple-schema}%
\end{table}

In addition, MDL-based cost function involves $HP(S)$ and $EP(S)$. While $EP(S)$ are
stored in $GroundMap$, we query the schema over the whole knowledge graph, and store
a sampled list\footnote{Due to memory limit, we set the maximal sampling number
as 100,000 in the implementation.} of $HP(S)$.
\tabref{tab:bfs-grand-father} shows the picked simple schemas for ``hasGrandFather''
relation with $EP(S)$, $HP(S)$ size and cover ratio ($|EP(S)| / |EP|$).

\subsection{Schema Expansion}
%DFS part
%6 sents
%1. input?
In the second part of searching process, we take the simple schemas with
corresponding ground graphs as input, and explore more specific schemas.
%2. how to expand
The algorithm of schema expansion is a depth-first search (DFS) framework.
Intuitively, for the schema at some state of DFS, we add a new predicate (called branch)
on it, which produces a more specific schema.
We expand the new schema recursively, until it's ``too specific''
to be a suitable schema, which we back trace and go on searching with another branch .

There has a huge number of specific schemas derived from one simple schema.
We follow the running example ``hasGrandFather'', for the ``parent + parent'' simple
schema, we could add $\langle x_2, \text{gender}, \text{male} \rangle$ dashed edge,
forming the schema shown in \figref{fig:fb-schema}(b);
or we may add another edge
$\langle x_1, \text{directed}, \text{Titanic} \rangle$ to get a new schema.
Actually there are so many choice just for adding one branch, not to mention the
branch combinations.
Therefore, we use MDL-based cost function to guide the searching order,
also we designed a bunch of pruning strategies, which helps us filter out
meaningless schemas. We state our pruning strategies as follows.

First, we limit that we can add at most one
branch to each skeleton variable (the variable in the corresponding simple schema).
The intuition is that, a natural language relation seldom infers more than one
constraint on one argument, we follow this idea and avoid overfitted schemas.

Second, the cover ratio ($|EP(S)| / |EP|$) of a schema should above a threshold $\theta$.
\eqnref{eqn:cost-ep-s} shows that, a schema costs too many bits transmitting
uncovered entity pairs directly when it has a poor coverage,
which dramatically increases the summation cost.

Third, a branch needs to connect with enough \textit{distinct} entities in $EP(S)$.
Suppose schema $S$ is the ground truth schema for a binary relation,
then for any variable $x_i$ in the schema,
we could get various entities $e_i$ among all its ground graphs.
Still considering ``hasGrandFather'', there have more than 100 thousands ground graphs
in Freebase, resulting in various children, parents and grandfathers.
If some relation violates this rule, we could easily split it into several
unary relations, which is not a typical binary relation.
Therefore, we set up another parameter $\phi$ as the threshold of minimal connecting entities.

\begin{algorithm}
\caption{Depth-First Schema Expansion}
\label{alg:dfs}
\textbf{Input}: Positive entity pair $EP$, simple schemas $SchemaSet$,

cover limit $\theta$, distinct entity threshold $\phi$, $GroundMap$ storing

grounds in $EP$, $HpMap$ storing sampled grounds in $HP$, and schema budget $B$.


\textbf{Output}: $SchemaSet, GroundMap, HpMap$, saving newly expanded schemas.
in $EP$ for each output schema.
\begin{algorithmic}[1]
\Procedure{DfsExpander}{}
\State $expandTask \gets sizeof(SchemaSet)$
\For { $sc$ in $SchemaSet$ }
	\State $budget \gets B / expandTask$
	\State $sc.gEpSet \gets GroundMap(schema)$
	\State $sc.gHpSet \gets HpMap(schema)$
	\State $collectNum \gets DfsKernel(sc, budget)$
	\State $expandTask \gets expandTask - 1$
	\State $B \gets B - collectNum$
\EndFor
\Return
\EndProcedure

\Procedure{DfsKernel}{sc, budget}
	\If { $budget$ = 0 }
		\Return {0}
	\EndIf
	\State $SchemaSet \gets SchemaSet + sc$
	\State $GroundMap \gets GroundMap + \langle sc, sc.gEpSet \rangle$
	\State $HpMap \gets HpMap + \langle sc, sc.gHpSet \rangle$
	\State $collectNum \gets 1$
	\State $budget \gets budget - 1$
	\State $candBranchList \gets extractBranches(schema, \phi)$
	\State $newSchemaSet \gets \varnothing$
	\For { $branch$ in $candBranchSet$ }
		\State $nSc \gets specify(schema, branch)$
		\State $nSc.gEpSet \gets filterGround(sc.gEpSet, branch)$
		\State $nSc.gHpSet \gets filterGround(sc.gHpSet, branch)$
		\State $covRatio \gets getCover(EP, nSc)$
		\If { $covRatio > \theta$ }
			\State $newSchemaSet \gets newSchemaSet + nSc$
		\EndIf
	\EndFor
	\State $sortSchemaByMDL(newSchemaSet)$
	\For { $nSc$ in $newSchemaSet$ }
		\State $collect \gets DfsKernel(nSc, budget)$
		\State $budget \gets budget - collect$
		\State $collectNum \gets collectNum + collect$
	\EndFor
\Return {$collectNum$}
\EndProcedure

\end{algorithmic}
\end{algorithm}


Finally we present the DFS expansion approach as Algorithm 2.
We use total budget $B$ to control the total searching space,
for each simple schema, we assign its own budget as an average number (line 4).
In the main part of DFS expansion, we extract all candidate branches
for the current schema under threshold $\phi$ (line 17), and build a
specific schema.
All new schemas which passed cover limit $\theta$ (line 24)
will be saved.
Then we sort new schemas by $cost(EP,S)$ (line 26), and the further searching
direction are guided by the sorted list: schema with lower cost goes first.
The whole DFS expansion would retrieve at most $B$ specific schemas.
Both simple and specific schemas are sent to the next step.


\subsection{Schema Weight Learning}
In the last step, we generate the feature space and build a discriminative
model to learn weights for each schema.
Each dimension in the feature space indicates a independent schema graph.
Both positive and negative entity pairs are used for model training.
For an entity pair $\langle e_1, e_2 \rangle$, the $i$-th feature function
is define as:
\begin{equation}
\begin{aligned}
    &f(i, e_1, e_2)      = \left\{
        \begin{aligned}
        \ 1 & ~     & isHit(e_1, e_2, S_i)  \\
        \ 0 & ~     & otherwise \\
        \end{aligned}
	\right.	\\
\end{aligned}
\end{equation}
We applied Max Entropy classifier for the training task, resulting in the
final output schema graphs with weights.
