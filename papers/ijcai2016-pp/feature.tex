\section{Schema Features}
\label{sec:feature}

%16 sentences
%3 sents: talk about "element", encoding, 3 aspects
As what we described in \secref{sec:problem}, a schema is a
composition of elements: predicates, entities, types and variables.
In our model, feature functions are used to encode a schema from
3 aspects: the association between elements and relation words,
the similarity between the relation and element description, and
the structural information of a schema.
The first 2 parts focus on single element, and the last one is in
the view of the entire schema.

%3 sents: association, gives the intuition of assoc, and example
\textbf{Association Features} are built by pairwise combining 
elements with words in the surface form of a relation.
Each entity, type and predicated has a unique identifier in the
knowledge base.
By doing such combination, the model is able to mine certain
associations between KB nodes and single words.
For example, the feature ``starring | film.performance.actor''
indicates the association between word ``starring'' and a Freebase
predicate denoting the relationship from a film performance to an
actor, and training model is likely to learn a higher weight for 
this feature.

%5 sents: WN & w2v feature, and talk about positional encoding
\textbf{Similarity Features} takes the name of an element as input,
and measure the similarity from the name to the relation. 
We first compute the similarity at lexical level, we use WordNet as 
an external lexical resource, for each word in the element name,
we find its synsets in WordNet, and expand the word to a set of 
words containing its synonyms and derivationally related forms.
By counting the percentage of word sets which have co-occurr with 
relation surface form, we get a similarity score at lexical level.
And we also leverage the embedding model to measure the similarity
over statistics. 
In our method, word2vec model is used to calculate the maximum 
similarity score among all word combinations between the relation 
and the element name.

%5 sents. structural feature: EP, HP, size, constraints
\textbf{Structural Features} are extracted for each schema during
candidate generation.
Firstly, the length of skeleton and the total number of constraints
are picked as features.
Seondly, the percentage of entity pairs covered among all positive 
instances is also useful and picked as a feature.
In addition, two more features are extracted from the comparison 
between a schema and its skeleton (a more general candidate).
One is the ratio of covered positive instances between the schema 
and its skeleton, which shows how much positive evidence it loses 
when we try attaching constraints to a skeleton.
And the other one is the ratio of the coverage over the whole KB 
between the schema and its skeleton, indicating how specific the
schema is.

