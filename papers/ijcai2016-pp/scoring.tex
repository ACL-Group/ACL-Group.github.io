\section{Information-Theoretic Scoring}
\label{sec:scoring}

%>>>>0. we follow information theory to define the cost function
In this section, we follow information theory to define the function
$cost(EP, S)$, which is used to measure the cost of a schema describing
a set of positive entity pairs.

%>>>>1. what's MDL? bytes transmitting data & desc
    % with the form cost(EP, OS) = cost(OS) + cost(EP | OS), which is most suitable in our work
% some similar words in popl2008 can be put here, for example, what is compact and precise
The \textit{Minimum Description Length} (MDL) \cite{grunwald2007minimum}
is a common principle from information theory,
which states that a good description is one that minimizes the cost (in bits) of transmitting
the data.
% MDL is used in tasks of data representations \cite{}.
The intuition of MDL comes from compression problem: a user is going to transmit some data
to another user through network, he/she could send original data directly, or instead, try to compress
the data by a program, then sending both the program and compressed data to others.
A compression program (data description) is the best, if it minimizes the sum cost of transmitting both
the program itself and the compressed data.

Similar idea also holds in our paraphrasing scenario.
A schema is the description for positive entity pairs. One can query schema over knowledge
graph first, producing all its hit pairs, and then apply constraints (mentioned later) to hit pairs,
which reconstructs all positive pairs.
Since a highly general schema brings large hit pairs (increasing the size of constraints),
while a perfect specific schema costs too much for transmitting itself, MDL principle
could handle this trade-off when searching for a most suitable schema.
% why we have this idea?  data compression, data representing
% similar: use schema to represent
% then we have a tradeoff: too general v.s. too specific


%cite Grunwald 2007 here
Formally, the transmitting cost consists of two parts in our task:
\begin{equation}
    \label{eqn:mdl}
    cost(EP,\, S) = cost(S) + cost(EP\, |\, S),
\end{equation}

\noindent
where $cost(S)$ is the number of bits to transmit the description (schema) itself, and
$cost(EP\, |\, S)$ is the number of bits to transmit the data (positive entity pairs) given the schema.

%>>>>2. cost(S)
%Obviously, a good  is a connected graph. If $S$ contains some vertices which are not connected
%to $x_{subj}$ and $x_{obj}$ through predicates, those vertices are nothing with the relation that
%$S$ wants to describe. Removing those vertices can decrease the cost of transmitting $S$ itself without
%harming the effectiveness for transmitting entity pairs.
%Due to multiple schemas available, the cost of transmitting $\vec{S}$ equals to the sum of transmitting
%each schema.

Now we define the cost (in bits) of transmitting a schema graph as \eqnref{eqn:costs}.
The first row defines the cost to be the sum of transmitting each predicate and node in the schema.
the second row presents the cost of transmitting one predicate. It's made up of transmitting
indices of two nodes, the predicate name with its category.
The remaining part defines the cost for a node and a predicate name.
Note that the predicate name can only be ``isa'' if the category is $isa$,
therefore we don't need to transmit this duplicated information.
\begin{equation}
\label{eqn:costs}
\begin{aligned}
    & cost(S)      = \sum\limits_{p_s \in P_S} cost(p_s) + \sum\limits_{v \in E' \cup T' \cup X} cost(v), \\
    & cost(p_s)    = 2\log{|E' \cup T' \cup X|} + cost(pred) + \log{|C|}, \\
    & cost(v)      = \left\{
        \begin{aligned}
        \log{|X|} & ~     & v \in X  \\
        \log{|E|} & ~     & v \in E' \\
        \log{|T|} & ~     & v \in T'
        \end{aligned}
    \right., \\
    & cost(pred)   = \left\{
        \begin{aligned}
        & 0         & ~     & pred = \text{isa}  \\
        & \log{|L|} & ~     & pred \in L \\
        \end{aligned}
    \right.,
\end{aligned}
\end{equation}

%>>>>3. cost'(EP | S) (EP \in Hit(S))
Next, we focus on the cost of transmitting entity pairs given a schema.
Due to the fact that a single schema is not able to cover all positive pairs,
we split $EP$ into two parts.
We define $EP(S)$ as the hit pairs in $EP$, that is, $EP(S) = EP \cap HP(S)$.
For entity pairs excluded from $EP(S)$, the schema has no way to describe them,
and we have to transmit this pair directly, costing $2\log{|E|}$ bits.
Therefore the cost of transmitting pairs given schema is defined as \eqnref{eqn:cost-ep-s}:
\begin{equation}
\label{eqn:cost-ep-s}
    cost(EP|S) = cost(EP(S)|S) + |EP \backslash EP(S)|*2\log{|E|}
\end{equation}

Recap that $HP(S)$ are retrieved by querying the schema in the knowledge graph.
The retrieval step resembles querying a view in relational database,
where the resulting $HP(S)$ is a table with 2 columns, representing $e_1$ and $e_2$.
Note that $S$ can only hit at most $|E|^2$ different entity pairs, the more specific
a schema is, the fewer pairs it hits.

For the calculation of $cost(EP(S)\, |\, S)$,
since $EP(S) \subseteq HP(S)$, when transmitting one entity pair,
we can either send two entities directly, or send the row number of the pair in $HP(S)$.
The former one costs $2\log|E|$ bits, while the latter costs $\log{|HP(S)|}$ bits,
which is always no larger than $2\log|E|$.
We call this method as \textbf{inclusive} strategy.

There is an alternative \textbf{exclusive} strategy: suppose $EP(S)$ covers more than
half of pairs in $HP(S)$, instead of transmitting all pairs belonging to $EP(S)$,
we can only transmit pairs excluded from the set. Under this circumstance, the exclusive
strategy goes more efficient.

Combining both inclusive and exclusive strategies, the cost of transmitting entity pairs
given a schema is defined as \eqnref{eqn:costd}:
\begin{equation}
\label{eqn:costd}
\begin{aligned}
cost(EP(S)\, |&\, S) = 1 + \log|HP(S)| \cdot   \\
    &\min\{|EP(S)|,\, |HP(S)\, \backslash\, EP(S)|\},
\end{aligned}
\end{equation}
\noindent
where it cost extra 1 bit to describe whether to use inclusive or exclusive strategy.

%>>>>4. Removing assignment and define f.
Finally, merging equations together, we get the full definition of $cost(EP, S)$,
which guides the paraphrase system picking a suitable searching direction
in the schema generation step.
%Finally, we define the MDL-based cost function $f$
%over a schema and entity pairs as follows:
%\begin{equation}
%\label{eqn:f}
%f(EP,\, S) = cost(S) + cost(EP(S)\, |\, S).
%\end{equation}
%As we can see, minimizing the whole cost $Z$ in our paraphrasing problem (see \defref{def:pp})
%is equivalent to minimizing the transmitting cost in \eqnref{eqn:mdl}.
%In the next section, we will discuss the method to generate candidate schemas
%and necessary information ($|HP(S)|$ and $|EP(S)|$) used in the cost function.

%
%%>>>>4. state "assignment matrix" A, because one schema is not enough, we need to assign tasks.
%    % 1-to-many, we will explain why later
%    % use a diagram to show "hit" v.s. "assign"
%    % formal def. of assign
%    % Then we have f(EP, S) = min_A \in A_space {cost(S) + cost(Grp_A(S) | S)}
%In order to properly define the cost function $f$ based on MDL principle, we are going to
%explore what does a single schema contributes to the transmitting cost in \eqnref{eqn:mdl}.
%Since each schema in $OS$ is independent of each other, therefore
%$cost(OS) = \sum\nolimits_{S \in OS} cost(S)$, which is the summation cost of transmitting each schema.
%
%Now we focus on $cost(EP\, |\, OS)$. Due to the fact that a single schema is not able to cover
%all pairs in $EP$, we assign each schema with a small part of entity pairs,
%such that each pair will be described by some schemas.
%
%\begin{defn}
%\textit{Assignment Matrix}
%
%Let $EP$ be entity pairs $\{ep_1, ..., ep_n\}$, $OS$ be output schemas $\{S_1, ..., S_m\}$.
%An \textit{assignment matrix} $A_{n \times m}$ is a 0/1 matrix satisfying:
%\begin{itemize}
%    \item[-] $A_{ij} = 1 \Rightarrow ep_i \in HP(S_j)$,
%    \item[-] $\forall ep_i, \exists S_j$, such that $A_{ij} = 1$.
%\end{itemize}
%Besides, we define $EP_A(S)$ as all entity pairs assigned to $S$.
%\end{defn}
%
%Given an assignment matrix $A$, the cost of transmitting entity pairs is
%made up of each schema transmitting its own pairs:
%$cost(EP\, |\, OS) = \sum\nolimits_{S \in OS} cost(EP_A(S)\, |\, S)$.
%
%\KQ{figure again, different assignments.}
%Figure b shows two example assignments on ``parentOf'' entity pairs.
%\KQ{Note that one entity pair can be assigned to more than one schema.
%We will explain why later.}
%As we can see in the example, more than one assignment is available.
%Let $\mathbb{A}$ be the space of all assignments, we define $\hat{A}$ as
%the assignment at minimum cost:
%\begin{equation}
%\hat{A} = \arg\underset{A \in \mathbb{A}}{\min} \sum\limits_{S \in OS} cost(S) + cost(EP_A(S)\, |\, S)
%\end{equation}
%Therefore, we define MDL-based cost function $f$ based on the assignment bringing
%the minimum cost, as follows:
%\begin{equation}
%f(EP,\, S) = cost(S) + cost(EP_{\hat{A}}(S)\, |\, S)
%\end{equation}
%
%
%
%
%%>>>>5. summary
%In summary, we follow the MDL principle and define the cost function of a schema
%describing entity pairs. In the next section, we model our paraphrasing problem
%as a task of integer linear programming.

%
%The cost of transmitting entity pairs given schemas is defined as Equation 3.
%The whole cost is the sum of transmitting each individual entity pair.
%For each $\langle e_1, e_2 \rangle,$ if it's hit by at least one schema in $\vec{S}$,
%the schema with smallest hits is chosen to transmit the pair;
%otherwise, we transmit $e_1$ and $e_2$ directly.
%Besides, we need extra $\log{(|\vec{S}|+1)}$ bits to indicate which schema it chooses or not.
%
%\begin{equation}
%\begin{aligned}
%    & cost(EP|\vec{S})= \sum\limits_{ep \in EP} cost(ep | \vec{S}), \\
%    & cost(ep | \vec{S}) = \log{(|\vec{S}|+1)} + \\
%    & \left\{
%        \begin{aligned}
%        & \min_{S : ep \in HP(S)} & \log{|HP(S)|} &    & \exists S \in \vec{S}, ep \in HP(S)\\
%        &                         & 2 \log{|E|}   &    & \text{otherwise}  \\
%        \end{aligned}
%    \right. \\
%\end{aligned}
%\end{equation}

%
%The intuition that a good $S$ can reduce the cost of transmitting $EP$ is,
%the entity pairs are highly related with the hitting pairs of $S$, such that we
%do not need to transmit every entity in $EP$, instead, we just transmit constraints
%to control which pairs are transmitted from its hitting set $HP(S)$. \KQ{I'm not sure the word
%``constraint'' is suitable or not}
%
%For example, suppose user has already get $S$ \KQ{we use a true graph $S$ representing ``mother of'' relation},
%and the user want to transmit $\langle Bill Gates, Mary Gates\rangle$ in $EP$. Since $HP(S)$ has the pair,
%specifying ``subject is $Bill Gates$'' is informative enough to return the pair in $HP(S)$.
%
%%Given an entity pair $\langle e_1, e_2 \rangle$, the the intuition of reducing transmission
%%cost is that, with the knowledge of $S$ and $KB$,
%
%The intuition that a good $S$ can reduce the cost of transmitting $EP$ is,
%given an entity $e$, if a ground graph $G$ generated from $S$ has $e_{subj} = e$,
%then the object argument $e_{obj}$ is likely to form a pair with $e$ in $EP$.
%Similar intuition holds when $e$ occurs in the object side.
%%Alright, that sentence is so long......
%
%For example, suppose a group $\{ \langle e', e_1 \rangle, ..., \langle e', e_{20} \rangle \}$ with
%the same $e'$ at the subject side.
%If $S$ generates 20 ground graphs with $e_{subj} = e'$, hitting every pair in the group, then
%we can transmit entity $e'$ only, instead of 21 different entities.
%Even if $S$ generates 25 related ground graphs, where 5 hits are not in $EP$, the entities
%for transmitting is 6, which is still much fewer than original 21 entities.
%
%We define the cost of transmitting a pair group as below:
%\begin{itemize}     % we can add a running example figure illustrating these notions
%  \item[*] A pair group $grp$ is a set of pairs $\{\langle e', e_1 \rangle, ..., \langle e', e_n \rangle\}$
%  sharing subjects, or $\{\langle e_1, e' \rangle, ..., \langle e_n, e' \rangle\}$ sharing objects.
%  The side of sharing entity $e'$ is called \textbf{sharing side}, the other side called \textbf{different side}.
%  \item[*] A hit group $hgrp(S, grp)$ is a set of entity pairs $\langle e_{subj}, e_{obj} \rangle$
%  hit by $S$, where all $e_{subj} = e'$ if $grp$ shares subject, or vice versa.
%  \item[*] $ED(grp)$ is the set of \textbf{E}ntities at the \textbf{D}ifferent side of $grp$.
%  \item[*] $EH(S, grp)$ is the set of \textbf{E}ntities at the different side
%  of \textbf{H}it group $hgrp(S, grp)$.
%  \item[*] The cost of transmitting the pair group given $S$ is:
%\end{itemize}
%
%\begin{equation}
%\begin{aligned}
%cost & (grp|S) = \{\min \{|grp|, |ED(grp) \cup EH(S, grp)| \\
%     & - |ED(grp) \cap EH(S, grp)|\} + 1\} * |E|
%\end{aligned}
%\end{equation}
%
%
%Learnt from Equation 3, for one $grp$, the sharing entity $e'$ must be transmitted.
%For the remaining part, we have two strategies:
%transmitting entities at different side one by one,
%or only transmitting part of entities not in ED or EH.
%
%$cost(EP | S)$ is defined as Equation 4.
%We assign each pair in $EP$ into groups (could be more than one), which satisfies that
%each relation group shares one subject or object. For one assignment, the bits of transmitting
%$EP$ equals to the summation cost of transmitting each group.
%Therefore, the cost of transmitting $EP$ is the minimum cost among all possible assignments.
%
%\begin{equation}
%cost(EP | S) = \min_{ass \in assSpace} \sum\limits_{i=1}^{|ass|} cost(grp_i | S)
%\end{equation}
%
%% Add formula here.
%% Therefore, this is really a searching problem.
%

