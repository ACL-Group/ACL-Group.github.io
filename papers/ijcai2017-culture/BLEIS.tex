\subsection{Bilingual Lexicon Extraction for Internet Slangs}
\label{sec:bleis}
In this section, we evaluate our model on the second task. This task aims to translate, define or understand the meaning of emerging Internet slangs from Chinese to English and from English to Chinese. We first construct the ground truth and then implement several baseline methods for comparison. Finally, we analyze the experimental results quantitatively and qualitatively.

\subsubsection{Data Preparation}
We use an online Chinese Internet slang glossary\footnote{\url{https://www.chinasmack.com/glossary}} consisting of  200 popular Chinese Internet slangs with English explanation. For English slangs, we resort to another slang dictionary and crawl their word list\footnote{\url{http://onlineslangdictionary.com/word-list/}} as well as explanations. We randomly downsampled the list to 200 English slangs.
\subsubsection{Ground Truth}
To evaluate the performance of our model on automatically translating and explaining slangs in another language, 
we propose to build the ground truth based on above-mentioned explanations of slangs. Since the subtle and latent semantics of Internet slangs are too difficult to translate without losing any information, exact translation of them are always missing from all the dictionary. Thus, we argue that using the relevant terms in the explanation as the target word list and then computing the similarity between the results and the list is a better approach to evaluate bilingual slang lexicon induction system.
In the following example, we manually selected and annotated words from original glossary that are related to the meaning of the slang, constructing the ground truth:
\begin{description}
	\item[二百五] A \textbf{\textit{foolish}} person who is lacking in sense but still \textbf{\textit{stubborn}}, \textbf{\textit{rude}}, and \textbf{\textit{impetuous}}.
\end{description}

%what are the good word translation that best preserve and convey the meaning, sentiment tendency and usage context ot the original slang. 
%However most of the slangs possess most subtle senses that are related to native culture background, general characteristic and language style of netizens in different language worlds. 
%Thus it is really difficult, if not impossible, to find a exact word in another language that carries the exact same meaning with the original slang. 
%Because of this, our ground truth does not pursue exact Internet slang translation from one language to another's corresponding slang, since most likely such slang does not exist yet. 
%Our aim for the task is using IV (in-vocabulary) normal related words in target language to describe and translate the OOV (out-of-vocabulary) slang words in another language, which is more viable and feasible, and easier for people in different culture/language to understand not just literal meaning but the deeper buried and more subtle sense and context it conveys.
%
%Following this goal, we could build our ground truth based on filtered glossary. For Chinese slangs, we tokenize and lemmatize the definition sentences in English and manually remove the stop words that does not contribute to the meaning of a specific slang, left with a list of English words that have either same meaning or high relatedness to the Chinese slang.
%The same process is applied to English slang glossary as well, with the only difference is that due to the lack of direct Chinese definitions, we have to use Google Translate to translate the definition sentences to Chinese and human annotators tokenize, filter and paraphrase the definitions into Chinese word lists, resulting in the ground truth of slangs in the same format as the Chinese ones.
%Note that we do not manually add word translations by ourselves, we only delete irrelevant words for later evaluation, thus lowering the human error, cross-cultural and bilingual requirement to the minimum.

\subsubsection{Baseline and Our Methods}

We proposed several baselines regarding Internet slang translation. 
One type of baseline for translation comparison are from Internet translators. Google, Bing and Baidu are all well-known online translators, thus with our test set's slang as input, we retrieve the output of translation.
An additional baseline specific to Chinese slangs is from CC-CEDICT\footnote{\url{https://cc-cedict.org/wiki/}} (CC), an online public-domain Chinese-English dictionary, which is well updated with some popular slang inside.

Except for simply using Linear Transform (LTrans) to find the most cross-lingually similar words of the given slang, we propose a strong baseline leveraging our \textit{bilingual lexicon} (BLex). Given an Internet slang of one language, for each common word of target language in our bilingual lexicon, we obtain its translations back into the the source language and then calculate the word similarities between the input slang and aforementioned translation word(s) within monolingual word vector. 
%Now over 20,000 words in the other language of the bilingual lexicon have their corresponding similarity score to the given slang. 
A word may have multiple possible translation words in the other language. In this case, we choose to take average over all of them in terms of similarity score.
We then rank the words by their scores and take top 5 words to form a word set, while other online translation baselines directly produce a word set for later comparison with the ground truth word set. 

Our method simply uses top 5 most similar words (in target language) with the given slang in SocVec space.
\subsubsection{Experimental Results}
In~\tabref{tab:bleis_3}, we present several examples of translation results for Chinese and English slangs with their explanations from the glossaries. Our results are highly correlated with these explanations and capture their core semantics, whereas most online translators are inadequate for extracting subtle meanings of such slangs. They often give just literal meanings as translation or nothing.
\begin{table*}[th]
	\footnotesize
	\centering
	\caption{\small Slang Translation Examples}
	\begin{tabular}{|L{1.2cm}|L{6.1cm}|L{1.7cm}|L{1.7cm}|L{1.7cm}|L{3cm}|}
		\hline
		\textbf{Slang} & \textbf{Explanation} & \textbf{Google}& \textbf{Bing}& \textbf{Baidu} & \textbf{Ours} \\ \hline \hline
		浮云 &something as ephemeral and unimportant as ``passing clouds''& clouds& nothing& floating clouds & nothingness, illusion \\ \hline
		水军 &``water army'', referring to people paid to post comments on the Internet to help shape public opinion by slandering competitors and promoting themselves & Water army& Navy& Navy & propaganda, complicit, fraudulent\\ \hline
		城管 & ``City administrators'', who enforce city regulations, with poor reputation as being corrupt and violent, best known for physically bullying illegal street peddlers & urban management& urban management& urban management & terrorist, rioting, threaten\\ \hline \hline
		floozy & a woman with a reputation for promiscuity & floozy&劣根性 (depravity)&荡妇(slut)&骚货(slut),妖精(promiscuous)\\ \hline
		fruitcake & a crazy person, someone who is completely insane & 水果蛋糕 \quad(fruit cake)&水果蛋糕 \qquad(fruit cake)&水果蛋糕 \quad(fruit cake)& 怪诞(bizarre),令人厌烦(annoying)\\ \hline
		nonce &  A person convicted (or simply guilty) of sexual crimes, especially pedophilia. Or a common British insult regardless of the tendencies of the person &随机数 (random numbers)&杜撰 (fabricate)&杜撰 (fabricate) & 伤风败俗(immoral),十恶不赦(extremely evil),畜类(beast),令人发指(heinous)\\ \hline
	\end{tabular}
	\label{tab:bleis_3}
\end{table*}

Additionally, we take a step forward to directly translate between English slangs and Chinese slangs by simply filtering out common words in the original result. Examples are shown in~\tabref{tab:bleis_4}. 

To quantitatively compare our methods with the baselines, we need to measure the similarity between the translation word set and the ground truth word set. 
Jaccard similarity coefficient is too strict to capture valuable relatedness between two word sets, since it takes only exact matches into account.
We argue that average cosine similarity (ACS) between two sets of word vectors is a better metrics to evaluate similarity between two word sets. The following equation illustrates such computation, where $A$ and $B$ are the two word sets, $\mathbf{A_i}$ and $\mathbf{B_j}$ denotes the word vector of the $i^{th}$ word in $A$ and $j^{th}$ word in $B$ respectively. 
\small
\begin{equation*}
ACS (A,B)=
{\frac{1}{|A||B|}}{\sum_{i=1}^{|A|}{\sum_{j=1}^{|B|}} \frac{\mathbf{A_i }\cdot \mathbf{B_j}}{\|\mathbf{A_i }\|\|\mathbf{B_j }\|}} 
\end{equation*}
%We can see that our \textit{SocVec} model outperforms other approaches by a large margin.
Experiment results of Chinese and English online slangs translation are in~\tabref{tab:bleis_1} and~\tabref{tab:bleis_2}.
Typically, the performance of online translators in translating slangs depends on the a number of human-set rules and supervised learning on well-annotated parallel corpora. However, such parallel corpora are rare and costly, especially for social media where Internet slangs emerge the most. It could be a possible reason why they do not perform well. Linear transformation model is trained on translation pairs with high confidence in the bilingual lexicon, which contains no Internet slangs and almost no opinion-related words. Thus, the matrix transforms slangs badly.
\textit{BL} is competitive for its similarity is based on monolingual word similarity, while it is limited by the bilingual lexicon and consequently loses the information from the related words which are not in the lexicon. 

Our~\textit{SocVec} utilizes comparable English and Chinese social media corpora and encodes the context and usage of a given slang by computing its similarities with opinion and sentiment words in the socio-linguistic vocabulary of the source language. Therefore, our model keeps the cross-cultural socio-linguistic features, which is a most important reason why we outperform baselines.
%the best among all the baseline methods. 
%Then, we are able to find the most similar counterparts in the target language by computing the similarity in \textit{SocVec} space through \textit{BSL}. 
%Therefore, our performance is better than the others.    
\begin{table}[th]
	\small
	\centering
	\caption{\small Slang-to-Slang Translation Examples}
	\begin{tabular}{|C{1.92cm}|L{3.0cm}|L{2.2cm}|}
		\hline
		\textbf{Chinese Slang} & \textbf{English Slang} & \textbf{Explanation} \\ \hline
		萌 & adorbz, adorb, adorbs, tweeny, attractiveee & cute, adorable \\ \hline
		二百五 & shithead, stupidit, douchbag & A foolish and senseless person\\ \hline
		鸭梨 & antsy, stressy, fidgety, grouchy, badmood & stress, pressure, burden \\ \hline
	\end{tabular}
	\label{tab:bleis_4}
\end{table}

\begin{table}[th] 
	\small
	\centering
	\caption{\small ACS Result of Slang Translation}
	\begin{subtable}[h]{\columnwidth}
		\small
		\centering
		\begin{tabular}{|c|c|c|c|c|c|c|}
		\hline
		Google&  Bing& Baidu & CC & LTrans  & BLex  & SocVec \\ \hline
		18.24 &  16.38&  17.11 & 17.38& 9.14&  20.92& \textbf{23.01} \\ \hline
		\end{tabular}
	 	\caption{\small Chinese Slang Translation}
		\label{tab:bleis_1}
	\end{subtable}
	\vfill
	\begin{subtable}[h]{\columnwidth}
		\small
		\centering
		\begin{tabular}{|c|c|c|c|c|c|}
			\hline
			Google &  Bing & Baidu & LTrans & BLex  & SocVec  \\ \hline
			6.40 &  15.96 &  15.44 &  7.32 &  11.43& \textbf{17.31} \\ \hline  
		\end{tabular}
		\caption{\small English Slang Translation}
		\label{tab:bleis_2}
	\end{subtable}
\end{table}
