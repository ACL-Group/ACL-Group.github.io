\section{Conclusion}
In this paper, we conclude that the cultural properties and usages of a term (including named entities and slangs) can be effectively represented by its similarities to socio-linguistic words. 
Bilingual socio-linguistic lexicon enables two incomparable monolingual semantic spaces to be comparable with each other. 
Our proposed framework can assist cross-cultural social studies and cross-lingual linguistic research, such as detection of cross-cultural differences in named entities and extraction of bilingual lexicon for Internet slangs.
