\section{Related Work}
\label{sec:related}

Most existing approaches for learning cross-lingual word representations  
rely on expensive parallel corpora with word or sentence 
alignments~\cite{Klementiev:2012uk,kovcisky2014learning} or  a 
supervised model to learn a transformation matrix between two monolingual 
vector spaces~\cite{Mikolov:2013tp}. 
These work aims for  improving monolingual tasks and cross-lingual 
document classification, which does not require cross-cultural signals. However, they fails to  capture socio-linguistic information. We propose an uncomplicated framework to quantify the cross-cultural differences by leveraging publicly accessible resources such as Bing Translator and OpinionFinder lexicon.
%Our work uses a bilingual 
%lexicon, socio-linguistic vocabularies and comparable documents.
%all of which are publicly available and easy to obtain. 
%Actually there are other interesting culture related applications such as the
%two tasks presented in this paper, which require special treatment from 
%sociolinquistic point of view.
%However, none of existing methods worked on it specifically. 

Cross-cultural studies have been conducted in 
sociology, anthropology and psychology for many years. Recently, some researchers propose 
studying cross-cultural analysis through text mining and natural language processing. 
Nakasaki et al. \shortcite{nakasaki2009visualizing} and 
Elahi et al. \shortcite{elahi2012examination} show that 
User Generated Content (UGC) like microblogs, 
is a valuable resources to cross-cultural analysis. 
The most relevant work to our first task is Pennebaker 
et al.~\shortcite{PennebakerIdentifying}, which studies the cross-cultural 
differences in word usage between Australian and American English through 
socio-linguistic features. Nevertheless, their supervised model are dependent on large volume of training data and limited to identifying differences of monolingual word usage. To the best of our knowledge, we are among the first to focus on cross-cultural differences in named entities and to propose an effective unsupervised approach.
%While their supervised model is not generalizable to  cross-lingual differences and differences of named entities. 
%Their results show that socio-linguistic vocabulary 
%are essential in cross-cultural analysis of text. 
%However, their research, which uses traditional topic modeling and SVM classifier,
%cannot be applied directly to cross-lingual tasks. To our knowledge, 
%we are the first to focus on cross-cultural differences on named entities and 
%to propose an approach to conduct cross-lingual cross-cultural social studies 
%through vector representation of words. 

Previous work about Internet slangs mainly focuses on automatic 
discovering of slangs~\cite{elsahar2014a} and normalization of noisy texts ~\cite{han2012automatically}. However, research on automatic 
translation and explanation for slangs in another language is missing from literature. 
Our work on the second task fills the void  by directly computing cross-lingual 
similarities to find the most related words in another language.

