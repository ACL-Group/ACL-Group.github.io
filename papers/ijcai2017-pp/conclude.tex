\section{Conclusion}
This work mines the equivalence between natural language relations and structured 
knowledge known as schemas.  It generalizes the simple path representation 
by adding constraints along the path and thus support more complex semantics.
Experiments show that schema representation is able to describe the concrete 
and precise semantic meaning. Schemas thus learned have higher quality than
those learned by existing rule induction approaches. Our approach can also be applied
to the traditional knowledge base completion problem and yield good results. 
%and performs as well as the state-of-the-art methods on ordinary relations.
%To the best of our knowledge, this is the first attempt to directly model complex NL relations in knowledge bases.
%We have observed that this data-driven approach benefits from 
%increasing amount of training data, but it's sensitive to the noisy pairs in the input.
%Future research may aim at designing a self-adaptive algorithm to detect and filter noisy data
%from relation instances, and leverage the structural knowledge to improve performance of QA systems.
%
%Future direction of this research includes schema inference from
%small data sets, efficient generation of
%more complex schemas, including comparative and aggregative constraints, 
%and the exploration of other features during
%schema weighting.
%
