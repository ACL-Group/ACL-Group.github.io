\section{Schema Features}
\label{sec:feature}

%3 sents: talk about "element", encoding, 3 aspects
%As what we described in \secref{sec:problem}, 
A schema is composed of predicates, entities and variables, where
the first two are called ``lexical elements'' of a schema or elements
in short.
%composition of elements: predicates, entities, types and variables.
In this paper, we extract {\em association features}, {\em semantic features},
and {\em structural features} from the schemas and their ground graphs 
in the KB, the input relation instances as well as NL relation patterns.
%3 aspects: the association between elements and relation words,
%the similarity between the relation and element description, and
%the structural information of a schema.
%The first 2 parts focus on single element, and the last one is in
%the view of the entire schema.

%3 sents: association, gives the intuition of assoc, and example
\textbf{Association Features} are formed by pairwise combining 
the schema elements with the words in the relation pattern.
%Each entity, type and predicated has a unique identifier in the
%knowledge base. 
This feature allows us to capture the association between 
KB nodes and NL words.
For example, the feature ``starring | film.performance.actor''
indicates the association between word ``starring'' and a Freebase
predicate linking a film performance to an actor. The training model 
is expected to learn a significant weight for this feature.

%5 sents: WN & w2v feature, and talk about positional encoding
\textbf{Similarity Features} computes the lexical and 
semantic similarity between schema elements and 
relation pattern words. For lexical similarity, 
We collect all words in an element into a set and augment it
with the WordNet synonyms and derivational forms of those words.
We then compute the Jaccard similarity between this augmented set of
words with the set of words in the relation pattern, 
and use it as a feature value. For semantic similarity,
we use the word2vec representation of words. 
The similarity between an element 
and the pattern is then computed as the maximum cosine
similarity between a word from the element and a word from the relation
pattern.

%5 sents. structural feature: EP, HP, size, constraints
\textbf{Structural Features} include the length of skeleton, 
the total number of constraints on the schema, and the support of
the schema. In addition, two more features are extracted to capture
the difference betwee a schema and its skeleton (a more general schema). 
One is the ratio between the support of the schema and its skeleton, 
which shows how much positive evidence it loses 
when additional constraints are attached to a skeleton.
The other one is the ratio of the KB coverage of the schema 
over the KB coverage of its skeleton, indicating how specific the
schema is.

