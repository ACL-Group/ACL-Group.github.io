\section{Conclusion}

In this paper, we propose an unsupervised framework ExtRA for 
extracting the most prominent aspect terms about a type of product 
or service from user reviews, which benefits both qualitative and 
quantitative aspect-based review summarization.
Using WordNet as a backbone, and by running personalized page rank
on the network, we can produce aspect terms that are both important
and non-overlapping. Results show that this approach is more
effective than a number of other strong baselines.
%We find that directly performing topic modeling alone is 
%not adequate to attack this problem because user reviews tend to
%switch aspects very quickly within a short text and the 
%topics are not balanced. 
%The proposed unsupervised framework ExtRA solves this problem by slicing
%the review documents into sentences and then clustering them in 
%sentence level and topic level respectively. 
%Finally we design word ranking algorithm and propose our 
%AspVec to represent the semantics of each aspects, 
%so that we can extract both aspect terms and phrases by 
%computing the semantic similarities. Extensive experimental results 
%show that our approach outperforms many baseline models.
%Although LSTM-based RNN has been used widely in various 
%sentiment analysis tasks like sentiment prediction, we find out that
%paragraph vector is more effective in our tasks.

%As for the future work, we believe improving the representation models for sentence and topics could offer more improvement.
%The proposed framework ExtRA can be applied more downstream applications if more information is provided, such as detecting user communities with similar preferred review aspects. 
%Also, ExtRA  can be modified to extract aspect terms for many other domains, like question answering forums.
%% 
%we first made a general introduction to sentiment analysis, 
%talking about the motivation and potential power of such research; 
%we talked about the tasks in sentiment analysis, including classification, 
%extraction and summarization; we introduced several popular methods for 
%sentiment analysis, including linguistics features, topic model and its 
%variations, and finally deep learning for natural language processing, 
%with an emphasis on language modeling and sentence representation. 
%We then focus on the problem we try to tackle in this paper, that is, 
%aspect extraction for aspect-based review summarization; 
%we explained why this task is difficult and why models like LDA won't work well. 
