%%%% ijcai18.tex

\typeout{IJCAI-18 Instructions for Authors}

% These are the instructions for authors for IJCAI-18.
% They are the same as the ones for IJCAI-11 with superficical wording changes only.

\documentclass{article}
\pdfpagewidth=8.5in
\pdfpageheight=11in
% The file ijcai18.sty is the style file for IJCAI-18 (same as ijcai08.sty).
\usepackage{ijcai18}

% Use the postscript times font!
\usepackage{times}
\usepackage{xcolor}
\usepackage{soul}
\usepackage[utf8]{inputenc}
\usepackage[small]{caption}
\usepackage{graphicx}

\usepackage{CJKutf8}
\usepackage{amsmath,amsfonts,amsthm}
\usepackage{booktabs}
\usepackage{url}
\usepackage{color}
\usepackage{latexsym}
\usepackage{epsfig}
\usepackage{graphicx}
\usepackage{booktabs}
\usepackage{array}
\usepackage{multirow}
\usepackage{multicol}
\usepackage{hhline}
\usepackage{pbox}
\usepackage{threeparttable}
\usepackage{epstopdf}
\usepackage{amsmath}
\usepackage{amssymb}
\usepackage{amsthm}
\usepackage{subfigure}

\usepackage{amsmath}
%\usepackage{algorithm}
\usepackage[noend]{algpseudocode}

\usepackage{diagbox}
\usepackage{paralist}
\usepackage[font=normalsize]{caption}
\usepackage{color}
\usepackage{epsfig}
\usepackage{amsfonts}
\usepackage{graphicx,color}
\usepackage{color}
\usepackage{ragged2e}
\usepackage[ruled,vlined,boxed,linesnumbered]{algorithm2e}
\usepackage{xcolor}

\usepackage{CJK}
\usepackage{tikz}
\usepackage{calc}

\usepackage{float}
%\restylefloat{table}
\newcolumntype{P}[1]{>{\RaggedRight\hspace{0pt}}p{#1}}
\newcolumntype{L}[1]{>{\raggedright\let\newline\\\arraybackslash\hspace{0pt}}m{#1}}
\newcolumntype{C}[1]{>{\centering\let\newline\\\arraybackslash\hspace{0pt}}m{#1}}
\newcolumntype{R}[1]{>{\raggedleft\let\newline\\\arraybackslash\hspace{0pt}}m{#1}}

\newcommand{\ZY}[1]{\textcolor{blue}{Zhiyi: #1}}
\newcommand{\KZ}[1]{\textcolor{red}{Kenny: #1}}
\newcommand{\FX}[1]{\textcolor{cyan}{Frank: #1}}
\newcommand{\BL}[1]{\textcolor{green}{Bill: #1}}
\newcommand{\SHY}[1]{\textbf{\textcolor{yellow}{Hanyuan: #1}}}

\definecolor{mygray}{gray}{0.6}
\newcommand{\red}[1]{\colorbox{red}{#1}}

\newcommand{\secref}[1]{Section \ref{#1}}
\newcommand{\figref}[1]{Figure \ref{#1}}
\newcommand{\eqnref}[1]{Eq. (\ref{#1})}
\newcommand{\exref}[1]{Example \ref{#1}}
\newcommand{\algoref}[1]{Algorithm \ref{#1}}
\newcommand{\tabref}[1]{Table \ref{#1}}
\newcommand{\socvec}{SocVec}
\newcommand{\argmin}{\operatornamewithlimits{argmin}}
\newcommand{\argmax}{\operatornamewithlimits{argmax}}
\newtheorem{example}{Example}
\newtheorem{lemma}{Lemma}
\newtheorem{definition}{Definition}
\newcommand{\cut}[1]{}

\newcommand{\li}{\uline{\hspace{0.5em}}}

\newcommand{\bi}[1]{\textbf{\textit{#1}}}

% the following package is optional:
\usepackage{latexsym} 

\title{Supplementary Content}
\begin{document}
	\maketitle
	
	\subsection*{Evaluation Dataset}
	For each category, we ask 5 annotators who are proficient in English to give 5 aspect terms which they think are important to each product type.
	The annotating principle is to pursue the best coverage of the prominent aspects as well as minimum overlapping between them.
	In total, there are 25 aspect terms for each product type (including duplicates).
	The annotated evaluation dataset is shown in \tabref{table:labels}.
	
	\begin{table}[th!]
		\small
		\centering
		\caption{Evaluation dataset. Each row is provided by one annotator.}
		\label{table:labels}
		\begin{tabular}{|c|l|}
			\hline
			\multirow{5}{*}{hotel}
			& room price location service utility \\
			& room service price food location  \\
			& sleep service room price location  \\
			& location price bedroom bath staff  \\
			& room price bath staff location  \\\hline
			\multirow{5}{*}{mp3 player}
			& price quality sound earphone battery \\
			& carry price design sound screen  \\ 
			& price quality carry earphone sound \\
			& quality price battery sound carry \\
			& price quality sound carry screen
			\\\hline
			\multirow{5}{*}{camera}
			& image lens battery memory carry \\
			& picture lens price battery mode \\
			& image price battery design operation \\
			& image lens battery focus storage \\
			& image appearance lens portability battery \\\hline
			\multirow{5}{*}{mobile phone}
			& brand price quality battery screen \\
			& price quality service touch battery \\
			& quality price design screen carry \\
			& quality price OS battery service \\
			& price quality screen battery color \\\hline
			
			\multirow{5}{*}{laptop}
			& price quality brand OS battery \\
			& quality price battery memory CPU \\
			& disk memory CPU screen keyboard \\
			& price battery screen CPU performance \\ 
			& quality price appearance battery keyword \\\hline
			
			\multirow{5}{*}{restaurant}
			& location price food service cleanness \\
			& food price location environment service \\
			& price food quietness location staff \\
			& food price service environment location \\
			& food price location service environment \\\hline
			
		\end{tabular}
	\end{table}
	
	\subsection*{Hard Accuracy Metric}
	Formally, 
	$Aspects(m, c) = [a_1, a_2, a_3, a_4, a_5]$ denotes the $5$ 
	prominent aspects generated from model $m$ given the category $c$.
	$[l_1, l_2, ..., l_{25}]$ is the 25 ground-truths annotated by
	humans. 
	We formulate the hard accuracy measure (i.e. {\em hacc }) as follows:
		\begin{equation}
	hit(Aspects(m, c), l_i) = 
	\begin{cases}
	1 &  \text{, if $l_i\in Aspects(m, c)$} \\
	0 &  \text{, otherwise}
	\end{cases}
	\end{equation}
	
	\begin{equation}
	hacc(m, c) = \frac{\sum_{i=1}^{25}{hit(Aspects(m, c), l_i)}}{25}
	\end{equation}

	
\end{document}



