\section{Introduction}
\label{sec:intro}
Causal reasoning, the core challenge in artificial intelligence, which aims to understand causal dependency between events, is receiving more and more attention \cite{Pearl2009}.
It is of great interest in many domains, including finance, where financial events predictions can provide significant opportunities for economic benefits. 
In the reasoning process, causal knowledge plays a critical role in people's daily behavior and decision-making \cite{waldmann2013causal}.
%, which is crucial in many natural language research applications, such as event prediction, question answering and so on. It has also aroused great interest in many areas of real life, including finance, where understanding causality between events can provide significant opportunities for economic interests. 
For example, consider the following,
\begin{enumerate}[label=(\alph*)]
	\item If a large disaster happens in a country and this country is rich in certain metal, the price of this metal will rise. \label{intro:natural-language-rule-1}
	\item If the price of some kind of metal rises, the price of the products based on this metal will also go up. \label{intro:natural-language-rule-2}
\end{enumerate}
 Expressed in the form of natural language, the causal knowledge given above is easy for us to understand and has great practical value in real life. For example, if there's an earthquake in Chile, according to rule \ref{intro:natural-language-rule-1}, it is easy to infer that the price of copper will rise. Furthermore, we can infer that the price of the household appliances, such as air conditioners and refrigerators, will also rise via rule \ref{intro:natural-language-rule-2}. 

However, it is difficult for machines to understand such sort of causal knowledge and apply them into causal reasoning. 
These are precisely the challenges we faced : causal knowledge representation and causal reasoning.
%However, it is a daunting task for humans, especially traders, to learn many of these rules and use them for realtime reasoning in the real world. We hope machines can learn these rules automatically and reason quickly with them to help us get rid of the heavy burden of rule learning and realtime massive information processing. we propose to solve what the format of the causal knowledge should be stored into the machine and how to do efficient and automatic reasoning with such kind knowledge.
Many efforts have been made to represent causal knowledge, but with shortcomings. Existing representation schemes\cite{Zhao2017,Luo2016a,Speer2016} using word, phrase or short text to represent the event, are less informative in the rule. For example, (`smoking', Cause, `cancer') represents the specific rather than general causal knowledge, which is less informative since there is no variable or concept in the cause or effect. In \cite{sap2018atomic}, the logic variables used in the cause or effect are in the fixed position, which makes the representation less powerful, e.g., ``if X calls the police, then X needs to dial 911''. Moreover, the unstructured event representation is hard for machines to process. \cite{Radinsky2012} uses clusters of causal event pairs to represent general causal knowledge, where, however, the general knowledge expressed in each cluster is implicit and the boundaries of each cluster are not clear.
%the general causal knowledge in each cluster of causal event pairs is implicit and the boundaries of causal event pairs in each cluster are not clear. %
Different knowledge representations usually require different reasoning methods.
\cite{Radinsky2012} is based on searching for the nearest cluster, which is ambiguous for reasoning. Most other existing works do not need additional reasoning engines since they build direct causal relation between specific events.

In this paper, we propose an informative causal knowledge representation with structured logic rule form, see rule (1), and develop the WoLong system to make financial events prediction by uncertain causal reasoning in Prolog. To the best of our knowledge, there is still no study on causal reasoning with uncertainty in logical form. The demonstration system is available online \footnote{\url{https://adapt.seiee.sjtu.edu.cn/intelligentsia}}.


(rise(Z,price,`',`') :- suffer(`',X,Y,attack), isA(X,country), isA(Y,disaster), isA(Z,product), atLocation(Z,X) conf:0.84 ruleID:936) (1)


%The rule frmat is `上涨/rise'(Z,`价格/price',`',`'):-`遭受/suffer'(`',X,Y,` 袭击/attack'),isA(X,`国家/country'),isA(Y,`自然灾害/ disaster'),isA(Z,`金属/metal'),atLocation(Z,X) conf:0.842 \ \ (1)




%However, it is a daunting task for humans, especially traders, to learn many of these rules and use them for real-time reasoning in the real world.  

%We hope machines can learn these rules automatically and reason quickly with them to help us get rid of the heavy burden of rule learning and real-time massive information processing. 
%In order to achieve this goal, we face two challenges: how to represent causal knowledge in a machine-actionable way and how to automatically acquire a large amount of this type of causal knowledge. We separately explain these two aspects.
	
%	contribution: 
%	rule to do causal reasoning.(fundamental)
%	financial price prediction from text.(applicable)