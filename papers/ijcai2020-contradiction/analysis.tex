\section{Data Analysis}
In this part, two primary data analysis are given for a general dataset overview. 
\subsection{Number of Characters}
% number of characters
We first count number of characters of each phrase in our CoCon dataset and the result is shown in Table \ref{tab:numCharCoCon}. %, which conforms to normal distribution. 
%As our benchmark aimed at %the commonsense contradiction in 
%short phrases, 
The statistics show that all the samples have characters between 2 to 7, and they mostly possess 4 or 5 characters, which demonstrates that our benchmark aimed at short phrases. 

\begin{table}[h!]
	\small
	\centering
	\begin{tabular}{ccccccc}
		\hline
		\# characters &2&3&4&5&6&7 \\
		%Models & Accuracy on CoCon & Accuracy on CoCon testset\\
		\hline
		count&1&394&4678&3006&149&1\\
		\hline
	\end{tabular}
	\caption{Distribution of the phrases by number of characters}
	\label{tab:numCharCoCon}
\end{table}


\subsection{Diversity of Contradiction Type}
% kind of contradiction types
To investigate the contradiction types in our constructed CoCon dataset, we classify all phrases into several categories and show statistics in Table \ref{tab:contradictionType}, %From Table \ref{tab:contradictionType}, 
where we can see that the contradiction types are diverse. Our CoCon dataset cover numerous properties of object, and can evaluate the commonsense knowledge of machine in multiple aspects. Besides, the main contradiction comes from functional, stylistic or material domain, making up 59.7\% of the data. 

\begin{table}[h!]
	\small
	\centering
	\begin{tabular}{m{1.5cm}|c|m{5cm}}
		Contradiction Type & \% & Example \\
		\hline
		Function & 19.6 & figure-slimmer shoe (修身凉鞋) vs hole-prevent shoe (防踢凉鞋) \\
		Style/Design & 25.8 & loose table (宽松桌子) vs double-layer table (双层桌子) \\
		Material & 14.3 & cotten linen biscuits (棉麻饼干) vs whole wheat biscuits (全麦饼干) \\
		User & 8.8 & female child supplementary food (女童辅食) vs baby supplementary food (宝宝辅食)  \\
		Measure & 8.9 & big shelf (大码置物架) vs large shelf (大型置物架) \\
		Space & 2.6 & sneakers at the office (办公室运动鞋) vs sneakers at the gym (健身房运动鞋) \\
		Event & 9.2 & converted shoes (改装凉鞋) vs knitted shoes (编织凉鞋) \\
		Thing & 7.7 & salt coat (海盐外套) vs baseball coat (棒球外套) \\
		Time & 1.9 & summer teakettle (夏季烧水壶) vs winter teakettle (冬天烧水壶) \\
		Other & 1.1 & transparent suit pant (透明西装裤) vs grey suit pant (灰色西装裤) 
	\end{tabular}
	\caption{Contradiction Types in CoCon dataset}
	\label{tab:contradictionType}	
\end{table}



