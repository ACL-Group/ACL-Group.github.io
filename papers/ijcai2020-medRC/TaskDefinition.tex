\section{Problem Formulation}
In few-shot relation classification, there is a training set 
$D_{\rm{train}}$ and a test set $D_{\rm{test}}$. 
Each instance in both sets can be represented as a triple $(s,e,r)$, 
where $s$ is a sentence of length $T$, $e=(e_1, e_2)$ is the entities and 
$r$ is the semantic relation between $e_1$ and $e_2$ conveyed by $s$.
$r \in R$ where $R=\{r_1,...,r_N\}$ is the set of all candidate relation classes.
$D_{\rm{train}}$ and $D_{\rm{test}}$ have disjoint relation sets, i.e., 
if a relation $r$ appears in a triple of the training set, it must not appear 
in any triples of the test set and vice versa.
$D_{\rm{test}}$ is further split into a support set $D_{\rm{test\text{-}s}}$ 
and a query set $D_{\rm{test\text{-}q}}$. The problem is to predict 
the classes of instances in $D_{\rm{test\text{-}q}}$ given 
$D_{\rm{test\text{-}s}}$ and $D_{\rm{train}}$.
In a $N$ way $K$ shot few-shot relation classification scenario, 
$D_{\rm{test\text{-}s}}$ contains $N$ relation classes and $K$ instances 
for each relation. Both $N$ and $K$ are supposed to be small (e.g., 5 way 1 shot, 10 way 5 shot). Few-shot relation classification is hard because 
$D_{\rm{test\text{-}s}}$ is small (totally $N\times K$ instances).
The task is more difficult if $D_{\rm{train}}$ is also small.
