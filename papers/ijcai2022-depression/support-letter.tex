\documentclass[11pt]{article}

\usepackage{graphicx,adjustbox}% http://ctan.org/pkg/graphicx
\usepackage{hyperref}

\setcounter{secnumdepth}{4}
\setlength{\parindent}{0pt}% Remove paragraph indent

% \newcommand{\centered}[1]{\begin{tabular}{l} #1 \end{tabular}}
\def\imagetop#1{\vtop{\null\hbox{#1}}}


\begin{document}

Kenny Q. Zhu, Ph.D.\\
Professor\\
Department of Computer Science and Engineering \\
Shanghai Jiao Tong University \\

\noindent\today 
\hfill 
\resizebox{0.5\linewidth}{!}{\includegraphics{vi/sjtu-vi-logo-blue.pdf}}
\vspace{1cm}


Dear IJCAI Student Volunteer Chairs,
\vspace{0.2cm}

As the research adviser of Mr. Zhiling Zhang who is the first author of a paper accepted by IJCAI AI for Good Track:

\begin{enumerate}
    \item Paper \#AI4G63 - Psychiatric Scale Guided Risky Post Screening for Early Detection of Depression\\
    \textbf{Zhiling Zhang}, Siyuan Chen, Mengyue Wu, Kenny Q. Zhu,
\end{enumerate}
I'm writing to give my full support for his application of the IJCAI 2022 student volunteer program.

Zhiling is currently a second year master of science student at our department. His current research interest is
the application of text classification and generation. 
%Depression is a prominent health challenge to the world, and early risk detection (ERD) of depression from online posts can be a promising technique for combating the threat. Early depression detection faces the challenge of efficiently tackling streaming data, balancing the tradeoff between timeliness, accuracy and explainability. To tackle these challenges, we propose a psychiatric scale guided risky post screening method that can capture risky posts related to the dimensions defined in clinical depression scales, and providing interpretable diagnostic basis. A Hierarchical Attentional Network equipped with BERT (HAN-BERT) is proposed to further advance explainable predictions. For ERD, we propose an online algorithm based on an evolving queue of risky posts that can significantly reduce the number of model inferences to boost efficiency. Experiments show that our method outperforms the competitive feature-based and neural models under conventional depression detection settings, and achieves simultaneous improvement in both efficacy and efficiency for ERD.
The current IJCAI paper proposes a novel ``Risky Post Screening'' approach, which can simultaneously improve 
the efficiency and effectiveness of the early detection of depression online. This will push the boundary of 
research on this important global health issue. Zhiling has also published papers in other top AI conferences like WWW, CIKM, 
ICASSP, but never had the chance to work as a volunteer at these major conferences. It would be a wonderful experience for
him to interact with the student body of IJCAI and also to have the opportunity to give back to the general AI community.
From my past 3 years experience working with him, I consider Zhiling is very responsible, efficient and energetic young man
who is ideal for the volunteer job. 
% give the presentation and have active discussions, which is both beneficial for himself and the audience.

Moreover, given the current pandemic and the economic downturn, our university has completely scrapped 
the financial support for student conference travel. The financial burden for such trips is mounting both for the student
and the research group. Being able to cover the registration fee is a huge help to us at this difficult time. 
Therefore, I would sincerely appreciate it if you could give Zhiling's application full considerations. 

%As a graduate student, he only gets limited stipend from the university, which is difficult for him to cover the registration fee. He has tried best to seek supports from the university and our research group. However, due to the policy adjustment of the university, he is unable to apply for any support now. Therefore, it would reduce his burden greatly if his registration can be covered by the volunteer program.

If there are any questions, please feel free to contact me at \href{mailto:kzhu@cs.sjtu.edu.cn}{kzhu@cs.sjtu.edu.cn}.

\vspace{0.5cm}
Sincerely, \par \medskip

\includegraphics[width=0.2\linewidth]{vi/signature.jpg} \par
Kenny Q. Zhu \par

\vspace{0.5cm}
\begin{flushright}
Department of Computer Science and Engineering \\
Shanghai, P.R. China, 200240 \\
% \href{tel:862134205285}{Tel: +86 21 34205285} \qquad 
E-mail: \href{mailto:kzhu@cs.sjtu.edu.cn}{kzhu@cs.sjtu.edu.cn}
\end{flushright}
\end{document}
