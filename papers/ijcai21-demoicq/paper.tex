%%%% ijcai21.tex

\typeout{IJCAI--21 Instructions for Authors}

% These are the instructions for authors for IJCAI-21.

\documentclass{article}
\pdfpagewidth=8.5in
\pdfpageheight=11in
% The file ijcai21.sty is NOT the same than previous years'
\usepackage{ijcai21}

% Use the postscript times font!
\usepackage{times}
\usepackage{soul}
\usepackage{url}
\usepackage{color}
\usepackage[hidelinks]{hyperref}
\usepackage[utf8]{inputenc}
\usepackage[small]{caption}
\usepackage{graphicx}
\usepackage{amsmath}
\usepackage{amsthm}
\usepackage{booktabs}
\usepackage{algorithm}
\usepackage{algorithmic}
\urlstyle{same}

% added
\usepackage{microtype}
\usepackage{multirow}
\usepackage{multicol}
\newtheorem{example}{Example}
\usepackage{amsmath}
\newcommand{\secref}[1]{Section \ref{#1}}
\newcommand{\figref}[1]{Figure \ref{#1}}
\newcommand{\eqnref}[1]{Eq. (\ref{#1})}
\newcommand{\tabref}[1]{Table \ref{#1}}
\newcommand{\exref}[1]{Example \ref{#1}}
\newcommand{\cut}[1]{}
\newcommand{\KZ}[1]{\textcolor{blue}{Kenny: #1}}
\newcommand\BibTeX{B\textsc{ib}\TeX}
\newcommand{\sgn}{\text{sgn}}
\usepackage{color, colortbl}
\usepackage{bbding}
%\usepackage{subfigure}
\usepackage{subcaption} 
\usepackage{tikz}
\newcommand{\crosssymbol}{{\color{red} \XSolidBrush} }
\newcommand{\checksymbol}{{\color{green} \Checkmark} }
% the following package is optional:
%\usepackage{latexsym}

% See https://www.overleaf.com/learn/latex/theorems_and_proofs
% for a nice explanation of how to define new theorems, but keep
% in mind that the amsthm package is already included in this
% template and that you must *not* alter the styling.
%\newtheorem{example}{Example}
%\newtheorem{theorem}{Theorem}

% Following comment is from ijcai97-submit.tex:
% The preparation of these files was supported by Schlumberger Palo Alto
% Research, AT\&T Bell Laboratories, and Morgan Kaufmann Publishers.
% Shirley Jowell, of Morgan Kaufmann Publishers, and Peter F.
% Patel-Schneider, of AT\&T Bell Laboratories collaborated on their
% preparation.

% These instructions can be modified and used in other conferences as long
% as credit to the authors and supporting agencies is retained, this notice
% is not changed, and further modification or reuse is not restricted.
% Neither Shirley Jowell nor Peter F. Patel-Schneider can be listed as
% contacts for providing assistance without their prior permission.

% To use for other conferences, change references to files and the
% conference appropriate and use other authors, contacts, publishers, and
% organizations.
% Also change the deadline and address for returning papers and the length and
% page charge instructions.
% Put where the files are available in the appropriate places.

%PDF Info Is REQUIRED.
\pdfinfo{
/TemplateVersion (IJCAI.2021.0)
}

\title{ICQ: Visualizing Statistical Label Biases in Discriminative Text Inference Models}

\begin{document}

\maketitle

\begin{abstract}
Recent work has shown that sophisticated neural models may 
take advantage of the label biases in some NLI datasets and 
the model effectiveness may be overrated. 
%However, none of the work has been able to
%easily pinpoint what these cues are. 
%Inspired by black-box test in software engineering, 
%Some human-designed stress tests have been proposed to mitigate this problem
%but they are expensive to create and do not generalize to arbitrary models. 
%\KZ{There seems to be too  much before here.}
We propose a visualization tool, ICQ (I-See-Cue), 
that automatically identifies such biases
in any multiple choice datasets and visually validates
if a model takes advantage of these biases.  Our demo includes 
12 well-known datasets and 3 popular neural models to solve
these tasks for users to try and allows users to upload their own
datasets and models.
\end{abstract}

\section{Introduction}

Protein$-$protein interactions (PPIs) are of central importance for the majority of biological functions, such as signal transduction, metabolic pathways, molecular dynamics, and protein networks\cite{Hoffmann.Krallinger.ea:2005}, for they serve as the most fundamental building blocks of the entire interacademic systems of any organisms. Collecting data on pairwise interaction relationships is essential for multiple purpose, including identification of modules with certain functionality\cite{Spirin.Mirny.03}, mapping diseases to dominated genes\cite{Ideker.Sharan.08}, and after all, understanding wholistic metabolic/genetic networks from a system biology perspective.

A lot of databases have been built to store protein and genetic interactions from major model organism species and are available in various standardized formats, such as MINT\cite{Zanzoni.Montecchi-Palazzi.ea:2002}, BIND\cite{Bader.ea:2003}, BIOGRID\cite{DBLP:journals/nar/StarkBRBBT06}, etc. Among those mainstream databases, the data largely rely on voluntary reports by scientists or researchers, besides, comprehensive curation efforts become indispensable for the sake of accuracy. However, the amount of biology-related literatures with respect to protein interactions grows explosively and thus make it either impossible or impractical to manually detect PPI information anymore.

Considering huge amount of PPI information with great wealth hidden in published papers, in recent years, numerous mining techniques have been proposed that aim to extract PPI information automatically from free text, especially machine learning, information retrieval, and natural language processing\cite{DBLP:journals/bib/WinnenburgWPDS08}.These approaches can be roughly categorized into three classes: co$-$occurrence, rule$-$based, and machine learning. 

Co$-$occurrence is the approach with most simplicity and naivete. Just as its name implies, this method intends to find out pairs of proteins that co-occur in the same context. The scope of "same context" ranges from phrase, sentence, paragraph to whole abstract, even document. The underlying assumption is that whenever two proteins are mentioned together by authors, chances are high that there is some kind of relationship between them. However, however, in-context closeness even semantic relation does not necessarily represent actual biological interaction. As a consequence, a large fraction of candidate pairs are mismatched inevitably, causing a high recall but low precision.

The second approach is rule-based extraction, in other words, pattern matching. There are many types of rules, most of them concern natural language processing (NLP). One way is to specify hand-crafted regular expressions before hand, which mostly lean on language usage preference. Besides, by using full or partial (shallow) parsing strategies, more information would be acquired, such as part-of-speech taggers, local dependencies between syntactic components, context-free grammar\cite{DBLP:journals/bioinformatics/TemkinG03}, and full sentence structure. Compared to co$-$occurrence, rule-based approach enjoy better precision but much lower recall. In addition, since the rules are usually derived from training data, that is to say, the improper choice of training data would be significantly lethal, therefore quality of extraction is invariably instable and may not applicable to other data.

The third and most commonly used approach use machine learning techniques, in this case, the task to extract protein$-$protein interactions turns out to be a binary classification problem. Each protein pairs are represented along with a set of features, which is associated with their context, then a well$-$defined classifier gives the answer whether the candidate protein pairs is classified to be qualified PPI. (TO BE FURTHER FILLED!!!)

In this paper, we introduce a general bootstrapping framework for Protein$-$protein interaction extraction from natural text.Our method differs from most of the previous works in three aspects:

(1)The extraction process is driven by only tiny fraction of training data, which are regarded as seed data. In each round, it would derive reliable patterns automatically from seed data, then extract more positive PPI pairs consequently, what's more, the seed data would be augmented by the newly extracted results with high confidence.

(2)multiple graph kernel. 

(3)various evaluation.




\section{Preliminary}
\label{sec:formulation}
\subsection{Task Definition}
%\textcolor{red}{Hongru: these are concrete examples, which i think should be given after the problem
%formulation} 
We define an instance $x$ of an NLU task 
dataset $X$ as
\begin{equation}
    x = (p, h, l) \in X, \label{eq:nli}
\end{equation}
\noindent
where $p$ is the context against which to do the reasoning ($p$ corresponds 
to ``premise'' in~\exref{exp:snli});
$h$ is the hypothesis given the context $p$; 
$l \in \mathcal{L}$ is the label that 
depicts the type of relation between $p$ and $h$. 
The size of the relation set $\mathcal{L}$ varies with tasks. 

%There is another type of natural language reasoning tasks which 
%are also in the form of multiple-choice questions, 
%but their choices are a fixed set of labels, as shown below. 

%\begin{center}
%\begin{example}\label{exp:roc}
%A story in ROCStory dataset, with ground truth bolded~\cite{mostafazadeh2016corpus}.
%\begin{description}
%\item{Context:} Rick grew up in a troubled household. 
%He never found good support in family, and turned to gangs.           
%It was n't long before Rick got shot in a robbery.             
%The incident caused him to turn a new leaf.
%\item{Ending 1:} He joined a gang. 
%\item{Ending 2:}  \textbf{He is happy now.}
%\end{description}
%\end{example}
%\end{center}
%We can transform the this case into two separate 
%problem instances, still
%in the same form as in \eqnref{eq:nli}, 
%$u_1=(context, ending1, false)$ and $u_2=(context, ending2, true)$, where $L = {true, false}$.

\subsection{Linguistic Features}
\label{sec:extract}

As demonstrated in previous work~\cite{naik2018stress,checklist2020acl}, 
we consider the following linguistic features: 

\indent\textbf{Word}: The existence of a specific word in the premise or hypothesis of a dataset instance.

\indent\textbf{Sentiment}: The sentiment value of an instance, calculated as the sum of sentiment polarities of individual words.

\indent\textbf{Tense}: The tense feature (past, present, or future) of an instance, determined by the POS tag of the root verb.

\indent\textbf{Negation}: The existence of negative words (e.g.,``no'', ``not'', or``never'') in an instance, determined by dependency parsing.

\indent\textbf{Overlap}: The existence of at least one word (excluding stop words) that occurs in both the premise and hypothesis.

\indent\textbf{NER}: The presence of named entities (e.g., PER, ORG, LOC, TIME, or CARDINAL) in an instance, detected using the NLTK NER toolkit.

\indent\textbf{Typos}: The presence of at least one typo in an instance, identified using a pretrained spelling model.

For multiple-choice datasets, all features except Overlap are applied exclusively to hypotheses.

%\subsubsection{Word} 
%For a dataset $X$, we collect a set of all words 
%$V$ that ever exist in $X$. 
%A word feature is defined as the existence of a word $w \in V$
%either in the premise or the hypothesis. 
%Because $V$ is generally very large, in practice, we may narrow it down
%to words that are sufficiently popular in $X$. That is, we may remove
%words that seldom appear in $X$.
%\subsubsection{Sentiment}
%
%For each data instance $x$, we can compute its sentiment value as:
%\begin{equation}
%S(x) = \sgn(\sum_{w \in x} polar(w),
%\end{equation}
%where $polar(w)$ is the sentiment polarity (-1, 0, or 1)
%of $w$ determined by a look-up from a pretrained sentiment 
%lexicon~\footnote{NLTK: \url{https://www.nltk.org}}.
%We say $x$ has a positive/negative/neutral sentiment feature if $S(x)$ = 1, -1 or 0,
%respectively.
%
%\subsubsection{Tense}
%We say that an instance $x$ has  
%\textit{past}, \textit{present} or \textit{future} tense feature if $x$
%carries one of these tenses, respectively, by the POS tag of the root verb
%in $p$ or $h$. 
%
%\subsubsection{Negation}
%Previous work has observed that negative words (``no'', ``not'' or ``never'') 
%may be indicative of a certain label in NLI tasks for some models.
%The existence of a negation feature in $x$ is decided by dependency 
%parsing~\footnote{Scipy: \url{https://spacy.io}}. 
%
%\subsubsection{Overlap}
%In many models, substantial word-overlap between the premise and the
%hypothesis sentences causes incorrect inference, 
%even if they are unrelated~\cite{mccoy2019right}. 
%We define that an overlap feature exists in $x$ if there's at least one word
%(except for stop words) that occurs both in $p$ and $h$. 
%
%\subsubsection{NER}
%We define the NER feature as the existence of either PER,
%ORG, LOC, TIME, CARDINAL entity in $x$.
%We use the NLTK ner toolkit for this purpose. 
%
%\subsubsection{Typos}
%We say an instance $x$ has typo feature if there exists at least one
%typo in $x$.
%We use a pretrained spelling model~\footnote{\url{https://github.com/barrust/pyspellchecker}} 
%to detect all typos in a sentence. We don't distinguish the types of misspellings here. 
%
%In Example \ref{exp:roc}, we noted that multiple-choice 
%questions are split into two instances with opposite 
%labels (T or F) and identical premises. Thus, 
%detecting features within premises alone is unproductive. 
%For MCQ datasets, all features except Overlap are applied exclusively to hypotheses.
%
%

\section{Framework}
%\BF{workflow figure to show the framework}
%\begin{multicols}{2}
%\begin{figure*}
%\centering
%\includegraphics[width=2\columnwidth]{sysoverviewgrapheps.eps}
%\caption{Framework wrokflow} \label{fig:workflow}
%\end{figure*}
%\end{multicols}
%\KZ{In the framework, say ``Output Parse'' instead of ``Output File.''}
The general architecture of the our parser is shown in \figref{fig:workflow}
and is divided into training phase and parsing phase.
%We take training treebank as input, which carries the
%essential information (we only use FORM and POSTAG) and
%gold dependency parses.

\begin{figure}[th]
\centering
\epsfig{file=sysoverviewgrapheps.eps, width=\columnwidth}
\caption{Sequence Based Parser Framework}
\label{fig:workflow}
\end{figure}

{\bf Training:} The preprocessing step generates oracle sequences
from the gold standard parse trees. Only the word forms and the POS tags 
in these parse trees are used. Here, we assume that a child node is
easier to process than its parent node and it is supposed to be attached
before its parent. \footnote{By this rule, multiple gold sequences
can be generated from one dependency tree. In this paper, when a parent node
has multiple children, we generate the sequence by a left-to-right order.}
%\KZ{Which one do we use or do we use all of them?}
%\footnote{
%For example, a bottom-up, breadth-first traversal of the gold parse tree or oracle transition
%process order from Malt Parser are both gold sequences.}
%and further discussion is deferred to Section 4.
%\TJ{maybe they will ask which one is the best; needs some explanations here}
We then train respectively a graph-based head mapper (a.k.a. decoder)
from the gold sequences and the gold parses, and a sequence predictor
from the gold sequences.

{\bf Parsing:} Given an input sentence, the sequence predictor
outputs a feasible decoding sequence, which is a permutation of
the words in the input. For each word in this sequence,
the head mapper returns its best head word according to a scoring function
while employing a cycle detection mechanism.
The process continues until all words in the sentence have found their
heads.
%(except manually introduce ROOT node in dependency parsing).
%For a sentence with $N$ words, the final result consists of ($N+1$) nodes
%and constructed $N$ arcs.
The procedure guarantees to produce a tree structure eventually.
\cut{
We implemented a simple version of this framework,
%and released the source code as well as the evaluation data\footnote{\urlstyle{same}\url{https://github.com/littlebeanfang/BeanParser}}.
%To reproduce the experiments refered in this paper, all our data and related commands are offered in the compressed file.
%\BF{add the data download source}
%\KZ{Besides the open-source system, create an online demo using default model
%and allow users to type in
%a sentence to have it parsed.}
and built an online demo\footnote{\urlstyle{same}\url{http://202.120.38.146/BeanParser}} to show parses of eight languages with the model
trained in our experiment.}

In the current implementation, we generate the decoding sequence by
{\em stackproj} algorithm~\cite{nivre2009non} in
malt parser and scorer-based greedy head mapper.
%\KZ{Consider rephrase this sentence.
%What does graph-based head mapper have to do with sequence?}
%The training and testing data are both in
%CoNLL format~\footnote{http://ilk.uvt.nl/conll/}.


%\section{Demonstration}
% how to evaluate?

% random picking relation, labeling true or false?
% the count precision@x?

% check: 1. GY's paper
%        2. ritter's paper

% Check their page size
% write about our size.
% and compare the version of MI-Equal, MI-Uniform, TfIdf-Equal, TfIdf-Uniform
%%In the demonstration part, we first introduce the experimental setup.
%%Secondly we evaluate the accuracy of relation type inferring.
%%Then we present our web interface of RvSp system, and finally
%%we provide some example relations with the inferred argument types.

\section{Evaluation}
% Add Freebase dump citation
%\KZ{First, say a bit about the ReVerb dataset, and the specifics of
%Freebase. Then say something about our implementation details in the
%3 steps, such as parser we used, etc. What about the numbers in the table?}


Freebase \cite{bollacker2008freebase} is a collaboratively generated knowledge base,
which contains more than 40 million entities, and more than 1,700 real types
\footnote{Freebase types are identified by type id, for example, $sports.pro\_athlete$ stands for ``professional athlete''.}.
In our experiment, We use the 16 Feb. 2014 dump of Freebase as the knowledge
base.

ReVerb \cite{fader2011identifying} is an Open IE system
which aims to extract verb based relation instances from web corpora.
The release ReVerb dataset contains more than 14 millions of relation tuples with high quality.
%Each entity belongs to at least one type.
%When compared with other knowledge bases, Freebase has a much greater focus on named entities than {\tt WordNet}.
%Besides, the type hierarchy of {\tt Yago} is too fine-grained, which is not suitable for schema inferring.
%Considering aspects mentioned above, we adapt Freebase as our knowledge base in our work.
%The input ReVerb dataset is released by Lin et al.\shortcite{lin2012entity}, containing 3 millions of relation tuples with high quality.
We observed that in ReVerb, some argument is unlikely to be an entity in Freebase, for example:

$\langle Metro\ Manila,\ consists\ of,\ \textbf{12 cities}\rangle$,

\noindent
where the object argument is not an entity but a type. Since types are usually represented by lowercase common words,
we remove the tuple if one argument is lowercase, or if it is made up
completely of common words in WordNet.
In addition, because date/time such as ``Jan. 16th, 1981''
often occurs in the object argument while Freebase does not have any
such specific dates as entities,
we use SUTime \cite{chang2012sutime} to recognize dates as an virtual entity.
After cleaning, the system collects 3,234,208 tuples and
171,168 relation groups.

% Talk about entity linking.
%We make the following parameter settings by empirics:
The following parameters are tuned using a development set:
$\tau = 0.667$,
$\epsilon=0.6$, $\lambda = 5\%$ and $\rho = e^{-50}$.
For relation grouping, we use Stanford Parser \cite{klein2003accurate}
to perform POS tagging, lemmatizing and parsing on relations.
%1. data come from Rv 3M
%2. Freebase.
% set tau to be 0.667 as the empirical value
%2. lowercased are removed
%3. remaining xxx relations, and xxx tuples.
% Important part, how to evaluate ?
%All the data sets involved in the evaluation are available at
%\url{http://202.120.38.146/schema/}.


We first evaluate the results of entity linking.
We randomly pick 200 relation instances from ReVerb, and manually
labeled arguments with Freebase entities.
For both naive and ensemble strategy, we evaluate the precision, recall, F1 and MRR score on the labeled set.
An output entity pair is correct, if and only if both arguments
are correctly linked. Experimental results are listed in \tabref{tab:linking_result}.

%We assigned 3 human annotators to judge whether both arguments are linked to correct entities.
%We don't have a gold set for entity linking, but we assume that each unlinked relation tuple corresponds to a linked tuple.
%Therefore, we can approximate the recall of entity linking as:
%\begin{equation}
%recall\ =\ \frac {precision * \#Linked\ Tuples} {\#Relation\ Tuples}
%\end{equation}
%For each strategy, the total number of linked tuples, precision, recall and F1 are listed in \tabref{tab:linking_result}.

\begin{table}[ht]
\small
	\centering
	\caption{Entity Linking Result}
	\begin{tabular}{Ic|c|c|c|cI}
		%\toprule
        \whline
		Strategy & P & R & F1 & MRR \\
        \whline
        Naive    & 0.371 & 0.327 & 0.348 & 0.377 \\
        \hline
        Ensemble & 0.386 & 0.340 & 0.361 & 0.381 \\
        \whline
	\end{tabular}%
	\label{tab:linking_result}%
\end{table}

For the evaluation of relation schema, we first randomly pick
50 binary relations with support larger than 500 from the system.
For each relation, we selected top 100 type pairs with the largest
support, as what we evaluated.
We assigned 3 human annotators to label the fitness score of type
pair for the relation. The labeled score ranges from 0 to 3.
Then we merge these 3 label sets, forming 50 gold standard rankings.
When evaluating a relation schema list from our system,
we calculate the MRR score~\cite{liu2009learning} by the top schemas in the gold rankings.

For comparison, we use Pointwise Mutual Information~\cite{church1990word}
as our baseline model, which is
used in other selectional preference tasks \cite{resnik1996selectional}.
We define the association score between relation and type pair as:
\begin{equation}
PMI(r, tp) = p(r, tp) \log \frac {p(r, tp)}{p(r, *) p(*, tp)}
\end{equation}
Where $p(r, tp)$ is the joint probability of relation and type pair in the whole linked
tuple set, and $*$ stands for any relations or type pairs.


\tabref{tab:precision} shows the MRR scores by using both baseline model (PMI) and our approach.
As the result shows, our approach improves the MRR score by 10.1\%.
%proving that RvSp can find enough correct type pairs.

\begin{table}[ht]
\small
	\centering
	\caption{End-to-end Schema Inference Results}
	\begin{tabular}{Ic|c|c|c|cI}
		%\toprule
        \whline
		Approach & MRR Score \\
        \whline
        PMI Baseline & 0.306 \\
        \hline
        Our Approach & 0.337 \\
        \whline
	\end{tabular}%
	\label{tab:precision}%
\end{table}

%\begin{figure%}[htp]
%\centering \scalebox{0.6}{\includegraphics{eval.eps}}
%%\epsfig{file=figure1-cropped.eps, width=2\columnwidth}
%%\scalebox{0.35}
%\caption{Average precision at different ranks.}
%\label{fig:precision}
%\end{figure}

% We randomly sample K relations, use 3 annotators to annotate whether a type pair is true or not.
% count precision@px

%\subsection{Web Interface}
%In addition, we set up a website \footnote{http://202.120.38.146/rvsp} for users to query the schemas of a binary relation.
%Users can search for type pairs by providing the binary relation alone, or the relation with the type of either arg1 or arg2.
%The interface will output the ranked list of schemas satisfying the input constraint along with its support instances.
%Before querying, the interface will transform the relation pattern, using the method introduced in section 4.

%Due to argument types in RvSp is recognized by Feebase type id, which doesn't match its name exactly, we provide typing suggestion in the web interface, %making users easily enter Freebase types.
%Users can browse Freebase website \footnote{http://www.freebase.com} for detail information about type id.

%
%\\
%\\
%
%\begin{figure}[ht]
%\centering
%\epsfig{file=cropped-demo1.eps, width=0.6\columnwidth, angle=270}
%\caption{Query Interface}
%\label{fig:demo1}
%\end{figure}
%
%
%\figref{fig:demo1} shows the result page.
%User can click ``page up'' and ``page down'' to check more results.
%Besides, for each relation schema, user can click ``detail'' link too check all its support tuples.
%The schema details are shown in \figref{fig:demo2}.
%\\
%\\
%
%\begin{figure}[ht]
%\centering
%\epsfig{file=cropped-demo2.eps, width=0.6\columnwidth, angle=270}
%\caption{Schema Details}
%\label{fig:demo2}
%\end{figure}
%

Finally, \tabref{tab:sample_relation} shows some example binary relations,
and their schemas inferred by our system.  We can see that
with a well-defined type hierarchy, our system is able to extract both
coarse-grained and fine-grained type information from entities,
resulting in a informative type lists.

%
%\begin{table*}[htbp]
%	\centering
%	\caption{Sample Relation Schemas}
%	\begin{tabular}{Ic|l|lI}
%		%\toprule
%        \whline
%		Relation & Arg1 Type & Arg2 Type \\
%        \whline
%        & book.author & book.book \\
%        & book.author & book.written\_work \\
%        be the writer of & tv.tv\_writer & award.award\_nominated\_work \\
%        & people.person & book.book \\
%        & people.person & book.written\_work  \\
%        \hline
%        & fictional\_universe.fictional\_character & tv.tv\_actor  \\
%        & fictional\_universe.fictional\_character & film.actor  \\
%        be play by & fictional\_universe.fictional\_character & people.person  \\
%        & fictional\_universe.fictional\_character & influence.influence\_node  \\
%        & people.person & tv.tv\_actor  \\
%        \hline
%        & organization.organization\_founder & organization, organization \\
%        & people.person & organization, organization \\
%        found & people.deceased\_person & organization, organization \\
%        & organization.organization\_founder & business.business\_operation \\
%        & organization.organization\_founder & business.employer \\
%        \whline
%	\end{tabular}%
%	\label{tab:sample_relation}%
%\end{table*}
\begin{table}[ht]
\small
	\centering
	\caption{Sample Relation Schemas}
	\begin{tabular}{Ic|lI}
		%\toprule
        \whline
		Relation & Top Schemas \\
        \whline
        & $\langle location,\ location \rangle$ \\
        be find at & $\langle employer,\ location \rangle$ \\
        & $\langle organization,\ location \rangle$\\
        \hline
        & $\langle person,\ tv\ program \rangle$\\
        appear on & $\langle person,\ nominated\ work \rangle$\\
        & $\langle person,\ winning\ work \rangle$\\
        \hline
        & $\langle person,\ nominated\ work \rangle$\\
        be the writer of & $\langle person,\ film \rangle$\\
        & $\langle person,\ book\ subject \rangle$\\
        \whline
	\end{tabular}%
	\label{tab:sample_relation}%
\end{table}

%\section{Conclusion}
We implement a novel sequence-based dependency parsing
framework which takes advantage of high order features 
in parsing history. 
%We can also adapt beam search to this framework so as to
%relax the strictly greedy nature. Vine pruning\cite{rush2012vine} could
%be incorporated to speed up the parsing.
More importantly, we discovered that the parsing accuracy is very sensitive to
the quality of parsing sequence. Future work can be focused on
developing better sequence predictors that outperform Malt action classifier.
Furthermore, we use two sets of features for sequence predictor and
head mapper right now. A unified set of features between these two components
are worth exploring.
%Besides, better sequence predicting method and unified feature
%representation of two components are worth exploring.
%
%Though we currently get a not bad result,
%the sequence predictor still needs more exploration.
%According to our experiment, slightly changes
%on the sequence can lead to a fatal decline on accuracy. Ensuring the match degree of training sequence and testing
%sequence demands a high quality of sequence predictor.
%
%Further, the features in our current implementation are not expanded and well tuned yet  and we are free to define high order features to make use of parsing history. Our framework is flexible to merge other technics to enhance the performance. Introducing beam could make up for our greedy decoder and improve our accuracy. Vine pruning\cite{rush2012vine} could speed up parsing process. Besides, better sequence predicting method and unified feature representation of two components are worth exploring.


\bibliographystyle{named}
\bibliography{ijcai21}

\end{document}

