% IJCAI-ECAI 2022 Author's response

% Template file with author's response

\documentclass{article}
\pdfpagewidth=8.5in
\pdfpageheight=11in
\usepackage{ijcai22-authors-response}


\usepackage{times}
\usepackage{soul}
\usepackage{url}
\usepackage[hidelinks]{hyperref}
\usepackage[utf8]{inputenc}
\usepackage[small]{caption}
\usepackage{graphicx}
\usepackage{amsmath}
\usepackage{amsthm}
\usepackage{booktabs}
\usepackage{algorithm}
\usepackage{algorithmic}
\usepackage{lipsum}
\urlstyle{same}

\newtheorem{example}{Example}
\newtheorem{theorem}{Theorem}

\begin{document}

\noindent\fbox{
	\parbox{\linewidth}{
		{\bf IMPORTANT NOTE (DO NOT DELETE)}
		
		The rebuttal to the reviews {\bf must} be restricted to:
		\begin{itemize}
			\item 
			      answering the specific “pressing” questions raised by the reviewers in slot 4 of the reviews; and
			\item
			      pointing out factual errors in the reviews.
		\end{itemize}
		Furthermore, it {\bf must be one page-only}.
		
		The goal of the rebuttal is not to enter a dialogue with the reviewers. Therefore, it is {\bf not allowed to include new results in the rebuttal, provide links to such results, or describe how you might rewrite your paper to address concerns of the reviewers}. It is the current version of your paper that is under review, not possible future versions. 
		
		The program committee will be instructed to ignore rebuttals violating these principles.
	}
}

\section*{Rebuttal}
\subsubsection*{Review 1, Question 1: Was a statistical analysis performed and...}

\textbf{Answer:} We conducted 12 experiments in Table 4: 3 models on 4 different
datasets. According to t-tests, with $p<0.05$, +C+M is significantly more
accurate than (w/o) and +B in the stress test of all 12 experiments.
For original tests, with $p<0.05$, 4 of 12 experiments, 
+C+M is significantly better than (w/o); 
and in 2 of 12 experiments, +C+M is significantly better than +B.
In all other experiments, there are no significant differences between +C+M and
(w/o) or +B.

\subsubsection*{Review 1, Question 2: Can we interpret a performance in he ``choice-only'' condition...}

\textbf{Answer:} In fact, the ``choice-only'' test is proposed to test whether the problems in 
a dataset are easy to solve or not. The performance difference of a certain model 
with the ``choice-only'' test on different datasets can 
partly illustrate the degree of bias contained in the datasets.
%We can't interpret a performance in he ``choice-only'' 
%condition that is better than random guess as a sign of a biased dataset. 
In Figure 5, we can find the ``choice-only'' test performance 
of models drops with our data augmentation methods. It indicates there is less bias in 
the augmented datasets. We enhance model robustness by encouraging 
models to pay more attention to the premise with augmented data. 
As a by-product, we have also alleviated the data bias in the datasets.

%It can be a sign of model bias. The source of model bias can be dataset or 
%the training process of model itself (C Lovering, 2020). 

\subsubsection*{Review 1, Question 3: I don't understand why stress test operator PI works as intended...}

\textbf{Answer:} While the stress operators may not succeed in invalidating
the choice as you pointed out, this seems very unlikely. We sampled 
100 cases, and after PI operation, only one case failed to be invalidated.
In that particular case, and also in the case you pointed out, 
the reasoning is obviously weakened after the modification. 

And since crossover and mutation are operators for data augmentation, 
the modified questions do not need to be strictly correct. 
We also sampled 100 cases for each operator. 95\% of the cases turned out
to be correct. 
%These two operators can generate high-quality data for training. 
%Besides,  mutation is designed to encourage the models the model to pay more attention to the premise of
%questions due to its two very similar choices with minimum changes. 
%Models can't get the bias feature easily only from 
%choices and have to look forward (premise). 
%

\subsubsection*{Review 1, Question 4: I have some general reservations about MCQ for text understanding...}

\textbf{Answer:} 
%Thanks for your suggestion that we should discuss more for the 
%utility of MCQ for reasoning in NLP. 
We totally agree with your concern and it is true that MCQs are prone to
data biases. Unfortunately, existing datasets for commonsense reasoning in NLP 
are almost exclusively in the form of MCQs. We will add some discussion
about this and suggest it as future work in the revised version of the paper.
%thought about what is the best format or method to test reasoning capabilities. 
%Current MCQ test datasets contain lots of bias features or statistical nature cues you mentioned. 
%To remedy the weakness of original test, we propose to test the models with stress tests whose 
%choices are very similar. We believe this kind of test can better test whether models consider permise and 
%choices at the same time. However, we know the stress test is still not perfect. We will consider how 
%to construct more informative MCQ questions to interpret model strengths and weaknesses. 
%I will consider your suggestion carefully.

\subsubsection*{Review 2, Question 1: Why mutation works?}

\textbf{Answer:} 
\begin{enumerate}
\item Mutation makes the two choices of a question very similar except for 
the order of the words. This forces the model to look to the premise to avoid
short-circuit problems. 
\item It also further strengthens the model's grammatical capabilities.
\end{enumerate}
For more details, please refer to Sec 2.2 para 4.
%
%As we have described in the forth paragraph of Sec 2.2,
%mutation is designed to encourage the models the model to pay more attention to the premise of
%questions due to its two very similar choices. Models can't get the bias feature easily only from 
%choices and have to look forward (premise). 
%
\subsubsection*{Review 2, Question 2: Can the system learn that grammatically incorrect...}

\textbf{Answer:} As you think, the system can learn grammatical knowledge in choices. 
We also illustrated this in Sec 2.2 (fourth paragraph). 
In our experiments which didn't show in this paper, we test on grammatical test cases and 
the system gets great performance. However, this test is an in-domain test for mutation which can not 
show the reasoning ability of models. Thus, it didn't appear in our paper as a stress test. 
The main contribution of mutation is to teach models to learn the relation between premise and choices. 

\subsubsection*{Review 3, Question 1: Why not augment the dataset by using the stress test operators...}

\textbf{Answer:} 
Please refer to Sec 2.2, para 1:
\begin{enumerate}
\item These stress tests cannot generate a sufficient amount of data for
training;
\item The purpose of this paper is to design data augment to promote
the model's general ability to avoid short-circuit, while most of the stress
operators work on a specific linguistic capability. 
\end{enumerate}
%We noted in Sec 2.2 (paragraph 1), ``Intuitively,
%all the operators that can generate stress tests can
%be used to create more training data ...not to
%mention their combinations.'' 
%The main difference between our operators and stress tests is that 
%our operators are general rather than specific on a fine-grained aspect. Besides, 
%our operators can construct more sufficient data.

%\textbf{Review 3, Question 2: It would be interesting to see an example of question/answer on which one and both...}
%
%\textbf{Answer:} We showed an example of question/answer on which one and both 
%the two operators are applied in Sec 3.4 as case study. Furthermore, we give extra examples in Appendix C. 
%We think these examples can show that short-circuit does take
%place and our augmentation methods alleviate it.

%\textbf{Review 3, Question 3: As reported in the strength section, I have some concerns about...} 
%
%\textbf{Answer:} 
%Mutation is designed to encourage the models the model to pay more attention to the premise of
%questions due to its two very similar choices. Models can't get the bias feature easily only from 
%choices and have to look forward. Besides, it also makes the model
%more sensitive to the differences in word orders and enhances
%the model’s pre-existing grammatical knowledge (the last paragraph in Sec 2.2). 
%The most direct experiment that can show the effectiveness of mutation is the stress test result in Sec 3.1. 
%The choices of stress test are very familiar and doesn't have any grammatical difference 
%(both are grammatically correct). 
%It it obvious that mutation can improve the performance of models on stress tests in Table 4 and Fogure 4.
\end{document}

