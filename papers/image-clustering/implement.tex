\section{Implementation}
\label{sec:implement}
In the section, we show some implementation details of our system.
We first tune the threshold of the tri-stage clustering algorithm,
then, we discuss integrating visual features to our system.

\subsection{Threshold of Tri-stage Clustering}
The tri-stage clustering algorithm is based on HAC\_CC
algorithm. Similar to traditional HAC algorithm, HAC\_CC has
a threshold to control the granularity of the clustering
result. We tune different threshold $\tau_t$ of HAC\_CC on a training
data collected from top 100 images of 10 different queries.
We assign the cluster label to each image by human judgement.
\figref{fig:thtri} shows the clustering result on
different thresholds of HAC\_CC.
We prefer to choose a threshold which can ensure high purity, F1 and NMI 
at the same time. NMI reaches a peak value at $\tau_t=0.15$ while at this threshold,
the purity have a significant improvement comparing to 0.1 and the
F1 score stays at a high value.
Thus, in our system, we choose the threshold $\tau_t$ to be 0.15.

\begin{figure}[th]
\centerline{\psfig{figure=tht.eps,width=0.8\columnwidth}}
	\caption{Clustering Result on Different $\tau_t$}
	\label{fig:thtri}
\end{figure}

%\subsection{Combining Visual Features}
%Because visual features are complementary and may not necessary
%represent the true semantics of the image, in our framework,
%clustering by visual features is biased toward high purity,
%%to make sure further clustering based on visual features is reliable.
%We experiment on different visual clustering threshold $\tau_v$ on
%the same data set as $\tau_t$.
%As shown in \figref{fig:thv}, using $\tau_v$ = 0.6, we can obtain
%an acceptable result of purity higher than 0.9.
%\begin{figure}[th]
%	\caption{Clustering result on different $\tau_v$}
%	\label{fig:thv}
%\end{figure}

\subsection{Integrating with Image Search Engine}
In this paper, our primary goal for web image clustering is to
improve the user experience of image search on the web.
Instead of a mix bag of multiple entities, existing image search engine
can leverage our framework to deliver classified search results.
When serving images online, responsiveness is critical.
%The resulting clusters of different entities
%is shown to users.
Clustering on all of the images in the internet offline is an option
but is infeasible since the complexity of all clustering algorithms
is super-linear. We design our system in a online-offline division
manner, and the conceptualization of all host web pages (which is linear of the number of
images) can be done off-line while the search engine indexes the web pages.
Then during online query processing, our system can cluster the images
returned by search engine page by page. Take the query of ``bean'' as example,
we group the first 100 results returned by the search engine
into clusters as shown in \figref{fig:demo-bean}.
By limiting the number of images, online clustering can
be done in under 1 second. User can quickly find out the image they want.
If they want to discover more about a particular entity,
they can click on one of the concepts (tags) of each clusters
which will be transformed into another query for additional images.
