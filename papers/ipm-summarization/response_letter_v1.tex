%%%%%%%%%%%%%%%%%%%%%%%%%%%%%%%%%%%%%%%%%
% Plain Cover Letter
% LaTeX Template
% Version 1.0 (28/5/13)
%
% This template has been downloaded from:
% http://www.LaTeXTemplates.com
%
% Original author:
% Rensselaer Polytechnic Institute 
% http://www.rpi.edu/dept/arc/training/latex/resumes/
%
% License:
% CC BY-NC-SA 3.0 (http://creativecommons.org/licenses/by-nc-sa/3.0/)
%
%%%%%%%%%%%%%%%%%%%%%%%%%%%%%%%%%%%%%%%%%

%----------------------------------------------------------------------------------------
%	PACKAGES AND OTHER DOCUMENT CONFIGURATIONS
%----------------------------------------------------------------------------------------

\documentclass[11pt]{letter} % Default font size of the document, change to 10pt to fit more text

\usepackage{newcent} % Default font is the New Century Schoolbook PostScript font 
%\usepackage{helvet} % Uncomment this (while commenting the above line) to use the Helvetica font

% Margins
\topmargin=-1in % Moves the top of the document 1 inch above the default
\textheight=8.5in % Total height of the text on the page before text goes on to the next page, this can be increased in a longer letter
\oddsidemargin=-10pt % Position of the left margin, can be negative or positive if you want more or less room
\textwidth=6.5in % Total width of the text, increase this if the left margin was decreased and vice-versa

\let\raggedleft\raggedright % Pushes the date (at the top) to the left, comment this line to have the date on the right

\usepackage{color}
\newcommand{\ZY}[1]{\textcolor{blue}{Zhiyi: #1}}
\newcommand{\KZ}[1]{\textcolor{red}{Kenny: #1}}

\begin{document}
	
	%----------------------------------------------------------------------------------------
	%	ADDRESSEE SECTION
	%----------------------------------------------------------------------------------------
	
	\begin{letter}{Dr. Jim Jansen \\
			Editor-in-Chief  \\
			Information Processing and Management} 
		
		%----------------------------------------------------------------------------------------
		%	YOUR NAME & ADDRESS SECTION
		%----------------------------------------------------------------------------------------
        \begin{center}
        \large\bf Yizhu Liu, Xinyue Chen, Xusheng Luo and Kenny Q. Zhu \\ % Your name
        %\vspace{20pt} \hrule height 1pt % If you would like a horizontal line separating the name from the address, uncomment the line to the left of this text
        Department of Computer Science and Engineering \\ Shanghai Jiao Tong University \\ 800 Dongchuan Road, Shanghai, China 200240 \\
        liuyizhu@sjtu.edu.cn
         % Your address and phone number
        \end{center} 
        \vfill

        \signature{Yizhu Liu} % Your name for the signature at the bottom
		
		%----------------------------------------------------------------------------------------
		%	LETTER CONTENT SECTION
		%----------------------------------------------------------------------------------------
		
		\opening{Dear Dr. Jim Jansen,} 
		
		We appreciate the opportunity to revise and resubmit our manuscript. 
		Thank you for the editors' and reviewers' comments concerning our 
		manuscript entitled ``Reducing Repetition in Convolutional Abstractive Summarization". 
		Those comments are all valuable and very helpful
		for revising and improving our paper.
		A point-by-point response to the Editors' and Reviews' comments is below. 
		We believe that the revisions prompted by these comments have strengthened our manuscript.
		\newline\newline
		On behalf of all co-authors,\\
		Yizhu Liu, Xinyue Chen, Xusheng Luo and Kenny Q. Zhu
		\newline\hrule

		\flushleft
		\begin{enumerate}
			\item Your abstract does not highlight the specifics of your research or findings but contains too much background information.
			\begin{itemize}
				\item[] AUTHORS' REPONSE: We highlight the specifics of our research and findings in our abstract as follows: 
                ``In this paper, we identify the repetition problem in abstractive summarization based on CNN seq2seq model and attention mechanism. The repetition is caused by decoders attending to the same locations in source document and attending to similar (but different) sentences in the source. 
				We propose to reduce the repetition in summaries by Attention Filter mechanism (ATTF) and Sentence-level Backtraching Decoder (SBD), which dynamically redistributes attention over the input sequence as the output sentences are generated. 
				The ATTF can record previously attended locations in the source document directly and prevent decoder from attending to these locations. The SBD prevents the decoder from generating similar sentences more than once via backtracking at test.''
			\end{itemize}
			\item Your references need to tie this research into the information science community. Certainly, there has been more recent (within the last two years) research on this topic published in Information Processing and Management and other information science outlets. Therefore, you must update your literature review. 
			\begin{itemize}
				\item[] AUTHORS' REPONSE: We have added more recent research in related work section.
			\end{itemize}
			\item Your reference list needs tidying up, as there are several references missing items or formatting issues. Please be consistent with the formatting. 
			\begin{itemize}
				\item[] AUTHORS' REPONSE: We have fixed the format of references in Reference section.
			\end{itemize}
			\item Given the topic, your reference list seems quite short. Therefore, you must update your literature review.
			\begin{itemize}
				\item[] AUTHORS' REPONSE: We have updated our literature in the preamble of related work section.
			\end{itemize}
			\item Please update the references from the ‘preprints', ‘working papers’, ‘tech reports’, etc. (i.e., to actual peer-reviewed articles), if available.
			\begin{itemize}
				\item[] AUTHORS' REPONSE: We have update the references list.
			\end{itemize}
			\item you must more clearly highlight the theoretical and practical implications of your research; it is not clear how this research differs from existing work.
			\begin{itemize}
				\item[] AUTHORS' REPONSE: We have updated highlights in ``highlightdiff.pdf'' and highlighted the theoretical and practical implications of our research as follows:
                \begin{itemize}
                \item We identify the repetition problem in abstractive summarization based on CNN seq2seq models.
                \item We propose two new evaluation metrics: Repeatedness and Readability. 
                \item We propose an effective repetition reduction approach that consists of the Attention Filter mechanism (ATTF) and Sentence-level Backtracking Decoder (SBD).
                \item Our proposed approach outperforms the state-of-the-art (SOTA) CNN-based baselines substantially by all evaluation metrics.
                \end{itemize}
            \end{itemize}
		\end{enumerate}
	
	
		We appreciate for the editors' and reviewers' warm work earnestly,
		and hope that the correction will meet with approval.
		
		Once again, thank you very much for  your comments and suggestions.
		
		
		Thank you and best regards.
		
		\closing{Sincerely yours,}
		
		
		%\encl{Curriculum vitae, employment form} % List your enclosed documents here, comment this out to get rid of the "encl:"
		
		%----------------------------------------------------------------------------------------
		
	\end{letter}
	
\end{document}
