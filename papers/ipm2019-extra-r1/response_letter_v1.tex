%%%%%%%%%%%%%%%%%%%%%%%%%%%%%%%%%%%%%%%%%
% Plain Cover Letter
% LaTeX Template
% Version 1.0 (28/5/13)
%
% This template has been downloaded from:
% http://www.LaTeXTemplates.com
%
% Original author:
% Rensselaer Polytechnic Institute 
% http://www.rpi.edu/dept/arc/training/latex/resumes/
%
% License:
% CC BY-NC-SA 3.0 (http://creativecommons.org/licenses/by-nc-sa/3.0/)
%
%%%%%%%%%%%%%%%%%%%%%%%%%%%%%%%%%%%%%%%%%

%----------------------------------------------------------------------------------------
%	PACKAGES AND OTHER DOCUMENT CONFIGURATIONS
%----------------------------------------------------------------------------------------

\documentclass[11pt]{letter} % Default font size of the document, change to 10pt to fit more text

\usepackage{newcent} % Default font is the New Century Schoolbook PostScript font 
%\usepackage{helvet} % Uncomment this (while commenting the above line) to use the Helvetica font

% Margins
\topmargin=-1in % Moves the top of the document 1 inch above the default
\textheight=8.5in % Total height of the text on the page before text goes on to the next page, this can be increased in a longer letter
\oddsidemargin=-10pt % Position of the left margin, can be negative or positive if you want more or less room
\textwidth=6.5in % Total width of the text, increase this if the left margin was decreased and vice-versa

\let\raggedleft\raggedright % Pushes the date (at the top) to the left, comment this line to have the date on the right

\usepackage{color}
\newcommand{\ZY}[1]{\textcolor{blue}{Zhiyi: #1}}
\newcommand{\KZ}[1]{\textcolor{red}{Kenny: #1}}

\begin{document}
	
	%----------------------------------------------------------------------------------------
	%	ADDRESSEE SECTION
	%----------------------------------------------------------------------------------------
	
	\begin{letter}{Dr. Jim Jansen \\
			Editor-in-Chief  \\
			Information Processing and Management} 
		
		%----------------------------------------------------------------------------------------
		%	YOUR NAME & ADDRESS SECTION
		%----------------------------------------------------------------------------------------
		
		\begin{center}
			\large\bf Zhiyi Luo, Shanshan Huang, Kenny Q. Zhu \\ % Your name
			%\vspace{20pt} \hrule height 1pt % If you would like a horizontal line separating the name from the address, uncomment the line to the left of this text
			Department of Computer Science and Engineering \\ Shanghai Jiao Tong University \\ 800 Dongchuan Road, Shanghai, China 200240 \\
			jessherlock@sjtu.edu.cn
			% Your address and phone number
		\end{center} 
		\vfill
		
		\signature{Zhiyi Luo} % Your name for the signature at the bottom
		
		%----------------------------------------------------------------------------------------
		%	LETTER CONTENT SECTION
		%----------------------------------------------------------------------------------------
		
		\opening{Dear Dr. Jim Jansen,} 
		
		We appreciate the opportunity to revise and resubmit our manuscript. 
		Thank you for the editors' and reviewers' comments concerning our 
		manuscript entitled ``Knowledge empowered prominent aspect extraction from product reviews". Those comments are all valuable and very helpful
		for revising and improving our paper.
		A point-by-point response to the Editors' and Reviews' comments is below. 
		We believe that the revisions prompted by these comments have strengthened our manuscript.
		\newline\newline
		On behalf of all co-authors,\\
		Zhiyi Luo, Shanshan Huang, Kenny Q. Zhu
		\newline\hrule

		\flushleft
		\begin{enumerate}
			\item Please update your literature with recent developments in deep learning, e.g., Ma et al.'s Targeted aspect-based sentiment analysis via embedding commonsense knowledge into an attentive LSTM. 
			\begin{itemize}
				\item[] AUTHORS' REPONSE: We have updated our literature in the preamble of related work section (page 4, Section 2).
			\end{itemize}
			\item The organization of the figures and tables also needs to be rearrange.
			\begin{itemize}
				\item[] AUTHORS' REPONSE: We reorganized the figures and tables
				in the experiments section (page 20, Table 5 and page 21, Table 6).
			\end{itemize}
			\item The Abstract can be improved by highlighting the results found in the experiments before making the conclusion.
			\begin{itemize}
				\item[] AUTHORS' REPONSE: We have improved the Abstract to
				highlight the results (page 1, Abstract).
			\end{itemize}
			\item The contribution part also needs to be improved as the 3rd point is not a contribution.
			\begin{itemize}
				\item[] AUTHORS' REPONSE: We strengthen our third contribution as follows: ``We create a new evaluation dataset for prominent aspects extraction (Section 5.1). We will release the evaluation dataset as well as the implementation of the system for future work in this research area." (page 4, second paragraph).
			\end{itemize}
			\item Since the term `prominent' is the keyword in this work, kindly define the term clearly, to differentiate between prominent and fine-grained. 
			\begin{itemize}
				\item[] AUTHORS' REPONSE: We define the `prominent aspects' and
				distinguish it with `fine-grained aspects' in introduction section (page 2, last paragraph).
			\end{itemize}
			\item I think the statement of ``is that they typically use only word frequency and co-occurrence information, and thus the performance degrades when extracting aspects from sentences that appear different on the surface but actually discuss similar aspects." is not so accurate and may need to be revised.
			\begin{itemize}
				\item[] AUTHORS' REPONSE: We restate the problem of topic modeling based approaches as follows:
				``Although topic modeling based methods can group topically related words into one topic, it can be difficult to properly control the granularity of the topic. Thus, the performance degrades when extracting too fine-grained aspect words as prominent aspects. ''
				(page 3, line 61)
			\end{itemize}
			\item In Section 4.1, I think it's better to use a table to list all the rules and their corresponding examples to make it easy for readers to understand these rules.
			\begin{itemize}
				\item[] AUTHORS' REPONSE: We improve the representation accordingly, demonstrating
				the extraction process in Table 1. (page 8)
			\end{itemize}
			\item When introducing some of the previous work, the paper is missing the relevant references. There are some grammatical errors and typo errors in the paper.
			\begin{itemize}
				\item[] AUTHORS' REPONSE: We proofreading the paper thoroughly and fix those errors.
			\end{itemize}
		
		\end{enumerate}
	
	
		We appreciate for the editors' and reviewers' warm work earnestly,
		and hope that the correction will meet with approval.
		
		Once again, thank you very much for  your comments and suggestions.
		
		
		Thank you and best regards.
		
		\closing{Sincerely yours,}
		
		
		%\encl{Curriculum vitae, employment form} % List your enclosed documents here, comment this out to get rid of the "encl:"
		
		%----------------------------------------------------------------------------------------
		
	\end{letter}
	
\end{document}
