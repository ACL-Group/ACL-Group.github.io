\section{Related Work}
\label{sec:related}
This paper is an extended version of the best
paper award winning work published at DASFAA 2014 \cite{JiaPXZL14}.
In this paper, we formalize the problem more rigorously,
give more details of the algorithm and prove a few of its
interesting properties. Furthermore, additional
experiments as well as in-depths analysis of the experimental results
are added to this extended version. We first present a number
of well-known privacy models, then compare and contrast several general
anonymization techniques, and finally show how our anonymizaton method
can protect several kinds of unsual attacks.

\subsection{Privacy Models}

Privacy-preserving data publishing of relational tables has been well
studied in the past decade since the original proposal of $k$-anonymity by
Sweeney \etal \cite{Sweeney2002:k-anonymity}.
Recently, privacy protection of set-valued data has received increasing
interest. The original set-valued data privacy problem was defined in the
context of association rule hiding
\cite{atallah99:disclosure,tkde:VerykiosEBSD04:ARH,tkde:WuCC07:hiding},
in which the data publisher
wishes to ``sanitize'' the set-valued data (or {\em micro-data}) so that all
sensitive or ``bad'' associate rules cannot be discovered while all (or most)
``good'' rules remain in the published data.
Subsequently, a number of privacy models
including $(h,k,p)$-coherence \cite{Xu:2008:ATD},
$k^m$-anonymity \cite{Terrovitis:2008:PAS},
$k$-anonymity \cite{He:2009:ASD} and
$\rho$-uncertainty \cite{Cao:2010:rho} have been proposed.
$k^m$-anonymity and $k$-anonymity are carried over directly from
relational data privacy,
while $(h,k,p)$-coherence and $\rho$-uncertainty protect the
privacy by bounding the confidence and the support of
any sensitive association rule inferrable from the data. This is
also the privacy model this paper adopts.

\subsubsection{The $k$-anonymity Model}
Many datasets are published simply with key identifiers (e.g. name and
social-security number) removed so the records are not related to specific
people.  However, some pseudo-identifiers (e.g. age and zip-code) can be
combined to narrow down to or even identify a small number of individuals.
In order to prevent identification, the $k$-anonymity model requires that
every such combination in the dataset occurs at least $k$
times so that
every record is indistinguishable from at least $k-1$ other records.

\subsubsection{The $l$-diversity Model}
Kifer \etal \cite{Kifer:l-diversity} showed using two simple attacks that a
$k$-anonymized dataset has some subtle but severe privacy problems, and
proposed a novel and powerful privacy criterion called $l$-diversity that
can defend against such attacks.

While $k$-anonymity is effective in preventing identification of a record,
$l$-diversity focuses on maintaining the diversity of the sensitive attributes \cite{aggarwal2008general}.
Therefore, the $l$-diversity model is defined as follows:

\begin{Definition}
  Let a $q^*$-block be a set of tuples such that its non-sensitive values
  generalize to $q^*$.  A $q^*$-block is $l$-diverse if it contains $l$
  ``well-represented'' values for the sensitive attribute $S$.  A table
  is $l$-diverse, if every $q^*$-block in it is $l$-diverse.
\end{Definition}

A number of different instantiations for this definition are discussed
in \cite{Kifer:l-diversity}, where the term
``well-represented'' is attached with different meanings.

\subsubsection{The $(h,k,p)$-coherence Model}
The $(h,k,p)$-coherence model by Xu \etal \cite{Xu:2008:ATD}
requires that the attacker's prior knowledge to be no more than $p$ public
(non-sensitive) items, and any inferrable rule must be supported by at least
$k$ records while the confidence of such rules is at most $h$\%. Xu believe
private items are essential for research and therefore only remove public
items to satisfy the privacy model, and developed an efficient greedy
algorithm using global suppression. In this paper, we do not restrict the
size or the type of the background knowledge, and we use a partial
suppression technique to achieve less information loss and also better retain
the original data distribution.

\subsubsection{The $\rho$-uncertainty Model}

Cao \etal \cite{Cao:2010:rho} proposed a similar $\rho$-uncertainty model
which is used in this paper. They developed a global suppression method and a top-down generalization-driven global suppression method (known as TDControl) to eliminate all sensitive inferences with confidence above
a threshold $\rho$.
Their methods suffer from same woes discussed earlier for generalization and
global suppression. Furthermore, TDControl assumes that data exhibits some monotonic property under a generalization hierarchy. 
%This assumption is questionable. 
Experiments show that our algorithm significantly outperforms the two methods in preserving data distribution and useful inference rules, and in minimizing information losses.

\subsubsection{Differential Privacy}

Differential privacy \cite{Dwork08:diff:survey}
is targeted at a statistical database. The goal of differential privacy is to release statistical information without compromising the privacy of the individual respondents.
It ensures that the removal or addition of a single database item does not affect the outcome of any analysis. It follows that no risk is incurred by joining the database,
providing a mathematically rigorous means of coping with
the fact that distributional information may be divulged.
However, 
%in the scenario of a statistical database,
some users do not have exclusive access to all of the data set, but can only gain some statistics of the required data.
Such mechanism is then less efficient and harder to
deploy than where the users can load the
whole dataset into memory and do arbitrary computations.
Anonymization techniques of differential privacy add appropriately chosen random noise to produce response to the queries, which will lead to spurious rules when doing data mining.

\subsection{Anonymization Techniques}

A number of anonymization techniques were developed for these models.
These generally fall in four categories\cite{Machanavajjhala12}: {\em
global/local generalization}
\cite{samarati1998,Iyengar:2002:TDS,LeFevre:2006:Mondrian,Terrovitis:2008:PAS,He:2009:ASD,Cao:2010:rho}, {\em global suppression} \cite{atallah99:disclosure,Xu:2008:ATD,Cao:2010:rho},
{\em permutation} \cite{2011:TKDE:Anonymous} and {\em perturbation}
\cite{Zhang:2007:agg,ChenMFDX11:Diff,Javier2012,WangW05}. Next we briefly discuss the pros and
cons of these anonymization techniques.

\subsubsection{Generalization}

Generalization\cite{samarati1998,Iyengar:2002:TDS} replaces a specific value by
a generalized value, e.g., ``beer'' by ``drink'',
according to a generalization hierarchy \cite{FungWCY10:Survey}.
\cite{samarati1998} first proposed the idea of generalization.
Then generalization is divided into global generalization \cite{Iyengar:2002:TDS}
and local generalization \cite{LeFevre:2006:Mondrian}.
Table \ref{tab:samesample} illustrates generalization by reusing the
same dataset in Table \ref{tab:sample}, where both ``beer'' and ``coffee''
are generalized to ``drink'', according to some generalization hierarchy.
%While generalization preserves the correctness of the data,
%it compromises accuracy and preciseness. 
Worse still, association rule mining is impossible
unless the data users have access to the same generalization taxonomy
and they agree to the target level of generalization. For instance, if
the users don't intend to mine rules involving ``drink'', then
all generalizations to ``drink'' are useless.

\begin{table*}[thb]
\caption{The Same Dataset in Table \ref{tab:sample} and Generalization Anonymization Result
\label{tab:samesample}}
%\small
\centering
\subtable[Original Dataset]{
\begin{tabular}{|c|l|}
\hline
% after \\: \hline or \cline{col1-col2} \cline{col3-col4} ...
{\bf ID} & {\bf Transaction} \\ \hline
1 & bread, {\bf beer}, {\em condom} \\ \hline
2 & {\bf coffee}, fruits  \\ \hline
3 & {\bf beer}, {\em condom}  \\ \hline
4 & {\bf coffee}, fruits  \\ \hline
5 & flour, {\em condom}\\ \hline
6 & bread, {\bf coffee}  \\ \hline
7 & fruits, {\em condom}  \\ \hline
\end{tabular}
\label{tab:orig-sample-same}
}
\subtable[Generalization Result]{
\begin{tabular}{|c|l|}
\hline
% after \\: \hline or \cline{col1-col2} \cline{col3-col4} ...
{\bf ID} & {\bf Transaction} \\ \hline
1 & bread, {\bf drink}, {\em condom} \\ \hline
2 & {\bf drink}, fruits  \\ \hline
3 & {\bf drink}, {\em condom}  \\ \hline
4 & {\bf drink}, fruits  \\ \hline
5 & flour, {\em condom}\\ \hline
6 & bread, {\bf drink}  \\ \hline
7 & fruits, {\em condom}  \\ \hline
\end{tabular}
\label{tab:sample-generalization}
}

\end{table*}

\subsubsection{Global Suppression}

Global suppression is a technique that deletes all instances of some items
so that the resulting dataset is safe.
The advantage is that it preserves the support of
existing rules that don't involve deleted items and hence retains these rules
\cite{Xu:2008:ATD}. The obvious disadvantage is that it can cause unnecessary
information loss. In the past, partial suppression
has rarely been attempted mainly due to its perceived side effects of
changing the support of inference rules in the original data
\cite{Xu:2008:ATD,Cao:2010:rho,tkde:VerykiosEBSD04:ARH,tkde:WuCC07:hiding}.
Our work shows that partial suppression introduces limited
amount of new rules while preserving many more original ones than
global suppression. Furthermore,
it preserves the data distribution much better than
other competing methods.
Some work integrated suppression and generalization in anonymizing set-valued date \cite{liu2010anonymizing,Cao:2010:rho}.
\subsubsection{Permutation}

Permutation was introduced by Xiao \etal \cite{Xiao:2006:Anatomy} for
relational data and was extended by
%. With generalization technique severely compromising the
%accuracy of data aggregation analysis, Xiao \etal propose the
%\textit{Anatomy} which releases quasi-identifier and sensitive values in two
%separate tables. Specifically quasi-identifier values are not changed and
%organized into groups, and for every such group the corresponding sensitive
%values are aggregated. After that,
Ghinita \etal \cite{2011:TKDE:Anonymous}.
%for transactional data.
Ghinita \etal propose two novel anonymization techniques for sparse
high-dimensional data by introducing two representations for transactional
data. However the limitation is that the quasi-identifier is restricted to
contain only {\em non-sensitive items}, which means they only
consider associations between quasi-identifier
and sensitive items, and not {\em among} sensitive items.
Manolis \etal \cite{terrovitis:privacy} introduced ``disassociation''
which also severs the links between values attributed to the
same entity but does not
set a clear distinction between sensitive and non-sensitive attributes.
%They set those frequent itemsets into a cluster and partition the table into
%several parts, which eliminates the rules with
%a high confidence and a certain support.
%In this paper, we consider all kinds of associations and try best to
%retain them.
%While in our paper, we consider all kinds of associations and try best to
%retain the accuracy of those associations
%
%\PC{Moreover, the model introduced in their method is not strong. The attack
%effects in the following steps. Step 1:The attacker gains some background
%knowledge such that the number of people buying creams and pregnancy tests
%are around five times more than people buying butter and pregnancy tests. .
%Step 2: The attacker downloads the result processed by permutation and one of
%the group in the result has such form that only contains people buying cream
%or butter with probability of sensitive item pregnancy test $\frac{1}{3}$.
%Step 3: with a simple equation $\frac{1}{3}(x+y)=5Px+Py$, where x represents
%people buying butter and y represents people buying cream, the attacker can
%get the exact probability of people who buy cream buy the pregnancy
%test($5P$) which is likely to be a high value with different x and y.
% }

\subsubsection{Perturbation}

Perturbation is developed for statistical disclosure control
\cite{evfimievski2003limiting,evfimievski2004privacy,FungWCY10:Survey}. Common perturbation methods include {\em additive
noise}, {\em data swapping}, and {\em synthetic data generation}. Their
common criticism is that they damage the data integrity by adding noises and
spurious values, which makes the results of downstream analysis unreliable.
Perturbation, however, is useful in non-deterministic privacy model such as
differential privacy \cite{Dwork08:diff:survey}, as
attempted by Chen \etal~ \cite{ChenMFDX11:Diff} in a probabilistic top-down
partitioning algorithm based on a context-free taxonomy.
 %Interestingly, the
%algorithm proposed in this paper is probabilistic in nature as well.
%Considering the fact that the noise introduced by randomization leads to
%severe data utility, some work related with differential privacy focuses on
%releasing certain data mining results
%\cite{Barak:2007:PAC:1265530.1265569,Bhaskar:2010:DFP:1835804.1835869,Friedman:2010:DMD:1835804.1835868,Korolova:2009:RSQ:1526709.1526733}.
%However, the usability of the published data is constrained by the pattern
%the owner decide to release and moreover the assumption that the data owner
%is able to perform data mining tasks is also weak. In addition, Leoni \etal
%\cite{DBLP:journals/corr/abs-1205-2726} also indicates the weakness of
%differential privacy model itself.
%
%\textbf{
%Recently, a new method called slicing was firstly proposed in \cite{10.1109/TKDE.2010.236} and was
%further developed in \cite{terrovitis:privacy}. Slicing, also called disassociation, aims to protect
%identity or attribute disclosure using identify combinations. However, the privacy model they introduce is
%different from ours and the type of targeted
%data utility is also completely different from ours.
%(I am in a dilemma. Since their methods are totally different
%from ours, I can't make comparison with ours.
%Their methods are somewhat useful in association rule mining and distribution remaining.
%The only disadvantage is that they changed the original structure of the table, but
%according to their data utility such change is acceptable.  )}

\subsection{Adversarial Attacks}
It is straight-forward to see that the anonymized data under our algorithm
is immune from the {\em record linking attack} \cite{FungWCY10:Survey,samarati1998}.
Furthermore, our technique can protect transactional data from {\em minimality
attacks} \cite{Wong:2007:Minimality} and {\em composition attacks} \cite{Ganta:2008:Composition}.

The minimality attack \cite{Wong:2007:Minimality}
is proposed for relational data. Assume an adversary knows the whole original
quasi-identifier values as external data, also knows the privacy model and
anonymization technique,  the
adversary can successfully predict some privacy. The minimality attack relies
on the generalization anonymization technique, while our method uses
suppression technique. Also for set-valued data it doesn't have fixed
combination of items as quasi-identifiers, so it's unrealistic for an
adversary to obtain the satisfactory external data.

The composition attack \cite{Ganta:2008:Composition} is proposed for relational data by using the overlap population of multiple organizations' independent release of anonymized data through intersection. For example $l$-diversity \cite{Ganta:2008:Composition} model can be violated by composition attack. 
The reason why composition attack can succeed is that quasi-identifier attribute values are generalized and sensitive attribute values are retained.
%, when performing intersection the probability of a correlation %between quasi-identifier and sensitive values will definitely %increase. 
On the contrary, our partial suppression algorithm anonymizes set-valued data by randomly suppressing some sensitive items.  The probability that a sensitive item correlated with quasi-identifier items can be changed, which makes the composition attack not plausible. So our partial suppression technique is ideal to avoid composition attack depending on the randomized characteristic of suppression.

\cut{%%%%%%%%%%
\subsection{Improvements over the Previous Version}
This paper is an improved version of our previous paper\cite{JiaPXZL14}.
In the former version, we stated the fundamental goal of the work and the basic algorithms.
Some vague or confusing definitions are clarified in this paper and
additional experiments are conducted.
Detailed explanation and analysis of the algorithms and implementation is also included in this paper.

First, in \secref{sec:algo}, the possible scenarios of regression are discussed.
We gave the reasons for some specific implementation techniques,
such as why we randomly picked items to be deleted.
Analysis on the algorithms is presented in \secref{sec:analysis},
which shows the validity of our algorithms.
The correctness of the {\em number of suppressions} is proved.
We also proved that \PartialSuppressor can terminate with a correct solution,
and that the divide-and-conquer optimization is correct.
The analysis of the time complexity is given at the end of that section.

Second, we add clear definitions of the evaluation metrics, such as the \emph{info loss},
which makes the experiments easier to understand.
In \secref{sec:eval}, the distribution of each data set and the difference among them
is shown in \figref{fig:datasets}.
Analysis on the results of different data sets is also added.
The analysis on the effects of different parameters is restructured such that
it's more straightforward to see the individual function of the parameters.

More related work is added in \secref{sec:related}.
Well-known privacy models, together with their advantages and disadvantages, are introduced.
Some anonymization techniques are introduced and we analyze the pros and cons of them.
Several adversarial attacks are discussed,
in order to show the robustness of our partial suppression technique.
}%%%%%%%%%%%
