%\vspace{-15pt}

\section{Conclusion}
In this paper, we propose a new research topic in Information Extraction and Text Mining, and develop a framework to compute cross-cultural differences of named entities.
Leveraging the quantity of corpus from news articles and the quality of named entity information from Wikipedia and Wikidata, we managed to come up with an approach of calculating the distance to represent cross-cultural differences. 
%We quantify the social phenomenon into a text mining problem, resulting not only to be able to find out whether a given entity term is different in two cultural backgrounds, but also which entity is similar to another in two different standings.
%During the process, we have some interesting discoveries, too. 
\section{Future work}
%
Firstly, popular images on search engines for different
languages are a good source for detecting cultural difference. 
%We use these
%images to create the ground truth of our evaluation.
%, but further research
%maybe instead focus on mining the images themselves.
%However, the manual labeling process using Bing Images search result may be biased since the volunteers who helped us may have not have the same criteria of cultural differences, and when we are collecting the labeling results, many terms received similar number of votes from both sides, how we deal with the labeling error or biased criteria among people also affects our result.
Also, the reliability of the human annotators can be improved by more
annotators from more diverse cultures.
%This may be possible through
%crowd-sourcing.
Apart from that, the cultural differences were mined from news articles which often
reflect the official opinions rather than the opinion of the masses.
%data source, we discovered that news articles maybe too objective for such international matters.
%For example, a Chinese newspaper may have the same point of view on something that happens where both countries are unrelated as that of an America newspaper. The news may not representing the people's thoughts and will, thus the cultural difference we mined from news articles may be limited only to less subtle and hidden ones.
It would be interesting to do similar research using other text
resources, such as the social media data. Obviously this poses new
challenges as social media data is a lot noisier and more ambiguous.
%is harder to collect and parse the information. It would be a challenging task to do good entity linking on those informal text with various new usages and words, yet they certainly represent people's mind directly in multiple cultures.
%\BL{the news maybe too objective and contains less culture-related comments. social network maybe better choice but it's harder to collect.}
%\newpage
