\section{Introduction}
\label{sec:intro}
%{\bf Outline:}
%\begin{verbatim}
%{\bf Motivation of the problem:}
%(1) Too many choices: how to select my favorite 
%show to watch at real time (on the fly)?
With the rapid development of digital television service
in China, viewers are spoiled with choices. 
Over 30,000 programmings are available each week 
on 112 cable channels. 
However, this massive variety can be a mixed blessing:
viewers often have a hard time deciding which program
to watch. In the worst case, one might spend more time
flipping through the channels than watching anything
meaningful, or simple miss the programs that they really
should watch. 
What we viewers really desire is a real-time, personalized
recommender system that makes timely suggestions of 
what programmings to watch, or not to watch,
based on {\em who we are} and {\em what we like}.
 
%(2) Users demand personalized services
%Many users, especially youths, are in pursuit of
%personality today. Thery are eager to show their
%differences with others. In such situation, An
%behavior-oriented recommendation system can not
%only save users' time to select programs, but
%also provide users personalized services. It can
%help users to make decision in other area, or
%send information to users which they may be
%interested in.

%{\bf Challenges of the problem: }
%(1) Large data volume.

However, to determine who we are and what we like, and ultimately
provide good TV recommendations poses significant challenges.
First, user profile or preference information is 
often non-existent, incomplete or changing over time given
the common cable TV subscription practice in China. 

Second, television viewing patterns are complicated. Unlike
video sharing sites such as YouTube, which typically
interacts with a single user, and records user sessions,
televisions are usually not viewed by an
individual person only but by a group of people such as
family or friends. And these people can watch the same television
at different times of the day. The profile information of 
one user (e.g. the TV service subscriber) is inadequate 
for determining the preferences of the whole group.
And there's no obvious user sessions to help 
distinguish the different viewers.

Third, in contrast to products and services offered by e-commerce 
or video sharing sites, whose supply can be considered ``infinite''
and ever ready, TV shows are ``time-consumable'' resources which
are only available at specific times. Once aired, they are probably 
never available again, at least not controlled by the user. Consequently
the time factor is important in the TV recommendation system.
In addition, viewers often have their preferred TV times.
This means, even if a user may like a show, she probably will not
watch it unless it is shown at the right time.

Finally, web-based recommender systems can often take
advantage of user feedback including ratings and
comments. Such explicit feedback is almost certainly
not available to a TV recommender system. The only
input a TV recommender can leverage is the viewing histories
and behaviors, which is what we are targeting in this paper.



%based on the complexity of TV programs and
%user behavior, there are some problems left that
%needed to be solved. Our system will handle large
%amount of data such as users viewing behaviour,
%TV schedule and TV info. Processing such data
%really cost time and resource.
%TV shows are "time-consumable" resources, once
%aired, probably never aired again, at least not
%controlled by the viewers. This is different from
%recommendation systems in some online video website
%like YouTube. For online video website, resources
%are stored on the server and will be available in
%quite a period. Users can pick up a video to watch
%whenever they like. So this kind of recommendation
%systems can just take users's preference in account.
%In contrast, TV recommendation system should also
%involve the time factor, which requires our system
%to recommend programs to specified user at
%specified time.
%View patterns are more complex than other video
%sharing sites. Also take YouTube as an example.
%When users start to browse this website, their
%viewing histories will be stored via session or
%cookie. Even if different users use the same
%machine to browse, their behaviours can be divided
%in different session or cookie. But set-top box can
%not finish such distinguishing. So when different
%users use the same TV set, their viewing histories
%will merge together and make it difficult to analyze
%the patterns.
%TV schedule info is very limited. In mainland China,
%there is no standard TV program infomations available.
%Our system have to obtain such informations itself.
%%(5) Users have no or little profile info, no
%%direct feedback after watching.
%Our system is not based on predefined user profile
%but form users' model via users' behaviours. Also,
%we evaluate our recommendation result based on
%users' behaviour. So no direct feedback is needed.

%{\bf Our solution: }
To address the above challenges, 
we propose PredicTV, a new TV recommender system based on
user viewing behaviors. The system tracks viewing
behavior by recording the time and the channel whenever
the user switches to a new channel.
Such channel switching information,
also known as the {\em viewing history}, is
then streamed to the remote recommender server.
The server automatically analyzes
the programs being watched as well as the viewing patterns,
and recommends current or future shows which best match the inferred
preferences of the viewers associated with that viewing
history. Because the system works dynamically, it
adapts to the user tastes as it changes over time.

%(3) Digital TV set-top box provides a mechanism 
%to track viewing behaviors

PredicTV contains two main parts: an offline program information
extraction system, and an online recommendation system.
The offline system downloads TV program schedule every week,
extract attribute-value information for each TV program in
that schedule from the web, and build special vector models
for them. The online system records the channel switching
request in real time, and builds a viewing model based on the
history, while recommending the most relevant programs according to
the current model of the users.

%In the server part, we use a background program to obtain
%TV schedule and gather info about each program on the
%internet. Then convert the info of each program 
%(including air time) into a multi-dimensional
%vector. Users' preferences are also represented using
%vector. Their viewing behaviours will
%be used to update user vector. Gradually, this
%vector will match users' preference better and 
%better.
%At runtime, our system takes as input the 
%streams of user viewing behavior, identifies the 
%programs watched by each user, and dynamically
%updates their viewing preference vectors with
%the program being watched at the same time and damps
%the effects of shows watched earlier at the same time.
%When doing recommendation, we match the program
%vector against the users summary vector and get
%the result.

Some work has been done in developing an TV recommendation
system. Srinivas Gutta has introduced a TV content recommendation
system
\cite{Gutta00:ContentRecommender}, which mainly uses
Bayesian classifier and decision tree to analyze users'
preference. SeungGwan Lee has also introduced a program
recommendation system
\cite{Lee10:RecommendunderCloud}. It is an on-demand
system and its services is deployed on cloud. When
designing our system,
we test and find that some classical machine-learning
method like decision tree, content-base and collaborative
method are too slow to support large amount of users.
So we try to use vector in our recommendation algorithm.
Claire Laudy has discussed information fusion for TV program
information
\cite{Laudy08:InfoFusion}. But in our system, we cannot get
standard TV information directly. So we add an information
extraction module in our system. It fetches information then
construct it as a vector. Zhiwen Yu has discussed a TV
recommendation for multiple viewers
\cite{Zhiwen06:RecommendMultiViewers}.
It use some algorithms to merge users' profile and get a
general profile. This profile is used in recommendation. We
don't use user profile in our system. Instead, users' viewing
patterns are represented as the
sum of all the program vector they have viewed. The vector
algorithm can not only used to recommend TV program, but
also ad, merchandise, which makes our system flexible.

Next we will briefly introduce the extraction and recommendation 
algorithms in PredicTV (Section \ref{sec:tech}), and present
some preliminary experimental results (Section \ref{sec:eval})
 as well as a demo plan (Section \ref{sec:demoplan}).
