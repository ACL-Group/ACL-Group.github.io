\section{Technical Specification}
\label{sec:algo}

\begin{figure*}[th]
\centering
\epsfig{file=pic/SystemOverview2.eps,width=1.8\columnwidth}
\caption{System Overview}
\label{fig:sys}
\end{figure*}

Figure \ref{fig:sys} shows the block diagram of our system.
As the input of the system, the web page is first parsed by 
a HTML parser\cite{winista} to obtain a complete DOM representation.
Then the title classifier attempts to recognize the page title.
If it is a ``top-$k$ like'' title, 
the classifier outputs the list size (the number $k$) 
and a set of possible concepts mentioned in the title.
With the number $k$, the candidate picker extracts all lists of size $k$ 
from the page body as candidate lists. Only one of them will be the actual
list of interest. With the concept set, 
the top-$k$ ranker can score each candidate list and pick the best one 
as the ``top-$k$'' list.  Finally the content processor  
analyzes the list content and extracts the entity names and attributes. 
%and conceptualize the main entities in the list
%as well as their attributes, if any. 

\subsection{Title Classifier}
\label{sec:title}
%
%\begin{figure}
%\centering
%\epsfig{file=pic/PageTitle.eps,width=0.8\columnwidth}
%\caption{A Sample Top-K Like Title}
%\label{fig:title}
%\end{figure}

The title of a web page (string enclosed in {\tt<title>} tag) 
helps us identify a ``top-$k$'' page.
The goal of the classifier is to recognize ``top-$k$ like'' titles, 
the likely name of a ``top-$k$'' page. In general, 
a ``top-$k$ like'' title represents the topic of ``top-$k$'' list.
%Figure \ref{fig:title} shows a typical ``top-$k$ like'' title.
Note that a ``top-$k$ like'' title may contain multiple segments, and
usually only one segment describes the topic or concept of the list.

%Besides the features we mentioned in Section \ref{sec:intro} 
%(concept and number $k$),
%a ``top-$k$ like'' title could include some other elements;
%also as a web page, it may contain multiple segments, 
%among which only one segment is the main part.

%Therefore, the actual task for Title Classifier is 
%trying to recognize a proper number k with proper context in the title.
%If no such k is found, we consider the title not a ``top-$k$ like'' title.

%In our implementation, we build our classifier using a supervised machine-learning method.
We trained a Conditional Random Fields (CRF) \cite{CRFLafferty} model
from both positive and negative sample titles.
\cut{%%%%%%%%%%%%%%%%%
from 4000 negative titles (titles that contains a number but 
are not actually ``top-$k$ like'') and 2000 positives titles. The number $k$
is especially important because it serves as an anchor to a phrase that 
represent a ``top-$k$ like'' concept or topic.
We use \textit{word, lemma,} and \textit{POS tag} \cite{StanfordParser}
as the basic feature set.
Among these features, the number k is especially important for 
our system for the following reasons:
\begin{enumerate}
\item The number k is the common feature among all ``top-$k$ like'' titles, 
while other features may omit in some titles
\item The number k is indispensable for following components in our system: 
we need to extract a list with exact k items.
\item We can reduce our target page group to 
``those pages whose title contains at least one number''.
\end{enumerate}

Before we test an input title with the model we learned, 
%we need to transfer it to the format that our model can recognize
%(the same format for training data).
%Thus 
the following preprocessing steps are needed:

\begin{enumerate}
\item \textit{Normalizer}:
Fix some ill-formatted writing in the title and lowercase all the words.
\item \textit{Title Splitter}:
Split the title into segments by splitters such as ``|'' and ``-'', 
and select the longest one with a number as the main segment. 
\item \textit{Feature Generator}:
Generate mentioned features for each word in the main segment.
We use Stanford Parser \cite{StanfordParser} to get the lemma and POS tag features.
After this, we can get a table with words as rows and features as columns.
\end{enumerate}

After that, we can test the feature table of the input title. 
The model will label the number in the title with ``T'' or ``F'',
where ``T'' means the whole title is ``top-$k$ like''.
}%%%%%%%%%%%%%%%%%%%%
In addition to recognizing a ``top-$k$ like'' title, 
%the classifier need to provide information for the following components:
the classifier also transfers the cardinal digit word 
(word like ``ten'' or ``fifteen'') into the number $k$,
and outputs a set of Probase concepts such as 
``scientists'' which are mentioned in the title.



\subsection{Candidate Picker}
\label{sec:picker}
Given an HTML page body and the number $k$, 
the candidate picker collects a set of lists as candidates.
Each list item is a text node in the page body.

We define a {\em tag path} of a node as a path from the root to this node 
in the DOM tree.
Items in a ``top-$k$'' list usually have similar format and style, 
and therefore they share an identical tag path. 
For example, in Table \ref{tab:sampleoutput},
the tag path corresponding to the second column {\em Name} is 
{\tt html/body/.../p/strong}. Our {\em Default} algorithm takes
advantage of this observation to identify lists of size $k$. Note
that there might be multiple lists of the same size 
from a given page.

\cut{%%%%%%%%%%%%%%
Based on this observation, our algorithm runs in four steps:
First, we preprocess the DOM tree to normalize the content of text nodes 
(remove non-printable characters and shorten continuous spaces, etc.). 
Second, we prune the DOM tree by cutting subtrees that include ``blacklisted''
attributes such as ``sidebar'' and ``comment'', because these often indicate
they are not the main content of the page. 
%so that we can get avoid of most advertisements and user comments.
Third, we compute the tag path for every node in the DOM tree of the 
input page. Finally, we group nodes with an identical tag path into 
one {\em equivalence class}, and we
select those equivalence classes which have exactly $k$ members as our
candidate lists. 
}%%%%%%%%%%%%%%%

The {\em Default} algorithm achieves good 
recall but may produce noise. In a modified algorithm, known as
 {\em Def+Patt}, we introduce filters based on more reliable patterns
such as indices and highlighting.

\cut{%%%%%%%%%%%%%%%%%%
We then introduce three additional pattern-based rules to 
filter the candidate lists:

\begin{enumerate}
\item \textit{Index}:
There exists an integer number in front of every list item, serving as
a rank or index: e.g., ``1.'',``2.'',``3.'', ..., the numbers are in sequence
and within the range of $[1, k]$.

\item \textit{Highlighting tag}:
The tag path of the candidate list contains at least one tag 
among {\em <b>,<strong>,<h1-h6>} for highlighting purposes.

\item \textit{Table}:
The candidate list is shown in a table format.
\end{enumerate}

In this modified algorithm, a.k.a. {\em Def+Patt} algorithm,
only candidates that satisfy at least one of the rules above are
kept and output to the next step. 
For example the ``top-$k$'' list in Figure \ref{fig:topscientists} 
satisfies rules 1 and 2.
}%%%%%%%%%%%%%%%



\subsection{Top-K Ranker}
\label{sec:ranker}

When there are multiple candidate lists,
we select only one of them as the {\em main list}.
Intuitively, the main list is the one that best matches the title.
In Subsubsection \ref{sec:title}, we extract a set of concepts from
the title, and one of them should be the central concept of the top-$k$ list.
Our key idea is that one or more items from the main list should be instances
of one of the concepts extracted from the title. For example, if the title
contains the concept ``scientist'', then the items of the main list should
be {\em instances} of the ``scientist'' concept. The Probase taxonomy provides
large number of concepts and their instances. 
For instance, ``scientist'' concept has 2054 instances in Probase.
%Considering the fact that Probase cannot cover all the instances and
%concepts in the world,
We calculate the score of each candidate list $L$ as:

\[Score(L)= \frac{1}{k} \sum_{n \in L} \frac{LMI(n)}{Len(n)}\]
where $LMI(n)$ is the word count of the longest matched
instance in the text of node $n$,
while $Len(n)$ means the word count of the entire text in node $n$.

If there is a tie in $score(L)$, we prefer the list with the largest
{\em visual area} in the page.
The visual area is estimated by calculating text area
of the candidate list:

\[Area(L)= \sum_{n \in L} (TextLength(n)\times FontSize(n)^2).\]

%After we know the main list, we can also get attribute lists that
%are interleaved with the main list.


\subsection{Content Processor}
The content processor takes as input a ``top-$k$'' list and 
extracts the main entities as well
as their attributes. 
%normalized and conceptualized ``top-k list'' to the output.
%It has two major tasks:
Sometimes the text within an HTML text node contains a structure itself, e.g.
``Hamlet By William Shakespeare''. 
The content processor infers the structure of
the text \cite{Fisher08:dirttoshovels} by building a histogram for
all potential separator tokens such as ``By'', ``:'' and ``,'' from all the items
of the ``top-$k$'' list. If we identify a sharp spike in the histogram for a 
particular token, then we successfully find a separator token, and we use that
token to separate the text into multiple fields. 

%It is sometimes useful to provide names to the extracted attribute values. 
%%For example, we want to infer ``name'', ``image'', and ``Wikipedia link'' as 
%%attribute names from the list in Figure \ref{fig:topscientists}. 
%To do this, we conceptualize the extracted columns \cite{Song11:Conceptualize}, 
%using Probase and a Bayesian model.
%%%who utilized Probase \cite{WuLWZ12:Probase} as knowledgebase and
%%%developed a short text understanding system based on Bayesian model.
%%In addition, for special columns like indexes, pictures and long paragraphs,
%%we apply specified rules to conceptualize them.



