\section{Top-K List Extraction Problem}
\label{sec:problem}
This section defines the top-k list extraction problem. 
First, a {\em top-k web page} is an HTML page that describes a list of
exactly $k$ instances of a particular topic or concept. 
That list is known as the {\em top-k list}.
The title of this page must contain the number $k$
either in numeric or alphabetic form. A list item is a segment in
the HTML page that contains multiple facets of the instance, e.g.
the name of the instance, a short textual description, an image and
even an user comment. The top-k list extract problem is defined as,
given an arbitrary web page, automatically extract the top-k list 
if the input page is a top-k page, or return ``fail'' otherwise.

Regarding this definition, there are a few points to note. First,
the list items are HTML segments that may be discontinuous or disjoint
in the whole page. In other words, noise such as an occasional advertisement
segment may exist between two consecutive items. 
Second, even though the list items may look
very similar when displayed in a browser within the same list, 
the actual HTML code segments may be
inconsistent or contain errors, because many of such top-k pages 
are manually edited. Moreover, some lists are styled in a way that is visually
appealing to the viewer (e.g. by using different background colors for
alternate list items), and therefore introduce irregularity in the
HTML code of the list items. Third, while a top-k page contains a ``main'' list
that is the target top-k list, it may very well contain other lists,
and some of them may very well consist of exactly $k$ elements also. 
Therefore identifying the right list to extract is critical to the solving
the problem. 


%A so-called ``Top-K'' web page is a web page that contains a K-item data list,
%and the content of the list should be the main information that 
%the page presents. Such pages are usually featured by their 
%titles, like ``top 10 NBA players'', ``Top 20 Best Video Games'', etc.
%Thus we may get the number K, the count of the list items from the title.
%
%The problem we try to solve is to automatically detect the existence of data list from a potential ``Top-K'' web page 
%and extract the K-item list if it exists.
%
%In other words, our work is to develop an algorithm which satisfy:
%
%
%INPUT: 
%
%A web page; 
%
%A number K.
%
%OUTPUT: 
%
%A K-item data list, 
%
%or an empty list(not exists). 
%
%The fig?? shows the expected result of the algorithm, the input is the page showed in fig?? and the number K.
