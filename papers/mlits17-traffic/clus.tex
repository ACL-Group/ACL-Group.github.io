\subsection{CTMP: A Two-stage Clustering-based Transfer Model }
We introduce our novel \textbf{C}lustering-based \textbf{T}ransfer \textbf{M}odel for \textbf{P}rediction  (CTMP), which first clusters links in both source and target areas based on their spatial features and then do time series based prediction for the target links based on neighboring source links with historical data.

\subsubsection{Intuition Behind the CTMP}
Our intuition behind CTMP is that given a link in target areas with its spatial features, we can first find the most similar links in source areas and then leverage the source data to predict the speed of links in target areas.
Our basic assumption is that links with similar spatial features should also share similar traffic patterns.
However, simply clustering road links based on spatial features performs not very well in practice, because not all the features are equally important and the importances cannot be obtained in such an unsupervised way; 
Therefore, we incorporate a regularization term in the distance metric for feature reduction and selection.

\subsubsection{Clustering with Regularized Distance Metric}
We use the $s_i$ and $s_j$ to denote two spatial feature vectors of any two links $i$ and $j$ respectively. 
We choose to capture the distance between the two feature vectors by computing $$\text{s\_dis}(i,j) = 1-\cos(s_i,s_j).$$
To regularize the time series similarities between two links, we add a regularization term $\text{t\_dis}(i,j)$, which has multiple options.
A desirable option is the DTW similarities between the weekly HAM traffic speed series of the two links.
Thus, the total distance between two links can be regarded as follows, where $\lambda$ is a hyper parameter to control the weight of temporal distance:
$$\text{dis}(i,j) =  \text{s\_dis}(i,j) + \lambda \text{t\_dis}(i,j) $$

With such supervision in the source area data,
we can use K-means as our clustering algorithm.
For each query instance $(l,t)$ \footnote{Typically, the link $l$ has no available historical data.}, we first find the closest $k$ neighboring source links with historical data $\{l_1,...l_k\}$.
We compute all the distances between them and the target link $l$ respectively, and obtain the set of spatial feature distances $$\{\text{dis}(l,l_1),...,\text{dis}(l,l_k)\}.$$
Also, we can get the predicted typical traffic speed for such neighboring links based on existing time-series models  at the time $t$: $\{y(l_1,t),...,y(l_k,t)\}$.
Finally, we can compute the predicted result for the query instance $(l,t)$ is:
$$ y(l,t) = \sum_{i=1}^k \left( \frac{\text{dis}(l,l_i)}{\sum_{j=1}^k \text{dis}(l,l_j)} 
y(l_i,t)
\right)$$ 

%\BL{add a figure to illustrate this novel model}