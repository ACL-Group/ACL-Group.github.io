\section{Data}
In this section, we briefly introduce a public dataset named \textit{UIUC New York City Traffic Estimates}\footnote{\url{https://publish.illinois.edu/dbwork/open-data/}}, on which we extract spatial features and conduct our following experiments.
This dataset covers 700 million trips from 2010 to 2013 in New York City. 
Most importantly, it contains accurate hourly traffic speed measurement for almost all individual links of the NYC road networks.  

Specific data format is described as follows:
1) the road network is represented as a directed graph composed of \textit{nodes} and \textit{links}; 
2) each \textit{node} is an intersection of the road network, with multiple properties like latitude and longitude;  
3) each directed \textit{link} is a small road segment connecting two such nodes;   
4) generally, a real street consists of multiple links;
two-way streets are often represented as two directed links with opposite directions; 
5) each row of the traffic speed data is the average traffic speed of a particular link at a particular hour.
To evaluate transfer learning approaches, we split the road network into five areas as shown in Table~\ref{tab:regions}.
%\footnote{The ``link\_num'' represents the number of the links in each region; 
%	the ``travel\_time\_num'' represents the number of the links in each region; 
%	the ``num\_trips\_sum'' represents the sum of the number of the trips involving all links in each region; }

\begin{table}[th!]
	\centering
	\small
	\caption{Statistics of the Five Areas in New York City}
	\label{tab:regions}
	\begin{tabular}{|c|c|c|c|c|c|}
		\cline{1-6}
		~& \textbf{Hudson}  & \textbf{Manhatten}  & \textbf{Brooklyn}  & \textbf{Bronx}    & \textbf{Queens}    \\ \hline
		\#link           & 730     & 8,578       & 7,790      & 2,113     & 8,173      \\ 
%		travel   & 4949074 & 4450541890 & 241506751 & 15451922 & 412756499 \\ 
		\#trips\_sum & 477,596  & 52,394,074   & 24,121,488  & 5,002,542  & 20,531,664  \\ \hline
	\end{tabular}
\end{table}


