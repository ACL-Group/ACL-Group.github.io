\section{Introduction}
Traffic speed prediction is a challenging problem and has various downstream machine learning applications, many of which are fundamental to intelligent transportation systems, such as
congestion management, transportation planning,  vehicle routing, etc. \cite{Pan2012, Xu2015,Mchugh2015}
%Prediction of traffic speed is important in many ways. It
%can help urban planners optimize roads and arrange land
%usage. Also, it can assist people to organize their itinerary
%and routes more reasonably. In addition, it can help with
%transport control and distribution.
Most of the existing approaches heavily rely on the historical data of the areas being predicted~\cite{Ren2014,Clark2003Traffic}.
However, accurate and reliable historical traffic data collected from road sensors is very expensive and available in only a few areas, where the government can afford the large cost.
Consequently, most state-of-the-art time-series based models cannot be applied easily on those areas where little historical traffic data is available.

Another problem of most existing models is that they only utilize temporal features and do not capture the relationship between spatial features and temporal patterns~\cite{yao2017short,lin2017road}.
Such knowledge can benefit many practical applications of urban computing~\cite{zheng2014urban}, 
but research on extracting effective spatial features for traffic speed prediction is almost missing from the literature.
%Traffic prediction has been well studied by using many classical or newly developing models. Nevertheless, most existing work can only capture the time series pattern. 
%Attempts to explain the traffic speed patterns with both  temporal and spatial features are still left blank. 
%A new framework to form a more efficient and universal prediction to solve the spatial and temporal correlation of traffic is in urgent need.

%To solve the above ``Cold Start Problem'' and 
In this paper, we aim to answer this research questions:
\textit{ How can we exploit the existing historical traffic speed data of some areas to make speed predictions for other areas without their data?}

We attack this problem by proposing a feature-based \textit{transfer learning}~\cite{Pan:Survey, wei2016transfer} approach that exploits both historical traffic speed data of other areas and effective spatial features.
Our contribution in this paper is as follows: 
1) The proposed transfer learning approach supports many classic regression models. 
2) We extract various spatial features in multiple levels and combine them with temporal features to support the transfer learning scenario and improve interpretability of the proposed model. 
3) Experimental results show that proposed transfer learning models based on our effective spatiotemporal features can perform competitively with two classic regression models for predicting traffic speed.
4) To the best of our knowledge, we are among the first to study transfer learning for traffic speed prediction~\cite{xu2016cross}.
%	\item We also built a preliminary immature framework based on the idea of Generative Adversarial Networks to capture the dependency across the neighboring road segments by learning traffic data as images. 

%\section{Related Work}
%In the following sections, we will first talk about the dataset we are interested in and how we extracted spatial features from multiple data sources.
%Then, we will introduce our framework in detail and our experiments. 
%Finally, we would like to talk our preliminary attempts to learn traffic data as images and utilize the idea of Generative Adversarial Networks to generate the joint prediction results. 