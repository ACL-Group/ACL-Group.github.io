\section{Related Work}
\label{sec:related}
%Given a dynamic network represented as network snapshots, the task of temporal clustering focuses on identifying clusters of nodes and detecting their lifetimes. 
%Dynamic community detection is challenging because it is usually difficult to decide the appropriate number for clusters, whose value can significantly affects clustering results. Moreover, clusters are evolving, making lifetime detection difficult.

Indirect dynamic community detection methods usually involves many clusters generated from all the snapshots. 
Previous work such as \cite{c7} only maintains those most frequently appearing ones.
Evolutionary clustering (EC) \cite{c6} applies K-means clustering to a similarity matrix generated from the current snapshot and the clustering result of previous snapshots. 
Researchers have replaced the K-means clustering with other clustering methods or proposed new ways to generate similarity matrices \cite{c26, c16, c3, c14, c28} to improve EC. 
However, these methods use local information (more specifically, a similarity matrix generated from a small subset of snapshots) to identify clusters. 
When snapshots are sparse and contain too few edges, the similarity matrices provide little information for clustering, and the detected clusters can be small and fail to correspond to meaningful clusters. Moreover, nodes are usually assigned to only one cluster in a time step, while in real-world social networks or email networks, a person may be part of multiple communities at the same time.

Evolving stochastic block models (SBM) \cite{c9, c30, c15} are proposed for a dynamic network where each node has a mixed membership of communities defined by the model. 
A probability model is applied to learn the model. 
However, since these methods focus on computing the memberships, the formation, dissolution and lifetime of a community remains unknown. Tensor decomposition based methods such as \cite{c10, c17} model a network as a three mode tensor and apply low-rank tensor factorization to obtain R components. 
Each component consists of three vectors named “loading vectors”. Two of the loading vectors relate to nodes and are used to generate a community with binary classification. 
The other loading vector contains temporal information for tracking the community lifetime. 
However, previous analysis with tensor decomposition fails to provide a good model for the dynamic network; more precisely, the physical interpretations of the vectors related to nodes and the time are unclear. As a result, they inaccurately determine the lifetime of a community when network snapshots are sparse and contain few edges. 
They also fail to provide a way to accurately calculate the lifetime of communities. 
Some methods merge snapshots to analyze data at a large time granularity, but this can result in inaccurate lifetime detection for a cluster because of the loss in details of the change in a cluster.