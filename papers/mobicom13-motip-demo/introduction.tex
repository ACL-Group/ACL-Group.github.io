\section{Introduction}
The rapid growth in mobile applications has called for effective and
inexpensive localization technology, targeting different application domains. 
Most of the past localization research has relied on
GPS, Wi-Fi or cellular based schemes 
\cite{constandache2010did,mobisys:EnTracked,zhuang2010improving}. 
%using traditional ways, like GPS and WiFi/GSM based schemes. 
These schemes essentially represent the trade-off between 
energy consumption and localization accuracy. 
GPS is accurate only when the device sees an open sky, and it's very
power consuming and thus not suitable for extended use on unplugged mobile
devices such as a cell phone. Wi-Fi/GSM based localization consumes less power
and can be used indoors, but does not offer high accuracy. 
Furthermore, it requires the location map 
of all wireless access points which requires either a cooperative 
infrastructure or wardriving which is expensive and unreliable due 
to the dynamic nature of Wi-Fi access points. Other authors attempted to
leverage other sources of information \cite{ravi2007fiatlux,ravi2006indoor}
for indoor localization, but all of the existing techniques focus on
the localization of a single object.

%its continuous usage can drain a phones' battery in less than 8 hours. Alternative WiFi/GSM based schemes offer longer battery life, but at the expense of lower accuracy. GPS cannot be used in-door or in street, but surrounded by tall buildings. Today's in-door localization techniques either require special infrastructure or continuous war-driving. Some authors breaking away from coordinate-based localization\cite{ravi2007fiatlux,ravi2006indoor}, they propose ways to identify logical locations as opposed to physical coordinates. We see this problem from a different angle. Exists technique focus on single person, there is no relation between people. We expand the notion of localization to the social context. By analysis the contact history, infer the location information.  


This paper treats the localization problem from a different perspective. 
Our goal is to simultaneously infer the trajectories of multiple moving objects 
which interact with each other. Picture a large exhibition in which people
roam around a maze of booths and kiosks. When Alice and Bob meet each other, 
they generate a {\em contact} record which consists of two persons' 
ids and a timestamp of the encounter. 
When Alice completes her tour and uploads her own contact history, 
and so does Bob and everybody else, the system will then be able 
to infer the likely trajectories of Alice, Bob and others, i.e.,
when and where they have had the encounters with other people. 
This approach essentially utilizes the {\em social context} as the chief 
input of localization. The other required inputs include a map of the area 
(e.g., map of the exihibition hall) with the right dimensions and the initial 
locations of the objects. 
Our approach doesn't require expensive infrastructure set-up,
or consume a lot of energy. In practice, contacts can be detected by
peer-to-peer cell phone signals, Wi-Fi signals or even high pitched sound \cite{mobileUltraSound}.
%\KZ{cite here!}.
This demonstration generalizes the previously reported
{\em trace inference problem} \cite{wang2011automatic} by allowing objects to
move at variable speed, though following a reasonable speed distribution. 
Such speed distribution can be determined from the type of movement in 
question, e.g., walking, biking, vehicle, etc.

%In this article, we provide a low cost and accurate localization technique. It takes advantage of the mobile devices owner's social context, knowing the people you meet with when you are moving around.
%Except the contacts history, start and end points are required. We suppose people moving speed follow some reasonable distribution.
%We describe the problem with a hypothetical scenario. Imagine that in a big exhibition, lots of people are moving around. Every one carrying a mobile phone which can record when and who he meet with. Kenny is also in the exhibition. When the exhibition ends, Kenny got home, upload his data to the Web site, after some calculation. The system will tell him, whom and where he ever meet. Maybe he can someone shares the same interest with him.
%
%This paper is concerned with the \textit{Trance Inference Problem}, which we proposed in our previous paper \cite{wang2011automatic}, the approach has some limitations. It assume that people moving under a constant speed, this makes it far from reality. We introduce a new method in this paper, which has the same accuracy and complexity, but it does not have a constant speed assumption.
