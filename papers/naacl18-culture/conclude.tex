\vspace{-5pt}
\section{Conclusion}
\vspace{-5pt}
In this paper, we proposed two novel tasks in cross-cultural social media analysis including mining cross-cultural differences of named entities and finding similar terms for slang across languages.
Through extensive experiments with ablation  test, we demonstrated that our proposed lightweight yet effective SocVec-based methods outperform a number of baseline methods.\footnote{We will make our code and data available at \#url\#.}

Future directions include: 1) investigations on cross-cultural differences of concepts, 2) how to connect such knowledge with existing knowledge base, and 3) benefit downstream applications like machine translation for social media messages.

%
% that the cultural properties and social elements of a term (including both entity names and slang terms) can be effectively embedded by its similarities to social words with the help of our proposed 
%\textit{SocVec}, which enables the comparison between two incompatible 
%monolingual semantic spaces. 
%Our proposed framework can be valuable assistance to cross-cultural 
%social studies, acting as a building block for 
%computing such cross-cultural differences and similarities. 
%The two novel tasks with datasets as well as benchmark results also benefit 
%further study in related topics of computational social science.
%, such as detection of cross-cultural differences in named entities and extraction of a bilingual lexicon for Internet slangs.
