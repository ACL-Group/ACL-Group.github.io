%
%\label{sec:eval}
%In this section, 
%Task 1 is detecting cross-cultural differences of named entities and Task 2 is 
%explaining slang terms with terms in another language. 
%Following subsections first discuss preliminary experimental setup and then present our experiments for the two tasks.

\section{Experimental Setup}
\label{sec:prelim}
Prior to evaluating \textit{SocVec} with our two proposed tasks in \secref{sec:mcdne} and \secref{sec:bleis}, we present our preparation steps as follows. 
% need to pre-process the social media copora, perform entity linking, then train mono-lingual word embeddings, and finally collect the bilingual lexicon for common words and
%social word vocabularies, which contain opinion and sentiment related words. 

\textbf{Social Media Corpora}~~
%As the name of our model suggests, microblog corpora are extensively used for both our model and the two tasks (MCDNE and BLEIS).
%One highlight about this dataset is that it contains not only normal microblogs, but also those deleted or censored by force of authority. It enables our model to better look into the true and thorough opinions from netizens in China unhindered by censorship.
Our English Twitter corpus is obtained from Archive Team's Twitter stream 
grab\footnote{\scriptsize{\url{https://archive.org/details/twitterstream}}}.
Our Chinese Weibo corpus comes from Open Weiboscope Data Access\footnote{\scriptsize{\url{http://weiboscope.jmsc.hku.hk/datazip/}}}~\cite{fu2013assessing}.
Both corpora cover the whole year of {2012}.
We then randomly down-sample each corpus to 100 million messages where
each message contains at least 10 characters, normalize the text~\cite{han2012automatically}, 
lemmatize the text~\cite{manning2014stanford} and use LTP~\cite{che2010ltp} to perform word segmentation (for Chinese).

\textbf{Entity Linking and Word Embedding}~~
%After preprocessing the corpora, we first do entity linking.
Entity linking a preprocessing step which links various entity mentions (surface forms) to the identity of corresponding entities~\cite{Ren2017CoTypeJE,Pan2017CrosslingualNT}. 
For the Twitter corpus, we use Wikifier~\cite{cheng2013relational,ratinov2011local}, a widely used
entity linker. 
Because no sophisticated tool for Chinese short text is available, 
we implement our own tool that is greedy for high precision, which prefers to link an entity with a surface form that appears more frequently in our corpus. 
%We utilize context information for selecting entity candidates. 
%That is, only when a surface form is an exact match, or the candidate has been 
%linked in the surrounding text before, or satisfies the occurrence 
%frequency criterion, will it be linked by our tool. Also, we only focus on entities that have both English and Chinese 
%Wikipedia pages.
%We argue that such precision oriented approach is sufficiently good for our tasks, 
%because even if an entity is not recognized, it can still be captured as a normal
%word and contribute to the semantics of other terms.
%Due to the lack of existing state-of-the-art tools for Chinese entity linking and the fact that other tools pursue the balance of precision and recall and tend to generate too many links for our task requiring high precision. Thus we devised our own entity linking method suitable for this particular task.
%
%Basically, we obtain the set of possible surface forms for an entity by collecting all anchor texts of the entity in the
%Wikipedia corpus. In addition, we leverage a redirect system
%of Wikipedia and merge all entities that redirect to each other
%as one entity.
%We can also compute anchor-entity linking frequency
%from the Wikipedia corpus.
%Our entity linking algorithm starts by looking for potential
%anchors in the plain text corpus. 
%We adopt a longest match strategy here that prefers longer anchors. This is because we assume longer anchors are more reliable and bring about higher precision. 
%In Entity linking, we aim for high precision rather than recall because even if an entity is not recognized in the text, its constituent words will still be captured later in the ordinary word vector space and contribute to the semantics of other entities or words.
%After this process we turned our original bilingual corpora into annotated ones with all the Named Entities linked to a shared common entity identifier, the Wikipedia title with both English and Chinese page for a given entity. This bridges an inter-language link for named entities that we would like to mine cultural differences on afterwards.
%
Finally, we use Word2Vec~\cite{Mikolov2013distributed} to train English and Chinese monolingual word embedding respectively.

\textbf{Bilingual Lexicon}~~
\label{sec:blpre}
Our bilingual lexicon is collected from Microsoft Translator\footnote{\scriptsize{\url{http://www.bing.com/translator/api/Dictionary/Lookup?from=en&to=zh-CHS&text=<input_word>}}}, which translates English words to multiple
Chinese words with confidence scores. 
\footnote{All named entities and slang terms used
	in the following experiments are excluded from this bilingual lexicon.}


%between Twitter/Weibo frequent words and the  
%In order to bridge the two cross-lingual word vector spaces for using cooperatively, we require a bilingual lexicon that maps the words between two languages.
%Our bilingual lexicon is built in an one-way manner, i.e, from English words to translated Chinese words. 
%We first derive the intersection of the word set from our Twitter corpus with a publicly available word count list\footnote{http://norvig.com/ngrams/count\_1w100k.txt} with 100,000 most popular words, which is our English Tweets common word list with over 20,000 words. 
%Using Bing Translate API\footnote{http://www.bing.com/translator}, we translate those common words into Chinese. As each English word can be translated into multiple Chinese
%words, and each Chinese word can be translated into multiple English words, with translation confidence ranging from 0 - 1 and sums to 1, this phase generates a many-to-many translation mapping, or bilingual lexicon.

\textbf{Social Word Vocabulary}~~
%As stated in previous sections, in order to capture the sociolinguistic properties and features in social media texts, we need another set of lexicons which represents the meanings that are crucial to our goal and tasks.
\label{sec:sv}
Our social word vocabularies come from 
Empath~\cite{fast2016empath} and OpinionFinder~\cite{choi2005identifying} 
for English, and TextMind~\cite{gao2013developing} for Chinese.
Empath is similar to LIWC~\cite{pennebaker2001linguistic},
but with more words and more categories and publicly available. 
%but agrees with LIWC on many categories they share. 
%It's based on a combination of word embedding models, 
%knowledge bases and crowdsourcing, wherein 
We manually select 91 categories of words that are 
relevant to human perception and psychological processes. 
OpinionFinder consists of words relevant to opinion and sentiment. 
TextMind is a Chinese counterpart for Empath.
%, which is similar to LIWC for analysis the preferences and degrees of different categories in text with an emphasize on Chinese characteristics.
%All of these three vocabularies are used on a combinatorial basis 
%acting as a parameter to our \textit{SocVec} model.
%
In summary, we obtain 3343 words from Empath, 3861 words from OpinionFinder, 
and 5574 unique words in total. 



