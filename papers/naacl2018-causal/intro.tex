\section{Introduction}
\label{sec:intro}
Commonsense causality is the causality between actions or events that is acknowledged by human beings. Commonsense is one thing that shared by nearly all people. It is embraced in large amount of text corpus but hard to mine out because of its sparsity. Commonsense causal reasoning aims to find out the possible causality between events, measuring whether one event can lead to another.
Current work on this area can be summarized into two categories. One is manual annotation such as ConceptNet \cite{singh2002open}. Knowledge using these methods is accurate but quite expensive and rare. It can hardly be used in large scale domain reasoning tasks. The other one is data-driven method, which tries to capture the statistical features between causal events \cite{luo2016commonsense}. These methods give a strength or score for each causal pair and then be applied on word-level or event-level applications. We name the amout of causality between events as \emph{causal strength} short for $SC$ in this paper.

Our method belongs to the second category. However, we are the first to generate the vector representations of cause/effect role for each word and those vectors can be used to not only calculate the word-level scores directly but also use for further sentence-level reasoning. For a word $w$, we believe that it plays a cause(effect) role when it appears in a causal(effect) span in one cause-effect pair. 
For example:
\begin{example}
\noindent
\label{eg:sen}
\begin{itemize}
	\item[(1)] In 1998 the Pont du Gard was hit by major \textbf{flooding} which caused widespread damage in the area.
	\item[(2)] Rainfall is occurring as heavy downpours that cause \textbf{flooding}.
\end{itemize}
\end{example}
In the first sentence, ``flooding'' plays a cause role while in the second sentence it plays an effect role. Therefore, we propose two vectors, $\overrightarrow{c_w}$ and $\overrightarrow{e_w}$ to represent word $w$ for the two roles.
Our main contributions are:
\begin{itemize}
	\item We are the first to represent one word with two vectors of different causal roles (cause and effect).
	\item Word vectors can be used to calculate causal strength directly or applied to further applications.
	\item The dataset we generated shows good reasoning ability.
\end{itemize}