

\renewcommand\arraystretch{1.2}
\setlength\parskip{0.1\baselineskip}
\setlength{\textfloatsep}{0.5cm}
% This is not strictly necessary, and may be commented out,
% but it will improve the layout of the manuscript,
% and will typically save some space.

%\usepackage{geometry}
%\usetikzlibrary{automata,positioning}
%\geometry{left=2.0cm, right=2.0cm, top=2.5cm, bottom=2.5cm}


\appendix

\label{appendix}

\section{Implement details}
% \label{appendix}
We take two group of hyper-parameters: learning rate=5e-5, batch size=32 and learning rate=1e-5, batch size=16. We use $\mathcal{D}_{dev}$ to chose the best one. warm-up steps are set as 10. For each model, we train it for 30 epochs in total. We evaluate it for each epoch starting from epoch 10. 

\section{Analysis of the choice of retrievers}

\begin{table}[!h]
	\centering
	\small
	\begin{tabular}{l|cccc}
		\toprule
		\multirow{2}{*} & \multicolumn{2}{c}{\textbf{SNLI(acc)}} & \multicolumn{2}{c}{\textbf{MNLI(acc)}} \\
		& \textbf{top} & \textbf{mm} & \textbf{top} & \textbf{mm}\\
		\midrule
		random &  68.49 & 69.35 & 59.02 &57.37 \\
		SBERT + static &  66.79 & 71.81 &59.39 &59.78 \\
		BERT-base + dynamic & 72.43 & 72.80 &  61.96& \textbf{62.88}\\
		SBERT + dynamic & \textbf{73.69} & \textbf{74.44} & \textbf{62.57}& 62.57\\
		\bottomrule
	\end{tabular}
	\caption{Impact of the choice of the retriever. $\textbf{random}$: randomly sample demonstrations.}
	\label{table:bert}
\end{table}

We also investigate the effect of the choice of the retriever on the performance. As showed in Table \ref{table:bert}, random retriever performs worst and dynamic SBERT retriever performs best. We also find dynamic pre-trained BERT-base outperforms static BERT, where BERT-base has similar model size with SBERT but it is not fine-tuned with extra dataset to learn a better sentence embedding. This indicates that with dynamic demonstration, the retriever can learn a better sentence embedding with training samples, so a small size of training set is enough to train a retriever well, which makes dynamic demonstration more useful in a low-resource domain.

\section{Generated prompts}

We show the prompt selected by top method and our max-margin method on SNLI and MNLI dataset in Table \ref{table:template1}, Table \ref{table:template2}, Table \ref{table:template3} and Table \ref{table:template4}

%\begin{table*}[!h]
%	\centering
%	\small
%	\begin{tabular}{l|ccccc}
%		\toprule
%		\textbf{condition} & \textbf{13} & \textbf{21} & \textbf{42} & \textbf{87} & \textbf{100}\\
%		\midrule
%		entailment  & Close up of  & A photo of & A black and white photograph of & Description: & A photo of \\
%		neutral &  Here &  Then, & At the end of the day,  & In this video,  &  And \\
%		contradiction & Now &  In the background & At the same time, & The back of & In the distance,\\
%		\bottomrule
%	\end{tabular}
%	\caption{Generated templates of top method in SNLI dataset}
%	\label{table:template1}
%\end{table*}


\begin{table*}[!h]
	\centering
	\small
	\begin{tabular}{l|ccc}
		\toprule
		\textbf{seed} & \textbf{contradiction} & \textbf{neutral} & \textbf{entailment}\\
		\midrule
		13  & Now &  Here & Close up of \\
		21   & In the background &  Then, & A photo of \\
		42   & At the same time, & At the end of the day, & A black and white photograph of \\
		87   & The back of  & In this video, & Description: \\
		100   & In the distance, &  And & A photo of \\
		\bottomrule
	\end{tabular}
	\caption{Generated templates of top method in SNLI dataset}
	\label{table:template1}
\end{table*}

%\begin{table*}[!h]
%	\centering
%	\small
%	\begin{tabular}{l|ccccc}
%		\toprule
%		\textbf{condition} & \textbf{13} & \textbf{21} & \textbf{42} & \textbf{87} & \textbf{100}\\
%		\midrule
%		entailment  & Close up of & One of & A black and white photograph of & A black and white photo of & The photo shows\\
%		neutral &  Later, &  In this photograph, & On the right  & A woman and  &  Here \\
%		contradiction & As  & At the same time & At the same time, & A close up of & Then, \\
%		\bottomrule
%	\end{tabular}
%	\caption{Generated templates of max-margin method in SNLI dataset}
%	\label{table:template2}
%\end{table*}

\begin{table*}[!h]
	\centering
	\small
	\begin{tabular}{l|ccc}
		\toprule
		\textbf{seed} & \textbf{contradiction} & \textbf{neutral} & \textbf{entailment}\\
		\midrule
		13  & As &  Later, & Close up of \\
		21   & At the same time  &  In this photograph, & One of \\
		42   & At the same time, & On the right & A black and white photograph of \\
		87   & A close up of  & A woman and & A black and white photo of \\
		100   & Then, &   Here & The photo shows \\
		\bottomrule
	\end{tabular}
	\caption{Generated templates of max-margin method in SNLI dataset}
	\label{table:template2}
\end{table*}

%\begin{table*}[!h]
%	\centering
%	\small
%	\begin{tabular}{l|ccccc}
%		\toprule
%		\textbf{condition} & \textbf{13} & \textbf{21} & \textbf{42} & \textbf{87} & \textbf{100}\\
%		\midrule
%		entailment  & The &  At the same time & That & For example, & Again,\\
%		neutral &  Although &  For the most part, & At the end of the day,  &Even  &  For the most part \\
%		contradiction & Though  & At the end of the day & Now & Although & Yet \\
%		\bottomrule
%	\end{tabular}
%	\caption{Generated templates of top method in MNLI dataset}
%	\label{table:template3}
%\end{table*}


\begin{table*}[!h]
	\centering
	\small
	\begin{tabular}{l|ccc}
		\toprule
		\textbf{seed} & \textbf{contradiction} & \textbf{neutral} & \textbf{entailment}\\
		\midrule
		13  & Though &   Although & The  \\
		21   & At the end of the day  &  For the most part, &  At the same time \\
		42   & Now & At the end of the day, & That \\
		87   & Although  & Even & For example, \\
		100   & Yet &   For the most part & Again, \\
		\bottomrule
	\end{tabular}
	\caption{Generated templates of top method in MNLI dataset}
	\label{table:template3}
\end{table*}

%\begin{table*}[!h]
%	\centering
%	\small
%	\begin{tabular}{l|ccccc}
%		\toprule
%		\textbf{condition} & \textbf{13} & \textbf{21} & \textbf{42} & \textbf{87} & \textbf{100}\\
%		\midrule
%		entailment  & In the meantime, & As you can see & At least & Finally, & ,\\
%		neutral &  So & However, & The  & Even  &  In fact  \\
%		contradiction & For some reason  & For some reason & Even though & It's not that & But in the end, \\
%		\bottomrule
%	\end{tabular}
%	\caption{Generated templates of max-margin method in MNLI dataset}
%	\label{table:template4}
%\end{table*}

\begin{table*}[!h]
	\centering
	\small
	\begin{tabular}{l|ccc}
		\toprule
		\textbf{seed} & \textbf{contradiction} & \textbf{neutral} & \textbf{entailment}\\
		\midrule
		13  & For some reason &   So & In the meantime,  \\
		21   & For some reason  & However, &  As you can see\\
		42   & Even though & The & At least \\
		87   & It's not that  & Even & Finally, \\
		100   & But in the end, &  In fact & , \\
		\bottomrule
	\end{tabular}
	\caption{Generated templates of max-margin method in MNLI dataset}
	\label{table:template4}
\end{table*}

%\begin{table*}[!h]
%	\centering
%	\small
%	\begin{tabular}{l|ccc}
%		\toprule
%		\textbf{seed} & \textbf{contradiction} & \textbf{neutral} & \textbf{entailment}\\
%		\midrule
%		13  & Now &  Here & Close up of \\
%		21   & In the background &  Then, & A photo of \\
%		42   & At the same time, & At the end of the day, & A black and white photograph of \\
%		87   & The back of  & In this video, & Description: \\
%		100   & In the distance, &  And & A photo of \\
%		\bottomrule
%	\end{tabular}
%	\caption{Statistics of SNLI and MNLI dataset.}
%	\label{table:template}
%\end{table*}