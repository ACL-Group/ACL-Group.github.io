\section{Analysis of Causal Relationships}
\label{sec:case study}
% 还没改 TODO
\KZ{You might wanna merge this section with the prev section. Put a preamble after the heading of every section!}


\subsection{Causality Results}

% Using the mentioned method and experimental setup, we obtained two types of causality: causality between symptoms, and causality between life events and symptoms. The significance of causality is measured by the magnitude of the Average Treatment Effect (ATE) values\MY{We have defined it previously, no need to use full name again. But this argument needs citation}. Here, ATE values indicate how much the probability of other symptoms increases when a symptom/life event occurs.


\subsection{Analysis of Causality}
Figure \ref{fig:LEs_symps_disease} displays the visualization results of causality between life events (LEs) and symptoms. Starting from LEs, lines are drawn to connect related symptoms, with the thickness of the lines representing the strength of the causality. Finally, the symptoms are aggregated according to diseases.
Green nodes represent life events (LEs).Blue nodes represent somatic symptoms. Yellow nodes represent psychological symptoms. Red nodes represent diseases. The size of the LEs nodes indicates the number of symptoms they may cause, with larger nodes indicating more symptoms. The thickness of the lines connecting LEs nodes to symptom nodes represents the magnitude of the Average Treatment Effect (ATE) values, indicating the strength of the causal relationship. This graph only displays one type of disease, anxiety.



\begin{figure}[th]
	\centering
	\includegraphics[width=\linewidth]{figures/LEs_symps_disease.PNG}
	\caption{Visualization results for the causality between life events and symptoms with the time window of 1 year. \KZ{Fonts too small in the fig!}}
	\label{fig:LEs_symps_disease}
\end{figure}
