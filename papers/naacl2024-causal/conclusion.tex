\section{Conclusion}
% \KZ{This conclusion still sounds like abstract. Emphasize the findings and lessons we learned.}
Mental disorders, situated as chronologically evolving diseases, highlight the importance of considering the entire spectrum of symptom development and progression over time. 
In our study, we delved into such chronological aspects of social media posts, uncovering significant causal relationships between symptoms and life events through an observational causal method, the Propensity Scoring Matching analysis. We identified causal links among 38 symptoms and their connections to 11 life event categories. By corroborating our results on \textit{symptom-to-symptom} and \textit{life event-to-symptom} with existing clinical literature, we provided a direct analysis of the causal relationships identified. These findings were then applied to two practical tasks, namely the Diagnosis Point Detection and Early Risk Detection of Depression. Enhanced performance when incorporating causal features on both tasks suggested the effectiveness and necessity of long-term causing relations. Our research underscores the critical importance of causal relations in understanding the complex interplay between symptoms, life events and mental disorders, thus advancing the science of mental disorder prevention and early detection. 
%. First, in the Diagnosis Point Detection, we established the optimal timing for depression diagnosis. Second, in the established Early Recognition of Depression task, we enhanced its effectiveness using an interpolation approach. 
%In this study, our exploration of the chronological characteristics in social media posts led to significant findings regarding causal relationships among symptoms and life events. These causal relationships are inferred using Propensity Scoring Matching.

%Through analysis using PSM, we infer causal relationships among 38 symptoms and their association with 11 categories of life events. Importantly, we applied these causal relationships to practical tasks – firstly, in DPD, where we determined the temporal window for depression diagnosis, and secondly in the well-established ERD task, enhancing its efficacy through an interpolation approach. By substantiating our findings with references to existing literature, we directly analyzed the obtained causal relationships. Our study emphasize the crucial role of causal relationships in early detection and understanding the intricate interplay between symptoms and life events in the realm of mental health analysis.
%In this study, we leveraged the temporal characteristics of social media posts to establish causal relationships among various symptoms and between symptoms and life events. Employing propensity score matching, we identified causal relationships among 38 symptoms and explored connections between 11 categories of life events and these symptoms. Subsequently, we employed these identified causal relationships in two downstream tasks. The first task, DPD, involved determining the temporal window within which an individual was diagnosed with a mental disorder. The second task is ERD, a well-established objective, also benefited from our causal relationships through an interpolation approach. Furthermore, beyond the applications in DPD and ERD, we conducted a comprehensive analysis of the obtained causal relationships. We bolstered our conclusions with references to relevant literature in support of our analysis, grounding our work in the broader context of existing research. 
%In this work, we utilized the temporal characteristics of social media posts to establish causal relationships between symptoms and between symptoms and life events. 
%Using propensity score matching, we identified causal relationships among 38 symptoms and between 11 categories of life events and 38 symptoms. We then applied these causal relationships to two downstream tasks. The first is DPD, \KZ{No need to give full name and then abbrev again.} which aims to identify the time window of an individual's diagnosis of a mental disorder. The second is ERD, a well-established task. We applied the causal relationships to both tasks using interpolation. Additionally, we directly analyzed the obtained causal relationships and cited supporting literature.