\section{Limitations}
In our study, there are some limitations that could be addressed in future research:
\begin{enumerate}
    \item Although the causal relationships between life events and symptoms we identified achieved good results in downstream tasks, and we considered as many common and impactful life events as possible, the 11 categories life events we selected might not cover all events that could potentially affect mental health in life. 
    \item In addition to studying the causal relationships between life events and symptoms, as well as between symptoms themselves, we could also consider other factors and their causal relationships with mental disorders and symptoms. 
    \item Exploration of other downstream tasks involving temporal analysis of mental disorders is necessary. We identified diagnosis point detection and early risk detection here while more tasks can benefit from causal relations.
\end{enumerate}


% (, the 11 categories life events \KZ{Is it 11 categories?} we selected do not cover all events that could potentially affect mental health in life. Achieving this is extremely challenging). 
%Furthermore, 
% \KZ{As a pioneer study, we don't really have to consider all the possible life events.}

%In the downstream task of diagnosis point detection (DPD), we used the RuLSIF\cite{Liu2013Change} model as our baseline, which is a well-established classical model for change point detection (CPD). However, with the rapid development in the CPD field, many models with superior performance have emerged, which might show better results in our task.