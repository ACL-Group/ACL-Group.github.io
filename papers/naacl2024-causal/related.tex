\section{Related Work}
% \KZ{Put a preamble here!}
In this section, we present relevant literature on causality analysis in the context of social media and mental health.
\paragraph{Causal Analysis Methods} 
Generally, there are two ways to establish causal relationships. The first is Randomized Controlled Trial (RCT), typically applied in strict clinical settings~\cite{cipriani2018comparative}; the second one is observational~\cite{PMID:32164822}, including methods like Propensity Score Matching (PSM)~\cite{Rosenbaum1983TheCR} and Regression Discontinuity Design (RDD)~\cite{Cattaneo2019API}—strategies that mimic RCT conditions by meticulously controlling numerous covariates.

RCT provides controlled and stringent conditions, ensuring a high level of internal validity. However, their drawback lies in potential limited generalizability~\cite{KennedyMartin2015ALR}, as the strict conditions and carefully selected participants may not fully mirror real-world diversity. Moreover, RCTs can be resource-intensive and ethically challenging in specific situations, constraining their feasibility for certain research inquiries. 

In this study, we adopt the observational method to infer causality from social media data. We do acknowledge that observational studies are weaker than RCTs in making conclusive causal claims, 
but their validity is supported by statistical literature~\cite{Caliendo2005SomePG}, and they can provide complementary advantages since the analysis is conducted in large population.

\paragraph{Mental-health-related Causality Inference on Social Medial}
Exploring mental-health-related causal relationships is common in clinical studies (i.e. the theoretical and mechanism research in psychiatric diseases). For example, the Network Theory suggests that mental disorders arise from the causal interplay between symptoms~\citep{borsboom2013network}. Additionally, research on the network analysis of depression and anxiety symptoms has revealed potential causal relationships among them, with findings empirically supporting the idea that certain symptoms may act as central hubs, influencing the dynamics of the entire network~\citep{beard2016network}.

However, there is a scarcity of prior research utilizing computational methods to infer the causes of specific psychiatric symptoms or mental disorders on social media. 
Some prior works, such as \citet{Saha2019social}, utilize PSM to infer causal relationships between psychiatric medication use and symptom outcomes on Reddit Corpus. Additionally, \citet{Yuan2023Mental} also utilize similar method to mine the causality between mental health coping and the severity of mental disorders. However, these studies mainly made qualitative conclusions about the inferred causality, while our work makes further exploration by utilizing them quantitatively as features for downstream tasks like mental disease detection.
What's more, other works~\cite{garg2022cams, Saxena2023Explainable} use information retrieval methods to extract causal relationships between stressful events and mental disorders from social media posts. However, these studies focus on extracting direct causal relationships from the semantic information within one post, which may overlook the various long-term, subtle causal relationships that may be absent in a single post.

% between psychiatric medication use and symptom outcomes~\cite{Saha2019social} and the causal relationship between mental health coping and the severity of mental disorders~\cite{Yuan2023Mental}.

% \paragraph{Causality in Mental Health} 
% % \MY{exploration of causal relationships is common in clinical studies, i.e. the theoretical and mechanism research in psychiatric diseases but is there much in detecting causal in CS community and use these causal features for computational purpose? I doubt it. If no, you have to make it clear that this exploration is prevalent in clinical medicine while we overlook such chronological characteristic in automatic prediction and detection tasks. }
% The investigation into causal relationships within the realm of mental health has gained significant attention in recent years. Traditional models in mental health research often focused on identifying singular causal factors contributing to the onset or progression of disorders. In contrast to these traditional models, the network theory of mental illness proposes that mental disorders arise from the causal interplay between symptoms~\citep{borsboom2013network}. Furthermore, research has delved into the network analysis of depression and anxiety symptoms, shedding light on potential causal relationships among these symptoms~\citep{beard2016network}. Their findings empirically support the theory that certain symptoms may function as central hubs, influencing the dynamics of the entire network. 

% The application of statistical and computational methods have facilitated the identification of causal relationships within mental health networks. These methods contribute to determining whether specific symptoms act as causal drivers, leading to the emergence or exacerbation of other symptoms~\citep{epskamp2018estimating}. 
% This exploration of causal relations is prevalent in clinical medicine, yet, intriguingly, such chronological characteristics are often overlooked in automatic prediction and detection tasks.

% This nuanced approach to understanding mental health, emphasizing the interconnectivity of symptoms and the temporal aspects of their emergence, stands as a valuable departure from conventional models. Integrating these insights into automatic prediction and detection tasks could potentially enhance the accuracy and effectiveness of interventions and preventive measures in mental health contexts.

% \paragraph{Causality Analysis through Social Media} In recent years, the exploration of causality\MY{this para is on mental health via social media or causality with social media? Your para title and the first sentence doesn't match.} within the context of mental health analysis has extended to the realm of social media platforms. As individuals increasingly share their thoughts, emotions, and experiences on platforms like Twitter, Facebook, and Instagram, these platforms have become valuable sources of data for understanding mental health dynamics 
% and linguistic features extracted from the posts can be indicative of depression symptoms~\citep{de2013predicting}.
% \citet{saravia2016midas} proposes a data collection mechanism and built predictive models to determine whether a user is suffering from a mental disorder.\citet{kim2020deep} and \citet{uban2021emotion} use deep learning models  to distinguish between users diagnosed with a mental disorder and healthy users.\citet{kelley2022using} studies the network dynamics of depression using social media posts.
% \citet{garg2022cams} introduce a new dataset for Causal Analysis of Mental health issues in Social media posts (CAMS) and make causal interpretation and causal categorization to causal analysis.\citet{garg2023nlp} infer mental health through causal analysis and perception mining on social media. \MY{Minghao, a general comment: related works are not a presentation slide where you listed all relevant works, but to compared the existing works with your work. What has been investigated and what hasn't, what's left and your work shall stand as a hero to fulfill it.}



% Effectively exploring causal relationships involves employing causal methods to mitigate biases associated with the observed symptoms following previously reported symptoms or life events. Previous studies typically use randomized controlled trials(RCTs) conducted in clinical settings~\cite{cipriani2018comparative} to measure causal relationships. RCTs offer controlled and stringent conditions, ensuring a high level of internal validity. This method is particularly effective in minimizing confounding variables and establishing a clear cause-and-effect relationship.

% However, RCTs also come with limitations.However, RCTs also come with limitations. Despite their controlled nature, RCTs may lack external validity~\cite{KennedyMartin2015ALR}, as the strict conditions and carefully selected participant populations may not fully represent real-world diversity. Additionally, RCTs can be resource-intensive, time-consuming, and ethically challenging in certain situations, limiting their feasibility for certain research questions.

% In our work, we adopt an observational study design~\cite{PMID:32164822}, which simulates an RCT setting by diligently controlling for as many covariates as possible. In contrast to RCTs, observational studies, including methods such as Propensity Score Matching (PSM) and Regression Discontinuity Design (RDD)~\cite{Cattaneo2019API}, present an alternative approach to establishing causal relationships. These methods allow for the exploration of causal links in real-world settings, offering a more diverse and potentially generalizable perspective. However, observational studies are susceptible to confounding biases, and achieving balance in covariates may be challenging~\cite{Caliendo2005SomePG}, requiring careful consideration and sophisticated statistical techniques.

%Effectively exploring causal relationships necessitates the use of causal methods to reduce biases associated with the observed symptoms following the previously reported symptoms or life events. 
%Previous studies typically measure causal relationship through RCTs in clinical settings~\cite{cipriani2018comparative}, which is controlled, strict but might lack in generalizability and requires carefully-disgned clinical trials.
%Our work adopts an observational study design. We adopt a causal inference framework based on matching, which simulates an RCT setting by controlling for as many covariates as possible.
%Previous studies typically measure causal relationship through Randomized Controlled Trials (RCTs) in clinical settings~\cite{cipriani2018comparative}. 
%RCTs can control both known and unknown confounders, but they may have significant limitations in terms of generalizability. Additionally, due to ethical considerations, many critical questions in the field of psychiatry cannot be addressed through RCTs. Observational studies, while weaker than RCTs in making definitive causal conclusions, have complementary advantages in various aspects, as they possess deterministic causal relationship~\cite{hannan2008randomized}. The viewpoint that only RCTs can establish causation, and other methods cannot provide useful inferences, is overly simplistic
% \MY{We don't critisize other works in this way! Just say there are two ways to establish causal relations, one is controlled, strict but might lack in generalizability and requires carefully-disgned clinical trials. The other one being obesrvational, including PSM xxx etc. Here you have to mention different observational methods in causality}. 
%Each method has distinct strengths and limitations. We should avoid an overly enthusiastic stance on causal relationships and extreme views that suggest causation cannot be obtained when RCTs are not feasible. We explored propensity scores, which are most suitable for samples where the exposure to risk factors has strong predictive capability.
