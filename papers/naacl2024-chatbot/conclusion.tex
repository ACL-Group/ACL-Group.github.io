\section{Conclusions}
\label{sec:conclusion}

In this work, we proposed a framework named PsyDial to simulate both psychiatrist and patient in depression diagnosis conversation. We collaborated with professional psychiatrists and individuals with mental disorders to precisely define the objectives of these two kinds of chatbots. With their guidance, we developed a comprehensive evaluation framework that takes into account the distinctive characteristics of diagnostic conversations within the mental health domain. Moreover, we explored the potential of using LLM as the underlying technology for developing these chatbots and assessed their performance within our framework, offering valuable insights for future research in this field.

% In this work, we investigated the capacity of ChatGPT to serve as the underlying technology for developing chatbots that can simulate psychiatrists and patients with depressive disorder, respectively. To ensure the validity of our approach, we collaborated with professional psychiatrists who provided their expertise throughout the study. With their guidance, we developed a comprehensive evaluation framework that takes into account the distinctive characteristics of diagnostic conversations within the mental health domain. We then evaluated the performance of these chatbots and observed how varying designs can influence chatbot behavior, which provides valuable insights for future studies in this area.