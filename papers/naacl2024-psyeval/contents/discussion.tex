% !TEX root = ../main.tex

\section{Discussion}

\paragraph{Context Window Limitations.} The model's context window is pivotal, evident in in several tasks within our experiments. The challenge arises when dealing with extensive textual information, such as social media posts or doctor-patient dialogues, exceeding 2k tokens. In these situations, some models struggled to provide effective outputs. Overcoming this challenge is crucial for accurate mental health assessments, as comprehensive understanding often demands analysis of lengthy and context-rich textual information.

\paragraph{Language-Specific Training.} Training models on language-specific data for mental health diagnosis and therapy is paramount. Our experiments emphasize the poor performance of models not trained on the specific language of mental health scenarios. Tokenization issues during encoding further exacerbate the challenge of limited context windows. Future research should prioritize pretraining models on diverse datasets with representative language samples from mental health contexts. Exploring language-adaptive techniques during training and fine-tuning can enhance the model's sensitivity to language nuances, ensuring effective processing of unique linguistic features in mental health scenarios.

\paragraph{Specialized Training for Psychological Diagnosis and Counseling Scenarios.} Targeted training on datasets curated from psychological diagnosis and counseling scenarios is essential. Our experiments reveal the need for models to understand the nuanced dynamics of psychological consultations. Specialized training should expose models to diverse examples capturing the complexity of patient responses. Incorporating multi-modal data and developing techniques to discern discrepancies between expressed concerns and underlying mental health conditions will enhance models' accuracy in navigating mental health scenarios. Augmenting training with authentic datasets from psychological diagnosis and counseling improves models' reliability in these specialized domains.