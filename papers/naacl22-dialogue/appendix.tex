\section{Related Work}
\label{sec:relatedwork}
%Consideration of our approach over previous approaches on 
Dialogue summarization and pretrained language models are discussed as follows.

%\subsection{Dialogue Summarization}

\textbf{Dialogue Summarization:} A growing number of works have been proposed for dialogue summarization in recent years. 
In this work, we mainly refer to the chat summarization defined in~\cite{feng2021survey}.
Previous works widely explore dialogue features explicitly and input them as known labels to enhance the dialogue understanding ability of summarization models.
Features, including dialogue acts~\cite{goo2018abstractive}, topic transitions~\cite{chen2020multi}, discourse dependencies~\cite{chen2021structure}, coreference relations~\cite{liu2021coreference}, argument graphs~\cite{fabbri2021convosumm}, semantic structures or slots~\cite{lei-etal-2021-finer-grain,zhao-etal-2021-give-truth}, etc. are carefully designed and collected by transferring tools pre-trained on other corpus or unsupervised methods with multiple hyper-parameters. 
%Such specific features are not suitable for all scenarios and they highly relied on human labors or experts to redesign heuristic rules or hyper-parameters when implementing on a new dataset.
These work also modify the basic transformer-based models with additional encoders~\cite{chen2020multi} or attention layers~\cite{chen2021structure,liu2021coreference,lei-etal-2021-finer-grain,zhao-etal-2021-give-truth} to utilize the injected features.
\citet{liu-etal-2021-topic-aware} propose a contrastive learning approach for dialogue summarization with multiple training objectives. They also introduce a number of hyper-parameters for contrastive dataset construction and balancing among those objectives.

%Different from previous approaches, we propose to do post-training before dialogue summarization tasks, exploiting the vanilla model's representation ability. And our approach is proved to be more effective with less implementation costs. 


%Multi-view~\cite{chen2020multi} proposes to add separators between input utterances by topic segments and conversational stages with a double-encoder design.
%Structure-BART~\cite{chen2021structure} constructs inter-utterance discourse graphs and intra-utterance action graphs with a graph encoder layer between the encoder and decoder structure.
%ConvoSumm~\cite{fabbri2021convosumm} improves summarization by constructing an argument graph for each dialogue session and linearizing the graph as the input for models.
%DialoBART~\cite{feng2021language} takes advantage of the pretrained response generator model DialoGPT\cite{zhang2020dialogpt} to label features with special tokens including keywords extraction, redundancy detection and topic segmentation.
%Coref-Attention~\cite{liu2021coreference} utilizes coreference resolution tools accompanied with human-designed rules to get coreference links between words in dialogues.

%and may also modify the vanilla pretrained encoder-decoder model to utilize their additional labels.



%\subsection{Pretrained Language Models}

\textbf{Pretrained Language Models:} Previous pretrained seq-to-seq models can be divided into two categories by training data formats.
One is models pretrained on narrative text, such as BART~\cite{lewis2020bart}, PEGASUS~\cite{zhang2020pegasus}, and T5~\cite{raffel2020exploring}. They use training data from Wikipedia, BookCorpus~\cite{zhu2015aligning} and C4~\cite{raffel2020exploring}. These models show great potentials for tasks such as translation and story ending generation. %, where both the input and output are in narrative text format.
The other is models pretrained on dialogue, such as DialoGPT~\cite{zhang2020dialogpt} and PLATO~\cite{bao2020plato}. Their training data are general-domain dialogues, such as Reddit~\cite{henderson2019repository} and Twitter~\cite{cho2014learning}. These models work for dialogue response selection and generation tasks. %, aiming at finding or generating the most suitable utterance given the dialogue history.
All of the above models are trained to exploit language features within the same data format, with pre-training tasks such as masked token/sentence prediction and utterance permutation.
%on tasks including masked language modeling, next utterance or sentence prediction, and utterance or sentence permutation, exploiting the language modeling features within the same data format.
Pretraining with cross-format data hasn't been researched so far. 
%With the rapid growth of single format data understanding and generation mentioned above, cross-format tasks become new hot points, such as dialogue summarization. 
%Different from the previous language modeling training target, 
%such cross-format pretraining focuses on rephrasing between a third-person point of view and multi-speaker views.
As a first step, we focus on narrowing the gap by learning to rephrase unidirectionally from dialogue to narratives.



\section{Implementation Details}
\label{sec:details}
We use BART\footnote{\url{https://huggingface.co/facebook/bart-large}} as our basic language model. For both post-training and fine-tuning, the speakers and utterances of each dialogue are concatenated into a single sequence and truncated to the first $1024$ tokens.
The learning rate is set to $3e-5$ with weight decay equaling $0.01$. The number of warmup steps is $500$ and dropout is $0.1$. The model is tested on the corresponding validation set after each training epoch and % It stops training when the Rouge-2 F1 score doesn't improve on the validation set or it reach the maximum training epoch.
the early-stop is activated if there is no improvement in the Rouge-2 F1 score.
The early-stop and maximum training epochs are set to $3$ and $10$.
During inference, i.e., validation and testing,
the beam size is set to $4$ with length penalty equaling $1.0$
and no-repeat-n-gram size equaling $3$. 
The minimum and maximum lengths are set to the corresponding lengths
of the reference summaries based on statistics of each dataset, allowing for free-length text generation.
Besides, for the inference on the validation set during the post-training stage, we also set the first $3$ tokens as the known prefix. This constant number enables a fair comparison of performances on validation sets under different experimental settings.
All of our experiments are done on an RTX 2080Ti with 11G GPU
memory. We run experiments three times and show the best results following~\cite{feng-etal-2021-language}. 
%We will open-source all of the used datasets and codes.

%We also reimplement the baseline BART fine-tuning
%directly with the above hyper-parameters on different datasets for
%a fair comparison.

\section{Other Types of Paraphrase Datasets}
\label{sec:para}

\begin{table}
	\small
	\centering
	\begin{tabular}{lll}
		\toprule[1pt]
		\textbf{Datasets} & \textbf{Input} & \textbf{Output} \\
		\midrule[1pt]
		{DialIndirect} & $U_{1\sim8}$& \makecell[l]{Katarina says,``Hello, I got ...\\ we work together'' Jill says,\\``Hi :) ......  nice and sunny''}\\
		\midrule[1pt]
		{ExtSum} &$U_3$, $U_6$ & \makecell[l]{Katarina ...... a flat from Liz.\\ She will ...... after 6 pm.} \\
		\midrule[1pt]
		{ExtSumM} &$U_{3\sim6}$&\makecell[l]{Katarina ...... a flat from Liz.\\ She will ...... after 6 pm.}  \\
		\midrule[1pt]
		\multirow{2}{1cm}{{ExtSent/\\ExtSentM}}&$U_3$&Katarina ...... a flat from Liz. \\
		\cmidrule{2-3}
		& $U_6$ &Katarina will ...... after 6 pm. \\
		\midrule[1pt]
		{DSum} & $U_{1\sim8}$& \makecell[l]{Katarina ...... a flat from Liz.\\ She will ...... after 6 pm.}\\
		
		\midrule[1pt]
		\multirow{2}{1cm}{{DialSent}} & $U_{1\sim8}$ &Katarina ...... a flat from Liz. \\
		\cmidrule{2-3}
		&$U_{1\sim8}$ &Katarina will ...... after 6 pm. \\
		
		\bottomrule[1pt]
	\end{tabular}
	\caption{An illustration of post-training pairs generated from the example in Figure~\ref{fig:example}. ExtSent and ExtSentM get the same training pairs in this case.}
	\label{tab:datasets2}
\end{table}


\begin{table}[h]
	\small
	\centering
	\begin{tabular}{lrrrr}
		\toprule[1pt]
		\textbf{Datasets} & \textbf{Train/Val} & \textbf{IW} & \textbf{OW} & \textbf{CR} \\
		\midrule[1pt]
		\multicolumn{5}{l}{\textit{SAMSum}} \\
		{DialIndirect} & 14,731/818 & 124.10 & 157.41 & 1.31 \\
		{ExtSum} & 14,731/818 & 31.23 & 23.44 &0.94  \\
		{ExtSumM} & 14,731/818 & 66.09 &23.44 & 0.69 \\
		{EntSent} & 29,757/1,654 & 31.05 & 11.93 &0.68  \\
		{ExtSentM} & 29,757/1,654 & 46.45 & 11.93 & 0.60 \\
		{DSum} & 14,731/818 & 124.10 & 23.44 & 0.25 \\
		{DialSent} &29,757/1,654  & 149.93 & 11.93 & 0.13 \\
		\midrule[1pt]
		\multicolumn{5}{l}{\textit{DialSumm}} \\
		{DialIndirect} & 12,460/500 & 187.52 & 215.30 & 1.16 \\
		{ExtSum} & 12,460/500 & 44.43 & 30.02 &0.84 \\
		{ExtSumM} & 12,460/500 & 94.32 &31.02 &0.61  \\
		{EntSent} & 22,407/840 & 39.27 &17.78  &0.65  \\
		{ExtSentM} &22,407/840  &61.17  & 17.78 & 0.56 \\
		{DSum} & 12,460/500 & 187.52 & 31.02 & 0.18 \\
		{DialSent} &22,407/840  &214.00 & 17.78 & 0.10 \\
		\bottomrule[1pt]
	\end{tabular}
	\caption{Statistics of constructed datasets. IW and OW refer to the number of words in the input and output of corresponding dataset. DSum and DialSent are in-list for easier comparison.}
	\label{tab:rephrasedatasets}
\end{table}


To make the input and output carry the same amount of information,
one way is to fix $D$ as input and convert utterances into indirect speech
as the output. \citet{ganesh2019restructuring} restructured dialogue into text with complicated rules
%considering discourse relations among utterances for zero-shot scenarios. However, their complicated rules
which are not released and difficult to transfer among datasets under different scenarios. Thus, we only use simple rules to convert all of the utterances into [$r_t$ says,``$u_t$''] and concatenated as the output. We call this dataset as \textbf{DialIndirect}.

Another way is fixing $S$ as output and removing the redundant utterances in $D$ to get the rephrasing input. We take advantage of the idea of oracle extraction for news summarization~\cite{zhou-etal-2018-neural-document} and regard the combination of dialogue utterances with the highest Rouge scores computed with $S$ as the input. 
Considering that utterances are highly dependent, we modify the original extraction algorithm by extracting all of the utterances lying between the extracted ones, 
different from the window-sized snippet selection in~\cite{liu-etal-2021-topic-aware}.  Datasets with or without this modification are called \textbf{ExtSum} and \textbf{ExtSumM} respectively.

A summary $S$ is divided into sentences to construct more rephrase pairs.
%We use Spacy~\footnote{\url{https://spacy.io/}} to
Similar extraction operations can be done between $D$ and $p$, and we 
get \textbf{ExtSent} and \textbf{ExtSentM} datasets.

An example of the paraphrase pair generated from the dialogue-summary pair 
in Figure~\ref{fig:example} is shown
in Table~\ref{tab:datasets2}.
The statistics of post-training datasets derived from SAMSum and DialSumm are shown in Table~\ref{tab:rephrasedatasets}. We compare the performances between different rephrasing approaches with these datasets of our two-stage approach with the fine-tuning-only BART. The results are in Table~\ref{tab:rephrasing}. 

DialIndirect performs incredibly well on SAMSum. However, if we use the converted dialogue as input and directly fine-tune the original BART, the results are only 50.91/28.51/50.25 for Rouge-1/2/L.
It shows that when accompanied with the post-training stage, the model can learn relationships between speakers and utterances, and boundaries of utterances better than a direct transformation of dialogue inputs. 
This rule-based transformation falls on DialSumm compared with BART baseline. More complicated rules may lead to better results, but such labored work is not what we are after. 

\begin{table}[h!]
	\small
	\centering
	\begin{tabular}{lccc}
		\toprule[1pt]
		\textbf{Models} & \textbf{Rouge-1} & \textbf{Rouge-2} & \textbf{Rouge-L} \\
		\midrule[1pt]
		\multicolumn{4}{l}{\textit{SAMSum}} \\
		{BART} &52.06 &27.45 &48.89 \\
		{DialIndirect} &  53.08&28.51  & \textbf{50.25} \\
		{ExtSum} &  53.20& 28.26 & 49.80 \\
		{ExtSumM} & 52.20 & 27.91 & 49.74   \\
		{EntSent} & 51.82 & 27.43 & 49.19   \\
		{ExtSentM} & 51.66 & 27.27 &  48.96  \\
		{DSum} & 52.52&27.51 &49.03 \\
		{DialSent} & \textbf{53.54} & \textbf{28.91} &  50.21 \\
		
		\midrule[1pt]
		
		\multicolumn{4}{l}{\textit{DialSumm}} \\
		{BART} & 53.01&29.18 &51.34 \\
		{DialIndirect} &  52.54&29.13  &51.68 \\
		{ExtSum} & 51.83 & 27.92 &50.33  \\
		{ExtSumM} & 52.29 & 27.72 & 50.09   \\
		{EntSent} & 51.41 & 27.81 & 49.65   \\
		{ExtSentM} & 52.46 & 28.86 & 51.36 \\
		{DSum} & 53.27 & 28.64 & 51.69 \\
		{DialSent} &\textbf{54.73} & \textbf{30.47}&  \textbf{53.46}  \\
		\bottomrule[1pt]
	\end{tabular}
	\caption{Comparisons among different post-training approaches and fine-tuning-only BART baseline on dialogue summarization.}
	\label{tab:rephrasing}
\end{table}


The extraction-based methods fall behind the others. The modification to the algorithm tends to bring more noises than useful information to the input as the results drop mostly. 
Besides, splitting the summary into sentences doesn't improve the results here.
In a word, such hard extractions hurt the intricate discourse and coreference relations among utterances and are not suitable for cross-format data construction.

DialSent with PGG task outperforms other methods and BART consistently across datasets, while DSum with PGG performs almost the same as BART.
If we use DialSent data to augment the original DSum during fine-tuning, the results on SAMSum are 44.61/22.81/44.15 for Rouge-1/2/L respectively showing that the data in both datasets is not compatible. Thus, our approach is different from data augmentation.
Overall, post-training with cross-format rephrasing intuition does help with dialogue summarization,




\section{Case Studies}
\label{sec:cases}
We show more cases as follows.


\begin{table}[th]
	\small
	\centering
	\begin{tabular}{lp{4.9cm}}
		\toprule[1pt]
		{Dialogue} &  \makecell[l]{\textbf{Kate}: Hey, do you know if our medical \\insurance covers hospital costs? \\\textbf{Greg}: Hm, it depends \\\textbf{Mel}: What happened dear? \\\textbf{Kate}: I broke my arm and they're \\sending me to the hospital :/ \\\textbf{Greg}: Call Linda or ask someone at the \\reception, they should be able to tell \\you what kind of package you have \\\textbf{Kate}: thnx} \\
		
		\hline
		{Reference} & \textbf{Kate} broke her arm and she's going to the hospital. She'd like to know whether her medical insurance covers hospital costs. \textbf{Greg} suggests her to call \textbf{Linda} or ask someone at the reception about it. \\
		
		
		\hline
		{BART} &   \textbf{Kate} broke her arm and they're sending her to the hospital. \textbf{Greg} doesn't know if their medical insurance covers hospital costs. \textbf{(53.33/37.93/53.19)}\\
		
		\hline
		{Multi-view} &  \textbf{Kate} broke her arm and they're sending her to the hospital. \textit{\textbf{Greg} will call \textbf{Linda} or ask someone} at the reception to find out if their insurance covers hospital costs.\textbf{(67.64/51.52/56.15)}\\
		
		\hline
		{DialoBART} &   \textbf{Kate} broke her arm and they're sending her to the hospital . \textbf{Greg} advises her to call \textbf{Linda} or ask someone at the reception .\textbf{(65.57/50.85/67.62)}\\
		
		\hline
		{DialSent-PGG} &\textbf{Kate} broke her arm and they're sending her to the hospital. \textbf{Greg} advises her to call \textbf{Linda} or ask someone at the reception if their insurance covers hospital costs. \textbf{(71.64/55.38/62.39)}\\
		\bottomrule[1pt]
	\end{tabular}
	\caption{A case from SAMSum. \textbf{Names} are in bold and \textit{unfaithful contents} are in italic. Rouge-1/2/L scores(\%) are in parentheses.}
	\label{tab:case1}
\end{table}

The case in Table~\ref{tab:case1} is a dialogue happened between three speakers from SAMSum. 
The labeled dialogues, which are directly extracted from Multi-view's and DialoBART's released datasets are shown in Table \ref{tab:case1inputs}. ``$\mid$'' label for Multi-view refers to the topic transitions and stage transitions for the same dialogue respectively. 
We can see that topic segments by Multi-view BART are reasonable. However, such linear segmentation is not quite suitable for this dialogue since the first and third topics are the same.
``$\mid$'' in DialoBART just refers to the end of each utterance.
DialoBART failed to label any topic transitions or redundant utterances.

Compared to the reference summary, the summary generated by BART lost the information about Greg's suggestion, and DialoBART lost the information about ``medical insurance'' even though it recognized ``medical insurance'' as a keyword.
Multi-view did incorrect reasoning on who will call Linda.
Our model generated a more condensed summary covering the same key points as the reference with the original dialogue as input.

\begin{table}[h]
	\small
	\centering
	\begin{tabular}{lp{5cm}}
		\toprule[1pt]
		\makecell[l]{{Multi-view}\\ {Topic}}& Kate: Hey, do you know if our medical insurance covers hospital costs? Greg: Hm, it depends $\mid$ Mel: What happened dear? Kate: I broke my arm and they're sending me to the hospital :/ $\mid$ Greg: Call Linda or ask someone at the reception, they should be able to tell you what kind of package you have Kate: thnx $\mid$
		\\
		
		\hline
		\makecell[l]{{Multi-view}\\ {Stage}} & $\mid$ Kate: Hey, do you know if our medical insurance covers hospital costs? Greg: Hm, it depends Mel: What happened dear? $\mid$ Kate: I broke my arm and they're sending me to the hospital :/ $\mid$ Greg: Call Linda or ask someone at the reception, they should be able to tell you what kind of package you have Kate: thnx
		\\
		
		\hline
		\makecell[l]{{DialoBART}} & Kate : Hey , do you know if our medical insurance covers hospital costs ? $\mid$ Greg : Hm , it depends $\mid$ Mel : What happened dear ? $\mid$ Kate : I broke my arm and they're sending me to the hospital $\mid$ Greg : Call Linda or ask someone at the reception , they should be able to tell you what kind of package you have $\mid$ Kate : thnx \#KEY\# Mel Kate Greg Hey do you know if our medical insurance covers hospital costs happened dear Linda reception package\\
		
		\bottomrule[1pt]
	\end{tabular}
	\caption{Modified inputs by Multi-view and DialoBART.}
	\label{tab:case1inputs}
\end{table}

%\begin{table}[h]
%	\small
%	\centering
%	\begin{tabular}{lp{5cm}}
%		\toprule[1pt]
%		\textbf{Dialogue} & \makecell[l]{\textbf{Linda}: Hi Dad, I want to buy flowers for \\mum! But I don't remember which one\\ she likes :( \\ \textbf{Michael}: Well, she likes all the flowers\\ I believe \\\textbf{Linda}: That doesn't help! I'm on a \\flower market right now! \\ \textbf{Michael}: Send me some pics then \\ \textbf{Linda}: <file\_photo>   \ \textbf{Michael}: Tulips\\ are nice, roses too \\ \textbf{Linda}:  What about carnations? \\ \textbf{Michael}: No, carnations are boring :D \\ \textbf{Linda}: Thanks Dad, srsly… \\ \textbf{Michael}:  What about freesias? She \\likes them a lot, are there any there? \\ \textbf{Linda}: <file\_photo>  \\ \textbf{Michael}: Take those!} \\
		
%		\hline
%		\textbf{Reference}& \textbf{Linda} wants to buy flowers for her mother and asks
%		\textbf{Michael} which flowers does she like. \textbf{Michael} suggests \textbf{Linda} to buy freesias.\\
%		\hline
%		\textbf{BART}&\textbf{Linda} wants to buy flowers for her mum, but she doesn't remember which one she likes. She's on a flower market right now. \textbf{Michael} recommends tulips, roses and freesias. \textbf{(52.83/27.45/51.78)}\\
%		\hline
%		\textbf{Multi-vew}& \textbf{Linda} wants to buy flowers for her mum. She is on a flower market right now. She will buy freesias. \textbf{(52.38/35.00/58.35)}\\
%		\hline
%		\textbf{DialoBART}& \textbf{Linda} wants to buy flowers for her mum . \textbf{Linda} is on a flower market right now . She will buy freesias . \textbf{(57.14/35.00/58.35)} \\
%		\hline
%		\textbf{DialSent-PGG}& \textbf{Linda} is on a flower market right now. She wants to buy flowers for her mother. \textbf{Michael} advises her to buy freesias.\textbf{(63.63/38.10/60.34)}\\
%		\bottomrule[1pt]
%	\end{tabular}
%	\caption{Another case from SAMSum.}
%	\label{tab:case2}
%\end{table}



Another case from DialSumm between two speakers is in Table~\ref{tab:case3}. BART recognized ``him'' in the second utterance as ``\#Person1\#'' incorrectly. DialoBART regarded the man as ``\#Person1\#'s friends'' which isn't mentioned in the original dialogue.  Our model, DialSent-PGG generates a more accurate summary.
%\footnote{The missing or redundancy of ``\#'' for DialSumm exists in generated summaries from all of the neural models.}. 

\begin{table}[th!]
	\small
	\centering
	\begin{tabular}{lp{5cm}}
	\toprule[1pt]
	{Dialogue} & \makecell[l]{\textbf{\#Person1\#}: Like a cat on hot bricks, as \\you might say. I don ' t believe you are\\ listening at all.\\ \textbf{\#Person2\#}: Sorry, I just worried about\\ him. You know, he should be here an \\hour ago. \\ \textbf{\#Person1\#}: Don ' t worry him, he has \\been grown up and I think he can take\\ himself very well. \\ \textbf{\#Person2\#}: But he still does not come\\ back. \\ \textbf{\#Person1\#}: Maybe he is on the way\\ home now.
	} \\
	
	\hline
	{Reference-1} & \textbf{\#Person2\#} is worried about one man, and \textbf{\#Person1\#} thinks that that man might be on the way home now. \\
	\hline
	{Reference-2} & \textbf{\#Person2\#} is worried about a man, but \textbf{\#Person1\#} thinks it would be fine.\\
	\hline
	{Reference -3} &\textbf{\#Person2\#} is worried about a man but \textbf{\#Person1\#} is not.\\
	\hline
	{BART} & \textit{ \textbf{\#Person2\#} is worried about \textbf{\#Person1\#} }because he hasn't come back from work. \textbf{(43.48/28.57/50.01)} \\
	\hline
	{DialoBART} &  \textbf{\#Person2\#} is worried about \textit{\textbf{\#Person1\#}'s friend }who hasn't come back. \textbf{(45.45/30.00/51.87)} \\
	\hline
	{DialSent-PGG} & \textbf{\#Person2\#} is worried about a boy who hasn't come back.\textbf{(47.62/42.11/53.90)}\\
	\bottomrule[1pt]
	\end{tabular}
	\caption{A case from DialSumm.}
\label{tab:case3}
\end{table}






