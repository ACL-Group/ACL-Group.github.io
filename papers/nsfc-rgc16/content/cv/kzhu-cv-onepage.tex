\documentclass[10pt,a4paper]{article}
\usepackage{epsfig}
\usepackage{fancyhdr}
%\usepackage{pslatex}
%\usepackage{time}
\setlength{\textheight}{10in}
\setlength{\textwidth}{6.8in}
\setlength{\columnsep}{0.3125in}
\setlength{\topmargin}{-0.5cm}
\setlength{\headheight}{0in}
\setlength{\headsep}{0cm}
\setlength{\footskip}{0mm}
\setlength{\parindent}{1pc}
\setlength{\oddsidemargin}{-0.5cm}
\setlength{\evensidemargin}{-0.5cm}
\newcommand{\sk}{\vspace{1mm}}
\newcommand{\hs}{\hspace{10mm}}
\newcommand{\cut}[1]{}
%\renewcommand{\baselinestretch}{1.2}
\pagestyle{fancy}
\fancyfoot{}
%\title{\bf Kenny Qili Zhu}
%\author{\small Department of Computer Science \& Engineering\\
%\small Shanghai Jiao Tong University\\
%\small 800 Dongchuan Road, Shanghai 200240, China\\
%\small http://www.cs.sjtu.edu.cn/\~{}kzhu/\\
%\small kzhu@cs.sjtu.edu.cn \\
%\small Tel: +86-21-3420-7231  Mobile: +86-13918272740  Fax: +86-21-34204728 \\
%}
%\date{\small \today}
\begin{document}
%\maketitle
%\sf
\section*{Kenny Q. Zhu}

%\noindent
%\begin{minipage}[r]{0.5\columnwidth}
%%\epsfig{file=kzhu_official.eps, width=100pt} 
%Department of Computer Science \& Engineering\\
%Shanghai Jiao Tong University \\
%800 Dongchuan Road, \\
%Shanghai 200240, China\\
%\end{minipage}
%\hfill
%\begin{minipage}[l]{0.5\columnwidth}
%Phone: +86-139-1827-2740\\
%Fax: +86-21-3420-4728\\ 
%Email: {\tt kzhu@cs.sjtu.edu.cn} \\
%URL: \verb#http://www.cs.sjtu.edu.cn/~kzhu/# \\
%%Date of Birth: January 2, 1976 \\
%%Place of Birth: Nanjing, China \\
%%Nationality: Singaporean\\
%%Residency: US Resident (H1B)
%\end{minipage}
%%\sk

\subsection*{Positions}
\begin{quote}
Distinguished Research Professor (PhD Advisor), 
Shanghai Jiao Tong University, January 2010-. 

%Associate Professor, Shanghai Jiao Tong University, September 2009-.
%
Visiting Professor, Microsoft Research Asia, February 2010 - September 2010.

Postdoctoral Researcher \& Lecturer, Princeton University, January 2007 - August 2009.

Software Design Engineer, Microsoft Corp., October 2005 - December 2006.
%Research Fellow, National University of Singapore, June 2005 - September 2005.
%Teaching Assistant, National University of Singapore, July 2001 - June 2005.\\
%Research Assistant, National University of Singapore, July 1999 - June 2001.\\
%Intern, Hewlett Packard Singapore Pte Ltd., January 1998 - June 1998.
\end{quote}

\subsection*{Research Interests}
\begin{quote}
Data and Knowledge Engineering, Artificial Intelligence, Programming Languages
\end{quote}

%Coordination systems \\
%Declarative languages\\
%Database indexing and transaction management\\
%Agent-based programming languages, reactive systems \\
%Grid/peer-to-peer computing\\
%Combinatorial optimization, evolutionary computing\\

%\noindent
%\begin{minipage}[t]{0.2\columnwidth}
%TEACHING\\ INTEREST 
%\end{minipage}
%\hfill
%\begin{minipage}[t]{0.8\columnwidth}
%Introduction to computing\\
%Introductory programming courses\\
%Programming languages\\
%Parallel/distributed systems\\
%Parallel/concurrent programming \\
%Heuristics in combinatorial problems, etc. \\
%\end{minipage}
%
\subsection*{Education}
\begin{quote}
Ph.D. in Computer Science, National University of Singapore, 2006.

B.Eng. in Computer Engineering (Hons), National University of Singapore, 1999.
\end{quote}

\subsection*{Selected Awards and Honors}
\begin{quote}
2014 DASFAA Best Paper Award -
``$\rho$-uncertainty Anonymization by Partial Suppression''.

2013 Google Faculty Research Award - ``Action Conceptualization from Text Documents'' ({\em only winner in mainland China}).

%2013 Model University Course Instructed in English - ``Database System Concepts'', Shanghai.
%
%%2013 Advisor for the Chun-Tsung Program.
%%2012 Candle Light Award, SJTU.
%
%2012 First Prize (\#1), Teach-in-English Competition, SJTU.

2010 Microsoft Research Asia Young Faculty Visiting Program.

%2004 Deans Graduate Award, School of Computing, NUS.
%\item Participated in Undergraduate Research Opportunity Program.\\
%\item Project title: Color Image Segmentation Using Neural Network with Application in Computer Aided Surgery. \\
%Project supervisor: Dr. S. Ranganath.\\
\end{quote}

\subsection*{Selected Recent Publications}
%(* marks the contact author.)

\begin{quote}
Yu Gong, Kaiqi Zhao and {\bf Kenny Q. Zhu}. Representing Verbs as Argument Concepts.
{\it In AAAI 2016.}

Peipei Li, Haixun Wang, {\bf Kenny Q. Zhu}, Zhongyuan Wang, Xuegang Hu, Xindong Wu. A Large Probabilistic Semantic Network based Approach to Compute Term Similarity.
{\it In TKDE, 27(5), 2015.}

Keyang Zhang, {\bf Kenny Q. Zhu} and Seung-Won Hwang. An Association Network for Computing Semantic Relatedness. {\it In AAAI 2015.}

Ke Wu, Song Yang and {\bf Kenny Q. Zhu}. False Rumor Detection on Sina Weibo by Propagation Structures. {\it In ICDE 2015.}

%Kaiqi Zhao, Gao Cong, Quan Yuan and {\bf Kenny Q. Zhu}.
%SAR: A Sentiment-Aspect-Region Model for User Preference Analysis in
%Geo-tagged Reviews. {\it In ICDE 2015.}

Hongsong Li, {\bf Kenny Q. Zhu} and Haixun Wang.
Data-Driven Metaphor Recognition and Explanation.
{\it In the Transactions of ACL, Vol 1. Appeared at ACL 2014.}

%Zhiyuan Cai, Kaiqi Zhao, {\bf Kenny Q. Zhu} and Haixun Wang.
%Wikification via Link Co-occurrence.
%{\it In CIKM 2013.}
%%
%%Peipei Li, Haixun Wang, {\bf Kenny Q. Zhu}, Zhongyuan Wang and Xindong Wu.
%%Computing Term Similarity by Large Probabilistic isA Knowledge.
%%{\it In CIKM 2013.}
%
%Zhixian Zhang, {\bf Kenny Q. Zhu}, Haixun Wang, Hongsong Li. Automatic Extraction
%of Top-k Lists from the Web. {\it In ICDE 2013.}

%Yue Wang, Hongsong Li, Haixun Wang and {\bf Kenny Q. Zhu}.
%Toward Topic Search on the Web. {\it In ER 2012.}
%
%Jingjing Wang, Haixun Wang, Zhongyuan Wang and {\bf Kenny Q. Zhu}.
%Understanding Tables on the Web. {\it In ER 2012.}

%Zhixian Zhang, {\bf Kenny Q. Zhu}, Haixun Wang. A System for Extracting Top-K Lists from the Web. {\it In KDD 2012.}

%Xiao Jia, {\bf Kenny Q. Zhu}, Joxan Jaffar, Roland H.C. Yap.
%A Runtime System for Generalized Committed Choice.
%{\it In ACM Asia-Pacific Programming Language and Compilers
%Workshop (APPLC'12, affiliated with PLDI 2012).}

Wentao Wu, Hongsong Li, Haixun Wang, {\bf Kenny Q. Zhu}.
Probase: A Probabilistic Taxonomy for Text Understanding.
{\it In SIGMOD 2012.}

%{\bf Kenny Q. Zhu}, Kathleen Fisher and David Walker. 
%LearnPADS++: Incremental Inference of Ad Hoc Data Formats.
%{\it In 14th International Symposium of Practical Aspects of Declarative Languages (PADL 2012).}
%
%Pengcheng Wang, Zhaoyu Gao, Xinhui Xu, Yujiao Zhou, Haojin Zhu 
%and Kenny Q. Zhu. 
%Automatic Inference of Movements from Contact Histories. 
%{\it In ACM SIGCOMM 2011.}
%
%Kathleen Fisher, Nate Foster, David Walker and Kenny Q. Zhu. 
%Forest: A Language and Toolkit For Programming with Filestores.
%{\it In ICFP 2011.} 
%
%Kenny Q. Zhu, Kathleen Fisher and David Walker.
%Incremental Learning of System Log Formats.
%{\it ACM OS Review. 44(1), 2010.}

%Kenny Q. Zhu, Kathleen Fisher, and David Walker. 
%Incremental Learning of System Log Formats. 
%{\it In SOSP Workshop of Analysis of System Logs,
%WASL 2009.   
%} 

%Kenny Q. Zhu, Daniel S. Dantas, Kathleen Fisher,
%Limin Jia, Yitzhak Mandelbaum, Vivek Pai and David Walker. 
%Language Support for Processing Distributed Ad Hoc Data. 
%{\it In 11th International ACM SIGPLAN Symposium on
%Principles and Practice of Declarative Programming, PPDP 2009.
%} 

%Kathleen Fisher, David Walker, Kenny Q. Zhu* and Peter White.
%From Dirt to Shovels: Fully Automatic Tool Generation from Ad Hoc Data. 
%{\it ACM SIGPLAN Notices. Volume 43, Issue 1, pp. 421--434.}
%
%Qian Xi, Kathleen Fisher, David Walker and Kenny Q. Zhu.
%Ad Hoc Data and the Token Ambiguity Problem. 
%{\it In 11th International Symposium on Practical Aspects of 
%Declarative Languages, PADL 2009.
%}

%Kathleen Fisher, David Walker and Kenny Q. Zhu*. 
%LearnPADS: Fully Automatic Tool Generation from Ad Hoc Data. 
%{\it In SIGMOD 2008, pp.1299--1302.}
%
%Kathleen Fisher, David Walker, Kenny Q. Zhu and Peter White.
%From Dirt to Shovels: Fully Automatic Tool Generation from Ad Hoc Data. 
%{\it In POPL 2008, pp.421--434.}

%David Burke, Kathleen Fisher, David Walker, Peter White and Kenny Q. Zhu*.
%Towards 1-click Tool Generation with PADS. 
%{\it ICML Workshop on Challenges and Applications of Grammar Induction, 2007.} 
%
%Joxan Jaffar, Roland H.C. Yap and Kenny Q. Zhu*.
%Generalized Committed Choice.
%{\it In the 9th International Conference on 
%Coordination Models and Languages, COORDINATION 2007, 
%pp.191--210.}
%
%Joxan Jaffar, Roland H.C. Yap and Kenny Q. Zhu*.
%Indexing for Dynamic Abstract Regions.
%{\it In ICDE 2006.}
%
%Joxan Jaffar, Roland H.C. Yap and Kenny Q. Zhu*.
%Coordinating Many Agents.
%{\it In ICLP 2005.} 
%
%Kenny Q. Zhu. Open Constraint Programming. {\it Doctor of 
%Philosophy Dissertation, 2005.} 
%URL: \verb#http://www.cs.princeton.edu/~kzhu/papers/phd_thesis.pdf#
%\end{quote}
%
%\noindent
%{\bf Before 2005}
%\begin{quote}
%Joxan Jaffar, Andrew E. Santosa, Roland H.C. Yap and Kenny Q. Zhu*.
%Scalable Distributed Depth-First Search with Greedy Work Stealing. 
%{\it In Proceedings of the 16th IEEE International Conference on Tools 
%for Artificial Intelligence, ICTAI 2004, pp. 98--103.}
%
%Kenny Q. Zhu and Ziwei Liu. 
%Population Diversity in Permutation-Based Genetic Algorithm. 
%{\it In Proceedings of European Conference on Machine Learning, ECML 2004, 
%pp. 537--547.}
%
%Kenny Q. Zhu and Ziwei Liu. Empirical Study of Population Diversity 
%in Permutation-Based Genetic Algorithm. 
%{\it In Proceedings of Genetic and Evolutionary Computation Conference, 
%GECCO 2004, pp. 420--421.}
%
%Kenny Q. Zhu. A Diversity-controlling Adaptive Genetic Algorithm for the 
%Vehicle Routing Problem with Time Windows. 
%{\it In Proceedings of 15th IEEE International Conference on Tools for Artificial Intelligence, 
%ICTAI 2003, pp. 176--183.}
%
%Kenny Q. Zhu and Andrew E. Santosa. A Web Meeting Scheduling 
%System Based on Open Constraint Programming. 
%{\it In Proceedings of International Conference of Advance 
%Informations System Engineering, CAiSE'02, pp. 792--795.}
%
%Kenny Q. Zhu, Wee-Yeh Tan, Andrew Santosa and Roland Yap. 
%Reactive Web Agents with OCP. 
%{\it In Proceedings of International Symposium of 
%Autonomous Decentralized Systems, ISADS 2001, Dallas, Texas, pp. 251--254.}
%
%K. C. Tan, L. H. Lee, Q. L. Zhu and K. Ou. Heuristic methods for vehicle routing problem with time windows. {\it Artificial Intelligence in Engineering (2001) pp. 281--295.}
%
%Kenny Q. Zhu, Kar-Loon Ong, 
%A Reactive Method for Real Time Dynamic Vehicle Routing Problems. 
%{\it In Proceedings of the 12th IEEE International Conference 
%on Tools for Artificial Intelligence, ICTAI 2000, Vancouver, Canada, pp. 176--180.}
%
%K.C. Tan, L.H. Lee and Kenny Q. Zhu. Heuristics for VRPTW. 
%{\it In Proceedings of 6th International Symposium on 
%Artificial Intelligence and Mathematics, AMAI 2000.}
%%{\bf Work in Progress}
%%
%%\noindent
%%\begin{minipage}[t]{0.2\columnwidth}
%%\end{minipage}
%%\hfill
%%\begin{minipage}[t]{0.8\columnwidth}
%%Kenny Q. Zhu, Kathleen Fisher and David Walker.
%%Incremental Learning of Formats for Large-Scale Ad Hoc Data.
%%{\it Manuscript.}
%%\\
%%\end{minipage}
%%
%%\noindent
%%\begin{minipage}[t]{0.2\columnwidth}
%%\end{minipage}
%%\hfill
%%\begin{minipage}[t]{0.8\columnwidth}
%%Yibo Fan and Kenny Q. Zhu
%%A SystemVerilog Extension for Rapid and Interactive IP Synthesis.
%%{\it Manuscript.}
%%\\
%%\end{minipage}
%
%%\noindent
%%\begin{minipage}[t]{0.2\columnwidth}
%%\end{minipage}
%%\hfill
%%\begin{minipage}[t]{0.2\columnwidth}
%%\end{minipage}
%%\hfill
%%\begin{minipage}[t]{0.8\columnwidth}
%%Joxan Jaffar, Roland H.C. Yap and Kenny Q. Zhu.
%%RC-tree: A Multi-Dimensional Spatial Index for Overlapping Shapes\\
%%\end{minipage}
%%
%%\noindent
%%\begin{minipage}[t]{0.2\columnwidth}
%%\end{minipage}
%%\hfill
%%\begin{minipage}[t]{0.2\columnwidth}
%%\end{minipage}
%%\hfill
%%\begin{minipage}[t]{0.8\columnwidth}
%%Joxan Jaffar, Roland H.C. Yap and Kenny Q. Zhu.
%%Speculated Transactions. \\
%%\end{minipage}
%
%Kenny Q. Zhu. Heuristics Methods for Vehicle Routing Problem with time Windows. {\it Bachelor of Engineering Thesis, 1999.} 
\end{quote}

\subsection*{Selected Grants}
\begin{quote}
%Morgan Stanley Joint Research Scheme (PI), ``Dead Code Detection in Scala
%Compiler Source'', 2015. 
%
NSFC China-Korea (PI), ``A Study on
Multi-lingual, Cross-cultural Association Networks'', 2014. 

Google Faculty Research Award (PI), ``Action Conceptualization from
Text Documents'', 2013. 

Oracle Joint Research Scheme (PI), ``Automatic Extraction of Computing Topics
from Large Code Repositories'', 2013-2014. 

AstraZeneca Joint Scheme (PI), ``Information Extraction from Medical
Documents'', 2013-2015. 

NSFC General (No. 61373031) (PI), ``An Ontology Based on Action
Concepts and Its Applications on Text Processing'', 2014-2017.

%%SJTU Distinguished Research Fellow Award (PI), 2010-2012.
%%
%%NSFC Key Program (No. 61033002) (Co-PI), 2011-2014.
%
%%SJTU-Morgan Stanley Joint Research Scheme (Co-PI), 2011-2012, RMB 120,000.
%
%Microsoft Joint Research Scheme (PI), 2011-2012.
%
NSFC Youth (No. 61100050) (PI), ``Incremental Format Inference from Ad Hoc Data'', 2012-2014. 
%
%Chinese Ministry of Education New Faculty Award (No. 20110073120023) (PI), 2012-2014.
%%
%%Shanghai Bai Yu Lan Science and Technology Grant (PI), 2012, RMB 22,000.
\end{quote}
%
%\subsection*{Selected Patents}
%\begin{quote}
%US Patent No. 2007-1053 (Granted). ``Format Inference for Ad Hoc Data''. 
%
%
%US Patent No. 2009-0801 (Granted). ``Incremental Learning of Format 
%Descriptions''. 
%
%Chinese Patent No. 201210109622.9 (Granted). ``A TV Program Recommendation 
%System for Real Time Streams''. 
%
%%Chinese Patent No. 201210110031.3 (Pending). ``Automatic Labeling of TV 
%%Programs''. Filed in April, 2012.
%
%Chinese Patent No. 201210285469.5 (Granted). ``A System for Anonymizing 
%Set-valued Data by Partial Suppression''. 
%\end{quote}
%
\subsection*{Professional Services}
\begin{quote}
Area Chair: WAIM 2011.

PC member: ECML 2015, SAC 2015, SAC 2014, COLING 2014, ECML 2014, APLAS 2013, SAC 2013, ODBASE 2013, WWW 2013, CIKM 2012, ECML 2011, NDBC 2011.
\end{quote}
\end{document}
