\section{Conclusion}
%In this study, we introduced ICQ, a lightweight yet effective framework 
%for uncovering statistical biases and cues in NLU datasets.
%Utilizing two straightforward tests, 
%ICQ enables the evaluation of a model's dependency 
%on spurious cues, both quantitatively and visually. 
%Our extensive examination of 10 prominent NLU datasets 
%and 4 models substantiates prior findings and unveils 
%novel insights. The supplementary online demonstration 
%platform empowers users to assess their own datasets and models. 
%Moreover, we delved into the art of designing prompts to 
%minimize ChatGPT's biases, providing invaluable 
%guidance for real-world applications.
%
%We develop a light-weight 
%framework that evaluates the potential biases and cues in NLR multiple-choice 
%datasets and further shed light on
%the exploration of models at least from the perspective of the statistical cues. 
%We experimented on a large range of datasets covering different tasks and 
%conclude that the new evaluation framework is effective in discovering
%bias problems in both the datasets and some popular models.
%Ultimately, ICQ paves the way for a better understanding and 
%optimization of large language models, 
%fostering the development of more robust and unbiased AI systems.
%
We have developed a lightweight framework, ICQ, that evaluates potential biases and cues in NLU multiple-choice datasets and sheds light on model exploration from a statistical cue perspective. Our experiments span a wide range of datasets covering various tasks, demonstrating the effectiveness of our evaluation framework in uncovering bias issues in both datasets and popular models. Through a case study, we investigate ChatGPT's bias and provide valuable recommendations for practical applications. Ultimately, ICQ paves the way for a better understanding and optimization of large language models, fostering the development of more robust and unbiased AI systems.
