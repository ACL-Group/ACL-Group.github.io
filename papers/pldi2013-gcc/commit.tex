\subsection{More on Commit}\label{sec:commit}
Commit is a special and important operation in speculative nondeterminism because:
(i) it gives the agent the power to specify preferences among different choices,
e.g., in Listing \ref{lst:intro-commit-sleep}; 
(ii) it prunes some of the virtual worlds to make it easier for agents to exit
because the fewer remaining worlds means easier consistency checks,
and thus it improves the responsiveness and overall throughput of the system;
(iii) by pruning worlds, it also reclaims precious system resources such as
memory and CPU cycles, which is critical to the viability of the multi-world 
speculation , even though the naive combinatorial problem space is exponential and
prohibitive.

However, it is also important to understand that whenever commit is used and
pruning is done, potential solutions can be pruned away, and deadlock or blocking
may arise as a result. So from the system point of view, we design a commit
semantics that is {\em localized} and less aggressive because the scope of the pruning
is restricted to be of height 1 only and affects only a small fraction of
the whole tree (see Rule \ref{rule:cm}). This commit semantics is also
more efficient because pruning decisions are made locally. There are of course
other possible commit semantics \cite{JaffarYZ07} which are more eager in pruning. 
For example, one type of commit requires agent $T$ to commit in all left or right
subtree rooted at $\oplus_T$. These are either too aggressive and lose 
too many solutions or require coordination
among various worlds which is more costly to execute in practice.

From the programmer's point of view, she should use commit with care, knowing that
without commit, the multi-world space grows very quickly and the program probably
cannot scale, but with commit, the program could potentially lose valuable solution.
It makes more sense to put commit later rather than early in the choice. 
If one must put a commit in the middle of a sequence of operations like
$op_1.op_2.\cm.op_3.op_4$, she should ensure that the remaining operations
after $\cm$, i.e., $op_3$ and $op_4$, are not likely to block, 
e.g., when they are local computations.

%The commit semantics described in Section \ref{sec:semantics} 
%is called {\em localized commit} as it restricts
%the scope of pruning to be of height 1 only.  
%%$\cm$ by agent $X$ cannot
%%prune until the direct parent of the committed world is $\oplus_X$.
%%$\cu$ kills the current world immediately without coordination with other worlds.
%Besides localized commit, there is a number of other possible commit
%semantics \cite{JaffarYZ07}.
%They differ in their eagerness to prune the worlds.
%They are ordered roughly from the most eager to the most conservative:
%{\em absolutely eager commit}, {\em eager commit}, {\em coordinated commit}, 
%{\em late commit} and {\em no commit}. 
%

