\subsection{Data Model}\label{sec:datamodel}

In speculative nondeterminism, we assume a 
centralized data store acting as the primary programming environment. The implementation of this store is
not part of the language and therefore independent from the operational semantics of the language.
Here we present the abstract data model of the store that can be instantiated into different
concrete models.

A data model is a 6-tuple
$$\dm=\langle D_0,\alltype,\allop,\allstore,\vdash,\psi\rangle$$
where
\begin{description}
  \item[$D_0$] defines an empty data store, for example, 
    an empty set $\emptyset=\{\}$ or an empty multi-set $\lbb\rbb$.
  \item[$\alltype$] is the set of all possible data items in the data model.
    For example, $\alltype=\nat$ for data models such as a set of natural numbers. 
  \item[$\allop$] is the set of all data operations available in the data model.
  \item[$\allstore$] is the set of all possible data stores.
    For example, $\allstore=2^\nat$ for data models such as a set of natural numbers.
  \item[$\vdash\subseteq\allstore\times\allop$] is a binary relation denoted by $D\vdash d$.
    It defines when operation $d$ can be executed on the data store $D$.
    For example, if the operation $d$ has a precondition, $D\vdash d$ is
    true if and only if the condition is true when evaluated on the data store $D$.
  \item[$\psi:\allop\times\allstore\to\alltype\times\allstore$] is the transition function that
    defines the effect of data operations on the store. 
    In general, a data operation can do read, update, or both. 
    For read purpose, a data item $t\in\alltype$ in the data store must be returned from $\psi$. 
\end{description}

Next we give some examples of concrete data models under the above framework.

\subsubsection*{Tuple Space}

The idea of tuple space comes from Linda \cite{Gelernter85:Linda}, where
a tuple space is a multi-set of tuples that can be accessed concurrently.
A tuple is an ordered collection of fields. 
Agents can post their data to the tuple space in the form of tuples, and 
retrieve tuples as data from the tuple space that match a certain pattern.
There are three major operations in the tuple space data model:
(i) $\In$ reads and removes a tuple from a tuple space;
(ii) $\Out$ produces a tuple, writing it into a tuple space;
(iii) $\Rd$ non-destructively reads a tuple space and gets a copy of a tuple. 
Both $\In$ and $\Rd$ are blocking while $\Out$ is non-blocking.
Formal definitions and operational semantics for
the $\In/\Out/\Rd$ operations \cite{Ciancarini95}
can be easily adapted to this framework.

\subsubsection*{Key-value Store}

{\em Key-value store} is a mapping from keys 
to values, e.g.
$[a\mapsto 3, b\mapsto 5]$ is a key-value store
where the key $a$ gets value $3$ and $b$ gets $5$.
Typical data operations include: (i) creating a new key-value pair,
(ii) updating an existing key with a new value,
(iii) getting a value according to a key, and
(iv) removing a key and its corresponding value.
Conditional guards can also be combined with these data operations,
e.g., $a>4\Rightarrow b\gets 2$ waits (blocks) until
the condition $a>4$ is true in the store,
and then updates the value of $b$ to $2$.

\subsubsection*{Other Models}

Other more structured data models include relational data and logic programs.
Data operations are insertion and deletion of tuples/predicates. Applications
using these models are also typical.
