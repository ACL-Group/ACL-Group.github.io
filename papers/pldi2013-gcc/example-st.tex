\subsubsection*{Stock Trading (ST)}

In this example, we set up a mini stock market which follows the prices in the public exchange 
but is not openly available to the public. 
Traders can join the market at any time but their trades will not affect the public market.
Markets like this do exist in real life in the form of 
``Dark Pools''\cite{degryse2008shedding}  
which offers anonymous and private investors to
trade away from the public exachanges.

Our market starts with $N$ stocks, each has a fixed amount of shares dedicated for this market.
In general, each trader has a fixed amount of money in the beginning, 
and can buy (or sell) stocks from (or to) the market according to the public prices.
% Each trader also has a high watermark and a low watermark of cash.
% When the cash value is less than the low watermark (or greater than the high watermark), 
% the trader will keep selling (or buying) stocks in order to increase (or decrease) the cash value. 
% This strategy comes from real-world strategies where people want to control the risk.
% Different watermarks lead to different behaviors, and for this example, 
% the traders speculate to be either conservative or aggressive.
Listing \ref{lst:trader} shows the trader's agent program. 
Each trader trades several rounds until his cash value reaches the goal. 
For each round, the trader randomly picks a stock {\tt X}, buys it as much as possible, 
and waits for the price of {\tt X} to change. 
$\alpha$ is a value between 0 and 1 which indicates the profit/loss margin percentage. 
For example, if a trader buys {\tt X} at a price of {\tt P}, 
he sells it at a price of either higher than $(1+\alpha)\times{\tt P}$ (to profit)
or lower than $(1-\alpha)\times{\tt P}$ (to prevent further loss). 
Different $\alpha$ values lead to different trading behaviors, and for this example,
the traders speculate to be either 
conservative (smaller $\alpha$) or aggressive (larger $\alpha$).
Note that there is no atomicity guarantee between checking the price of a stock 
and actually buying it, which is also the behavior of a real market.
Also an intention to buy can be partially filled by the stock-serving agent
(see Listing \ref{lst:stock}).
The stock-serving agent also waits for price updating (i.e. {\tt (update, ...)})
from the public exchange in a realtime fashion. 

\begin{figure}[tb]
\begin{lstlisting}[label=lst:trader,caption={Trader's Agent. Numbers are irrelevant and just for illustration purposes.}]
Me     = unique_account_name
Stocks = [...]  // a list of stock names
Start  = 5000   // start with cash of $5000
Goal   = 5500   // the goal is to end up with $5500

trade(Cash, |$\alpha$|):
  X = Stocks[rand(N-1)]

  |$\Rd$|(price, X, _) = (_, _, P1)
  Q1 = |$\lfloor$|Cash / P1|$\rfloor$|
  |$\Out$|(buy, X, Q1, Me)
  |$\In$|(ack, Me, _, _) = (_, _, P2, Q2)
  C1 = Cash - P2 * Q2

  |$\Rd$|(price, X, |$\lambda x.x\ge(1+\alpha)\times{\tt P2}\lor x\le(1-\alpha)\times{\tt P2}$|)
  |$\Out$|(sell, X, Q2, Me)
  |$\In$|(ack, Me, _) = (_, _, P3)
  C2 = C1 + P3 * Q2

  if C2 < Goal
    trade(C2, |$\alpha$|)
  else
    |$\cm$|

trade(Start, 0.01) |$\oplus$| trade(Start, 0.05)
|$\exit$|
\end{lstlisting}
\begin{lstlisting}[label=lst:stock,caption=Stock-serving Agent. \texttt{P} is the current price of the stock while \texttt{Q} is the quantity of stocks available for sale.]
Me = unique_stock_name

serve(P, Q):
  case |$\In$|(_, Me, _, _)
    (update, _, NewPrc, _): |$\In$|(price, Me, _)
                            |$\Out$|(price, Me, NewPrc)
                            serve(NewPrc, Q)

    (buy, _, Q1, X): Q2 = min(Q1, Q)
                     |$\Out$|(ack, X, P, Q2)
                     serve(P, Q - Q2)
    
    (sell, _, Q2, X): |$\Out$|(ack, X, P)
                      serve(P, Q + Q2)

|$\Out$|(price, Me, Start_Price)
serve(Start_Price, Start_Quantity)
\end{lstlisting}
\vspace*{-6mm}
\end{figure}
