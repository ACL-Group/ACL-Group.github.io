\section{System Implementation}\label{sec:impl}

We have implemented a proof-of-concept system using Erlang,
to show the viability of 
the speculative nondeterminism in a practical setting. 
Both the system and the agent programs are written in Erlang, 
and communicate by using global registered names and message passing. 
Every node in the tree of worlds is a process
\footnote{In this section, {\em process} 
means the ultra-lightweight processes in the Erlang VM
while {\em native processes/threads} means OS processes/threads} 
in the Erlang VM. 
Different instances of the same agent are also different processes. 
Agent instances talk to the worlds they live in directly by message passing. 
The system also assumes a tuple space data model and uses data
operations similar to Linda, such as $\In/\Out/\Rd$.
Tuples in the tuple space are simply Erlang tuples, 
while the tuple space is a list of tuples along with additional indexing structures. 
$\In/\Rd$ conditions are represented as Erlang anonymous functions
(i.e. {\tt fun(...) -> ... end}). 
We also defined a set of macros in Erlang to help building conditions. 
For example, {\tt ?Any} is defined as {\tt fun(\_) -> true end}, 
{\tt ?LessThan(X)} is defined as {\tt fun(V) -> V < X end}, etc.

Our system benefits from Erlang in two aspects:
    (i) Processes in Erlang are not native OS processes/threads, 
        but {\em ultra-lightweight processes}
        with extremely fast creation and tiny memory footprint. 
        There are large amount of world splitting and pruning operations in runtime.
        By reducing the overhead caused by the system, agent programs can have 
        more CPU time and thus more opportunities to capture their needs in realtime. 
        Tiny memory footprint allows us to run millions of processes at the same time 
        even on a single machine, which provides the possibility of running large scale applications. 
    (ii) Erlang processes share no memory, and the communication model among them is asynchronous 
        {\em message passing}. Erlang guarantees causality and eventually delivery,
        provided that the recipient exists, and provides process linking and monitoring functionality
        such as {\tt erlang:link/1} and {\tt erlang:monitor/2}. 
        These guarantees and facilities help the system being robust and well-structured and thus
        provide possibilities of customizability and extensibility. 

Next we describe a few optimizations that make the implementation more practical.

\subsection*{Lazy Forking}

Lazy forking is a \emph{fork-on-need} mechanism which enables 
\emph{asynchronous coordination} of the agents.
For example, initially program $P$ lives in world $w$. 
Then program $Q$ joins and creates two choices in world $w$, 
thus conceptually splitting $w$ into two descendant worlds $w_1$ and $w_2$,
and $Q$ itself forks into two processes.
In a na\"ive implementation, $P$ has to fork into two processes as well,
each corresponds to a program instance in one of the new worlds.
By lazy forking, $P$ remains one process at this point, 
and delays the forking until it starts interacting with the store.
This mechanism helps control the excessive forking and the number of
running processes in the system.

\subsection*{Exponential Backoff of Splitting}

In some cases, a large number of agents issue the choice construct 
in the same world simultaneously. Na\"ively splitting this world 
in this case doesn't do any real work but can easily get stuck in 
the exponentiality. 
From the perspective of an agent instance $a$, the world $w$ it lives in 
is stored locally and should be a leaf in the tree of worlds. 
However, because of the lazy forking mechanism described above,
that world $w$ may be already split into two descendant worlds $w_1$ and $w_2$,. 
When $a$ wants to issue the choice construct in this case, 
the system uses exponential backoff to delay the choice creation,
and later allows $a$ to split in the two descendant worlds of $w$.
The delay gets exponential when $w_1$ (or $w_2$) is also split 
at the time $a$ is allowed to split it. 
There is also a limit of the delay so agents can have reasonable responsiveness. 
This mechanism injects nondeterminism into the scheduler so that programs that are just
about to commit can commit first and start pruning the worlds.

\subsection*{Condition Triggering}

Both $\In$ and $\Rd$ in the tuple space data model are blocking on the condition. 
When there's an update to a world's tuple space, instead of waking up all the agents 
in the world to check the condition, the system only wakes up part of them by using triggering. 
In our prototype system, the triggering is enabled by \emph{indexing} 
the fields in the tuples, and is available for tuple operations $\In$ and $\Rd$.
% The system also provides variants of $\In$ and $\Rd$ 
% to allow more flexible control of tuple spaces. 
% Parameters for $\In$ and $\Rd$ operations are: 
% non-blocking, time-out and block until a condition.

\subsection*{Tree Partitioning}

Besides the optimizations for reducing overhead, we also provide a 
{\em tree partitioning} mechanism in order to reduce exponentiality. 
When there are large number of agents, not all of them are interested in 
the same kind of resources. For $n$ agents with no interleaving interests 
and each with 2 choices, na\"ively there should be $2^n$ worlds. 
However, since their interests are non-interleaving, we only 
need $2n$ worlds corresponding to their $2n$ choices in total. 
The tree partitioning mechanism works in such a way that 
each agent starts with its own world, instead of joining the leaf worlds 
of a common tree. As long as the agent doesn't have interest interleaving 
with other agents at this point of time, it splits and prunes on its own tree. 
In this way, agents with non-interleaving interests are separated 
and partitioned into different trees. 
When at some point of time, an agent $a_i$ starts to be interested in some 
resources which some other agent $a_j$ is also interested in, 
the system merges the trees of $a_i$ and $a_j$ together. 
Suppose the trees of $a_i$ and $a_j$ have $p$ and $q$ leaf worlds respectively,
the resulting tree after merging them together will have exactly $pq$ leaf worlds 
to represent the combination of choices. 

In order to concretely illustrate the use of tree partitioning, 
we reuse the FR example described in Section \ref{sec:examples}. 
Suppose, in Figure \ref{fig:flight}, a traveler $T_1$ wants to 
buy either $a\to b$ or $a\to c\to b$, while another traveler $T_2$
wants to buy either $d\to e$ or $d\to b\to e$. 
Then $R_1=\{ab,ac,cb\}$ are resources that $T_1$ is interested in, 
and $R_2\{de,db,be\}$ are resources that $T_2$ is interested in. 
For $T_1$ and $T_2$ there is no need to ``multiply'' their choices 
together. Also suppose the agents with interleaving interests in $R_1$ 
and $R_2$ create a tree of $p$ and $q$ leaf worlds respectively. 
By using tree partitioning, we only get $p+q$ instead of $pq$ worlds. 

Intuitively, tickets between different cities should be separated in this way. 
However, in Listing \ref{lst:tsell}, the ticket seller's agent touches 
every ticket, so it's considered by the system to be interested in 
both $ab\in R_1$ and $de\in R_2$. In this case, two trees containing 
$T_1$ and $T_2$ will be merged together. 
In order to benefit from tree partitioning, the ticket seller 
should be moved out of the system and creates a short-lived actor 
for each of the tickets (see Listing \ref{lst:tsell-ex}).
There's no need to modify the ticket buyer's agent program. 

\begin{figure}[tb]
\begin{lstlisting}[label=lst:tsell-ex,caption=Ticket Seller's Program Making Use of Tree Partitioning. {\tt spawn\_process()} creates a short-lived actor for interacting with the system of virtual worlds. {\tt \{...\}} denotes a separate program code instead of evaluation.]
Ts = ...  // tickets as in Listing |\ref{lst:tsell}|
loop
  T = Ts[rand(6)]
  if rand(1) = 0
    spawn_process{|$\Out$|(ticket, T[0].T[1], T[2])
                  |$\exit$|}
  else
    spawn_process{|$\Out$|(ticket, T[1].T[0], T[2])
                  |$\exit$|}
\end{lstlisting}
\end{figure}

