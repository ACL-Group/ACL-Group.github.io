\section{Related Work}
\label{sec:related}
%Our proposed model is related to two aspects: geographical
%topic modeling and sentiment analysis. In this section,
%we first introduce several existing work on these two aspects,
%and then review some existing studies on point-of-interest(POI)
%recommendation.

\subsection{Geographical Topic Modeling}
\label{sec:GTM}
%Our proposed model is related to the studies on
%geographical topic modeling and we proceed to review them.
Based on the traditional topic modeling techniques such
as Latent Dirichlet Allocation (LDA) \cite{BleiLDA:2003}
and Probabilistic Latent Semantic Analysis (PLSA),
some recent studies
\cite{Eisenstein:2010,Geofolk:2010,Hong:2012,YuanW4:2013}
incorporate geographical information in topic models.
Benefit from the geographic information, these models
can discover regional topics, which are shown
to be useful in POI recommendation.

Some of these studies
\cite{MeiST:2006, Eisenstein:2010, Geofolk:2010, Sakaki:2010, Yin:2011}
focus on analyzing
the relation between locations and words/topics without considering
users. Mei et al. \cite{MeiST:2006} sample each word in their model
conditioned on the time, location (id only) and background words.
Eisenstein et al. \cite{Eisenstein:2010} consider the coordinates
of locations and use the Gaussian distribution to generate
coordinates for locations from latent regions. GeoFolk
\cite{Geofolk:2010} generates latitude and longitude
of a location from two Gaussian distributions determined by
the topic of the location's document.
%Thus, near locations correlates to the
%same topic which makes the model fails to capture locations
%that are far away from each other but have the same topic.
Yin et al. \cite{Yin:2011} propose a PLSA model with two latent
variables, region and topic, in two levels.
Regions are generated in the document level, which is shared
by all of the documents. Topics are generated in the word level.
Each latent region is modeled as a Gaussian distribution
in their model. Different from these proposals, our work focuses
on modeling user preferences and takes users and users' sentiment
into account.

There also exist proposals \cite{Hong:2012,YuanW4:2013,Yin:2014TKDD} that model user
preferences to geographical topics based on geo-tagged tweets.
Hong et al. \cite{Hong:2012} propose a model to analyze the
geographical topics in geotagged tweets.
In this model, latent regions are modeled as Gaussian distributions
and each region contains a topic distribution.
%For each geotagged tweet, this model first generates a latent region
%based on the region distribution over all users and the
%preference of the tweet author over region. Then a topic is generated
%dependent on topic distribution, the region and the author. The
%words of a tweet are generated both from region and topic.
Yuan et al. \cite{YuanW4:2013}
explore the temporal information together with the
location, topic, and user information to model the time-aware
personalized topic region.
Yin et al. \cite{Yin:2014:TCM} consider that the user behaviors 
depend on both the user's topic preferences and temporal topic
distributions. They incorporate category
and locality preference into consideration to make
further improvement on modelling user profiles \cite{Yin:2014TKDD}.
Our work is
different from these proposals in that we focus on user reviews
rather than tweets, and we not only explore the topic-regions
but also analyze the topical-aspects and
corresponding sentiments in the reviews.
%By analyzing the opinions
%expressed in a user's reviews, our model can discover
%which aspects in a category is really favoured by the user.

\subsection{Sentiment Analysis}
%Our work is also related to the work on
%sentiment aspect joint modeling.
Our work is also related to the work on sentiment
aspect joint modeling.
According to different granularities of sentiment,
we divide the existing studies into two types:
sentence level, and phrase level.

Sentence level sentiment analysis supposes each
sentence expresses one aspect of the product.
%For example, a review of a digital device
%may be related to aspects like battery, screen, price, etc.
Titov and McDonald \cite{TitovMGLDA:2008}
present a sentence level model, namely MG-LDA, to extract
aspects from reviews.
Based on MG-LDA, Titov and McDonald further
propose a method \cite{TitovMAS:2008} which jointly
models the aspect and rating. The model is then used to predict
aspect ratings from a review. This model can extract aspect level
sentiment/rating, but needs aspect rating as
observable variable, which is often not available.
Jo and Oh \cite{JoASUM:2011}
model aspect and sentiment jointly in an LDA-based framework.
Both aspect and sentiment are modeled as latent variables in the model, and each
word has a joint distribution over topics and sentiment polarities. To identify
the polarities, they keep a list of seed words for each sentiment polarity,
and give higher probability to generate a
seed word from its corresponding polarity.

Studies on phrase level sentiment
\cite{MeiTSM:2007,WangLRR:2010, Lu:2011:ACC,MoghaddamILDA:2011,Zhang:2014:URR}
use NLP tools to
analyze the dependency between the words in a sentence
and extract aspect-opinion phrase pairs, e.g., $<screen, bright>$,
with some predefined patterns.
The aspect-opinion phrase pairs are then used for
further analysis on the sentiment polarities on the aspects.
Mei et al. \cite{MeiTSM:2007} build a PLSA
model in which a topic is modeled as a linear mixture
of multinomials from neutral topics and two sentiments (positive
and negative). 
%This model is in document level, which means
%it can only discover sentiment for the whole document.
Wang et al. \cite{WangLRR:2010} build a regression model to capture the aspect
ratings from the overall ratings. 
%Each word has different sentiment
%polarities on each aspect and 
The overall ratings are
modeled as the weighted sum of the sentiments from all aspects. However, aspects
are fixed and extracted using a list of seed words and a boot-strapping
algorithm.
Moghaddam and Ester
\cite{MoghaddamILDA:2011} propose an interdependent LDA model in
which the aspects of a product is modeled and the corresponding ratings
to each aspect is drawn depending on the aspect. However, the ratings
are treated as 5 clusters and the model cannot tell which cluster
has the rating of 1 and which has the rating of 5.

%Mei et al. \cite{MeiTSM:2007} build a PLSA
%model in which a topic is modeled as a linear mixture
%of multinomials from neutral topics and two sentiments (positive
%and negative). This model is in document level, which means
%it can only discover sentiment for the whole document.
%Titov and McDonald \cite{TitovMGLDA:2008}
%present a sentence level model, namely MG-LDA, to extract
%aspects from reviews.
%Titov and McDonald further
%propose a method \cite{TitovMAS:2008} based on MG-LDA which jointly
%models the aspect and rating. The model is then used to predict
%aspect ratings from a review. This model can extract aspect level
%sentiment/rating, but needs aspect rating as
%observable variable, which is often not available. Wang et al.
%\cite{WangLRR:2010} build a regression model to capture the aspect
%ratings from the overall ratings. Each word has different sentiment
%polarities on each aspect and the overall ratings are
%obtained from the weighted sum of all aspect sentiments. However, aspects
%are fixed and extracted using a list of seed words and a boot-strapping
%algorithm.
%Moghaddam and Ester
%\cite{MoghaddamILDA:2011} propose an interdependent LDA model in
%which the aspects of a product is modeled and the corresponding ratings
%to each aspect is drawn depending on the aspect. However, the ratings
%are treated as 5 clusters and the model cannot tell which cluster
%has the rating of 1 and which has the rating of 5.
%Jo and Oh \cite{JoASUM:2011}
%model aspect and sentiment jointly in an LDA-based framework.
%Both aspect and sentiment are modeled as latent variables in the model, and each
%word has a joint distribution over topic and sentiment. To identify
%the polarities, they keep a list of seed words for each sentiment polarity,
%and give higher probability to generate a
%seed word from its corresponding polarity.

Our model differs from these models in that
we jointly consider region, aspect and sentiment in a unified model.

\subsection{POI recommendation}
%POI recommendation is a popular research topic in recent years.
%Different from traditional product recommendation, POI has two
%additional features: user check-ins and coordinates.
%Most of the
%existing works are based on the assumption that the more a user
%check-in in a POI, the more probable the user likes it.
%The coordinate of the POI is then used as a feature to rank down
%POIs that far away from the user's activity area.
We divide the existing approaches into
three categories: memory-based collaborative filtering,
matrix factorization and topic models. Since topic models are mentioned
in \secref{sec:GTM}, we focus on memory-based collaborative filtering and
matrix factorization in this section.
%Since most geographical topic models
%discussed in \secref{sec:GTM} can
%be applied to POI recommendation tasks, we focus on collaborative
%filtering techniques in this section.

Several proposals recommend POIs based on collaborative filtering (CF)
\cite{YeGeoSocial:2011,Levandoski:2012,YuanPOI:2013}.
Ye et al. \cite{YeGeoSocial:2011} propose a fusion framework to
combine user-based, friend-based and geo-based collaborative
filtering. In the geographic model, the probability of transporting
from one POI to another is drawn from a power law distribution over
the distances between the two POIs. Levandoski et al. \cite{Levandoski:2012}
use an item-based CF for POI recommendation, but they mainly
focus on how to make the memory based method efficient on a large dataset.
%A user is modeled as a set
%of his historically visited POIs. Then the probability of a user
%visits a POI is given by considering the distance between the given
%POI and the POIs in the user's visited POI set.
Yuan et al.
\cite{YuanPOI:2013} propose the problem of recommending POIs
for a user specified time, and incorporate the temporal factor 
into the user-based CF model for recommendation.  Yuan et al.
\cite{Yuan:2014:GBPB} also propose a graph-based approach for time-aware
POI recommendation which integrates geographical and temporal
influences.
%a time-aware model with
%time, coordinates and user check-ins. In this work, check-ins
%are divided into different groups by different time segments to
%model user interests by time.

In the proposals based on matrix factorization,
Liu et al. \cite{LiuBin:13} and Cheng et al. \cite{ChengYKL12}
propose latent factor models
by incorporating the geographical information using
Gaussian distribution.
Yang et al. \cite{YangSenti:2013}
propose a sentiment-enhanced personalized location recommendation
system using probabilistic matrix factorization.
Very recently, Zhang et al. \cite{ZhangYF14} propose an explicit factor model
which takes aspect and sentiment into account.
However, these proposals
do not consider the geographical information.

%The system combines both check-ins and
%sentiment extracted by a dictionary from user
%tips to construct a fusion user preference matrix,
%and takes user friendship and location similarity into account.

In summary, no existing work models aspect, sentiment, spatial
information and category at the same time. And no existing work is
able to discover the latent relation between these variables.
%which is
%proved to be helpful in applications like POI recommendation.
