\section{Related Works}
There has been some works performed to improve the search 
results by improving the query input by user, by providing 
a list of related queries or by directly expanding or 
substituting the original query. A work in 
\cite{Baeza-yates04queryrecommendation} showed the use of 
query logs to provide user with a list of recommendation of 
related queries. In this work, a query was represented as 
a vector of terms found in the document (URL) clicked by user. 
The queries were then clustered using the similarity formula 
defined by the authors. Recommendation queries were extracted 
by finding the appropriate cluster for user input query and 
then ranking the queries using the combination of similarity 
value and fraction of clicked documents for scoring.

There is also a work in the area of query rewriting (substitution) 
as in \cite{Jones06generatingquery}. The work was focused on the 
domain of sponsored search and targeted to find the ads that match 
user query. Their data set consisted of user session (query log) 
that recorded the query refinement made by user in each search 
session. Using this information, the authors derived the 
log-likelihood ratio and produce pairs of interchangeable queries or 
phrases. The machine learning techniques were also used to detect 
the important features. Other similar works in the domain of sponsored 
search was in \cite{Radlinski08optimizingrelevance}, which focused 
on increasing revenues gained from ads. The authors utilized the 
pseudo-relevance feedback method, to collect query substitution 
candidates from bid phrases that were placed by advertiser.

A work in using concepts to retrieve information was performed in 
\cite{semsearch-10}. The authors tried to extend the current simple 
keyword search with the concepts found in user query. The concepts 
were identified by parsing the query, taking the noun phrases found, 
and then matched the noun phrases with the meaning in Wordnet 
\cite{Miller95wordnet:a}. Modified inverted index was then built, 
using concepts as their index instead of simple keyword. Another work 
was in \cite{Bendersky:2008}, in which the author tried to improve 
the retrieval performance for verbose query input. The authors worked 
on discovering key concepts in verbose queries by extracting the noun 
phrases. The works showed the importance of concepts which provided 
better retrieval performance by weighting and adding the concepts 
into the original query.

We defined our work as a query rewriting or query substitution task. 
We differ in which we focused on the task on helping the user to 
transform their vague query that contains concepts, into a list of 
queries that are more specific. The query input by user will be 
replaced by more specific queries in our query logs that contains 
instances of concepts found in the user query. In addition, we used 
Probase \cite{website:Probase} as our knowledge base, by utilizing 
the taxonomy of concepts and instances extracted from millions of 
documents found in web.
