\section{Related Work}
\subsection{News recommendation}
Many news recommendation systems \cite{wu_neural_2019-1, wu2019npa} are based on long history of user. Some of them use the knowledge graph of entities in the news title as affiliated information \cite{wang2018dkn, wang_ripplenet:_2018}, others excavate the quality of news articles \cite{lu_quality_2019} or the backtracking behavior as the user feedback \cite{smadja_understanding_2019}. In this case, they formulate the recommendation task as conventional recommendation tasks, and recommend articles for users based on their collaborative click history \cite{zhu2019dan}. Some work use well-designed deep neural network to combine the power of factorization machines, such as Wide\& Deep \cite{cheng2016wide} and deepFM \cite{guodeepfm2017}, where the relation of items and users are well exploited. Unfortunately, in real-time scenarios, new articles and anonymous users emerge, causing severe cold start problem. Then if we want to capture users' short term preference and recommend immediately after their several interactions, this kind of approaches with static user-item matrix is not suitable. Some propose incremental matrix factorization algorithm based on classic MF algorithm by adding a new user or a new article to the matrices with a random initialization \cite{al2018adaptive}, and others apply meta-learning which aims to train a model that can rapidly adapt to a new task with a few examples \cite{lee_melu:_2019}. 
\subsection{Session-based recommendation}
Many online recommender systems are proposed to deal with the session-based scenarios \cite{epure_recommending_2017, zhou_variational_2019}, where the user interaction information is limited and items are increasing generated. Some traditional content-based or hybrid recommendation approaches are transferred to session-based recommendation \cite{sottocornola2018session}.  They combine session-based CF similarity and content-based similarity to recommend top-K similar articles. Some combine a variant of MF adapted to the incremental nature of data streams and a topic drift modeling module \cite{al2018adaptive}. As for deep learning methods, Recurrent Neural Network (RNN), Long Short-Term Memory (LSTM) and Graph Neural Network (GNN) possess properties that make them attractive for sequence modeling of user sessions \cite{guo_streaming_2019, hidasi2015session, wang2019modeling, moreira_news_2018, wu2019session}. Further, a hybrid encoder with an attention mechanism is introduced to model sequential behavior of users \cite{li2017neural, liu2018stamp, xu2019time, song_islf_2019, zhang_feature-level_2019}. 
Besides, many sequential recommendation systems \cite{pereira2019online, xu2019graph} on music listening, e-commerce purchasing and games playing construct assorted RNN-related architectures (e.g, Recurrent Convolutional Neural Network (RCNN) \cite{xu_recurrent_2019}, Gated Recurrent Unit (GRU) \cite{hidasi2018recurrent}, Hierarchical Gating Network (HGN) \cite{xiao2019hierarchical, ma2019hierarchical}), showing RNN's high capacity to modeling user shift preference. On the other hand, some reproduce the result of these recent algorithmic proposals, showing that many of them do not consistently outperform a well-tuned non-neural linear ranking method \cite{dacrema_are_2019, ludewig_performance_2019}.

Although sequential modeling naturally take preference shifting into account, the inter-session and intra-session temporal information is neglected. When sampling negative articles, adaptive negative sampling method based on Generative Adversarial Nets (GAN) is proposed \cite{wang_neural_2018}. Beyond that, few work pays attention to the effective sampling strategy when building negative samples. Randomly sampling from such continuously increasing and high-volume news articles might be fast but won't be effective enough.

\subsection{Time aware sequential recommendation}
Sequence and Time Aware Neighborhood (STAN) \cite{garg2019sequence} takes vanilla SKNN as its special case. They build static time decay functions for three factors: position of an item in the current session, recency of a past session w.r.t. to the current session, and position of a recommendable item in a neighboring session. This approach can be regarded as rule-based SKNN, with exponential decay function, and the experiment result on e-commerce
website even outperforms some deep-learning based approaches. However, the decay function of news articles is fixed, which may undermine the ability to model user's short-term preference towards different articles.

A time-interval-based GRU is proposed to model user session-level representations \cite{lei_tissa_2019}. Some work \cite{rakkappan2019context, xu2019time, wu_recommender_2019} treat the time feature of interactions as a temporal context, and some also pay attention to item's lifetime \cite{wang_modeling_2019}, while they fail to consider the temporal gap between different sessions. Clearly, a session in close proximity to the target session weights more than a session from long time ago.

