\section{Introduction}
\label{sec:intro}
%{\bf Outline:}
%\begin{verbatim}
%{\bf Motivation of the problem:}
%(1) Too many choices: how to select my favorite
%show to watch at real time (on the fly)?

With the rapid development of digital television service
in China, viewers are spoilt with choices.
Over 30,000 programs are available each week
on 112 cable channels.
However, this massive variety can be a mixed blessing:
viewers often have a hard time deciding which program
to watch. In the worst case, one might spend more time
flipping through the channels than watching anything
meaningful, or even miss the programs that they really
want to watch.
What the viewers really desire is a real-time, personalized
recommender that makes realtime suggestions of
what programs to watch, or not to watch,
based on {\em who we are} and {\em what we like}.

%(2) Users demand personalized services
%Many users, especially youths, are in pursuit of
%personality today. Thery are eager to show their
%differences with others. In such situation, An
%behavior-oriented recommendation system can not
%only save users' time to select programs, but
%also provide users personalized services. It can
%help users to make decision in other area, or
%send information to users which they may be
%interested in.

%{\bf Challenges of the problem: }
%(1) Large data volume.

However, determining who we are and what we like,
and ultimately providing good TV recommendations poses
significant challenges.

Firstly, user profile or preference information is
often non-existent, incomplete or changing over time given
the common cable TV subscription cases in China and the
rest of the world.

Secondly, television viewing patterns are complicated. Unlike
video sharing sites such as YouTube, which typically
interacts with a single user, and records user sessions,
televisions are usually not viewed by an
individual person only but by a group of people such as
families or friends. And these people with different tastes
can watch the same television at different times of the day.
The profile information of
one user (e.g. the TV service subscriber) is inadequate
for determining the preferences of the whole group.
And there's no obvious user sessions to help
distinguishing the different viewers.

Thirdly, in contrast to products and services offered by e-commerce
or video sharing sites, whose supply can be considered ``infinite''
and ever ready, TV shows are ``time-consumable'' resources which
are often available at specific times. Once aired, they are probably
never available again, at least at times controlled by the user.
Consequently the time factor is important in the TV recommender system.
In addition, viewers often have their preferred TV viewing times.
This means, even if a user may like a show, she probably will not
watch it unless it is shown at the right time.

Finally, web-based recommender systems can often take
advantage of user feedbacks including ratings and
comments. Such explicit feedback is almost certainly
not available to a TV recommender system. The only
input that a TV recommender can leverage is the viewing histories
and behaviors, which is what we are targeting in this
paper.

%based on the complexity of TV programs and
%user behavior, there are some problems left that
%needed to be solved. Our system will handle large
%amount of data such as users viewing behaviour,
%TV schedule and TV info. Processing such data
%really cost time and resource.
%TV shows are "time-consumable" resources, once
%aired, probably never aired again, at least not
%controlled by the viewers. This is different from
%recommendation systems in some online video website
%like YouTube. For online video website, resources
%are stored on the server and will be available in
%quite a period. Users can pick up a video to watch
%whenever they like. So this kind of recommendation
%systems can just take users's preference in account.
%In contrast, TV recommendation system should also
%involve the time factor, which requires our system
%to recommend programs to specified user at
%specified time.
%View patterns are more complex than other video
%sharing sites. Also take YouTube as an example.
%When users start to browse this website, their
%viewing histories will be stored via session or
%cookie. Even if different users use the same
%machine to browse, their behaviours can be divided
%in different session or cookie. But set-top box can
%not finish such distinguishing. So when different
%users use the same TV set, their viewing histories
%will merge together and make it difficult to analyze
%the patterns.
%TV schedule info is very limited. In mainland China,
%there is no standard TV program infomations available.
%Our system have to obtain such informations itself.
%%(5) Users have no or little profile info, no
%%direct feedback after watching.
%Our system is not based on predefined user profile
%but form users' model via users' behaviours. Also,
%we evaluate our recommendation result based on
%users' behaviour. So no direct feedback is needed.

%{\bf Our solution: }
To address the above challenges,
we propose PredicTV, a new TV recommender system based on
user viewing behaviors. The system tracks viewing
behaviors by recording the time and the channel whenever
the user switches to a new channel on an Internet-ready
digital TV set-top box. Such channel switching information,
also known as the {\em viewing history}, is
then streamed to the remote recommender server in pseudo-real time.
We say pseudo-real time because viewing histories may be batched
and sent periodically (e.g. hourly) in order to save network bandwidth and
reduce load on the server.
The server automatically analyzes
the programs being watched as well as the viewing patterns,
and recommends future programs which best match the inferred
preferences of the viewers associated with the viewing
history. Because the system works dynamically, it
adapts to the user tastes as it changes over time. The
details of the system architecture will be presented in
Section \ref{sec:archi}.

Our key technique (detailed in Section \ref{sec:algo})
to solve the TV recommendation problem
is  to discover as much information about
the TV programs as possible from the Web and transform each watched
program into data points in a multiple dimensional space, where
each dimension represents a certain attribute of a TV program, such
as producer, director, cast, air time, language, etc. The values in
these dimensions collectively form the {\em semantics} of a program.
From viewer's past viewed programs and their respective viewing durations,
the system can dynamically learn weights of these attributes. These weights,
together with a time-decaying agglomerate of all programs watched in
the past form a real-time user preference model. We make recommendations
to a user by selecting programs that are most similar to the current
preference model. For example,
if a viewer has viewed two movies of Jackie Chan and a drama involving
police and gangsters recently,
then the system knows the viewer cares about the actor and genre attributes,
and specifically he is interested in ``Jackie Chan'' being the actor and
``action'' being the genre.

Next we will first present an overview of the system architecture
in Section \ref{sec:archi} and discuss the details of the two most
important components of our system in Section \ref{sec:algo},
namely the information extraction
module and the user model and recommendation module, and finally present
some evaluation results in Section \ref{sec:eval}.

%For example, if the viewer watches a lot of
%spy related movies or dramas set in the 1940's,
%the recommender should suggest a similar show if it's
%on air tonight. On the other hand, if the viewer only watches
%TV in the afternoon, then this show shouldn't be
%recommended even if the content matches her usual
%taste.
%
%%(3) Digital TV set-top box provides a mechanism
%%to track viewing behaviors
%
%Facing the problems above, we point out a new method
%to form user model via user behavior and do
%recommendation. The detailed algorithm process will
%be discussed in section
%\ref{sec:algo},
%here we just give a brief introduction.
%We use a background program to obtain
%TV schedule and gather info about each program on the
%internet. Then convert the info of each program
%(including air time) into a multi-dimensional
%vector. Users' preferences are also represented using
%vector. Their viewing behaviours will
%be used to update user vector. Gradually, this
%vector will match users' preference better and
%better.
%At runtime, our system takes as input the
%streams of user viewing behavior, identifies the
%programs watched by each user, and dynamically
%updates their viewing preference vectors with
%the program being watched at the same time and damps
%the effects of shows watched earlier at the same time.
%When doing recommendation, we match the program
%vector against the users summary vector and get
%the result. We also design some methods to evaluate
%the accuracy of our recommendation result. This will
%be covered in section
%\ref{sec:eval}.
