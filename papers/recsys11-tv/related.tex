\section{Related Work}
\label{sec:relate}
Much work has been done in developing an TV recommendation
sytem.
Srinivas Gutta has introduced a TV content recommendation
system
\cite{Gutta00:ContentRecommender}, which mainly uses
bayesian classifier and decision tree to analyze users'
preference. SeungGwan Lee has also introduced a program
recommendation system
\cite{Lee10:RecommendunderCloud}. It is an on-demand
system and its services is deployed on cloud. When
designing our sytem,
we test and find that some classical machine-learning
method like decision tree, content-base and collaborative
method are too slow to support large amount of users.
So we try to use vector in our recommendation algorithm.
Claire Laudy has discussed information fusion for TV program
information
\cite{Laudy08:InfoFusion}. But in our system, we cannot get
standard TV information directly. So we add an information
extraction module in our system. It fetches information then
construct it as a vector. Zhiwen Yu has discussed a TV
recommendation for multiple viewers
\cite{Zhiwen06:RecommendMultiViewers}.
It use some algorithms to merge users' profile and get a
general profile. This profile is used in recommendation. We
don't use user profile in our system. Instead, users'viewing
pattens are represented as the
sum of all the program vector they have viewed. Jeroen Van
Barneneld pays attention to the interface of TV recommender
system and has discussed how to design a usable interface
for recommender system
\cite{Barneveld04:Interface}.
He mainly divides interface into several parts and designs
several proposals for each part. Then he takes a survey
to ask users which proposal they like most. This interface
is used in PC end while in our system, the recommendation
result will be directly sent to the set-top box and display
on the TV screen. How to evaluate a recommendation
system is also an interesting topic. Kazuki Ikawa has
introduced an evaluation method for TV programs
recommendation with viewer's log data
\cite{Ikawa10:Evaluate}.
Since we haven't deployed our system, we cannot get similar
log data. That's why we use survey to evaluate our system.

Recomendation system can not only be
used to recommend TV program, but also other area.
Sudha Velusamy has discussed an ad recommendation system
for TV programs
\cite{Velusamy08:AdRecommend}.
Several algorithms are used in that system
like content-based, collaborative and cluster method. He
also emphasizes that the time ad plays should fits the
TV schedule time. {\`O}scar Celma has introduced a music
recommendation system
\cite{CelmaRH05:MusicRecommend}.
Their system requires two input: user's preference and music
information which come from FOAF and RSS. Badrul M. Sarwar
has discussed different kinds of recommmendation algorithms
used in E-commerce
\cite{Sarwar00:ECommerceRecommend}. We can see that
recommendation system has wide application prospect.
Regard to our system, the vector algorithm can not only
used to recommend TV program, but also ad, merchandise,
which makes our system flexible.


Item-based collaborative filtering recommendation algorithm
\cite{SarwarKKR01:itembased}
provides dramatically better performance than user-based algorithms.
It calculates the similarities according to all users' behaviors. 
Then the method computes the prediction on an item $i$ for a user $u$ 
by computing the sum of the ratings given by the user on the items similar to $i$.

