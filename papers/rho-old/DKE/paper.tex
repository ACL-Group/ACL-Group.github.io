%%
%% Copyright 2007, 2008, 2009 Elsevier Ltd
%%
%% This file is part of the 'Elsarticle Bundle'.
%% ---------------------------------------------
%%
%% It may be distributed under the conditions of the LaTeX Project Public
%% License, either version 1.2 of this license or (at your option) any
%% later version.  The latest version of this license is in
%%    http://www.latex-project.org/lppl.txt
%% and version 1.2 or later is part of all distributions of LaTeX
%% version 1999/12/01 or later.
%%
%% The list of all files belonging to the 'Elsarticle Bundle' is
%% given in the file `manifest.txt'.
%%

%% Template article for Elsevier's document class `elsarticle'
%% with numbered style bibliographic references
%% SP 2008/03/01
%%
%%
%%
%% $Id: elsarticle-template-num.tex 4 2009-10-24 08:22:58Z rishi $
%%
%%
\documentclass[preprint,12pt]{elsarticle}

%% Use the option review to obtain double line spacing
%% \documentclass[preprint,review,12pt]{elsarticle}

%% Use the options 1p,twocolumn; 3p; 3p,twocolumn; 5p; or 5p,twocolumn
%% for a journal layout:
%% \documentclass[final,1p,times]{elsarticle}
%% \documentclass[final,1p,times,twocolumn]{elsarticle}
%% \documentclass[final,3p,times]{elsarticle}
%% \documentclass[final,3p,times,twocolumn]{elsarticle}
%% \documentclass[final,5p,times]{elsarticle}
%% \documentclass[final,5p,times,twocolumn]{elsarticle}

%% if you use PostScript figures in your article
%% use the graphics package for simple commands
%% \usepackage{graphics}
%% or use the graphicx package for more complicated commands
%% \usepackage{graphicx}
%% or use the epsfig package if you prefer to use the old commands
%% \usepackage{epsfig}

%% The amssymb package provides various useful mathematical symbols
\usepackage{amssymb}
%\usepackage{times,amsmath,amsfonts}
\usepackage{float}
\usepackage{graphicx}
\usepackage{enumerate}
\usepackage{subfigure}
\usepackage{url}
\usepackage{multirow}
\usepackage{makecell}
\usepackage{algcompatible}
\usepackage{eqparbox}
\usepackage{epsfig}
\usepackage{epstopdf}
\usepackage{mathrsfs}
\usepackage{amsmath}
\usepackage{balance}
\usepackage{algcompatible}
%% The amsthm package provides extended theorem environments
\usepackage{amsthm}

%% The lineno packages adds line numbers. Start line numbering with
%% \begin{linenumbers}, end it with \end{linenumbers}. Or switch it on
%% for the whole article with \linenumbers after \end{frontmatter}.
%% \usepackage{lineno}

%% natbib.sty is loaded by default. However, natbib options can be
%% provided with \biboptions{...} command. Following options are
%% valid:

%%   round  -  round parentheses are used (default)
%%   square -  square brackets are used   [option]
%%   curly  -  curly braces are used      {option}
%%   angle  -  angle brackets are used    <option>
%%   semicolon  -  multiple citations separated by semi-colon
%%   colon  - same as semicolon, an earlier confusion
%%   comma  -  separated by comma
%%   numbers-  selects numerical citations
%%   super  -  numerical citations as superscripts
%%   sort   -  sorts multiple citations according to order in ref. list
%%   sort&compress   -  like sort, but also compresses numerical citations
%%   compress - compresses without sorting
%%
%% \biboptions{comma,round}

% \biboptions{}

\renewcommand\algorithmiccomment[1]{%
  \hfill\#\ \eqparbox{COMMENT}{#1}%
}
\newcommand\LONGCOMMENT[1]{%
  \hfill\#\ \begin{minipage}[t]{\eqboxwidth{COMMENT}}#1\strut\end{minipage}%
}

\clubpenalty=10000
\widowpenalty=10000
\makeatletter
\newif\if@restonecol
\makeatother
\let\algorithm\relax
\let\endalgorithm\relax
\usepackage[linesnumbered,vlined,ruled]{algorithm2e}

\usepackage{xspace}
\newcommand{\qid}{\emph{qid}\xspace}
\newcommand{\qids}{\emph{qid}s\xspace}

\newcommand{\SA}{\mathcal{SA}}

\newcommand{\argmax}{\mathop{\mathrm{argmax}}}
\newcommand{\pss}{\sigma^2_{\eta_c^i}}
\newcommand{\qss}{\sigma^2_{\eta_c^j}}
\newcommand{\pww}{w^2_i}
\newcommand{\qww}{w^2_j}
\newcommand{\container}{\mathcal{C}}
\newcommand{\csize}{\kappa}
\newcommand{\linked}{\mathcal{L}}
\newcommand{\enum}{\mathcal{E}}
\newcommand{\safe}{\mathcal{S}}
\newcommand{\policy}{\mathcal{P}}
\newcommand{\natnum}{\mathbb{N}}
\newcommand{\breach}{P_\text{breach}}
\newcommand{\wrt}{w.r.t.}
\newcommand{\TRUE}{\textbf{true} }
\newcommand{\FALSE}{\textbf{false} }
\newcommand{\AND}{\textbf{and} }
\newcommand{\NOT}{\textbf{not} }
\newcommand{\RETURN}{\textbf{return} }
\newcommand{\reference}{\textbf{ref}~}
\newcommand{\shrink}{\vspace*{-3pt}}

\newcommand{\PartialSuppressor}{\textsc{PartialSuppressor} }
\newcommand{\SplitData}{\textsc{DNCSplitData} }
\newcommand{\SanitizeBuffer}{\textsc{SanitizeBuffer} }
\newcommand{\HandleLongRecord}{\textsc{HandleLong} }
\newcommand{\HandleShortRecords}{\textsc{HandleShort} }
\newcommand{\MakeRed}{\textcolor{red}}

\newcommand{\Enum}{\textsc{EnumQids} }
\newcommand{\EnumLong}{\textsc{EnumerateLong} }
\newcommand{\ClearBuffer}{\textsc{ClearBuffer} }
\newcommand{\UpdateBuffer}{\textsc{UpdateBuffer} }
\newcommand{\FindAffected}{\textsc{FindAffected} }
\newcommand{\SuppressionPolicy}{\textsc{SuppressionPolicy} }

\newcommand{\bmax}{b_\text{max}}
\newcommand{\lmin}{l_\text{min}}
\newcommand{\tmax}{t_\text{max}}
\newcommand{\dnum}{\sigma}

\newcommand{\PartialR}{\textsc{Partial(r)} }
\newcommand{\PartialL}{\textsc{Partial(l)} }
\newcommand{\PartialALL}{\textsc{Partial(all)} }

\newtheorem{theorem}{Theorem}
%\newproof{proof}{Proof}
\newtheorem{lemma}{Lemma}
\newtheorem{proposition}{Proposition}
\newtheorem{corollary}{Corollary}
\newtheorem{definition}{Definition}
\newtheorem{property}{Property}
\newtheorem{axiom}{Axiom}
\newtheorem{example}{Example}
\newcommand{\etal}{{\em et al.}\xspace}

\usepackage[usenames]{color}
\newcommand{\XH}[1]{\textcolor{blue}{[Xinhui: #1]}}
\newcommand{\KZ}[1]{\textcolor{Green}{[KZ: #1]}}

\journal{Data \& Knowledge Engineering}

\begin{document}

\begin{frontmatter}

%% Title, authors and addresses

%% use the tnoteref command within \title for footnotes;
%% use the tnotetext command for the associated footnote;
%% use the fnref command within \author or \address for footnotes;
%% use the fntext command for the associated footnote;
%% use the corref command within \author for corresponding author footnotes;
%% use the cortext command for the associated footnote;
%% use the ead command for the email address,
%% and the form \ead[url] for the home page:
%%
%% \title{Title\tnoteref{label1}}
%% \tnotetext[label1]{}
%% \author{Name\corref{cor1}\fnref{label2}}
%% \ead{email address}
%% \ead[url]{home page}
%% \fntext[label2]{}
%% \cortext[cor1]{}
%% \address{Address\fnref{label3}}
%% \fntext[label3]{}

\title{Partial Suppression of Set-Valued Data}

%% use optional labels to link authors explicitly to addresses:
%% \author[label1,label2]{<author name>}
%% \address[label1]{<address>}
%% \address[label2]{<address>}

\author{Xinhui ~Xu}
\ead{xuxinhui08@gmail.com}
\author{Xiao ~Jia}
\ead{xjia@cs.sjtu.edu.cn}
\author{Chao ~Pan}
\ead{pc660@sjtu.edu.cn}
\author{Kenny Q. ~Zhu}
\ead{kzhu@cs.sjtu.edu.cn}
\address{
Department of Computer Science \& Engineering, Shanghai Jiao Tong University,
800 Dongchuan Road, Shanghai 200240, China}

\address{}

\begin{abstract}
%% Text of abstract
Privacy-preserving set-valued data publishing is an important yet challenging
problem. In this paper, we present a novel framework for set-valued data
anonymization by partial suppression which ensures no strong inference of
sensitive information is possible regardless of the amount of background
knowledge the attacker possesses, and can be adapted to both space-time and
quality-time trade-offs as a ``pay-as-you-go'' approach.
%
While minimizing the number of item deletions, the framework attempts to
either preserve the original data distribution or retain mineable useful
association rules with limited spurious rules invented in the anonymized
data, without particular assumptions of the downstream utility of the data.
%
We also devise two novel heuristics to drive the partial suppression process
in two different directions. The first heuristic seeks to preserve the
original data distribution by minimizing the {\em Kullback-Leibler
divergence} between the original and anonymized data sets. The second
heuristic seeks to reduce the number of spurious rules when the anonymized
data is used for rule mining. Our extensive experiments show that partial
suppression significantly outperforms generalization and global suppression,
two of the existing popular anonymization techniques, in maximizing the data
utility with reasonable cost in time.
%space and time.
\end{abstract}

\begin{keyword}
%% keywords here, in the form: keyword \sep keyword

%% MSC codes here, in the form: \MSC code \sep code
%% or \MSC[2008] code \sep code (2000 is the default)

Data Anoymization \sep Partial Suppression \sep Global Suppression \sep
Generalization \sep Set-valued Data \sep Information Loss
\end{keyword}

\end{frontmatter}

%%
%% Start line numbering here if you want
%%
% \linenumbers

%% main text
%\section{}
%\label{}

\section{Introduction}

Protein$-$protein interactions (PPIs) are of central importance for the majority of biological functions, such as signal transduction, metabolic pathways, molecular dynamics, and protein networks\cite{Hoffmann.Krallinger.ea:2005}, for they serve as the most fundamental building blocks of the entire interacademic systems of any organisms. Collecting data on pairwise interaction relationships is essential for multiple purpose, including identification of modules with certain functionality\cite{Spirin.Mirny.03}, mapping diseases to dominated genes\cite{Ideker.Sharan.08}, and after all, understanding wholistic metabolic/genetic networks from a system biology perspective.

A lot of databases have been built to store protein and genetic interactions from major model organism species and are available in various standardized formats, such as MINT\cite{Zanzoni.Montecchi-Palazzi.ea:2002}, BIND\cite{Bader.ea:2003}, BIOGRID\cite{DBLP:journals/nar/StarkBRBBT06}, etc. Among those mainstream databases, the data largely rely on voluntary reports by scientists or researchers, besides, comprehensive curation efforts become indispensable for the sake of accuracy. However, the amount of biology-related literatures with respect to protein interactions grows explosively and thus make it either impossible or impractical to manually detect PPI information anymore.

Considering huge amount of PPI information with great wealth hidden in published papers, in recent years, numerous mining techniques have been proposed that aim to extract PPI information automatically from free text, especially machine learning, information retrieval, and natural language processing\cite{DBLP:journals/bib/WinnenburgWPDS08}.These approaches can be roughly categorized into three classes: co$-$occurrence, rule$-$based, and machine learning. 

Co$-$occurrence is the approach with most simplicity and naivete. Just as its name implies, this method intends to find out pairs of proteins that co-occur in the same context. The scope of "same context" ranges from phrase, sentence, paragraph to whole abstract, even document. The underlying assumption is that whenever two proteins are mentioned together by authors, chances are high that there is some kind of relationship between them. However, however, in-context closeness even semantic relation does not necessarily represent actual biological interaction. As a consequence, a large fraction of candidate pairs are mismatched inevitably, causing a high recall but low precision.

The second approach is rule-based extraction, in other words, pattern matching. There are many types of rules, most of them concern natural language processing (NLP). One way is to specify hand-crafted regular expressions before hand, which mostly lean on language usage preference. Besides, by using full or partial (shallow) parsing strategies, more information would be acquired, such as part-of-speech taggers, local dependencies between syntactic components, context-free grammar\cite{DBLP:journals/bioinformatics/TemkinG03}, and full sentence structure. Compared to co$-$occurrence, rule-based approach enjoy better precision but much lower recall. In addition, since the rules are usually derived from training data, that is to say, the improper choice of training data would be significantly lethal, therefore quality of extraction is invariably instable and may not applicable to other data.

The third and most commonly used approach use machine learning techniques, in this case, the task to extract protein$-$protein interactions turns out to be a binary classification problem. Each protein pairs are represented along with a set of features, which is associated with their context, then a well$-$defined classifier gives the answer whether the candidate protein pairs is classified to be qualified PPI. (TO BE FURTHER FILLED!!!)

In this paper, we introduce a general bootstrapping framework for Protein$-$protein interaction extraction from natural text.Our method differs from most of the previous works in three aspects:

(1)The extraction process is driven by only tiny fraction of training data, which are regarded as seed data. In each round, it would derive reliable patterns automatically from seed data, then extract more positive PPI pairs consequently, what's more, the seed data would be augmented by the newly extracted results with high confidence.

(2)multiple graph kernel. 

(3)various evaluation.




\section{Problem Definition}
\label{sec:prob}

In this section, the problem of partial suppression is formally described,
and before which, some preliminary mathematical concepts are defined as
follows.

%\subsection{Notations and Definitions}
%A {\em multiset} $S$ is a set which allows repetitive elements, while the
%{\em power multiset} $\mathbb{N}^S$ is the set of all subsets of the multiset
%$S$.
%Examples or formal definitions of these are given below.
%\begin{description}
%  \item[Multiset] $[a,a,b]$ is the same as $(\{a,b\},\{(a,2),(b,1)\})$
%  \item[Power set] $2^S$ is the power set of the set $S$
%  \item[Power multiset] $\mathbb{N}^S$ is the power multiset of the set $S$
%\end{description}

A {\em multiset} is a set which allows repetitive elements, while a {\em
power set} is the set of all subsets of a set. Examples or formal definitions
of these are given below.
\begin{description}
  \item[Multiset] \hspace{2em} $[a,a,b]$ is the same as
      $(\{a,b\},\{(a,2),(b,1)\})$
  \item[Power set]  \hspace{2em}  $2^S$ is the power set of the set $S$
  \item[Power multiset] \hspace{4em} $\mathbb{N}^S$ is the power multiset
      of the set $S$
\end{description}

\begin{table}[th]
\centering
\caption{Notations for Problem Definition}
\label{table:problem_notations}
\begin{tabular}{m{0.28\columnwidth}|m{0.6\columnwidth}}
  \hline
  \textbf{Symbol} & \textbf{Definition} \\
  \hline
  $D = D_S \cup D_N$ & the domain, which is the set of all possible items \\ \hline
  $D_S$ & the sensitive domain \\ \hline
  $D_N$ & the non-sensitive domain \\ \hline
  $T\in\natnum^{2^D}$ & a set-valued table, which is a multiset of $m$ transaction records \\ \hline
  $T[i]\in T$ & the $i$-th record of $T$ \\ \hline
  $R\in T$ & a transaction record %, which is a set of items drawn from $D$
  \\ \hline
  $|R|$ & the number of items contained in $R$ \\ \hline
  $I$ & an itemset, which is a subset of $D$\\ \hline
  $q$ & a \emph{quasi-identifier} (also \qid), which is a set of items (\textbf{sensitive items also allowed}) taken from any record in table $T$ \\ \hline
  $\enum(R) = 2^R - \{\emptyset\}$ & the \qid enumeration of $R$, which is the power set of $R$ except the $\emptyset$ \\ \hline
  $\displaystyle Q(T)=\bigcup_{R\in T} \enum(R)$ & the set of all \qids in table $T$ \\ \hline
  $\rho$ & the strong inference/association rule threshold \\ \hline
  $\mathcal{A}(q,e)$ & an inference/association rule $q\rightarrow e$\\  \hline
  $\SA(q,e)$ & a sensitive association rule $\mathcal{A}(q,e)$  if $e$ contains at least one sensitive item\\  \hline
  $sup_{T}(I)$ & the support of $I$ is the number of transactions $t\in T$ such that $I\subset t$\\ \hline
  $conf_{T}(q,e)$& the confidence of inference $\mathcal{A}(q,e)$ is $\frac{sup_T(q\cup e)}{sup_T(q)}$\\ \hline
\end{tabular}
\end{table}
%A set-valued table $T$ is a multiset of transaction records,
%  each record $R \in T$ is a set of items drawn from domain $D$.
%  $D$ is the union of two non-intersecting set, sensitive domain $D_S$ and non-sensitive domain $D_N$.
%We follow the step of \cite{Sweeney2002:k-anonymity} and
%  extend the definition \emph{quasi-identifier} ($qid$) in relational database for set-valued data.
%Then we give a series of other definitions related with $qid$.
%As simple as you can imagine, a $qid$ is just a set of items taken from $D$.
%$Q$ is a set of $qid$s, $\Omega(R_i)$ is the $qid$ enumeration of $R$ which is the power set of $R$ except the $\emptyset$.
%The column count $cc$ of row $R$ is the number of items contained in $R$.
Table \ref{table:problem_notations} lists the detailed notations used in the
rest of this paper. Next we define a number of important notations before
presenting the problem definition.

\begin{definition}[Container]
The \emph{container} of an itemset $I$ in table $T$ is defined as \[
\container_T(I) = \{ i \in \natnum : I \subseteq T[i], 1 \leq i \leq |T| \}
.\]
\end{definition}

\begin{definition}[Linked Items]
All sensitive items linked by a \qid $q$ in table $T$ is defined as \[
\linked_T(q) = \{ e \in D_S-q : sup_{T}(q\cup\{e\}) > 0 \} .\]
\end{definition}
 According to the definition, $sup_{T}(I)$=$|\container_T(I)|$.
 Also we will use $\container(I)$, $\linked(q)$, $sup(I)$,
$conf(q,e)$ to represent $\container_T(I)$, $\linked_T(q)$, $sup_{T}(I)$ and
$conf_{T}(q,e)$ respectively when $T$ is the only table within discussion.

\subsection{Privacy Model}
We reuse the $\rho$-uncertainty privacy model
proposed by Cao \etal \cite{Cao:2010:rho} which requires that the confidence of any sensitive
association rules is not above $\rho$. Next we formalize the
privacy requirement.
%Let $T$ be a set-valued table and the domain $D$ is
%divided into two subsets: the sensitive domain $D_S$ and the non-sensitive
%domain $D_N$. We assume that an attack will know any \qids $q$ such that
%$q\subset R, where~ R \in T$. The dataset is safe if and only if the attack
%will not infer any items , with a high probability (e,g,$\geq \rho$),
% $e$ from $D_S$ such that if $q \subset R$ then $e\in R$. Such an inference can be defined
%as a sensitive inference or sensitive association rule if $e \in D_S$,
% Our objective is to prevent the attack from mining any sensitive association rules.

%The model defined is immune to
%{\em
%record/attribute linkage attack} \cite{FungWCY10:Survey}
%However, we  prove that our technique promises a strong privacy result immune from  {\em Minimalitiy
%attack} \cite{Wong:2007:Minimality} and {\em Composition attack}
%\cite{Ganta:2008:Composition}
%To reach our objective, we have to keep the $conf(q,e)$ lower than $\rho$
%for any $\mathcal{A}(q,e)$. Therefore, we introduce the following definitions.
\begin{definition}
\label{def:safety_rule} A sensitive association rule $\SA(q,e)$ is safe
\wrt~$\rho$ if and only if $conf(q,e)\leq\rho$ .
\end{definition}

The $\rho$-uncertainty privacy model requires that all sensitive association
rules are safe.

\begin{definition}[Breach Probability]
\label{def:probability} The \emph{breach probability} of a \qid $q$ is \[
\breach(q) = \max_{e\in\linked(q)} conf(q,e) \]
%where $P(e|q) = P(q\rightarrow
%e) = \csize(q\cup \{e\})/\csize(q)$, and $P(e|q) = conf(e,q)$.
\end{definition}

\begin{definition}%[Safety of qid]
\label{def:safety_qid}
A \qid $q$ is safe \wrt~$\rho$ if and only if $\breach(q)\leq\rho$.
\end{definition}

\begin{definition}%[Safety of Table]
\label{def:safety_table}
A table $T$ is safe \wrt~$\rho$ if and only if $q$ is safe \wrt~$\rho$ for any \qid $q\in Q(T)$.
\end{definition}

As proved by Cao \etal \cite{Cao:2010:rho}, by ensuring confidences of all
sensitive association rules with one consequent below $\rho$,
confidences of all sensitive association rules with more consequents are also
made below $\rho$. As a result, Definition \ref{def:safety_table} can satisfy
the $\rho$-uncertainty privacy model.

\begin{definition}[Suppressor]
\label{def:suppressor}
A function $S : \natnum^{2^D}\times[0,1]\rightarrow\natnum^{2^D}$ is a \emph{suppressor} if and only if $S(T,\rho)$ is safe \wrt~$\rho$ for any table $T$.
\end{definition}
%\begin{property}
%\label{prop:container_size} For two\qids $p$ and $q$, if $p\subseteq q$ then
%$\container(q)\subseteq\container(p)$ and $\csize(q)\leq \csize(p)$.
%\end{property}
%
%\begin{proof}
%  $\forall i\in\container(q), q\subseteq T[i]\Rightarrow p\subseteq q\subseteq T[i] \Rightarrow i\in\container(p)$,
%  so $\container(q)\subseteq\container(p)$. Hence, $\csize(q)\leq \csize(p)$.
%\end{proof}
%
%Property \ref{prop:container_size} guarantees that $P(e|q)$ in Definition
%\ref{def:probability}, we can only consider sensitive association rules with
%a singleton consequent then all

\subsection{Information Loss}
\begin{definition}[Information Loss]
\label{def:infoloss}
Let $T^\prime = S(T)$ be the suppression result of table $T$ by the suppressor $S$.
The information loss of an item $e$ is defined as
\[ IL(e) = sup_{T}(\{e\}) - sup_{T^\prime}(\{e\}), \]
and the information loss of a suppression from $T$ to $T^\prime$ is defined as
\[ IL(T,T^\prime) = \frac{\sum_{e\in D}IL(e)}{\sum_{e\in D}sup_{T}(\{e\})}. \]
\end{definition}

Information loss of $T$ is essentially the number of items deleted in $T$
divided by total number of items in $T$. Our aim is to find a suppressor
defined in Definition \ref{def:suppressor} which reduces information loss as
much as possible. Information
loss is caused by suppression of items, and the best way to consider it
depends on the specific downstream utility \cite{Xu:2008:ATD}. Xu \etal
\cite{Xu:2008:ATD} requires the data publisher assigning a certain
information loss function to the global suppression of an item $e$, denoted
$IL(e)$.
%\textcolor{red}{ \emph{Classification Metric}
%\cite{Iyengar:2002:TDS} is better for anonymizing data for classifier
%training. \emph{Normalized Certainty Penalty} \cite{Xu:2006:UAU} is proposed
%for anonymization using generalization. Remove them?}
Cao \etal
\cite{Cao:2010:rho} introduces a metric for computing the information loss
caused by generalization.
We adopt the information loss metric used in \cite{Xu:2008:ATD}
and revised it in order to be consistent with the metric introduced in \cite{Cao:2010:rho}.
%\PC {We have to mention that actually this metric of information loss makes sense, since the value of it is exactly the value
%calculated by the avgloss\cite{Cao:2010:rho} under the condition that only suppression is executed}
%Because most of the existing metrics of information loss are proposed
%for anonymization using generalization,

To make a summary,
we use three metrics to measure the effectiveness of our partial suppression
algorithm and peers. The first one is the \emph{information loss}. The second
one is the \emph{symmetric relative entropy} which measures the change in
data distribution. The third one is the \emph{number of rules mined}
(including original and spurious rules) from the anonymized data. The first
measure is more general and used by other work. The last two metrics target
the utility of the anonymized data for statistical analysis and rule mining,
and they will be introduced in Section \ref{sec:eval}.

%\textcolor{red}{ We'll consider these three metrics in our
%heuristic solution later. }
%\begin{definition}[Optimal Suppression Problem]
%\label{def:osp}
%The optimal suppression problem is to find an optimal suppressor $S_\text{OPT}$ for a given table $T$ such that
%\[ IL(T,S_\text{OPT}(T))\leq IL(T,S(T)) \] for any suppressor $S$.
%\end{definition}
%\begin{definition}[Minimum suppression]
%Minimum occurrence suppression of item type $t$, to make confidence of
%inference $\mathcal{A}(q,e)$ below $\rho$:
% \hspace{4mm}
%\[MS(t,\mathcal{A}(q,e))=
%\begin{cases}
%sup(q\cap \{e\})-sup(q)\rho & t=e  \\
%\frac{sup(q\cap \{e\})-sup(q)\rho}{1-\rho} & t\in q \\
% \infty & otherwise
%\end{cases} \]
%\end{definition}
%\PC {
%\begin{definition}[Remaining probability of item $i$]
%The Remaining probability of item $i$ is defined as
%\[ remain(i)=\frac{\kappa_{T^\prime}(\{i\})}{\kappa_T(\{i\})} \]
%where $T$ is the original
%dataset and $T^\prime$ is the current dataset processed by suppression but
%not finished
%\end{definition}
%}

\subsection{The Problem}
The problem is to find a {\em Partial Suppressor} which anonymizes the input
set-valued table $T$ to satisfy the $\rho$-uncertainty privacy model and
minimizes item deletions.
%preserves the original data distribution or retains mineable useful
%association rules with limited spurious rules invented, and also minimizes
%item deletions.

\begin{definition}[Optimal Partial Suppressor]
The \emph{optimal partial suppressor} $S_\text{OPT}$ is the suppressor such that
\[ IL(T,S_\text{OPT}(T))\leq IL(T,S(T)) \] for any suppressor $S$.
\end{definition}

% \begin{definition}[Optimal Partial Suppressor for Distribution]
% \label{def:distribution}
% \[ Dist_{distance}(S_\text{OPT}(T), T) \leq  Dist_{distance}(S(T), T)\]
% while
% \[ IL(T,S_\text{OPT}(T))\leq IL(T,S(T)) \] for any suppressor.
% \end{definition}
% \begin{definition}[Optimal Partial Suppressor for Mining]
% \[ Spurious(S_\text{OPT}(T)) \leq  Spurious(S(T)) \] and
% \[OriginalRule(S_\text{OPT}(T)) \geq OriginalRule(S(T))\]
% while
% \[ IL(T,S_\text{OPT}(T))\leq IL(T,S(T)) \] for any suppressor.
% \end{definition}

 \section{Partial Suppression Algorithm}
\label{sec:algo}

\renewcommand{\algorithmicforall}{\textbf{for each}}
%\algnotext{ENDFOR}
%\algnotext{ENDIF}
%\algnotext{ENDWHILE}
%Main algo and its variants. Give pseudo code and description.
%Use pics whenever possible. Discuss the ``knob''.

%First we come to the question why we opt for partial suppression.
%We are inspired by the global suppression proposed in $\rho$-uncertainty \cite{Cao:2010:rho}.
%Apparently, we had a immediate impression that it would suppress
%many items including both non-sensitive and sensitive ones.
%Although privacy model has been satisfied, but the data is highly
%cut sparse. \cite{Cao:2010:rho} also used global generalization method to anonymize transactions.
%As we can see, data is generalized, rules mined from the generalized data also become vague.
%It not only creates disambiguated rules but also influences the practical usage of the mined rules.
%In the last part of \cite{Cao:2010:rho}, authors also addressed problems of generalization method,
%the non-monotonicity property.
%Since both algorithms in \cite{Cao:2010:rho} rely on global suppression,
%the authors clarified that both algorithms are meant for datasets which may include large number of transactions,
%but the number of distinct items per transaction is relatively small.

%Partial suppression in our mind is the one that can suppress as few items as possible to
%satisfy the criteria of a safe set-valued table (see Definition \ref{def:safety_table}).
%If we are required to suppress only one item we won't suppress more.
%This is the intuition of partial suppression.
%Partial suppression also draws some doubts in
%\cite{Cao:2010:rho, tkde:VerykiosEBSD04:ARH,tkde:WuCC07:hiding} mainly for
% its side-effects in rule mining.
%But no one really takes action to challenge the
%side-effects in anonymization of transaction datasets.
%So we are the first to try and we believe its not that bad
%as you see and also the partial suppressed result is practical in usage.
%In experimental section we will compare results of our
%partial suppression algorithm with results of both algorithms in $\rho$-uncerntainty \cite{Cao:2010:rho},
%we will show that partial suppression method
%is much better than you think and will be acceptable in practical usage.


%%%
% TODO comments
%\XH{do we need to declare that, later when we say update $linked(q)$
%we also mean to update $\csize(q \cup \{e\})$ for
%each sensitive item $e \in \linked(q)$ implicitly?
%}
%%%

The Optimal Suppression Problem defined in Section \ref{sec:prob} is
a combinatorial optimization problem. To find the optimal suppressor, it needs
to check all combinations of items suppressed, which yields $O(2^N)$
complexity and $N$ is the total number of items in $T$. Instead we present
the partial suppression algorithm as a heuristic solution to Optimal
Suppression Problem.
%By definition of a safe set-valued table (Definition
%\ref{def:safety_table}), the breach probability of each \qid in $Q$ must be
%under the threshold $\rho$ (Definition \ref{def:safety_qid}).
%For
%example, given a \qid $q=\{a, b, \alpha\}$ and all its linked sensitive items
%$\linked(q)=\{\beta\}$, where $\{a, b\}$ are non-sensitive items and
%$\{\alpha, \beta\}$ are sensitive items, if $\csize(q)=5$, $\csize(q \cup
%\{\beta\})=3$ and $\rho=\frac{1}{3}$, then $\breach(q)=\frac{3}{5}>\rho$.
%\begin{definition}[Data Structures]
%We define three key data structures in this framework.
%$B$ is a \qid buffer which is a set of \qids.
%$K$ is a mapping which is essentially a materialized function from \qid $q$ to $\csize(q)$.
%$L$ is a mapping from \qid $q$ to $\linked(q)$.
%\end{definition}
Before we present the details of our algorithm, we define some notations for
simplifying the description.
\begin{definition}[Data Structures]
We define three key data structures in this framework. $B$ is a \qid buffer
which stores a set of \qids. $S$ is a \textbf{mapping} which is essentially a
materialized function from \qid $q$ to $sup(q)$. $L$ is a \textbf{mapping}
from \qid $q$ to $\linked(q)$.
\end{definition}
In the following algorithms, we will also use $sup(\cdot)$ and
$\linked(\cdot)$ to denote the computations of these two functions, and use
$S(\cdot)$ and $L(\cdot)$ to denote the access of the elements of the two
data structures.
\begin{definition}[Minimum suppression]
\label{minimum}
Minimum suppression of item $t$ to make strong sensitive association rule
$\SA(q,e)$ safe, i.e. $conf(q,e)\leq\rho$:
 \hspace{4mm}
\[MS(t,\SA(q,e))=
\begin{cases}
sup(q\cup \{e\})-sup(q)\rho & t=e  \\
\frac{sup(q\cup \{e\})-sup(q)\rho}{1-\rho} & t\in q \\
 \infty & otherwise
\end{cases} \]
\end{definition}
%\emph{Minimum suppression} is essentially the number of item $t$ suppressed
%to fix the inference $\mathcal{A}(q,e)$.
Specifically, it represents that we do not intend to suppress more items to
make $\SA(q,e)$ a safe one as it can make $conf(q,e)$ even smaller.
\begin{definition}[Residual of Item type $t$]
The residual of item $t$ is defined as
\[ residual(t)=\frac{sup_{T^\prime}(\{t\})}{sup_T(\{t\})} \]
where $T$ is the original data and $T^\prime$ is the internal suppressing
result.
% being\XH{??} anonymized data.
\end{definition}

%\XJ{move this para to algo sec} \MakeRed{
Our main idea is although the total number of ``dangerous'' association rules
maybe the worst case exponential in the original data, incremental
``invalidation'' of some of the rules through partial suppression of small
number of affected items can dramatically decrease the number of these bad
rules, which leads to quicker convergence to a solution.
%}
%%%
%
%In this paper, we adopts three kinds of partial suppression policies.
%The first one is \PartialR, which suppresses only sensitive items
%in the consequents. The second one is \PartialL, which
%suppresses only items in the antecedents.
%The third one is \PartialALL, which suppresses items in
%both the consequents and the antecedents.
%\PartialL and \PartialALL suppress both sensitive
%and non-sensitive items, whereas \PartialR suppresses only
%sensitive items.
Next we present the \textbf{general framework}, known as the \textbf{basic
algorithm}.

\subsection{The Basic Algorithm}
%
%Table \ref{table:problem_notations} and \ref{table:algo_notations}
%include notations for some global variables and parameters.
%They are defined in advance in order to omit the declarations in
%related functions.

\begin{algorithm}[th]
\caption{$\PartialSuppressor(T, D_S, \rho, \bmax)$}
\label{algo:partialsuppressor}
\begin{algorithmic}[1]
    \STATE Initialize $safe\leftarrow\TRUE$, $i\leftarrow 1$;
%    \STATE Initialize $safe\leftarrow\TRUE$
    \LOOP
        \STATE Initialize $sup$, $\linked$, $S$, $L$;
        \WHILE {$|B|<b_{max}$ \AND $i\leq |T|$} \label{algo:enu_s}
             \STATE Fill $B$ with \qids generated by $T[i]$, \label{algo:enumerate1}
             \STATE update $sup$, $\linked$, $S$, $L$; \label{algo:enumerate2}
             \STATE $i\leftarrow i+1$;
        \ENDWHILE \label{algo:enu_e}
		\STATE \textcolor{red}{update $sup$, $\linked$, $S$, $L$;}
%        \STATE Calculate $\rho$ of each $qid$ in $|B|$ \label{algo:update}
        \IF {$B$ contains an unsafe \qids}\label{line:containunsafe}
            \STATE $\SanitizeBuffer(T, B, sup, \linked, S, L)$;\label{line:sanitizebuffer}
            \STATE $safe\leftarrow\FALSE$;
        \ENDIF
        \IF {$i \ge |T|$ \AND $safe$}
            \STATE \textbf{break};\label{algo:partialbreak}
        \ELSIF {$i \ge |T|$}
                \STATE $i\leftarrow 1$;
                \STATE $safe\leftarrow\TRUE$;
%                \STATE \textbf{continue}
        \ENDIF
    \ENDLOOP
\end{algorithmic}
\end{algorithm}

\begin{algorithm}[th]
\caption{$\SanitizeBuffer(T, B, sup, \linked, S, L)$}
\label{algo:sanitize}
\begin{algorithmic}[1]
\STATE $\policy \leftarrow \SuppressionPolicy()$; \label{choose_heur}
\REPEAT
\label{algo:pick_rs}
    \STATE pick any $q,e$ such that $conf(q,e)>\rho$;
    \IF {$\policy = Distribution$}
        \STATE $d\gets\underset{d\in q\cup\{e\}}{\arg\max}\,H_{dist}(d)$;
        \label{algo:heur_dist}
    \ELSE
        \STATE $d\gets\underset{d\in q\cup\{e\}}{\arg\min}\,H_{mine}(d)$;
        \label{algo:heur_mine}
    \ENDIF
%    \IF {$\policy = Distribution$}
%        \STATE  find $\SA(q,e)$ with maximal $H_{dist}(d)$, where
%        \STATE  $d\in q\cup\{e\}$ \AND $conf(q,e)>\rho$
%        \label{algo:heur_dist}
%    \ELSE
%        \STATE find $\SA(q,e)$ with minimal $H_{mine}(d)$, where
%        \STATE  $d\in q\cup\{e\}$ \AND $conf(q,e)>\rho$
%        \label{algo:heur_mine}
%    \ENDIF
    \STATE $X\leftarrow q\cup\{e\}$;
    \STATE $k\leftarrow MS(d,\SA(q,e))$;\label{line:sanitize-k}
    \WHILE{$k>0$}\label{line:sanitize-whilek}
        \STATE $R\leftarrow$ a record containing $X$, i.e. $R\subseteq
        \container(X)$;
        \label{pick_row}
        \STATE $R\leftarrow R-\{d\}$;\label{line:sanitize-suppress}
        \STATE Update $sup$, $\linked$, $S$, $L$ of \qids contained in $R$;
        \label{algo:update_kl}
        \STATE $k \leftarrow k-1$;
    \ENDWHILE
\UNTIL{there is no unsafe \qid in $B$}   \label{algo:pick_re}
\end{algorithmic}
\end{algorithm}

To ensure a table is safe, we must make sure all \qids in $Q$ are safe.
\PartialSuppressor (Algorithm \ref{algo:partialsuppressor}) presents the
top-level algorithm. The partial suppressor iterates over the table $T$, and
for each record $T[i]$, the algorithm first generates  \qids from $T[i]$ and
sanitizes the unsafe ones. The suppressor terminates when the whole table is
scanned and there is no unsafe \qid.
%This termination condition ensures the algorithm achieves at
%least a local {\em optimum} in terms of minimizing the suppressions.

A \qid can be considered as a combination of different item types,
  and the number of distinct \qids is in exponential scale.
So one of the most time-consuming phases in partial suppression is
  the generation of distinct \qids.
An ideal solution to reduce the time cost in \qid generation is to enumerate
all distinct \qids in main memory. But this is impractical since the number
of distinct \qids can be prohibitive. We therefore introduce a \qid buffer of
capacity $\bmax$ to balance the space consumption with the generation time.
The value of $\bmax$ is significant. Small $\bmax$ values will cause
repetitive generation of \qids, while large $\bmax$ values will cause useless
generation of to-be non-existing \qids. Effects of different $\bmax$ values are
shown in Section \ref{sec:eval}.

As described in Algorithm \ref{algo:partialsuppressor}, starting from the
current record $i$, the algorithm repeatedly enumerates \qids in $T[i]$ to
fill the buffer and increments $i$ until $B$ is full (Lines
\ref{algo:enu_s}-\ref{algo:enu_e}).
% In algorithm \ref{algo:partialsuppressor}
%while buffer $B$ is not full, it recursively generates \qids from the record
%$T[i]$ (Lines \ref{algo:enu_s}-\ref{algo:enu_e}).
For each $q$ generated (Line \ref{algo:enumerate1}-\ref{algo:enumerate2}), if
it is already in $B$, increment $sup(q)$. Otherwise, insert $q$ into the
buffer and set $sup(q)$ to $1$. At the same time, update $\linked(q)$ with
the complementary sensitive set (only sensitive items) of $q$ in $T[i]$.
%When $B$ is full, line
%\ref{algo:update} in algorithm \ref{algo:partialsuppressor} is invoked to
%scan $T$ from index $i$ to $j$, and update $K(\cdot)$ and $L(\cdot)$.
Furthermore, in \qid generation we utilize a fixed size cache to remember \qids
whose linked items $\linked(\cdot)$ is empty, to prevent generating \qids that
are superset of them. It's a pruning technique we can leverage
 since if a short \qid does not link to sensitive items in $T$, neither do its supersets.
After that, if $B$ contains an unsafe \qid, we call \SanitizeBuffer to
suppress items from $T$ so that the unsafe \qid becomes safe.
\textcolor{red}{add something about the added statement in Algorithm 1}

\subsection{Buffer Sanitization}
\label{sec:sanitize} Each time \qid buffer $B$ is ready, \SanitizeBuffer
(Algorithm \ref{algo:sanitize}) is invoked to start processing \qids in $B$
and make all of them safe. We first partition \qids in $B$ into two groups,
{\em safe} and {\em unsafe} according to Definition \ref{def:probability} and
\ref{def:safety_qid}.
 Then in each iteration (Lines \ref{algo:pick_rs}-\ref{algo:pick_re}), \SanitizeBuffer
 picks the `best` (according to below heuristic functions) strong sensitive association
 rule %whose confidence is above $\rho$
 to sanitize (Lines \ref{algo:heur_dist} and \ref{algo:heur_mine}).

As we mentioned before we target to preserve the original data distribution
or retain mineable useful association rules with limited spurious rules
invented in the anonymized data, while minimizing the item deletions. By
noticing the reason why global suppression does not introduce spurious rules,
we introduce the following heuristic function to imitate global suppression
in order to avoid introducing spurious rules, while minimizing
item deletions.
\subsubsection{Spurious Rule Impediment}

%Then we introduce our second heuristic function which trys to retain minable
%rules with fewer spurious rules invented.
As we mentioned before, to learn from the characteristic of global
suppression - no spurious rules introduced, we devise a heuristic which takes
deletions towards global suppression while using partial suppression. We
define \[H_{mine}=residual(t)*MS(t, \SA(q,e))\] as the heuristic
function which prevents introducing spurious rules.
This product expresses
the strategy that we want to introduce fewer spurious rules by imitating the
effect of global suppression while suppressing as few items as possible.
We try to maintain the support
of some items and suppress some items as many as possible to
satisfy our privacy model. Notice that a spurious rule ($\mathcal{A}(q,e)$) is introduced when
 $sup(q)$ becomes small and if $sup(q)$ is small enough to ignore, $\mathcal{A}(q,e)$
will not be learned even $conf(q,e)$ larger than a certain threshold.
 Therefore, we are prone to
suppress those items which have been suppressed before to minimize the
support of spurious rules. As
you can see, each time we choose the sensitive association rule with lowest
$H_{mine}$ to perform sanitization (Line \ref{algo:heur_mine}).

However, $H_{mine}$ goes the opposite side of preserving the original data
distribution. Therefore, we introduce the following heuristic function
separately to preserve the original data distribution and minimizes the
deletion in local optimal.
\subsubsection{Minimum Distance}
%We first present the heuristic function which helps to preserve data
%distribution.
Consider a sensitive association rule $\SA(q,e)$, where $e$ is a
sensitive item and $conf(q,e) > \rho$.
%$q$ is not a safe \qid following the
$\SA(q,e)$ is not a safe rule following the
 Definition \ref{def:safety_rule}. To make $conf(q,e)$ not
above $\rho$,
%we must make the confidence
%of the inference not larger than $\rho$, thus
we should choose an item type $t\in q\cup \{e\}$ to suppress from
$\container(q\cup \{e\})$.
%$\mathcal{A}(q,e)$ and decide the minimum number of occurrences of $t$ to be
%deleted from  $\mathcal{A}(q,e)$ or just eliminate this inference from $T$.
Kullback-Leibler divergence is defined as
 \[H(Q||P)=\sum_{t\in
D}Q(t)log\frac{Q(t)}{P(t)}\]
where P(t) is the original distribution of $t$
and $Q(t)$ is the current
% (before selecting this removal)
 distribution of $t$, which is often used to characterize the distribution
 distance.
Noticing that by suppressing some of item $t$s when $Q(t)>P(t)$, it may make
the K-L divergence smaller, thus we define
\[H_{dist}=\frac{Q(t)log\frac{Q(t)}{P(t)}}{MS(t, \SA(q,e))}\]
 as the heuristic
function which helps to preserve data distribution. The ratio expresses the
strategy that picking an item type $t$ to suppress which maximally recovers
the data distribution and minimizes the deletion in local optimal.
 The numerator indicates the divergence of the data distribution on $t$.
 The larger the absolute value is,
the larger distribution difference of $t$ now is compared with the original
situation. However, if the value is less than 0, i.e. $Q(t)<P(t)$, we'd
better not suppress $t$, since it may further decreases $Q(t)$ and
deteriorates the data distribution. Therefore the larger the numerator is,
the item has more priorities to be chosen to suppress.
%Furthermore, if the value is positive, applying suppression will
%improve the data distribution. So the removal of $t$ is urgent and valuable
%when this value is large.
% However, if the numerator is less than 0, we better not suppress $i$ since
%$Q(i)$ has already been less than $P(i)$.
The denominator indicates the minimum suppression. The smaller the number is,
the fewer items are suppressed. Each time we choose sensitive association
with the highest $H_{dist}$ to perform sanitization (Line
\ref{algo:heur_dist}).


%
% By iterating over all unsafe \qids, we can all sensitive
%inferences whose confidence larger than $\rho$ from those {\em unsafe} \qids.
%Then in each iteration, the algorithm fetches one sensitive association
%$\SA(q,e)$ and fix it.

 Depending on whether the downstream utilities require to preserve
 data distribution or do mining tasks,
 \SuppressionPolicy (Line \ref{choose_heur}) in \SanitizeBuffer determines
 which heuristic function should be used.
% Next we will show the two heuristic
% functions that help to find the `best` association to fix.

After suppressing the item $d$, unsafe \qids may become safe, while safe ones
may become unsafe again, an undesirable situation known as {\em regression}.
Algorithm \SanitizeBuffer Line \ref{algo:update_kl} determines the set of
\qids that would be affected by suppressing $d$. This step is like the $qid$
generation step in Algorithm \PartialSuppressor Line \ref{algo:enumerate1}
and \ref{algo:enumerate2}.
 There are also different ways to pick a
record from $\container(q\cup \{e\})$ to suppress item $d$ (Line
\ref{pick_row}), currently for simplicity we just use the randomized way.

\subsection{Implementation Specifics}
\label{algo:impmentation}
The partial suppressor is a framework under which our algorithm
implementation relies on the following two approximation techniques to make
the performance practically acceptable.
%. Although
%the termination of the algorithm is proved in the analysis section, the time
%performance doesn't seem very good.
%Therefore, we introduce two approximation
%method to improve the time performance but maintain the quality of the
%result.

\subsubsection{Handle Long Records}
%We notice that the result has a property that the sensitive items don't exist
%in a long record in the final result. Such property indicates that the
%algorithm converges very quickly when it handles a long record. However, the
% \qids generated by long records are enormous and most of them are useless
%since sensitive items will be suppressed after handling only a small amount
%of those  \qids. Therefore we decide to handle long records separately. First,
%we consider only sensitive items in that record versus all sensitive items in
%$T$ as an optimization. Second, we only enumerate some  \qids, e.g,1000, and
%sanitize them. Third, when there are no sensitive items in this record, we do
%not consider this record anymore (We don't need to generate  \qids from this
%record) because those  long \qids are not expected frequently in the dataset
%which means that they hardly contribute to the support of association rules.
%However, if  the suppressor cannot eliminate all sensitive items by only generating
%\qids combined with all sensitive items in one long record, we should
%still generate \qids from that record.
A record $R$ has as many as $2^{|R|}$ \qids. Suppose $\bmax = 10^6$, then a
single record of length $20$ can fill up $B$. However, long \qids are not
expected frequently in the dataset, and thus can be handled separately. Since
we can't possibly enumerate all the \qids within a long record, we do this in
batches. Because the number of sensitive items are limited in any record,
sanitizing the first batch of \qids in a long record may already cause the
removal of all the sensitive items in that record. A record without sensitive
items doesn't contribute any unsafety to sensitive association rules, on the
contrary provides some safety , and hence no further actions are needed for this long
record.

There are three implementation issues in handling long records. First, each
time when {\em batch size of } \qids are generated, it will be costly to scan
the whole table to determine the related $sup(\cdot)$ and
$\linked(\cdot)$. Instead we use the $\container(\cdot)$ of each item in
\qid, and calculate the intersection of all the containers to determine the
$sup(\cdot)$ and $\linked(\cdot)$. This is much faster than the whole
table-scanning method when \qid is not a frequent itemset in $T$.

 Second, when computing
$\linked(q)$ in a long record, we consider only sensitive items in that
record versus all sensitive items in $T$ as an optimization. This is correct
because if there is a sensitive item $e$ in another record $R$ which is
linkable by $q$, update of $L(q)$ and $S(q)$ for $e$ can be done when we
handle $R$ later.

 Third, our algorithm provides a preprocessing option that
suppresses all sensitive items from records whose size is regarded large
enough.
%
%
\subsubsection{Heuristic Approximation}
In Algorithm \SanitizeBuffer Lines \ref{algo:heur_dist} and
\ref{algo:heur_mine}, in each iteration it is very time-consuming to iterate
over all strong sensitive association rules to find the `best` one with
maximal $H_{dist}$ or minimal $H_{mine}$ to sanitize, since the number of
\qids are enormous and each \qid can link to various sensitive items.
%For each $\SA(q,e)$, we have to calculate the highest
%$H_{dist}$ or the lowest $H_{mine}$.
While the iteration number are also propositional to the number of \qids, so
the cost is tremendous. Therefore, we use an approximation way which randomly
pick small amount of \qids in the buffer, e.g 1,000, and calculate the
maximal $H_{dist}$ or the minimal $H_{mine}$ among them to find the locally
`best` one to sanitize.
%We pick the candidate item by a local greedy algorithm.
%
%The two approximation strategies do improve the time performance dramatically
%but may deteriorate the result. However, we conduct experiments to justify
%that even by using such approximations, our result is much better than the
%previous work.\XH{need Kenny revise}

\subsection{Speedup By Divide-and-Conquer}
\label{subsec:speedup}
%\subsection{Optimization in Time By Divide-and-Conquer}
%\KZ{Let's not talking about experiments here. We haven't done any experiments yet
%in this section. Just say ``when execution time is of concern, we use the
%following divide-and-conquer framework to speed up the execution''. And
%avoid saying "deeply go".}
%Although our algorithm can terminate within an acceptable time,e,g,2 hours,for most datasets
% by using the two approximation strategies, it is still too long for practical issues.
The above \textbf{basic algorithm} incurs cost which relies on the average
transaction size, the table size, and the domain size. When data is very
large and execution time is of concern, we can use a divide-and-conquer (DnC)
framework that partitions the input data dynamically, runs \PartialSuppressor
on them individually and combines the results in the end. This approach is
correct in the sense that of each suppressed partition is safe, so is the
combined data (See Lemma \ref{CorrectnessOfPartitioning} in Section
\ref{sec:analysis}). This approach also gives rise to the parallel execution
on multi-core or distributed environments which provides further speed-up
(this will be shown in Section \ref{sec:eval}).

\begin{algorithm}
%\caption{Top level partitioning controller}
\caption{$\SplitData(T, D_S, \rho, \bmax, \tmax)$} \label{algo:splitdata}
\begin{algorithmic}[1]
    \IF { $Cost(T) > \tmax$ }
        \STATE Split $T$ equally into $T_1$ , $T_2$
        \STATE $\SplitData(T_1, D_S, \rho, \bmax, \tmax)$
        \STATE $\SplitData(T_2, D_S, \rho,  \bmax, \tmax)$
    \ELSE
        \STATE $\PartialSuppressor(T, D_S, \rho, \bmax)$
    \ENDIF
\end{algorithmic}
\end{algorithm}

The divide-and-conquer algorithm is shown in Algorithm \ref{algo:splitdata}.
The main idea is that we split the input table whenever the estimated cost of
suppressing that table is greater than a predefined parameter $\tmax$. And
for simplicity we are just using an randomized two-equal-size-partition
splitting on the original data when the estimated time is above $\tmax$.
Equation (\ref{eq:costfunc}) is defined to estimate that cost.

\begin{equation}\label{eq:costfunc}
Cost(T)=\frac{|T|*2^{\frac{N}{|T|}}}{D}
\end{equation}
%where $\mathcal{AVG}$ is the average transaction size, i.e. $\frac{N}{|T|}$,
where $N$ is the total number of items in $T$. The function estimates the
average number of \qids per item type. The larger this value is, the more
sensitive association rules we should handle. The cost function is used when
data size is large enough. We claim that in cases when $|T|$ is relatively
small there is no need to apply DnC.

\subsection{Analysis}
\label{sec:analysis}

% Analyze the main algo: time complexity, space complexity.
% Some properties to consider:
%
% \begin{itemize}
% \item give a bound on the total number of items suppressed;
% \item give a bound on the deviation in distribution from the original data;
% \item give a bound on the number of association rules that we eliminate;
% \item and what else??
% \end{itemize}

In this section, we present several theorems and lemmas along with their proofs
in order to provide an all-aspect analysis of the problem and our algorithm.

%\begin{theorem}
%  The Optimal Suppression Problem in Definition \ref{def:osp} is NP-hard.
%\end{theorem}
%% TODO prove the whole problem hierarchy
%\begin{proof}
%  TODO
%\end{proof}
%
%\begin{lemma}
%\label{lemma:rule}
%  If the inference $\mathcal{A}(q,a)$ is safe,
%  then  $\mathcal{A}(q,a, b_1,b_2,\dots,b_n)$ is safe for any sequence of $\{b_i\}$.
%\end{lemma}
%\begin{proof}
%  \begin{align*}
%    \text{$q\rightarrow a$ is safe}
%    \Rightarrow
%    &\, \frac{\csize(q\cup\{a\})}{\csize(q)} \le \rho \\
%    \Rightarrow
%    &\, \csize(q\cup\{a\}) \le \rho\cdot\csize(q)
%  \end{align*}
%  \begin{align*}
%    &\, (q\cup\{a\}) \subset (q\cup\{a, b_1,b_2,\dots,b_n\}) \\
%    \Rightarrow
%    &\,  \csize(q\cup\{a, b_1,b_2,\dots,b_n\}) \leq \csize(q\cup\{a\}) \le \rho\cdot\csize(q) \\
%    \Rightarrow
%    &\, \frac{\csize(q\cup\{a, b_1,b_2,\dots,b_n\}}{\csize(q)} \le \rho \\
%    \Rightarrow
%    &\, \text{$q\rightarrow a, b_1,b_2,\dots,b_n$ is safe}
%  \end{align*}
%\end{proof}
%
%Lemma \ref{lemma:rule} shows that we do not have to consider rules with consequent of length 2 or longer.

\begin{lemma}%[Correctness of partitioning]
\label{CorrectnessOfPartitioning}
  If $q$ is safe in both $T_1$ and $T_2$, then $q$ is safe in $T = T_1 \cup T_2$.
\end{lemma}
\begin{proof}
For any item $a$,
  \begin{align*}
   q~\text{is safe in}~T_1 &\Rightarrow sup_{T_1}(q\cup\{a\}) \le \rho\cdot sup_{T_1}(q) \\
   q~\text{is safe in}~T_2 &\Rightarrow sup_{T_2}(q\cup\{a\}) \le \rho\cdot sup_{T_2}(q)
  \end{align*}
  So \begin{align*}
   sup_{T_1}(q\cup\{a\}) + sup_{T_2}(q\cup\{a\}) &\le \rho\cdot sup_{T_1}(q) + \rho\cdot sup_{T_2}(q)
  \end{align*}
  And \begin{align*}
    sup_T(q\cup\{a\}) &= sup_{T_1}(q\cup\{a\}) + sup_{T_2}(q\cup\{a\}) \\
    sup_T(q) &= \csize_{T_1}(q) + sup_{T_2}(q)
  \end{align*}
  So $$ \frac{sup_T(q\cup\{a\})}{sup_T(q)} \le \rho~\Rightarrow q~\text{is safe in}~T .$$
\end{proof}

\begin{theorem}
\label{CorrectnessOfPartialSuppressor}
  \PartialSuppressor always terminates with a correct solution.
\end{theorem}
\begin{proof}
We first prove that if the algorithm terminates, the suppressed table is safe.
Note that the algorithm can only terminate on Line \ref{algo:partialbreak}
  in Algorithm \ref{algo:partialsuppressor}.
  Therefore, two conditions must be satisfied. First the record cursor $i$ should
  exceed the table size $|T|$. Second the value $Safe$ must be \TRUE. $Safe$ is true
  if and only if there is no unsafe \qids in the table, otherwise  Line \ref{algo:sanitize}
 will assign $Safe$ to \FALSE. If $i$ exceed $|T|$ and $Safe$ is \TRUE, the algorithm
must scans the table at least once and doesn't find any unsafe \qids. Hence, the
 suppressed table is safe.


Then we prove that \PartialSuppressor always terminates by measuring the
  number of items left (denoted $l$) in the table after each step of suppression.
Initially, $l=l_0=\sum_{i=1}^{|T|} |T[i]|\le |D| |T|$.
We state that for every invocation of \SanitizeBuffer, Line \ref{line:sanitize-suppress}
  in Algorithm \ref{algo:sanitize} is always executed at least once.
So the value $l$ strictly decreases in a positive integer
when \SanitizeBuffer is invoked.
And before the table becomes safe, \SanitizeBuffer will be invoked for
  every iteration of the loop in Algorithm \ref{algo:partialsuppressor}.
So $l$ strictly decreases for each loop iteration in \PartialSuppressor.
Because $l$ starts from a finite number which is at most $l_0=\sum_{i=1}^{|T|} |T[i]|$,
  \PartialSuppressor must terminate.
Otherwise there will be an infinite descending chain of all the $l$ values.


Now we prove that Line \ref{line:sanitize-suppress} in Algorithm \ref{line:sanitizebuffer}
  is always executed once \SanitizeBuffer is invoked.
Whenever \SanitizeBuffer is invoked, it is guaranteed that there exists
  an unsafe \qid $q\in B$ (see Line \ref{line:containunsafe}  in Algorithm \ref{algo:partialsuppressor}).
$q$ is unsafe so that there always exists an item $e\in\linked(q)$ such that $conf(q,e)>\rho$,
  i.e. \[ \frac{sup(q\cup\{e\})}{sup(q)}>\rho \Rightarrow
   sup(q\cup\{e\})-\rho\cdot sup(q)>0 .\]
For $k$ on Line \ref{line:sanitize-k} in Algorithm \ref{algo:sanitize},
  \[ k = MS(d,\SA(q,e))\]
  and
  \[MS(d,\SA(q,e)) \geq sup(q\cup\{e\})-\lfloor\rho\cdot sup(q)\rfloor \ge 1\]
  as is shown in Definition \ref{minimum}.Therefore,
  it is guaranteed that the number of deletions is at least 1
  because the rule $q\rightarrow e$ is unsafe and there must be some deletions to make it safe.
So $k\ge 1$ on Line \ref{line:sanitize-whilek} for the first time.
Thus the condition is satisfied and Line \ref{line:sanitize-suppress} is executed.
\end{proof}

\begin{corollary}
The divide-and-conquer optimization \SplitData is correct.
\end{corollary}
\begin{proof}
It follows directly from Lemma \ref{CorrectnessOfPartitioning} and
Theorem \ref{CorrectnessOfPartialSuppressor}.
\end{proof}

%\begin{theorem}
%Let %$M = |T|$ be the size of table $T$,
%$l$ be the average record length,
%$c = r_r r_d$ where $r_r$ is the regression rate and $r_d$ is the qid duplicate rate.
%The average time complexity of \PartialSuppressor is
%\[ O(c \cdot 2^l |T|^2 l (\bmax (1-\rho) + l)). \]
%\end{theorem}
%\begin{proof}
%{\small\begin{verbatim}
%  general idea:
%  l1 <- estimate the number of iterations
%    for the loop in algorithm 1 -- assume
%    there is a parameter: regression ratio
%  may have to assume the data in some distribution
%    (e.g. power-law) -- related to Figure 1
%    count the number of invocations of HandleShort
%      --> b_max
%    count the number of invocations of HandleLong
%  l2 <- estimate the number of iterations
%    for the loop in algorithm 3
%  l1+l2 -> the number of invocation of SanitizeBuffer
%  estimate complexity of SanitizeBuffer
%  estimate complexity of line 7 to 13 in algorithm 3
%    (related to distribution in Figure 2)
%  estimate the complexity of UpdateBuffer
%\end{verbatim}}]

%Let $n_1$ be the number of short records,
%$n_2$ be the number of long records,
%$p(i)$ be the probability of a record being length $i$,
%$l_m$ be the maximum record length,
%$t=1-\rho$.
%
%Let $v_1$ be the number of invocations of \HandleShortRecords.
%In the process of generating qids and filling them into the qid buffer $B$,
%duplicates cannot be counted.
%If duplicates are allowed, the number of qids generated by a record is
%just $r=\sum_{i=1}^{\lmin-1} p(i)\cdot\#qid(i)$ where
%$\#qid(i)$ is the expected number of qids generated by a record of length $i$.
%Then $\bmax/r$ records are used to fill the buffer (duplicates allowed).
%So roughly $v_1=\frac{n_1\cdot r}{\bmax r_d}$ times to consider all distinct qids
%in short records, for a single pass.
%
%For \HandleLongRecord, the number of invocations is $v_2=n_2$ if
%we do not take multiple passes of table scanning (i.e., the loop in algorithm 1) into consideration.
%Assume the loop in \HandleLongRecord is iterated for $v_3$ times, then \SanitizeBuffer is
%roughly invoked for $v_1 + v_2 v_3$ times.
%
%There are 4 major parts in the computation. We will consider them one by one.
%
%The first part is Line 4 to 7 in Algorithm 2, since the buffer capacity is $\bmax$,
%the maximum number of iterations here is roughly $\bmax r_d$.
%\HandleShortRecords will be invoked for $v_1$ times, so
%the total time cost by this part of computation is roughly $v_1 \bmax r_d$.
%
%The second part is \UpdateBuffer in Algorithm 2. For the invocation
%$\UpdateBuffer(B, T, i, j, K, L)$, the purpose is to update $K$ and $L$
%by considering qids in records $T[i..j]$ which are also in $B$.
%So a single call invocation of \UpdateBuffer costs roughly $(j-i+1)|B|$.
%Hence, the total time cost by this part of computation is roughly
%$v_1 (|T| - \frac{\bmax}{r}) \bmax$.
%
%The third part is Line 7 to 13 in Algorithm 3. Note that in reality
%the computation from Line 10 to 12 can be done at the same time when
%calculating Line 8. And in the worst case, the total time cost by calculating
%these intersections is $\dnum l_m |T|$. And the total time cost by
%this part of computation is $v_2 v_3 \dnum l_m |T|$.

%The fourth part is all the invocations of \SanitizeBuffer.
%For a single invocation of \SanitizeBuffer, there are two sub-parts to consider.
%The first sub-part is the intersection calculation on Line 5 in Algorithm 4,
%which costs roughly $l |T|$.
%The second sub-part is the computation from Line 14 to 25 in Algorithm 4,
%which costs roughly $k |B|$, where $k$ is determined on Line $9$.
%In the worst case, $k= t |T|$.
%So for a single invocation of \SanitizeBuffer, the time cost is roughly
%$|B| r_r l (|T| L + t |T| |B|)$ where $|B|$ is the size of the buffer.
%Because \SanitizeBuffer is invoked $v_1$ times in \HandleShortRecords,
%with buffer size $\bmax$, and $v_2 v_3$ times in \HandleLongRecord,
%with buffer size $\dnum$,
%the total time cost by this part of computation is roughly
%$v_1 \bmax r_r l (l |T| + t |T| \bmax) + v_2 v_3 \dnum r_r l (l |T| + t |T| \dnum)$.
%
%Summing up these four parts we can get the following time cost.
%\begin{align*}
%  n_1 r
%+ \frac{n_1 (r |T| - \bmax)}{r_d}
%+ \frac{n_1 |T| r l r_r (\bmax t + l)}{r_d} \\
%+ l_m |T| n_2 \dnum v_3
%+ l |T| n_2 \dnum v_3 r_r (l + \dnum t)
%\end{align*}
%By eliminating non-denominating terms, we get the order of \[ O(c \cdot 2^l |T|^2 l (\bmax (1-\rho) + l) ) .\]
%\end{proof}
%
%For a given dataset, the expected time complexity is actually quadratic to the size of the table.

%\begin{theorem}
%  The space complexity of \PartialSuppressor on table $T$ is \[ O(\sum_{i=1}^{|T|} |T[i]| + \bmax) .\]
%\end{theorem}
%\begin{proof}
%Let $N = \sum_{i=1}^{|T|} |T[i]|$, then $N$ is the sum of the numbers
%of all item occurrences.
%This term is easy to explain since the algorithm has to store
%the original table $T$.
%So we only need to consider local data structures
%created in \PartialSuppressor
%and related functions for the term $O(\bmax)$.
%
%For \PartialSuppressor, the most significant memory cost is from the \qid buffer of size $\bmax$.
%For \HandleShortRecords, there is only $O(1)$ extra memory space for loop variables like $j$.
%For \HandleLongRecord, there is also $O(1)$ extra memory cost.
%For \SanitizeBuffer, except for the $O(1)$ memory space for local variables, it also involves
%  the storage of $\linked(\cdot)$ and $\csize(\cdot)$.
%Because all the \qids updated are from the buffer $B$, the total number of \qids being active at any time
%  is no greater than the capacity of the buffer, i.e. $\bmax$.
%In order to keep the information of $\linked(\cdot)$ and $\csize(\cdot)$,
%  there will be $O(\bmax)$ extra memory space used.
%\end{proof}

%\begin{theorem}
%  The algorithm suppresses at most $O(xxx)$ item occurrences on average.
%\end{theorem}
%\begin{proof}
%  TODO
%\end{proof}
%
%\KZ{Say something about the property of regression?}
%
%\KZ{A property for DnC time performance? The experiment seems to show that
%time decreases exponetially with $t_{max}$ for Retail, which is amazing!}

%\begin{property}
%  Idea: distribution similarity ...
%\end{property}
%\begin{proof}
%  TODO
%\end{proof}

\section{Experiment}
In this section, we experiment on different NLG tasks. We first present the experimental setup on different tasks. Then, we show the quantitative and qualitative results together with comprehensive analysis and ablation studies.

\subsection{Implementation Details}
We evaluate the newly proposed ICL strategy on five commonly-researched natural language generation tasks: reading comprehension, dialogue summarization, style transfer, question generation and news summarization. Details on the task description, the strong baseline, corresponding  dataset, evaluation metrics and key hyper-parameters for each task are presented as follows.

\begin{table*}[th]
	\scriptsize
	\centering
	\begin{tabular}{lp{1.1cm}rrrcccc}
		\hline
		Task & Dataset & \#Train & \#Val & \#Test & Input & Output & Avg & Std\\
		\hline
		Reading Comprehension & DREAM & 6,116 & 2,040 & 2,041 & ``Q:''+ question + dialogue & answer & 5.59 & 2.61\\
		Dialogue Summarization & SAMSum & 14,732 & 818 & 819 & dialogue & summary  & 24.99 & 13.06\\
		Style Transfer & Shakespeare & 36,790 & 2,436 & 2,924 & original/modern  & modern/original  & 11.63 & 8.19 \\
		Question Generation & SQuAD1.1 & 75,722 & 10,570 & 11,877 & passage + [SEP] + answer & question & 13.09 & 4.27 \\
		News Summarization & CNNDM & 287,227& 13,368& 11,490 & document & summary & 70.97 & 29.59\\ 
		\hline
	\end{tabular}
	\caption{A summary of tasks and datasets. \#Train, \#Val and \#Test refers to the number of samples in the corresponding dataset. Avg and Std are the statistics for the number of output tokens. ``+'' refers to the concatenation operation.}
	\label{tab:taskdata}
\end{table*}

\textbf{Reading comprehension} is the task that answering questions about a piece of text. We use the DREAM dataset~\cite{sun2019dream} where questions are about corresponding dialogues and the answer is a complete sentence in natural language. We neglect the negative choices in the original dataset and formulate it as a NLG task. We adopt the pre-trained language model BART~\cite{lewis2020bart} as the baseline, where the input is a concatenation of a question and the corresponding dialogue made up of speakers and utterances. 
We experiment with  transformers\footnote{\url{https://github.com/huggingface/transformers}} based on the publically available ``facebook/bart-large'' checkpoint \footnote{\url{https://huggingface.co/facebook/bart-large}}.
%The preceding BART model is also adopted as the baseline, whereas the input is a concatenation of question and a dialogue.
The generated answers are evaluated by BLEU scores\footnote{The BLEU-1/2/3/4 scores are computed according the Google's implementation(\url{https://github.com/tensorflow/nmt/blob/master/nmt/scripts/bleu.py}).}~\cite{papineni2002bleu} widely used for QA systems, together with Meteor and Rouge-L F1 as mentioned above. The parameters are also the same as dialogue summarization, except that the early-stop is activated if there is no improvement on the perplexity of the validation set. 


\textbf{Dialogue summarization} is to generate a concise summary covering the salient information in the input dialogue. The preceding model BART has shown to be a strong baseline for this task, where only the dialogue is concatenated into a single sequence as the input. We experiment with  %transformers\footnote{\url{https://github.com/huggingface/transformers}} based on the publically available ``facebook/bart-large'' checkpoint \footnote{\url{https://huggingface.co/facebook/bart-large}} and 
SAMSum dataset\footnote{\url{https://arxiv.org/src/1911.12237v2/anc/corpus.7z}}~\cite{gliwa2019samsum} for daily-chat dialogues. 
The generated summaries are evaluated by comparing with the reference through evaluation metrics, including Rouge-1/2/L F1 scores\footnote{\url{https://github.com/pltrdy/files2rouge}}~\cite{lin2004rouge}, Meteor~\cite{banerjee2005meteor} and BertScore F1\footnote{Both Meteor and BertScore are calculated by SummEval(\url{https://github.com/Yale-LILY/SummEval}), and the latter one is based on the default bert-base-uncased model.}. We evaluate the model on the validation set after each training epoch and the early-stop patience will be added 1 if there is no improvement according to the Rouge-2 F1 score. The training process terminates when the early-stop patience equals or is larger than 3.  During the inference, the minimum and maximum output length is set to 5 and 100 respectively, with no\_repeat\_ngram\_size=3, length\_penalty=1.0 and num\_beams=4.


% The answer is either a span of words in the original text or a complete sentence in natural language.
\textbf{Style transfer} preserves the semantic meaning of a given sentence while modifies it's style, such as positive to negative, formal to informal, etc.
We adopt the Shakespeare author imitation dataset~\cite{xu2012paraphrasing}, containing William Shakespeare's original plays and corresponding modernized versions. Krishna el al.~\shortcite{krishna2020reformulating} proposed to do unsupervised style transfer by training paraphrase models based on the GPT-2 language model~\cite{radford2019language}. We re-implemented their approach STRAT\footnote{\url{https://github.com/martiansideofthemoon/style-transfer-paraphrase}} and evaluated with the provided script. Evaluation metrics includes 
transfer accuracy(ACC), semantic similarity(SIM), Fluency(FL) and two aggregation metrics, i.e., geometric averaging(GM) and their newly introduced $J(\cdot)$ metric. The hyper-parameter $hp$ equaling 0.0, 0.6 or 0.9  in Table~\ref{tab:end2endst} is the sampling parameter for trades off between ACC and SIM in their approach. 
In the training stage, we evaluate the model after updating every 500 steps. The perplexity on the validation set is used to activate the early-stop which equals 3. The inference is done as default.
 
\textbf{Question generation}~\cite{zhou2017neural} aims at generating a question given an input document and its corresponding answer span. SQuAD 1.1~\cite{rajpurkar2016squad} is generally used for evaluation. We adopt the data split as in \cite{du2017learning} and fine-tune the pre-trained UniLM~\cite{dong2019unified} as the strong baseline according to their official implementation\footnote{\url{https://github.com/microsoft/unilm/tree/master/unilm-v1}}. Generated questions are evaluated by metrics including BLEU-1/2/3/4, Meteor and Rouge-L with the provided scripts. The model is evaluated every 1000 steps and the early-stop equaling 3 is associated with the perplexity on the validation set. Other parameters are unchanged following the official guideline.

\textbf{News summarization} differs from dialogue summarization where the input is a document instead of a dialogue. We adopt the same strong baseline BART and evaluation metrics as dialogue summarization. Experiments are done with CNNDM dataset~\cite{HermannKGEKSB15} consisting of news articles and multi-sentence summaries\footnote{\url{https://github.com/pytorch/fairseq/blob/main/examples/bart/README.summarization.md}}. The model is evaluated every 2000 steps and the early-stop equaling 3 is associated with the Rouge-2 on the validation set. During the inference, the minimum and maximum output length is set to 45 and 140 respectively, with no\_repeat\_ngram\_size=3, length\_penalty=2.0 and num\_beams=4.
%\footnote{Inference parameters are borrowed from \url{https://github.com/pytorch/fairseq/blob/main/examples/bart/summarize.py}}

The summary of each task is listed in Table~\ref{tab:taskdata}. For fair comparisons, we re-implemented baselines following the above instructions on our machine. On top of the above baselines, we further arm them with the ICL strategy according to the Algorithm~\ref{alg:picl}. The settings of newly introduce Start and Stride are specified and discussed in following sub-sections. All of our experiments are done on a single RTX 3090 or a single RTX 2080Ti with 24G and 11G GPU memory respectively.
%and the result are averaged over three runs.


 
\subsection{Automatic Evaluations on Different Tasks}
\label{sec:taskperformances}

We compare our approach with the vanilla models mentioned above and the approach from~\citet{liang-etal-2021-token-wise} as baselines.
The performances on different NLG tasks are shown in Table~\ref{tab:end2end}. 
These tasks not only focus on solving different problems, but also has various amount of training data as well
as reference output lengths as shown
Table~\ref{tab:taskdata}.
Besides, the basic model are also different, including BART, GPT-2 and UniLM. 
Our new training strategy achieves significantly improvements among different tasks on most evaluation metrics, which shows that our method not only works well, but also has strong generalization abilities.

We explain the some specific results as follows:

(1) Our training strategy boosts the performances of the original STRAT with different $hp$ in the style transfer task. GM and J are two comprehensive evaluation metrics, with our approach topping the ranks with significant improvements.

(2) TCL generally performs poorly on tasks
with more training data. For example, it failed on question generation without any improvements over the vanilla model under the same parameter setting, while ICL still 
logs gains. This is mainly due to two reasons.
First, because the nature of TCL is data augmentation which is more effective in low-resource settings,
when training data is abundant, it becomes less useful. 
Second, the way they calculate the loss as sub-sequence generation better suites paraphrasing tasks, such as machine translation tested in their paper, as the order of 
the corresponding tokens between input and output 
are almost the same. Learning such forward mapping can 
be regarded as a kind of ``easy-to-hard'' 
in these limited scenarios.
However, this doesn't hold true for other tasks, 
such as summarization and question generation. 
Therefore, we didn't further test it on CNNDM since
CNNDM has the large amount of training data among
the five.

(3) For news summarization, Rouge-1 scores (precision, recall) for the baseline and our method on CNNDM are (38.16, 52.72) and (40.84, 49.23) correspondingly. Our method made substantial improvements on the precision with a compromise on the recall. 
The meteor score based on the unigram precision and recall emphasizes more on the recall than the Rouge-1 F1. As a result, it drops while Rouge-1 F1 increases. Overall, our method still outperforms BART on this task, especially on F1 scores of Rouge-2 and Rouge-L.




\begin{table}[th]
	\small
	\centering
	\begin{subtable}{\linewidth}
		\scriptsize
		\centering
		\begin{tabular}{lcccccc}
			\hline
			{Method} & {B1} & {B2} & {B3} & {B4} & {Met} & {RL}\\
			\hline
			w/o CL &  32.03 & 16.01 & 8.77 & \textbf{4.80} & 19.84 & 38.89\\
			TCL & 32.53 & 16.25 & 8.52 &4.67 &19.88 & 39.65 \\
			ICL &  \underline{\textbf{33.99}} & \underline{\textbf{17.43}} & \underline{\textbf{9.18 }}& 4.64 & \textbf{20.60} & \textbf{40.78}\\

			\hline
		\end{tabular}
		\caption{Reading Comprehension}
		\label{tab:end2endrc}
	\end{subtable}
	\\[5pt]
	\begin{subtable}{\linewidth}
		\scriptsize
		\centering
		\begin{tabular}{lccccc}
			\hline
			{Method} & {R1} & {R2} & {RL} & {Met} & {BertS} \\
			\hline
			%BART & 52.60&27.00 &42.10 &- & - \\
			w/o CL & 51.88 & 27.30 & 42.77 & 24.75 & 71.38 \\
			TCL  & 52.33 & 27.80 & \textbf{43.91} & 24.59 & 71.77 \\
			ICL & \underline{\textbf{53.07}} & \underline{\textbf{28.23}} & {43.83} & \underline{\textbf{26.12}}& \underline{\textbf{72.17}} \\
			
			\hline
		\end{tabular}
		\caption{Dialogue Summarization}
		\label{tab:end2endds}
	\end{subtable}
	\\[5pt]
	\begin{subtable}{\linewidth}
		\scriptsize
		\centering
		\begin{tabular}{lcccccc}
			
			\hline
			{Method}&$hp$ &  {ACC} & {SIM} & {FL} & {GM} & {J}\\
			\hline
			%\multirow{3}{*}{STRAT}& 0.0 & 71.70 & \textbf{56.40} & 85.20 & 70.10 & 34.70 \\
			%& 0.6 & 75.70 & 53.70 & 82.70 & 69.50 & 33.50 \\
			%& 0.9 & 79.80 & 47.60 & 71.70 & 64.80 & 27.50 \\
			%\hline
			\multirow{3}{*}{w/o CL}& 0.0 & 70.49 & 55.70 & 85.98 & 69.63& 33.72 \\
			& 0.6 &75.31 & 53.46 & 82.56 & 69.27& 33.30\\
			& 0.9 & 78.76 & 47.38 & 74.42 &65.24 & 27.88\\
						\hline
			\multirow{3}{*}{TCL } & 0.0 & 70.31 & \textbf{55.95} &\textbf{87.24} &  70.01& 34.71 \\
			& 0.6 & 74.79 & 53.14 & 82.56 & 68.97 & 33.21 \\
			& 0.9 & 79.41 & 46.88 & 71.92 &64.45 & 26.92 \\
			\hline
			\multirow{3}{*}{ICL}& 0.0 & \underline{73.72} & 55.91 & 86.30 & \underline{\textbf{70.60}} &\underline{\textbf{35.81}}\\
			& 0.6 & 77.26 & \underline{53.80} & \underline{83.87} & \underline{70.38} & 34.64\\
			& 0.9 & \textbf{79.65} & 48.16 & 76.06 & 66.32 & 29.03\\

			\hline
		\end{tabular}
		\caption{Style Transfer.}
		\label{tab:end2endst}
	\end{subtable}
	\\[5pt]
	\begin{subtable}{\linewidth}
		\scriptsize
		\centering
		\begin{tabular}{lcccccc}
			\hline
			{Method} & {B1} & {B2} & {B3} & {B4} & {Met} & {RL}\\
			\hline
			w/o CL & \textbf{50.38} & 35.67 & 27.24 & 21.36 & 24.40 & 50.67 \\
			TCL &\textbf{50.38} & 35.67 & 27.24 & 21.36 & 24.40 & 50.67\\
			ICL &  50.18 & \textbf{35.72} & \textbf{27.36} & \textbf{21.54} & \textbf{24.57} & \underline{\textbf{51.09}} \\
			\hline
		\end{tabular}
		\caption{Question Generation}
		\label{tab:end2endqg}
	\end{subtable}
		\\[5pt]
	\begin{subtable}{\linewidth}
		\scriptsize
		\centering
		\begin{tabular}{lccccc}
			\hline
			{Method} & {R1} & {R2} & {RL} & {Met} & {BertS}\\
			\hline
			%BART &  \\
			w/o CL &  43.07 & 20.01 & 35.94 & \textbf{21.44} & 63.72 \\
			TCL & - & -&- &- &- \\
			ICL & \textbf{43.39} & \underline{\textbf{20.55}} & \underline{\textbf{36.63}} & 19.68 & \textbf{64.05}\\
			\hline
		\end{tabular}
		\caption{News Summarization}
		\label{tab:end2endns}
	\end{subtable}
	\caption{Performances on different NLG tasks. ICL represents the models trained with our ICL algorithm. TCL refers to the previous work from~\cite{liang-etal-2021-token-wise}. Scores underlined are statistically significantly better than both re-implemented baselines with $p<0.05$ according to t-test. }	
	\label{tab:end2end}
\end{table}


\subsection{Human Evaluations}

To further prove the improvement of ICL, we hired three proficient English speakers for human evaluation. 20 samples from the test set of each task are randomly selected, ignoring the ones with totally same generations among three models, including the vanilla model, TCL and ICL. The original input, reference output and three generations are shown to annotators together, while the order of three generations are unknown and different among samples. 3-point Likert Scale is adopted for scoring for each generation~\cite{gliwa2019samsum}, where [1, 3, 5] represent 
excellent, moderate and disappointing results 
respectively. The average scores and agreements 
among the annotators are shown in 
Table~\ref{tab:humaneval}.

The Fleiss Kappa on the first four tasks indicates the fair to moderate agreements. It shows the promising improvement of ICL over the vanilla model and TCL especially on DREAM, SAMSum, and SQuAD1.1, which is consistent with the conclusion based on automatic metrics.
Although the agreement on style transfer is fair, 
our annotators without Shakespeare background 
tend to give low scores to all outputs.
Therefore, the absolute improvement is 
only $0.04$ compared to both baselines.
%This mainly due to the indistinguishable styles between
%Shakespeare’s plays with are quite different from modern languages. 
Besides, the poor agreement on CNNDM reflects the 
diverse concerns of summarization from different 
annotators. Without more specific instructions, they 
tends to focus more on the content coverage instead 
of checking the detailed facts. This is also 
consistent with the higher Meteor scores of the 
vanilla model over ICL.

\begin{table}[th]
	\scriptsize
	\centering
	\begin{tabular}{l|ccc|c}
		\hline
		{Datasets} & {w/o CL} & {TCL} & {ICL} & {Agreement}  \\
		\hline
		DREAM  &3.07 & 2.50&3.20 &0.48 \\
		SAMSum &2.97 &3.57 &3.97 &0.40 \\
		Shakespeare &2.23 &2.23 & 2.27&0.32 \\
		SQuAD1.1 &3.43 & 3.43 &3.77 &0.35 \\
		CNNDM & 3.45 &- &3.40 &0.11 \\
	%	\hline
	%	overall & & & &\\
		\hline
	\end{tabular}
	\caption{Human evaluations. The agreement is calculated by Fleiss Kappa.}
	\label{tab:humaneval}
\end{table}




%Following Liu et al.\shortcite{liu2021competence}'s work, we asked annotators to comparing the performance between our generated results and baselines by choosing from ``Better, Tie, Worse''. 
%The counts for each choice are shown in Table~\cite{}, where the Fleiss Kappa among annotators is ??.

%Analysis





%\subsection{Analysis on Variable Generation Lengths}

%Teacher forcing, which predicts each token given the reference summary tokens during training and given the previous generated tokens during inference, leads to the exposure bias problem for NLG tasks.
%Since ICL starts the training process by predicting the last few tokens of outputs and gradually calculates the loss based on more tokens when the model is stronger, we hypothesis that it can alleviate the exposure bias for training Seq2Seq models to some extent.
%As stated in~\cite{pang2020text}, the output quality tends to degrade as the output length increase with the exposure bias.
%So, we divided the test set of each task according to the length of the generated output into 4 buckets and randomly picked 20 samples in each buckets for both the corresponding baselines and our approach. Each generation is annotated by 5 point Likert Scale, where 1 is the worst and 5 is the best. 

%The trends of performances on variable generation lengths are in Figure~\ref{}.


\section{Related Work}
This section surveys previous works on question generation and tree encoding
respectively.

Text question generation has attracted the attention 
after the work of ~\citeauthor{du2017learning}~\shortcite{du2017learning}, who uses deep seq2seq model 
to generate questions from a raw text paragraph. 
Before that, text question generation relied heavily on hand-craft 
question patterns~\cite{HeilmanS10,LabutovBV15,MostowC09} which is time and 
labor consuming. 

However, this pure seq2seq model is not focused and 
has no control over part in the paragraph to generate question. 
~\citeauthor{zhou2017neural}~\shortcite{zhou2017neural} proposed to encode 
key phrase information using binary indicators to generate 
key-aware questions and they assumes the answer to be key phrase. 
Considering key phrase (answer) is unavailable in reality, 
~\citeauthor{SubramanianWYT17}~\shortcite{SubramanianWYT17} applied 
a two-stage approach. First, key phrases are extracted by 
pointer network~\cite{ptrnet}. Second, 
key phrases are encoded in the same way as 
Zhou et al. With the intuition that questions could be asked in many ways, 
~\citeauthor{Yao2018vae}~\shortcite{Yao2018vae} used conditional-VAE to 
increase the diversity of questions. More recently, models with 
auxiliary feature information~\cite{HarrisonW18} helped improve 
the question quality. Structure question generation aims at 
converting structured data such as triples in knowledge graph to questions. 
~\citeauthor{SerbanGGACCB16}~\shortcite{SerbanGGACCB16} proposed a model to generate factoid questions from knowledge base triples.  None of the above work
considered using parse tree structures to aid question generation process,
which is the focus of this paper.

Sequential RNN model takes sentence as a sequence of words, 
ignoring the syntactic information. In order to utilize
such syntactic information with sequential information, 
~\citeauthor{tai2015improved}~\shortcite{tai2015improved} proposed Tree-LSTM to 
encode the binary parse tree recursively in a bottom-up fashion to 
classify sentiment. In text generation task, 
\citeauthor{eriguchi2016tree}~\shortcite{eriguchi2016tree} 
proposed a tree-to-sequence model with attention mechanism to do 
machine translation and 
~\citeauthor{liang2018automatic}~\shortcite{liang2018automatic} proposed a 
tree-to-sequence model which could handle arbitrary trees, 
to do code comment generation. Our work is inspired by these previous
attempts and we are first to adapt structure encoded neural models to
textual question generations.
\section{Conclusion}
We implement a novel sequence-based dependency parsing
framework which takes advantage of high order features 
in parsing history. 
%We can also adapt beam search to this framework so as to
%relax the strictly greedy nature. Vine pruning\cite{rush2012vine} could
%be incorporated to speed up the parsing.
More importantly, we discovered that the parsing accuracy is very sensitive to
the quality of parsing sequence. Future work can be focused on
developing better sequence predictors that outperform Malt action classifier.
Furthermore, we use two sets of features for sequence predictor and
head mapper right now. A unified set of features between these two components
are worth exploring.
%Besides, better sequence predicting method and unified feature
%representation of two components are worth exploring.
%
%Though we currently get a not bad result,
%the sequence predictor still needs more exploration.
%According to our experiment, slightly changes
%on the sequence can lead to a fatal decline on accuracy. Ensuring the match degree of training sequence and testing
%sequence demands a high quality of sequence predictor.
%
%Further, the features in our current implementation are not expanded and well tuned yet  and we are free to define high order features to make use of parsing history. Our framework is flexible to merge other technics to enhance the performance. Introducing beam could make up for our greedy decoder and improve our accuracy. Vine pruning\cite{rush2012vine} could speed up parsing process. Besides, better sequence predicting method and unified feature representation of two components are worth exploring.


%% The Appendices part is started with the command \appendix;
%% appendix sections are then done as normal sections
%% \appendix

%% \section{}
%% \label{}

%% References
%%
%% Following citation commands can be used in the body text:
%% Usage of \cite is as follows:
%%   \cite{key}         ==>>  [#]
%%   \cite[chap. 2]{key} ==>> [#, chap. 2]
%%

%% References with bibTeX database:

\bibliographystyle{elsarticle-num}
\bibliography{privacy}

%% Authors are advised to submit their bibtex database files. They are
%% requested to list a bibtex style file in the manuscript if they do
%% not want to use elsarticle-num.bst.

%% References without bibTeX database:

% \begin{thebibliography}{00}

%% \bibitem must have the following form:
%%   \bibitem{key}...
%%

% \bibitem{}

% \end{thebibliography}


\end{document}

%%
%% End of file `elsarticle-template-num.tex'.
