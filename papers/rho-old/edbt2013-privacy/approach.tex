\section{Approach}
\label{sec:app}

We adopt the $\rho$-uncertainty privacy model
\cite{Cao:2010:rho} which requires that the confidence of any sensitive
association rules obtained from the data is less than or equal to $\rho$.
A sensitive association
rule is one whose consequent contains at least one sensitive item.
To find a partial suppression method
which incurs minimum information loss,
it needs to check all combinations of items suppressed, which requires
$O(2^N)$ time where $N$ is the total number of items in dataset. Problem of
this sort has been shown to be an NP-hard problem
\cite{atallah99:disclosure,Xu:2008:ATD}.
In this paper, we present a heuristic approach
to this problem.

Our main intuition is, although the total number of unsafe
sensitive association rules (whose confidence is above $\rho$) may
be the worst case exponential in the original data,
incremental ``invalidation'' of some of the
rules through partial suppression of small number of affected items can
dramatically decrease the number of these bad rules, which leads to quicker
convergence to a solution.

\subsection{Preliminaries}
\begin{definition}
\label{def:sup_conf} Support $sup_{T}(I)$ of an itemset $I$ is the number of
records in $T$ including $I$. The confidence $conf_{T}(q,e)$ of
association rule $q\rightarrow e$ is $sup_T(q \cup e)/sup_T(q)$.
\end{definition}

\begin{definition}
\label{def:quasi_id} A \emph{quasi-identifier} (also \qid) $q$ is a set of
items (including both sensitive and non-sensitive items) taken from any
record.
\end{definition}

\begin{definition}
\label{def:safety_rule} A sensitive association rule $q\rightarrow e$ is safe
\wrt~$\rho$ if and only if $conf(q,e)\leq\rho$.
\end{definition}

\begin{definition}[Breach Probability]
\label{def:probability} The \emph{breach probability} of a \qid $q$ is
$\breach(q) = \max_{e\in item domain} conf(q,e)$
\end{definition}

\begin{definition}%[Safety of qid]
\label{def:safety_qid} A \qid $q$ is safe \wrt~$\rho$ if and only if
$\breach(q)\leq\rho$.
\end{definition}

\begin{definition}%[Safety of Table]
\label{def:safety_table} A dataset $T$ is safe \wrt~$\rho$ if and only if $q$
is safe \wrt~$\rho$ for any \qid $q$ appears in $T$.
\end{definition}
%The $\rho$-uncertainty privacy model\cite{Cao:2010:rho}
%requires that all sensitive association rules are safe. By ensuring
%confidences of all sensitive association rules with one consequent below
%$\rho$, confidences of all sensitive association rules with more consequents
%are also made below $\rho$. As a result, Definition \ref{def:safety_table}
%can satisfy the $\rho$-uncertainty privacy model.


\subsection{A Divide-and-Conquer Framework}
\label{subsec:speedup}
Since the complexity of verifying the privacy of a data set depends on
the scale of the data (including total number of records,
average record size, and the domain size), at the top level,
we propose to partition the
input data dynamically in a divide-and-conquer manner. Lemma
\ref{CorrectnessOfPartitioning} below ensures that if each partition is
partially suppressed to satisfy the $\rho$-uncertain model, then the
whole data set also satisfies the model and is deemed safe. The average
of this approach is that it not only linearizes the input but also
gives rise to the parallel execution which provides
further speed-up, which will be shown in Section \ref{sec:eval}.

\begin{lemma}%[Correctness of partitioning]
\label{CorrectnessOfPartitioning}
  If $q$ is safe in both $T_1$ and $T_2$, then $q$ is safe in $T = T_1 \cup T_2$.
\end{lemma}
\begin{proof}
For any item $a$,
  \begin{align*}
   q~\text{is safe in}~T_1 &\Rightarrow sup_{T_1}(q\cup\{a\}) \le \rho\cdot sup_{T_1}(q) \\
   q~\text{is safe in}~T_2 &\Rightarrow sup_{T_2}(q\cup\{a\}) \le \rho\cdot sup_{T_2}(q)
  \end{align*}
  So \begin{align*}
   sup_{T_1}(q\cup\{a\}) + sup_{T_2}(q\cup\{a\}) &\le \rho\cdot sup_{T_1}(q) + \rho\cdot sup_{T_2}(q)
  \end{align*}
  And \begin{align*}
    sup_T(q\cup\{a\}) &= sup_{T_1}(q\cup\{a\}) + sup_{T_2}(q\cup\{a\}) \\
    sup_T(q) &= sup_{T_1}(q) + sup_{T_2}(q)
  \end{align*}
  So $$ \frac{sup_T(q\cup\{a\})}{sup_T(q)} \le \rho~\Rightarrow q~\text{is safe in}~T .$$
\end{proof}

In this framework, we split the input data into two
equally-sized chunks when the estimated cost of suppressing the input is
greater than a threshold $\tmax$. Equation
(\ref{eq:costfunc}) defines the cost of suppressing an input $T$.

\begin{equation}\label{eq:costfunc}
Cost(T)=\frac{|T|*2^{\frac{N}{|T|}}}{D}
\end{equation}
where $N$ is the total number of items in $T$. The function estimates the
average number of association rules per item type. The larger this value is,
the more sensitive association rules we should handle (assuming that
the number of sensitive rules are positively related to the total number of
rules).
%The cost function is
%used/useful? when average transaction size and table size are large.
%We claim that in cases when $|T|$ is relatively small there is no need to
%apply DnC.

\subsection{Partial Suppressor}
\label{subsec:partial_sup}
To ensure that each partition is safe, we must make sure all \qids in the
partition are safe. Since a \qid is a combination of different item types,
the number of distinct \qids is large. It is thus impractical to enumerate all
distinct \qids in one go. Instead, we enumerate one set of \qids at a time
and fix these \qids by suppressing just enough items so that these \qids are
safe. Once this batch of \qids are fixed, we move on to fix the next batch,
until all \qids are safe. We do this by introducing a \qid buffer of fixed
capacity $\bmax$. The \qid buffer contains the set of \qids to be checked and
fixed in each batch. The value of $\bmax$ is significant. Small $\bmax$
values can cause repetitive enumeration of \qids because a \qid made safe in
one batch may become unsafe again after processing another batch, a
phenomenon known as {\em regression}. Large $\bmax$ values causes unnecessary
enumeration of massive \qids, many of which do not exist
by the time all previous \qids in the buffer are fixed, due to the item
suppressions.

%\begin{algorithm}[th]
%\caption{$\PartialSuppressor(T, D_S, \rho, \bmax)$}
%\label{algo:partialsuppressor}
%\begin{algorithmic}[1]
%    \STATE Initialize $safe\leftarrow\TRUE$, $i\leftarrow 1$;
%%    \STATE Initialize $safe\leftarrow\TRUE$
%    \LOOP
%        \STATE Initialize $sup$, $\linked$, $S$, $L$;
%        \WHILE {$|B|<b_{max}$ \AND $i\leq |T|$} \label{algo:enu_s}
%             \STATE Fill $B$ with \qids generated by $T[i]$, \label{algo:enumerate1}
%             \STATE update $sup$, $\linked$, $S$, $L$; \label{algo:enumerate2}
%             \STATE $i\leftarrow i+1$;
%        \ENDWHILE \label{algo:enu_e}
%                \STATE update $sup$, $\linked$, $S$, $L$;
%%        \STATE Calculate $\rho$ of each $qid$ in $|B|$ \label{algo:update}
%        \IF {$B$ contains an unsafe \qids}\label{line:containunsafe}
%            \STATE Sanitize Buffer $B$;\label{line:sanitizebuffer}
%            \STATE $safe\leftarrow\FALSE$;
%        \ENDIF
%        \IF {$i \ge |T|$ \AND $safe$}
%            \STATE \textbf{break};\label{algo:partialbreak}
%        \ELSIF {$i \ge |T|$}
%                \STATE $i\leftarrow 1$;
%                \STATE $safe\leftarrow\TRUE$;
%%                \STATE \textbf{continue}
%        \ENDIF
%    \ENDLOOP
%\end{algorithmic}
%\end{algorithm}

%Algorithm \ref{algo:partialsuppressor} gives the pseudo-code of the
%partial suppressor.
Starting from the beginning of the partition,
the partial suppressor scans through the records and performs
the following two-phase sanitization repeatedly. In phase 1,
it takes a number of records starting from the current position, and enumerates
\qids to fill up the \qid buffer. Meanwhile, the sensitive items associated with
each \qid is also determined and the support of the \qid computed.
In phase 2, it sanitizes all the unsafe \qids in the buffer.
The algorithm terminates when the whole partition is scanned in one go
and there is no unsafe \qid discovered.

If a \qid is unsafe, it is involved in at least one {\em unsafe} sensitive
association rule. Sanitizing the \qid buffer is equivalent to disabling all
the unsafe sensitive rules associated with the \qids in the buffer by
suppressing some items in the original data. To disable a rule $q \rightarrow
e$, the number of items of type $t \in q\cup e$ that need to be suppressed is
given by
\[N_s(t, q\rightarrow e)=
\begin{cases}
sup(q\cup \{e\})-sup(q)\rho & t=e  \\
\frac{sup(q\cup \{e\})-sup(q)\rho}{1-\rho} & t\in q %\\
% \infty & otherwise
\end{cases} \]
Of all item types in $q\cup e$, there exists an item type $t_{min} \in q\cup
e$ which results in to minimum suppression. In other words, for each unsafe
sensitive rule $r$, we need to delete $N_s(t_{min}, r)$ items of type
$t_{min}$ to make it safe. In this work, we delete {\em any} $N_s(t_{min},
r)$ items from the original data.

Since there are multiple unsafe rules associated to the buffer, we disable one
rule at a time. Each time we pick a rule to disable according to one of the following
two heuristics.

The first heuristic prefers to focus the suppression on a few item types and
retain the support of other items as much as possible.
This is in the same spirit of global
suppression though we don't have to suppress all items of the same type. The
advantage of this heuristic is that it doesn't introduce {\em spurious rules}
and therefore produces data that is more suitable in rule mining tasks.

The second heuristic prefers to disable a rule which results in a data distribution
which is closest to the orginal data distribution, by Kullback-Leibler divergence.
The effect of this heuristic is to preserve the original data distribution and hence
is more suitable for statistical analysis tasks.

%The above two heuristics not only determine the best strong sensitive
%association rule to fix, also need to fully decide how to fix a specific
%rule. As shown below, for a strong sensitive association rule $q\rightarrow
%e$ whose confidence $conf(q,e)$ is bigger than $\rho$, we have two
%conservative ways to fix it: one is suppressing $sup(q\cup \{e\})-sup(q)\rho$
%replications of e, the other one is suppressing $\frac{sup(q\cup
%\{e\})-sup(q)\rho}{1-\rho}$ replication of any $t\in q$. In this way the
%heuristics follow the strategy explained above and fixed a picked rule.
%we follow the Definition \ref{def:minimum} which decides the minimum
%information loss it must incur.
%
%\begin{definition}[Minimum suppression]
%\label{def:minimum} Minimum suppression of item $t$ to make strong sensitive
%association rule $q\rightarrow e$ safe, i.e. $conf(q,e)\leq\rho$:
% \hspace{4mm}
%\[MS(t, q\rightarrow e)=
%\begin{cases}
%sup(q\cup \{e\})-sup(q)\rho & t=e  \\
%\frac{sup(q\cup \{e\})-sup(q)\rho}{1-\rho} & t\in q \\
% \infty & otherwise
%\end{cases} \]
%\end{definition}

%After fixing one strong association rule with several replication of one
%item type suppressed, it'll determine the set of \qids in \qid buffer that
%would be affected.


%\subsection{Implementation Specifics}
%\begin{definition}[Minimum suppression]
%\label{minimum} Minimum suppression of item $t$ to make strong sensitive
%association rule $q\rightarrow e$ safe, i.e. $conf(q,e)\leq\rho$:
% \hspace{4mm}
%\[MS(t, q\rightarrow e)=
%\begin{cases}
%sup(q\cup \{e\})-sup(q)\rho & t=e  \\
%\frac{sup(q\cup \{e\})-sup(q)\rho}{1-\rho} & t\in q \\
% \infty & otherwise
%\end{cases} \]
%\end{definition}
%Specifically, it represents that we do not intend to suppress more items to
%make $q\rightarrow e$ a safe one as it can make $conf(q,e)$ even smaller.
%
%\subsubsection{Pruning}
%In first phase - \qid generation, we utilize a fixed size cache to remember
%\qids which are not associated with any sensitive item any more, to prevent
%generating \qids that are superset of them. It��s a pruning technique we can
%leverage since if a short \qid does not associated with sensitive items,
%neither do its supersets.
%
%\subsubsection{Handle Long Records}
%A record $R$ has as many as $2^{|R|}$ \qids. Suppose $\bmax = 10^6$, then a
%single record of length $20$ can fill up $B$. However, long \qids are not
%expected frequently in the dataset, and thus can be handled separately. Since
%we can't possibly enumerate all the \qids within a long record, we do this in
%batches (setting of batch size). Because the number of sensitive items are
%limited in any record, sanitizing the first batch of \qids in a long record
%may already cause the removal of all the sensitive items in that record.
%
%\subsubsection{Heuristic Approximation}
%In second phase - \qid buffer sanitization, it is very time-consuming to
%iterate over all strong sensitive association rules to find the `best` one
%according to two optional heuristics, since the number of \qids are enormous
%and each \qid can link to various sensitive items. While the iteration number
%are also propositional to the number of \qids, so the cost is tremendous.
%Therefore, we use an approximation way which randomly pick small amount of
%\qids in the buffer, e.g 1,000, to find the locally `best` one to sanitize.
