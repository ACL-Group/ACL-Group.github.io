 \section{Partial Suppression Algorithm}
\label{sec:algo}

\renewcommand{\algorithmicforall}{\textbf{for each}}
%\algnotext{ENDFOR}
%\algnotext{ENDIF}
%\algnotext{ENDWHILE}
%Main algo and its variants. Give pseudo code and description.
%Use pics whenever possible. Discuss the ``knob''.

%First we come to the question why we opt for partial suppression.
%We are inspired by the global suppression proposed in $\rho$-uncertainty \cite{Cao:2010:rho}.
%Apparently, we had a immediate impression that it would suppress
%many items including both non-sensitive and sensitive ones.
%Although privacy model has been satisfied, but the data is highly
%cut sparse. \cite{Cao:2010:rho} also used global generalization method to anonymize transactions.
%As we can see, data is generalized, rules mined from the generalized data also become vague.
%It not only creates disambiguated rules but also influences the practical usage of the mined rules.
%In the last part of \cite{Cao:2010:rho}, authors also addressed problems of generalization method,
%the non-monotonicity property.
%Since both algorithms in \cite{Cao:2010:rho} rely on global suppression,
%the authors clarified that both algorithms are meant for datasets which may include large number of transactions,
%but the number of distinct items per transaction is relatively small.

%Partial suppression in our mind is the one that can suppress as few items as possible to
%satisfy the criteria of a safe set-valued table (see Definition \ref{def:safety_table}).
%If we are required to suppress only one item we won't suppress more.
%This is the intuition of partial suppression.
%Partial suppression also draws some doubts in
%\cite{Cao:2010:rho, tkde:VerykiosEBSD04:ARH,tkde:WuCC07:hiding} mainly for
% its side-effects in rule mining.
%But no one really takes action to challenge the
%side-effects in anonymization of transaction datasets.
%So we are the first to try and we believe its not that bad
%as you see and also the partial suppressed result is practical in usage.
%In experimental section we will compare results of our
%partial suppression algorithm with results of both algorithms in $\rho$-uncerntainty \cite{Cao:2010:rho},
%we will show that partial suppression method
%is much better than you think and will be acceptable in practical usage.


%%%
% TODO comments
%\XH{do we need to declare that, later when we say update $linked(q)$
%we also mean to update $\csize(q \cup \{e\})$ for
%each sensitive item $e \in \linked(q)$ implicitly?
%}
%%%

The Optimal Suppression Problem defined in Section \ref{sec:prob} is
a combinatorial optimization problem. To find the optimal suppressor, it needs
to check all combinations of items suppressed, which yields $O(2^N)$
complexity and $N$ is the total number of items in $T$. Instead we present
the partial suppression algorithm as a heuristic solution to Optimal
Suppression Problem.
%By definition of a safe set-valued table (Definition
%\ref{def:safety_table}), the breach probability of each \qid in $Q$ must be
%under the threshold $\rho$ (Definition \ref{def:safety_qid}).
%For
%example, given a \qid $q=\{a, b, \alpha\}$ and all its linked sensitive items
%$\linked(q)=\{\beta\}$, where $\{a, b\}$ are non-sensitive items and
%$\{\alpha, \beta\}$ are sensitive items, if $\csize(q)=5$, $\csize(q \cup
%\{\beta\})=3$ and $\rho=\frac{1}{3}$, then $\breach(q)=\frac{3}{5}>\rho$.
%\begin{definition}[Data Structures]
%We define three key data structures in this framework.
%$B$ is a \qid buffer which is a set of \qids.
%$K$ is a mapping which is essentially a materialized function from \qid $q$ to $\csize(q)$.
%$L$ is a mapping from \qid $q$ to $\linked(q)$.
%\end{definition}
Before we present the details of our algorithm, we define some notations for
simplifying the description.
\begin{definition}[Data Structures]
We define three key data structures in this framework. $B$ is a \qid buffer
which stores a set of \qids. $S$ is a \textbf{mapping} which is essentially a
materialized function from \qid $q$ to $sup(q)$. $L$ is a \textbf{mapping}
from \qid $q$ to $\linked(q)$.
\end{definition}
In the following algorithms, we will also use $sup(\cdot)$ and
$\linked(\cdot)$ to denote the computations of these two functions, and use
$S(\cdot)$ and $L(\cdot)$ to denote the access of the elements of the two
data structures.
\begin{definition}[Minimum suppression]
\label{minimum}
Minimum suppression of item $t$ to make strong sensitive association rule
$\SA(q,e)$ safe, i.e. $conf(q,e)\leq\rho$:
 \hspace{4mm}
\[MS(t,\SA(q,e))=
\begin{cases}
sup(q\cup \{e\})-sup(q)\rho & t=e  \\
\frac{sup(q\cup \{e\})-sup(q)\rho}{1-\rho} & t\in q \\
 \infty & otherwise
\end{cases} \]
\end{definition}
%\emph{Minimum suppression} is essentially the number of item $t$ suppressed
%to fix the inference $\mathcal{A}(q,e)$.
Specifically, it represents that we do not intend to suppress more items to
make $\SA(q,e)$ a safe one as it can make $conf(q,e)$ even smaller.
\begin{definition}[Residual of Item type $t$]
The residual of item $t$ is defined as
\[ residual(t)=\frac{sup_{T^\prime}(\{t\})}{sup_T(\{t\})} \]
where $T$ is the original data and $T^\prime$ is the internal suppressing
result.
% being\XH{??} anonymized data.
\end{definition}

%\XJ{move this para to algo sec} \MakeRed{
Our main idea is although the total number of ``dangerous'' association rules
maybe the worst case exponential in the original data, incremental
``invalidation'' of some of the rules through partial suppression of small
number of affected items can dramatically decrease the number of these bad
rules, which leads to quicker convergence to a solution.
%}
%%%
%
%In this paper, we adopts three kinds of partial suppression policies.
%The first one is \PartialR, which suppresses only sensitive items
%in the consequents. The second one is \PartialL, which
%suppresses only items in the antecedents.
%The third one is \PartialALL, which suppresses items in
%both the consequents and the antecedents.
%\PartialL and \PartialALL suppress both sensitive
%and non-sensitive items, whereas \PartialR suppresses only
%sensitive items.
Next we present the \textbf{general framework}, known as the \textbf{basic
algorithm}.

\subsection{The Basic Algorithm}
%
%Table \ref{table:problem_notations} and \ref{table:algo_notations}
%include notations for some global variables and parameters.
%They are defined in advance in order to omit the declarations in
%related functions.

\begin{algorithm}[th]
\caption{$\PartialSuppressor(T, D_S, \rho, \bmax)$}
\label{algo:partialsuppressor}
\begin{algorithmic}[1]
    \STATE Initialize $safe\leftarrow\TRUE$, $i\leftarrow 1$;
%    \STATE Initialize $safe\leftarrow\TRUE$
    \LOOP
        \STATE Initialize $sup$, $\linked$, $S$, $L$;
        \WHILE {$|B|<b_{max}$ \AND $i\leq |T|$} \label{algo:enu_s}
             \STATE Fill $B$ with \qids generated by $T[i]$, \label{algo:enumerate1}
             \STATE update $sup$, $\linked$, $S$, $L$; \label{algo:enumerate2}
             \STATE $i\leftarrow i+1$;
        \ENDWHILE \label{algo:enu_e}
		\STATE \textcolor{red}{update $sup$, $\linked$, $S$, $L$;}
%        \STATE Calculate $\rho$ of each $qid$ in $|B|$ \label{algo:update}
        \IF {$B$ contains an unsafe \qids}\label{line:containunsafe}
            \STATE $\SanitizeBuffer(T, B, sup, \linked, S, L)$;\label{line:sanitizebuffer}
            \STATE $safe\leftarrow\FALSE$;
        \ENDIF
        \IF {$i \ge |T|$ \AND $safe$}
            \STATE \textbf{break};\label{algo:partialbreak}
        \ELSIF {$i \ge |T|$}
                \STATE $i\leftarrow 1$;
                \STATE $safe\leftarrow\TRUE$;
%                \STATE \textbf{continue}
        \ENDIF
    \ENDLOOP
\end{algorithmic}
\end{algorithm}

\begin{algorithm}[th]
\caption{$\SanitizeBuffer(T, B, sup, \linked, S, L)$}
\label{algo:sanitize}
\begin{algorithmic}[1]
\STATE $\policy \leftarrow \SuppressionPolicy()$; \label{choose_heur}
\REPEAT
\label{algo:pick_rs}
    \STATE pick any $q,e$ such that $conf(q,e)>\rho$;
    \IF {$\policy = Distribution$}
        \STATE $d\gets\underset{d\in q\cup\{e\}}{\arg\max}\,H_{dist}(d)$;
        \label{algo:heur_dist}
    \ELSE
        \STATE $d\gets\underset{d\in q\cup\{e\}}{\arg\min}\,H_{mine}(d)$;
        \label{algo:heur_mine}
    \ENDIF
%    \IF {$\policy = Distribution$}
%        \STATE  find $\SA(q,e)$ with maximal $H_{dist}(d)$, where
%        \STATE  $d\in q\cup\{e\}$ \AND $conf(q,e)>\rho$
%        \label{algo:heur_dist}
%    \ELSE
%        \STATE find $\SA(q,e)$ with minimal $H_{mine}(d)$, where
%        \STATE  $d\in q\cup\{e\}$ \AND $conf(q,e)>\rho$
%        \label{algo:heur_mine}
%    \ENDIF
    \STATE $X\leftarrow q\cup\{e\}$;
    \STATE $k\leftarrow MS(d,\SA(q,e))$;\label{line:sanitize-k}
    \WHILE{$k>0$}\label{line:sanitize-whilek}
        \STATE $R\leftarrow$ a record containing $X$, i.e. $R\subseteq
        \container(X)$;
        \label{pick_row}
        \STATE $R\leftarrow R-\{d\}$;\label{line:sanitize-suppress}
        \STATE Update $sup$, $\linked$, $S$, $L$ of \qids contained in $R$;
        \label{algo:update_kl}
        \STATE $k \leftarrow k-1$;
    \ENDWHILE
\UNTIL{there is no unsafe \qid in $B$}   \label{algo:pick_re}
\end{algorithmic}
\end{algorithm}

To ensure a table is safe, we must make sure all \qids in $Q$ are safe.
\PartialSuppressor (Algorithm \ref{algo:partialsuppressor}) presents the
top-level algorithm. The partial suppressor iterates over the table $T$, and
for each record $T[i]$, the algorithm first generates  \qids from $T[i]$ and
sanitizes the unsafe ones. The suppressor terminates when the whole table is
scanned and there is no unsafe \qid.
%This termination condition ensures the algorithm achieves at
%least a local {\em optimum} in terms of minimizing the suppressions.

A \qid can be considered as a combination of different item types,
  and the number of distinct \qids is in exponential scale.
So one of the most time-consuming phases in partial suppression is
  the generation of distinct \qids.
An ideal solution to reduce the time cost in \qid generation is to enumerate
all distinct \qids in main memory. But this is impractical since the number
of distinct \qids can be prohibitive. We therefore introduce a \qid buffer of
capacity $\bmax$ to balance the space consumption with the generation time.
The value of $\bmax$ is significant. Small $\bmax$ values will cause
repetitive generation of \qids, while large $\bmax$ values will cause useless
generation of to-be non-existing \qids. Effects of different $\bmax$ values are
shown in Section \ref{sec:eval}.

As described in Algorithm \ref{algo:partialsuppressor}, starting from the
current record $i$, the algorithm repeatedly enumerates \qids in $T[i]$ to
fill the buffer and increments $i$ until $B$ is full (Lines
\ref{algo:enu_s}-\ref{algo:enu_e}).
% In algorithm \ref{algo:partialsuppressor}
%while buffer $B$ is not full, it recursively generates \qids from the record
%$T[i]$ (Lines \ref{algo:enu_s}-\ref{algo:enu_e}).
For each $q$ generated (Line \ref{algo:enumerate1}-\ref{algo:enumerate2}), if
it is already in $B$, increment $sup(q)$. Otherwise, insert $q$ into the
buffer and set $sup(q)$ to $1$. At the same time, update $\linked(q)$ with
the complementary sensitive set (only sensitive items) of $q$ in $T[i]$.
%When $B$ is full, line
%\ref{algo:update} in algorithm \ref{algo:partialsuppressor} is invoked to
%scan $T$ from index $i$ to $j$, and update $K(\cdot)$ and $L(\cdot)$.
Furthermore, in \qid generation we utilize a fixed size cache to remember \qids
whose linked items $\linked(\cdot)$ is empty, to prevent generating \qids that
are superset of them. It's a pruning technique we can leverage
 since if a short \qid does not link to sensitive items in $T$, neither do its supersets.
After that, if $B$ contains an unsafe \qid, we call \SanitizeBuffer to
suppress items from $T$ so that the unsafe \qid becomes safe.
\textcolor{red}{add something about the added statement in Algorithm 1}

\subsection{Buffer Sanitization}
\label{sec:sanitize} Each time \qid buffer $B$ is ready, \SanitizeBuffer
(Algorithm \ref{algo:sanitize}) is invoked to start processing \qids in $B$
and make all of them safe. We first partition \qids in $B$ into two groups,
{\em safe} and {\em unsafe} according to Definition \ref{def:probability} and
\ref{def:safety_qid}.
 Then in each iteration (Lines \ref{algo:pick_rs}-\ref{algo:pick_re}), \SanitizeBuffer
 picks the `best` (according to below heuristic functions) strong sensitive association
 rule %whose confidence is above $\rho$
 to sanitize (Lines \ref{algo:heur_dist} and \ref{algo:heur_mine}).

As we mentioned before we target to preserve the original data distribution
or retain mineable useful association rules with limited spurious rules
invented in the anonymized data, while minimizing the item deletions. By
noticing the reason why global suppression does not introduce spurious rules,
we introduce the following heuristic function to imitate global suppression
in order to avoid introducing spurious rules, while minimizing
item deletions.
\subsubsection{Spurious Rule Impediment}

%Then we introduce our second heuristic function which trys to retain minable
%rules with fewer spurious rules invented.
As we mentioned before, to learn from the characteristic of global
suppression - no spurious rules introduced, we devise a heuristic which takes
deletions towards global suppression while using partial suppression. We
define \[H_{mine}=residual(t)*MS(t, \SA(q,e))\] as the heuristic
function which prevents introducing spurious rules.
This product expresses
the strategy that we want to introduce fewer spurious rules by imitating the
effect of global suppression while suppressing as few items as possible.
We try to maintain the support
of some items and suppress some items as many as possible to
satisfy our privacy model. Notice that a spurious rule ($\mathcal{A}(q,e)$) is introduced when
 $sup(q)$ becomes small and if $sup(q)$ is small enough to ignore, $\mathcal{A}(q,e)$
will not be learned even $conf(q,e)$ larger than a certain threshold.
 Therefore, we are prone to
suppress those items which have been suppressed before to minimize the
support of spurious rules. As
you can see, each time we choose the sensitive association rule with lowest
$H_{mine}$ to perform sanitization (Line \ref{algo:heur_mine}).

However, $H_{mine}$ goes the opposite side of preserving the original data
distribution. Therefore, we introduce the following heuristic function
separately to preserve the original data distribution and minimizes the
deletion in local optimal.
\subsubsection{Minimum Distance}
%We first present the heuristic function which helps to preserve data
%distribution.
Consider a sensitive association rule $\SA(q,e)$, where $e$ is a
sensitive item and $conf(q,e) > \rho$.
%$q$ is not a safe \qid following the
$\SA(q,e)$ is not a safe rule following the
 Definition \ref{def:safety_rule}. To make $conf(q,e)$ not
above $\rho$,
%we must make the confidence
%of the inference not larger than $\rho$, thus
we should choose an item type $t\in q\cup \{e\}$ to suppress from
$\container(q\cup \{e\})$.
%$\mathcal{A}(q,e)$ and decide the minimum number of occurrences of $t$ to be
%deleted from  $\mathcal{A}(q,e)$ or just eliminate this inference from $T$.
Kullback-Leibler divergence is defined as
 \[H(Q||P)=\sum_{t\in
D}Q(t)log\frac{Q(t)}{P(t)}\]
where P(t) is the original distribution of $t$
and $Q(t)$ is the current
% (before selecting this removal)
 distribution of $t$, which is often used to characterize the distribution
 distance.
Noticing that by suppressing some of item $t$s when $Q(t)>P(t)$, it may make
the K-L divergence smaller, thus we define
\[H_{dist}=\frac{Q(t)log\frac{Q(t)}{P(t)}}{MS(t, \SA(q,e))}\]
 as the heuristic
function which helps to preserve data distribution. The ratio expresses the
strategy that picking an item type $t$ to suppress which maximally recovers
the data distribution and minimizes the deletion in local optimal.
 The numerator indicates the divergence of the data distribution on $t$.
 The larger the absolute value is,
the larger distribution difference of $t$ now is compared with the original
situation. However, if the value is less than 0, i.e. $Q(t)<P(t)$, we'd
better not suppress $t$, since it may further decreases $Q(t)$ and
deteriorates the data distribution. Therefore the larger the numerator is,
the item has more priorities to be chosen to suppress.
%Furthermore, if the value is positive, applying suppression will
%improve the data distribution. So the removal of $t$ is urgent and valuable
%when this value is large.
% However, if the numerator is less than 0, we better not suppress $i$ since
%$Q(i)$ has already been less than $P(i)$.
The denominator indicates the minimum suppression. The smaller the number is,
the fewer items are suppressed. Each time we choose sensitive association
with the highest $H_{dist}$ to perform sanitization (Line
\ref{algo:heur_dist}).


%
% By iterating over all unsafe \qids, we can all sensitive
%inferences whose confidence larger than $\rho$ from those {\em unsafe} \qids.
%Then in each iteration, the algorithm fetches one sensitive association
%$\SA(q,e)$ and fix it.

 Depending on whether the downstream utilities require to preserve
 data distribution or do mining tasks,
 \SuppressionPolicy (Line \ref{choose_heur}) in \SanitizeBuffer determines
 which heuristic function should be used.
% Next we will show the two heuristic
% functions that help to find the `best` association to fix.

After suppressing the item $d$, unsafe \qids may become safe, while safe ones
may become unsafe again, an undesirable situation known as {\em regression}.
Algorithm \SanitizeBuffer Line \ref{algo:update_kl} determines the set of
\qids that would be affected by suppressing $d$. This step is like the $qid$
generation step in Algorithm \PartialSuppressor Line \ref{algo:enumerate1}
and \ref{algo:enumerate2}.
 There are also different ways to pick a
record from $\container(q\cup \{e\})$ to suppress item $d$ (Line
\ref{pick_row}), currently for simplicity we just use the randomized way.

\subsection{Implementation Specifics}
\label{algo:impmentation}
The partial suppressor is a framework under which our algorithm
implementation relies on the following two approximation techniques to make
the performance practically acceptable.
%. Although
%the termination of the algorithm is proved in the analysis section, the time
%performance doesn't seem very good.
%Therefore, we introduce two approximation
%method to improve the time performance but maintain the quality of the
%result.

\subsubsection{Handle Long Records}
%We notice that the result has a property that the sensitive items don't exist
%in a long record in the final result. Such property indicates that the
%algorithm converges very quickly when it handles a long record. However, the
% \qids generated by long records are enormous and most of them are useless
%since sensitive items will be suppressed after handling only a small amount
%of those  \qids. Therefore we decide to handle long records separately. First,
%we consider only sensitive items in that record versus all sensitive items in
%$T$ as an optimization. Second, we only enumerate some  \qids, e.g,1000, and
%sanitize them. Third, when there are no sensitive items in this record, we do
%not consider this record anymore (We don't need to generate  \qids from this
%record) because those  long \qids are not expected frequently in the dataset
%which means that they hardly contribute to the support of association rules.
%However, if  the suppressor cannot eliminate all sensitive items by only generating
%\qids combined with all sensitive items in one long record, we should
%still generate \qids from that record.
A record $R$ has as many as $2^{|R|}$ \qids. Suppose $\bmax = 10^6$, then a
single record of length $20$ can fill up $B$. However, long \qids are not
expected frequently in the dataset, and thus can be handled separately. Since
we can't possibly enumerate all the \qids within a long record, we do this in
batches. Because the number of sensitive items are limited in any record,
sanitizing the first batch of \qids in a long record may already cause the
removal of all the sensitive items in that record. A record without sensitive
items doesn't contribute any unsafety to sensitive association rules, on the
contrary provides some safety , and hence no further actions are needed for this long
record.

There are three implementation issues in handling long records. First, each
time when {\em batch size of } \qids are generated, it will be costly to scan
the whole table to determine the related $sup(\cdot)$ and
$\linked(\cdot)$. Instead we use the $\container(\cdot)$ of each item in
\qid, and calculate the intersection of all the containers to determine the
$sup(\cdot)$ and $\linked(\cdot)$. This is much faster than the whole
table-scanning method when \qid is not a frequent itemset in $T$.

 Second, when computing
$\linked(q)$ in a long record, we consider only sensitive items in that
record versus all sensitive items in $T$ as an optimization. This is correct
because if there is a sensitive item $e$ in another record $R$ which is
linkable by $q$, update of $L(q)$ and $S(q)$ for $e$ can be done when we
handle $R$ later.

 Third, our algorithm provides a preprocessing option that
suppresses all sensitive items from records whose size is regarded large
enough.
%
%
\subsubsection{Heuristic Approximation}
In Algorithm \SanitizeBuffer Lines \ref{algo:heur_dist} and
\ref{algo:heur_mine}, in each iteration it is very time-consuming to iterate
over all strong sensitive association rules to find the `best` one with
maximal $H_{dist}$ or minimal $H_{mine}$ to sanitize, since the number of
\qids are enormous and each \qid can link to various sensitive items.
%For each $\SA(q,e)$, we have to calculate the highest
%$H_{dist}$ or the lowest $H_{mine}$.
While the iteration number are also propositional to the number of \qids, so
the cost is tremendous. Therefore, we use an approximation way which randomly
pick small amount of \qids in the buffer, e.g 1,000, and calculate the
maximal $H_{dist}$ or the minimal $H_{mine}$ among them to find the locally
`best` one to sanitize.
%We pick the candidate item by a local greedy algorithm.
%
%The two approximation strategies do improve the time performance dramatically
%but may deteriorate the result. However, we conduct experiments to justify
%that even by using such approximations, our result is much better than the
%previous work.\XH{need Kenny revise}

\subsection{Speedup By Divide-and-Conquer}
\label{subsec:speedup}
%\subsection{Optimization in Time By Divide-and-Conquer}
%\KZ{Let's not talking about experiments here. We haven't done any experiments yet
%in this section. Just say ``when execution time is of concern, we use the
%following divide-and-conquer framework to speed up the execution''. And
%avoid saying "deeply go".}
%Although our algorithm can terminate within an acceptable time,e,g,2 hours,for most datasets
% by using the two approximation strategies, it is still too long for practical issues.
The above \textbf{basic algorithm} incurs cost which relies on the average
transaction size, the table size, and the domain size. When data is very
large and execution time is of concern, we can use a divide-and-conquer (DnC)
framework that partitions the input data dynamically, runs \PartialSuppressor
on them individually and combines the results in the end. This approach is
correct in the sense that of each suppressed partition is safe, so is the
combined data (See Lemma \ref{CorrectnessOfPartitioning} in Section
\ref{sec:analysis}). This approach also gives rise to the parallel execution
on multi-core or distributed environments which provides further speed-up
(this will be shown in Section \ref{sec:eval}).

\begin{algorithm}
%\caption{Top level partitioning controller}
\caption{$\SplitData(T, D_S, \rho, \bmax, \tmax)$} \label{algo:splitdata}
\begin{algorithmic}[1]
    \IF { $Cost(T) > \tmax$ }
        \STATE Split $T$ equally into $T_1$ , $T_2$
        \STATE $\SplitData(T_1, D_S, \rho, \bmax, \tmax)$
        \STATE $\SplitData(T_2, D_S, \rho,  \bmax, \tmax)$
    \ELSE
        \STATE $\PartialSuppressor(T, D_S, \rho, \bmax)$
    \ENDIF
\end{algorithmic}
\end{algorithm}

The divide-and-conquer algorithm is shown in Algorithm \ref{algo:splitdata}.
The main idea is that we split the input table whenever the estimated cost of
suppressing that table is greater than a predefined parameter $\tmax$. And
for simplicity we are just using an randomized two-equal-size-partition
splitting on the original data when the estimated time is above $\tmax$.
Equation (\ref{eq:costfunc}) is defined to estimate that cost.

\begin{equation}\label{eq:costfunc}
Cost(T)=\frac{|T|*2^{\frac{N}{|T|}}}{D}
\end{equation}
%where $\mathcal{AVG}$ is the average transaction size, i.e. $\frac{N}{|T|}$,
where $N$ is the total number of items in $T$. The function estimates the
average number of \qids per item type. The larger this value is, the more
sensitive association rules we should handle. The cost function is used when
data size is large enough. We claim that in cases when $|T|$ is relatively
small there is no need to apply DnC.

\subsection{Analysis}
\label{sec:analysis}

% Analyze the main algo: time complexity, space complexity.
% Some properties to consider:
%
% \begin{itemize}
% \item give a bound on the total number of items suppressed;
% \item give a bound on the deviation in distribution from the original data;
% \item give a bound on the number of association rules that we eliminate;
% \item and what else??
% \end{itemize}

In this section, we present several theorems and lemmas along with their proofs
in order to provide an all-aspect analysis of the problem and our algorithm.

%\begin{theorem}
%  The Optimal Suppression Problem in Definition \ref{def:osp} is NP-hard.
%\end{theorem}
%% TODO prove the whole problem hierarchy
%\begin{proof}
%  TODO
%\end{proof}
%
%\begin{lemma}
%\label{lemma:rule}
%  If the inference $\mathcal{A}(q,a)$ is safe,
%  then  $\mathcal{A}(q,a, b_1,b_2,\dots,b_n)$ is safe for any sequence of $\{b_i\}$.
%\end{lemma}
%\begin{proof}
%  \begin{align*}
%    \text{$q\rightarrow a$ is safe}
%    \Rightarrow
%    &\, \frac{\csize(q\cup\{a\})}{\csize(q)} \le \rho \\
%    \Rightarrow
%    &\, \csize(q\cup\{a\}) \le \rho\cdot\csize(q)
%  \end{align*}
%  \begin{align*}
%    &\, (q\cup\{a\}) \subset (q\cup\{a, b_1,b_2,\dots,b_n\}) \\
%    \Rightarrow
%    &\,  \csize(q\cup\{a, b_1,b_2,\dots,b_n\}) \leq \csize(q\cup\{a\}) \le \rho\cdot\csize(q) \\
%    \Rightarrow
%    &\, \frac{\csize(q\cup\{a, b_1,b_2,\dots,b_n\}}{\csize(q)} \le \rho \\
%    \Rightarrow
%    &\, \text{$q\rightarrow a, b_1,b_2,\dots,b_n$ is safe}
%  \end{align*}
%\end{proof}
%
%Lemma \ref{lemma:rule} shows that we do not have to consider rules with consequent of length 2 or longer.

\begin{lemma}%[Correctness of partitioning]
\label{CorrectnessOfPartitioning}
  If $q$ is safe in both $T_1$ and $T_2$, then $q$ is safe in $T = T_1 \cup T_2$.
\end{lemma}
\begin{proof}
For any item $a$,
  \begin{align*}
   q~\text{is safe in}~T_1 &\Rightarrow sup_{T_1}(q\cup\{a\}) \le \rho\cdot sup_{T_1}(q) \\
   q~\text{is safe in}~T_2 &\Rightarrow sup_{T_2}(q\cup\{a\}) \le \rho\cdot sup_{T_2}(q)
  \end{align*}
  So \begin{align*}
   sup_{T_1}(q\cup\{a\}) + sup_{T_2}(q\cup\{a\}) &\le \rho\cdot sup_{T_1}(q) + \rho\cdot sup_{T_2}(q)
  \end{align*}
  And \begin{align*}
    sup_T(q\cup\{a\}) &= sup_{T_1}(q\cup\{a\}) + sup_{T_2}(q\cup\{a\}) \\
    sup_T(q) &= \csize_{T_1}(q) + sup_{T_2}(q)
  \end{align*}
  So $$ \frac{sup_T(q\cup\{a\})}{sup_T(q)} \le \rho~\Rightarrow q~\text{is safe in}~T .$$
\end{proof}

\begin{theorem}
\label{CorrectnessOfPartialSuppressor}
  \PartialSuppressor always terminates with a correct solution.
\end{theorem}
\begin{proof}
We first prove that if the algorithm terminates, the suppressed table is safe.
Note that the algorithm can only terminate on Line \ref{algo:partialbreak}
  in Algorithm \ref{algo:partialsuppressor}.
  Therefore, two conditions must be satisfied. First the record cursor $i$ should
  exceed the table size $|T|$. Second the value $Safe$ must be \TRUE. $Safe$ is true
  if and only if there is no unsafe \qids in the table, otherwise  Line \ref{algo:sanitize}
 will assign $Safe$ to \FALSE. If $i$ exceed $|T|$ and $Safe$ is \TRUE, the algorithm
must scans the table at least once and doesn't find any unsafe \qids. Hence, the
 suppressed table is safe.


Then we prove that \PartialSuppressor always terminates by measuring the
  number of items left (denoted $l$) in the table after each step of suppression.
Initially, $l=l_0=\sum_{i=1}^{|T|} |T[i]|\le |D| |T|$.
We state that for every invocation of \SanitizeBuffer, Line \ref{line:sanitize-suppress}
  in Algorithm \ref{algo:sanitize} is always executed at least once.
So the value $l$ strictly decreases in a positive integer
when \SanitizeBuffer is invoked.
And before the table becomes safe, \SanitizeBuffer will be invoked for
  every iteration of the loop in Algorithm \ref{algo:partialsuppressor}.
So $l$ strictly decreases for each loop iteration in \PartialSuppressor.
Because $l$ starts from a finite number which is at most $l_0=\sum_{i=1}^{|T|} |T[i]|$,
  \PartialSuppressor must terminate.
Otherwise there will be an infinite descending chain of all the $l$ values.


Now we prove that Line \ref{line:sanitize-suppress} in Algorithm \ref{line:sanitizebuffer}
  is always executed once \SanitizeBuffer is invoked.
Whenever \SanitizeBuffer is invoked, it is guaranteed that there exists
  an unsafe \qid $q\in B$ (see Line \ref{line:containunsafe}  in Algorithm \ref{algo:partialsuppressor}).
$q$ is unsafe so that there always exists an item $e\in\linked(q)$ such that $conf(q,e)>\rho$,
  i.e. \[ \frac{sup(q\cup\{e\})}{sup(q)}>\rho \Rightarrow
   sup(q\cup\{e\})-\rho\cdot sup(q)>0 .\]
For $k$ on Line \ref{line:sanitize-k} in Algorithm \ref{algo:sanitize},
  \[ k = MS(d,\SA(q,e))\]
  and
  \[MS(d,\SA(q,e)) \geq sup(q\cup\{e\})-\lfloor\rho\cdot sup(q)\rfloor \ge 1\]
  as is shown in Definition \ref{minimum}.Therefore,
  it is guaranteed that the number of deletions is at least 1
  because the rule $q\rightarrow e$ is unsafe and there must be some deletions to make it safe.
So $k\ge 1$ on Line \ref{line:sanitize-whilek} for the first time.
Thus the condition is satisfied and Line \ref{line:sanitize-suppress} is executed.
\end{proof}

\begin{corollary}
The divide-and-conquer optimization \SplitData is correct.
\end{corollary}
\begin{proof}
It follows directly from Lemma \ref{CorrectnessOfPartitioning} and
Theorem \ref{CorrectnessOfPartialSuppressor}.
\end{proof}

%\begin{theorem}
%Let %$M = |T|$ be the size of table $T$,
%$l$ be the average record length,
%$c = r_r r_d$ where $r_r$ is the regression rate and $r_d$ is the qid duplicate rate.
%The average time complexity of \PartialSuppressor is
%\[ O(c \cdot 2^l |T|^2 l (\bmax (1-\rho) + l)). \]
%\end{theorem}
%\begin{proof}
%{\small\begin{verbatim}
%  general idea:
%  l1 <- estimate the number of iterations
%    for the loop in algorithm 1 -- assume
%    there is a parameter: regression ratio
%  may have to assume the data in some distribution
%    (e.g. power-law) -- related to Figure 1
%    count the number of invocations of HandleShort
%      --> b_max
%    count the number of invocations of HandleLong
%  l2 <- estimate the number of iterations
%    for the loop in algorithm 3
%  l1+l2 -> the number of invocation of SanitizeBuffer
%  estimate complexity of SanitizeBuffer
%  estimate complexity of line 7 to 13 in algorithm 3
%    (related to distribution in Figure 2)
%  estimate the complexity of UpdateBuffer
%\end{verbatim}}]

%Let $n_1$ be the number of short records,
%$n_2$ be the number of long records,
%$p(i)$ be the probability of a record being length $i$,
%$l_m$ be the maximum record length,
%$t=1-\rho$.
%
%Let $v_1$ be the number of invocations of \HandleShortRecords.
%In the process of generating qids and filling them into the qid buffer $B$,
%duplicates cannot be counted.
%If duplicates are allowed, the number of qids generated by a record is
%just $r=\sum_{i=1}^{\lmin-1} p(i)\cdot\#qid(i)$ where
%$\#qid(i)$ is the expected number of qids generated by a record of length $i$.
%Then $\bmax/r$ records are used to fill the buffer (duplicates allowed).
%So roughly $v_1=\frac{n_1\cdot r}{\bmax r_d}$ times to consider all distinct qids
%in short records, for a single pass.
%
%For \HandleLongRecord, the number of invocations is $v_2=n_2$ if
%we do not take multiple passes of table scanning (i.e., the loop in algorithm 1) into consideration.
%Assume the loop in \HandleLongRecord is iterated for $v_3$ times, then \SanitizeBuffer is
%roughly invoked for $v_1 + v_2 v_3$ times.
%
%There are 4 major parts in the computation. We will consider them one by one.
%
%The first part is Line 4 to 7 in Algorithm 2, since the buffer capacity is $\bmax$,
%the maximum number of iterations here is roughly $\bmax r_d$.
%\HandleShortRecords will be invoked for $v_1$ times, so
%the total time cost by this part of computation is roughly $v_1 \bmax r_d$.
%
%The second part is \UpdateBuffer in Algorithm 2. For the invocation
%$\UpdateBuffer(B, T, i, j, K, L)$, the purpose is to update $K$ and $L$
%by considering qids in records $T[i..j]$ which are also in $B$.
%So a single call invocation of \UpdateBuffer costs roughly $(j-i+1)|B|$.
%Hence, the total time cost by this part of computation is roughly
%$v_1 (|T| - \frac{\bmax}{r}) \bmax$.
%
%The third part is Line 7 to 13 in Algorithm 3. Note that in reality
%the computation from Line 10 to 12 can be done at the same time when
%calculating Line 8. And in the worst case, the total time cost by calculating
%these intersections is $\dnum l_m |T|$. And the total time cost by
%this part of computation is $v_2 v_3 \dnum l_m |T|$.

%The fourth part is all the invocations of \SanitizeBuffer.
%For a single invocation of \SanitizeBuffer, there are two sub-parts to consider.
%The first sub-part is the intersection calculation on Line 5 in Algorithm 4,
%which costs roughly $l |T|$.
%The second sub-part is the computation from Line 14 to 25 in Algorithm 4,
%which costs roughly $k |B|$, where $k$ is determined on Line $9$.
%In the worst case, $k= t |T|$.
%So for a single invocation of \SanitizeBuffer, the time cost is roughly
%$|B| r_r l (|T| L + t |T| |B|)$ where $|B|$ is the size of the buffer.
%Because \SanitizeBuffer is invoked $v_1$ times in \HandleShortRecords,
%with buffer size $\bmax$, and $v_2 v_3$ times in \HandleLongRecord,
%with buffer size $\dnum$,
%the total time cost by this part of computation is roughly
%$v_1 \bmax r_r l (l |T| + t |T| \bmax) + v_2 v_3 \dnum r_r l (l |T| + t |T| \dnum)$.
%
%Summing up these four parts we can get the following time cost.
%\begin{align*}
%  n_1 r
%+ \frac{n_1 (r |T| - \bmax)}{r_d}
%+ \frac{n_1 |T| r l r_r (\bmax t + l)}{r_d} \\
%+ l_m |T| n_2 \dnum v_3
%+ l |T| n_2 \dnum v_3 r_r (l + \dnum t)
%\end{align*}
%By eliminating non-denominating terms, we get the order of \[ O(c \cdot 2^l |T|^2 l (\bmax (1-\rho) + l) ) .\]
%\end{proof}
%
%For a given dataset, the expected time complexity is actually quadratic to the size of the table.

%\begin{theorem}
%  The space complexity of \PartialSuppressor on table $T$ is \[ O(\sum_{i=1}^{|T|} |T[i]| + \bmax) .\]
%\end{theorem}
%\begin{proof}
%Let $N = \sum_{i=1}^{|T|} |T[i]|$, then $N$ is the sum of the numbers
%of all item occurrences.
%This term is easy to explain since the algorithm has to store
%the original table $T$.
%So we only need to consider local data structures
%created in \PartialSuppressor
%and related functions for the term $O(\bmax)$.
%
%For \PartialSuppressor, the most significant memory cost is from the \qid buffer of size $\bmax$.
%For \HandleShortRecords, there is only $O(1)$ extra memory space for loop variables like $j$.
%For \HandleLongRecord, there is also $O(1)$ extra memory cost.
%For \SanitizeBuffer, except for the $O(1)$ memory space for local variables, it also involves
%  the storage of $\linked(\cdot)$ and $\csize(\cdot)$.
%Because all the \qids updated are from the buffer $B$, the total number of \qids being active at any time
%  is no greater than the capacity of the buffer, i.e. $\bmax$.
%In order to keep the information of $\linked(\cdot)$ and $\csize(\cdot)$,
%  there will be $O(\bmax)$ extra memory space used.
%\end{proof}

%\begin{theorem}
%  The algorithm suppresses at most $O(xxx)$ item occurrences on average.
%\end{theorem}
%\begin{proof}
%  TODO
%\end{proof}
%
%\KZ{Say something about the property of regression?}
%
%\KZ{A property for DnC time performance? The experiment seems to show that
%time decreases exponetially with $t_{max}$ for Retail, which is amazing!}

%\begin{property}
%  Idea: distribution similarity ...
%\end{property}
%\begin{proof}
%  TODO
%\end{proof}
