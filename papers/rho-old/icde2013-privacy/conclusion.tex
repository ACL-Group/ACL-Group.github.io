\section{Conclusion}
\label{sec:conclude} Generalization and global suppression, two
state-of-the-art approaches for anonymizing set-valued data, have been
studied in this paper. Generalization preserves the correctness of data, but
gives up the preciseness. The use of a generalization hierarchy imposes the
adoption of the same hierarchy on the data users. Otherwise, rule mining will
not be possible on the anonymized data. Global suppression aims at preserving
the support of the association rules that can be mined from the resulting
data. However, since global suppression typically deletes more data than
necessary to protect privacy, the amount of rules remaining after
anonymization is very limited. We argue that a partial suppression framework
is more general than global suppression in that it deletes just enough items
to satisfy the privacy model.
%At the same time, it deletes items in probabilistic fashion
%so that the resulting data possesses roughly the same statistical
%properties of the original data.
It is true that the partial suppression method does introduce
a small amount of spurious association rules in the data.
But we argue that through the use of a heuristic function, we can
effectively limit the impact of such spurious rules. In addition,
another heuristic which minimizes the KL-Divergence between the
anonymized data and the original data can help preserve the
data distribution.
%losing most of the rules, which is the case in generalization and
%global suppression.
The optimization with divide-and-conquer also effectively
controls the execution time of our algorithm with limited sacrifice in
the solution quality.
