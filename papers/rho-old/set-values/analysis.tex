\section{Analysis}
\label{sec:analysis}

% Analyze the main algo: time complexity, space complexity.
% Some properties to consider:
%
% \begin{itemize}
% \item give a bound on the total number of items suppressed;
% \item give a bound on the deviation in distribution from the original data;
% \item give a bound on the number of association rules that we eliminate;
% \item and what else??
% \end{itemize}

In this section, we present several theorems and lemmas along with their proofs
in order to provide an all-aspect analysis of the problem and our algorithm.

%\begin{theorem}
%  The Optimal Suppression Problem in Definition \ref{def:osp} is NP-hard.
%\end{theorem}
%% TODO prove the whole problem hierarchy
%\begin{proof}
%  TODO
%\end{proof}

\begin{lemma}
\label{lemma:rule}
  If the rule $q\rightarrow a$ is safe,
  then $q\rightarrow a, b_1,b_2,\dots,b_n$ is safe for any sequence of $\{b_i\}$.
\end{lemma}
\begin{proof}
  \begin{align*}
    \text{$q\rightarrow a$ is safe}
    \Rightarrow
    &\, \frac{\csize(q\cup\{a\})}{\csize(q)} \le \rho \\
    \Rightarrow
    &\, \csize(q\cup\{a\}) \le \rho\cdot\csize(q)
  \end{align*}
  \begin{align*}
    &\, (q\cup\{a\}) \subset (q\cup\{a, b_1,b_2,\dots,b_n\}) \\
    \Rightarrow
    &\,  \csize(q\cup\{a, b_1,b_2,\dots,b_n\}) \leq \csize(q\cup\{a\}) \le \rho\cdot\csize(q) \\
    \Rightarrow
    &\, \frac{\csize(q\cup\{a, b_1,b_2,\dots,b_n\}}{\csize(q)} \le \rho \\
    \Rightarrow
    &\, \text{$q\rightarrow a, b_1,b_2,\dots,b_n$ is safe}
  \end{align*}
\end{proof}

Lemma \ref{lemma:rule} shows that we do not have to consider rules with consequent of length 2 or longer.

\begin{lemma}%[Correctness of partitioning]
\label{CorrectnessOfPartitioning}
  If $q$ is safe in both $T_1$ and $T_2$, then $q$ is safe in $T = T_1 \cup T_2$.
\end{lemma}
\begin{proof}
For any item $a$,
  \begin{align*}
   q~\text{is safe in}~T_1 &\Rightarrow \csize_{T_1}(q\cup\{a\}) \le \rho\cdot\csize_{T_1}(q) \\
   q~\text{is safe in}~T_2 &\Rightarrow \csize_{T_2}(q\cup\{a\}) \le \rho\cdot\csize_{T_2}(q)
  \end{align*}
  So \begin{align*}
   \csize_{T_1}(q\cup\{a\}) + \csize_{T_2}(q\cup\{a\}) &\le \rho\cdot\csize_{T_1}(q) + \rho\cdot\csize_{T_2}(q)
  \end{align*}
  And \begin{align*}
    \csize_T(q\cup\{a\}) &= \csize_{T_1}(q\cup\{a\}) + \csize_{T_2}(q\cup\{a\}) \\
    \csize_T(q) &= \csize_{T_1}(q) + \csize_{T_2}(q)
  \end{align*}
  So $$ \frac{\csize_T(q\cup\{a\})}{\csize_T(q)} \le \rho~\Rightarrow q~\text{is safe in}~T .$$
\end{proof}

\begin{theorem}
\label{CorrectnessOfPartialSuppressor}
  \PartialSuppressor always terminates with a correct solution.
\end{theorem}
\begin{proof}
We first prove that if the algorithm terminates, the suppressed table is safe.
Note that the algorithm can only terminates on Line \ref{line:partial-suppressor-break}
  in Algorithm \ref{algo:partialsuppressor}.
So the value of $u$ on Line \ref{line:partial-suppressor-if-u} must always be \FALSE
  until the record cursor $i$ exceeds the table size $|T|$.
That means both \HandleShortRecords and \HandleLongRecord always return
  a pair with \FALSE as the first element during some pass of scanning of the whole table.
For \HandleShortRecords, returning \FALSE on Line \ref{line:handle-short-return-false}
  in Algorithm \ref{algo:handleshort} indicates there is no unsafe\qid in the buffer $B$.
For \HandleLongRecord, returning \FALSE on Line \ref{line:handle-long-return}
  in Algorithm \ref{algo:handlelong} indicates all the\qids generated by \Enum are safe.
So these return values of the two functions indicate there is no unsafe\qid in the table.
Hence, the suppressed table is safe.

Then we prove that \PartialSuppressor always terminates by measuring the
  number of items left (denoted $l$) in the table after each step of suppression.
Initially, $l=l_0=\sum_{i=1}^{|T|} |T[i]|\le |D| |T|$.
We state that for every invocation of \SanitizeBuffer, Line \ref{line:sanitize-suppress}
  in Algorithm \ref{algo:sanitize} is always executed at least once.
So the value $l$ strictly decreases when \SanitizeBuffer is invoked.
And before the table becomes safe, \SanitizeBuffer will be invoked for
  every iteration of the loop in Algorithm \ref{algo:partialsuppressor}.
So $l$ strictly decreases for each loop iteration in \PartialSuppressor.
Because $l$ starts from a finite number which is at most $l_0=\sum_{i=1}^{|T|} |T[i]|$,
  \PartialSuppressor must terminate.
Otherwise there will be an infinite descending chain of all the $l$ values.

Now we prove that Line \ref{line:sanitize-suppress} in Algorithm \ref{algo:sanitize}
  is always executed once \SanitizeBuffer is invoked.
Whenever \SanitizeBuffer is invoked, it is guaranteed that there exists
  an unsafe\qid $q\in B$ (see Line \ref{line:handle-short-if-contains-unsafe} in Algorithm \ref{algo:handleshort}
  and Line \ref{line:handle-long-if-contains-unsafe} in Algorithm \ref{algo:handlelong}).
$q$ is unsafe so that there always exists an item $e\in\linked(q)$ such that $P(e|q)>\rho$,
  i.e. \[ \frac{\csize(q\cup\{e\})}{\csize(q)}>\rho \Rightarrow \csize(q\cup\{e\})-\rho\cdot\csize(q)>0 .\]
For $k$ on Line \ref{line:sanitize-k1} in Algorithm \ref{algo:sanitize},
  \[ k = |X|-\lfloor\rho\cdot\csize(q)\rfloor = \csize(q\cup\{e\})-\lfloor\rho\cdot\csize(q)\rfloor \ge 1 .\]
For $k$ on Line \ref{line:sanitize-k2} in Algorithm \ref{algo:sanitize},
  it is guaranteed that the number of deletions is at least 1
  because the rule $q\rightarrow e$ is unsafe and there must be some deletions to make it safe.
So $k\ge 1$ on Line \ref{line:sanitize-while-k} for the first time.
Thus the condition is satisfied and Line \ref{line:sanitize-suppress} is executed.
\end{proof}

\begin{corollary}
The divide-and-conquer optimization \SplitData is correct.
\end{corollary}
\begin{proof}
It follows directly from Lemma \ref{CorrectnessOfPartitioning} and 
Theorem \ref{CorrectnessOfPartialSuppressor}. 
\end{proof}

\begin{theorem}
Let %$M = |T|$ be the size of table $T$, 
$l$ be the average record length, 
$c = r_r r_d$ where $r_r$ is the regression rate and $r_d$ is the qid duplicate rate.
The average time complexity of \PartialSuppressor is 
\[ O(c \cdot 2^l |T|^2 l (\bmax (1-\rho) + l)). \]
\end{theorem}
\begin{proof}
%{\small\begin{verbatim}
%  general idea:
%  l1 <- estimate the number of iterations 
%    for the loop in algorithm 1 -- assume 
%    there is a parameter: regression ratio
%  may have to assume the data in some distribution 
%    (e.g. power-law) -- related to Figure 1
%    count the number of invocations of HandleShort
%      --> b_max
%    count the number of invocations of HandleLong
%  l2 <- estimate the number of iterations 
%    for the loop in algorithm 3
%  l1+l2 -> the number of invocation of SanitizeBuffer
%  estimate complexity of SanitizeBuffer
%  estimate complexity of line 7 to 13 in algorithm 3 
%    (related to distribution in Figure 2)
%  estimate the complexity of UpdateBuffer
%\end{verbatim}}]

Let $n_1$ be the number of short records,
$n_2$ be the number of long records,
$p(i)$ be the probability of a record being length $i$,
$l_m$ be the maximum record length, 
$t=1-\rho$.

Let $v_1$ be the number of invocations of \HandleShortRecords.
In the process of generating qids and filling them into the qid buffer $B$, 
duplicates cannot be counted. 
If duplicates are allowed, the number of qids generated by a record is 
just $r=\sum_{i=1}^{\lmin-1} p(i)\cdot\#qid(i)$ where
$\#qid(i)$ is the expected number of qids generated by a record of length $i$.
Then $\bmax/r$ records are used to fill the buffer (duplicates allowed).
So roughly $v_1=\frac{n_1\cdot r}{\bmax r_d}$ times to consider all distinct qids 
in short records, for a single pass.

For \HandleLongRecord, the number of invocations is $v_2=n_2$ if 
we do not take multiple passes of table scanning (i.e., the loop in algorithm 1) into consideration. 
Assume the loop in \HandleLongRecord is iterated for $v_3$ times, then \SanitizeBuffer is 
roughly invoked for $v_1 + v_2 v_3$ times. 

There are 4 major parts in the computation. We will consider them one by one.

The first part is Line 4 to 7 in Algorithm 2, since the buffer capacity is $\bmax$, 
the maximum number of iterations here is roughly $\bmax r_d$. 
\HandleShortRecords will be invoked for $v_1$ times, so 
the total time cost by this part of computation is roughly $v_1 \bmax r_d$. 

The second part is \UpdateBuffer in Algorithm 2. For the invocation 
$\UpdateBuffer(B, T, i, j, K, L)$, the purpose is to update $K$ and $L$ 
by considering qids in records $T[i..j]$ which are also in $B$. 
So a single call invocation of \UpdateBuffer costs roughly $(j-i+1)|B|$.
Hence, the total time cost by this part of computation is roughly 
$v_1 (|T| - \frac{\bmax}{r}) \bmax$.

The third part is Line 7 to 13 in Algorithm 3. Note that in reality 
the computation from Line 10 to 12 can be done at the same time when 
calculating Line 8. And in the worst case, the total time cost by calculating
these intersections is $\dnum l_m |T|$. And the total time cost by
this part of computation is $v_2 v_3 \dnum l_m |T|$.

The fourth part is all the invocations of \SanitizeBuffer.
For a single invocation of \SanitizeBuffer, there are two sub-parts to consider. 
The first sub-part is the intersection calculation on Line 5 in Algorithm 4, 
which costs roughly $l |T|$.
The second sub-part is the computation from Line 14 to 25 in Algorithm 4,
which costs roughly $k |B|$, where $k$ is determined on Line $9$. 
In the worst case, $k= t |T|$. 
So for a single invocation of \SanitizeBuffer, the time cost is roughly
$|B| r_r l (|T| L + t |T| |B|)$ where $|B|$ is the size of the buffer.
Because \SanitizeBuffer is invoked $v_1$ times in \HandleShortRecords,
with buffer size $\bmax$, and $v_2 v_3$ times in \HandleLongRecord,
with buffer size $\dnum$, 
the total time cost by this part of computation is roughly
$v_1 \bmax r_r l (l |T| + t |T| \bmax) + v_2 v_3 \dnum r_r l (l |T| + t |T| \dnum)$. 

Summing up these four parts we can get the following time cost.
\begin{align*}
  n_1 r 
+ \frac{n_1 (r |T| - \bmax)}{r_d} 
+ \frac{n_1 |T| r l r_r (\bmax t + l)}{r_d} \\
+ l_m |T| n_2 \dnum v_3 
+ l |T| n_2 \dnum v_3 r_r (l + \dnum t)
\end{align*}
By eliminating non-denominating terms, we get the order of \[ O(c \cdot 2^l |T|^2 l (\bmax (1-\rho) + l) ) .\]
\end{proof}

For a given dataset, the expected time complexity is actually quadratic to the size of the table.


\begin{theorem}
  The space complexity of \PartialSuppressor on table $T$ is \[ O(\sum_{i=1}^{|T|} |T[i]| + \bmax) .\]
\end{theorem}
\begin{proof}
Let $N = \sum_{i=1}^{|T|} |T[i]|$, then $N$ is the sum of the numbers 
of all item occurrences.
This term is easy to explain since the algorithm has to store 
the original table $T$.
So we only need to consider local data structures 
created in \PartialSuppressor
and related functions for the term $O(\bmax)$.

For \PartialSuppressor, the most significant memory cost is from the\qid buffer of size $\bmax$.
For \HandleShortRecords, there is only $O(1)$ extra memory space for loop variables like $j$.
For \HandleLongRecord, there is also $O(1)$ extra memory cost.
For \SanitizeBuffer, except for the $O(1)$ memory space for local variables, it also involves
  the storage of $\linked(\cdot)$ and $\csize(\cdot)$.
Because all the\qids updated are from the buffer $B$, the total number of\qids being active at any time
  is no greater than the capacity of the buffer, i.e. $\bmax$.
In order to keep the information of $\linked(\cdot)$ and $\csize(\cdot)$,
  there will be $O(\bmax)$ extra memory space used.
\end{proof}

%\begin{theorem}
%  The algorithm suppresses at most $O(xxx)$ item occurrences on average.
%\end{theorem}
%\begin{proof}
%  TODO
%\end{proof}
%
%\KZ{Say something about the property of regression?}
%
%\KZ{A property for DnC time performance? The experiment seems to show that
%time decreases exponetially with $t_{max}$ for Retail, which is amazing!}

%\begin{property}
%  Idea: distribution similarity ...
%\end{property}
%\begin{proof}
%  TODO
%\end{proof}

