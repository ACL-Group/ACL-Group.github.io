\documentclass[a4paper]{article}
\usepackage{enumerate}
\usepackage{hyperref}
\usepackage{longtable}
\usepackage{graphicx}

\title{type-1 dead code}
\author{ADAPT lab}

\begin{document}
\maketitle

\section{Problem Definition}
Based on a set of testing programs, we want to find the type1 dead code (code not executed) in Scala Compiler source code. As it's meaningless to say some code is dead without any sample programs, we must first pick a particular set of testing programs. So when you talk about "dead code", you must know you should already have a specific set of testing programs.


\section{Testing Data}
We pick four projects and many individual Scala programs to test our program. Detailed information about them is listed below.
\begin{enumerate}
\item scalastyle [57 files, 307KB]
\begin{description}
\item[Introduction]
•Scalastyle examines your Scala code and indicates potential problems with it.
\end{description}•
\item scala-dddbase-develop [91 files, 390KB]
\begin{description}
\item[Introduction]
•scala-dddbase-develop is a library used to develop applications based on Domain Driven Design advocated by Eric Evans.
\end{description}•
\item scala-stm [107 files, 479KB]
\begin{description}
\item[Introduction]
•scala-stm is a lightweight software transactional memory for Scala, inspired by the STMs in Haskell and Clojure.
\end{description}•
\item Scala-Algorithms [64 files, 287KB]
\begin{description}
\item[Introduction]
•Java to Scala translations of around 50 algorithms from Robert Sedgewick and Kevin Wayne's website for their book Algorithms.
\end{description}•
\item individual programs [162 files, 287KB]
\begin{description}
\item[Introdution]
•Individual scala programs, mostly from scala reference books' code example, covering most basic syntaxes.
\item[Trait]
89 files have compilation errors, including: unclosed comment, type mismatch, String does not take parameters, x expected but x found, not found: type/value/object x, expected class or object definition, x is not a member of xxx, trait Logger is abstract; cannot be instantiated, unclosed string literal, illegal start of declaration, x expected but xxx found, missing arguments for method x in class x, overriding method equals in class Object of type Boolean, x is already defined as trait x, throws.type does not take parameters, method parent overrides nothing, Missing closing brace `\}' assumed here
\end{description}•

\end{enumerate}•


\section{Our Approach}
\subsection{Basic Idea}
In order to find out which part of compiler source code is "dead", we can first find which part is executed when we use scala compiler to compile our testing programs. Then it's easy to solve our problem. So our basic idea is to make compiler able to output the line numbers of code it executes. 

\subsection{Scala Compiler Plugin}
A compiler plugin lets you modify the behavior of the compiler itself without needing to change the main Scala distribution. \url{http://www.scala-lang.org/old/node/140} refers to a web page where you can get some introduction to scala compiler plugin.

/**I will explain our plugin code here, including its advantages...**/

\subsection{Ant}

\subsection{sbt}

\subsection{Tool for Processing Results}


\section{Result}
\begin{table}[!htb]
\begin{tabular}{|l|l|l|l|}
\hline
file name & number of total lines & number of total dead lines & percentage of dead lines \\
\hline
actors & 2908 & 2908 & 1.0\\ \hline
build & 568	& 568 & 1.0\\ \hline
compiler & 72206 & 38021 & 0.527\\ \hline
eclipse	& 0 & 0 & 0\\ \hline
ensime	& 0	& 0 & 0\\ \hline
forkjoin & 0 & 0 & 0\\ \hline
intellij & 0 & 0 & 0\\ \hline
interactive & 2941 & 2941 & 1.0\\ \hline
library & 35750 & 27211 & 0.761\\ \hline
library-aux & 36 & 36 & 1.0\\ \hline
manual & 1460 & 1460 & 1.0\\ \hline
partest-extras & 611 & 611 & 1.0\\ \hline
partest-javaagent & 0 & 0 & 0\\ \hline
reflect & 31039 & 13885	& 0.447\\ \hline
repl & 5193 & 5193 & 1.0\\ \hline
repl-jline & 241 & 241 & 1.0\\ \hline
scaladoc & 7471	& 7471 & 1.0\\ \hline
scalap & 2306 & 2306 & 1.0\\ \hline
\end{tabular}•
\caption{Dead code percentage within files under directory "scala/src"}
\end{table}

\begin{table}[!htb]
\newcommand{\tabincell}[2]{\begin{tabular}{@{}#1@{}}#2\end{tabular}}
\begin{tabular}{|l|l|l|l|}
\hline
file path & \tabincell{c}{number of \\total lines} & \tabincell{c}{number of total\\ dead lines} & percentage of dead lines \\ \hline
\tabincell{c}{scala/src/compiler/scala/tools/\\nsc/typechecker/Typers.scala}	& 4574 & 1595&		0.349 \\ \hline
\tabincell{c}{scala/src/reflect/scala/reflect/\\internal/Types.scala}	& 3172 & 901	&0.284 \\ \hline
\tabincell{c}{scala/src/compiler/scala/tools/nsc/\\symtab/classfile/ICodeReader.scala} &	959 &831 & 0.866 \\ \hline
\tabincell{c}{scala/src/reflect/scala/reflect/\\runtime/JavaMirrors.scala}	& 1034 & 819	&0.792 \\ \hline
\tabincell{c}{scala/src/compiler/scala/tools/nsc/\\backend/jvm/BCodeBodyBuilder.scala}	& 1020 & 787&0.771 \\ \hline
\tabincell{c}{scala/src/compiler/scala/tools/\\nsc/backend/jvm/GenASM.scala}	&2533 &765	&0.302 \\ \hline
\tabincell{c}{scala/src/interactive/scala/tools/\\nsc/interactive/Global.scala}	&1000	&753&	0.753 \\ \hline
\tabincell{c}{scala/src/repl/scala/tools/nsc/\\interpreter/IMain.scala}	&	1026&745	&0.726 \\ \hline 
\tabincell{c}{scala/src/reflect/scala/reflect/\\internal/ReificationSupport.scala}	&	966&735&	0.761 \\ \hline 
\tabincell{c}{scala/src/library/scala/collection/\\parallel/ParIterableLike.scala}	&967&726		&0.751 \\ \hline
\hline
\end{tabular}•
\caption{Ten files with most lines of dead code}
\end{table}

\begin{figure}[!htb]
number of lines of testing programs, number of dead lines in "src/compiler"(blue), number of dead lines in "src/library"(green-high), number of dead lines in "src/reflect"(green-low), total number of dead lines(orange).
\centering
\includegraphics[width=1.3\textwidth]{p1.jpg}
\end{figure}

\end{document}
