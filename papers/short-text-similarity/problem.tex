\section{Problem and Applications}
\subsection{What is short text?}
Short texts are everywhere in the digital world. Examples include
search queries, tweets, SMS text messages, questions on QA systems,
user comments on e-commerce sites, etc. 
Short texts have a few important characteristics that differentiate
them from longer, complete documents. A short text
\begin{itemize}
\item has limited length;
\item has limited number of words;
\item may contain incomplete sentences;
\item may contain abbreviations, special symbols or errors.
\end{itemize}

\begin{example}
A Web Query.
\end{example}
\begin{example}
A Tweet.
\end{example}
\begin{example}
A typical question on Yahoo Answers.
\end{example}

The similarity between two short texts can be measured in two dimensions:
{\em syntactic similarity} and {\em semantic similarity}. 
Syntactic similarity usually refers to common substrings or common
subsequences \cite{} of words in two text. [Need examples] 
It can also be measured by the edit
distance \cite{} between two texts. Semantic similarity can be measured
at the word level (in terms of synonymy), at the concept level or at the 
topic level. [Need examples]

\subsection{Features of similarity}
Semantic similarity is often confused with another term, 
``semantic relatedness'', which actually subsumes the former as a special
case. Two objects can be related to each other for a variety of reasons byt
they are similar to each other because they share {\em common attributes}.
For example, {\em car} and {\em gasoline} are related because the former
cannot operate without the latter as fuel. On the other hand,
{\em car} and {\em bike} are more similar because they are both means of
transportation, they both have wheels and they both can carry people, etc.

\subsection{Major applications}

\subsection{Main approaches}
