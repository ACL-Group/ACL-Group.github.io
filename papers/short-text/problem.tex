\section{Short Text Understanding Problem}
\label{sec:problem}
What do we mean by {\em understanding}? Let's reconsider the previous example
in Section \ref{sec:intro}:

{\em ``Superfresh sells apples and oranges.''}

When we human read this sentence from left to right, using our
prior knowledge, we probably would detect
a few key {\em signal terms} in it, like {\em superfresh}, {\em sell}, 
{\em apple}, {\em orange}, and {\em apples and oranges}. Notice that
the last term overlaps with the previous two terms. This is the result of
different ways of {\em parsing} the sentence. Two of them
are shown below with signal terms underlined:

Parse 1: \underline{Superfresh} \underline{sells} 
\underline{apples} and \underline{oranges}.

Parse 2: \underline{Superfresh} \underline{sells} 
\underline{apples and oranges}.

Most of these terms have one or more meanings,
some of the meanings may be semantically far away from others. For instance,
Superfresh is a grocery chain in eastern US;
{\em apple} may mean a popular US technology company or a fruit; 
{\em orange} may be a fruit, a color, or the name of several counties
in the US; {\em apples and oranges} may be a colloquial term to mean
things that cannot be compared or the title of a Pink Floyd song. 
Depending on our individual knowledge, different people may 
come up with different sets of meanings for each of these terms.
To truly understand this sentence, we have to identify the correct meaning
for each term, a process known as {\em word sense disambiguation (WSD)}
\cite{Navigli09:WSD}. 
Humans are good at this because our knowledge tells us that while a store
cannot sell a colloquial term or a song, but it can sell 
Apple company's product or sell fruits. However it
is not likely to sell IT products and fruits at the same time.  
Therefore the final and correct interpretation of the above sentence is
a grocery store sells two kinds of fruits, apples and oranges.

From the above example, we see that knowledge is indispensable in understanding
human language text. We need three kinds of knowledge: a 
{\em lexicon} or a vocabulary of all possible signal terms (words and phrases), 
a knowledge base that allows one to {\em conceptualize} a term, e.g.
``apple'' can be conceptualized into a company or a fruit, and
finally the {\em relationship} between any two concepts.

The problem of understanding a short text message can be informally 
defined as: given a knowledge base, 
assign the correct concepts to as many terms in the message 
as possible so that these concepts collectively represent a coherent meaning. 
What's more, when we process multiple short texts, we often need to compare 
different text messages for classification purpose. Therefore, besides the 
measure of relatedness between concepts, we also need to compute the
relatedness between any two messages. Specifically, the short text
understanding problem can be broken down into the following
subproblems:

\begin{enumerate}
\item parse a text message into a sequence of terms;

\item associate a concept for each term in the sequence so that the coherence
for the sequence is maximized;

\item given a set of short text messages, classify the messages into different
clusters based on their semantic relatedness. 

%\item defining semantic relatedness(similarity) between two terms; 
%\item defining semantic relatedness(similarity) between two messages.
\end{enumerate}

{\bf 
Kaiqi to fill in: Coherence score ...
}

In the next section, we will present our approach to solve these subproblems.
