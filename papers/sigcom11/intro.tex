\newdef{definition}{Definition}

\section{Introduction}
\label{sec:intro}

Location privacy and, in particular, privacy of 
individuals' timestamped movements 
is receiving increasing interest. 
Recent research indicates that the widespread use of WiFi and Bluetooth
enabled smartphones opens new doors for malicious attacks, 
including geolocating individuals by illegitimate means 
(e.g. spreading worm based malware) and even legitimate means 
(e.g. location based advertisement networks and theft locators) 
\cite{Ma10:PrivacyTraces,HustedM10:Malnets,Constandache10:SeeBob}. 
However, most existing geo-localization techniques require GPS 
information or additional control of hardware infrastructures such 
as WiFi access points or GSM base stations.  
This paper presents a new technique to infer individuals' 
complete movement traces using only their mutual contact histories and a map. 
This technique can be deployed both indoors and outdoors, 
so long as the area map is available. To the best of our knowledge, 
the only other work that attempts to infer traces from 
bluetooth contact histories is by Whitbeck et al. \cite{Whitbeck10:Plausible}.
However, they did not make use of a map and therefore only produces rough
moving trajectories. It also relies on the existence of various imaginary
forces that are supposed to bias the person's movement, though it is not
clear how to determine the parameters involved in the
forces calculation. The technique in this paper, on the contrary, produces
detailed movement traces according to a map.

Next we describe the {\em Trace Inference Problem}.
We define a map $M$ as a graph $(V,E)$, where $V$ is a
set of road junctions each with geographic coordinates $(x,~ y)$,
and $E$ is a set of straight road segment. Note that a curved road
can be approximated by a sequence of straight road segments.
Let a set of nodes $N$ move on the edges of the map at various
but constant speeds.
We assume that a node never backtracks unless it's at a dead end.
Let the trace of node $i$ be a location function on time $l_i(t)$,
and a {\em contact} between node $i$ and $j$ be a 4-tuple:
$(i, j, t_{in}, t_{out})$
where $t_{in}$ is when the encounter of $i$ and $j$ begins, and $t_{out}$ is
when their encounter ends. Further, a {\em contact history} is a set of contacts.
Given the set of traces of $N$, we can {\em induce} all the contacts by solving
inequality $||l_i(t)-l_j(t)|| \leq r$
for all pairs of traces by node $i$ and $j$, where $r$ is the range
in which two nodes are considered in contact.
For simplicity, we assume $r = 0$ in the rest of this paper, and
the contact induction can be computed in $O(|N|^2|V|^2)$ time.

\textbf{Trace Inference Problem(TIP):} Given a map $M$, a set of moving nodes $N$,
their speeds $\{v_i\}$, their initial locations $\{l_i(0)\}$, and a contact history $H$,
find the traces of $N$ whose induced contacts $H_{ind} = H$ .

Suppose the last contact in $H$ is at $t_{max}$, the maximum speed is $v_{max}$,
the length of the shortest edge in $M$ is
$e_{min}$, and largest degree of any vertex on $M$ is $d_{max}$, then
a naive search across all possible paths costs
\[O\left(|N|^2|V|^2 (d_{max}-1)^ {\frac{|N| v_{max} t_{max}}{e_{min}}}\right).\]

