\section{Related Work}

We introduce related works in this section. Section \ref{related_1} discusses user-centric Entity Linking systems and section \ref{related_2} introduces works studying visual features.

\subsection{User-Centric Entity Linking} \label{related_1}

When reading texts, users may encounter unfamiliar entities. The reading experience can be enhanced via Entity Linking, i.e., assigning hyperlinks directing to a given knowledge base \cite{milne2008learning, csomai2008linking,mihalcea2007wikify, cucerzan-2007-large}. Different from most existing Entity Linking works, user-centric Entity Linking systems \cite{von2007leveraging} aim to improve reading experience. In such cases, excessive linking would lead to distraction and degrade the reading experience.

It is thus an important problem to decide whether to link or not to link the entities \cite{guo2013link}. For example, the interestingness of entities can be modeled with deep neural networks \cite{gao2014modeling}. Besides, an Entity Linking system named Linkify was proposed to evaluate the helpfulness of linked entities \cite{yamada2014evaluating, yamada2018linkify}. There are some other works concentrating on key entity prediction. In \cite{kraft2011contextual}, the authors tried to extract the most interesting and relevant keyword phrases from the article. Different from these line of works, we consider the situation where all the hyperlinks are given and the task is to predict the click popularity of all the links. The task of click popularity prediction was studied in \cite{thruesen2016link}. The authors proposed textual and graph-based features for click popularity prediction. Unfortunately, the impact of visual factors was neglected.

\subsection{Visual Factors} \label{related_2}

An empirical study on Wikipedia \cite{lamprecht2017structure} showed that a large share of user clicks comes from links in the lead section or the infobox, suggesting that visual factors can affect users' navigation behavior. Having a better understanding of the impact of visual factors can be instructive to the amelioration of website structure \cite{lamprecht2016evaluating, paranjape2016improving}. For example, we can reinforce some specific links on their website by changing their visual appearance, for example, by locating them on the top of the page \cite{geigl2016assessing}. Besides, predicting hyperlink click popularity by visual factors are also important in online advertising \cite{richardson2007predicting, zhu2017optimized}.

In the limited work that studied visual factors \cite{dimitrov2016visual, dimitrov2017makes}, the Html pages were rendered to study the effect of visual position on hyperlink click popularities. The coordinate positions of hyperlinks are used as a type of visual feature for prediction. However, it requires rendering to obtain the visual features. The visual features highly depend on specific devices and resolution. Given this drawback, it is thus non-trivial to propose render-agnostic visual features which can be obtained without rendering.
