%\documentclass{sig-alternate}%{acm_proc_article-sp}%{sig-alternate}%
\documentclass{vldb}
%\documentclass[10pt,conference,letterpaper]{IEEEtran}
\usepackage{amsmath}
\usepackage{multicol,multirow}
%\usepackage{algorithm}
\usepackage{algorithm}
%\usepackage[ruled,vlined]{algorithm}
%\usepackage{algorithmic}
\usepackage[noend]{algpseudocode}
%\usepackage{hyperref}%[colorlinks, citecolor=blue, hyperindex]
\usepackage{graphicx,subfigure}
\usepackage[normalem]{ulem}
\usepackage{url}
\usepackage{color}
%\usepackage{algorithm2e}
%\usepackage{subfigure}
\graphicspath{{figures/}}
\newcommand{\KZ}[1]{\textcolor{blue}{[KZ: #1]}}
\newcommand{\secref}[1]{Section \ref{#1}}
\newcommand{\figref}[1]{Figure \ref{#1}}
\newcommand{\eqnref}[1]{Eq. (\ref{#1})}
\newcommand{\tabref}[1]{Table \ref{#1}}
\newcommand{\exref}[1]{Example \ref{#1}}
\newcommand{\argmin}{\operatornamewithlimits{argmin}}
\newcommand{\argmax}{\operatornamewithlimits{argmax}}
%\newcommand{\exref}[1]{Example \ref{#1}}
\newtheorem{example}{Example}
\newtheorem{lemma}{Lemma}
\newcommand{\cut}[1]{}
%\usepackage{caption}
%\newfont{\mycrnotice}{ptmr8t at 7pt}
%\newfont{\myconfname}{ptmri8t at 7pt}
%\let\crnotice\mycrnotice%
%\let\confname\myconfname%
%
%\permission{Permission to make digital or hard copies of all or part of this work for personal or classroom use is granted without fee provided that copies are not made or distributed for profit or commercial advantage and that copies bear this notice and the full citation on the first page. Copyrights for components of this work owned by others than ACM must be honored. Abstracting with credit is permitted. To copy otherwise, or republish, to post on servers or to redistribute to lists, requires prior specific permission and/or a fee. Request permissions from Permissions@acm.org.
%}
%\conferenceinfo{CIKM'13,}{October 27 - November 01 2013, San Francisco, CA, USA.}
%\copyrightetc{Copyright 2013 ACM \the\acmcopyr}
%\crdata{978-1-4503-2263-8/13/10\ ...\$15.00.\\
%http://dx.doi.org/10.1145/2505515.2505567}
%
%\clubpenalty=10000
%\widowpenalty = 10000
%
\begin{document}
\title{Action Conceptualization}

\numberofauthors{5} %  in this sample file, there are a *total*
% of EIGHT authors. SIX appear on the 'first-page' (for formatting
% reasons) and the remaining two appear in the \additionalauthors section.
%
\author{
% You can go ahead and credit any number of authors here,
% e.g. one 'row of three' or two rows (consisting of one row of three
% and a second row of one, two or three).
%
% The command \alignauthor (no curly braces needed) should
% precede each author name, affiliation/snail-mail address and
% e-mail address. Additionally, tag each line of
% affiliation/address with \affaddr, and tag the
% e-mail address with \email.
%
% 1st. author
\alignauthor
Kaiqi Zhao\\
%\titlenote{Peipei Li was a student intern in Microsoft Research Asia when the paper was
%developed.}\\
       \affaddr{Shanghai Jiao Tong University}\\
%       \affaddr{Shanghai, China}\\
       \email{kaiqi\_zhao@163.com}
%% 2nd. author
\alignauthor
Zhiyuan Cai\\
       \affaddr{Shanghai Jiao Tong University}\\
%       \affaddr{Shanghai, China}\\
       \email{luckyvega@163.com}
%% 3rd. author
\alignauthor
Yu Gong\\
       \affaddr{Shanghai Jiao Tong University}\\
%       \affaddr{Shanghai, China}\\
       \email{gy910210@gmail.com}
% use '\and' if you need 'another row' of author names
\and
%% 4th. author
\alignauthor
Youer Pu\\
       \affaddr{Shanghai Jiao Tong University}\\
%       \affaddr{Shanghai, China}\\
       \email{puyouer@gmail.com}
%% 5th. author
\alignauthor
Kenny Q. Zhu\\
       \affaddr{Shanghai Jiao Tong University}\\
%       \affaddr{Shanghai, China}\\
       \email{kzhu@cs.sjtu.edu.cn}
}
% There's nothing stopping you putting the seventh, eighth, etc.
% author on the opening page (as the 'third row') but we ask,
% for aesthetic reasons that you place these 'additional authors'
% in the \additional authors block, viz.
%\additionalauthors{Additional authors: John Smith (The Th{\o}rv{\"a}ld Group,
%email: {\texttt{jsmith@affiliation.org}}) and Julius P.~Kumquat
%(The Kumquat Consortium, email: {\texttt{jpkumquat@consortium.net}}).}
%\date{30 July 1999}
% Just remember to make sure that the TOTAL number of authors
% is the number that will appear on the first page PLUS the
% number that will appear in the \additionalauthors section.
\maketitle
\begin{abstract}
Verbs play a central role in both syntax and semantics of natural
language. This paper studies the problem of abstracting a verb and
its arguments into what we call ``action concepts''. An action concept
represents a particular fine-grained usage and semantics of the verb
under certain context. We define this
problem as a combinatorial optimization problem and present two
heuristic solutions. We applied the algorithms
on large web data and automatically extract a lexicon of action concepts
for 200 most frequently used verbs in English language. Such a lexicon
lies between the coarse-grained FrameNet which provides shallow semantics
and ReVerb which are too specific to recognize verb uses which are never
encountered before. Furthermore, our lexicon is suitable for both
human reading and machine computation. We demonstrate several potential
applications of such action concept lexicon and show that
it enables improved performance
in the tasks of text understanding.
\end{abstract}

% A category with the (minimum) three required fields
%\category{H.4}{Information Systems Applications}{Miscellaneous}
%A category including the fourth, optional field follows...
%\category{D.2.8}{Software Engineering}{Metrics}[complexity measures, performance measures]

%\terms{Theory}

% A category with the (minimum) three required fields
%\category{H.3}{INFORMATION STORAGE AND RETRIEVAL}{Miscellaneous}
%%A category including the fourth, optional field follows...
%\category{I.2}{Computing Methodologies}{ARTIFICIAL INTELLIGENCE}[Knowledge Representation Formalisms and Methods]
%
%%\terms{Algorithms,Measurement}
%
%\keywords{Term Similarity; Multi-word Expression; Clustering; Semantic Network} % NOT required for Proceedings
%
%\keywords{ACM proceedings, \LaTeX, text tagging} % NOT required for Proceedings
\section{Introduction}

Protein$-$protein interactions (PPIs) are of central importance for the majority of biological functions, such as signal transduction, metabolic pathways, molecular dynamics, and protein networks\cite{Hoffmann.Krallinger.ea:2005}, for they serve as the most fundamental building blocks of the entire interacademic systems of any organisms. Collecting data on pairwise interaction relationships is essential for multiple purpose, including identification of modules with certain functionality\cite{Spirin.Mirny.03}, mapping diseases to dominated genes\cite{Ideker.Sharan.08}, and after all, understanding wholistic metabolic/genetic networks from a system biology perspective.

A lot of databases have been built to store protein and genetic interactions from major model organism species and are available in various standardized formats, such as MINT\cite{Zanzoni.Montecchi-Palazzi.ea:2002}, BIND\cite{Bader.ea:2003}, BIOGRID\cite{DBLP:journals/nar/StarkBRBBT06}, etc. Among those mainstream databases, the data largely rely on voluntary reports by scientists or researchers, besides, comprehensive curation efforts become indispensable for the sake of accuracy. However, the amount of biology-related literatures with respect to protein interactions grows explosively and thus make it either impossible or impractical to manually detect PPI information anymore.

Considering huge amount of PPI information with great wealth hidden in published papers, in recent years, numerous mining techniques have been proposed that aim to extract PPI information automatically from free text, especially machine learning, information retrieval, and natural language processing\cite{DBLP:journals/bib/WinnenburgWPDS08}.These approaches can be roughly categorized into three classes: co$-$occurrence, rule$-$based, and machine learning. 

Co$-$occurrence is the approach with most simplicity and naivete. Just as its name implies, this method intends to find out pairs of proteins that co-occur in the same context. The scope of "same context" ranges from phrase, sentence, paragraph to whole abstract, even document. The underlying assumption is that whenever two proteins are mentioned together by authors, chances are high that there is some kind of relationship between them. However, however, in-context closeness even semantic relation does not necessarily represent actual biological interaction. As a consequence, a large fraction of candidate pairs are mismatched inevitably, causing a high recall but low precision.

The second approach is rule-based extraction, in other words, pattern matching. There are many types of rules, most of them concern natural language processing (NLP). One way is to specify hand-crafted regular expressions before hand, which mostly lean on language usage preference. Besides, by using full or partial (shallow) parsing strategies, more information would be acquired, such as part-of-speech taggers, local dependencies between syntactic components, context-free grammar\cite{DBLP:journals/bioinformatics/TemkinG03}, and full sentence structure. Compared to co$-$occurrence, rule-based approach enjoy better precision but much lower recall. In addition, since the rules are usually derived from training data, that is to say, the improper choice of training data would be significantly lethal, therefore quality of extraction is invariably instable and may not applicable to other data.

The third and most commonly used approach use machine learning techniques, in this case, the task to extract protein$-$protein interactions turns out to be a binary classification problem. Each protein pairs are represented along with a set of features, which is associated with their context, then a well$-$defined classifier gives the answer whether the candidate protein pairs is classified to be qualified PPI. (TO BE FURTHER FILLED!!!)

In this paper, we introduce a general bootstrapping framework for Protein$-$protein interaction extraction from natural text.Our method differs from most of the previous works in three aspects:

(1)The extraction process is driven by only tiny fraction of training data, which are regarded as seed data. In each round, it would derive reliable patterns automatically from seed data, then extract more positive PPI pairs consequently, what's more, the seed data would be augmented by the newly extracted results with high confidence.

(2)multiple graph kernel. 

(3)various evaluation.




\section{Problem Definition}
\label{sec:problem}

In this section we formally define the problem of short title extraction.
A char is a single Chinese or English character.
A segmented word (or term) $x$ is a sequence of several chars such as 
``Nike'' or ``牛仔裤''(jean).
A product title, denoted as $X$, is a sequence of words $\{x_1, x_2, ..., x_n\}$.
Let $Y$ be a sequence of labels $\{y_1, y_2, ..., y_n\}$ over $X$, where $y_i \in \{0, 1\}$.
The corresponding short title is a subsequence of $X$, denoted as $S = \{x_i\}$, 
where $y_i = 1$ and $|S| \le n$.

%we are interesting in obtaining a short title which can represent the most important information about the product.

We regard short title extraction task as a sequence classification problem.
Each word is sequentially visited in the original product title order
and a binary decision is made.
We do this by scoring each word $x_i$ within $X$ and predicting a label $y_i \in \{0, 1\}$, 
indicating whether the word should or should not be included in the short title $S$.
As we apply supervised training, the objective is to maximize the likelihood of all word labels
$Y=\{y_1,y_2,...,y_n\}$, given the input product title $X$ and model parameters $\theta$:
\begin{equation}
\label{eqn:problem}
\log{p(Y|X,\theta)}=\sum_{i=1}^{n}{\log{p(y_i|X,\theta)}}.
\end{equation}

%Our problem is different from Sequece Labelling problem, as ...

%In a more restrictive scenario, the number of words $m$ in the short title is strictly limited, where $m$ is some fixed number and $m \le \sum_{i=1}^{n} len(x_i)$. $len(x_i)$ is the number of words (chars) in term $x_i$.


\section{Technical Specification}
\label{sec:tech}

\begin{figure}[h]
\begin{center}
\epsfig{file=archi.eps,width=0.85\columnwidth}
\caption{PredicTV System Architecture}
\shrink
\label{fig:archi}
\end{center}
\end{figure}

Figure \ref{fig:archi} illustrates the system architecture.
The core recommendation engine consists of two modules:
the offline web information extraction module and the online
recommendation module. The first module collects in advance weekly TV schedule,
identifies program titles and show times, and then extracts relevant
information about programs from web. 
The second module maintains user viewing model
dynamically and by comparing the similarity between the 
current user model and programs, 
recommends the most relevant programs to the user in real time.
We next present these two modules in greater details.

%Communication module receives users' channel switching requests,
%transfers them to Recommendation module, and outputs
%recommendations to the user. Database module stores all the program
%models and dynamically changing user viewing models. We next focus on
%the other key components: information extraction module and
%recommendation module.
%
%The Information Extraction module has two tasks. One is to crawl HTML pages  
%about TV programs from the internet, another is to extract 
%important attribute-value pairs from the pages. 
%We use Baidu and Google for the first task. 
%%For a program
%%that we want to collect its information, we form an URL based on Baidu
%%and Google's rules. Then we send HTTP requests to both these two sites
%%and get replies. Baidu and Google will reply the URLs of websites most fit
%%the key word we provide, so we send HTTP requests again and get HTML files
%%we need. Though Baidu and Google help us to search the websites most
%%related to our key words, there will still be some noise in the HTML files,
%%so we define some patterns to match information we need.
%
%Recommendation module is the core of our system. It uses users' viewing
%behaviour information as the
%input of its analysis process then replies recommendation results to users
%via Communication module. Since in mainland China, we don't have standard
%and detailed TV program information available, we have to obtain such
%information ourself. That's why Information Extraction module is involved
%in our system. It grabs HTML files from the internet, filters unnecessary
%parts, then passes them to Recommedation module. 

\subsection{Program and Viewer Model}
Before discussing the details of the two modules, we first introduce the
structure of our program and viewer model. A program model is basically a list
of attribute-value pairs. Attribute is the property of 
a program such as director, 
cast of a movie, or host of a talk show.
For certain attribute like cast of a movie, 
the value can be a set of names rather than
a single value.
In order to extract such properties or attribute values, 
we employ a statistical method.
First, we gather the list of programs, and then use 
search engine to find related pages of these programs. 
In the pages returned, we use simple but strict patterns to
match key-value pairs then choose keys which appeared more frequently 
to form an attribute list. 

A viewer model is the accumulation of program models, so the structure
is the same with program model. 
What's different is that in program model, each attribute-value
pair contains a set of values, whereas in viewer model, each value is also
is associated with a weight, which represents how important that 
value is for the owner of the model.

\subsection{Web Information Extraction}
The module first run a crawler to download TV
schedules for the coming week. As long as schedules are ready, 
before we use search engine to find out related pages for programs, 
we need to do some preprocessing on the program
name, because program names in the TV schedule may contain 
noises like type, subtitle and episode number, which affect the results
returned from search engine. After preprocessing, 
the modified program names are put on search engines like Google and Baidu. 
Ideally we will get many related
pages containing the information we need. 
As discussed in previous section, we automatically
form an attribute list for program model. 
For each attribute, we use a simple pattern to match
the context of that attribute and extract values for that attribute.

\subsection{Model Update and Recommendation}
When a viewer turns on TV, she may switch between different channels.
We capture her actions and try to predict her likes and dislikes. 
From the viewing history, we can collect the duration the viewer
views each program. We assume that the longer viewer views one program, 
the more she likes that program. 
We use this criteria to update viewer model. 
When the viewer switch to a new channel, 
a viewing record containing the old channel id, 
timestamp and duration viewer stays on that
channel is sent to server. According to channel id and timestamp, 
we find out what program viewer was previously 
watching and get the program model for viewer model updating. 
We merge program model into viewer model. 
The duration the user has spent on a program is translated into
the weight for each value in viewer model. 
All values in program model will be added to viewer model. 
If viewer model already contains that value, the weight of value will be
increased, otherwise that value is directly added to viewer model. 
We have a restriction on the size of value
set in each attribute. If the set is full, 
we remove value with the lowest weight. 

In the recommendation step, we first get a list of candidate programs to
be recommended. These are typically programs shown right now or in the near 
future. For each program in the list, we calculate the similarity 
between that program model and viewer model using Equation (\ref{eq-sim}),
and we return the top 3 programs as recommendation result. 

\begin{equation}
P_u(p) = \sum_{i=1}^nw_i\delta(v_{u_i},v_{p_i})
\label{eq-sim}
\end{equation}

$\delta(v_{u_i}, v_{p_i})$ is used to calculate the similarity between value of users and value of programs.
For each attribute in the attribute-value list, 
taking into account the different types of attributes, we design three
different implementations of function $\delta(v_{u, i}, v_{p_i})$.

For title, its value is normally a single chinese string. 
Comparing the whole string is not good enough,
therefore we use a segmentation tool to split the whole string 
into several phrases to form a set, and use Jaccord similarity here.

For time, since its value is a range, we can calculate the intersection and union of two time
range. Then divide the length of intersection by the length of union and use the ratio as the similarity.

For other attributes, the value of the attribute is also a set, so we also use Jaccord similarity, but we make some change this
time. When calculating both the intersection and union, for every value in the intersection or union, instead of
adding 1 to the size of intersection or union, we add the weight of that value. So the size here is a weighted size.

In our equation, each attribute similarity has a weight parameter $w_i$. 
Our concern is that
for different users, the effect of one attribute which influences user's choice is different. $w_i$ is
used to define the importance of one attribute to users. For each attribute, 
$w_i$ is the max value weight divided by
the sum of value weight in that attribute.

\section{Experiment}
In this section, we experiment on different NLG tasks. We first present the experimental setup on different tasks. Then, we show the quantitative and qualitative results together with comprehensive analysis and ablation studies.

\subsection{Implementation Details}
We evaluate the newly proposed ICL strategy on five commonly-researched natural language generation tasks: reading comprehension, dialogue summarization, style transfer, question generation and news summarization. Details on the task description, the strong baseline, corresponding  dataset, evaluation metrics and key hyper-parameters for each task are presented as follows.

\begin{table*}[th]
	\scriptsize
	\centering
	\begin{tabular}{lp{1.1cm}rrrcccc}
		\hline
		Task & Dataset & \#Train & \#Val & \#Test & Input & Output & Avg & Std\\
		\hline
		Reading Comprehension & DREAM & 6,116 & 2,040 & 2,041 & ``Q:''+ question + dialogue & answer & 5.59 & 2.61\\
		Dialogue Summarization & SAMSum & 14,732 & 818 & 819 & dialogue & summary  & 24.99 & 13.06\\
		Style Transfer & Shakespeare & 36,790 & 2,436 & 2,924 & original/modern  & modern/original  & 11.63 & 8.19 \\
		Question Generation & SQuAD1.1 & 75,722 & 10,570 & 11,877 & passage + [SEP] + answer & question & 13.09 & 4.27 \\
		News Summarization & CNNDM & 287,227& 13,368& 11,490 & document & summary & 70.97 & 29.59\\ 
		\hline
	\end{tabular}
	\caption{A summary of tasks and datasets. \#Train, \#Val and \#Test refers to the number of samples in the corresponding dataset. Avg and Std are the statistics for the number of output tokens. ``+'' refers to the concatenation operation.}
	\label{tab:taskdata}
\end{table*}

\textbf{Reading comprehension} is the task that answering questions about a piece of text. We use the DREAM dataset~\cite{sun2019dream} where questions are about corresponding dialogues and the answer is a complete sentence in natural language. We neglect the negative choices in the original dataset and formulate it as a NLG task. We adopt the pre-trained language model BART~\cite{lewis2020bart} as the baseline, where the input is a concatenation of a question and the corresponding dialogue made up of speakers and utterances. 
We experiment with  transformers\footnote{\url{https://github.com/huggingface/transformers}} based on the publically available ``facebook/bart-large'' checkpoint \footnote{\url{https://huggingface.co/facebook/bart-large}}.
%The preceding BART model is also adopted as the baseline, whereas the input is a concatenation of question and a dialogue.
The generated answers are evaluated by BLEU scores\footnote{The BLEU-1/2/3/4 scores are computed according the Google's implementation(\url{https://github.com/tensorflow/nmt/blob/master/nmt/scripts/bleu.py}).}~\cite{papineni2002bleu} widely used for QA systems, together with Meteor and Rouge-L F1 as mentioned above. The parameters are also the same as dialogue summarization, except that the early-stop is activated if there is no improvement on the perplexity of the validation set. 


\textbf{Dialogue summarization} is to generate a concise summary covering the salient information in the input dialogue. The preceding model BART has shown to be a strong baseline for this task, where only the dialogue is concatenated into a single sequence as the input. We experiment with  %transformers\footnote{\url{https://github.com/huggingface/transformers}} based on the publically available ``facebook/bart-large'' checkpoint \footnote{\url{https://huggingface.co/facebook/bart-large}} and 
SAMSum dataset\footnote{\url{https://arxiv.org/src/1911.12237v2/anc/corpus.7z}}~\cite{gliwa2019samsum} for daily-chat dialogues. 
The generated summaries are evaluated by comparing with the reference through evaluation metrics, including Rouge-1/2/L F1 scores\footnote{\url{https://github.com/pltrdy/files2rouge}}~\cite{lin2004rouge}, Meteor~\cite{banerjee2005meteor} and BertScore F1\footnote{Both Meteor and BertScore are calculated by SummEval(\url{https://github.com/Yale-LILY/SummEval}), and the latter one is based on the default bert-base-uncased model.}. We evaluate the model on the validation set after each training epoch and the early-stop patience will be added 1 if there is no improvement according to the Rouge-2 F1 score. The training process terminates when the early-stop patience equals or is larger than 3.  During the inference, the minimum and maximum output length is set to 5 and 100 respectively, with no\_repeat\_ngram\_size=3, length\_penalty=1.0 and num\_beams=4.


% The answer is either a span of words in the original text or a complete sentence in natural language.
\textbf{Style transfer} preserves the semantic meaning of a given sentence while modifies it's style, such as positive to negative, formal to informal, etc.
We adopt the Shakespeare author imitation dataset~\cite{xu2012paraphrasing}, containing William Shakespeare's original plays and corresponding modernized versions. Krishna el al.~\shortcite{krishna2020reformulating} proposed to do unsupervised style transfer by training paraphrase models based on the GPT-2 language model~\cite{radford2019language}. We re-implemented their approach STRAT\footnote{\url{https://github.com/martiansideofthemoon/style-transfer-paraphrase}} and evaluated with the provided script. Evaluation metrics includes 
transfer accuracy(ACC), semantic similarity(SIM), Fluency(FL) and two aggregation metrics, i.e., geometric averaging(GM) and their newly introduced $J(\cdot)$ metric. The hyper-parameter $hp$ equaling 0.0, 0.6 or 0.9  in Table~\ref{tab:end2endst} is the sampling parameter for trades off between ACC and SIM in their approach. 
In the training stage, we evaluate the model after updating every 500 steps. The perplexity on the validation set is used to activate the early-stop which equals 3. The inference is done as default.
 
\textbf{Question generation}~\cite{zhou2017neural} aims at generating a question given an input document and its corresponding answer span. SQuAD 1.1~\cite{rajpurkar2016squad} is generally used for evaluation. We adopt the data split as in \cite{du2017learning} and fine-tune the pre-trained UniLM~\cite{dong2019unified} as the strong baseline according to their official implementation\footnote{\url{https://github.com/microsoft/unilm/tree/master/unilm-v1}}. Generated questions are evaluated by metrics including BLEU-1/2/3/4, Meteor and Rouge-L with the provided scripts. The model is evaluated every 1000 steps and the early-stop equaling 3 is associated with the perplexity on the validation set. Other parameters are unchanged following the official guideline.

\textbf{News summarization} differs from dialogue summarization where the input is a document instead of a dialogue. We adopt the same strong baseline BART and evaluation metrics as dialogue summarization. Experiments are done with CNNDM dataset~\cite{HermannKGEKSB15} consisting of news articles and multi-sentence summaries\footnote{\url{https://github.com/pytorch/fairseq/blob/main/examples/bart/README.summarization.md}}. The model is evaluated every 2000 steps and the early-stop equaling 3 is associated with the Rouge-2 on the validation set. During the inference, the minimum and maximum output length is set to 45 and 140 respectively, with no\_repeat\_ngram\_size=3, length\_penalty=2.0 and num\_beams=4.
%\footnote{Inference parameters are borrowed from \url{https://github.com/pytorch/fairseq/blob/main/examples/bart/summarize.py}}

The summary of each task is listed in Table~\ref{tab:taskdata}. For fair comparisons, we re-implemented baselines following the above instructions on our machine. On top of the above baselines, we further arm them with the ICL strategy according to the Algorithm~\ref{alg:picl}. The settings of newly introduce Start and Stride are specified and discussed in following sub-sections. All of our experiments are done on a single RTX 3090 or a single RTX 2080Ti with 24G and 11G GPU memory respectively.
%and the result are averaged over three runs.


 
\subsection{Automatic Evaluations on Different Tasks}
\label{sec:taskperformances}

We compare our approach with the vanilla models mentioned above and the approach from~\citet{liang-etal-2021-token-wise} as baselines.
The performances on different NLG tasks are shown in Table~\ref{tab:end2end}. 
These tasks not only focus on solving different problems, but also has various amount of training data as well
as reference output lengths as shown
Table~\ref{tab:taskdata}.
Besides, the basic model are also different, including BART, GPT-2 and UniLM. 
Our new training strategy achieves significantly improvements among different tasks on most evaluation metrics, which shows that our method not only works well, but also has strong generalization abilities.

We explain the some specific results as follows:

(1) Our training strategy boosts the performances of the original STRAT with different $hp$ in the style transfer task. GM and J are two comprehensive evaluation metrics, with our approach topping the ranks with significant improvements.

(2) TCL generally performs poorly on tasks
with more training data. For example, it failed on question generation without any improvements over the vanilla model under the same parameter setting, while ICL still 
logs gains. This is mainly due to two reasons.
First, because the nature of TCL is data augmentation which is more effective in low-resource settings,
when training data is abundant, it becomes less useful. 
Second, the way they calculate the loss as sub-sequence generation better suites paraphrasing tasks, such as machine translation tested in their paper, as the order of 
the corresponding tokens between input and output 
are almost the same. Learning such forward mapping can 
be regarded as a kind of ``easy-to-hard'' 
in these limited scenarios.
However, this doesn't hold true for other tasks, 
such as summarization and question generation. 
Therefore, we didn't further test it on CNNDM since
CNNDM has the large amount of training data among
the five.

(3) For news summarization, Rouge-1 scores (precision, recall) for the baseline and our method on CNNDM are (38.16, 52.72) and (40.84, 49.23) correspondingly. Our method made substantial improvements on the precision with a compromise on the recall. 
The meteor score based on the unigram precision and recall emphasizes more on the recall than the Rouge-1 F1. As a result, it drops while Rouge-1 F1 increases. Overall, our method still outperforms BART on this task, especially on F1 scores of Rouge-2 and Rouge-L.




\begin{table}[th]
	\small
	\centering
	\begin{subtable}{\linewidth}
		\scriptsize
		\centering
		\begin{tabular}{lcccccc}
			\hline
			{Method} & {B1} & {B2} & {B3} & {B4} & {Met} & {RL}\\
			\hline
			w/o CL &  32.03 & 16.01 & 8.77 & \textbf{4.80} & 19.84 & 38.89\\
			TCL & 32.53 & 16.25 & 8.52 &4.67 &19.88 & 39.65 \\
			ICL &  \underline{\textbf{33.99}} & \underline{\textbf{17.43}} & \underline{\textbf{9.18 }}& 4.64 & \textbf{20.60} & \textbf{40.78}\\

			\hline
		\end{tabular}
		\caption{Reading Comprehension}
		\label{tab:end2endrc}
	\end{subtable}
	\\[5pt]
	\begin{subtable}{\linewidth}
		\scriptsize
		\centering
		\begin{tabular}{lccccc}
			\hline
			{Method} & {R1} & {R2} & {RL} & {Met} & {BertS} \\
			\hline
			%BART & 52.60&27.00 &42.10 &- & - \\
			w/o CL & 51.88 & 27.30 & 42.77 & 24.75 & 71.38 \\
			TCL  & 52.33 & 27.80 & \textbf{43.91} & 24.59 & 71.77 \\
			ICL & \underline{\textbf{53.07}} & \underline{\textbf{28.23}} & {43.83} & \underline{\textbf{26.12}}& \underline{\textbf{72.17}} \\
			
			\hline
		\end{tabular}
		\caption{Dialogue Summarization}
		\label{tab:end2endds}
	\end{subtable}
	\\[5pt]
	\begin{subtable}{\linewidth}
		\scriptsize
		\centering
		\begin{tabular}{lcccccc}
			
			\hline
			{Method}&$hp$ &  {ACC} & {SIM} & {FL} & {GM} & {J}\\
			\hline
			%\multirow{3}{*}{STRAT}& 0.0 & 71.70 & \textbf{56.40} & 85.20 & 70.10 & 34.70 \\
			%& 0.6 & 75.70 & 53.70 & 82.70 & 69.50 & 33.50 \\
			%& 0.9 & 79.80 & 47.60 & 71.70 & 64.80 & 27.50 \\
			%\hline
			\multirow{3}{*}{w/o CL}& 0.0 & 70.49 & 55.70 & 85.98 & 69.63& 33.72 \\
			& 0.6 &75.31 & 53.46 & 82.56 & 69.27& 33.30\\
			& 0.9 & 78.76 & 47.38 & 74.42 &65.24 & 27.88\\
						\hline
			\multirow{3}{*}{TCL } & 0.0 & 70.31 & \textbf{55.95} &\textbf{87.24} &  70.01& 34.71 \\
			& 0.6 & 74.79 & 53.14 & 82.56 & 68.97 & 33.21 \\
			& 0.9 & 79.41 & 46.88 & 71.92 &64.45 & 26.92 \\
			\hline
			\multirow{3}{*}{ICL}& 0.0 & \underline{73.72} & 55.91 & 86.30 & \underline{\textbf{70.60}} &\underline{\textbf{35.81}}\\
			& 0.6 & 77.26 & \underline{53.80} & \underline{83.87} & \underline{70.38} & 34.64\\
			& 0.9 & \textbf{79.65} & 48.16 & 76.06 & 66.32 & 29.03\\

			\hline
		\end{tabular}
		\caption{Style Transfer.}
		\label{tab:end2endst}
	\end{subtable}
	\\[5pt]
	\begin{subtable}{\linewidth}
		\scriptsize
		\centering
		\begin{tabular}{lcccccc}
			\hline
			{Method} & {B1} & {B2} & {B3} & {B4} & {Met} & {RL}\\
			\hline
			w/o CL & \textbf{50.38} & 35.67 & 27.24 & 21.36 & 24.40 & 50.67 \\
			TCL &\textbf{50.38} & 35.67 & 27.24 & 21.36 & 24.40 & 50.67\\
			ICL &  50.18 & \textbf{35.72} & \textbf{27.36} & \textbf{21.54} & \textbf{24.57} & \underline{\textbf{51.09}} \\
			\hline
		\end{tabular}
		\caption{Question Generation}
		\label{tab:end2endqg}
	\end{subtable}
		\\[5pt]
	\begin{subtable}{\linewidth}
		\scriptsize
		\centering
		\begin{tabular}{lccccc}
			\hline
			{Method} & {R1} & {R2} & {RL} & {Met} & {BertS}\\
			\hline
			%BART &  \\
			w/o CL &  43.07 & 20.01 & 35.94 & \textbf{21.44} & 63.72 \\
			TCL & - & -&- &- &- \\
			ICL & \textbf{43.39} & \underline{\textbf{20.55}} & \underline{\textbf{36.63}} & 19.68 & \textbf{64.05}\\
			\hline
		\end{tabular}
		\caption{News Summarization}
		\label{tab:end2endns}
	\end{subtable}
	\caption{Performances on different NLG tasks. ICL represents the models trained with our ICL algorithm. TCL refers to the previous work from~\cite{liang-etal-2021-token-wise}. Scores underlined are statistically significantly better than both re-implemented baselines with $p<0.05$ according to t-test. }	
	\label{tab:end2end}
\end{table}


\subsection{Human Evaluations}

To further prove the improvement of ICL, we hired three proficient English speakers for human evaluation. 20 samples from the test set of each task are randomly selected, ignoring the ones with totally same generations among three models, including the vanilla model, TCL and ICL. The original input, reference output and three generations are shown to annotators together, while the order of three generations are unknown and different among samples. 3-point Likert Scale is adopted for scoring for each generation~\cite{gliwa2019samsum}, where [1, 3, 5] represent 
excellent, moderate and disappointing results 
respectively. The average scores and agreements 
among the annotators are shown in 
Table~\ref{tab:humaneval}.

The Fleiss Kappa on the first four tasks indicates the fair to moderate agreements. It shows the promising improvement of ICL over the vanilla model and TCL especially on DREAM, SAMSum, and SQuAD1.1, which is consistent with the conclusion based on automatic metrics.
Although the agreement on style transfer is fair, 
our annotators without Shakespeare background 
tend to give low scores to all outputs.
Therefore, the absolute improvement is 
only $0.04$ compared to both baselines.
%This mainly due to the indistinguishable styles between
%Shakespeare’s plays with are quite different from modern languages. 
Besides, the poor agreement on CNNDM reflects the 
diverse concerns of summarization from different 
annotators. Without more specific instructions, they 
tends to focus more on the content coverage instead 
of checking the detailed facts. This is also 
consistent with the higher Meteor scores of the 
vanilla model over ICL.

\begin{table}[th]
	\scriptsize
	\centering
	\begin{tabular}{l|ccc|c}
		\hline
		{Datasets} & {w/o CL} & {TCL} & {ICL} & {Agreement}  \\
		\hline
		DREAM  &3.07 & 2.50&3.20 &0.48 \\
		SAMSum &2.97 &3.57 &3.97 &0.40 \\
		Shakespeare &2.23 &2.23 & 2.27&0.32 \\
		SQuAD1.1 &3.43 & 3.43 &3.77 &0.35 \\
		CNNDM & 3.45 &- &3.40 &0.11 \\
	%	\hline
	%	overall & & & &\\
		\hline
	\end{tabular}
	\caption{Human evaluations. The agreement is calculated by Fleiss Kappa.}
	\label{tab:humaneval}
\end{table}




%Following Liu et al.\shortcite{liu2021competence}'s work, we asked annotators to comparing the performance between our generated results and baselines by choosing from ``Better, Tie, Worse''. 
%The counts for each choice are shown in Table~\cite{}, where the Fleiss Kappa among annotators is ??.

%Analysis





%\subsection{Analysis on Variable Generation Lengths}

%Teacher forcing, which predicts each token given the reference summary tokens during training and given the previous generated tokens during inference, leads to the exposure bias problem for NLG tasks.
%Since ICL starts the training process by predicting the last few tokens of outputs and gradually calculates the loss based on more tokens when the model is stronger, we hypothesis that it can alleviate the exposure bias for training Seq2Seq models to some extent.
%As stated in~\cite{pang2020text}, the output quality tends to degrade as the output length increase with the exposure bias.
%So, we divided the test set of each task according to the length of the generated output into 4 buckets and randomly picked 20 samples in each buckets for both the corresponding baselines and our approach. Each generation is annotated by 5 point Likert Scale, where 1 is the worst and 5 is the best. 

%The trends of performances on variable generation lengths are in Figure~\ref{}.


\section{Related Work}
This section surveys previous works on question generation and tree encoding
respectively.

Text question generation has attracted the attention 
after the work of ~\citeauthor{du2017learning}~\shortcite{du2017learning}, who uses deep seq2seq model 
to generate questions from a raw text paragraph. 
Before that, text question generation relied heavily on hand-craft 
question patterns~\cite{HeilmanS10,LabutovBV15,MostowC09} which is time and 
labor consuming. 

However, this pure seq2seq model is not focused and 
has no control over part in the paragraph to generate question. 
~\citeauthor{zhou2017neural}~\shortcite{zhou2017neural} proposed to encode 
key phrase information using binary indicators to generate 
key-aware questions and they assumes the answer to be key phrase. 
Considering key phrase (answer) is unavailable in reality, 
~\citeauthor{SubramanianWYT17}~\shortcite{SubramanianWYT17} applied 
a two-stage approach. First, key phrases are extracted by 
pointer network~\cite{ptrnet}. Second, 
key phrases are encoded in the same way as 
Zhou et al. With the intuition that questions could be asked in many ways, 
~\citeauthor{Yao2018vae}~\shortcite{Yao2018vae} used conditional-VAE to 
increase the diversity of questions. More recently, models with 
auxiliary feature information~\cite{HarrisonW18} helped improve 
the question quality. Structure question generation aims at 
converting structured data such as triples in knowledge graph to questions. 
~\citeauthor{SerbanGGACCB16}~\shortcite{SerbanGGACCB16} proposed a model to generate factoid questions from knowledge base triples.  None of the above work
considered using parse tree structures to aid question generation process,
which is the focus of this paper.

Sequential RNN model takes sentence as a sequence of words, 
ignoring the syntactic information. In order to utilize
such syntactic information with sequential information, 
~\citeauthor{tai2015improved}~\shortcite{tai2015improved} proposed Tree-LSTM to 
encode the binary parse tree recursively in a bottom-up fashion to 
classify sentiment. In text generation task, 
\citeauthor{eriguchi2016tree}~\shortcite{eriguchi2016tree} 
proposed a tree-to-sequence model with attention mechanism to do 
machine translation and 
~\citeauthor{liang2018automatic}~\shortcite{liang2018automatic} proposed a 
tree-to-sequence model which could handle arbitrary trees, 
to do code comment generation. Our work is inspired by these previous
attempts and we are first to adapt structure encoded neural models to
textual question generations.
%\input{Knowledge}
%\section{Similarity Computation}
That is, given a pair of terms, our approach has basically two steps: We first
represent the semantic contexts for each of the terms, modeling a probability distribution over contexts and then we compare how similar
these two discrete probability distributions are by encoding them as vectors and computing the cosine between the vectors.


In this section, we give the similarity computation method between terms.
The terms mentioned in this paper involve two categories, namely the concept and the instance in our Probase. Hence, we could get three categories corresponding to the type of the term in the given pair, such as the concept-instance pair (e.g.,~$<Company,~Microsoft>$), the concept-concept~pair (e.g.,~$<Company,~Country>$) and the instance-instance~pair (e.g.,~$<Apple,~Microsoft>$). However, if the given pair of terms has a concept term, we can use the contexts of seeds as the context of the current concept. For example, given a concept term $Company$, the top 10 seed instances include \emph{Microsoft}, \emph{IBM}, \emph{Google}, \emph{Apple}, \emph{Dell}, \emph{Intel}, \emph{Sony}, \emph{Motorola}, \emph{HP} and \emph{Samsung}. Therefore, we can transform the issue of measuring semantic similarity between terms into that between instances.
Correspondingly, we can formalize the issue below. Given the pairwise terms $<e_{1}, e_{2}>$, we can get the semantic contexts, such as attribute-based and isA-based contexts for each instance as shown in Figure~\ref{fig:Information-structure-of-terms}. According to the collected semantic contexts, our current task is hence to evaluate the similarity between contexts. That is,
\begin{figure}[t]
%\makeatletter\def\@captype{figure}\makeatother
 \centerline{
 \includegraphics[width=0.5\textwidth]{Information-structure-of-terms.eps}}
\caption{Semantic contexts of given terms} \label{fig:Information-structure-of-terms}
\end{figure}
\begin{equation}
\begin{aligned}
sim(T(e_{1}), T(e_{2})) = f(sim(A_{e_{1}}, A_{e_{2}}), sim(\mathcal{I}_{e_{1}}, \mathcal{I}_{e_{2}}))
\label{eq:task1}
\end{aligned}
\end{equation}
\begin{displaymath}
{s.t.,
\begin{aligned}
sim(A_{e_{1}}, A_{e_{2}}) = cosine(A_{e_{1}}, A_{e_{2}})\\
sim(\mathcal{I}_{e_{1}}, \mathcal{I}_{e_{2}}) = cosine(\mathcal{I}_{e_{1}}, \mathcal{I}_{e_{2}})~~
\end{aligned}
}
\end{displaymath}
where $f()$ indicates the similarity evaluation function between contexts. In our approach, we use logistic regression to combine the attribute based similarity (e.g., $sim(A_{e_{1}}, A_{e_{2}})$) and the isA-based similarity (e.g., $sim(\mathcal{I}_{e_{1}}, \mathcal{I}_{e_{2}})$). 
\section{Conclusion}
\label{sec:conclude}
In this work,
we propose a new data creation method to generate
 a semi-structured synthetic training data for 
opinion summarization,
which is known for lacking training data.
\cut{We showed that by extracting an aspect-opinion pairs and 
implicit sentences from multiple reviews
first and then synthesizing them into semi-structured data, we achieve
better performance on opinion summarization.}
%\KZ{It is critical to show in your experiments that the proposed
%synthetic data is better than other possible alternatives.}, 
We also designed an aspect-guided model with opinion-aspect pair encoder and implicit sentence encoder.
The results showed that
the proposed model can make full use of semi-structured data
and generate high-quality summaries.




%\section{Acknowledgments}
%This work is supported in part by NSFC grants 61100050 and 61373031.
% The following two commands are all you need in the
% initial runs of your .tex file to
% produce the bibliography for the citations in your paper.
\bibliographystyle{abbrv}
\bibliography{../bibs/probase}  % sigproc.bib is the name of the Bibliography in this case
% You must have a proper ".bib" file
%  and remember to run:
% latex bibtex latex latex
% to resolve all references
%
% ACM needs 'a single self-contained file'!
%
%APPENDICES are optional
%\balancecolumns
%\appendix
%Appendix A
%\balancecolumns
% That's all folks!
%\end{multicols}
\end{document}
