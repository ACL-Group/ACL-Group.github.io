\section{Problem Definition}
\label{sec:problem}

We begin with an informal definition of the
{\em action conceptualization problem}.
Given a collection of argument instances (either subjects or objects) for a
verb, which are extracted from text corpus, we would like to pick
a smallest set of abstract concepts $O$ from a taxonomy to cover all these
argument instances. That is, for each argument instance,
there must be at least one concept in $O$ which subsumes the argument
in an isA relation. A {\em concept} is defined as any {\em non-leaf} node
in the taxonomy.
At the same time, we require that the overlap between any two concepts in
$O$ to be as small as possible so each of these concepts essentially
represent a different use of the verb.
The resulting action concepts are similar to the definition entries of
words in a dictionary.
For example, for verb ``wear'', the list of extracted
objects might be ``t-shirt'', ``hoodie'', stetson hat'', ``bracelet'',
``ear ring'', ``pink'', etc. One way of conceptualizing the object of
``wear'' might be ``clothing'', ``jewelery'' and ``color''.
%We rely on a taxonomy to provide such isA relations.
%Probase is a large scale of IsA knowledge base derived from
%large collection of web pages.
%It contains information about relation between concept and entity
%which entity is an instance of concept and the probability of that relation.
%If we view each concept in Probase as a set of entities,
%the target of our problem is to pick a list of concepts in Probase
%which the union of these concepts contains all the objects in the
%given collection so those concepts can cover the usage of the verb.
%For example, given objects ``shoe'', ``hat'', ``shirt'',
%``red'' and ``yellow'' for verb ``wear'', we may obtain the
%following concepts, which ``clothing'' covers ``shoe'', ``hat'' and
%``shirt" while ``color" covers ``red'' and ``yellow.''
%``Clothing'' and ``color'' represent the different usages of ``wear.''
%
%\KZ{First define the problem informally, and then go into the
%mathematical definition.}
%We first define the problem of \emph{action conceptualization} informally.
%
%\KZ{Do not refer to Probase here, but instead define the problem
%purely on set of entities.}
%We want to find a set of
%concepts which can cover all the objects extracted from the corpus,
%that is, for each object, we can at least find one concept in the set having
%IsA relation with that object. During the extraction, we just keep those
%objects that can be mapped to entities. If we view each Probase
%concept as a set of entities, our process is similar to the classical set
%cover problem. All the objects extracted from the entities set $U$ to be
%covered, Probase itself is a set $S$ of entities set so the problem is to
%identify a smallest subset of $S$ whose union contains all the entities in $U$.
%The Difference is that, in our problem, the union of sets in $S$ may be
%larger than $U$, while in the classical set cover problem, $U$ and the union
%of sets in $S$ is equal. Formally, our problem can be formulated as follow:

Let the set of all extracted argument instances of a verb be $U$, and
let $S$ be the collection of all concepts in the taxonomy where each concept
is a set of entities (i.e., $S$ is a set of sets of entities). The
action conceptualization problem can be formalized as
a minimization problem:
\begin{eqnarray}
\min && \sum_{s\in S} x_s \\
\rm{subject~to} && \sum_{s:e\in s} x_s \ge 1,\ \forall e \in U \label{eq:cover}\\
&& x_s\in \{0,1\},\ \forall s \in S \label{eq:binary} \\
&& \frac{|s\cap U|}{|s|} \ge \delta, \forall s:x_s=1 \label{eq:corpuscover}\\
% && \sum_{s,t:x_s=1,x_t=1} |s\cap t| \le \tau
&& \frac{|s\cap t|}{min\{ |s|,|t| \}} \le \tau,\ \mbox{for~}  x_s=1, x_t=1
\label{eq:overlap}
\end{eqnarray}
% If $\bigcup_{s\in S}s = U$, then $|s\cap U| = |s|$, constraint (4) is always satisfied so we can remove it.

\begin{lemma}
Action conceptualization problem is NP-hard.
\end{lemma}

\begin{proof}
A special case of the above problem is
when $\tau = 0$ \textcolor{red}{and $\delta=0$}, that is,  when all the sets we pick from $S$
are disjoint from each other.
In this case, each entity is required to be covered by exactly one set.
By removing constraint \textcolor{red}{(\ref{eq:corpuscover})}(\ref{eq:overlap})
and change the inequality in constraint (\ref{eq:cover})
into equality, we obtain the following equations which is equivalent to
the special case of $\tau=0$ \textcolor{red}{and $\delta=0$}.

\begin{eqnarray}
\min && \sum_{s\in S} x_s \\
\mbox{subject~to} && \sum_{s:e\in s} x_s = 1, \forall e \in U \\
&& x_s\in \{0,1\}, \forall s \in S
\end{eqnarray}

This new problem is the optimization version of the
{\em exact cover problem} \cite{karp72} in its binary form,
which is known to be NP-hard.
And since extract cover problem is a special case of our problem,
our problem is also NP-hard.
\qed
\end{proof}
Given that our problem is NP-hard, next we present two approximate
approaches to solve it.
