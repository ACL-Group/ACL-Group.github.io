\subsection{Ranking of Action Concepts}
\label{sec:rank}
Previous two algorithms return an approximately  minimum set of
concepts that cover all the input argument instances of a given verb
which satisfy the overlap constraint. This subsection introduce a
simple but effective way to rank these concepts. This ranking is
important for three reasons.
I think ranking has the following purposes:
\begin{enumerate}
\item There are still many concepts (in the hundreds) in the AC results,
thus there is need to present the most representative concepts for
human consumption;
\item Not all the concepts are useful in argument identification,
as we will point out later in \secref{sec:eval}, the top concepts for
an argument are often sufficient to determine whether an argument is
legal and correct.
\item There are noisy concepts which are too general and often contain
incorrect entities due to the inaccuracy of dependency parsers. A
good ranking algorithm may ``rank down'' these undesirable concepts.
\end{enumerate}

The basic idea behind our ranking algorithm is that the best concept
(the concept that has the largest representative score) has the highest
priority to pick the arguments in corpus. After one argument is
picked by a concept, it will be removed from the candidate argument set
(CAS). And the iteration ends up when CAS is empty.
In this process, every concept has an attribute named $CN$
which is the number of unique entities a concept covers.
Finally, $CN$ is used as criterion to rank the concepts.
Our algorithm is shown in Algorithm \ref{rank_concepts}.

\begin{algorithm}[th]
\caption{Rank concepts}
\label{rank_concepts}
\begin{algorithmic}[1]
\Function{RankConcepts}{corpusE,resultC}
\State rank $resultC$ by representative $score$
\For{$c \in resultC$}
\State $num \leftarrow 0$
\For{$e \in corpusE$}
\If{$e\ isA\ c$}
\State $num \leftarrow num+1$
\State remove $e$ from $corpusE$
\EndIf
\EndFor
\State $c.CN\leftarrow num$
\EndFor
\State rank $resultC$ by $CN$
\State \textbf{return} $resultC$
\EndFunction
\end{algorithmic}
\end{algorithm}

