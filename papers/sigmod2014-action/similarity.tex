\subsection{Term Similarity Computation}
% We need the lexicon for large number of verbs... This could be difficult.
Term similarity computation is generally based on a predefined taxonomy.
While computing the similarity of two terms, taxonomy is used to provide
hyponyms and hypernyms information of both terms. Those hyponyms and hypernyms
can be used to represent the terms in set or vector forms, then Jaccard similarity
or cosine similarity can be applied to compute the similarity of the terms.
However, for some ``related'' terms which do not share common hyponyms or
hypernyms, taxonomy may not work well, like ``company'' and ``stock''. We
argue that by complementing our action concept lexicon with taxonomies, we
can obtain a better similarity result. In the example above, ``company'' and
``stock'' can be connected by verb ``release'' or ``sell'', our lexicon can provide
extra knowledge for calculating the similarity of these two terms.

Li\cite{LiWZWW13} introduces a term similarity computation method based on Probase.
In this experiment, we implement Li's method and also define our term similarity
function based on our action concept lexicon. When computing the similarity
of two terms, we combine these two methods and use the higher score produced as
the final similarity score. Similar to Li's method, we apply different similarity
function on terms according to their types(concepts or entities):
\begin{eqnarray*}
sim_{c}(c_1,c_2) &=& \max(\frac{V_o(c_1)\cap V_s(c_2)}{V_o(c_1)\cup V_s(c_2)},\frac{V_s(c_1)\cap V_o(c_2)}{V_s(c_1)\cup V_o(c_2)}) \\
sim_{e}(e_1,e_2) &=& \frac{ \sum_{c_i \in C(e_1)}\sum_{c_j \in C(e_2)} sim_{c}(c_i,c_j) }{|C(e_1)||C(e_2)|} \\
sim_{c\&e}(c,e) &=& \max_{c_i \in C(e)}sim_c(c,c_i)
\end{eqnarray*}
which $V_o(c)$ is the set of verbs taking c as object in action concept lexicon, $V_s(c)$
is the set of verbs taking $c$ as subject. $C(e)$ is the set of concepts for entity $e$ in
Probase.

We test our method(verb based) with Li's method(noun based) on a word similarity label data set
``Word Similarity 353''\cite{LiWZWW13}, using Pearson correlation as the metric. The result is shown below:
\begin{table}[th]
\small
\centering
\begin{tabular}{|c|c|c|}
\hline
Noun based & Verb based & Combined \\
\hline
0.27 & 0.24 & 0.35 \\
\hline
\end{tabular}
\end{table}

We can see that, by combining knowledge provided by Probase taxonomy and our action concept
lexicon, we can get a better similarity result.

