\section{Conclusion}
\label{sec:conclude}
In this paper, we present the action conceptualization 
problem which seeks to generalize the arguments 
(either object or subject) 
of a verb into a set of concepts drawn from 
a taxonomy with limited pairwise overlap among them. 
Each concept then denotes approximately the dictionary
``sense'' of the verb.
We developed a data-driven approach which automatically
extracts such concepts by abstracting from large 
number of verb-argument pairs parsed from raw text.
Despite it's NP-hardness, the problem can be solved by
an approximation algorithm using branch-and-bound principles.
We applied the approach to Google syntactic N-gram data and
obtained a comprehensive, human-readable, machine-computable 
lexicon for 10,116 common English
verbs and verb phrases. This lexicon sits comfortably on the
semantic spectrum between FrameNet,
and instance-based dataset like ReVerb. It has better
coverage but more refined semantics than FrameNet, and
thus finds applications in a range of NLP applications.

%Our problem can be thought of as a
%variant of the class-based selectional preference. 
%The difference
%is that we can control the semantic granularity of the 
%generated concepts and the fact that they have limited overlaps
%means that our action concepts are closer to the dictionary
%``senses'' of the verb being analyzed.
%We have generated an action concept lexicon for 
%3734 most popular verbs in English and showed that 
%such a lexicon can be useful in a variety of NLP applications.
%These concepts cover most of the valid arguments 
%in the input data and have small
%overlap with each other. We automatically generate
%such action concepts for 3734 most frequently used English verbs and
%showed that such a lexicon represent finer grained semantics of
%verbs, thus can be used in a variety of applications for
%understanding text and outperforms several well-known related approaches
%or lexicons. One possible future direction is to generate mappings
%from actions to noun concepts, such as from
%``\underline{play}/instrument'' to ``performance'' and from
%``country/\underline{invade}'' to ``war.''
%
