\section{Introduction}

% Since the internet is growing fast and e-commerece is more and more popular, there is a massive amount of user reviews of all kinds of products available online. Although the amout of reviews is huge, they cannot be very helpful if not organized - user simply cannot read all those reviews. So summarizing user reviews has motivated a lot of efforts and study.

Summarization of online reviews has been very popular and useful. In order to make the massive amount of reviews readable and intelligible to the users, we summarize the reviews into some structure. One popular and reasonable form is \emph{aspect-based opinion mining}\cite{hu2004mining} , which focuses on the \emph{aspects} of the product and summarize the reviews around those aspects. For example, a sentence like ``The battery life of my phone is too short.'' talks about an aspect of mobile phone, namely \emph{battery life}, and describes the aspect of battery life as ``too short''. 

Aspect-based structure is both informative and human-readable thus \emph{aspect extraction} task becomes important. Websites that focuses on one kind of product like TripAdvisor may manually set those aspects and let users rate each of them. But for huge e-commerece websites like Amazon and Ebay which feature their great variety of products, manually setting the aspects for each kind of product may not be effective or plausible. Also, pre-defined aspects may not reflect what customer care about. For example phone users may care more about battery life than some feature the producer advertise about, so aspects chosen by producer or retailer may not be a good summarization of the opinion of the users.

Our goal in this paper is to develop a system that automatically does this task. We propose a unsupervised, multi-stage clustering method that given reviews of some kind of products, finds the best aspects. In developing the system, we leverage the lastest practice of Deep Learning in natural language processing, Sentence2Vec. Based on it we run multiple level of clusterings and result in clusters that each consists of words related to this potential aspect, that is, the aspect candidates. Then we select the best aspect candidate from each cluster, and for this purpose we propose a ranking system that rank the words in each cluster on how they summarize the whole cluster, and also rank the clusters on how they represent the opinon of the users.
