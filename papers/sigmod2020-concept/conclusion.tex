\section{Conclusion}
\label{sec:conclusion}

In this paper,
we point out that there is a huge semantic gap between user needs and current ontologies in most e-commerce platforms.
This gap inevitably leads to a situation where e-commerce search engine and recommender system can not understand user needs well, 
which, however, are precisely the ultimate goal of e-commerce platforms try to satisfy.
To tackle it, we introduce a specially designed e-commerce cognitive concept net ``AliCoCo'' practiced in Alibaba, trying to conceptualize user needs as various shopping scenarios, 
also known as ``e-commerce concepts''. 
We present the detailed structure of AliCoCo and introduce how it is constructed with abundant evaluations.
AliCoCo has already benefited a series of downstream e-commerce applications in Alibaba.
Towards a subsequent version, our future work includes: 
1) Complete AliCoCo by mining more unseen relations containing commonsense knowledge, for example,
``boy's T-shirts'' implies the ``\textit{Time}'' should be ``\textit{Summer}'', even though term ``summer'' does not appear in the concept.
2) Bring probabilities to relations between concepts and items.
3) Benefit more applications in e-commerce or even beyond e-commerce.

%\section{Future Work}
%
%we have to infer unseen attributes for short concept phrases as well. For example, ``Boy's T-shirts'' implies the ``Time'' should be ``Summer'', which term ``summer'' does not appear in the phrase.
%Designing such an inference model is challenging.
%
%heterogeneous contents
%link with open-domain KG
%more applications in search \& recommendation

\section{acknowledgment}
We deeply thank Mengtao Xu, Yujing Yuan, Xiaoze Liu, Jun Tan and Muhua Zhu for their efforts on the construction of AliCoCo. 


