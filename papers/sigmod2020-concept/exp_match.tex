
\subsection{Concept-Item Semantic Matching}


In this subsection,
we demonstrate the superior of our semantic matching model
for the task of associating e-commerce concepts with billion of items in Alibaba.
We create a dataset with a size of $450m$ samples, among which $250m$ are positive pairs and $200m$ are negative pairs.
The positive pairs comes from strong matching rules and user click logs of the running application on Taobao mentioned in \secref{sec:intro}.
Negative pairs mainly comes from random sampling.
%共392个不同query,通过人工标注20w正、20w负,总计40w
For testing, we randomly sample $400$ e-commerce concepts,
and ask human annotator to label based on a set of candidate pairs.
In total, we collect $200k$ positive pairs and $200k$ negative pairs as testing set.

\begin{table}[th]
	\centering
	%\scriptsize
	\begin{tabular}{l|c|c|c}
		\hline
		Model & AUC & F1 & P@10   \\
		\hline
		BM25 & - & - & 0.7681 \\
		DSSM \cite{huang2013learning}  & 0.7885 & 0.6937 & 0.7971  \\
		MatchPyramid \cite{pang2016text} & 0.8127 & 0.7352 & 0.7813  \\
		RE2 \cite{yang2019simple} \ & 0.8664 & 0.7052 & 0.8977  \\
		\hline
		Ours & 0.8610 & 0.7532 & 0.9015  \\
		Ours + Knowledge & \textbf{0.8713} & \textbf{0.7769} & \textbf{0.9048}  \\
		\hline 
	\end{tabular}
	\caption{Experimental results in semantic matching between e-commerce concepts and items.}
	\label{tab:matching}
\end{table}

\tabref{tab:matching} shows the experimental result, 
where F1 is calculated by setting a threshold of $0.5$.
Our knowledge-aware deep semantic matching model outperforms all the baselines in terms of AUC, F1 and Precision at $10$,
showing the benefits brought by external knowledge.
To further investigate how knowledge helps, 
we dig into cases. Using our base model without knowledge injected,
the matching score of concept ``中秋节礼物 (Gifts for Mid-Autumn Festival)'' and item ``老式大月饼共800g云南特产荞三香大荞饼荞酥散装多口味 (Old big moon cakes 800g Yunnan...)'' is not confident enough to associate those two, since the texts of two sides are not similar.
After we introduce external knowledge for ``中秋节 (Mid-Autumn Festival)'' such as ``中秋节自古便有赏月、吃月饼、赏桂花、饮桂花酒等习俗。(It is a tradition for people to eat moon cakes in Mid-Autumn...)'', 
the attention score for ``中秋节 (Mid-Autumn Festival)'' and ``月饼 (moon cakes)'' increase to bridge the gap of this concept-item pair.
  
