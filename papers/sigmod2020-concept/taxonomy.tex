\section{Taxonomy}
\label{sec:taxonomy}

%\RV{[Separate Taxonomy from the layer of Primitive Concepts]:}

The taxonomy of AliCoCo is a hierarchy of pre-defined classes
to index million of (primitive) concepts.
A snapshot of the taxonomy is shown in \figref{fig:primitive}.
%\footnote{We attempt to release part of the taxonomy data after acceptance}.
Great efforts from several domain experts are devoted to manually define the whole taxonomy.
There are $20$ classes defined in the first hierarchy, 
among which there are $7$ classes are specially designed for e-commerce, including ``\textit{Category}'', ``\textit{Brand}'', ``\textit{Color}'', ``\textit{Design}'',
``\textit{Function}'', ``\textit{Material}'',
``\textit{Pattern}'', ``\textit{Shape}''
``\textit{Smell}'', ``\textit{Taste}'' and 
``\textit{Style}'',
where the largest one is ``\textit{Category}'' having nearly $800$ leaf classes, since the categorization of items is the backbone of almost every e-commerce platform.
Other classes such as ``\textit{Time}'' and  ``\textit{Location}'' are more close to general-purpose domain.
One special class worth mentioning is ``\textit{IP}'' (Intellectual Property), 
which contains millions of real world entities such as famous persons, movies and songs.
Entities are also considered as primitive concepts in AliCoCo.
The $20$ classes defined in the first hierarchy of the taxonomy are also called ``domains''.

\begin{figure}[th]
	\centering
	\epsfig{file=figures/primitive.eps, width=\columnwidth}
	\caption{Overview of the taxonomy in AliCoCo.}
	\label{fig:primitive}
\end{figure}