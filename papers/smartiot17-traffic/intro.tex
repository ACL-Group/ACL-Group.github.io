\section{Introduction}
Traffic speed prediction is a challenging problem and has various downstream applications of Internet of Things (IoT), many of which are fundamental to intelligent transportation systems and smart city, such as
congestion management, urban planning, vehicle routing, etc.~\cite{Pan2012,Xu2015,Mchugh2015}.

Most existing machine learning approaches heavily rely on the vast amount of historical data for the areas being predicted~\cite{Ren2014,Clark2003Traffic}.
However, accurate and reliable historical traffic data collected from road sensors is very expensive and available in urban areas where the government can afford the large cost.
Consequently, most state-of-the-art time-series based models cannot be applied directly on areas where little traffic data is available.

Another disadvantage of most existing approaches is that they only focus on temporal features and do not explicitly utilize semantic features from spatio-temporal patterns~\cite{yao2017short,lin2017road},
which benefit many practical applications of urban computing~\cite{zheng2014urban}.
Research on extracting such effective spatial features for traffic speed prediction is almost missing from the literature. 

The preliminary study aims to answer this research questions:
\textit{{{How can we exploit the data of data-abundant areas to predict traffic speed for areas without traffic data through their semantic spatio-temporal features?}}}
To the best of our knowledge, we are among the first to study transfer learning for traffic speed prediction~\cite{xu2016cross}.
The contributions of this paper are as follows: 
\begin{itemize}
	\item We extract various spatial features in multiple levels and combine them with temporal features to support this transfer learning scenario, which also improves the transparency of prediction models. 
	\item Based on the features, we propose a novel clustering-based transfer learning model. Experimental results show that proposed model perform competitively with classic regression methods, but using only distant supervision.
\end{itemize}
%1) Our proposed transfer learning~\cite{Pan:Survey} approach supports many classic regression models. 

%	\item We also built a preliminary immature framework based on the idea of Generative Adversarial Networks to capture the dependency across the neighboring road segments by learning traffic data as images. 

%\section{Related Work}
%In the following sections, we will first talk about the dataset we are interested in and how we extracted spatial features from multiple data sources.
%Then, we will introduce our framework in detail and our experiments. 
%Finally, we would like to talk our preliminary attempts to learn traffic data as images and utilize the idea of Generative Adversarial Networks to generate the joint prediction results. 
