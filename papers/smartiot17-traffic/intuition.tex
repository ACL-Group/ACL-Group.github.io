\section{Observation and Intuition}
\label{sec:intuition}
In the DBLP dataset we used, there are in total 71195 users, and 18638580 edges, i.e. (u,v) pairs,  exist between them. Each edge consists of several time pairs $(t_1,t_2)$ too. To analyze the data, we did some statistical analysis on the dataset. 

In the dataset, each user has his/her ground truth area of research. However, the ground truth fails to capture the temporal evolution of users. To come up with a better ground truth for temporal distribution between users, we define our ground truth under the temporal scale in the following way. 

For user $u$ and each research area $C_i$, we count his/her number of interactions in the data with the users $v\in C_i$. We count the total number of interactions for $u$ in each year. Using these numbers as the Y-axis and the time dimension as the X-axis, we can get a curve as a ground truth curve. We denote this curve as $Cur$. To some extent, because a user's community is reflected by the frequency he/she interacts with the people in the community, and the majority of the people's fields in every area are quite stable, the curve can reflect every user's temporal evolution of his area of research. 
%We plotted some representative figures of the curve $Cur$ shown in Figure 5.

We also counted the percentage of people that their research area changed in the dataset. We defined a threshold $\sigma$ for the variance of the curve $Cur$. Intuitively, if the variance is larger than $\sigma$, we will regard this user changed his research interest in a community. When using $\sigma $ as 0.5, we find that $17\%$ of the users in our dataset changed their interest. 

These observations imply that building a model for the temporal evolution of communities, i.e., dynamic community detection, is highly essential to analyze temporal networks.

