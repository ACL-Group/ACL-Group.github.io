\section{Future Work}
Following is some potential future work onto our extracted spatial features:

Considering that traffic of nearby path could have impact on a certain path, we try to model the underlying influence of several paths. 
For a single path, we construct a sub-path that consists of two more paths adjacent to current path. 
The input is a 3D tensor, denoted as X. Each row of it contains the feature of each path. 
The columns of tensor correspond to the time, whose length is 7$\times$24.

The features of each path for each column will totally the same if we only take spatial and temporal features into consideration. Since we have already introduced the impact of nearby paths by inserting it into the input tensor, we could further enhance this concept. 
The feature of each single path has 3 types of components, including spatial, temporal and coherence feature vectors. The spatial and temporal features are mentioned before. 
The coherence feature vectors include the spatial coherence of nearby paths and the temporal coherence of traffic data from other time of current path.
The size of input tensor X is $k \times h \times f$. k is the number of paths in sub-path, h is 7*24 and f is the length of feature vector. $\vec{X_{ij}} $ means the feature of $i^{th}$ path at time j.
The output y is a 2D matrix, whose size is $k \times h$. $y_{ij} $ stands for the typical speed of $i^{th}$ path at time j.
The training process is trying to map this tensor to output matrix.

In test, we construct the sub-path as what we did in training. As the output is unknown, we initialize the coherence vectors.
We will try to make the coherence approximate the true value iteratively.
We use the output $\hat{y}$ at ${(t-1)}^{th}$ iteration(denoted as $\hat{y}_{t-1}$) to calculate the newest coherence vectors and get the output $\hat{y}_t$. The iterations will continue until the optimal coherence vectors are derived.
Our model captures not only the feature of current path, but also naturally model the propagation of traffic information of nearby paths.