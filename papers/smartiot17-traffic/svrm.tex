\subsection{Support Vector Regression Model}
Support vector regression (SVR) depends only on a subset of the training data, 
because the cost function for building the model ignores any training data close to the model prediction. 
Training the original SVR means solving following optimization problem, where ${\displaystyle \mathbf{x_{i}}}$ is a training sample with target value ${\displaystyle y_{i}}$:
\begin{align*}
\min ~ {\displaystyle {\frac {1}{2}}\|\mathbf{w}\|^{2}}, ~~
\text{subject to} ~{\displaystyle {\begin{cases}y_{i}-\langle \mathbf{w},\mathbf{x_{i}}\rangle -b\leq \varepsilon \\\langle \mathbf{w},\mathbf{x_{i}}\rangle +b-y_{i}\leq \varepsilon \end{cases}}}
\end{align*}

%The inner product plus intercept ${\displaystyle \langle w,x_{i}\rangle +b} $
%is the prediction for that sample,
%and ${\displaystyle \varepsilon }$  
%is a free parameter that serves as a threshold: 
%all predictions have to be within an ${\displaystyle \varepsilon }$ 
%range of the true predictions. Slack variables are usually added into the above to allow for errors and to allow approximation in the case the above problem is infeasible.


%To predict the future traffic speed, we input spatial and temporal features as x and the true traffic speed as y for training. 
%After obtaining the $w$ and $b$, we use this model to predict traffic speed of test data and compare the prediction with true data. 