
%%%%%%%%%%%%%%%%%%%%%%% file typeinst.tex %%%%%%%%%%%%%%%%%%%%%%%%%
%
% This is the LaTeX source for the instructions to authors using
% the LaTeX document class 'llncs.cls' for contributions to
% the Lecture Notes in Computer Sciences series.
% http://www.springer.com/lncs       Springer Heidelberg 2006/05/04
%
% It may be used as a template for your own input - copy it
% to a new file with a new name and use it as the basis
% for your article.
%
% NB: the document class 'llncs' has its own and detailed documentation, see
% ftp://ftp.springer.de/data/pubftp/pub/tex/latex/llncs/latex2e/llncsdoc.pdf
%
%%%%%%%%%%%%%%%%%%%%%%%%%%%%%%%%%%%%%%%%%%%%%%%%%%%%%%%%%%%%%%%%%%%


\documentclass[a4paper]{llncs}

\usepackage{amssymb}
\setcounter{tocdepth}{3}
\usepackage{graphicx}

\usepackage{url}
\urldef{\mailsa}\path|{alex_wang@126.com,|
\urldef{\mailsb}\path|kzhu@cs.sjtu.edu.cn|
\urldef{\mailsc}\path|yibo.fan@gmail.com| 
\urldef{\mailsd}\path|every.elva@gmail.com|
\urldef{\mailse}\path|shajin0000@sjtu.edu.cn}|   
\newcommand{\keywords}[1]{\par\addvspace\baselineskip
\noindent\keywordname\enspace\ignorespaces#1}
\setcounter{secnumdepth}{3}
\setcounter{tocdepth}{3}


\begin{document}

\mainmatter  % start of an individual contribution

% first the title is needed
\title{Interactive precompiler extends Verilog to be able to describe reconfigurable hardware IP cores}

% a short form should be given in case it is too long for the running head
\titlerunning{Lecture Notes in Computer Science: Authors' Instructions}

% the name(s) of the author(s) follow(s) next
%
% NB: Chinese authors should write their first names(s) in front of
% their surnames. This ensures that the names appear correctly in
% the running heads and the author index.
%
\author{Donghua Wang%
\and Kenny Q. Zhu\and Yibo Fan\and Biman Tang\and\\
Jin Sha}
%
\authorrunning{Lecture Notes in Computer Science: Authors' Instructions}
% (feature abused for this document to repeat the title also on left hand pages)

% the affiliations are given next; don't give your e-mail address
% unless you accept that it will be published
\institute{Shanghai Jiaotong University,\\
Dongchuan road 800, 200240 Shanghai, China\\
\mailsa\\
\mailsb\\
\mailsc\\
\mailsd\\
\mailse\\
\url{http://www.sjtu.edu.cn}}

%
% NB: a more complex sample for affiliations and the mapping to the
% corresponding authors can be found in the file "llncs.dem"
% (search for the string "\mainmatter" where a contribution starts).
% "llncs.dem" accompanies the document class "llncs.cls".
%

\toctitle{Lecture Notes in Computer Science}
\tocauthor{Authors' Instructions}
\maketitle


\begin{abstract}
When using hardware description languages (HDL) to do circuit design, we need to handle much unnecessary details, such as the wire connections and timing. And different implementations of an algorithm vary in time and gate resources. The designer has to rewrite the source code and puts much effort on considering sequential signals if he/she wants to gain trade-offs between performance and cost. In this work, we designed some new HDL syntax that the designer can use to express the configurable parts of circuits. We also implement a Verilog compiler to handle the new syntax and dig out all the possible trade-offs between time and space of the design. The compiler generates Verilog RTL codes depends on the designer's optimization choices during the interactive compiling process. Different from traditional circuit compilers, like DFT compiler, that only work on single kind of algorithms, our syntax can be used in all kinds of Verilog program if there are any configurable parts. And by using the new syntax, designers have more configuration options.\newline\newline
\textbf{Key words: }IP core, Verilog, reconfigurable design, sequential circuit, interactive compiler
\end{abstract}
\end{document}
