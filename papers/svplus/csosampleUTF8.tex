\documentclass[UTF8]{csoarticle}


\newtheorem{theorem}{Theorem}
\newtheorem{lemma}{Lemma}
\renewcommand{\proofname}{Proof}

% \newtheorem{theorem}{Theorem}
% \newtheorem{lemma}{Lemma}

\begin{document}

%----------------------------------------------------------
% 1. Title information
%----------------------------------------------------------

\titleCHN{Chinese title}
\titleENG{Title of article}
\authorCHN{张三丰\affil{1},李四\affil{2}}
\authorENG{ZHANG San-Feng\affil{1}, LI Si\affil{2}}
\affiliationCHN{
    \affil{1} **大学**学院,城市 邮编 \\
    \affil{2} **大学**学院,城市 邮编
}
\affiliationENG{
    \affil{1} Department of Science, University of Noname, City 112233 \\
    \affil{2} Department of Computer, University of Noname, City 332211
}
\abstractCHN{综述文章:以背景、研究现状、研究用途的结构书写,篇幅以150~300字左右为宜,不用第一人称做主语,不与正文语句重复。一般研究性文章:以摘录要点的形式按目的、方法、结果、结论的结构报道出作者的主要研究成果,字数在200~400字左右为宜,不用第一人称做主语,不与正文语句重复。}
\abstractENG{Reviews: please describe the background, status and application of the research with 150-300 words. I and we can not be used as the subject, and the abstract must not the same as the sentences of the main text. General research paper: please extracts the key points of the paper, give the main research achievements with object, methods, results and conclusion with 200-400 words. I and we can not be used as the subject, and the abstract must not the same as the sentences of the main text.}
\keywordCHN{中文关键词要能反映文章的基本观点,避免广义词。第一个关键词为该文内容所属二级学科名称}
\keywordENG{The keywords must reflect the basic view points of the paper. The first keyword should be the secondary discipline of the paper. }
\cateidCHN{please refer to Chinese Library Classification}

\authorIntroduction{Zhang San(1978-),male,associate professor,major research direction:***. Correspondence author:Li Si(1960-),male,professor,major research direction:***.
}
\fund{***Foundation (00000000),*** Foundation (00000000)}

\maketitle


%----------------------------------------------------------
% 2. Main text
%----------------------------------------------------------

\section{Introduction}

Briefly introduce the background and object of the research in about 400~600 words, and do not exceed 800 words.

\section{Format instruction}

\subsection{Text}

\subsubsection{Text typeface}

Please do not use typeface setting commands of \LaTeX{} in the main text, for example, do not directly use following typeface commands in the main text: 
Song typeface \verb|\songti|, boldface \verb|\heiti|, imitation Song-Dynasty-style typeface \verb|\fangsong|, standard typeface \verb|\kaishu|.
If some words need to be emphasized, please use \verb|\emph{words}| command or \verb|\textbf{bold words}| or \verb|\textit{tilted words}|, etc.

\subsubsection{footnote}

Inserting footnote to the main text\verb|\footnote{footnote}| directive\footnote{content of the footnote}.

\subsubsection{reference citation}

The references should be cited in the main text by serial numbers and superscript or not according to the context. For example: the paper \cite{bib1} proposes a method, and the subsequent research improves the method upcite{bib1,bib2}.

\subsection{figure}

This template \verb|csoarticle| includes \verb|amsmath| and \verb|amsthm| micros, so the typical method is using \verb|\includegraphics| command to contain the figure into the floating environment \verb|figure|. The figure \ref{fig:sample} is an example. Except \verb|pdf| format, \verb|eps|, \verb|jpeg| and other formats are also supported. The specific usage methods are included in the micro document \verb|graphicx|.

\begin{figure}
\centering\includegraphics[height=6cm]{figsamp}
\caption{curve graph sample}
\label{fig:sample}
\end{figure}

\subsection{table}

The samples using floating environment \verb|table| are listed in the table \ref{tab:sample}。
\begin{table}
  \caption{table sample}
  \label{tab:sample}
  \centering
  \begin{tabular}{lcr}%{|l|c|r|}
    \hline
    algorithm                & scale of coefficient matrix    & execution time (s)  \\
    \hline
    Gauss elimination method (direct method)     & Mat1903           &  113.27       \\
     Jacobi point iteration            & Mat1784           &  201.36       \\
    \hline
  \end{tabular}
\end{table}

\subsection{mathematical formula}

This template \verb|csoarticle| includes \verb|amsmath| and \verb|amsthm| micros, so all the commands provided by the micros can be used. The mathematical formula with numbers should use \verb|align| environment. For example: quadratic equation of one variable
\begin{align}\label{eq:sample}
    a x^2 + b x + c = 0
\end{align}
的两个根为
\begin{align}\label{eq:root}
    x_1 &= \frac{-b + \sqrt{b^2 - 4ac}}{2a} \\
    x_2 &= \frac{-b - \sqrt{b^2 - 4ac}}{2a}
\end{align}
Wherein, coefficient of equation \eqref{eq:sample} $a \not= 0$。

The common mathematical theorem environment can be used directly. For example:(\emph{fictitious examples}):
\begin{lemma}\label{lem:levy}
     Lemma content.
\end{lemma}
\begin{proof}
   Please refer to mathematical analysis teaching material.
\end{proof}

\begin{theorem}[Newton’s second law]\label{thm:newton}
Mass $m$, force $F$ and acceleration of an object satisfy
\begin{align}\label{eq:f-eq-ma}
    F = m a
\end{align}
\end{theorem}
\begin{proof}
The result can be obtained by lemma \ref{lem:levy} and the theorem 4 of the 15th chapter of the reference \cite{bib1}.
\end{proof}

Please refer to the source codes of the above examples.

\subsection{Format instruction for references}

Not less than 6 for general research papers, and not less than 25 for review.

Note: the author should has carefully read the references and checked the authors’ name, paper title, journal title, year, volume number, issue number and page number of the references. The references should be numbered according to their order in the main text. [ ] should be added to the two sides of the number. If the reference has no more than 3 authors, all the author names should be given, otherwise, please add “, et al” after the third author. The family name (in capital letters) is presented before the given name. Each two author names are separated by a comma. The references are placed after the main text. All the non-English references should be given corresponding English translation with the following format.
\begin{enumerate}
\item\textbf{Journal:}     Author. Paper title[J]. Journal title, year, volume (issue): start page-ending page.
\item\textbf{Monograph:}      Author. Book title[M]. Location: Publisher, year.
\item\textbf{Translation:}     Author. Book title [M]. Translator. Location: Publisher, year.
\item\textbf{Collected papers:}   Author. Paper title[A]. Editor. Paper collection[C]. Location: Publisher, year. start page~ending page.
\item\textbf{Dissertation:}  Author. Dissertation title[D]. City: Conservation organization, year.
\item\textbf{Technical standard:} Responsible drafter, Code sequence number of the technical standard—publication year. Technical standard name[S]. Location: Publisher, year. (the responsible drafter, location, publisher and year are optional)
\item\textbf{Patent:}     Applicant. Patent name[P]. Country name and patent number, publication date.
\item\textbf{Technical report:} Author. Report name[R]. Location: responsible organization, report code and number, year.
\item\textbf{Newspaper article:} Author. Article title[N]. Newspaper title, publication date (edition No.).
\item\textbf{Online document:} Author. Document title[OL].[date]. http:…
\item\textbf{CD document:} Author. Document title[CD]. Location: Publisher, publication date.
\item\textbf{Other documents:} Author. Document title[Z]. Location: Publisher, publication date.
\end{enumerate}

\section{Conclusion}

In this paper……

\section*{Acknowledgements (Optional)}

Thanks to helpful people and companies.


%----------------------------------------------------------
% 3. References
%----------------------------------------------------------

\begin{thebibliography}{2} % 6 is the quantity of the references. Please give the number according to practical reference quantity.
    \bibitem{bib1} H.E.S.Said, T.Tan and K.Baker.Personal identification based on handwriting [J].Pattern Recognition, 33:149-160, Jan. 2000
    \bibitem{bib2} SAID H E S, TAN T, BAKER K. Personal identification based on handwriting[J]. Pattern Recognition, 2000, 33(1): 149~160.
\end{thebibliography}

\end{document}
