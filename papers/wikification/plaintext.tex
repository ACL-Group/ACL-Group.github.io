\subsection{Beyond the Wikipedia Corpus}
In this paper, Wikipedia acts as a lexicon which provides all the
surface terms as well as concepts to link to for wikification.
However, even though the Wikipedia text
corpus itself is very large, it is unlikely to contain all the co-occurrence
information there is between any two concepts. Additional data sources maybe
used to provide co-occurrence evidences not seen in Wikipedia itself.
For example, suppose in the Wikipedia corpus, concepts $a$ and $b$ co-occur,
concepts $a$ and $c$ also co-occur, but there's no evidence which supports
the co-occurrence of $b$ and $c$. Now, given a new plain text document, because of the co-occurrence between $a$ and $b$ and the co-occurrence between $a$ and $c$,
we may be able to disambiguate three terms $t_a$, $t_b$ and $t_c$ to $a$, $b$,
and $c$, respectively in the document. Consequently, a new occurrence ($b$, $c$)
which was never seen in Wikipedia itself, can be discovered and added to our
co-occurrence knowledge.
%Not only Wikipedia corpus, we also consider the possibility to bring in
%other source to help us collect more co-occurrence information. Plain
%text on web is a candidate. We are interested in the word and phrase
%distribution on both Wikipedia article and plain text.

We conduct the following experiment to verify our hypothesis.
We randomly sample 10,000 web pages from a Bing snapshot and
extract plain text from them.
Then we randomly pick two groups of 10, 000 Wikipedia articles, called
Wiki-1 and Wiki-2. We compute the word distribution and the Wikipedia terms
(phrase) distributions from these three groups of text and measure the
the cosine similarity between word distributions and between
phrase distributions in Table \ref{tab:vector}.

\begin{table}[th]
\centering
\begin{tabular}{|c|c|c|}
\hline
Data Sources           &  Word Similarity &  Phrase Similarity \\
\hline \hline
Wiki-1 vs. Wiki-2 &      0.992 &       0.990 \\
Plain text vs. Wiki-1 &      0.633 &      0.765 \\
Plain text vs. Wiki-2 &      0.629 &      0.764 \\
\hline
\end{tabular}
\caption{Word and Phrase Distribution Similarity}
\label{tab:vector}
\end{table}

Table \ref{tab:vector} shows that the word and phrase distribution
between two Wikipedia sets are very similar.
Whereas both word and phrase distribution of plain text share lower
similarity with those of the two Wikipedia sets. This indicates that
data sources outside of Wikipedia do have significant differences and hence
have the potential of introducing fresh co-occurrence information into
the Wikipedia corpus.

One straightforward way of incorporating the co-occurrence data from other sources
into Wikipedia is to wikify the plain text,
calculate co-occurrence frequency between the concepts inside a window, and then
update that information into the co-occurrence matrix we obtained from
Wikipedia itself.
We use the matrix generated from 10,000 sample Wikipedia articles to wikify
10,000 other web pages. Results show that this process introduces 2,802,392
fresh pairs, which is 10.47\% of the original matrix size.

%Our iterative algorithm that enrich the co-occurrence matrix can not only
%be applied on Wikipedia articles, but also plain text, which can bring us
%more knowledge. The process is similar to what we use to wikify plain text
%document. We set a sliding window and calculate the ($S_{SW}$)s. But instead
%of finding the best sense for each term, we delete the worst sense in each
%iteration with the lowest sum of ($S_{SW}$)s.
%
%The whole process on plain text starts with an initial co-occurrence matrix,
%which can be generated by the process on Wikipedia articles. In each iteration,
%each existing sense of a term is assigned a value which is the sum of all the
%($S_{SW}$)s this sense contributes to. The sense with the lowest value is deleted.
%Once there is only one sense left for a term, or to say the sense of that term
%is fixed, we add a link to the corresponding document and update the co-occurrence
%matrix. The iterations continue until no link can be added.

