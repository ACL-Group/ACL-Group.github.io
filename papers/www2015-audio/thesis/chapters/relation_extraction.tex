\chapter{Relation of scene and event extraction using knowledge base}

There are thousands of events in our world. For a given scene, there are a log of related events. It is difficult to build a relation between audio scenes and audible events directly from audio data of a scene, because it is not easy to collect a huge number of labeled audio clips. In this chapter, we will introduce a knowledge-based method to analyze huge corpus data, aimed to detect related events of a given scene, and build a probabilistic model for scene and event.

\section{Data collection}
It is not an easy work to label a paragraph which describes a scene. Some of researchers are working on topic model, such as {\em Latent Dirichlet Allocation (LDA)} \cite{Blei:2003:LDA:944919.944937}, which can express a paragraph as a mixture of various topics. But the model is too complicated and the accuracy is not high enough.

In our work, we choose drama scripts, which already have labels written by the writers, to be our corpus. We break the drama scripts and get the paragraphs which describe the scenes we are interested in.
\section{Building probabilistic model}
To build a probabilistic model, we firstly find all the events in paragraphs. We extract all the nouns, verbs, and verb-noun pairs and then do a filter\footnote{See chapter \ref{cha:sys} for more details.}, the result is used as our events.

After get the events, we calculate the term frequency of each event for each scene, using the following equation,
\begin{equation}
tf_{ij}=\frac{F_{ij}}{\sum_{k=0}^{N-1}F_{kj}},
\label{eq:tf}
\end{equation}
where $F_{ij}$ is the number of times the $i^{th}$ event occurred in the $j^{th}$ scene, and $N$ is the number of total events. Then, we calculate the probability $p(scece_j|event_i)$. We use the term frequency to estimate this probability. Formally, if we have $M$ scenes, we build our probabilistic model using the following equation:
\begin{equation}
p(scece_j|event_i) = \frac{tf_{ij}}{\sum_{k=0}^{M-1}tf_{ik}}
\label{eq:model}
\end{equation}

