\section{Problem Statement}
\label{sec:problem}
In this section, we formulate the problem of automatic slogan generation.
The objective of the task is to build a system
that can generate slogan automatically based on the
input text.
In the basic version of the problem,
we take surface text of topic as the input.
Given the topic represent as an input sequence of words
$x = (x_1, x_2, ..., x_n) \in \mathcal{X} $,
the objective of the system is to generate the textual slogan
$y = (y_1, y_2, ..., y_m) \in \mathcal{Y}$,
a sequence of words describing the topic.
Moreover, our ultimate goal is to obtain a personalized,
accurate and informative slogan.
We introduce the item preferences to
highlight the selling points to users and promote their interests
and further provide an improved definition.

%\begin{example}
%	This is an example.
%\end{example}

\theoremstyle{definition}
\begin{definition}
	\label{item_preference}
	\textbf{Item Preference}
	The item preference $p$ associated with topic $w$,
	consists of a set of items 
	$\{s_1, s_2, ..., s_u\}$. 
%	Different item preferences of a specific topic are expected to imply different focuses or selling-points according to their item members. 
	$s_i$ is represented as a sequence tokens of its title.
	Then, $p$ is the text sequence, concatenating $s_1, s_2, ..., s_u$.
\end{definition}

Given the above definitions, our problem
can be formally defined as generating an accurate and informative
slogan $y$, based on the topic $x$ and the provided item preference
$p$.

Next, we attack the problem of slogan generation in multiple steps. 
We present the workflow of our method in \figref{fig:flow}.
Firstly, we introduce the item preference thoroughly from 
daily query logs and CPV ontology (in \figref{fig:cpv}),
then combine it with given topic using item preference fusion module.
We then enhance the representations of encoder by semantics enhancement 
module, and finally integrate a pre-trained language model
at inference time to generate the slogan.

