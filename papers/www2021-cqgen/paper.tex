%%
%% This is file `sample-sigconf.tex',
%% generated with the docstrip utility.
%%
%% The original source files were:
%%
%% samples.dtx  (with options: `sigconf')
%% 
%% IMPORTANT NOTICE:
%% 
%% For the copyright see the source file.
%% 
%% Any modified versions of this file must be renamed
%% with new filenames distinct from sample-sigconf.tex.
%% 
%% For distribution of the original source see the terms
%% for copying and modification in the file samples.dtx.
%% 
%% This generated file may be distributed as long as the
%% original source files, as listed above, are part of the
%% same distribution. (The sources need not necessarily be
%% in the same archive or directory.)
%%
%% The first command in your LaTeX source must be the \documentclass command.
\documentclass[sigconf]{acmart}

\usepackage{latexsym}
\usepackage{makecell}
\usepackage[english]{babel}
\usepackage{ulem}

%%
%% \BibTeX command to typeset BibTeX logo in the docs
\AtBeginDocument{%
  \providecommand\BibTeX{{%
    \normalfont B\kern-0.5em{\scshape i\kern-0.25em b}\kern-0.8em\TeX}}}

%% Rights management information.  This information is sent to you
%% when you complete the rights form.  These commands have SAMPLE
%% values in them; it is your responsibility as an author to replace
%% the commands and values with those provided to you when you
%% complete the rights form.
\setcopyright{acmcopyright}
\copyrightyear{2021}
\acmYear{2021}
\setcopyright{iw3c2w3}
\acmConference[WWW '21]{Proceedings of the Web Conference 2021}{April 19--23, 2021}{Ljubljana, Slovenia}
\acmBooktitle{Proceedings of the Web Conference 2021 (WWW '21), April 19--23, 2021, Ljubljana, Slovenia}
\acmPrice{}
\acmDOI{10.1145/3442381.3449876}
\acmISBN{978-1-4503-8312-7/21/04}
%%
%% Submission ID.
%% Use this when submitting an article to a sponsored event. You'll
%% receive a unique submission ID from the organizers
%% of the event, and this ID should be used as the parameter to this command.
%%\acmSubmissionID{123-A56-BU3}

%%
%% The majority of ACM publications use numbered citations and
%% references.  The command \citestyle{authoryear} switches to the
%% "author year" style.
%%
%% If you are preparing content for an event
%% sponsored by ACM SIGGRAPH, you must use the "author year" style of
%% citations and references.
%% Uncommenting
%% the next command will enable that style.
%%\citestyle{acmauthoryear}

%%
%% end of the preamble, start of the body of the document source.
\begin{document}

%%
%% The "title" command has an optional parameter,
%% allowing the author to define a "short title" to be used in page headers.
\title{Diverse and Specific Clarification Question Generation with Keywords}

%%
%% The "author" command and its associated commands are used to define
%% the authors and their affiliations.
%% Of note is the shared affiliation of the first two authors, and the
%% "authornote" and "authornotemark" commands
%% used to denote shared contribution to the research.

\author{Zhiling Zhang}
\affiliation{%
  \institution{Shanghai Jiao Tong University}
  \city{Shanghai}
  \country{China}}
\email{blmoistawinde@sjtu.edu.cn}
\orcid{0000-0002-8081-704X}

\author{Kenny Q. Zhu}
\affiliation{%
  \institution{Shanghai Jiao Tong University}
  \city{Shanghai}
  \country{China}}
\email{kzhu@cs.sjtu.edu.cn}
\authornote{Corresponding author.}
%%
%% By default, the full list of authors will be used in the page
%% headers. Often, this list is too long, and will overlap
%% other information printed in the page headers. This command allows
%% the author to define a more concise list
%% of authors' names for this purpose.
%\renewcommand{\shortauthors}{Zhang and Zhu}

%%
%% The abstract is a short summary of the work to be presented in the
%% article.
\begin{abstract}
  Product descriptions on e-commerce websites often suffer from 
missing important aspects. \textit{Clarification question generation} (CQGen) can be a promising approach to help alleviate the problem. Unlike traditional QGen assuming the existence of answers in the context and generating questions accordingly, CQGen mimics user behaviors of asking for unstated information. The generated CQs can serve as a sanity check or proofreading to help e-commerce 
merchant to identify potential missing information before advertising their
product, and improve consumer experience consequently. 
Due to the variety of possible user backgrounds and use cases, 
the information need can be quite diverse but also specific to a detailed
topic, while previous works assume generating one CQ per context and 
the results tend to be generic. We thus propose the task 
of \textit{Diverse CQGen} and also tackle the challenge of specificity. 
We propose a new model named \textit{KPCNet}, which generates CQs with 
Keyword Prediction and Conditioning, to deal with the tasks. 
Automatic and human evaluation on 2 datasets (\texttt{Home \& Kitchen}, 
\texttt{Office}) showed that KPCNet can generate more specific questions 
and promote better group-level diversity than several competing baselines. \footnote{Our code is available at \url{https://github.com/blmoistawinde/KPCNet}}
\end{abstract}

%%
%% Keywords. The author(s) should pick words that accurately describe
%% the work being presented. Separate the keywords with commas.
\keywords{clarification question, text generation, diverse generation, keyword prediction, e-commerce}

%%
%% This command processes the author and affiliation and title
%% information and builds the first part of the formatted document.
\maketitle

\section{Introduction}

Protein$-$protein interactions (PPIs) are of central importance for the majority of biological functions, such as signal transduction, metabolic pathways, molecular dynamics, and protein networks\cite{Hoffmann.Krallinger.ea:2005}, for they serve as the most fundamental building blocks of the entire interacademic systems of any organisms. Collecting data on pairwise interaction relationships is essential for multiple purpose, including identification of modules with certain functionality\cite{Spirin.Mirny.03}, mapping diseases to dominated genes\cite{Ideker.Sharan.08}, and after all, understanding wholistic metabolic/genetic networks from a system biology perspective.

A lot of databases have been built to store protein and genetic interactions from major model organism species and are available in various standardized formats, such as MINT\cite{Zanzoni.Montecchi-Palazzi.ea:2002}, BIND\cite{Bader.ea:2003}, BIOGRID\cite{DBLP:journals/nar/StarkBRBBT06}, etc. Among those mainstream databases, the data largely rely on voluntary reports by scientists or researchers, besides, comprehensive curation efforts become indispensable for the sake of accuracy. However, the amount of biology-related literatures with respect to protein interactions grows explosively and thus make it either impossible or impractical to manually detect PPI information anymore.

Considering huge amount of PPI information with great wealth hidden in published papers, in recent years, numerous mining techniques have been proposed that aim to extract PPI information automatically from free text, especially machine learning, information retrieval, and natural language processing\cite{DBLP:journals/bib/WinnenburgWPDS08}.These approaches can be roughly categorized into three classes: co$-$occurrence, rule$-$based, and machine learning. 

Co$-$occurrence is the approach with most simplicity and naivete. Just as its name implies, this method intends to find out pairs of proteins that co-occur in the same context. The scope of "same context" ranges from phrase, sentence, paragraph to whole abstract, even document. The underlying assumption is that whenever two proteins are mentioned together by authors, chances are high that there is some kind of relationship between them. However, however, in-context closeness even semantic relation does not necessarily represent actual biological interaction. As a consequence, a large fraction of candidate pairs are mismatched inevitably, causing a high recall but low precision.

The second approach is rule-based extraction, in other words, pattern matching. There are many types of rules, most of them concern natural language processing (NLP). One way is to specify hand-crafted regular expressions before hand, which mostly lean on language usage preference. Besides, by using full or partial (shallow) parsing strategies, more information would be acquired, such as part-of-speech taggers, local dependencies between syntactic components, context-free grammar\cite{DBLP:journals/bioinformatics/TemkinG03}, and full sentence structure. Compared to co$-$occurrence, rule-based approach enjoy better precision but much lower recall. In addition, since the rules are usually derived from training data, that is to say, the improper choice of training data would be significantly lethal, therefore quality of extraction is invariably instable and may not applicable to other data.

The third and most commonly used approach use machine learning techniques, in this case, the task to extract protein$-$protein interactions turns out to be a binary classification problem. Each protein pairs are represented along with a set of features, which is associated with their context, then a well$-$defined classifier gives the answer whether the candidate protein pairs is classified to be qualified PPI. (TO BE FURTHER FILLED!!!)

In this paper, we introduce a general bootstrapping framework for Protein$-$protein interaction extraction from natural text.Our method differs from most of the previous works in three aspects:

(1)The extraction process is driven by only tiny fraction of training data, which are regarded as seed data. In each round, it would derive reliable patterns automatically from seed data, then extract more positive PPI pairs consequently, what's more, the seed data would be augmented by the newly extracted results with high confidence.

(2)multiple graph kernel. 

(3)various evaluation.




\section{Preliminaries}
Our starting point is the Allamanis bimodal model~\cite{allamanis2015bimodal}. This model is constructed
from the parse tree of source code where the internal nodes are
syntactic components such as if expression and while loops,
while the leaf nodes are all the variable names.
The model $P (C~ |~ T)$ is a generative one,
where $C$ is represented by a set of features extracted
from the parse tree, e.g., the n previous internal node types that are encountered by following the path from one node to the root,
%the first $n$ nodes in the code in the left-to-right in-depth traversal, 
and $T$ is the tag phrase made up of a number of words.

The core function is $s(v,T,C_{\leq n}) = (t \bigodot c)^{\top} r + b$, where
$\bigodot$ is an element-wise multiplication and $t$ and $c$ are
the representation vectors of $T$ and $C$, respectively.
The training is done by estimating the objective function as in
Mnih et al~\cite{mnih2013learning}. Allamains et al. used NCE method~\cite{gutmann2012noise} and AdaGrad method~\cite{duchi2011adaptive} to train model.

%During the implementation, we found that the parse tree of Allamains et al's model is too simple to contain enough information of source code. So we pay attention to improving the performance of parse tree and reconstruct the parse tree.

\section{Sentiment-Aspect-Region Model}
\label{sec:model}
We first present our objectives to build the
unified sentiment-aspect-region model.
To achieve the objectives, we present several intuitions
based on which we build our model.
We then describe the details of the model,
and propose a parameter estimation method.

\subsection{Intuitions}
\label{sec:motiv}
%We first introduce some notions that are used in
%explaining our objectives. There are three types of
%latent factors that are not observable in a geotagged review corpus, but
%are important for user preference analysis. They
%are topical-region, topical-aspect and sentiment.
%A topical-region represents a geographical area in which
%users do similar things (such as dining).
%%write region-specific words on their reviews.
%It comprises two components: geo-location and semantics.
%The geo-location component is usually modeled as a
%Gaussian distribution over
%POIs \cite{Yin:2011,YuanW4:2013}.
%The semantic component is modeled as a multinomial
%distribution over words \cite{Geofolk:2010}.
%Example topical-regions include shopping areas, education areas,
%streets of special snacks, etc.
%Topical-aspects are the aspects of POIs that
%are commented by users, such as environment, taste,
%price, etc. Sentiments are user's opinions over
%topical-aspects (e.g., positive, negative or neutral).
%%\KZ{Can sentiments be casted over regions? e.g., I hate
%%Clarke Quay!}
%Topical-aspects
%and sentiments can be modeled jointly \cite{JoASUM:2011}.

In this paper,
we aim at building a model that is able to 1) extract
latent variables, i.e., topical-aspect, sentiment,
and topical-region from the review
data; 2) capture the interdependencies among
category, POI, user, words and the three latent
variables; and 3) discover user's topical-region and
topical-aspect preferences.
To achieve these objectives,
we exploit the following intuitions in designing our model:

\textbf{Intuition 1}: A user visits POIs in a topical-region
because the region is geographically convenient to the user
(e.g., close to her activity areas) and its topics (e.g., shopping
street, education area, etc.) satisfy
the user's interest. Each user has her own preferences on the
topical-regions.
%We use a topical-region
%variable $r$ to model the mixture of topic and geographic
%information,
%i.e., each region exactly covers POIs of similar
%topic distribution and close in spatial.

\textbf{Intuition 2}: A user rates highly of a POI because
she likes some aspects of the POI. Such preferences might be
indicated in her review.
%i.e., user has preferences on some aspects of the POI.
Some users like to check the price range of a restaurant first while
others might be more concerned with the environment. Moreover, POIs in different
categories may have different aspects of interest.
%For
%instance, a traveler might care more about the environment
%of a hotel, while a hungry would-be diner might be more interested in
%the waiting time of a restaurant.

\textbf{Intuition 3}:
A user decides to visit a POI in a region
by considering the category, category-aware topical-aspects of the POI and
the distance to it. For example, users may visit POIs of the
restaurant category with good environment,
but she may first consider the restaurants nearby.
%to walk around a nearby shopping street.
%and visit
%POIs without being particular about the category.

\textbf{Intuition 4}: When a user writes a review on a POI, she
will use words for both the aspects of the POI and
her sentiments about the aspects.
The user may also use words for the topical-region of the POI.
For example, a review on a shop in Times Square may say:
``This shop offers best prices in Times Square.'' The reviewer
uses ``price'' for {\em aspect}, ``best'' for {\em sentiment}
and ``Times Square'' for {\em region}. %to construct the review.
%Moreover, each sentence in the review normally
%corresponds to exactly one aspect and
%users only associate one sentiment on each aspect. As a result,
%the words co-occurs in the same sentences are more likely to be correlated to
%the same aspect and sentiment.

\subsection{Model Description}
We first define the notations
to be used in the proposed model. Let $D$ be the set of user reviews,
and $U$ be the set of users. For each review, we denote the
number of its sentences by $M$ and number of words in each
sentence by $N$. In our model, a location has two attributes:
identifier and coordinates. We use $l$ to represent a location identifier
and $\boldsymbol{cd}_l$ to denote its corresponding coordinates.
Here $\boldsymbol{cd}_l$ is a latitude and longitude pair. We denote
the topical-aspect, sentiment and topical-region by $a$, $s$,
and $r$, respectively. The notations
used in this paper are listed in \tabref{tab:notation}.
Following the intuitions discussed in \secref{sec:motiv}, we
proceed to present our model.

\begin{table}[th]
\centering
%\scriptsize
\caption{Description of Symbols}
\begin{tabular}{l|l}
\hline
 Symbol & Description\\
\hline
$u$, $U$ & individual user and the set of users\\
\hline
$l$, $L$ & individual POI and the set of POIs  \\
\hline
$c$ & category  \\
\hline
$r$ & topical-region  \\
\hline
$a$, $s$ & topical-aspect and sentiment \\
\hline
$d$, $D$ & single review and the set of reviews \\
\hline
$M$ & the number of sentences in a review \\
\hline
$w$, $N$ & single word and the number of words in a sentence \\
\hline
\end{tabular}
\label{tab:notation}
\end{table}

Based on \textbf{Intuitions 1\&2}, we model the user
topical-region preferences and topical-aspect preferences
as multinomial distributions $p(r|u)$ and $p(a|u,c)$, respectively.

Based on \textbf{Intuition 3}, a user chooses a POI to visit
by considering both the category and the distance. We
define the probability of visiting a POI $l$ given
category $c$ and region $r$ proportional to $p(l|c)\cdot p(l|r)$.
Here $p(l|c)$ is the probability of selecting POI $l$ from
the category $c$; $p(l|r)$ is a the probability
of selecting POI $l$ in region $r$ by considering
the distance from $l$ to $r$. After normalization, we have the
definition $p(l|c,r)=\frac{p(l|c)p(l|r)}{\sum_{l'}{p(l'|c)p(l'|r)}}$.
%The denominator is used to normalize
%the $p(l|c)p(l|r)$ over all POIs.
To model the spatial distance, we use a
Gaussian mixture model, i.e.,
$p(l|r)\sim N(\boldsymbol{\mu}_r, \boldsymbol{\Sigma}_r)$, where
$\boldsymbol{\mu}_r$ is the center of region $r$ and
$\boldsymbol{\Sigma}_r$ is the co-variance matrix which depicts the
area of region $r$.
To model the membership of a POI to a category, we use a uniform
distribution for $p(l|c)$.
%$\kappa$ is tunable parameter used
%for balance the weights of generating POI from category and region.
%Note that $p(l|r)$ is a continuous distribution while $p(l|c)$ is
%a discrete distribution.
%To multiply the two distributions,
%we adopt the coordinate transformation approach for the Gaussian
%distribution that is proposed in Yuan et al.\cite{YuanW4:2013}.

Based on \textbf{Intuition 4},
we model the relationships among words, topical-aspects,
sentiments and topical-regions by
$p(w|a,s,r)=\lambda p(w|a,s)+(1-\lambda) p(w|r)$, where
$a$, $s$, $r$ are topical-aspect, sentiment and
topical-region, respectively.
Here $p(w|a,s)$ is the probability that the users write
word $w$ when they have sentiment $s$ on aspect $a$;
$p(w|r)$ is the probability that the users use word
$w$ to describe region $r$; parameter
$\lambda$ is used to balance the portion of
words drawn from topical-aspect, sentiment or topical-region.
We model $p(w|a,s)$ instead of $p(w|a)$ and $p(w|s)$
because aspects and sentiments are closely coupled,
and modeling by $p(w|a)$ and $p(w|s)$
needs an additional tuning parameter.
Similar to proposals of sentence level sentiment analysis
\cite{TitovMGLDA:2008,TitovMAS:2008, JoASUM:2011},
we assume each sentence expresses opinions on exactly one topical-aspect
and each topical-aspect is associated to a positive, negative or neutral sentiment.

\begin{figure}[th]
\centering
\epsfig{file=fig/modeldraft.eps,width=0.65\columnwidth}
\caption{Sentiment-Aspect-Region Model (SAR)}
\label{fig:model}
\end{figure}

In summary, the graphical representation of our model
is shown in \figref{fig:model} and
the generative process of the
reviews written by user $u$ is described as follows:
\begin{itemize}
\item For each review $d\in D_u$, where $D_u$ is the set of reviews written by user $u$.
    \begin{itemize}
    \item Draw topical region $r\sim p(r|u)$
    \item Draw category $c\sim p(c|u)$
    \item Draw location $l\sim p(l|c,r)=\frac{p(l|r)p(l|c)}{\sum_{l'}{p(l'|c)p(l'|r)}}$, where $p(l|r)\sim N(\boldsymbol{\mu}_r,\boldsymbol{\Sigma}_r)$
    \item For each sentence in review $d$
        \begin{itemize}
        \item Draw aspect $a\sim p(a|u,c)$
        \item Draw sentiment $s\sim p(s|a,l)$
        \item For each word position in the sentence
            \begin{itemize}
            \item Draw word $w\sim p(w|a,s,r)={\lambda}p(w|a,s)+(1-\lambda)p(w|r)$
            \end{itemize}
        \end{itemize}
    \end{itemize}
\end{itemize}

In the model, $p(l|c)$ and
$p(c|u)$ can be estimated directly from a given corpus. The
other distribution parameters need to be inferred.
We first present how to estimate $p(l|c)$ and
$p(c|u)$, and then show the inference algorithm for
the remaining distributions in \secref{sec:infer}.

As described in \textbf{Intuition 3},
a POI $l$ is generated from both category
and region. Since POI $l$ and category $c$ are
observable variables, we simply compute $p(l|c)$
by \equref{eq:plc}.
\begin{equation}
p(l|c)=\frac{I(l,c)}{\#\; of\; POIs\; in\; c}
\label{eq:plc}
\end{equation}
\begin{equation}
I(l,c)=
\begin{cases}
1 & l\in c \\
0 & otherwise \\
\end{cases}
\end{equation}

Similarly, we compute the category preferences of each user, i.e., $p(c|u)$,
directly from the corpus. To handle the overfitting problem,
we apply the additive smoothing technique. After smoothing, even though a user did
not a visit some category of POIs, the probability of
visiting that category still has a small value. The computation of $p(c|u)$ is shown in
\equref{eq:pcu}.
\begin{equation}
p(c|u)=\frac{n_c+\alpha}{N+{\alpha}C},
\label{eq:pcu}
\end{equation}
where $n_c$ is the number of reviews of POIs in category $c$ that user $u$
writes; $N$ is the total number of reviews on POIs in $c$; $C$
is the total number of categories; $\alpha$  is the smoothing
parameter which is usually set to a value smaller than 1. In this paper,
we set $\alpha=0.1$.

\subsection{Inference Algorithm}
\label{sec:infer}
To infer the parameters of the model, we
use the expectation-maximization (EM) approach.
In this section,
we present the computation of the corpus
likelihood, the two-step EM algorithm
used to infer our parameters, and
initialization of the EM algorithm.

\subsubsection{Likelihood Computation}
Our model has several levels, i.e., word level,
sentence level, and document level. The latent variables
are on two levels. Region $r$ is at document level while
aspect $a$ and sentiment $s$ are at sentence level.
This multi-level structure poses challenges to the estimation of
the log-likelihood. According to the generative
process, we have the likelihood of the corpus $D$:
\begin{equation}
p(D;\Phi)=\prod_{d}^{D}{p(u_d)\sum_{r}^{R}{p(r|u_d)}p(l_d,\mathbf{w}_d|r,u_d)}
\label{eq:likeli1}
\end{equation}
\begin{equation}
p(l_d,\mathbf{w}_d|r,u_d)=p(c_{l_d}|u_d)p(l_d|r,c_{l_d})\prod_{i}^{M}{p(\mathbf{w}_{d,i}|c_{l_d},r,u_d,l_d)}
\label{eq:likeli2}
\end{equation}
\begin{equation}
\begin{split}
&p(\mathbf{w}_{d,i}|c_{l_d},r,u_d,l_d) \\
&=\sum_{a,s}{p(a|c_{l_d},u_d)p(s|a,l_d)\prod_{j}^{N}{p(w_{d,i,j}|a,s,r)}}
\end{split}
\label{eq:likeli3}
\end{equation}
In \equref{eq:likeli1}, $\Phi$ is the set of parameters in the model,
i.e., $p(r|u)$, $p(a|c,u)$,$p(l|r)$,$p(s|a,l)$,$p(w|a,s)$,
$p(w|r)$,$\boldsymbol{\mu}_r$ and $\boldsymbol{\Sigma}_r$.
Variables $u_d$,$l_d$,$\mathbf{w}_d$ are the user, location and
words of review $d$, respectively. Variable $\mathbf{w}_{d,i}$
represents the set of words in sentence $i$ of review $d$
while $w_{d,i,j}$ is the $j^{th}$ word in sentence
$i$ of review $d$. Taking logarithm of
$p(D;\Phi)$ leads to a summation inside the logarithm:
\begin{equation}
L=\sum_{d}{\log{p(u_d)}+\log{\sum_{r}{p(r|u_d)p(l_d,\mathbf{w}_d|r,u_d)}}}
\label{eq:loglikeli}
\end{equation}
Since this likelihood cannot be estimated directly,
we adopt Jessen's
inequality to the log-likelihood, and estimate the
lower bound of the likelihood and the parameters
in an iterative manner.

\subsubsection{Expectation-Maximization}
Due to the aforementioned difficulty of computing
log-likelihood directly,
we apply Expectation-Maximization (EM)
algorithm to estimate the model parameters.

In \textbf{E-step}, we compute the expectation
of latent variables given the observed data.
By applying Jessen's inequality to \equref{eq:loglikeli},
we get the lower bound of the likelihood as:
\begin{equation}
\begin{split}
L_{LB}=&\sum_{d}{\log{p(u_d)}}\\
+&\sum_{d,r}{p(r|d)(\log{p(r|u_d)}+\log{p(l_d,\mathbf{w}_d|r,u_d)})}
\end{split}
\label{eq:loglikeli1}
\end{equation}
As shown in \equref{eq:loglikeli1},
we need to estimate $p(r|d)$ to compute the full likelihood.
We apply Bayes rule, and obtain the update
function of the posterior distribution as
\begin{equation}
p(r|d)=\frac{p(r,d)}{\sum_{r}{p(r,d)}}
\label{eq:prd}
\end{equation}
\begin{equation}
p(r,d)=p(u_d)p(r|u_d)p(l_d,\mathbf{w}_d|r,u_d)
\label{eq:prdjoint}
\end{equation}
In \equref{eq:prdjoint},
$p(l_d,\mathbf{w}_d|r,u_d)$ is computed by \equref{eq:likeli2}, and
$p(u_d)$ appears both in the numerator and the denominator,
and thus is not necessary to estimate.

In \textbf{M-step}, by maximizing the lower bound of likelihood,
we can obtain the update function of parameters at document level
that are related to topical region $r$ as below.
\begin{equation}
p(r|u)=\frac{\sum_{d\in D_u}{p(r|d)}}{\sum_{r}{\sum_{d\in D_u}{p(r|d)}}}
\label{eq:pru}
\end{equation}
%\begin{equation}
%\boldsymbol{\mu}_r=\frac{\sum_{d}{p(r|d)\cdot \boldsymbol{cd}_{l_d}}}{\sum_{d}{p(r|d)}}
%\label{eq:mu}
%\end{equation}
%\begin{equation}
%\boldsymbol{\Sigma}_r=\frac{\sum_{d}{p(r|d)\cdot (\boldsymbol{cd}_{l_d}-\boldsymbol{\mu}_r)^T(\boldsymbol{cd}_{l_d}-\boldsymbol{\mu}_r)}}{\sum_{d}{p(r|d)}}
%\label{eq:sigma}
%\end{equation}

However, we cannot obtain a close form solution for $\boldsymbol{\mu}_r$ and
$\boldsymbol{\Sigma}_r$ due to the normalization term. We adopt a gradient method
to obtain the update value of $\boldsymbol{\mu}_r$ and $\boldsymbol{\Sigma}_r$ in M-step.
Specifically, we use the BFGS quasi-Newton method \cite{Kurashima:2013,Liu:1989}.
In the gradient method, we compute the gradient of $\boldsymbol{\mu}_r$ and
$\boldsymbol{\Sigma}_r$ as follows:
\begin{equation}
\frac{\partial L_{LB}}{\partial \boldsymbol{\mu}_r}=
\sum_d{p(r|d)\boldsymbol{\Sigma}_r^{-1}\left(\frac{\sum_{l'}{q(l')(\boldsymbol{cd}_{l'}-\boldsymbol{\mu}_r)}}{\sum_{l'}{q(l')}}-
(\boldsymbol{cd}_{l_d}-\boldsymbol{\mu}_r)\right)}
\label{eq:gmu}
\end{equation}
\begin{equation}
\frac{\partial L_{LB}}{\partial \boldsymbol{\Sigma}_r}=\sum_d{p(r|d)(\frac{\sum_{l'}{q(l')g(l', r)}}{\sum_{l'}{q(l')}}-g(l_d, r))}
\label{eq:gsigma},
\end{equation}
%\begin{equation}
%g(l, r)=-\frac{1}{2}\boldsymbol{\Sigma}_r^{-1}+\frac{1}{2}\boldsymbol{\Sigma}_r^{-1}(\boldsymbol{cd}_{l}-\mu_r)(\boldsymbol{cd}_{l}-\mu_r)^T\boldsymbol{\Sigma}_r^{-1}
%\end{equation}
where $q(l')=p(l'|c_l)p(l'|r)$ and $\boldsymbol{cd}_{l}$ denotes the coordinates of POI $l$.
The function $g(l, r)$ in \equref{eq:gsigma} is the gradient of the Gaussian distribution for region $r$
w.r.t. $\boldsymbol{\Sigma}_r$ at point $l$.

Since sentiment and aspect are at the sentence level, we
cannot compute $\log p(l_d,\mathbf{w}_d|r,u_d)$
in \equref{eq:loglikeli1} using $p(r|d)$. 
Thus, we propose a second level of EM iterations.
Specifically, we introduce a new latent variable to estimate parameters related to
aspect and sentiment. Specifically, we use $\phi_{a,s,r,d_i}$ to identify
the probability that the $i^{th}$ sentence in a review $d$ from
region $r$ is assigned with aspect $a$ and sentiment $s$.
we use $\phi_{a,s,r,d_i}$ and $p(r|d)$ to compute the update
function of $p(a|c,u)$, $p(s|l,a)$,
$p(w|a,s)$, and $p(w|r)$.

Denote by $n(w,d_i)$ the number of occurrences of word $w$ in sentence $i$
of review $d$. We estimate $\phi_{a,s,r,d_i}$ as:
\begin{equation}
\phi_{a,s,r,d_i}=\frac{p(a,s,r,d_i)}{\sum_{a,s}{p(a,s,r,d_i)}}
\label{eq:pasrdi}
\end{equation}
\begin{equation}
\begin{split}
p(a,s,r,d_i)=p(u_d)p(r|u_d)p(c_{l_d}|u_d,r)p(l_d|r,c_{l_d})\\
p(a|c_{l_d},u_d)p(s|a,l_d)\prod_{w}{p(w|a,s,r)^{n(w,d_i)}}
\end{split}
\end{equation}

By maximizing the lower bound of the likelihood, we
obtain the update function of the rest parameters:
\begin{equation}
p(a|u,c)=\frac{\sum_{d\in D_u}{\sum_{r}{p(r|d)\sum_{i}{\sum_{s}{\phi_{a,s,r,d_i}}}}}}{\sum_{a'}{\sum_{d\in D_u}{\sum_{r}{p(r|d)\sum_{i}{\sum_{s}{\phi_{a',s,r,d_i}}}}}}}
\label{eq:pacu}
\end{equation}
\begin{equation}
p(s|l,a)=\frac{\sum_{d\in D_l}{\sum_{r}{p(r|d)\sum_{i}{\sum_{s}{\phi_{a,s,r,d_i}}}}}}{\sum_{s'}{\sum_{d\in D_l}{\sum_{r}{p(r|d)\sum_{i}{\sum_{s}{\phi_{a,s',r,d_i}}}}}}}
\label{eq:psal}
\end{equation}
\begin{equation}
p(w|s,a)=\frac{\sum_{d}{\sum_{r}{p(r|d)\sum_{i}{\phi_{a,s,r,d_i}n(w,d_i)}}}}{\sum_{w'}{\sum_{d}{\sum_{r}{p(r|d)\sum_{i}{\phi_{a,s,r,d_i}n(w',d_i)}}}}}
\label{eq:pwsa}
\end{equation}
\begin{equation}
p(w|r)=\frac{\sum_{d}{p(r|d)\sum_{i}{\sum_{a}{\sum_{s}{\phi_{a,s,r,d_i}n(w,d_i)}}}}}{\sum_{w'}{\sum_{d}{p(r|d)\sum_{i}{\sum_{a}{\sum_{s}{\phi_{a,s,r,d_i}n(w',d_i)}}}}}},
\label{eq:pwr}
\end{equation}
where $D_u$ is the set of reviews written by user $u$ and $D_l$
is the set of reviews for POI $l$.

\subsubsection{Initialization of EM Algorithm}
EM algorithm can only guarantee to find a local optima.
Different initializations may lead to different results.
In this section, we present our methods for initializing the assignment of
aspect, sentiment and region.

\textbf{Aspect} is extracted from sentence level in our model.
We initialize the aspect by a clustering process on
sentences. Each sentence is represented as a vector of words.
Given the number of aspects, we use K-means clustering
algorithm to assign each sentence an aspect.
We then initialize $p(w|a)$ by the probability that word
$w$ appears in sentences carrying aspect $a$.

\textbf{Sentiment} has 3 possible values in this paper:
positive, negative and neutral.
In order to know the polarity of each sentiment, we need some prior
knowledge. We use the same predefined set of sentiment seed words
as in Jo's proposal \cite{JoASUM:2011}. Moreover, we apply a syntactic parser to
extract negation of the sentiment words such as ``not good'' and
use a special word ``not\_good'' to represent the phrase ``not good''
in our vocabulary. For each word in the seed word set, we assign
a probability ($p(w|s)$) of 1 to its polarity and 0 to the other
two polarities. For words not in the seed word set, we assign an
equal probability for each polarity. We then use $p(w|a)p(w|s)$
to approximate $p(w|a,s)$.

\textbf{Region} is initialized by a K-means clustering
algorithm based on the coordinates (latitude and longitude).
The clustering algorithm partitions POIs to different
regions. Then for each region r, we compute $\boldsymbol{\mu}_r$
and $\boldsymbol{\Sigma}_r$ using a regression
over the POIs in the region.
We compute $p(w|r)$ by the distribution of
words in the reviews for POIs in region $r$ and $p(r|u)$ by the
portion of reviews that user $u$ writes in region $r$.

For other parameters: $p(a|c,u)$ and $p(s|a,l)$, we initialize
them by using the assignment of aspect and sentiment to a sentence
(We assign sentiment to a sentence by voting from sentiment seed words
extracted from the sentence). Specifically, $p(a|c,u)$ is proportional to the
number of sentences that are assigned to $a$ and that belong to a review
written by $u$ from category $c$; $p(s|a,l)$ is proportional to
the number of sentences that belong to location $l$ and
are assigned to sentiment $s$ and aspect $a$ at the same time.

\subsubsection{Efficiency Analysis}
Let the number of sentiment be 3 and we treat it as
constant. In E-step,
the computation of the expectation of latent variables in \equref{eq:prd}
and the variables $\phi_{a,s,r,d_i}$ in \equref{eq:pwr}
needs $O(|D|MNRA)=O(WRA)$, where $W$ is the number of words in the reviews of all
users' in training set,
$R$ is the number of regions and $A$ is the number of aspects.
In M-step, the cost for updating \equref{eq:pacu} to (\ref{eq:pwr})
is $O(UA+LA+VA+VR)$,
where $U,L,V$ are the number of users, POIs and unique words, respectively.
To update $\boldsymbol{\mu}$ and $\boldsymbol{\Sigma}$, we perform a
quasi-Newton method. Since each $\boldsymbol{\mu}_r$ and $\boldsymbol{\Sigma}_r$
are two dimensional vector and $2\times2$ matrix, respectively. The computation cost of matrix operation
can be treated as constant. Let $D$ be the number of reviews, the cost of
computing gradient in \equref{eq:gmu} and (\ref{eq:gsigma})
is $D+L$.
Therefore, the complexity of quasi-Newton is $O(I_qR(D+L))$, where $I_q$
is the number of iterations of quasi-Newton.
In summary, the total complexity of the learning
algorithm with $I$ iterations is $O(I(WRA+I_qR(D+L)+UA+LA+VA+VR))$.
Since $WRA\gg (UA+LA+VA+VR)$, we simplify the cost as $O(I(WRA+I_qR(D+L)))$.
%The training complexity is high, but
%fortunately, the training process can be done offline,
We can parallelize the computation
of both E-step and M-step. In E-step, since
the computation of $p(r|d)$ on each document is independent to others, we can compute $p(r|d)$
of each document in parallel. In M-step, the update of \equref{eq:pacu} to (\ref{eq:pwr}) and
the quasi-Newton iterations can also be
parallelized in the similar way as $p(r|d)$. Therefore, the algorithm can be fully parallelized.

\section{Applications}
\label{sec:app}
%Our model can be applied to POI recommendation and user recommendation.
%We show in detail how to use the estimated parameters for recommendation.
We present three applications of our model, namely POI recommendation,
user recommendation, and aspect satisfaction analysis in regions. In POI recommendation,
we provide a way to explain the reason of recommending a POI and
propose an efficient online recommendation algorithm.
% region-aware users' satisfaction estimations.

\subsection{POI recommendation}
\label{sec:model-poirec}
%Most of the existing proposals for POI recommendation are
%based on collaborative filtering.
%Ye et al. \cite{YeGeoSocial:2011} propose a fusion framework to
%combine user-based, friend-based and geo-based collaborative
%filtering. In the geographic model, the probability of transporting
%from one POI to another is drawn from a power law distribution over
%the distances between the two POIs. The probability of a user
%visiting a POI is given by considering the distances between the
%POI and the POIs visited by the user. Yuan et al.
%\cite{YuanPOI:2013} propose a time-aware model
%for recommendation where check-ins
%are divided into different groups by different time segments to
%model user interests by time. Yang et al. \cite{YangSenti:2013}
%propose a sentiment-enhanced location recommendation
%system. They combine both check-ins and
%the overall sentiment on each location
%and apply probabilistic matrix factorization for recommendation.
%Different from these proposals, our model recommends POIs based
%on user topical-aspect preferences, topical-region preferences
%and the aspect-level sentiment of the POIs.

We apply our model to two POI recommendation tasks
and propose an efficient online recommendation algorithm.
The two recommendation tasks are \emph{All-Category Recommendation}
and \emph{Single-Category Recommendation}.

\subsubsection{All-Category Recommendation}
All-Category Recommendation is a task of
generating a rank list of POIs in any category
given a set of POIs and a user.
%When
%a user wants to visit a place without specifying the category,
%she needs recommendation from all of the categories.
The aforementioned proposals are all for all-category recommendation.
We calculate the
probability of $p(l,s_+|u)$, i.e., the probability of user
$u$ visits POI $l$ with positive sentiment, to score $l$ for $u$
as shown in \equref{eq:poiacr}.
\begin{equation}
\begin{split}
p(l,s_+|u)=&\sum_{r}{p(r|u)p(c_l|u)p(l|r,c_l)}\\
&\sum_{a}{p(a|u,c_l)p(s_+|a,l)}
\label{eq:poiacr}
\end{split}
\end{equation}
According to \equref{eq:poiacr}, we make the recommendation
by considering the matching between user preferences (i.e., $p(r|u)$,
$p(c_l|u)$ and $p(a|u,c_l)$) and the attributes of the POI
(i.e., $p(s_+|a,l)$ and $p(l|r,c_l)$).
%Only when the location satisfy the preference, i.e., the probabilities
%$p(s_+|a,l)$ and $p(l|r)$ are high for the user's preferred aspects $a$
%and region $r$, will $l$ be probably visited
%and satisfied by user $u$.
%In summary, our model considers aspect($p(a|u,c)$),
%sentiment($p(s_+|a,l)$) and region($p(r|u)p(l|r)$) when
%giving a recommendation.

This recommendation model enables us to explain why we recommend
a POI to a user. We consider two factors: aspect and region.
First, we rank the aspects by $p(s_+|a,l)p(a|u,c_l)$ to reveal
which aspects match the user's preferences.
Second, we rank the regions by $p(r|u)p(l|r)$ to reveal which regions
contribute more to the recommendation. Finally, we choose top several
aspects and regions for explanation.

\subsubsection{Single-Category Recommendation}
Single-Category Recommendation aims at
ranking POIs given a user and
a specific category (e.g., restaurants).
It is a typical scenario for POI recommendation
although it has not been covered in previous work.
We compute $p(l,s_+|u,c)$ as shown in \equref{eq:poiscr}.
Compared to all-category recommendation, we fix the category
i.e., remove $p(c|u)$ from \equref{eq:poiacr}.
All locations that are not in $c$ will not be
considered in this scenario.
\begin{equation}
\begin{split}
p(l,s_+|u,c)=&\sum_{r}{p(r|u)p(l|r,c)}\\
&\sum_{a}{p(a|u,c)p(s_+|a,l)}
\label{eq:poiscr}
\end{split}
\end{equation}
We can also offer explanation for the single-category recommendation
by following similar method as we employ for the all-category recommendation.

\subsubsection{Efficient algorithm for Top-N Online Recommendation}
Time efficiency is an essential part of online recommendation. A straightforward
method of making recommendation is to compute the recommendation score as \equref{eq:poiacr}
or \equref{eq:poiscr}.
This method requires traversing all the regions which is highly time consuming.
Another choice is the threshold algorithm \cite{FaginTA:2001}
that may save the computation for some POIs.
However, in our applications, the
number of attributes (i.e., regions and aspects) is large, and thus it is expensive
to compute the recommendation score even for a single POI.
The threshold algorithm cannot help with this, either.
We propose an optimized top-N items recommendation algorithm that significantly
reduces the time cost. As to be shown in the experiment,
our algorithm is faster than the threshold algorithm
in the top-N POI recommendation using our model. Our algorithm
can be applied to all or single-category POI recommendation. % as well as user recommendation.
We use all-category POI recommendation (\equref{eq:poiacr})
as an example to explain the algorithm.

Our algorithm is based on two observations:
1) A user only prefers a small number of regions;
and 2) POIs in the center of the
region are more likely to be recommended. These two observations indicate that only when
a user prefers a region and the POI is near the center of the region, will the score
$p(r|u)p(l|r,c_l)$
contribute significantly to the recommendation score.
Therefore, after we have computed the most possible regions for a POI,
it may not be necessary to compute the remaining regions.
We design a branch and bound algorithm as shown in Algorithm \ref{oprec}
to prune the search space of the regions.
Our algorithm contains two steps: \emph{initialization} and \emph{pruning}.
%By using the second observation, we can produce a
%good initial top-N list.

%Consider the POI recommendation mentioned in \equref{eq:poiacr}.
In the \emph{initialization} step (line 2),
we find $N$ candidate POIs that are potentially
good for recommendation.
Specifically, we pick top $K$ regions which
cover most of the user's regional preferences
(i.e., $\sum_{i=1}^{K}{p(r_i|u)}>0.9$) with smallest $K$ (line 21).
If $K$ is larger than $N$, we pick at most $N$ regions
to ensure that we can select at least one candidate from each region.
In each of the top $K$ region, we choose top $\ceil*{\frac{N}{K}}$
POIs w.r.t. $p(l|r)$ as candidates.

In the \emph{pruning} step (line 9-10),
we check whether we can avoid traversing unnecessary regions for each POI.
%we check whether there is a POI that
%has a larger recommendation score than the smallest one in the candidate set.
%To compute the recommendation score, we need to traverse all regions
%to sum up $p(r|u)p(l|r,c_l)$ for each POI in the straightforward method.
We traverse the regions according to
the descending order of $p(l|r,c_l)$ for POI $l$. Suppose we have traversed regions
$\{r_1,...,r_{i-1}\}$. The partial score we have computed for the traversed regions is
\[PScore=\sum_{j=1}^{i-1}{p(r_j|u)p(l|r_j,c_l)}.\]
When we explore the i-th region, we compute the upper bound of
recommendation score for the POI as:
\begin{equation}
\label{eq:bound}
Bound^{(i)}(l)=PScore+(1-\sum_{j=1}^{i-1}{p(r_j|u)})p(l|r_i,c_l).
\end{equation}
%and $\sum_a{p(s_+|a,l)p(a|u,c_l)}=1$

Because we check the regions in the descending
order of $p(l|r,c_l)$, the actual value of $p(l|r,c_l)$
for the remaining regions should be less than the one
for the current region, i.e., $p(l|r_i,c_l)$.
Therefore, we have a partial recommendation
score for the rest of the regions, which is at most
\[(1-\sum_{j=1}^{i-1}{p(r_j|u)})p(l|r_i,c_l),\]
where
$1-\sum_{j=1}^{i-1}{{p(r_j|u)}}$ is the portion of user preferences for
the rest regions. The upper bound of
$\sum_r{p(l|r,c)p(r|u)}$ for all regions is
$PScore+(1-\sum_{r=r_1}^{r_{i-1}}{p(r|u)})p(l|r_i,c_l)$.
Since $\sum_a{p(a|u,c)p(s_+|a,l)}\le1$,
Finally, we obtain the upper bound of the recommendation score in \equref{eq:poiacr}
for the POI
by setting $\sum_a{p(a|u,c)p(s_+|a,l)}=1$, which results in
\equref{eq:bound}.

If the upper bound
is smaller than the $N^{th}$ candidate (Line 9),
we skip the current POI (no need to check the remaining regions).
Otherwise, we continue to check the
remaining regions.
If all regions are examined for the POI and the POI is not pruned by the aforementioned
upper bound, we compute the full score
of the POI to compare with the $N^{th}$ smallest candidate (line 12).
We remove the $N^{th}$ candidate
in the list and insert the POI to the list if the full score is
larger than the $N^{th}$ candidate (line 13-15). To maintain the
top-N candidate list, we use a binary min-heap.
%Details are shown in Algorithm \ref{oprec}.

\begin{algorithm}[th]
\caption{POI Recommendation}
\label{oprec}
\begin{algorithmic}[1]
\Function{Rec}{u, N}
\State {$H \leftarrow InitialCandidates(N)$}
\For {$l\in L\;and\;l\not\in H$}
\State $PartS \leftarrow 0, PartRPro\leftarrow 0, Skip\leftarrow false$
\While {there exists $r$ not examined for $l$}
\State {$r\leftarrow NextRegion()$}
\State {$PartS\leftarrow PartS+ p(r|u)p(l|r,c_l)$}
\State {$PartRPro\leftarrow PartRPro+ p(r|u)$}
\If {$PartS+(1-PartRPro)*p(l|r,c_l)<H.Top()$}
\State $Skip\leftarrow true, break$
\EndIf
\EndWhile
\If {$Skip=false$}
\State {$S\leftarrow PartS * p(c_l|u)\sum_a{p(s_+|a,l)p(a|u,c_l)}$}
\If {$S>H.Top()$}
\State {$H.DeleteTop()$}
\State {$H.Insert(<l,S>)$}
\EndIf
\EndIf
\EndFor
\State $Result\leftarrow\; Sort\; H\; by\; Score\; S$
\State \textbf{return} $Result$
\EndFunction
\Statex
\Function{InitialCandidates}{N}
\State {$H\leftarrow \emptyset$}
\State $r_1,...,r_R\leftarrow$ Sort the regions by $p(r|u)$
\State Pick top $K$ regions satisfies: $K=min(\{k|\sum_{i=1}^{k}{p(r_i|u)}>0.9\},N)$ %1) $\sum_{i=1}^K{p(r_i|u)}>0.9$; or 2) $K=N$
\State From $r_1$ to $R_K$, Insert top $\ceil*{\frac{N}{K}}$ POIs ordered by $p(l|r)$ to $H$ until $H$ contains $N$ POIs
\State \textbf{return} $H$
\EndFunction
\end{algorithmic}
\end{algorithm}

\subsection{User Recommendation}
We can also apply our model to recommend
users for a POI. Predicting which users may favor a given
POI is useful when the owner of the POI wants to target at or advertise
to some of the users.
%The users who favor the POI probably
%write positive reviews to the POI, which may then increase the
%overall ratings and attracts more users. User recommendation
%can be treated as an inverse process of POI recommendation.
%The goal is to generate a user list for
Given a POI $l$, we
compute the probability $p(u,s_+|l)$
of user $u$ favoring POI $l$ by considering both topical-region
and topical-aspect preferences of users as follows:
\begin{equation}
p(u,s_+|l)=\frac{p(u,s_+,l)}{\sum_{u,s}{p(u,s,l)}}
\label{eq:puspl}
\end{equation}
\begin{equation}
\begin{split}
p(u,s,l)=&p(u)p(c_l|u)\sum_{r}{p(r|u)p(l|r,c_l)}\\
&\sum_{a}{p(a|u,c_l)p(s|a,l)},
\end{split}
\label{eq:pusl}
\end{equation}
%Similar to POI recommendation, we also consider user's
%topic and aspect preferences. In additional to the two
%preferences, the computation in \equref{eq:pusl}
%involves $p(u)$ which
%is exactly proportional to contribution of user $u$.
%A user who is active to write reviews are more likely to
%write reviews to POI that she visits next. This user
%should be recommended to the POI with higher probability
%than inactive users.
where prior $p(u)$ is calculated
using the user's review history:
\[p(u)=\frac{\#\; of\; reviews\; u\; wrote}{\#\; of\; all\; reviews}.\]
Since the last two summations are the same as those in
POI recommendation,
Algorithm \ref{oprec} can also be used to speed up the user recommendation.

\subsection{Aspect Satisfaction Analysis in Regions}
\label{sec:asr}
Discovering which aspect is satisfied or not by
users in each region is useful when 1) someone wants to
set up a new business or make strategies to attract more customers,
or 2) policy makers make urban planning.
For example, most of the restaurants in a region
of a city may be complained for the
long waiting time. By knowing the dissatisfaction of this aspect,
a restaurant may think how to achieve competitive
advantage over other restaurants in the region.
We can infer the aspect satisfaction
in each region based on our model. Specifically, we compute the
aspect distribution of each sentiment $s$, category $c$ and
region $r$ as
\begin{equation}
%p(s|a,c,r)=\sum_{l}{p(s|a,l)p(l|r)p(l|c)}
p(a|s,c,r)=\frac{\sum_{u,l}{p(u)p(r|u)p(c|u)p(a|c,u)p(l|r,c)p(s|a,l)}}{
\sum_{a,u,l}{p(u)p(r|u)p(c|u)p(a|c,u)p(l|r,c)p(s|a,l)}}
\label{eq:sat}
\end{equation}
This probability shows which aspect is most probably liked/disliked
in POIs from category $c$ and region $r$.
%The weighted-sum of user
%aspect preferences $p(a|c,u)$
%in \equref{eq:sat} give higher weight for aspect that often mentioned
%by users who are active in the region.


\section{Experiments}
\label{sec:experiments}

% \subsection{Datasets and Metrics}

We experiment our methods on datasets with corresponding ontologies to verify the effectiveness of either ontological or temporal knowledge alone, and their combination.

\textit{AudioSet} is a large-scale multi-label audio tagging dataset collected from Youtube videos with the annotations of 527 categories out of the 632 tags defined in the ontology. Most recordings are processed into 10 seconds single-channel 16kHz, 16-bit wave format. Due to the changes of videos, it is not possible to recover the whole dataset. We downloaded 19,400 (87.5\%), 1,851,420 (90.7\%), and 17,756 (87.2\%) recordings for the balanced train, full train, and evaluation set, respectively. To simulate the low-resource scenario, we randomly sample 1\% of the unbalanced set as the training set, which has 18,514 samples, where 134 classes have no more than 5 samples, and 10 classes have no training sample. We also sample 5\% and 10\% sets for comparison. We use the commonly used mAP, mAUC as the evaluation metrics. 
\textit{SONYC} is the multi-label Urban Sound Tagging dataset used in DCASE 2019 Task5 (D19T5). It contains 2,351 train recordings and 443 validate recordings, which can be considered relatively low-resource. All recordings are 10 seconds single-channel 44.1kHz, 16-bit wave format. We use the official metrics, micro AUPRC and macro AUPRC.

\subsection{Experimental Setup}

\paragraph{Audio Features} For SONYC, we adopt a similar setting as \citep{kong2019cross}, all audios are re-sampled to 32 kHz and 64-Mel-bin log-Mel spectrograms are used to to represent the audios. The window size is 1024 samples, the hop size of 500 samples, and cut-off frequencies of 50 Hz to 14 kHz. 
For AudioSet, our setting is similar to \citep{kong2020panns}, all audios are re-sampled to 16 kHz and represented as 64-Mel-bin log-Mel spectrograms. The window size is 512 samples, the hop size of 160 samples, and cut-off frequencies of 50 Hz to 8 kHz.

\paragraph{Models and Baselines} We adopt standard CNN models as our baseline, that is, the CNN9 model use in \citep{kong2019cross} for SONYC and the CNN14 (16kHz) in \citep{kong2020panns} for AudioSet. We also introduce the SOTA method AT-GCN \citep{wang2020modeling} as baseline, which is based on co-occurrence graph mined from the whole AudioSet annotation, and uses tuned, dataset-specific hyperparameter for edge thresholding and smoothing. It is thus not directly applicable to SONYC. Our models include GCN(ASER), GCN(AudioSet), GCN(ASER+AudioSet), which refers to single GCN with temporal knowledge, AudioSet ontology, and their combination. D-GCN denotes double-GCN with 2 types of knowledge. We replace AudioSet with SONYC's ontology (OT) for experiments on SONYC dataset. We use batch size of 32 for all models and the learning rate is 1e-3 for all models except D-GCN using 3e-4.

\subsection{Results}

\subsubsection{AudioSet}


\begin{table}[tbp]
  \setlength{\belowcaptionskip}{-0.cm}
  \centering
  \small
  \begin{tabular}{lcccc}
      \hline
      {} & \multicolumn{2}{c}{Balance} & \multicolumn{2}{c}{Unbalance (100\%)} \\
      \cline{2-3}\cline{4-5} 
      Methods & mAP & mAUC & mAP & mAUC \\
      \hline
      CNN14 & 0.2441 & 0.8930 & 0.4090 & 0.9669 \\
      \hline
      AT-GCN & 0.2510 & 0.9278 & 0.4095 & 0.9664 \\
      GCN(ASER) & 0.2500 & 0.9283 & 0.3994 & 0.9660 \\
      GCN(AudioSet) & 0.2543 & \textbf{0.9420} & 0.4063 & 0.9665 \\
      GCN(ASER+AudioSet) & 0.2490 & 0.9277 & 0.3999 & \textbf{0.9690} \\
      D-GCN & \textbf{0.2554} & 0.9377 & \textbf{0.4109} & 0.9648 \\
      \hline
  \end{tabular}
  \caption{\label{tab:bal-unbal} Results on AudioSet evaluation set with models trained on balanced and unbalanced set.}
\end{table}

\begin{table}[tbp]
  \setlength{\belowcaptionskip}{-0.cm}
  \centering
  \small

  \begin{tabular}{lccc}
      \hline
      Methods &     1\% &     5\% &    10\% \\
      \hline
      CNN14                 & 0.1118 & 0.2343 & 0.2770 \\
      \hline
      AT-GCN                & 0.1243 & 0.2331 & 0.2785 \\
      GCN(ASER)             & 0.1252 & 0.2269 & 0.2735 \\
      GCN(AudioSet)         & 0.1280 & 0.2336 & 0.2747 \\
      GCN(AudioSet+ASER)    & 0.1214 & 0.2283 & 0.2741 \\
      D-GCN  & \textbf{0.1283} & \textbf{0.2387} & \textbf{0.2799} \\
      \hline
      D-GCN rel. improvement & 14.73\% & 1.88\% & 1.05\% \\
      \hline
  \end{tabular}   
  \caption{\label{tab:low-resource-map} mAP for each model trained on different portion of the unbalanced set, and the relative improvement(\%) of D-GCN over CNN14 backbone.}
  \vspace{-4mm}
\end{table}


\tabref{tab:bal-unbal} shows the performance of models trained on the official balanced and unbalanced set, while \tabref{tab:low-resource-map} shows the results on sampled subsets. We can see that all GCN-based models significantly outperform the baseline CNN14 on balanced and low-resource (1\%) set, suggesting the usefulness of the knowledge sources including the newly proposed temporal knowledge in low-resource scenarios. As the size of training data grows, the advantage of GCN models ceases to exist, expect for AT-GCN, possibly due to its knowledge of the co-occurrences on the whole training set, which matches more with the larger training data. D-GCN performs consistently better than single GCN with one KG or the simple addition of both KGs, showing the effectiveness of the separate relation modeling, and it also outperforms CNN14 by mAP in all settings despite the diminishing gain.

To study the reason for the effectiveness of GCN models in low-resource scenario (1\% set) and their degeneration in large-data settings, we divide the classes into groups according to the numbers of training samples, and calculate D-GCN's improvement over baseline on these groups. From Figure \ref{fig:delta_map_auc}, we can see that D-GCN can benefit classes with extremely few samples ([0, 5]), and the gain is the highest on classes with moderate number of samples, but not on the most prevalent classes. We may conclude that the prior knowledge in KG can effectively help the model learn the dependency between labels especially for the few-shot ones. However, as we have more resources, the large backbone model may be capable of learning such relations without KG, which explains why the advantage of GCN-based models would shrink.   

\begin{figure}[htbp]
\setlength{\abovecaptionskip}{0.cm}
\setlength{\belowcaptionskip}{-0.5cm}
\centering
\includegraphics[width=0.9\linewidth]{figures/1p_delta_map_auc.pdf}
\caption{Absolute improvement of D-GCN over CNN14 on classes with different number 
of training samples.}
\label{fig:delta_map_auc}
\end{figure}

\subsubsection{SONYC}

\begin{table}[tbp]
  \setlength{\belowcaptionskip}{-0.cm}
    \centering
    \small
    \begin{tabular}{lcccc}
        \hline
        {} & \multicolumn{2}{c}{Fine-level} & \multicolumn{2}{c}{Coarse-level} \\
        \cline{2-3}\cline{4-5} 
        Methods & Mi AUPRC & Ma AUPRC & Mi AUPRC & Ma AUPRC \\
        \hline
        CNN9 & 0.675 & 0.493 & 0.808 & 0.580 \\
        GCN(ASER) & 0.703 & 0.459 & 0.822 & 0.548 \\
        GCN(OT) & 0.680 & 0.494 & 0.821 & 0.596 \\
        GCN(ASER+OT) & 0.706 & 0.492 & \textbf{0.823} & 0.616 \\
        D-GCN & \textbf{0.709} & \textbf{0.516} & 0.820 & \textbf{0.647} \\
        \hline
    \end{tabular}
    \caption{\label{tab:SONYC} Results on SONYC validate set, OT: SONYC ontology, Mi: Micro, Ma: Macro.}
    \vspace{-4mm}
  \end{table}

The results on the SONYC dataset is shown in Table \ref{tab:SONYC}. Similar to AudioSet (1\%), all GCN models significantly outperform baseline by the main metric Micro AUPRC. The temporal knowledge of ASER seems to be more useful here compared to ontology, as the ontology for SONYC is more sparse, and the labels for each level are predicted separately, so that they don't co-occur. D-GCN again gives consistently best or competitive performance on both level, suggesting the generalizability of this method on effectively combining the strength of two knowledge types.

\section{Related Work}
This section surveys previous works on question generation and tree encoding
respectively.

Text question generation has attracted the attention 
after the work of ~\citeauthor{du2017learning}~\shortcite{du2017learning}, who uses deep seq2seq model 
to generate questions from a raw text paragraph. 
Before that, text question generation relied heavily on hand-craft 
question patterns~\cite{HeilmanS10,LabutovBV15,MostowC09} which is time and 
labor consuming. 

However, this pure seq2seq model is not focused and 
has no control over part in the paragraph to generate question. 
~\citeauthor{zhou2017neural}~\shortcite{zhou2017neural} proposed to encode 
key phrase information using binary indicators to generate 
key-aware questions and they assumes the answer to be key phrase. 
Considering key phrase (answer) is unavailable in reality, 
~\citeauthor{SubramanianWYT17}~\shortcite{SubramanianWYT17} applied 
a two-stage approach. First, key phrases are extracted by 
pointer network~\cite{ptrnet}. Second, 
key phrases are encoded in the same way as 
Zhou et al. With the intuition that questions could be asked in many ways, 
~\citeauthor{Yao2018vae}~\shortcite{Yao2018vae} used conditional-VAE to 
increase the diversity of questions. More recently, models with 
auxiliary feature information~\cite{HarrisonW18} helped improve 
the question quality. Structure question generation aims at 
converting structured data such as triples in knowledge graph to questions. 
~\citeauthor{SerbanGGACCB16}~\shortcite{SerbanGGACCB16} proposed a model to generate factoid questions from knowledge base triples.  None of the above work
considered using parse tree structures to aid question generation process,
which is the focus of this paper.

Sequential RNN model takes sentence as a sequence of words, 
ignoring the syntactic information. In order to utilize
such syntactic information with sequential information, 
~\citeauthor{tai2015improved}~\shortcite{tai2015improved} proposed Tree-LSTM to 
encode the binary parse tree recursively in a bottom-up fashion to 
classify sentiment. In text generation task, 
\citeauthor{eriguchi2016tree}~\shortcite{eriguchi2016tree} 
proposed a tree-to-sequence model with attention mechanism to do 
machine translation and 
~\citeauthor{liang2018automatic}~\shortcite{liang2018automatic} proposed a 
tree-to-sequence model which could handle arbitrary trees, 
to do code comment generation. Our work is inspired by these previous
attempts and we are first to adapt structure encoded neural models to
textual question generations.
\section{Conclusion}
We implement a novel sequence-based dependency parsing
framework which takes advantage of high order features 
in parsing history. 
%We can also adapt beam search to this framework so as to
%relax the strictly greedy nature. Vine pruning\cite{rush2012vine} could
%be incorporated to speed up the parsing.
More importantly, we discovered that the parsing accuracy is very sensitive to
the quality of parsing sequence. Future work can be focused on
developing better sequence predictors that outperform Malt action classifier.
Furthermore, we use two sets of features for sequence predictor and
head mapper right now. A unified set of features between these two components
are worth exploring.
%Besides, better sequence predicting method and unified feature
%representation of two components are worth exploring.
%
%Though we currently get a not bad result,
%the sequence predictor still needs more exploration.
%According to our experiment, slightly changes
%on the sequence can lead to a fatal decline on accuracy. Ensuring the match degree of training sequence and testing
%sequence demands a high quality of sequence predictor.
%
%Further, the features in our current implementation are not expanded and well tuned yet  and we are free to define high order features to make use of parsing history. Our framework is flexible to merge other technics to enhance the performance. Introducing beam could make up for our greedy decoder and improve our accuracy. Vine pruning\cite{rush2012vine} could speed up parsing process. Besides, better sequence predicting method and unified feature representation of two components are worth exploring.


\begin{acks}
This work was partly supported by the
SJTU-CMBCC Joint Research Scheme and SJTU Medicine-Engineering
Cross-disciplinary Research Scheme.
%To Robert, for the bagels and explaining CMYK and color spaces.
\end{acks}

%%
%% The next two lines define the bibliography style to be used, and
%% the bibliography file.
\bibliographystyle{ACM-Reference-Format}
\bibliography{cqgen}

%%
%% If your work has an appendix, this is the place to put it.
\appendix

\begin{table*}[th]
    \centering
    \tiny
    \resizebox{\linewidth}{!}{
        \begin{tabular}{cccccccc}
        \hline
        \textbf{Case} & \textbf{Character} & \textbf{Initial} & \textbf{Final} & \textbf{Rule} & \textbf{Initial IPA} \\
        \hline
        direct                      & \begin{CJK*}{UTF8}{gbsn}波\end{CJK*} &  \begin{CJK*}{UTF8}{gbsn}帮\end{CJK*} & \begin{CJK*}{UTF8}{gbsn}戈\end{CJK*} & \begin{CJK*}{UTF8}{gbsn}帮\end{CJK*}=[p] and \begin{CJK*}{UTF8}{gbsn}戈\end{CJK*}=[\textipa{uA}] & [p] \\
        \multirow{2}{*}{rule-based} 
        & \begin{CJK*}{UTF8}{gbsn}砩\end{CJK*} & \begin{CJK*}{UTF8}{gbsn}帮\end{CJK*} & \begin{CJK*}{UTF8}{gbsn}废\end{CJK*} & \multirow{2}{*}{if(initial=\begin{CJK*}{UTF8}{gbsn}帮\end{CJK*} and final=\begin{CJK*}{UTF8}{gbsn}废\end{CJK*}) then [f] else [p]} & [f] \\
        & \begin{CJK*}{UTF8}{gbsn}碑\end{CJK*} & \begin{CJK*}{UTF8}{gbsn}帮\end{CJK*} & \begin{CJK*}{UTF8}{gbsn}支\end{CJK*} &  & [p] \\
        arbitrary                   & \begin{CJK*}{UTF8}{gbsn}方\end{CJK*} & \begin{CJK*}{UTF8}{gbsn}帮\end{CJK*} & \begin{CJK*}{UTF8}{gbsn}阳\end{CJK*} & - & [f] \\
        \hline
        converted                   & \begin{CJK*}{UTF8}{gbsn}比\end{CJK*} & \begin{CJK*}{UTF8}{gbsn}帮\end{CJK*} & \begin{CJK*}{UTF8}{gbsn}旨\end{CJK*} & - & [p]\\
        \hline				
        \end{tabular}
        }
    \caption{Five different examples of reconstruction.}
    \label{tab:reconstruction}
\end{table*}
\section{Different Cases of Reconstruction}
\label{app:reconstruction}
\tabref{tab:reconstruction} presents five examples in four different cases constructing our ancient Chinese pronunciation dataset for each category. For an identical initial category, different rules applied can lead to different reconstruction result for initial IPA.

\section{Embedding for Medial Feature, Nucleus Feature, and Coda Feature}
\label{app:embedding}
This appendix supplements the embedding employed for the medial, nucleus, and coda features in GTenhanced Transformer, as shown in \figref{fig:embedding2}.

\begin{figure*}[th]
    \centering
    \includegraphics[width=0.4\textwidth]{images/embedding_layer2.png}
    \caption{Embedding for medial feature, nucleus feature, and coda feature.}
    \label{fig:embedding2}
\end{figure*}



\end{document}
\endinput
%%
%% End of file `sample-sigconf.tex'.
