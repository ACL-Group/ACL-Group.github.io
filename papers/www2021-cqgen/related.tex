\section{Related Work}
\paragraph{Clarification Question Generation} The concept of CQ can be naturally raised in a dialogue system where the speech recognition results tend to be erroneous so that we raise CQs for sanity check \citep{stoyanchev2014towards}, or the intents for a task is incomplete or ambiguous in a first short utterance and further CQs are needed to fill in the slots \citep{dhole2020resolving}. The concept is then extended to IR to clarify ambiguous queries \citep{aliannejadi2019asking}, and has been successfully put into practice \citep{zamani2020generating}. Other application areas including KBQA \citep{xu2019asking} and open-domain dialogue systems \citep{aliannejadi2020convai3}. CQGen can also be applied to help refine posts on websites like StackExchange \citep{Kumar_2020} and Amazon \citep{rao2019answer}. In this context, our work closely follows the research line of \citep{rao2018learning, rao2019answer, cao2019controlling}. \citet{rao2018learning} first adopted a retrieval-then-rank approach. They \citep{rao2019answer} then proposed a generation approach to train the model to maximize the utility of the hypothetical answer for the questions with GAN, to better promote specificity. \citet{cao2019controlling} propose to control the specificity by training on data with explicit indicator of specificity, but it requires additional specificity annotation. Towards the similar specificity goal, we adopted a different keyword-based approach. They also assume generating one question per context, which we claim is not sufficient to cover various possible information needs, and thus propose the task of the diverse CQGen.

\paragraph{Diverse Generation} The demand for diverse generation exists in many other fields~\cite{vijayakumar2018diverse, LiangZ18code, shen2019mixture}, and we've drawn inspirations from these literatures. For image captioning, we may use multiple descriptions for different focusing points of a scene. \textit{Diverse Beam Search} \citep{vijayakumar2018diverse} was proposed to broaden the searching space to catch such diversity by dividing groups in decoding and imposing repetition penalty between them. For machine translation, a context can be translated with different styles. \citet{shen2019mixture} thus proposed \textit{Mixture of Expert} models including hMup to reflect various styles with a discrete latent variable (\textit{expert}). And here for CQGen, diversity is required to cover various potentially missing aspects, so we come up with the idea to use keywords as a controlling variable like \textit{expert} to promote diversity.

