\section{Social Impact and Ethical Statement}


This work aims to help people distinguish subtle internal informational bias presented in news reports thus prevent the readers from being directed to a specific point of view. In this way we encourage the readers to freely develop own opinions and to think independently and critically.

The news media has a great influence on individuals and the public’s perception of the world, but news biases are widespread and the influence of biased news reports has been magnified by social media. Moreover, the recommendation systems make readers tend to focus only on news that is consistent with their established views and beliefs, resulting in the "echo chamber" effect. Over time, people will be trapped in such echo chamber and unable to contact or resist news that contradicts their views, and their internal prejudice will only be strengthened. Therefore, ideally the news reports should uphold the principle of objectiveness and neutrality, present the readers with a complete and impartial picture of the event.  However, even the news articles might hardly be completely objective and neutral, we hope that our research could make various conflicting views to be presented fairly by unmasking their internal bias, serving as a reference tool for the readers to avoid being misguided or getting trapped and to develop their own opinions. In this way we encourage the independent thinking and the critical thinking, maintain the communication of multiple viewpoints, and reduce the echo chamber effect.


The dataset used in this work is publicly available and is used under the data usage agreement. All news in examples are published by formal news agencies and can be found online. This study does not require IRB/ethical approval. 