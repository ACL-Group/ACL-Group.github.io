\section{Related Work}

\paragraph{Media bias Detection.} 
Datasets for media bias detection are limited and not standardized since it requires heavy workloads for humans to annotate manually the subtle bias with certain level of expertise. Besides, human annotators can suffer from implicit media bias and their judgments are subjective as well. However, there are still some sentence-level media bias datasets. \citet{wei2020newb} proposed a large sentence-level corpus for political bias detection, \citet{lim-etal-2020-annotating} proposed a sentence-level media bias dataset of 996 sentences from 46 news articles covering 4 topics, \citet{10.1145/3366423.3380158} further built a dataset consisting of more than 2000 sentences annoted with 43000 bias including subjectivity, hidden assumptions and representation tenddencies, \citet{spinde2021mbic} also created a sentence-level media bias dataset with the emphasis on annotators' backgrounds, \citet{fan-etal-2019-plain} presented the BASIL dataset used in our study.

Linguistic features-based techniques are initially utilized in media bias detection and they are still widely applied till today because they provide a strong descriptive and explanatory power. These techniques are systematically formulated in \citep{recasens-etal-2013-linguistic}. \citep{chen-etal-2020-analyzing} explored linguistic patterns at word-level and article-level to analyze political bias in news articles; \citet{SPINDE2021102505} engineered various linguistic, lexical and syntactic features to detect media bias; \citet{10.1145/3366423.3380158} made used of syntactic structure to obtain generalized text embedding for news credibility check; and \citet{baly-etal-2019-multi} used a multi-task ordinal regression framework.

With the rise of deep learning, neural-based aproaches are broadly used in media bias detection. \citet{iyyer-etal-2014-political} used RNNs to aggregate
the polarity of each word to predict political ideology on sentence-level. \citet{gangula-etal-2019-detecting}
made use of headline attention to classify article
bias. \citet{li-goldwasser-2019-encoding} captured social information by Graph Convolutional Network to identify political bias in news articles. \citet{fan-etal-2019-plain} used BERT and RoBERTa and \citet{van-den-berg-markert-2020-context} used BiLSTMs as well as BERT-based models to detect sentence-level informational bias.
\paragraph{Contextual information in media bias detection.} 
Contextual information is explored, though primarily, in media bias detection. \citet{baly2020detect} employed an adversarial news media adaptation using triplet loss; \citet{kulkarni-etal-2018-multi} proposed an attention based model to capture views from news articles' title, content and link structure; \citet{chen-etal-2020-detecting} explored the impact of sentence-level bias to article-level bias; \citet{li-goldwasser-2019-encoding} encoded social information using GCN; \citet{baly-etal-2018-predicting} made use of news media's cyber-features in news factuality prediction; \citet{10.1145/3366423.3380158} explored cross-media context by a news article graph.

Sentence-level informational bias is under-studied by only a few research and the methods described above are not applicable on this task. In order to infuse contextual information, we refer to extractive summarization \citet{10.1145/3397271.3401327} and \citet{christensen-etal-2013-towards} which used sentence graph to encode context.

% extractive summarization
% \paragraph{Contrastive learning}
% \paragraph{Graph Attention Network}