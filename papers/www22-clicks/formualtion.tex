\section {Task Formulation}

The set of articles is $P = \{p_1, p_2, \cdots, p_n\}$. For each article $p_i$, we use $L_i$ to represent the set of links on article $p_i$, where $L_i = \{l_1, l_2, \cdots, l_{N_i}\}$ and $N_i = |L_i|$. Given a page $p$ and its link $l$, we define the click counts as $click(p, l)$. The clickstream data on page hyperlinks $click(p, l)$ is then used as ground truth. 

Given an article $p_i$ and its associated set of links $L_i$, every link $l_j \in L_i$ has its label $y_j$ indicating its true ranking. Ranking linked entities can be viewed as a task of find permutation $\pi$ on indices $\{1,2,\cdots,N_i\}$. Permutation $\pi$ must satisfy that $click(p_i,l_{\pi(j)}) > click(p_i,l_{\pi(k)}) $ for all $1 \leq j < k \leq N_i$. The sequence of $l_{\pi(1)}$

Methodologically, we obtain features and then feed these features into the Learning to Rank algorithm.

The sequence $l_{\pi(1)}, l_{\pi(2)}, \cdots l_{\pi(N_i)}$ is the ordering of linked entities according the their number of clicks in an decreasing order with respect to the article $p$.

The Learning to Rank algorithm is fundamental. Each hyperlink can be represented as a feature vector.