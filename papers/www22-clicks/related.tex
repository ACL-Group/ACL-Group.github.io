\section{Related Work} 

This section introduces related works. Section \ref{related_1} is about some works on user-centric Entity Linking systems and Section \ref{related_2} introduces Wikipedia-related works.

\subsection{User-centric Entity Linking System} \label{related_1}

Entity Linking has been studied for decades. However, the majority of works study machine-centric Entity Linking systems, i.e., the detected entities are feed to a machine. Instead, we consider user-centric Entity Linking systems \cite{von2007leveraging}, where the users can perform actions on the linked entities. In user-centric Entity Linking systems, not all of the detected entities in a given piece of text are necessary to be linked. To link or not to link \cite{guo2013link} naturally becomes a problem for entities. It is thus an important problem to detect only the helpful entities from the article. Apart from improving users' reading experience, this task could also be applied to other applications like content recommendation \cite{joseph2019content}.

There are some other works concentrating on key entity prediction. In \cite{kraft2011contextual}, the authors tried to extract the most interesting and relevant keyword phrases from the article. \cite{gao2014modeling} modeled interestingness of entities with deep neural networks. Moreover, an Entity Linking system named Linkify has also been proposed to evaluate the helpfulness of linked entities \cite{yamada2018linkify, yamada2014evaluating}. Differently, we consider the situation where all the entities have been linked and the task is to predict the click popularity of entities.

\subsection{Wikipedia} \label{related_2}

In Wikipedia, many works have been proposed to enhance users' reading experience. Wikification is an example, which is the task of linking textual mentions to the corresponding Wikipedia pages. The seminal work \cite{cucerzan-2007-large} proposed one of the first entity linking systems on Wikipedia. This work tackled the task of wikification. Successively, other works were proposed to automatically link documents, such as Wikify \cite{csomai2008linking, mihalcea2007wikify} and document cross-reference \cite{milne2008learning}. However, these works mainly deal with some issues like keyword extraction and disambiguation. The overlinking problem has been rarely discussed. 

Some more recent works \cite{brochier2021predicting, gundala2018readers} study the problem of link prediction between documents to solve the overlinking problem. In these two works, some necessary links are missing and need to be predicted. Differently, we consider the situation where all the links are given and the task is to predict the click popularity of all the links.

However, there are only a few works regarding click popularity prediction on Wikipedia hyperlinks. In \cite{thruesen2016link}, the authors studied ranking hyperlinks based on textual and graph-based features. However, visual information is missing in this work. For visual features, \cite{lamprecht2017structure} studies how the structure of a Wikipedia article affects users' navigation. First in \cite{dimitrov2016visual}, the Html pages are rendered to study the effect of visual position. Later in \cite{dimitrov2017makes}, the coordinate position of the links are used as a type of visual feature to predict the click popularity. However, it requires rendering to obtain the visual features. Therefore, we propose several visual features that can be obtained without rendering so that they are more computationally efficient and naturally compatible with different resolutions.

