\section{Related Work}
Recently, depression detection has received much attention. Studies includes predicting depression diagnosis from clinical interviews \citep{gratch2014distress, dinkel2019text}, from medical records \citep{eichstaedt2018facebook} and from self-reported surveys \citep{guntuku2019twitter}. Depression detection on social media is especially promising, as proxy diagnostic signals can be relatively easy to get from self-reports or activities in depression communities (e.g. depression subreddit) \citep{losada2016test,yates2017depression,wolohan2018detecting,ernala2019methodological}. Early attempts~\citep{losada2016test} used TF-IDF and Logistics Regression on all user posts for depression detection. Later researchers further incorporate new features like LDA, LIWC dictionary and posting patterns \citep{trotzek2018utilizing, harrigian2020models}. For deep learning methods, \citet{yates2017depression} uses hierarchical CNN to process all the posts of a user at the first level and merge the output at the second level for user-level classification. However, most of them directly use all the user's posts without screening out the truly salient posts, which may negatively affect their accuracy and efficiency.

In terms of model interpretability, traditional feature-based methods are partially explainable on the level of global features. For example, \citet{shen2017depression} found different behaviors for depressed users in posting time, emotion catharsis, self-awareness and life sharing. However, these methods cannot make user-level explanations as personalized diagnostic basis. Moreover, it remains doubtful if these nuanced patterns can be regarded as robust depression indicators \citep{ernala2019methodological}. This may also explain why the performance of such models may degrade significantly when transferred to other datasets \citep{harrigian2020models}.

In practice, we also want to identify depression risk as early as possible, as is exemplified by the early risk detection competitions \citep{losada2019overview}. The majority of proposed methods can only achieve satisfying performance given almost the whole dataset, and few of them are able to make immediate response to each item update. To reduced the number of required posts, \citet{zogan2021depressionnet} uses extractive summarization to extract key posts of a user. However, it relies on K-means clustering to get the summaries, so the model cannot run online as well. 

