\section{Survey of Paraphrasing Techniques}

% Natural Language Paraphrasing

Paraphrase corpora contains a large number of paraphrase pairs, including lexical, phrasal paraphrases and paraphrase patterns. The paraphrase pairs are usually extracted from extremely large text corpora.

% ppdb
\textbf{PPDB} \cite{ganitkevitch2013ppdb}, the paraphrase database, is extracted from bilingual parallel corpora totaling over 100 million sentence pairs and over 2 billion English words. \textbf{PPDB} contains over 220 million paraphrase pairs, consisting of 73 million phrasal and 8 million lexical paraphrases, as well as 140 million paraphrase patterns, which capture many meaning-preserving syntactic transformations. Each paraphrase pair in \textbf{PPDB} also contains a set of associated scores, including paraphrase probabilities derived from the bitext data and a variety of monilingual distributional similarity scores computed from the Google n-grams and the Annotated Gigaword corpus.

% paralex
\textbf{PARALEX} \cite{fader2013paraphrase} is another large monolingual parallel corpora, containing 18 million pairs of question paraphrases from \textbf{wikianswers.com}, which is a collaboratively edited Question and Answering site. \textbf{Wikianswers.com} users can tag pairs of questions as alternate wordings of each other. \textbf{PARALEX} contains paraphrases of questions. At some degree, \textbf{PARALEX} is more suitable for our needs because it focuses on question paraphrases. Paraphrase pairs in \textbf{PARALEX} are word-aligned using the standard machine translation methods. They use the word alignments to construct a phrase table \cite{berant2014semantic}. At last, the result in a phrase table contains approximately 1.3 million phrase pairs.

% patty
\textbf{PATTY} \cite{nakashole2012patty} extracts a large number of relation patterns from Web corpora like Wikipedia and New York Time by leveraging techniques like dependency parse, named entity recognition (NER), and entity linking. Every pattern is generalized to a syntactically more general pattern after replacing words with POS-tags and some other tricks. Those raw patterns sharing a common representing pattern can be grouped together to become a synonymous set, which can be regarded as paraphrase dataset.

% MS
Besides the above mentioned representative paraphrase corpora, Microsoft Paraphrase Corpus \cite{automatically-constructing-a-corpus-of-sentential-paraphrases} is also considered. It is a text file containing 5800 pairs of sentences which have been extracted from news sources on the web, along with human annotations indicating whether each pair captures a paraphrase/semantic equivalence relationship. Microsoft Paraphrase Corpus contains only sentence paraphrases.
