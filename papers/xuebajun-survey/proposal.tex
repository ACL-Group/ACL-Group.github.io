\section{Research Proposal}

Now we propose two approaches to answer K12 questions in our scenario.
We first introduce a searching approach over community QA system, then we discuss
the other approach based on knowledge bases.

\subsection{Searching Based Approach}

In the searching based proposal, knowledge in community QA system will
help answer a new question. The intuition is to find the most similar questions
and summarize their answers.
In the following proposal, we assume using Baidu Zhidao as the cQA system.
Three main steps are listed below:

1. \textbf{Question clustering}. All questions in Baidu Zhidao needs to form into clusters.
Question clusters gather similar questions together, and replacing original questions by
summarized clusters can reduce the computation cost in further steps.
We separate a natural language question into two parts: focus entities, and relation template.
Focus entities (could be at least 1) are extracted by entity linking tools, and the relation template
is the question surface form with entities replaced by a placeholder ``\$e\$''.
Necessary tricks could be made to remove useless parts in a relation template.
As a clustering task, all questions within a cluster share the same focus entities,
and relation templates are close to each other.

In order to measure the similarity between relation templates, we plan to use an embedding model
to encode relation templates into a vector representation.
Fortunately, Baidu Zhidao have already provided ``similar questions'' in the web page of each question.
We all collect $\langle q, q' \rangle$ question pairs as our training data, if $q$ and $q'$ share the same
focus entities, and $q'$ is the similar question of $q$ provided by Zhidao.
Negative question pairs are also prepared via random combination.
Following the idea of Toutanova et al.~\shortcite{toutanova2015representing},
we build a CNN model for vector embedding, and train the model as a classification problem:
whether two questions are paraphrased or not.

After training step, we apply hierarchical clustering approach: initially each question is cluster,
for each step, the two nearest clusters (largest cosine similarity) are merged,
and update the embedding vector as the average over all questions in the cluster.
The merging stops if the cosine similarity between any two clusters are small than some threshold $\tau$.

2. \textbf{Question weighting}. Regarding clusters as documents, and relation templates as terms,
we calculate tf-idf score for each question in the cluster.
The relation template is more representative to a cluster if it mentioned multiple times in the cluster (different users
asked the same question in really similar style), and less representative if it occurs in many different clusters
(indicating somewhat common but not meaningful questions).

3. \textbf{Answer predicting}. When coming a new question $q_{new}$, we extract its focus entities and template,
and calculate the relation similarity to each cluster.
Answers of a question $q$ in the cluster $C$ contributes to the new question, the weight of contribution
is defined as:

$w(q, q_{new}) = sim(q_{new}, C) * tf\_idf(q, C)$.

Finally we summarize answers from the best answers of these weighted similar questions.
For factoid questions, we recognize all answer entities from answer text, and perform
weighted majority voting algorithm over all the similar questions;
for true-or-false questions, we first divide answers into two groups, then we show
the weighted supporting ratio of each group, as well as top opinions in each group;
for subjective questions, it's related to another task called answer summarization,
currently we plan to just output the top weighted answer.





\subsection{Ontology Based Approach}

We first discuss the construction of knowledge base.
We plan to build the knowledge base on both textbooks and Baidu Baike resources.
The knowledge base contain entities, types, and predicates between entities,
and it's better to provide aliases on both entities, types and predicates.
For the side of textbook, Xuebajun have already built the ontology,
but currently we don't have much knowledge about that (except one PPT slide).
If it's necessary to enrich the ontology, editorial efforts are considered because
data from textbooks are limited.
To build the ontology from Baidu Baike, we plain to leverage the infobox in each
article, and build the knowledge base upon infoboxes.
Some entity links in the infobox are missing, currently we are working on missing link completion.
Besides, predicates names in Baike infobox are not uniformed.
As an example, we found ``spouse'', ``wife'', ``first wife'', ``second wife'' representing
the same semantic meaning.
In order to retrieve uniform predicate names, techniques like clustering could be applied here.
Once knowledge bases from textbooks and Baidu Baike are prepared, we merge them into one big ontology.
Existing approaches~\cite{xiang2015ersom} can be applied here.

Now we discuss the query generation strategy.
We plan to follow Bast and Haussmann~\cite{bast2015more} as our starting point, because we don't
expect K12 users will ask question with too complex structure.
Therefore we prefer a slot filling strategy as the initial trial, which is easier to implement
and better to avoid over-fitting problem.
In order to match natural language words to predicates in the knowledge base,
we can use lexical, word derivational and synonym matching strategy to discover semantic
associations between input question and structured query.
Besides, we can search corpus for sentence that focus entity and candidate answer co-occurs.
Context words in the sentence can be indicators of selected predicates.
Taking F1 of each candidate query structure as the measurement,
we can adopt pair-wise ranking algorithm to learn feature weights.


