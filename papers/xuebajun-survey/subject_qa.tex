\section{Survey of QA on Subjects}



Ma and Song~\shortcite{ma2006framework} proposed a context-aware framework for community question answering. The authors argued that context information could be very important when answering questions, for example, given a question ``where is the Empire State Building?'', if it is asked on 33rd street in Manhattan, the answer might be ``just around the corner''; if it is asked in San Fransisco, the answer might be ``in New York''; or if it is asked in Shanghai, the answer might be ``in United States.'' The framework is based on a math forum, and it keeps tracks of users' browsing history along with level (primary school/middle school/high school/college) as the context. When user posts a question, the framework would search for QA pairs with similar lexical features and ontology terms, then rank them based on contextual similarity.

Abdi et al.~\shortcite{abdi2016qapd} proposed an ontology-based question answering system called QAPD whose domain is limited in physics. This system contains a knowledge base focusing on electricity and electromagnetism, and a database of question template-SQL template pairs. For new questions, the system first does some basic NLP processing, then find the best matched question template according to some measures, and use the corresponding SQL expression to query the knowledge base.

In the work of Ferr{\'e}s and Rodr{\'\i}guez~\shortcite{ferres2006experiments}, the authors described an approach to adapt an existing multilingual Open-Domain Question Answering (ODQA) system for factoid questions to a restricted domain, Spanish geography. They modified some of the components of the system and made use of some external resources like GNS Gazetteer for Named Entity (NE) Classification and Wikipedia or Google in order to obtain relevant documents for this domain.

Iyyer et al.~\shortcite{iyyer2014neural} introduced a recursive neural network model to answer factoid questions in the domain history and literature. The question consists of four to six sentences and are associated with factoid answers, such as specific battles, presidents or events. Authors built the recursive network over dependency parse trees to compute distributed representations for each individual sentence, then representations from multiple sentences are aggregated and fed into a classifier to predict the answer.

The Halo Project~\shortcite{angele2003ontology} targeted chemistry tests. In this application, authors manually modeled a complex ontology with rules, multiple representations of objects, and call-out functionality to particular problem-solving methods in F-Logic. Basic concepts of chemistry are represented as concepts in F-Logic and are arranged in is-a hierarchy. Complex chemical relationships and axioms are represented by rules.

In terms of subject QA corpus, Miyao and Kawazoe~\shortcite{miyao2013university} proposed a natural language corpora of university entrance examinations, aiming at benchmarking NLP systems for problem solving. Authors collected source texts from National Center Test in Japan, including examination texts from eleven subjects, such as world history, politics, physics and chemistry. In addition, 
textbooks of particular subjects are collected as knowledge sources for solving such questions. For these text data, authors annotated document structures, question types, technical terms, dependency trees and coreferences.
